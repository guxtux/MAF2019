\documentclass[12pt]{beamer}
\usepackage[Warsaw, crane]{../Estilos/BeamerMAF}
%\usepackage[Warsaw, crane]{../Estilos/BeamerEstilosMAF}
%\input{../Preambulos/preambulo_Beamer_Warsaw_seahorse}
%\input{../Preambulos/preambulo_presentacion_CambridgeUS_beaver}
\date{}
\title{\large{Objetivos del Tema 2 - Primeras técnicas de solución}}
\subtitle{Curso MAF}
\author{M. en C. Gustavo Contreras Mayén}


\begin{document}
\maketitle
\fontsize{14}{14}\selectfont
\spanishdecimal{.}
\section*{Contenido}
\frame[allowframebreaks]{\tableofcontents[currentsection, hideallsubsections]}
\section{Introducción}
\subsection{Las EDP}
\begin{frame}
\frametitle{Las Ecs. Diferenciales Parciales}
De la ecuación de movimiento de una partícula para conocer la fuerza por ejemplo, en general nos conduce a una ecuación diferencial.
\\
\bigskip
Por lo tanto, en casi toda la física  básica y en una mayor parte de la física teórica avanzada se expresa en términos de ecuaciones diferenciales.
\end{frame}
\end{document}