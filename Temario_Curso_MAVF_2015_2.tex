\documentclass[12pt]{article}
\usepackage[utf8]{inputenc}
\usepackage[spanish]{babel}
\usepackage{amsmath}
\usepackage{amsthm}
\usepackage{graphicx}
\usepackage{color}
\usepackage{float}
\usepackage{multicol}
\usepackage{enumerate}
\usepackage{anyfontsize}
\usepackage{anysize}
\usepackage{enumitem}
\setlist[enumerate]{itemsep=0mm}
\renewcommand{\baselinestretch}{1.2}
\let\oldbibliography\thebibliography
\renewcommand{\thebibliography}[1]{\oldbibliography{#1}
\setlength{\itemsep}{0pt}}
\marginsize{1.5cm}{1.5cm}{0cm}{2cm}
\author{M. en C. Gustavo Contreras Mayén. \texttt{curso.fisica.comp@gmail.com}\\
Fís. Abraham Lima Buendía. \texttt{abraham3081@ciencias.unam.mx}}
\title{Matemáticas Avanzadas de la Física \\ {\large Semestre 2015-2 Grupo 8810}}
\date{ }
\begin{document}
%\renewcommand\theenumii{\arabic{theenumii.enumii}}
\renewcommand\labelenumii{\theenumi.{\arabic{enumii}}}
\maketitle
\fontsize{12}{12}\selectfont
\section{Objetivos.}
Dentro el curso el alumno:
\begin{itemize}
\setlength{\itemsep}{0mm}
\item Reconocerá las ideas básicas del análisis de ecuaciones que involucran a funciones de varias variables.
\item Formulará aproximaciones consistentes a soluciones, con el fin de cuantificar los distintos mecanismos de la física que se involucran.
\item Consultará la literatura matemática que sea relevante para los problemas de física.
\item Identificará el papel moderno que juegan las funciones especiales, como auxiliares poderosos en el análisis cualitativo de problemas en varias variables.
\end{itemize}
También es nuestro objetivo demostrar al alumno que las funciones especiales y las transformadas integrales no son solamente un tema matemático, que involucra las ramas de la geometría diferencial, las ecuaciones diferenciales y el análisis matemático, sino que son las técnicas de estudio fundamentales en la electrostática, la electrodinámica, la mecánica cuántica en los límites relativista y  no relativista, la dinámica de medios deformables, la hidrodinámica clásica entre otras ramas de la física.
\\
\\
\textbf{Punto importante: } Considerando que MAF es una asignatura de sexto semestre, se espera que hayan cursado y acreditado: Cálculo I a Cálculo IV, Ecuaciones Diferenciales Ordinarias I, Algebra lineal I, Variable Compleja I, la llamada Física Clásica (Física contemporánea, Mecánica vectorial, Fenómenos colectivos, Electromagnetismo, Óptica e Introducción a la Mecánica cuántica)
\\
\\
\textbf{Lugar: }Aula P205.
\\
\textbf{Horario: } Martes de 15 a 16 horas y Jueves de 16 a 18 horas.
\section{Metodología de enseñanza.}
En las clases habrá exposición con dialógo por parte de los profesores, junto con el desarrollo de ejercicios que orientarán la solución de problemas en un primer momento de tipo analítico, para luego revisar un ejercicio tomado de la física.
\\
\\
Se dejarán lecturas complementarias, consulta de referencias bibliográficas, y en algunos casos, la solución numérica de ejercicios usando algún lenguaje de programación o paquetería.
\\
\\
El curso demandará como mínimo el mismo número de horas fuera de clase, es decir, que deberán de dedicarle al menos 5 horas a la semana para el desarrollo de lecturas, ejercicios, tareas y actividades.
\section{Temario}
\begin{enumerate}
\item La física y la geometría.
\item Primeras técnicas de solución.
\item Completez y ortogonalidad de una base.
\item Función Gamma($\Gamma$)y función Beta ($\beta$)
\item Separación de variables en coordenadas esféricas.
\item Funciones especiales.
\item Transformadas integrales.
\end{enumerate}
\section{Evaluación.}
La evaluación del curso contempla la entrega de tareas y de la solución de exámenes.
\\
\\
El peso para cada elemento de evaluación es el siguiente:
\begin{itemize}
\setlength{\itemsep}{0mm}
\item En total 7 tareas (una por cada tema): \hspace{1cm} 30 \%.
\item En total 3 exámenes parciales: \hspace{2.5cm} 70\%.
\end{itemize}
\textbf{Consideraciones importantes:}
\begin{itemize}
\setlength{\itemsep}{0mm}
\item La entrega de tareas se hará en el salón de clase, teniendo como límite de entrega: las 4:20 pm; no se recibirán tareas por medios electrónicos.
\item En caso de no acreditar un examen parcial de los tres programados, se podrá presentar una sola reposición.
\item Para presentar examen final del curso, deben de cumplirse de manera simultánea los siguientes criterios: tener dos o los tres exámenes no acreditados y haber entregado todas las tareas. En caso de que falte alguna tarea, no se aplica la política de examen final, por lo que el promedio final queda en 5 (cinco)
\end{itemize}
\section{Bibliografía.}
A continuación se indican algunos textos que serán útiles para consulta dentro del curso, de manera adicional, se proporcionarán algunos artículos científicos para revisión de ejemplos aplicados. Se habilitará un compendio de materiales de trabajo mediante una carpeta en Dropbox, tanto de materiales básicos como complementarios, tales como artículos, capítulos de libros, notas, etc. Para contar con la liga de acceso, deberán de enviar un mensaje de correo a Abraham para que les proporcione la liga de Dropbox.
\nocite{*}
\bibliography{LibrosCurso2015-2}\bibliographystyle{unsrt}
\section{Fechas importantes.}
\begin{itemize}
\item Lunes 26 de enero: Inicio del semestre 2015-2.
\item Del lunes 30 de marzo al viernes 3 de abril: Semana Santa.
\item Viernes 22 de mayo: Fin del semestre 2015-2.
\item Del lunes 26 al viernes 29 de mayo: Primera semana de finales.
\item Del lunes 1 al viernes 5 de junio: Segunda semana de finales.
\end{itemize}
\end{document}