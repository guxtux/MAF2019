\documentclass[12pt]{article}
\usepackage[utf8]{inputenc}
\usepackage[spanish,es-lcroman, es-tabla]{babel}
\usepackage[autostyle,spanish=mexican]{csquotes}
\usepackage{amsmath}
\usepackage{amssymb}
\usepackage{nccmath}
\numberwithin{equation}{section}
\usepackage{amsthm}
\usepackage{graphicx}
\usepackage{epstopdf}
\DeclareGraphicsExtensions{.pdf,.png,.jpg,.eps}
\usepackage{color}
\usepackage{float}
\usepackage{multicol}
\usepackage{enumerate}
\usepackage[shortlabels]{enumitem}
\usepackage{anyfontsize}
\usepackage{anysize}
\usepackage{array}
\usepackage{multirow}
\usepackage{enumitem}
\usepackage{cancel}
\usepackage{tikz}
\usepackage{circuitikz}
\usepackage{tikz-3dplot}
\usetikzlibrary{babel}
\usepackage{bm}
\usepackage{mathtools}
\usepackage{esvect}
\usepackage{hyperref}
\usepackage{relsize}
\usepackage{siunitx}
\usepackage{physics}
%\usepackage{biblatex}
\usepackage{standalone}
\usepackage{mathrsfs}
\usepackage{bigints}
\usepackage{bookmark}
\spanishdecimal{.}

\setlist[enumerate]{itemsep=0mm}

\renewcommand{\baselinestretch}{1.5}

\let\oldbibliography\thebibliography

\renewcommand{\thebibliography}[1]{\oldbibliography{#1}

\setlength{\itemsep}{0pt}}
%\marginsize{1.5cm}{1.5cm}{2cm}{2cm}


\newtheorem{defi}{{\it Definición}}[section]
\newtheorem{teo}{{\it Teorema}}[section]
\newtheorem{ejemplo}{{\it Ejemplo}}[section]
\newtheorem{propiedad}{{\it Propiedad}}[section]
\newtheorem{lema}{{\it Lema}}[section]

\usepackage{mathrsfs}
\usepackage{tikz}
\usepackage{bigints}
\spanishdecimal{.}
%\usepackage{enumerate}
%\author{M. en C. Gustavo Contreras Mayén. \texttt{curso.fisica.comp@gmail.com}}
\title{Funciones de Bessel 2 \\ {\large Matemáticas Avanzadas de la Física}}
\date{ }
\begin{document}
%\renewcommand\theenumii{\arabic{theenumii.enumii}}
\renewcommand\labelenumii{\theenumi.{\arabic{enumii}}}
\maketitle
\fontsize{14}{14}\selectfont
\section{Ejemplo.}
Partiendo del reposo a una distancia $L$ desde el origen $O$, una partícula $P$ de masa variable $m$ es atraída hacia el origen por una fuerza dirigida que apunta hacia el origen y que tiene magnitud proporcional al producto $my$, donde $y$ es la distancia de $P$ a partir de la origen. La masa $m$ de $P$ disminuye con el tiempo $t$ de acuerdo con la expresión
\begin{equation}
m = \dfrac{1}{a + bt}
\end{equation}
donde $a$ y $b$ son constantes. El problema es calcular el tiempo que necesita la partícula $P$ en llegar al origen $O$.
\\
\emph{Solución: }
\\
De Newton-2
\begin{equation}
\dfrac{d\overrightarrow{M}}{dt} = \overrightarrow{F}
\end{equation}
donde $\overrightarrow{M}$ es el vector momento y $\overrightarrow{F}$ es la fuerza actuante. El problema nos conduce a
\begin{equation}
\dfrac{d}{dt} \left( m \dfrac{dy}{dt} \right) = - k^{2} m y
\end{equation}
que es
\begin{equation}
m\dfrac{d^{2} y}{dt^{2}}+ \dfrac{dm}{dt}\dfrac{dy}{dt} + k^{2} m y = 0
\end{equation}
donde $k^{2}$ es la constante de proporcionalidad involucrada en la magnitud de $\overrightarrow{F}$.
\\
Si proponemos el cambio de variable
\begin{equation}
a + bt = bx
\end{equation}
tal que $m=1/bx$ y $dy/dt = dy/dx$, entonces la ecuación se puede expresar como
\begin{equation}
\dfrac{d^{2} y}{d x^{2}} - \dfrac{1}{x} \dfrac{dy}{dx} + k^{2} y = 0
\end{equation}
que es
\begin{equation}
x^{2} y'' - xy' + k^{2}x^{2} y = 0
\end{equation}
\end{document}