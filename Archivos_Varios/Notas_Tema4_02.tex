\documentclass[12pt]{article}
\usepackage[utf8]{inputenc}
\usepackage[spanish,es-lcroman, es-tabla]{babel}
\usepackage[autostyle,spanish=mexican]{csquotes}
\usepackage{amsmath}
\usepackage{amssymb}
\usepackage{nccmath}
\numberwithin{equation}{section}
\usepackage{amsthm}
\usepackage{graphicx}
\usepackage{epstopdf}
\DeclareGraphicsExtensions{.pdf,.png,.jpg,.eps}
\usepackage{color}
\usepackage{float}
\usepackage{multicol}
\usepackage{enumerate}
\usepackage[shortlabels]{enumitem}
\usepackage{anyfontsize}
\usepackage{anysize}
\usepackage{array}
\usepackage{multirow}
\usepackage{enumitem}
\usepackage{cancel}
\usepackage{tikz}
\usepackage{circuitikz}
\usepackage{tikz-3dplot}
\usetikzlibrary{babel}
\usetikzlibrary{shapes}
\usepackage{bm}
\usepackage{mathtools}
\usepackage{esvect}
\usepackage{hyperref}
\usepackage{relsize}
\usepackage{siunitx}
\usepackage{physics}
%\usepackage{biblatex}
\usepackage{standalone}
\usepackage{mathrsfs}
\usepackage{bigints}
\usepackage{bookmark}
\spanishdecimal{.}

\setlist[enumerate]{itemsep=0mm}

\renewcommand{\baselinestretch}{1.5}

\let\oldbibliography\thebibliography

\renewcommand{\thebibliography}[1]{\oldbibliography{#1}

\setlength{\itemsep}{0pt}}
%\marginsize{1.5cm}{1.5cm}{2cm}{2cm}


\newtheorem{defi}{{\it Definición}}[section]
\newtheorem{teo}{{\it Teorema}}[section]
\newtheorem{ejemplo}{{\it Ejemplo}}[section]
\newtheorem{propiedad}{{\it Propiedad}}[section]
\newtheorem{lema}{{\it Lema}}[section]

\usepackage{mathrsfs}
\spanishdecimal{.}
%\usepackage{enumerate}
%\author{M. en C. Gustavo Contreras Mayén. \texttt{curso.fisica.comp@gmail.com}}
\title{Funciones hipergeométricas \\ {\large Matemáticas Avanzadas de la Física}}
\date{ }
\begin{document}
%\renewcommand\theenumii{\arabic{theenumii.enumii}}
\renewcommand\labelenumii{\theenumi.{\arabic{enumii}}}
\maketitle
\fontsize{14}{14}\selectfont
\section{Funciones hipergeométricas.}
En el estudio de ED2 dentro de la matemática se ha buscado generalizar, llegando a una fomra general de una ecuación diferencial de segundo que tiene aplicaciones de interés en la física, este tipo de ecuaciones se expresa en términos de una serie de potencias, a una de estas ecuaciones se le llama \emph{ecuación diferencial hipergeométrica}.
\begin{equation}
 x(1-x) y'' + [ \gamma - (\alpha + \beta + 1) x ] y' - \alpha \beta y = 0
\label{eq:ecuacion_11_22}
\end{equation}
donde $\alpha$, $\beta$ y $\gamma$ son constantes\footnote{En algunos textos se utiliza $a$, $b$ y $c$ en lugar de $\alpha$, $\beta$, $\gamma$}
\\
La solución en series de esta ecuación diferencia es llama \textbf{función hipergeométrica}, que puede escribirse en términos de la función Gamma
\begin{equation}
F( \alpha, \beta; \gamma; x ) = \dfrac{\Gamma(\gamma)}{\Gamma(\alpha) \Gamma(\beta)} \sum_{n=0}^{\infty} \dfrac{\Gamma(\alpha + n) \Gamma(\beta + n)}{\Gamma(\gamma +n) \Gamma(n+1)} x^{n} 
\label{eq:ecuacion_11_23}
\end{equation}
De esta representación en series, notamos que la función hipergeométrica es simétrica al intercambiar $\alpha$ y $\beta$. 
\\
Una serie infinita puede ''integrarse'' de tal modo que podemos expresar el resultado en términos de una integral, en este caso, si multiplicamos y dividimos la serie de la ecuación (\ref{eq:ecuacion_11_23}) por $\Gamma(\gamma - \beta)$ resulta
\[ F( \alpha, \beta; \gamma; x ) = \dfrac{\Gamma(\gamma)}{\Gamma(\alpha) \Gamma(\beta) \Gamma(\gamma - \beta)}  \sum_{n=0}^{\infty} \dfrac{\Gamma(\alpha + n)}{ \Gamma(n+1)} \dfrac{\Gamma(\gamma - \beta) \Gamma(\beta + n)}{\Gamma(\gamma + n)} \]
pero
\[ B(\gamma - \beta, \beta -n) = \dfrac{\Gamma(\gamma - \beta) \Gamma(\beta + n)}{\Gamma(\gamma + n)} \]
Usando $\Gamma(n+1) = n!$ así como al representación integral de la función Beta
\[ \begin{split}
 F( \alpha, \beta; \gamma; x ) &= \dfrac{\Gamma(\gamma)}{\Gamma(\alpha) \Gamma(\beta) \Gamma(\gamma - \beta)}  \sum_{n=0}^{\infty} \int_{0}^{1} dt (1-t)^{\gamma - \beta -1} t^{\beta +n +1} \Gamma(\alpha + n) \dfrac{x^{n}}{n!} \\
&= \dfrac{\Gamma(\gamma)}{\Gamma(\beta) \Gamma(\gamma - \beta)} \int_{0}^{1} dt (1-t)^{\gamma - \beta -1} t^{\beta - 1} \sum_{n=0}^{\infty} \dfrac{\Gamma(\alpha + n)}{\Gamma(\alpha)} \dfrac{(tx)^{n}}{n!}
\end{split}  \]
Entonces podemos escribir
\begin{equation}
F( \alpha, \beta; \gamma; x ) = \dfrac{\Gamma(\gamma)}{\Gamma(\beta) \Gamma(\gamma - \beta)} \int_{0}^{1} dt (1-t)^{\gamma - \beta -1} t^{\beta - 1} (1-tx)^{-\alpha}
\label{eq:ecuacion_11_24}
\end{equation}
Que es la representación integral de función hipergeométrica.
\\
La generalidad de la ecuación diferencial hipergeométrica radica en la habilidad para expresar varias funciones elementales y las llamadas funciones especiales de la física matemática en términos de la función hipergeométrica.
\\
Por ejemplo: consideremos la integral elíptica de segunda clase:
\begin{equation}
E(k) = E \left( \dfrac{\pi}{2} ,k \right) = \int_{0}^{\pi/2} \sqrt{1-k^{2} \sin^{2} t} dt
\label{eq:ecuacion_11_15}
\end{equation}
que al utilizar la integral
\[ \int_{0}^{\pi/2} \sin^{2n} t dt = \dfrac{(2n-1)!!}{(2n)!!} \dfrac{\pi}{2} \]
se puede demostrar que
\[ E(k) = \dfrac{\pi}{2} \left[ 1 - \sum_{n=1}^{\infty} \left[ \dfrac{(2n-1)!!}{(2n)!!} \right]^{2} \dfrac{k^{2n}}{2n-1} \right] \]
Y podemos expresar la integral de segunda clase en términos de la función hipergeométrica
\[ \begin{split}
E(k) &= \dfrac{\pi}{2} \left[ 1 - \sum_{n=1}^{\infty} \dfrac{\Gamma(n+\frac{1}{2})\Gamma(n+\frac{1}{2}) \pi^{-1} }{\Gamma(n+1) \Gamma(n+1)} \dfrac{k^{2n}}{2(n - \frac{1}{2})} \right] \\
&= \dfrac{\pi}{2} - \dfrac{1}{4} \sum_{n=1}^{\infty} \dfrac{\Gamma(n+\frac{1}{2}) \Gamma(n-\frac{1}{2})}{\Gamma(n+1) \Gamma(n+1)} (k^{2})^{n}
\end{split} \]
donde se ha utilizado $\Gamma(n+ \frac{1}{2} = (n-\frac{1}{2} \Gamma(n-\frac{1}{2})$. La suma inicia en $1$. Para que realmente se parezca a una serie hipergeométrica, se necesita incluir el término $n=0$, por lo que si sumamos un cero
\[ \begin{split}
E(k) &= \dfrac{\pi}{2} - \dfrac{1}{4} \sum_{n=0}^{\infty} \dfrac{\Gamma(n+\frac{1}{2}) \Gamma(n-\frac{1}{2})}{\Gamma(n+1) \Gamma(n+1)} (k^{2})^{n} + \dfrac{1}{4} \left[ \dfrac{\Gamma(\frac{1}{2}) \Gamma(-\frac{1}{2})}{\Gamma(1) \Gamma(1)} (k^{2})^{0}  \right] \\
&= - \dfrac{1}{4} \sum_{n=0}^{\infty} \dfrac{\Gamma(n + \frac{1}{2}) \Gamma(n - \frac{1}{2})}{\Gamma(n+1) \Gamma(n+1)} (k^{2})^{n}
\end{split} \]

ya que $\Gamma(- \frac{1}{2} ) = -2 \Gamma( \frac{1}{2} ) = - 2 \sqrt{\pi}$
Con excepción de una constante de multiplicación, la suma es la función hipergeométrica con $\alpha=\frac{1}{2}= \beta$ y $\gamma=1$. Agregando la constante 
\[ \dfrac{\Gamma(1)}{\Gamma( \frac{1}{2} )\Gamma(- \frac{1}{2}) } = \dfrac{1}{(-2 \pi)} \]
Obtenemos
\begin{equation}
E(k) = \dfrac{\pi}{4} F(\frac{1}{2}, -\frac{1}{2}; 1; k^{2})
\label{eq:ecuacion_11_25}
\end{equation}
Queda por verificar que
\begin{equation}
K(k) = \dfrac{\pi}{4} F(\frac{1}{2}, -\frac{1}{2}; 1; k^{2})
\label{ecuacion_11_26}
\end{equation}
\end{document}