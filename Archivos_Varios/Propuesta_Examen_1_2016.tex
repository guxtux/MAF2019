\documentclass[12pt]{article}
\usepackage[utf8]{inputenc}
\usepackage[spanish,es-lcroman, es-tabla]{babel}
\usepackage[autostyle,spanish=mexican]{csquotes}
\usepackage{amsmath}
\usepackage{amssymb}
\usepackage{nccmath}
\numberwithin{equation}{section}
\usepackage{amsthm}
\usepackage{graphicx}
\usepackage{epstopdf}
\DeclareGraphicsExtensions{.pdf,.png,.jpg,.eps}
\usepackage{color}
\usepackage{float}
\usepackage{multicol}
\usepackage{enumerate}
\usepackage[shortlabels]{enumitem}
\usepackage{anyfontsize}
\usepackage{anysize}
\usepackage{array}
\usepackage{multirow}
\usepackage{enumitem}
\usepackage{cancel}
\usepackage{tikz}
\usepackage{circuitikz}
\usepackage{tikz-3dplot}
\usetikzlibrary{babel}
\usetikzlibrary{shapes}
\usepackage{bm}
\usepackage{mathtools}
\usepackage{esvect}
\usepackage{hyperref}
\usepackage{relsize}
\usepackage{siunitx}
\usepackage{physics}
%\usepackage{biblatex}
\usepackage{standalone}
\usepackage{mathrsfs}
\usepackage{bigints}
\usepackage{bookmark}
\spanishdecimal{.}

\setlist[enumerate]{itemsep=0mm}

\renewcommand{\baselinestretch}{1.5}

\let\oldbibliography\thebibliography

\renewcommand{\thebibliography}[1]{\oldbibliography{#1}

\setlength{\itemsep}{0pt}}
%\marginsize{1.5cm}{1.5cm}{2cm}{2cm}


\newtheorem{defi}{{\it Definición}}[section]
\newtheorem{teo}{{\it Teorema}}[section]
\newtheorem{ejemplo}{{\it Ejemplo}}[section]
\newtheorem{propiedad}{{\it Propiedad}}[section]
\newtheorem{lema}{{\it Lema}}[section]

\usepackage{enumerate}
%\usepackage[shortlabels]{enumitem}
\usepackage{pifont}
\renewcommand{\labelitemi}{\ding{43}}
%\author{M. en C. Gustavo Contreras Mayén. \texttt{curso.fisica.comp@gmail.com}}
\title{{Examen Parcial 1} \\ {\large Matemáticas Avanzadas de la Física}}
\date{ }
\begin{document}
\vspace{-4cm}
%\renewcommand\theenumii{\arabic{theenumii.enumii}}
\renewcommand\labelenumii{\theenumi.{\arabic{enumii}}}
\maketitle
\fontsize{14}{14}\selectfont
\textbf{Indicaciones:}
\begin{itemize}
\item Responde lo más claro posible cada una de las preguntas.
\item Expresa con tus propias palabras las ideas e interpretaciones que consideres.
\item Este primer examen cubre los dos primeros temas del curso.
\end{itemize}
\begin{enumerate}
\item \textbf{(1 punto.)} ¿Qué es un factor de escala?
\item \textbf{(1 punto.)} Escribe la ecuación de Helmholtz para un sistema de coordenadas generalizado.
\item \textbf{(2 puntos.)} La forma general de una ecuación diferencial de segundo orden es
\[ \begin{split} & A(x,y) \dfrac{\partial^{2} u(x,y)}{\partial x^{2}}  + 2B(x,y) \dfrac{\partial^{2} u(x,y)}{\partial x \partial y} + C(x,y) \dfrac{\partial^{2} u(x,y)}{\partial y^{2}} + \\
&+ a(x,y) \dfrac{\partial u(x,y)}{\partial x} + b(x,y) \dfrac{\partial u(x,y)}{\partial y} + c(x,y)u(x,y) = f(x,y) \end{split} \]
Se dice que la ecuación es:
\begin{enumerate} [label=\alph*)]
\item \textbf{Hiperbólica}, si $\Delta = B^{2} - AC > 0$.
\item \textbf{Parabólica}, si $\Delta = B^{2} - AC = 0$.
\item \textbf{Elíptica}, si $\Delta = B^{2} - AC < 0$.
\end{enumerate}
Indica el tipo de ecuación para cada una de las siguientes expresiones (explora todas las posibilidades):
\begin{enumerate}[label=\alph*)]
\item Para $ \dfrac{\partial^{2} u(x,y)}{\partial x^{2}} = \dfrac{\partial^{2} u(x,y)}{\partial y^{2}}$
\item Para $\dfrac{\partial^{2} u(x,y)}{\partial x \partial y} = 0$
\item Para $ \dfrac{\partial^{2} u(x,y)}{\partial x^{2}} + \dfrac{\partial^{2} u(x,y)}{\partial y^{2}} = 0$
\item Para $ y \dfrac{\partial^{2} u(x,y)}{\partial x^{2}} + \dfrac{\partial^{2} u(x,y)}{\partial y^{2}} = 0$
\end{enumerate}
\item \textbf{(1 punto.)} ¿Cómo se realiza el método de separación de variables con una ecuación diferencial parcial de segundo orden homogénea? ¿Por qué funciona este método?
\item \textbf{(2 puntos.)} Dada una ecuación diferencial de la forma
\[ p(x) y'' + q(x) y' + r(x) y = 0 \]
\begin{enumerate}[label=\alph*)]
\item Indica cuáles son las singularidades.
\item ¿Cuáles son los criterios para removerlas?
\item En la solución se construye la ecuación indicial, ¿qué es esta ecuación? y ¿qué representa su solución?
\end{enumerate}
\item \textbf{(1 punto.)} Platica cómo se lleva acabo el método de Frobenius para remover singularidades en el infinito.
\item \textbf{(1 punto.)} Explica en qué consiste el teorema del desarrollo, indicando cuáles son las propiedades de las funciones que se emplean para este teorema, menciona cómo se obtienen los coeficientes del desarrollo.
\item \textbf{(1 punto.)} ¿Qué es la delta de Dirac?, ¿Para qué nos sirve?
\end{enumerate}
\end{document}