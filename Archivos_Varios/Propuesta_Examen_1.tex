\documentclass[12pt]{article}
\usepackage[utf8]{inputenc}
\usepackage[spanish,es-lcroman, es-tabla]{babel}
\usepackage[autostyle,spanish=mexican]{csquotes}
\usepackage{amsmath}
\usepackage{amssymb}
\usepackage{nccmath}
\numberwithin{equation}{section}
\usepackage{amsthm}
\usepackage{graphicx}
\usepackage{epstopdf}
\DeclareGraphicsExtensions{.pdf,.png,.jpg,.eps}
\usepackage{color}
\usepackage{float}
\usepackage{multicol}
\usepackage{enumerate}
\usepackage[shortlabels]{enumitem}
\usepackage{anyfontsize}
\usepackage{anysize}
\usepackage{array}
\usepackage{multirow}
\usepackage{enumitem}
\usepackage{cancel}
\usepackage{tikz}
\usepackage{circuitikz}
\usepackage{tikz-3dplot}
\usetikzlibrary{babel}
\usetikzlibrary{shapes}
\usepackage{bm}
\usepackage{mathtools}
\usepackage{esvect}
\usepackage{hyperref}
\usepackage{relsize}
\usepackage{siunitx}
\usepackage{physics}
%\usepackage{biblatex}
\usepackage{standalone}
\usepackage{mathrsfs}
\usepackage{bigints}
\usepackage{bookmark}
\spanishdecimal{.}

\setlist[enumerate]{itemsep=0mm}

\renewcommand{\baselinestretch}{1.5}

\let\oldbibliography\thebibliography

\renewcommand{\thebibliography}[1]{\oldbibliography{#1}

\setlength{\itemsep}{0pt}}
%\marginsize{1.5cm}{1.5cm}{2cm}{2cm}


\newtheorem{defi}{{\it Definición}}[section]
\newtheorem{teo}{{\it Teorema}}[section]
\newtheorem{ejemplo}{{\it Ejemplo}}[section]
\newtheorem{propiedad}{{\it Propiedad}}[section]
\newtheorem{lema}{{\it Lema}}[section]

\usepackage{enumerate}
%\usepackage[shortlabels]{enumitem}
\usepackage{pifont}
\renewcommand{\labelitemi}{\ding{43}}
%\author{M. en C. Gustavo Contreras Mayén. \texttt{curso.fisica.comp@gmail.com}}
\title{{Examen Parcial 1} \\ {\large Matemáticas Avanzadas de la Física}}
\date{ }
\begin{document}
\vspace{-4cm}
%\renewcommand\theenumii{\arabic{theenumii.enumii}}
\renewcommand\labelenumii{\theenumi.{\arabic{enumii}}}
\maketitle
\fontsize{14}{14}\selectfont
\textbf{Indicaciones:}
\begin{itemize}
\item Responde lo más claro posible cada una de las preguntas.
\item Expresa con tus propias palabras las ideas e interpretaciones que consideres.
\item Este primer examen cubre los tres temas iniciales del curso.
\end{itemize}
\begin{enumerate}
\item \begin{enumerate} [label=\alph*)]
\item ¿Qué es un factor de escala?
\item Escribe la ecuación de Helmholtz para un sistema de coordenadas generalizado.
\end{enumerate}
\item La forma general de una ecuación diferencial de segundo orden es
\[ \begin{split} & A(x,y) \dfrac{\partial^{2} u(x,y)}{\partial x^{2}}  + 2B(x,y) \dfrac{\partial^{2} u(x,y)}{\partial x \partial y} + C(x,y) \dfrac{\partial^{2} u(x,y)}{\partial y^{2}} + \\
&+ a(x,y) \dfrac{\partial u(x,y)}{\partial x} + b(x,y) \dfrac{\partial u(x,y)}{\partial y} + c(x,y)u(x,y) = f(x,y) \end{split} \]
Se dice que la ecuación es:
\begin{enumerate} [label=\alph*)]
\item \textbf{Hiperbólica}, si $\Delta = B^{2} - AC > 0$.
\item \textbf{Parabólica}, si $\Delta = B^{2} - AC = 0$.
\item \textbf{Elíptica}, si $\Delta = B^{2} - AC < 0$.
\end{enumerate}
Indica el tipo de ecuación para cada una de las siguientes expresiones (explora todas las posibilidades):
\begin{enumerate}[label=\alph*)]
\item Para $ \dfrac{\partial^{2} u(x,y)}{\partial x^{2}} = \dfrac{\partial^{2} u(x,y)}{\partial y^{2}}$
\item Para $\dfrac{\partial^{2} u(x,y)}{\partial x \partial y} = 0$
\item Para $ \dfrac{\partial^{2} u(x,y)}{\partial x^{2}} + \dfrac{\partial^{2} u(x,y)}{\partial y^{2}} = 0$
\item Para $ y \dfrac{\partial^{2} u(x,y)}{\partial x^{2}} + \dfrac{\partial^{2} u(x,y)}{\partial y^{2}} = 0$
\end{enumerate}
\item ¿Por qué funciona el método de separación de variables con una ecuación diferencial parcial de segundo orden homogénea?
\item En términos prácticos, ¿qué representa el valor propio ($\lambda$, eigenvalor)?
\item Dada una ecuación diferencial de la forma
\[ p(x) y'' + q(x) y' + r(x) y = 0 \]
\begin{enumerate}[label=\alph*)]
\item Indica cuáles son las singularidades.
\item ¿Cuáles son los criterios para removerlas?
\item En la solución se construye la ecuación indicial, ¿qué es esta ecuación? y ¿qué representa su solución?
\end{enumerate}
\item Platica cómo se lleva acabo el método de Frobenius para remover singularidades en el infinito.
\item El método de Frobenius y las soluciones en series de Fourier aplicados a EDP2, ¿qué es lo que nos extienden?
\item ¿Qué es la delta de Dirac?, ¿Para qué nos sirve?
\item ¿Qué nos conviene más: usar funciones ortogonales u ortonormales?
\item En la construcción de los operadores Hermitianos, se mencionaron tres características importantes: ¿cuáles son? ¿en qué consisten? y ¿cuál es la ventaja de usarlos en la física?
\end{enumerate}
\end{document}