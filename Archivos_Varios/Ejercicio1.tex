\documentclass[12pt]{article}
\usepackage[utf8]{inputenc}
\usepackage[spanish,es-lcroman, es-tabla]{babel}
\usepackage[autostyle,spanish=mexican]{csquotes}
\usepackage{amsmath}
\usepackage{amssymb}
\usepackage{nccmath}
\numberwithin{equation}{section}
\usepackage{amsthm}
\usepackage{graphicx}
\usepackage{epstopdf}
\DeclareGraphicsExtensions{.pdf,.png,.jpg,.eps}
\usepackage{color}
\usepackage{float}
\usepackage{multicol}
\usepackage{enumerate}
\usepackage[shortlabels]{enumitem}
\usepackage{anyfontsize}
\usepackage{anysize}
\usepackage{array}
\usepackage{multirow}
\usepackage{enumitem}
\usepackage{cancel}
\usepackage{tikz}
\usepackage{circuitikz}
\usepackage{tikz-3dplot}
\usetikzlibrary{babel}
\usetikzlibrary{shapes}
\usepackage{bm}
\usepackage{mathtools}
\usepackage{esvect}
\usepackage{hyperref}
\usepackage{relsize}
\usepackage{siunitx}
\usepackage{physics}
%\usepackage{biblatex}
\usepackage{standalone}
\usepackage{mathrsfs}
\usepackage{bigints}
\usepackage{bookmark}
\spanishdecimal{.}

\setlist[enumerate]{itemsep=0mm}

\renewcommand{\baselinestretch}{1.5}

\let\oldbibliography\thebibliography

\renewcommand{\thebibliography}[1]{\oldbibliography{#1}

\setlength{\itemsep}{0pt}}
%\marginsize{1.5cm}{1.5cm}{2cm}{2cm}


\newtheorem{defi}{{\it Definición}}[section]
\newtheorem{teo}{{\it Teorema}}[section]
\newtheorem{ejemplo}{{\it Ejemplo}}[section]
\newtheorem{propiedad}{{\it Propiedad}}[section]
\newtheorem{lema}{{\it Lema}}[section]

\title{Matemáticas Avanzadas de la Física \\ {\large Ejercicio}}
\date{ }
\begin{document}
\maketitle
\fontsize{14}{14}\selectfont
En el espacio de Minkowski definimos $x_{1}=x$, $x_{2}=y$, $x_{3}=z$, $x_{4}=ict$. Esto hace que el intervalo de espacio-tiempo $d^{2}= dx^{2}+dy^{2}+dz^{2}-c^{2}dt^{2}$, (donde $c$ es la velocidad de la luz) llega a ser $d^{2} = \sum_{i=1}^{4} dx_{i}^{2}$.
\\
\\
Demuestra que la métrica en el espacio de Minkowski es $g_{ij} = \delta_{ij}$ o
\[ (g_{ij}) = \begin{pmatrix}
1 & 0 & 0 & 0 \\
0 & 1 & 0 & 0 \\
0 & 0 & 1 & 0 \\
0 & 0 & 0 & 1
\end{pmatrix} \]
\\
\\
Partimos del hecho que $ds^{2} = \sum\limits_{ij} g_{ij} dq_{i}dq_{j}$ y que los coeficientes $g_{ij}$ son 
\[ g_{ij} = \dfrac{\partial x}{\partial q_{i}} \dfrac{\partial x}{\partial q_{j}} + \dfrac{\partial y}{\partial q_{i}} \dfrac{\partial y}{\partial q_{j}} + \dfrac{\partial z}{\partial q_{i}} \dfrac{\partial z}{\partial q_{j}} + \dfrac{\partial T}{\partial q_{i}} \dfrac{\partial T}{\partial q_{j}}\]
\\
Solución:
\\
Lo que tenemos que hacer es obtener los sumandos del elemento de distancia, pero también necesitamos los coeficientes $g_{ij}$:
\begin{eqnarray*}
ds^{2} &=& g_{11} dq_{1}d_{1} = \left( \cancelto{1}{\dfrac{\partial x}{\partial q_{1}} \dfrac{\partial x}{\partial q_{1}}} + \cancelto{0}{\dfrac{\partial y}{\partial q_{1}} \dfrac{\partial y}{\partial q_{1}}} + \cancelto{0}{\dfrac{\partial z}{\partial q_{1}} \dfrac{\partial z}{\partial q_{1}}} + \cancelto{0}{\dfrac{\partial T}{\partial q_{1}} \dfrac{\partial T}{\partial q_{1}}} \right) d x^{2} + \\
&+& g_{12} dq_{1}d_{2} +  g_{13} dq_{1}d_{3} + g_{14} dq_{1}d_{4} + \\
&+& g_{21} dq_{2}d_{1} + \\
&+& g_{22} dq_{2}d_{2} = \left( \cancelto{0}{\dfrac{\partial x}{\partial q_{2}} \dfrac{\partial x}{\partial q_{2}}} + \cancelto{1}{\dfrac{\partial y}{\partial q_{2}} \dfrac{\partial y}{\partial q_{2}}} + \cancelto{0}{\dfrac{\partial z}{\partial q_{2}} \dfrac{\partial z}{\partial q_{2}}} + \cancelto{0}{\dfrac{\partial T}{\partial q_{2}} \dfrac{\partial T}{\partial q_{2}}} \right) d y^{2} + \ldots + \\
&+& \ldots g_{33} dq_{3}d_{3} = \left( \cancelto{0}{\dfrac{\partial x}{\partial q_{3}} \dfrac{\partial x}{\partial q_{3}}} + \cancelto{0}{\dfrac{\partial y}{\partial q_{3}} \dfrac{\partial y}{\partial q_{3}}} + \cancelto{1}{\dfrac{\partial z}{\partial q_{3}} \dfrac{\partial z}{\partial q_{3}}} + \cancelto{0}{\dfrac{\partial T}{\partial q_{3}} \dfrac{\partial T}{\partial q_{3}}} \right) d z^{2} + \ldots + \\
&+& \ldots g_{44} dq_{4}d_{4} = \left( \cancelto{0}{\dfrac{\partial x}{\partial q_{4}} \dfrac{\partial x}{\partial q_{4}}} + \cancelto{0}{\dfrac{\partial y}{\partial q_{4}} \dfrac{\partial y}{\partial q_{4}}} + \cancelto{0}{\dfrac{\partial z}{\partial q_{4}} \dfrac{\partial z}{\partial q_{4}}} + \cancelto{1}{\dfrac{\partial T}{\partial q_{4}} \dfrac{\partial T}{\partial q_{4}}} \right) (ic)^{2} d t^{2}
\end{eqnarray*}
Los términos entre paréntesis son los que ''sobreviven'' y valen uno, los demás se cancelan, ya que al derivar con respecto a las variables cruzadas, el resultado es cero.
\\
\\
Por tanto el conjunto de factores $g_{ij} = \delta_{ij}$.
\\
\\
Nota: como se menciona que se tiene un espacio de Minkowski, el que la matriz
\[ (g_{ij}) = \begin{pmatrix}
1 & 0 & 0 & 0 \\
0 & 1 & 0 & 0 \\
0 & 0 & 1 & 0 \\
0 & 0 & 0 & 1
\end{pmatrix} \]
tenga un valor de $1$ en la cuarta columna, no dice que recuperemos el espacio euclídeo, sino más bien, el espacio de Minkowski permance sin alteraciones ya que el valor de esa componente es $ict$.



















\end{document}