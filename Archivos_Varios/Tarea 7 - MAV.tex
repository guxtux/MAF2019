\documentclass[12pt]{article}
\usepackage[utf8]{inputenc}
\usepackage[spanish,es-lcroman, es-tabla]{babel}
\usepackage[autostyle,spanish=mexican]{csquotes}
\usepackage{amsmath}
\usepackage{amssymb}
\usepackage{nccmath}
\numberwithin{equation}{section}
\usepackage{amsthm}
\usepackage{graphicx}
\usepackage{epstopdf}
\DeclareGraphicsExtensions{.pdf,.png,.jpg,.eps}
\usepackage{color}
\usepackage{float}
\usepackage{multicol}
\usepackage{enumerate}
\usepackage[shortlabels]{enumitem}
\usepackage{anyfontsize}
\usepackage{anysize}
\usepackage{array}
\usepackage{multirow}
\usepackage{enumitem}
\usepackage{cancel}
\usepackage{tikz}
\usepackage{circuitikz}
\usepackage{tikz-3dplot}
\usetikzlibrary{babel}
\usetikzlibrary{shapes}
\usepackage{bm}
\usepackage{mathtools}
\usepackage{esvect}
\usepackage{hyperref}
\usepackage{relsize}
\usepackage{siunitx}
\usepackage{physics}
%\usepackage{biblatex}
\usepackage{standalone}
\usepackage{mathrsfs}
\usepackage{bigints}
\usepackage{bookmark}
\spanishdecimal{.}

\setlist[enumerate]{itemsep=0mm}

\renewcommand{\baselinestretch}{1.5}

\let\oldbibliography\thebibliography

\renewcommand{\thebibliography}[1]{\oldbibliography{#1}

\setlength{\itemsep}{0pt}}
%\marginsize{1.5cm}{1.5cm}{2cm}{2cm}


\newtheorem{defi}{{\it Definición}}[section]
\newtheorem{teo}{{\it Teorema}}[section]
\newtheorem{ejemplo}{{\it Ejemplo}}[section]
\newtheorem{propiedad}{{\it Propiedad}}[section]
\newtheorem{lema}{{\it Lema}}[section]

\usepackage{enumerate}
\usepackage{pifont}
\renewcommand{\labelitemi}{\ding{43}}
%\author{M. en C. Gustavo Contreras Mayén. \texttt{curso.fisica.comp@gmail.com}}
\title{{Tarea Examen Transformadas Integrales} \\ {\large Matemáticas Avanzadas de la Física}}
\date{ }
\begin{document}
%\renewcommand\theenumii{\arabic{theenumii.enumii}}
\renewcommand\labelenumii{\theenumi.{\arabic{enumii}})}
\maketitle
\fontsize{14}{14}\selectfont
Fecha de entrega: \textbf{Viernes 22 de mayo de 2015.} En el cubículo de Abraham a las 3 pm (hora del DF)
\\
\begin{enumerate}

\item Para cada una de las siguentes funciones determina si tiene una transformada de Laplace. En caso de que exista, indica cuál es, en caso de que, explica el por qué:
\begin{enumerate}
\item $ln(t)$
\item $e^{3t}$
\item $e^{t^{2}}$
\item $e^{1/t}$
\item $1/t$
\item $f(t) = \begin{cases} 1 & \mbox{si t es par} \\ 0 & \mbox{si t es impar} \end{cases}$
\end{enumerate}
\item La función
\[ f(x) = \begin{cases}
1, & \vert x \vert < 1 \\
0, & \vert x \vert > 0
\end{cases} \]
es una función simétrica finita.
\begin{enumerate}
\item Calcular $g_{c}(\omega)$ la transformada coseno de Fourier de $f(x)$.
\item\label{itm:inciso2} Con la transformada coseno inversa, demostrar que
\[ f(x) = \dfrac{2}{\pi} \int_{0}^{\infty} \dfrac{\sin \omega \cos \omega x}{\omega} d \omega \] 
\item Del inciso (2.2) demostrar que
\[ \int_{0}^{\infty} \dfrac{\sin \omega \cos \omega x}{\omega} d \omega = 
\begin{cases}
0, & \vert x \vert > 1 \\
\dfrac{\pi}{4}, & \vert x \vert = 1 \\
\dfrac{\pi}{2}, & \vert x \vert < 1
\end{cases} \]
\end{enumerate}
\item Resolver con la transformada de Laplace la ecuación diferencial de un oscilador armónico amortiguado
\[ m X''(t) + b X'(t) + k X(t) = 0 \]
con las condiciones $X(0) = X_{0}$, $X'(0) = 0$ y
\begin{enumerate}
\item $b^{2} = 4 km$ amortiguamiento crítico.
\item $b^{2} > 4 km$ sobreamortiguado
\end{enumerate}
\item Una bala de masa $m$ es disparada por un cañón con una velocidad $v_{0}$ dentro de un medio viscoso. Se sabe que el desplazamiento $y(t)$ en el tiempo $t > 0$ de la bala satisface la ecuación diferencial
\[ m y'' + k y' = 0 \]
donde $y(0)= 0$, $y'(0) =  v_{0}$. Calcula con trasformada de Laplace $y(t)$.
\item Si la corriente eléctrica $I$ en un circuito eléctrico es
\[ L \dfrac{dI}{dt} +  \dfrac{1}{C} \int_{0}^{t} I(u) du =  E \]
donde $L,C,E$ son constantes positivas e $I(0)=0$. Calcula el valor de la corriente eléctrica $I(t)$ con la transformada de Laplace.
\]
\end{enumerate}
\end{document}