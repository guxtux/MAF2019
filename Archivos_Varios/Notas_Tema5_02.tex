\documentclass[12pt]{article}
\usepackage[utf8]{inputenc}
\usepackage[spanish,es-lcroman, es-tabla]{babel}
\usepackage[autostyle,spanish=mexican]{csquotes}
\usepackage{amsmath}
\usepackage{amssymb}
\usepackage{nccmath}
\numberwithin{equation}{section}
\usepackage{amsthm}
\usepackage{graphicx}
\usepackage{epstopdf}
\DeclareGraphicsExtensions{.pdf,.png,.jpg,.eps}
\usepackage{color}
\usepackage{float}
\usepackage{multicol}
\usepackage{enumerate}
\usepackage[shortlabels]{enumitem}
\usepackage{anyfontsize}
\usepackage{anysize}
\usepackage{array}
\usepackage{multirow}
\usepackage{enumitem}
\usepackage{cancel}
\usepackage{tikz}
\usepackage{circuitikz}
\usepackage{tikz-3dplot}
\usetikzlibrary{babel}
\usetikzlibrary{shapes}
\usepackage{bm}
\usepackage{mathtools}
\usepackage{esvect}
\usepackage{hyperref}
\usepackage{relsize}
\usepackage{siunitx}
\usepackage{physics}
%\usepackage{biblatex}
\usepackage{standalone}
\usepackage{mathrsfs}
\usepackage{bigints}
\usepackage{bookmark}
\spanishdecimal{.}

\setlist[enumerate]{itemsep=0mm}

\renewcommand{\baselinestretch}{1.5}

\let\oldbibliography\thebibliography

\renewcommand{\thebibliography}[1]{\oldbibliography{#1}

\setlength{\itemsep}{0pt}}
%\marginsize{1.5cm}{1.5cm}{2cm}{2cm}


\newtheorem{defi}{{\it Definición}}[section]
\newtheorem{teo}{{\it Teorema}}[section]
\newtheorem{ejemplo}{{\it Ejemplo}}[section]
\newtheorem{propiedad}{{\it Propiedad}}[section]
\newtheorem{lema}{{\it Lema}}[section]

\usepackage{mathrsfs}
\usepackage{bigints}
\spanishdecimal{.}
%\usepackage{enumerate}
%\author{M. en C. Gustavo Contreras Mayén. \texttt{curso.fisica.comp@gmail.com}}
\title{Funciones de Bessel 2 \\ {\large Matemáticas Avanzadas de la Física}}
\date{ }
\begin{document}
%\renewcommand\theenumii{\arabic{theenumii.enumii}}
\renewcommand\labelenumii{\theenumi.{\arabic{enumii}}}
\maketitle
\fontsize{14}{14}\selectfont
\section{Ortogonalidad de las funciones de Bessel}
Si la ecuación de Bessel 
\begin{equation}
\rho^{2} \dfrac{d^{2}}{d \rho^{2}} Z_{v} (k \rho) + \rho \dfrac{d}{d \rho} Z_{v} (k \rho) + (k^{2} \rho^{2} - v^{2}) Z_{v} (k \rho) = 0
\label{eq:ecuacion_11_22a}
\end{equation}
se divide entre $x$, vemos que se convierte en autoadjunta, y por lo visto en la teoría de Sturm-Liouville, se espera que las soluciones sean ortogonales, si logramos ajustar las adecuadas condiciones de frontera.
\\
Teniendo cuidado en las condiciones de frontera, para un intervalo finito $[0,a]$, introducimos los parámetros $a$y $\alpha_{vm}$ en el argumento de $J_{v}$ para obtener $J_{v}(\alpha_{vm} \rho /a)$. Donde $a$ es el límite superior de la coordenada radial cilíndrica $\rho$.
\\
De la ecuación (\ref{eq:ecuacion_11_22a})
\begin{equation}
\rho \dfrac{d^{2}}{d \rho^{2}} J_{v} \left( \alpha_{vm} \dfrac{\rho}{a} \right) + \dfrac{d}{d \rho} J_{v} \left( \alpha_{vm} \dfrac{\rho}{a} \right) + \left( \dfrac{\alpha_{vm}^{2} \rho}{a^{2}} - \dfrac{v^{2}}{\rho} \right) J_{v} \left( \alpha_{vm} \dfrac{\rho}{a} \right) = 0
\label{eq:ecuacion_11_45}
\end{equation}
Cambiando el parámetro $\alpha_{vm}$ por $\alpha_{vn}$, encontramos que $J_{v}(\alpha_{vn} \rho /a)$ satisface
\begin{equation}
\rho \dfrac{d^{2}}{d \rho^{2}} J_{v} \left( \alpha_{vn} \dfrac{\rho}{a} \right) + \dfrac{d}{d \rho} J_{v} \left( \alpha_{vn} \dfrac{\rho}{a} \right) + \left( \dfrac{\alpha_{vn}^{2} \rho}{a^{2}} - \dfrac{v^{2}}{\rho} \right) J_{v} \left( \alpha_{vn} \dfrac{\rho}{a} \right) = 0
\label{eq:ecuacion_11_45a}
\end{equation}
Como vimos anteriormente, multiplicamos la ecuación (\ref{eq:ecuacion_11_45}) por $J_{v}(\alpha_{vn} \rho /a)$ y la ecuación (\ref{eq:ecuacion_11_45a}) por $J_{v}(\alpha_{vm} \rho /a)$, para luego restarlarla y obtener
\begin{eqnarray}
\begin{aligned}
J_{v} \left( \alpha_{vn} \dfrac{\rho}{a} \right) &  \dfrac{d}{d \rho} \left[ \rho \dfrac{d}{d \rho} J_{v} \left( \alpha_{vm} \dfrac{\rho}{a} \right) \right] - J_{v} \left( \alpha_{vm} \dfrac{\rho}{a} \right) \dfrac{d}{d \rho} \left[ \rho \dfrac{d}{d \rho} J_{v} \left( \alpha_{vn} \dfrac{\rho}{a} \right) \right] \\
&= \dfrac{\alpha^{2}_{vn} - \alpha^{2}_{vm}}{a^{2}} \rho J_{v} \left( \alpha_{vm} \dfrac{\rho}{a} \right) J_{v} \left( \alpha_{vn} \dfrac{\rho}{a} \right)
\end{aligned}
\label{eq:ecuacion_11_46}
\end{eqnarray}
Integrando de $\rho = 0$ a $\rho = a$, se obtiene
\begin{eqnarray}
\begin{aligned}
\int_{0}^{a} J_{v} \left( \alpha_{vn} \dfrac{\rho}{a} \right) &  \dfrac{d}{d \rho} \left[ \rho \dfrac{d}{d \rho} J_{v} \left( \alpha_{vm} \dfrac{\rho}{a} \right) \right] d \rho \\
& - \int_{0}^{a} J_{v} \left( \alpha_{vm} \dfrac{\rho}{a} \right) \dfrac{d}{d \rho} \left[ \rho \dfrac{d}{d \rho} J_{v} \left( \alpha_{vn} \dfrac{\rho}{a} \right) \right] d \rho \\
&= \dfrac{\alpha^{2}_{vn} - \alpha^{2}_{vm}}{a^{2}} \int_{0}^{a} J_{v} \left( \alpha_{vm} \dfrac{\rho}{a} \right) J_{v} \left( \alpha_{vn} \dfrac{\rho}{a} \right) \rho d \rho
\end{aligned}
\label{eq:ecuacion_11_47}
\end{eqnarray}
De la integración por partes, vemos que el lado izquierdo de la ecuación (\ref{eq:ecuacion_11_47}) es
\begin{equation}
\bigg\vert \rho J_{v} \left( \alpha_{vn} \dfrac{\rho}{a} \right) \dfrac{d}{d \rho} J_{v} \left( \alpha_{vm} \dfrac{\rho}{a} \right) \bigg\vert_{0}^{a} - \bigg\vert \rho J_{v} \left( \alpha_{vm} \dfrac{\rho}{a} \right) \dfrac{d}{d \rho} J_{v} \left( \alpha_{vn} \dfrac{\rho}{a} \right) \bigg\vert_{0}^{a} 
\label{eq:ecuacion_11_48}
\end{equation}
Para $v \geq 0$ el factor $\rho$ garantiza un cero en el límite inferior $\rho = 0$. De hecho el límite inferior en el índice $v$ puede reducirse a $v>-1$. En $\rho = a$, cada una de las expresiones se anulan si elegimos los parámetros $\alpha_{vn}$ y $\alpha_{vm}$ como ceros de las raíces de $J_{v}$, esto es, $J_{v}(\alpha_{vm})=0$. Los subíndices ahora toman significado: $\alpha_{vm}$ es el $m$-ésimo cero de $J_{v}$.
\\
Al elegir estos parámetros, el lado izquierdo de la igualdad se anula (las condiciones de frontera de Sturm-Loiuville se satisfacen) y para $m \neq n$
\begin{equation}
\int_{0}^{a} J_{v} \left( \alpha_{vm} \dfrac{\rho}{a} \right) J_{v} \left( \alpha_{vn} \dfrac{\rho}{a} \right) \rho d \rho = 0
\label{eq:ecuacion_11_49}
\end{equation}
Lo que nos proporciona la ortogonalidad en el intervalo $[0,a]$.
\section{Normalización.}
La normalización de la integral se desarrolla mediante el ajuste de $\alpha_{vn} = \alpha_{vm} + \varepsilon$ en la ecuación (\ref{eq:ecuacion_11_48}), y considerando el límite $\varepsilon \to 0$. Apoyándose de la relación de recuerrencia
\begin{equation}
J_{n+1}(x) = \dfrac{n}{x} J_{n}(x) - J'_{n}(x)
\label{eq:ecuacion_11_16}
\end{equation}
podemos escribir el resultado como
\begin{equation}
\int_{0}^{a} \left[ J_{v} \left( \alpha_{vm} \dfrac{\rho}{a} \right) \right] \rho d \rho = \dfrac{a^{2}}{2} [ J_{v+1} (\alpha_{vm} )]^{2}
\label{eq:ecuacion_11_50}
\end{equation}
\section{Series de Bessel.}
Si suponemos que el conjunto de funciones de Bessel $J_{v} (\alpha_{vm} \rho /a$ (con $v$ fijo, $m =1,2,3,\ldots$ es un conjunto completo, y que cualquier otra función arbitraria $f(\rho)$ es bien portada, podemos expandir el resultado en una serie de Bessel (Bessel-Fourier o Fourier-Bessel)
\begin{equation}
f(\rho) = \sum_{m=1}^{\infty} c_{vm} J_{v} \left( \alpha_{vm} \dfrac{\rho}{a} \right) \hspace{1cm} 0 \leq \rho \leq a, \hspace{0.5cm} v > -1
\label{eq:ecuacion_11_51}
\end{equation}
Los coeficientes $c_{vm}$ se determinan usando la ecuación (\ref{eq:ecuacion_11_50})
\begin{equation}
c_{vm} = \dfrac{2}{a^{2}[J_{v+1} (\alpha_{vm})]^{2}} \int_{0}^{a} f(\rho) J_{v} \left( \alpha_{vm} \dfrac{\rho}{a} \right) \rho d \rho 
\label{eq:ecuacion_11_52}
\end{equation}
\section{Funciones de Neumann. Funciones de Bessel de segunda clase $N_{v}(x)$.}
De la teoría de las ecuaciones diferenciales, sabemos que la ecuación de Bessel tiene dos soluciones independientes. De hecho para el orden $v$ no entero, se cuenta con dos soluciones $J_{v}(x)$ y $J_{-v}(x)$ usando series infinitas. El problema radica cuando $v$ es entero, de donde ya hemos obtenido una solución.
\\
Una segunda solución puede obtenerse por los métodos que hemos visto en el tema anterior, lo que nos entrega una segunda solución para la ecuación de Bessel, pero que no son de la forma usual.
\\
\textbf{Definición:}
Tomando una combinación particular de $J_{v}$ y $J_{v}$
\begin{equation}
N_{v} (x) = \dfrac{\cos v \pi J_{v}(x) - J_{-v} (x)}{\sin v \pi}
\label{eq:ecuacion_11_60}
\end{equation}
Esta es la función de Neumman. Para un valor de $v$ no entero, $N_{v}(x)$ satisface la ecuación de Bessel, por que es una combinación lineal de las soluciones conocidas $J_{v}(x)$ y $J_{-v}(x)$.
\\
Para un valor de $v$ entero, $v=n$, usando la ecuación
\begin{equation}
J_{-n}(x) = (-1)^{n} J_{n}(x) \hspace{1.5cm} \text{con $n$ entero}
\label{eq:ecuacion_11_8}
\end{equation}
la ecuación (\ref{eq:ecuacion_11_60}) se vuelve indeterminada. La definición de $N_{v}(x)$ se elige deliberadamente para esta propiedad. Evaluando $N_{n}(x)$ con la regla de L'Hopital para expresiones indeterminadas, tenemos
\begin{eqnarray}
\begin{aligned}
N_{n}(x) &= \dfrac{(d/d v)[\cos v \pi J_{v}(x) - J_{-v}(x)]}{(d/dv) \sin v \pi)} \bigg\vert_{v=n} \\
&= \dfrac{-\pi \sin n \pi J_{n}(x) + [\cos n \pi \partial J_{v} / \partial v - \partial J_{-v} / \partial v]}{\pi \cos n \pi} \bigg\vert_{v=n} \\
&= \dfrac{1}{\pi} \left[ \dfrac{\partial J_{v}(x)}{\partial v} - (-1)^{n} \dfrac{\partial J_{-v}(x)}{\partial v} \right] \bigg\vert_{v=n}
\end{aligned}
\label{eq:ecuacion_11_61}
\end{eqnarray}
Como en el caso de las funciones de Bessel de primera clase, $N_{v}(x)$ tiene una representación integral. Para $N_{0}(x)$, se expresa como
\begin{eqnarray}
\begin{aligned}
N_{0}(x) &= - \dfrac{2}{\pi} \int_{0}^{\infty} \cos (x \cosh t ) dt \\
&= - \dfrac{2}{\pi} \int_{1}^{0} \dfrac{\cos (xt)}{(t^{2}-1)^{1/2}} dt, \hspace{1cm}  x>0
\end{aligned}
\label{eq:ecuacion_11_65a}
\end{eqnarray}
Para verificar que las funciones de Neumman o funciones de Bessel de segunda clase, satisface la ecuación de Bessel para $n$ entero, hagamos lo siguiente. Diferenciando la ecucación de Bessel $J_{\pm v}(x)$ con respecto a $v$
\begin{equation}
x^{2} \dfrac{d^{2}}{dx^{2}} \left( \dfrac{\partial J_{\pm v}}{\partial v} \right) + x \dfrac{d}{dx} \left( \dfrac{\partial J_{\pm v}}{\partial v} \right) + (x^{2} - v^{2}) \dfrac{\partial J_{\pm v}}{\partial v} = 2 v J_{\pm v}
\label{eq:ecuacion_11_66}
\end{equation}
Multiplicando la ecuación con $J_{-v}$ por $(-1)^{v}$, restando de la ecuación con $J_{v}$, y tomando el límite cuando $v \to n$, se obtiene
\begin{equation}
x^{2} \dfrac{d^{2}}{dx^{2}} N_{n} +  x \dfrac{d}{dx} N_{n} + (x^{2} - n^{2}) N_{n} = \dfrac{2n}{\pi}[ J_{n} - (-1)^{n} J_{-n}]
\label{eq:ecuacion_11_67}
\end{equation}
Cuando $n=v$ entero, el lado derecho de la ecuación se anula y por tanto $N_{n}(x)$ es una solución a la ecuación de Bessel.
\\
La forma general de la solcuión para cualquier $v$ puede escribirse como
\begin{equation}
y(x) = A J_{v}(x) + B N_{v}(x)
\end{equation}
%\section{Ejemplo.}
%Partiendo del reposo a una distancia $L$ desde el origen $O$, una partícula $P$ de masa variante $m$ es atraída hacia el origen por una fuerza dirigida siempre hacia el origen y que tiene magnitud proporcional al producto $my$, donde $y$ es la distancia de $P$ a partir de la origen. La masa $m$ de $P$ disminuye con el tiempo $t$ de acuerdo con la expresión
%\begin{equation}
%m = \dfrac{1}{a + bt}
%\end{equation}
%donde $a$ y $b$ son constantes. El problema es calcular el tiempo que necesita la partícula $P$ en llegar al origen $O$.
%\\
%\emph{Solución: }
%\\
%De Newton-2
%\begin{equation}
%\dfrac{d\overrightarrow{M}}{dt} = \overrightarrow{F}
%\end{equation}
%donde $\overrightarrow{M}$ es el vector momento y $\overrightarrow{F}$ es la fuerza actuante. El problema nos conduce a
%\begin{equation}
%\dfrac{d}{dt} \left( m \dfrac{dy}{dt} \right) = - k^{2} m y
%\end{equation}
%que es
%\begin{equation}
%m\dfrac{d^{2} y}{dt^{2}}+ \dfrac{dm}{dt}\dfrac{dy}{dt} + k^{2} m y = 0
%\end{equation}
%donde $k^{2}$ es la constante de proporcionalidad involucrada en la magnitud de $\overrightarrow{F}$.
%\\
%Si proponemos el cambio de variable
%\begin{equation}
%a + bt = bx
%\end{equation}
\section{Funciones de Hankel.}
Se definen las funciones de Hankel $H_{v}^{(1)}$ y $H_{v}^{(2)}$ como
\begin{equation}
H_{v}^{(1)}(x) = J_{v}(x) + i N_{v}(x)
\label{eq:ecuacion_11_85}
\end{equation}
y
\begin{equation}
H_{v}^{(2)}(x) = J_{v}(x) - i N_{v}(x)
\label{eq:ecuacion_11_86}
\end{equation}
Que es análogo a tomar
\begin{equation}
e^{\pm i \theta} =  \cos \theta \pm i \sin \theta
\label{eq:ecuacion_11_87}
\end{equation}
Para argumentos reales, $H_{v}^{(1)}$ y $H_{v}^{(2)}$  son conjugados complejos.
\\
Ya que las funciones de Hankel son combinaciones lineales (con coeficientes constantes) de $J_{v}$ y $N_{v}$, satisfacen las mismas relaciones de recurrencia
\begin{equation}
H_{v-1} (x) + H_{v+1} (x) = \dfrac{2v}{x} H_{v} (x)
\label{eq:ecuacion_11_92}
\end{equation} 
y 
\begin{equation}
H_{v-1} (x) - H_{v+1} (x) = 2 H'_{v} (x)
\label{eq:ecuacion_11_92}
\end{equation} 
tanto para $H_{v}^{(1)}$ y $H_{v}^{(2)}$.
\section{Problema a cuenta del examen.}
Demostrar que el wronskiano
\begin{equation}
W_{v} = W_{v}(x) = \begin{vmatrix}
J_{v} & N_{v} \\
J'_{v} & N'_{v}
\end{vmatrix} = J_{v} N'_{v} - J'_{v} N_{v} = \dfrac{C}{x}
\end{equation}
Donde $C$ es una constante y donde (como se indica) el argumento de cada función involucrada es $x$.
\end{document}
