\documentclass[12pt]{article}
\usepackage[utf8]{inputenc}
\usepackage[spanish,es-lcroman, es-tabla]{babel}
\usepackage[autostyle,spanish=mexican]{csquotes}
\usepackage{amsmath}
\usepackage{amssymb}
\usepackage{nccmath}
\numberwithin{equation}{section}
\usepackage{amsthm}
\usepackage{graphicx}
\usepackage{epstopdf}
\DeclareGraphicsExtensions{.pdf,.png,.jpg,.eps}
\usepackage{color}
\usepackage{float}
\usepackage{multicol}
\usepackage{enumerate}
\usepackage[shortlabels]{enumitem}
\usepackage{anyfontsize}
\usepackage{anysize}
\usepackage{array}
\usepackage{multirow}
\usepackage{enumitem}
\usepackage{cancel}
\usepackage{tikz}
\usepackage{circuitikz}
\usepackage{tikz-3dplot}
\usetikzlibrary{babel}
\usetikzlibrary{shapes}
\usepackage{bm}
\usepackage{mathtools}
\usepackage{esvect}
\usepackage{hyperref}
\usepackage{relsize}
\usepackage{siunitx}
\usepackage{physics}
%\usepackage{biblatex}
\usepackage{standalone}
\usepackage{mathrsfs}
\usepackage{bigints}
\usepackage{bookmark}
\spanishdecimal{.}

\setlist[enumerate]{itemsep=0mm}

\renewcommand{\baselinestretch}{1.5}

\let\oldbibliography\thebibliography

\renewcommand{\thebibliography}[1]{\oldbibliography{#1}

\setlength{\itemsep}{0pt}}
%\marginsize{1.5cm}{1.5cm}{2cm}{2cm}


\newtheorem{defi}{{\it Definición}}[section]
\newtheorem{teo}{{\it Teorema}}[section]
\newtheorem{ejemplo}{{\it Ejemplo}}[section]
\newtheorem{propiedad}{{\it Propiedad}}[section]
\newtheorem{lema}{{\it Lema}}[section]

\usepackage{enumerate}
\usepackage{pifont}
\renewcommand{\labelitemi}{\ding{43}}
%\author{M. en C. Gustavo Contreras Mayén. \texttt{curso.fisica.comp@gmail.com}}
\title{{Tarea Examen 3} \\ {\large Matemáticas Avanzadas de la Física}}
\date{ }
\begin{document}
%\renewcommand\theenumii{\arabic{theenumii.enumii}}
\renewcommand\labelenumii{\theenumi.{\arabic{enumii}}}
\maketitle
\fontsize{14}{14}\selectfont
Fecha de entrega: \textbf{Jueves 26 de marzo de 2015.}
\\
\textbf{Consideraciones importantes:}
\begin{itemize}
\item Esta parte del examen, cubre los primeros cuatro temas del curso.
\item Las tareas de los temas 3 y 4 se sustituyen por esta tarea examen de casa.
\item Como forma parte de la primera evaluación, los problemas que entreguen el día señalado, serán los que se tomen en cuenta para la calificación, es decir, si les faltan soluciones, ya no se van a considerar como parte de su calificación, aunque los entreguen posteriormente, sólo se les revisarán. Por lo que aconsejamos entregar todos los ejercicios.
\end{itemize}
\begin{enumerate}
\item Para una esfera sólida homogénea con constante de difusión $K$, que no presenta fuentes de calor, la ecuación de conducción de calor es
\[ \dfrac{\partial T(r,t)}{\partial t} = K \nabla^{2} T(r,t) \]
Suponemos que tiene una solución del tipo
\[ T = R(r) T(t) \]
y que es separable. Demostrar que la ecuación radial puede tomar la siguiente forma
\[ r^{2} \dfrac{d^{2} R}{d r^{2}} + 2 R \dfrac{d R}{d r} + [\alpha^{2} r^{2} - n(n+1) ] R = 0, \hspace{1cm} n=\text{entero} \]
Las soluciones de esta ecuación son las llamadas funciones de Bessel esféricas.
\item Resolver la ecuación anterior mediante la técnica de Frobenius.
\item En una distribución tipo Maxwell la fracción de partículas moviéndose con velocidad $v$ y $v+dv$ es
\[ \dfrac{dN}{N} = 4 \pi \left( \dfrac{m}{2 \pi k T} \right)^{3/2} \exp(-mv^{2}/ k T) v^{2} dv \]
$N$ es el número total de partículas. El promedio o valor esperado de $v^{n}$ se define como $<v^{n}> = N^{-1} \int v^{n} dN$. Demostrar que
\[ < v^{n} > = \left( \dfrac{2 k T}{m} \right)^{n/2} \left( \dfrac{n+1}{2} \right) ! \Bigg/ \dfrac{1}{2} ! \]
\item Demostrar que
\[ \int_{0}^{\infty} e^{-x^{4}} dx = \left( \dfrac{1}{4} \right) !\]
\item Los polinomios de Legendre pueden escribirse como
\[ \begin{split}
P_{n}(\cos \theta) &= 2 \dfrac{(2n-1)!!}{(2n)!!} \left[ \cos n\theta + \dfrac{1}{1} \cdot \dfrac{n}{2n-1} \cos(n-2) \theta + \right. \\
&= \dfrac{1 \cdot 3}{1 \cdot 2} \dfrac{n(n-1)}{(2n-1)(2n-3)} \cos(n-4) \theta + \\
&= \left. \dfrac{1 \cdot 3 \cdot 5}{1 \cdot 2 \cdot 3} \dfrac{n(n-1)(n-2)}{(2n-1)(2n-3)(2n-5)} \cos(n-6) \theta + \ldots \right]
\end{split}
\]
Para $n=2s+1$, tenemos que
\[ P_{n}(\cos \theta) =  P_{2s+1} (\cos \theta) =  \sum_{m=0}^{s} a_{m} \cos(2m+1) \theta \]
Encontrar los $a_{m}$ en términos de factoriales y dobles factoriales.
\item Comprueba las siguientes identidades de la función Beta:
\begin{enumerate}
\item $B(a,b) = B(a+1,b) + B(a,b+1)$
\item $B(a,b) = \frac{a+b}{b} B(a,b+1)$ 
\item $B(a,b) = \frac{b-1}{a} B(a+1,b-1)$
\item $B(a,b) B(a+b,c) = B(b,c) B(a,b+c)$
\end{enumerate}
\item Demostrar que
\[ \int_{-1}^{1} (1-x^{2})^{1/2} x^{2n} dx =  
\begin{cases}
\pi/2 & n = 0 \\
\pi \dfrac{(2n-1)!!}{(2n+2)!!} & n=1,2,3,\ldots  \end{cases}
 \]

\end{enumerate}

\end{document}