\documentclass[12pt]{article}
\usepackage[utf8]{inputenc}
\usepackage[spanish,es-lcroman, es-tabla]{babel}
\usepackage[autostyle,spanish=mexican]{csquotes}
\usepackage{amsmath}
\usepackage{amssymb}
\usepackage{nccmath}
\numberwithin{equation}{section}
\usepackage{amsthm}
\usepackage{graphicx}
\usepackage{epstopdf}
\DeclareGraphicsExtensions{.pdf,.png,.jpg,.eps}
\usepackage{color}
\usepackage{float}
\usepackage{multicol}
\usepackage{enumerate}
\usepackage[shortlabels]{enumitem}
\usepackage{anyfontsize}
\usepackage{anysize}
\usepackage{array}
\usepackage{multirow}
\usepackage{enumitem}
\usepackage{cancel}
\usepackage{tikz}
\usepackage{circuitikz}
\usepackage{tikz-3dplot}
\usetikzlibrary{babel}
\usetikzlibrary{shapes}
\usepackage{bm}
\usepackage{mathtools}
\usepackage{esvect}
\usepackage{hyperref}
\usepackage{relsize}
\usepackage{siunitx}
\usepackage{physics}
%\usepackage{biblatex}
\usepackage{standalone}
\usepackage{mathrsfs}
\usepackage{bigints}
\usepackage{bookmark}
\spanishdecimal{.}

\setlist[enumerate]{itemsep=0mm}

\renewcommand{\baselinestretch}{1.5}

\let\oldbibliography\thebibliography

\renewcommand{\thebibliography}[1]{\oldbibliography{#1}

\setlength{\itemsep}{0pt}}
%\marginsize{1.5cm}{1.5cm}{2cm}{2cm}


\newtheorem{defi}{{\it Definición}}[section]
\newtheorem{teo}{{\it Teorema}}[section]
\newtheorem{ejemplo}{{\it Ejemplo}}[section]
\newtheorem{propiedad}{{\it Propiedad}}[section]
\newtheorem{lema}{{\it Lema}}[section]

\usepackage{standalone}
\usepackage{tikz}   
\usepackage{tikz-3dplot}
\usetikzlibrary{decorations.pathmorphing,patterns}
\usepackage{enumerate}
\usepackage{bigints}
\usepackage{hyperref}
\usepackage{float}
\usepackage[left=1.5cm,top=1.5cm,right=1.5cm,bottom=1.5cm]{geometry}
\title{Tarea 6 - Matemáticas Avanzadas de la Física \\ \Large{\textbf{Fecha de entrega: Jueves 26 de noviembre.}}}
\date{ }
\begin{document}
\vspace{-4cm}
%\renewcommand\theenumii{\arabic{theenumii.enumii}}
\renewcommand\labelenumii{\theenumi.{\arabic{enumii}}}
\maketitle
\fontsize{14}{14}\selectfont
\begin{enumerate}
\item La función $\exp(i\mathbf{k} \cdot \mathbf{r})$ describe una onda plana de momento $\mathbf{p} = \hbar \mathbf{k}$ normalizada a una densidad unitaria, se supone la dependencia del tiempo de $\exp(-i \omega t)$. Demuestra que esas funciones de onda plana satisfacen la relación de ortogonalidad
\[ \int ( \exp(i \mathbf{k} \cdot \mathbf{r} ))^{*}  \exp(i \mathbf{k}' \cdot \mathbf{r}) \; dx dy dx =  (2 \pi)^{3} \delta(\mathbf{k} - \mathbf{k}')  \]
\item Un oscilador lineal cuántico en su estado base tiene una función de onda
\[ \psi(x) = a^{-1/2} \pi^{-1/4} \exp(-x^{2}/2a^{2}) \]
Demuestra que la correspondiente función de momento es
\[ g(p) = a^{1/2} \pi^{-1/4} \hbar^{-1/2} \exp(-a^{2}p^{2}/2\hbar^{2}) \]
\item La ecuación unidimensional de onda de Schrödinger es
\[ - \dfrac{\hbar^{2}}{2m} \; \dfrac{d^{2} \psi(x)}{d x^{2}} +  V(x) \psi(x) = E \psi (x) \]
Para el caso especial de que $V(x)$ es una función analítica de $x$, demuestra que la correspondiente ecuación de onda para el momento es
\[ V \left( i \hbar \dfrac{d}{dp} \right) g(p) + \dfrac{p^{2}}{2m} g(p) =  E g(p)  \]
Recupera esta ecuación de onda para el momento usando la transformada de Fourier y su inversa. No utilices la sustitución directa $x \to i \hbar (d/dp)$.
\item Una masa $m$ está sujeta al extremo de un resorte sin comprimir, el resorte tiene constante $k$. Al tiempo $t=0$ el extremo libre del resorte presenta una aceleración constante $a$, lejos de la masa. 
\begin{figure}[H]
\centering
\includestandalone{Figura_01_Tarea6}
\caption{Sistema masa-resorte.}
\label{fig:figura1}
\end{figure}
Usando la transformada de Laplace
\begin{enumerate}[label=\alph*)]
\item Encuentra la posición $x$ de la masa $m$ como función del tiempo.
\item Determina la expresión que toma $x(t)$ para tiempos pequeños $t$.
\end{enumerate}
\item El decaimiento radiactivo de un núcleo sigue la ley
\[ \dfrac{d N}{d t} = - \lambda N \]
donde $N$ es la concentración para un núcleo dado y $\lambda$ es la constante de decaimiento. Esta ecuación indica que la tasa de decaimiento es proporcional al número de estos núcleos radiactivos presentes, que decaen de manera independiente. En una serie radiactiva de $n$ núcleos diferentes, iniciando con $N_{1}$,
\begin{eqnarray*}
\dfrac{d N_{1}}{dt} &=& - \lambda_{1} N_{1}, \nonumber \\
\dfrac{d N_{2}}{dt} &=& - \lambda_{1} N_{1} - \lambda_{2} N_{2}, \hspace{0.3cm} \mbox{y así} \nonumber \\
\dfrac{d N_{n}}{dt} &=& - \lambda_{n-1} N_{n-1}, \hspace{0.3cm} \mbox{estable} \nonumber
\end{eqnarray*} 
\begin{enumerate}[label=\alph*)]
\item Encuentra $N_{1}$, $N_{2}$ y $N_{3}$, $n=3$, con $N_{1}(0)=N_{0}$, $N_{2}(0)=N_{3}(0) =0$.
\item Encuentra una expresión aproximada para $N_{2}$ y $N_{3}$, para valores pequeños de $t$ cuando $\lambda_{1} \simeq \lambda_{2}$
\item Encuentra una expresión aproximada para $N_{2}$ y $N_{3}$ válidas para valores grandes de $t$, cuando
\begin{enumerate}[label=\roman*)]
\item $\lambda_{1} \gg \lambda_{2}$
\item $\lambda_{1} \ll \lambda_{2}$
\end{enumerate}
\end{enumerate}
\item El potecial electróstatico de una carga puntual $q$ en el origen dentro de un sistema de coordenadas cilíndrico es
\[ \dfrac{q}{4 \pi \varepsilon_{0}} \int_{0}^{\infty} \exp(-kz) \; J_{0}(k \rho) dk = \dfrac{q}{4 \pi \varepsilon_{0}}  \;\dfrac{1}{(\rho^{2} + z^{2})^{1/2}}, \hspace{1cm} Re(z) \geq 0\]
De esta relación, demostrar que las transformadas de Fouriere coseno y seno de $J_{0}(k \rho)$ son
\begin{enumerate}[label=\alph*)]
\item $\sqrt{\dfrac{\pi}{2}} F_{c} \left[ J_{0} (k \rho) \right] = \displaystyle  \int_{0}^{\infty} J_{0} (k \rho) \; \cos k \zeta dk
 = \begin{cases}
(\rho^{2} - \zeta^{2})^{-1/2}, & \rho > \zeta \\
0, & \rho < \zeta
\end{cases}$ 
\item $\sqrt{\dfrac{\pi}{2}} F_{s} \left[ J_{0} (k \rho) \right] = \displaystyle  \int_{0}^{\infty} J_{0} (k \rho) \; \sin k \zeta dk
 = \begin{cases}
0, & \rho > \zeta \\
(\zeta^{2} - \rho^{2})^{-1/2}, & \rho < \zeta
\end{cases}$ 
 \end{enumerate}
\item Resuelve la ecuación de un oscilador armónico amortiguado 
\[ m X^{\prime \prime} (t) + b X^{\prime}(t) + k X(t) = 0 \]
con $X(0) = X_{0}$, $X^{\prime}(0) = 0$ y
\begin{enumerate}[label=\alph*)]
\item $b^{2} = 4 \mbox{ km}$, caso con amortiguamiento crítico.
\item $b^{2} > 4 \mbox{ km}$, caso sobreamortiguado.
\end{enumerate}
\item Resuelve la misma ecuación de un oscilador armónico amortiguado del ejercicio anterior, con $X(0) = X_{0}$, $X^{\prime}(0) = v_{0}$ y
\begin{enumerate}[label=\alph*)]
\item $b^{2} < 4 \mbox{ km}$, caso subamortiguado.
\item $b^{2} = 4 \mbox{ km}$, caso con amortiguamiento crítico.
\item $b^{2} > 4 \mbox{ km}$, caso sobreamortiguado.
\end{enumerate}
\item Encuentra la transformada de Laplace para una función periódica ''diente de sierra'' con período $T$, dada por
\[ V(t) = V_{0} \dfrac{t}{T}, \hspace{0.5cm} \mbox{ para } 0 \leq t \leq T \]
\item Una función $N(t)$ se llama \emph{función nula}, si
\[ \int_{0}^{t} N(u) du = 0 \]
para todo $t > 0$. Demuestra que $\mathcal{L} [N(t)] = 0$.
\item Encuentra la transformada seno de Fourier de las siguientes funciones:
\begin{enumerate}[label=\roman*)]
\item $f(x) = x \exp(-a x), a>0$
\item $f(x) = \dfrac{1}{x} \exp(-a x), a>0$
\item $f(x) = \dfrac{1}{x}$
\item $f(x) = \dfrac{x}{a^{2} + x^{2}}$
\end{enumerate}
\item Usando la transformada de Fourier (seno o coseno) resuelve las siguientes ecuaciones integrales:
\begin{enumerate}[label=\roman*)]
\item $\displaystyle \int_{0}^{\infty} f(x) \cos kx dx = \sqrt{\dfrac{\pi}{2k}}$
\item $\displaystyle \int_{0}^{\infty} f(x) \sin kx dx = \dfrac{a}{a^{2} +k^{2}}$
\item $\displaystyle \int_{0}^{\infty} f(x) \sin kx dx = \dfrac{\pi}{2} J_{0} (ak)$
\item $\displaystyle \int_{0}^{\infty} f(x) \cos kx dx = \dfrac{\sin ak}{k}$
\end{enumerate}

\end{enumerate}
\end{document}