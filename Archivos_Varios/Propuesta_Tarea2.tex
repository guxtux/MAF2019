\documentclass[12pt]{article}
\usepackage[utf8]{inputenc}
\usepackage[spanish,es-lcroman, es-tabla]{babel}
\usepackage[autostyle,spanish=mexican]{csquotes}
\usepackage{amsmath}
\usepackage{amssymb}
\usepackage{nccmath}
\numberwithin{equation}{section}
\usepackage{amsthm}
\usepackage{graphicx}
\usepackage{epstopdf}
\DeclareGraphicsExtensions{.pdf,.png,.jpg,.eps}
\usepackage{color}
\usepackage{float}
\usepackage{multicol}
\usepackage{enumerate}
\usepackage[shortlabels]{enumitem}
\usepackage{anyfontsize}
\usepackage{anysize}
\usepackage{array}
\usepackage{multirow}
\usepackage{enumitem}
\usepackage{cancel}
\usepackage{tikz}
\usepackage{circuitikz}
\usepackage{tikz-3dplot}
\usetikzlibrary{babel}
\usetikzlibrary{shapes}
\usepackage{bm}
\usepackage{mathtools}
\usepackage{esvect}
\usepackage{hyperref}
\usepackage{relsize}
\usepackage{siunitx}
\usepackage{physics}
%\usepackage{biblatex}
\usepackage{standalone}
\usepackage{mathrsfs}
\usepackage{bigints}
\usepackage{bookmark}
\spanishdecimal{.}

\setlist[enumerate]{itemsep=0mm}

\renewcommand{\baselinestretch}{1.5}

\let\oldbibliography\thebibliography

\renewcommand{\thebibliography}[1]{\oldbibliography{#1}

\setlength{\itemsep}{0pt}}
%\marginsize{1.5cm}{1.5cm}{2cm}{2cm}


\newtheorem{defi}{{\it Definición}}[section]
\newtheorem{teo}{{\it Teorema}}[section]
\newtheorem{ejemplo}{{\it Ejemplo}}[section]
\newtheorem{propiedad}{{\it Propiedad}}[section]
\newtheorem{lema}{{\it Lema}}[section]

\usepackage{enumerate}
%\author{M. en C. Gustavo Contreras Mayén. \texttt{curso.fisica.comp@gmail.com}}
\title{{Tarea 2} \\ {\large Matemáticas Avanzadas de la Física}}
\date{ }
\begin{document}
%\renewcommand\theenumii{\arabic{theenumii.enumii}}
\renewcommand\labelenumii{\theenumi.{\arabic{enumii}}}
\maketitle
\fontsize{14}{14}\selectfont
Fecha de entrega: \textbf{Jueves 12 de marzo de 2015.}
\begin{enumerate}
\item Desarrollar la ecuación de Helmholtz en coordenadas esferoidales oblatas, para calcular los operadores diferenciales (gradiente, divergencia, rotacional y laplaciano), los factores de escala y demostrar que la ecuación es separable en ese sistema coordenado. Abraham les comentó que podrían apoyarse con la referencia del artículo del Dr. Eugenio Ley Koo que ya les envío previamente.
\item Mostrar que la ecuación de Helmholtz
\[ \nabla^{2} \psi + k^{2} \psi = 0 \]
\item Es separable en coordenadas cilíndricas, si $k^{2}$ se generaliza como $k^{2} + f(\rho) + (1/\rho^{2}) g(\varphi) +  h(z)$. Nota: en clase, se trabajó el caso cuando $k^{2}$ es constante.
\item Demuestra que
\[ \nabla^{2} \psi(r,\theta,\varphi) + \left[ k^{2} + f(\rho) + \dfrac{1}{\rho^{2}} g(\theta) + \dfrac{1}{r^{2}\sin^{2} \theta} h(\varphi) \right] \psi (r,\theta,\varphi) = 0 \]
es separable (en coordenadas esféricas). Las funciones $f,g,h$ son funciones sólo de las variables que se indican, $k^{2}$ es una constante.
\end{enumerate}
\item Reducir cada ecuación a una ecuación de valores propios y a otra ecuación con condiciones iniciales, y luego calcular las soluciones particulares:
\begin{enumerate}[label=(\alph*)]
\item \begin{fleqn}
\[ \dfrac{\partial^{2} u}{\partial t^{2}} - \dfrac{\partial^{2} u}{\partial x^{2}} - u = 0 \hspace{1cm} \text{para } 0 < x < 1, t>0 \]
\[ \dfrac{\partial^{2} u}{\partial t} (x,0) = 0\]
\[ u(0,t) = u(1,t) = 0\]

\item \[ \dfrac{\partial^{2} u}{\partial t^{2}} + 2 \dfrac{\partial u}{\partial t} - 4 \dfrac{\partial^{2} u}{\partial x^{2}} +  u = 0 \hspace{1cm} \text{para } 0 < x < 1, t>0 \]
\[ u(x,0) = 0\]
\[ \dfrac{\partial u}{\partial x} (0,t) = u(1,t) = 0\] 
\end{fleqn}
\end{enumerate}
\item Utiliza el método de Frobenius para obtener la solucion general de cada una de las siguientes ecuaciones diferenciales, para un entorno de $x = 0$:
\begin{enumerate}[label=(\alph*)]
\begin{fleqn}
\item  \[ 2 x \dfrac{d^{2} y}{d x^{2}} + (1 - x^{2}) \dfrac{d y}{d x} - y = 0 \]
\item \[ x^{2} \dfrac{d^{2} y}{d x^{2}} + x \dfrac{d y}{d x} + (x^{2} - 1) y = 0\]
\end{fleqn}
\end{enumerate}
\item De la expresión
\[ \delta_{n} (x) = \dfrac{n}{\pi} \dfrac{1}{1+n^{2}x^{2}}\]
Demostrar que
\[ \int_{-\infty}^{\infty} \delta_{n} (x) d x = 1 \]
\item Una solución a la ecuación diferencial de Laguerre
\[ xy'' + (1-x) y' + ny = 0\]
para $n=0$ es $y_{1}(x)=1$. Desarrolla una segunda solución linealmente independiente.
\item A partir del estudio en mecánica cuántica del efecto Stark (en coordenadas parabólicas), nos conduce a la ecuación difencial
\[ \dfrac{d}{d \xi} \left( \xi \dfrac{d u}{d \xi} \right) + \left( \dfrac{1}{2} E \xi + \alpha - \dfrac{m^{2}}{4 \xi} - \dfrac{1}{4} F \xi^{2} \right) u = 0 \]
donde
\begin{enumerate}[label=(\roman*)]
\item $\alpha$ es la constante de separación.
\item $E$ es la energía total del sistema.
\item $F$ es una constante.
\item $Fz$ es la energía potencial que se agrega al introducir un campo eléctrico.
\end{enumerate}
Usando la raíz más grande de la ecuación indicial, desarrolla una solución en series de potencias, alrededor de $\xi=0$. Evalúa los primeros tres coeficientes en términos de $a_{0}$
\[  \begin{split}
& \text{Ecuación indicial } \hspace{1.5cm} k^{2} - \dfrac{m^{2}}{4} = 0 \\
u(\xi) &=  a_{0} \xi^{m/2} \left\lbrace 1 - \dfrac{\alpha}{m+1} \xi + \left[ \dfrac{\alpha^{2}}{2(m+1)(m+2)} - \dfrac{E}{4(m+2)} \right] \xi^{2} + \ldots \right\rbrace
\end{split} \]
Checa que la perturbación $E$ no se presenta hasta que el término $a_{3}$ se incluye.
\item Para el caso especial en donde no hay dependencia en la coordenada azimutal, del estudio del ion molecular del hidrógeno $(H2^{+})$ en mecánica cuántica, se llega a la ecuación
\[ \dfrac{d}{d \eta} \left[ (1 - \eta^{2} ) \dfrac{d u}{d \eta} \right] + \alpha u + \beta \eta^{2} u = 0 \]
Desarrolla una solución en series de potencias para $u(\eta)$. Evalúa los primeros tres coeficientes no nulos en términos de $a_{0}$
\[  \begin{split}
& \text{Ecuación indicial } \hspace{1.5cm} k(k-1) = 0 \\
u_{k=1} &=  a_{0} \eta \left\lbrace 1 - \dfrac{2- \alpha}{6} \eta^{2} + \left[ \dfrac{(2-\alpha)(12-\alpha)}{120} - \dfrac{\beta}{20} \right] \eta^{4} + \ldots \right\rbrace
\end{split} \] 
\end{enumerate}
\end{document}