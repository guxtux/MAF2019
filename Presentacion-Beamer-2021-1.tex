\input{pre_documento}
\input{pre_plantilla_Warsaw_default}
\input{pre_define_footers_Warsaw_default}
%\usepackage[backend=biber]{biblatex}
%\bibliography{LibrosFC.bib}
%\input{../Preambulos/pre_codigo}
%\input{../Preambulos/pre_define_footers}
\title{Matemáticas Avanzadas de la Física}
\subtitle{Semestre 2021-1}
\newcommand\RBox[1]{%
  \tikz\node[draw,rounded corners,align=center,] {#1};%
}
\institute{Facultad de Ciencias - UNAM}
\titlegraphic{\includegraphics[width=2cm]{escudo-facultad-ciencias.jpg}\hspace*{3.75cm}~%
   \includegraphics[width=2cm]{escudo-unam.jpg}
}
\date{}
\setbeamertemplate{itemize items}{>>}
\begin{document}
\maketitle
\begin{frame}
\frametitle{Equipo académico}
\begin{center}
\RBox{
M. en C. Gustavo Contreras Mayén \\
\href{mailto:gux7avo@ciencias.unam.mx}{gux7avo@ciencias.unam.mx}
}
\vskip 1cm
\RBox{
M. en C. Abraham Lima Buendía \\
\href{mailto:abraham3081@ciencias.unam.mx}{abraham3081@ciencias.unam.mx}
}
\end{center}
\end{frame}
\section*{Contenido}
\frame{\frametitle{Contenido}\tableofcontents[currentsection, hideallsubsections]}
\fontsize{14}{14}\selectfont
\spanishdecimal{.}
\section{Presentación del curso}
\frame{\frametitle{Contenido}\tableofcontents[currentsection, hideothersubsections]}
\subsection{Objetivos.}
\begin{frame}
\frametitle{Objetivos del curso}
De acuerdo con el mapa curricular de la carrera de Física, la asignatura de Matemáticas Avanzadas de la física -se puede consultar \href{http://www.fciencias.unam.mx/asignaturas/610.pdf}{aquí}- contiene los siguientes objetivos:
\end{frame}
\begin{frame}
\frametitle{Objetivos del curso}
En donde el alumno:
\begin{itemize}
\setlength{\itemsep}{0mm}
\item Reconocerá las ideas básicas del análisis de ecuaciones que involucran a funciones de varias variables.
\item Formulará aproximaciones consistentes a soluciones, con el fin de cuantificar los distintos mecanismos de la física que se involucran.
\end{itemize}
\end{frame}
\begin{frame}
\frametitle{Objetivos del curso}
Además:
\begin{itemize}
\setlength{\itemsep}{0mm}
\item Consultará la literatura matemática que sea relevante para los problemas de física.
\item Identificará el papel moderno que juegan las funciones especiales, como auxiliares poderosos en el análisis cualitativo de problemas en varias variables.
\end{itemize}
\end{frame}
\begin{frame}
\frametitle{Objetivo adicional}
También es nuestro objetivo demostrar al alumno que \emph{las funciones especiales y las transformadas integrales} no son solamente un tema matemático, que involucra las ramas de la geometría diferencial, las ecuaciones diferenciales y el análisis matemático.
\end{frame}
\begin{frame}
\frametitle{Objetivo adicional}
Veremos que \emph{son las técnicas de estudio fundamentales} en la electrostática, la electrodinámica, la mecánica cuántica en los límites relativista y no relativista, la dinámica de medios deformables, la hidrodinámica clásica entre otras ramas de la física.
\end{frame}
\begin{frame}
\frametitle{Relevancia de la asignatura}
MAF les brindará un manejo más fluido y consistente para lo que van a cursar en el sexto semestre y los tres semestres que les restan para concluir la carrera.
\\
\bigskip
\pause
Es una asignatura con bastante relevancia para la formación del físico.
\end{frame}
\section{Metodología de enseñanza}
\frame[allowframebreaks]{\tableofcontents[currentsection, hideothersubsections]}
\subsection{Semestre a distancia}
\begin{frame}
\frametitle{Situación en la UNAM}
Como es conocimiento de todos, aún nos encontramos en contingencia sanitaria por Covid-19, por lo que las autoridades de la UNAM han decidido que el semestre 2021-1 se impartirá en modalidad a distancia.
\end{frame}
\begin{frame}
\frametitle{Forma distinta de trabajo}
El semestre anterior nos vimos en la necesidad de migrar la actividad docente de la modalidad presencial a distancia, siendo un reto en el que tanto alumnos como profesores tuvimos que adaptarnos y responder.
\end{frame}
\begin{frame}
\frametitle{Trabajo a distancia}
Para este semestre 2021-1 comenzaremos con la modalidad distancia, por lo que se replantean varias actividades y el desarrollo del trabajo para la asignatura.
\end{frame}
\begin{frame}
\frametitle{Plataforma de trabajo}
Para este curso se utilizará la plataforma Moodle, favoreciendo una estandarización con las demás asignaturas que se imparten en la Facultad de Ciencias.
\end{frame}
\begin{frame}
\frametitle{Acceso a Moodle}
Se proporcionarán las credenciales para ingresar a la plataforma en donde encontrarán las actividades de trabajo, materiales de consulta y referencias adicionales para el curso.
\end{frame}
\begin{frame}
\frametitle{Materiales de trabajo}
Se contará con materiales que deberán de revisar: en ellos se discute el tema, se presentan ejemplos, se incluyen ejercicios a cuenta.
\\
\bigskip
La lectura y trabajo con estos materiales es \emph{obligatoria.}
\end{frame}
\begin{frame}
\frametitle{Materiales adicionales}
De manera complementaria se dispondrá de materiales adicionales de consulta, en ellos se hará un revisión en particular de un ejercicio o problema.
\end{frame}
\begin{frame}
\frametitle{Materiales adicionales}
Buscando que el desarrollo se aborde con otro enfoque, pero que complementa lo que hayan revisado en los materiales de trabajo.
\\
\bigskip
Recomendamos mucho que revisen estos materiales adicionales.
\end{frame}
\subsection{Formato de trabajo en línea}
\begin{frame}
\frametitle{Formato de trabajo en línea}
Considerando el potencial de la enseñanza a distancia, tendremos un formato de trabajo mixto:
\setbeamercolor{item projected}{bg=blue!70!black,fg=yellow}
\setbeamertemplate{enumerate items}[circle]
\begin{enumerate}[<+->]
\item Con sesiones síncronas en el horario de 3 a 4 pm.
\item Sesiones asíncronas en donde podrán ingresar a la plataforma.
\end{enumerate}
\end{frame}
\begin{frame}
\frametitle{Sesiones síncronas}
Se tendrán sesiones de videoconferencia al inicio de cada tema del curso, con la finalidad de presentar el alcance para ese tema, señalar las actividades de trabajo y de evaluación del mismo.
\\
\bigskip
Las sesiones se llevarán a cabo en el horario de la asignatura: 3 a 4 pm, el día de la semana podrá variar en función del avance del curso. 
\end{frame}
\begin{frame}
\frametitle{Sesiones síncronas}
También se tendrán sesiones de videoconferencia los viernes de cada semana, con la finalidad de revisar los avances en la solución de los ejercicios de tarea y de evaluación del tema, siendo muy importante participar en estas sesiones síncronas.
\end{frame}
\begin{frame}
\frametitle{Sesiones asíncronas}
La modalidad de enseñanza a distancia requiere que el alumno realice actividades de manera asíncrona dentro de la plataforma, éstas actividades consistirán en la revisión de materiales de trabajo, lecturas adicionales, consulta de textos complementarios y una serie de ejercicios.
\end{frame}
\begin{frame}
\frametitle{Sesiones asíncronas}
El avance del curso se dará conforme al temario que se revisará en esta presentación, cabe señalar que las actividades (ejercicios y tareas) se deberán de completar en los tiempos señalados, la modalidad a distancia requiere también de un cumplimiento para el alcance de los objetivos.
\end{frame}
{
\setbeamercolor{background canvas}{bg=}
\includepdf[pages=1]{Calendario_Esquema_Trabajo_2021_1.pdf}
}
\subsection{Tiempo para atender el curso}
\begin{frame}
\frametitle{Requerimiento de tiempo}
De manera independiente a la modalidad de trabajo, deben de considerar \enquote{apartar} tiempo para la asignatura, por lo que hacemos la recomendación de que midan su carga académica durante este semestre.
\end{frame}
\section{Temario}
\frame[allowframebreaks]{\small\tableofcontents[currentsection, hideothersubsections]}
\subsection{La física y la geometría}
\begin{frame}
\frametitle{Tema 1 - La física y la geometría}
\setbeamercolor{item projected}{bg=blue!70!black,fg=yellow}
\setbeamertemplate{enumerate items}[circle]
\begin{enumerate}[<+->]
\item Sistemas de coordenadas curvilíneas ortogonales.
\item Operadores diferenciales en coordenadas curvilíneas.
\item Construcción de sistemas ortogonales.
\item Funciones Gamma y Beta.
\end{enumerate}
\end{frame}
\subsection{Primeras técnicas de solución}
\begin{frame}
\frametitle{Tema 2- Primeras técnicas de solución}
\setbeamercolor{item projected}{bg=blue!70!black,fg=yellow}
\setbeamertemplate{enumerate items}[circle]
\begin{enumerate}[<+->]
\item Separación de variables.
\item Método de Frobenius y remoción de singularidades.
\item Segunda solución linealmente independiente.
\end{enumerate}
\end{frame}
\subsection{Bases completas y ortogonales}
\begin{frame}
\frametitle{Tema 3 - Bases completas y ortogonales}
\setbeamercolor{item projected}{bg=blue!70!black,fg=yellow}
\setbeamertemplate{enumerate items}[circle]
\begin{enumerate}[<+->]
\item Ecuaciones de tipo Sturm-Liouville.
\item Ortogonalización de Gram-Schimdt.
\item Teorema del desarrollo.
\item Cálculo de funciones de Green.   
\end{enumerate}
\end{frame}
\subsection{Sep. variables en coordenadas esféricas}
\begin{frame}
\frametitle{Tema 4 - Sep. de variables en coordenadas esféricas}
\framesubtitle{Análisis del átomo de hidrógeno en la parte angular 1/2}
\setbeamercolor{item projected}{bg=blue!70!black,fg=yellow}
\setbeamertemplate{enumerate items}[circle]
\begin{enumerate}[<+->]
\item Momento angular.
\item Armónicos esféricos.
\item Teorema de adición.
\item Ecuación asociada de Legendre.
\item Ecuación de Legendre.
\end{enumerate}
\end{frame}
\subsection{Funciones especiales}
\begin{frame}
\frametitle{Tema 5 - Funciones especiales}
\setbeamercolor{item projected}{bg=blue!70!black,fg=yellow}
\setbeamertemplate{enumerate items}[circle]
\begin{enumerate}[<+->]
\item Funciones de Laguerre. \\ (Análisis del átomo de hidrógeno en la parte radial 2/2)
\item Funciones de Hermite. (Oscilador armónico cuántico)
\item Funciones de Bessel. (Propagación de ondas cilíndricas)
\item Funciones de Gegenbauer. 
\item Funciones hipergeométricas.
\end{enumerate}
\end{frame}
\subsection{Transformadas integrales}
\begin{frame}
\frametitle{Tema 6 - Transformadas integrales}
\setbeamercolor{item projected}{bg=blue!70!black,fg=yellow}
\setbeamertemplate{enumerate items}[circle]
\begin{enumerate}[<+->]
\item Transformada de Fourier.
\item Transformada de Laplace.
\item Transformada discreta usada en cómputo científico.
\end{enumerate}
\end{frame}
\section{Cronograma de trabajo}
\frame{\tableofcontents[currentsection, hideothersubsections]}
\subsection{Calendarización del curso}
\begin{frame}
\frametitle{Calendarización}
A continuación se presenta el cronograma de trabajo para el curso, durante las 16 semanas del semestre se han distribuido los 6 temas.
\end{frame}
{
\setbeamercolor{background canvas}{bg=}
\includepdf[pages=1-6]{Calendario_2021_1.pdf}
}
\section{Evaluación.}
\frame[allowframebreaks]{\small\tableofcontents[currentsection, hideothersubsections]}
\subsection{Ejercicios semanales en clase.}
\begin{frame}
\frametitle{Ejercicios semanales en clase}
Durante cada semana se presentarán ejercicios que se deberán de resolver a modo de tarea, para que puedan trabajarlos oportunamente, hacer consultas, resolver dudas, etc.
\\
\bigskip
\pause
Al concluir cada semana se tendrá un conjunto de ejercicios que deberán de ser resueltos, tendrán dos semanas para completarlos y hacer el envío el día viernes correspondiente.
\end{frame}
\begin{frame}
\frametitle{Muy importante}
Se considerarán como ejercicios a cuenta de calificación, aquellos ejercicios resueltos que representen el $100\%$ del total, es decir, deberán de entregar todos los ejercicios a cuenta.
\\
\bigskip
\pause
En el caso de que no se entreguen todos los ejercicios de la semana, sólo se revisarán éstos pero no contarán para el porcentaje de calificación.
\end{frame}
\subsection{Exámenes parciales.}
\begin{frame}
\frametitle{Exámenes parciales}
Habrá 2 exámenes parciales durante el semestre, el primero de ellos cubrirá los tres primeros temas, mientras que el segundo examen, los tres temas restantes.
\end{frame}
\begin{frame}
\frametitle{Exámenes parciales}
Se entregará con suficiente el tiempo el listado de problemas para cada examen, el examen es individual por lo que cada alumno devolverá su examen resuelto.
\end{frame}
\begin{frame}
\frametitle{Exámenes parciales}
El examen parcial deberá de devolverse con el $100\%$ de los ejercicios resueltos, sólo de esta manera se tomará en cuenta para la calificación, de lo contrario, sólo se revisarán los ejercicios entregados sin que aporten puntuación.
\end{frame}
\begin{frame}
\frametitle{De los ejercicios y tareas}
En esta asignatura \emph{se revisa y se evalúa el proceso de resolución de un problema, es decir, será necesario detallar cada paso en la solución}.
\\
\bigskip
\pause
Por lo que deberán de considerar que tanto los ejercicios semanales como los exámenes tendrán que ser resueltos a mano.
\end{frame}
\begin{frame}
\frametitle{De los ejercicios y tareas}
Posteriomente deberán de escanear las hojas que hayan ocupado y enviarlas mediante la plataforma.
\\
\bigskip
\pause
En caso de no contar con un escáner para la digitalización de las soluciones, se podrá enviar un archivo con las imágenes de la solución.
\end{frame}
\begin{frame}
\frametitle{De los ejercicios y tareas}
Se les pedirá encarecidamente, ser lo más claros en la escritura, orden y limpieza en sus soluciones, para que así recibamos un archivo que nos permita evaluar su trabajo.
\end{frame}
\subsection{Elementos para la calificación.}
\begin{frame}
\frametitle{Elementos para la calificación}
El porcentaje para cada elemento de evaluación es el siguiente:
\begin{itemize}
\setlength{\itemsep}{0mm}
\item Ejercicios semanales: $\mathbf{50\%}$.
\item Exámenes parciales: $\mathbf{50\%}$.
\end{itemize}
\end{frame}
\subsection{Consideraciones importantes.}
\begin{frame}
\frametitle{Consideraciones importantes}
\begin{itemize}
\setlength{\itemsep}{0mm}
\item No se recibirán ejercicios semanales de manera extemporánea.
\item No habrá reposición de exámenes.
\end{itemize}
\end{frame}
\begin{frame}
\frametitle{Consideraciones importantes}
\begin{itemize}
\setlength{\itemsep}{0mm}
\item En caso de que la calificación de un examen parcial (o los dos) sea menor a $6$ (seis), el alumno será candidato para presentar el examen final.
\item Para presentar examen final del curso se requiere que el alumno haya entregado los dos exámenes parciales del curso.
\end{itemize}
\end{frame}
\begin{frame}
\frametitle{Consideraciones importantes}
\begin{itemize}
\setlength{\itemsep}{0mm}
\item En caso de no haber entregado alguna tarea semanal de ejercicios y/o no haber presentado un examen del curso, no se tendrá derecho a presentar el examen final, por lo que la calificación final que se asentará en el acta del curso, será \textbf{NP (No presentó)}.
\end{itemize}
\end{frame}
\begin{frame}
\frametitle{Consideraciones importantes}
\begin{itemize}
\setlength{\itemsep}{0mm}
\item En caso de haber presentado al menos un examen y/o haber entregado una tarea semanal de ejercicios, y posteriormente no se tenga registro de otra entrega, se entenderá que abandonaron el curso, por lo que no se tendrá derecho para presentar el examen final, y la calificación que se asentará en el acta del curso, será $\mathbf{5}$ \textbf{(cinco)}.
\end{itemize}
\end{frame}
\begin{frame}
\frametitle{Consideraciones importantes}
\begin{itemize}
\setlength{\itemsep}{0mm}
\item En caso de presentar la primera ronda del examen final y la calificación obtenida sea no aprobatoria ($<6.0$), se puede presentar una segunda y última ronda del examen final.
\item La calificación obtenida en el examen final, es la que se asentará en actas.
\item Si el alumno no presenta el primer examen final, tendrá como calificación final en acta $\mathbf{5}$ \textbf{(cinco)}. 
\end{itemize}
\end{frame}
\section{Fechas importantes}
\begin{frame}
\frametitle{Fechas importantes}
\fontsize{12}{12}\selectfont
\begin{itemize}
\item \textcolor{red}{Lunes 21 de septiembre de 2020. Inicio del semestre 2021-1.}
\item Lunes 2 de noviembre del 2020. Día de asueto.
\item Lunes 16 de noviembre del 2020. Día de asueto.
\item \textcolor{blue}{Del Lunes 13 de diciembre al 1 de enero del 2021. Período vacional.}
\end{itemize}
\end{frame}
\begin{frame}
\frametitle{Fechas importantes}
\fontsize{12}{12}\selectfont
\begin{itemize}
\item Lunes 4 de enero. Regreso a las actividades.
\item \textcolor{red}{Viernes 29 de enero de 2021. Termina el semestre 2021-1.}
\item Lunes 1 de febrero del 2021. Día de asueto.
\item Del martes 2 de febrero al viernes 5 de febrero del 2021. \underline{Primera semana de exámenes}.
\item Del lunes 8 al viernes 12 de febrero del 2021. \underline{Segunda semana de exámenes}.
\end{itemize}
\end{frame}
\begin{frame}
\frametitle{Preguntas y respuestas}
Luego de haber revisado la presentación, les pedimos gentilmente realicen las preguntas que consideren necesarias para aclarar cualquiera de los puntos expuestos.
\end{frame}
\end{document}