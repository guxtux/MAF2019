\documentclass[12pt]{article}
\usepackage[utf8]{inputenc}
\usepackage[spanish,es-lcroman, es-tabla]{babel}
\usepackage[autostyle,spanish=mexican]{csquotes}
\usepackage{amsmath}
\usepackage{amssymb}
\usepackage{nccmath}
\numberwithin{equation}{section}
\usepackage{amsthm}
\usepackage{graphicx}
\usepackage{epstopdf}
\DeclareGraphicsExtensions{.pdf,.png,.jpg,.eps}
\usepackage{color}
\usepackage{float}
\usepackage{multicol}
\usepackage{enumerate}
\usepackage[shortlabels]{enumitem}
\usepackage{anyfontsize}
\usepackage{anysize}
\usepackage{array}
\usepackage{multirow}
\usepackage{enumitem}
\usepackage{cancel}
\usepackage{tikz}
\usepackage{circuitikz}
\usepackage{tikz-3dplot}
\usetikzlibrary{babel}
\usepackage{bm}
\usepackage{mathtools}
\usepackage{esvect}
\usepackage{hyperref}
\usepackage{relsize}
\usepackage{siunitx}
\usepackage{physics}
%\usepackage{biblatex}
\usepackage{standalone}
\usepackage{mathrsfs}
\usepackage{bigints}
\usepackage{bookmark}
\spanishdecimal{.}

\setlist[enumerate]{itemsep=0mm}

\renewcommand{\baselinestretch}{1.5}

\let\oldbibliography\thebibliography

\renewcommand{\thebibliography}[1]{\oldbibliography{#1}

\setlength{\itemsep}{0pt}}
%\marginsize{1.5cm}{1.5cm}{2cm}{2cm}


\newtheorem{defi}{{\it Definición}}[section]
\newtheorem{teo}{{\it Teorema}}[section]
\newtheorem{ejemplo}{{\it Ejemplo}}[section]
\newtheorem{propiedad}{{\it Propiedad}}[section]
\newtheorem{lema}{{\it Lema}}[section]

\usepackage{mathrsfs}
\usepackage{bigints}
\spanishdecimal{.}
%\usepackage{enumerate}
%\author{M. en C. Gustavo Contreras Mayén. \texttt{curso.fisica.comp@gmail.com}}
\title{Funciones de Bessel \\ {\large Matemáticas Avanzadas de la Física}}
\date{ }
\begin{document}
%\renewcommand\theenumii{\arabic{theenumii.enumii}}
\renewcommand\labelenumii{\theenumi.{\arabic{enumii}}}
\maketitle
\fontsize{14}{14}\selectfont
Las funciones de Bessel aparecen en una amplia variedad de problemas físicos: desde la separación con la ecuación de Helmholtz o la ecuación de onda en coordenadas cilíndricas, nos llevan a la ecuación de Bessel. La ecuación de Helmholtz en coordenadas esféricas nos dirigirá a una forma especial de la ecuación de Bessel.
\section{Función generatriz, Integral de orden $J_{n}(x)$}
Consideremos una función de dos variables:
\begin{equation}
g(x,t) = e^{(x/2)(t-1/t)}
\label{eq:ecuacion_11_1}
\end{equation}
Expandiendo esta función en una serie de Laurent, se obtiene
\begin{equation}
e^{(x/2)(t-1/t)} = \sum_{n=-\infty}^{\infty} J_{n} (x) t^{n}
\label{eq:ecuacion_11_2}
\end{equation}
El coeficiente de $t^{n}$, $J_{n}(x)$ se define como una función de Bessel de primera clase de orden $n$. Expandiendo las exponenciales, tenemos un producto de series de Maclaurin en $xt/2$ y $-x/2t$ respectivamente
\begin{equation}
e^{xt/2} \cdot e^{-x/2t} = \sum_{r=0}^{\infty} \left( \dfrac{x}{2} \right)^{r} \dfrac{t^{t}}{r!} \sum_{s=0}^{\infty} (-1)^{s} \left( \dfrac{x}{2} \right)^{s} \dfrac{t^{-s}}{s!}
\label{eq:ecuacion_11_3}
\end{equation}
Para un valor de $s$ dado, tenemos $t^{n}$ $(n \geq 0 )$ donde $r=n+s$
\begin{equation}
\left( \dfrac{x}{2} \right)^{n+s} \dfrac{t^{n+s}}{(n+s)!} (-1)^{s} \left( \dfrac{x}{2} \right)^{s} \dfrac{t^{-s}}{s!}
\label{eq:ecuacion_11_4}
\end{equation}
El coeficiente de $t^{n}$ es entonces
\begin{equation}
J_{n}(x) = \sum_{s=0}^{\infty} \dfrac{(-1)^{s}}{s! (n+s)!} \left( \dfrac{x}{2} \right)^{n+2s} = \dfrac{x^{n}}{2^{n}n!} - \dfrac{x^{n+2}}{2^{n+2}(n+1)! + \ldots}
\label{eq:ecuacion_11_5}
\end{equation}
Esta serie muestra el comportamiento de la función de Bessel $J_{n}(x)$ para valores de $x$ pequeños, y permite una evaluación numérica para $J_{n}(x)$.
\\
Las funciones de Bessel oscilan pero no son periódicas (excepto en el límite cuando $x \to \infty$. La amplitud de $J_{n}(x)$ no es constante, ya que decrece asintóticamente como $x^{-1/2}$.
\\
La ecuación (\ref{eq:ecuacion_11_5}) se ocupa si $n<0$, resultando
\begin{equation}
J_{-n} (x) = \sum_{s=0}^{\infty} \dfrac{(-1)^{s}}{s!(s-n)!)} \left( \dfrac{x}{2} \right)^{2s-n}
\label{eq:ecuacion_11_6}
\end{equation}
que se obtiene re-emplazando $n$ por $-n$ en la ecuación (\ref{eq:ecuacion_11_5}). Dado que $n$ es un entero, $(s-n)! \to \infty$ para $s=0, \ldots,(n-1)$. Por lo que se considera que inicia con $n=s$.
\\
Sustituyendo $s$ por $s+n$, se obtiene
\begin{equation}
J_{-n}(x) = \sum_{s=0}^{\infty} \dfrac{(-1)^{s+n}}{s!(n+s)!} \left( \dfrac{x}{2} \right)^{n+2s}
\label{eq:ecuacion_11_7}
\end{equation}
de donde se revisa que $J_{n}(x)$ y $J_{-n}(x)$ no son independientes, ya que están relacionadas por
\begin{equation}
J_{-n}(x) = (-1)^{n} J_{n}(x) \hspace{1.5cm} \text{con $n$ entero}
\label{eq:ecuacion_11_8}
\end{equation}
Las expresiones en series (ecuaciones \ref{eq:ecuacion_11_5} y \ref{eq:ecuacion_11_6}) pueden usarse al re-emplazar $n$ por $v$ para definir $J_{v}(x)$ y $J_{-v}(x)$ para $v$ un valor no entero.
\section{Relaciones de recurrencia.}
Las relaciones de recurrencia para $J_{n}(x)$ y sus derivadas se pueden obtener al operar las series, la ecuación (\ref{eq:ecuacion_11_5}) requiere un poco de clarividencia (o bastante ensayo y error).
\\
Diferenciando parcialmente la ecuación (\ref{eq:ecuacion_11_1}) con respecto a $t$, se encuentra
\begin{eqnarray}
\begin{aligned}
\dfrac{\partial}{\partial t} g(x,t) &= \dfrac{1}{2} x \left( 1 + \dfrac{1}{t^{2}} \right) e^{(x/2)(t-1/t)} \\
&= \sum_{n=-\infty}^{\infty} n J_{n}(x) t^{n-1}
\end{aligned}
\label{eq:ecuacion_11_9}
\end{eqnarray}
Sustituyendo la ecuación (\ref{eq:ecuacion_11_2}) para la exponencial y ajustando los coeficientes de las potencias de $t$, tenemos
\begin{equation}
J_{n-1}(x) + J_{n+1}(x) = \dfrac{2n}{x} J_{n} (x)
\label{eq:ecuacion_11_10}
\end{equation}
Que es una relación de recurrencia de tres términos. Dados por ejemplo $J_{0}$ y $J_{1}$, se puede calcular $J_{2}$ y cualquier otra valor de orden $J_{n}$.
\\
Ahora diferenciamos parcialmente la ecuación (\ref{eq:ecuacion_11_1}) con respecto a $x$, para obtener:
\begin{equation}
\dfrac{\partial}{\partial x} g(x,t) = \dfrac{1}{2} \left( 1 - \dfrac{1}{t} \right) e^{(x/2)(t-1/t)}
\label{eq:ecuacion_11_11}
\end{equation}
Nuevamente, sustituyendo en la ecuación (\ref{eq:ecuacion_11_2}) y ajustando los coeficientes de las potencias de $t$, obtenemos el resultado
\begin{equation}
J_{n-1}(x) - J_{n+1}(x) = 2 J'_{n}(x)
\label{eq:ecuacion_11_12}
\end{equation}
Como un caso especial de esta relación de recurrencia general
\begin{equation}
J'_{0}(x) = - J_{1}(x)
\label{eq:ecuacion_11_13}
\end{equation}
Sumando las ecuaciones (\ref{eq:ecuacion_11_10}) y (\ref{eq:ecuacion_11_12}), para luego dividir entre 2, resulta
\begin{equation}
J_{n+1}(x) = \dfrac{n}{x} J_{n}(x) + J'_{n}(x)
\label{eq:ecuacion_11_14}
\end{equation}
Multiplicando por $x^{n}$ y re-arreglando los términos
\begin{equation}
\dfrac{d}{dx} \left[ x^{n} J_{n} (x) \right] = x^{n} J_{n-1}(x)
\label{eq:ecuacion_11_15}
\end{equation}
Restando la ecuación (\ref{eq:ecuacion_11_12}) de (\ref{eq:ecuacion_11_10}) y dividiendo entre dos
\begin{equation}
J_{n+1}(x) = \dfrac{n}{x} J_{n}(x) - J'_{n}(x)
\label{eq:ecuacion_11_16}
\end{equation}
multiplicando por $x^{-n}$ y re-ordenando los términos, se tiene
\begin{equation}
\dfrac{d}{dx} \left[ x^{-n} J_{n} (x) \right] = -x^{-n} J_{n+1}(x)
\label{eq:ecuacion_11_17}
\end{equation}
\section{Ecuación diferencial de Bessel.}
Consideremos un conjunto de funciones $Z_{v}(x)$ tal que satisface las relaciones de recurrencia (ecuaciones \ref{eq:ecuacion_11_10} y \ref{eq:ecuacion_11_12}), pero donde $v$ no es necesariamente un entero y $Z_{v}$ no necesariamente está dada por las series (ecuación \ref{eq:ecuacion_11_15}). La ecuación (\ref{eq:ecuacion_11_14}) puede re-escribirse ($n \to v$) como
\begin{equation}
x Z'_{v} (x) = x Z_{v-1}(x) - v Z_{v} (x)
\label{eq:ecuacion_11_18}
\end{equation}
Ahora, diferenciando respecto a $x$, tenemos
\begin{equation}
x Z''_{v}(x) +  (v+1) Z'_{v} - x Z'_{v-1} - Z_{v-1} = 0
\label{eq:ecuacion_11_19}
\end{equation}
Multiplicando por $x$ y restando la ecuación (\ref{eq:ecuacion_11_18}) multiplicada por $v$, obtenemos
\begin{equation}
x^{2} Z''_{v} + x Z'_{v} - v^{2} Z_{v} + (v-1) x Z_{v-1} - x^{2} Z'_{v-1} = 0
\label{eq:ecuacion_11_20}
\end{equation}
Podemos re-escribir la ecuación (\ref{eq:ecuacion_11_16}) y re-emplazar $n$ por $v-1$
\begin{equation}
x Z'_{v-1} = (v-1) Z_{v-1} - x Z_{v}
\label{eq:ecuacion_11_21}
\end{equation}
Usando este último resultado para eliminar $Z_{v-1}$ y $Z'_{v-1}$ de la ecuación (\ref{eq:ecuacion_11_20}), se obtiene finalmente
\begin{equation}
x^{2} Z''_{v} + x Z'_{v} + (x^{2} - v^{2}) Z_{v} = 0
\label{eq:ecuacion_11_22}
\end{equation}
Que es la ecuación de Bessel. Aquí, cualesquiera funciones $Z_{v}(x)$ que satisfacen las relaciones de recurrencia (ecuaciones \ref{eq:ecuacion_11_10}, \ref{eq:ecuacion_11_12}, \ref{eq:ecuacion_11_14}, \ref{eq:ecuacion_11_16} o \ref{eq:ecuacion_11_15} y \ref{eq:ecuacion_11_17}) satisfacen la ecuación de Bessel, esto es, las $Z_{v}$ son las funciones de Bessel.
\\
En particular, hemos demostrado que las funciones $J_{n}(x)$ definidas por la función generatriz, satisface la ecuación de Bessel.
\\
Si el argumento es $k \rho$ en vez de $x$, la ecuación (\ref{eq:ecuacion_11_22}) se convierte en
\begin{equation}
\rho^{2} \dfrac{d^{2}}{d \rho^{2}} Z_{v} (k \rho) + \rho \dfrac{d}{d \rho} Z_{v} (k \rho) + (k^{2} \rho^{2} - v^{2}) Z_{v} (k \rho) = 0
\label{eq:ecuacion_11_22a}
\end{equation}
\section{Representación integral.}
Una particular manera útil y poderosa para el manejo de las funciones de Bessel, es la representación integral. Retomando la función generatriz (Ecuación \ref{eq:ecuacion_11_2}), y al sustituir $t=e^{i\theta}$
\begin{eqnarray}
\begin{aligned}
e^{ix\theta} &= J_{0}(x) + 2 (J_{2}(x) \cos 2 \theta + J_{4}(x) \cos 4 \theta + \ldots) + \\
&+ 2i (J_{1}(x) \sin \theta + J_{3}(x) \sin 3 \theta + \ldots)
\end{aligned}
\label{eq:ecuacion_11_23}
\end{eqnarray}
en donde se han utilizado las relaciones
\begin{eqnarray}
\begin{aligned}
J_{1}(x) e^{i \theta} + J_{-1}(x) e^{-i \theta} &= J_{1} (x) (e^{i \theta} - e^{-i \theta}) \\
&= 2 i J_{1}(x) \sin \theta \\
J_{2}(x) e^{2i \theta} + J_{-2}(x) e^{-2 i \theta} &= 2 J_{2} (x)\cos 2 \theta
\end{aligned}
\label{eq:ecuacion_11_24}
\end{eqnarray}
y así.
\\
En notación de suma
\begin{eqnarray}
\begin{aligned}
\cos (x \sin \theta) &= J_{0}(x) + 2 \sum_{n=1}^{\infty} J_{2n}(x) \cos (2n \theta) \\
\sin (x \sin \theta) &=  2 \sum_{n=1}^{\infty} J_{2n-1} (x) \sin [(2n-1) \theta]
\end{aligned}
\label{eq:ecuacion_11_25}
\end{eqnarray}
igualando las partes real e imaginaria, respectivamente. Nótese que el ángulo $\theta$ (en radianes) no tiene dimensiones. Dado que $\sin \theta$ no tiene dimensiones y la función $\cos (x \sin \theta)$ es correcta desde el punto de vista dimensional.
\\
Usando las propiedades de ortogonalidad del coseno y seno
\begin{eqnarray}
\int_{0}^{\pi} \cos n \theta \cos m \theta d \theta &=& \dfrac{\pi}{2} \delta_{nm} \label{eq:ecuacion_11_26a} \\
\int_{0}^{\pi} \sin n \theta \sin m \theta d \theta &=& \dfrac{\pi}{2} \delta_{nm} \label{eq:ecuacion_11_26b}
\end{eqnarray}
en donde $n$ y $m$ son enteros positivos (se excluye el cero), se obtiene
\begin{eqnarray}
\dfrac{1}{\pi} \int_{0}^{\pi} \cos (x \sin \theta) \cos n \theta d \theta &=& \begin{cases}
J_{n}(x) & n \text{ par} \\
0 & n \text{ impar} \end{cases} \label{eq:ecuacion_11_27} \\
\dfrac{1}{\pi} \int_{0}^{\pi} \sin (x \sin \theta) \sin n \theta d \theta &=& \begin{cases}
0 & n \text{ par} \\
J_{n}(x) & n \text{ impar} \end{cases}  \label{eq:ecuacion_11_28}
\end{eqnarray}
Si sumamos las dos ecuaciones
\begin{eqnarray}
\begin{aligned}
J_{n}(x) &= \dfrac{1}{\pi} \int_{0}^{\pi} [ \cos (x \sin \theta) \cos n \theta + \sin (x \sin \theta) \sin n \theta] d\theta \\
&= \dfrac{1}{\pi} \int_{0}^{\pi} \cos (n \theta - x \sin \theta) d \theta, \hspace{1cm} n=0,1,2,3,\ldots
\end{aligned}
\label{eq:ecuacion_11_29}
\end{eqnarray}
Un caso especial es
\begin{equation}
J_{0} (x) = \dfrac{1}{\pi} \int_{0}^{\pi} \cos (x \sin \theta) d \theta
\label{eq:ecuacion_11_30}
\end{equation}
Nótese que $\cos( x \sin \theta)$ se repite en los cuatro cuadrantes ($\theta_{1} = \theta, \theta_{2} = \pi - \theta, \theta_{3} = \pi + \theta, \theta_{4} = - \theta$), podemos re-escribir la ecuación (\ref{eq:ecuacion_11_30}) como
\begin{equation}
J_{0}(x) = \dfrac{1}{2 \pi} \int_{0}^{2 \pi} \cos (x \sin \theta) d \theta
\label{eq:ecuacion_11_30a}
\end{equation}
De otro manera, $\sin(x \sin \theta)$ cambia de signo en el tercer y cuarto cuadrante, así que
\begin{equation}
\dfrac{1}{2 \pi} \int_{0}^{2 \pi} \sin (x \sin \theta) d \theta = 0
\label{eq:ecuacion_11_30b}
\end{equation}
Sumando la ecuación (\ref{eq:ecuacion_11_30a}) con $i$ veces la ecuación (\ref{eq:ecuacion_11_30b}), se obtiene la representación exponencial compleja
\begin{eqnarray}
\begin{aligned}
J_{0}(x) &= \dfrac{1}{2 \pi} \int_{0}^{2 \pi} e^{ix \sin \theta} \\
&= \dfrac{1}{2 \pi} \int_{0}^{2 \pi} e^{ix \cos \theta}
\end{aligned}
\label{eq:ecuacion_11_30c}
\end{eqnarray}
\newpage
\section{Ejemplo: Difracción de Fraunhofer, apertura circular.}
\usetikzlibrary{shapes,snakes}
\begin{figure}[!h]
\centering
\begin{tikzpicture}
	\draw [->] (0,0) -- (10,0) node [pos=0.95, above] {$x$};
	\begin{scope}[shift={(0,-1.5)},rotate=20]	
	\draw (5,0) node[ellipse, ,rotate=15, thick, minimum height=2cm,minimum width=5cm,draw] {};
	\draw [->, thick] (2,0) -- (8,0) node [pos=0.95, above] {$y$};
	\end{scope}
	\begin{scope}[shift={(0,-5)},rotate=20]
	\draw [thick] (2,0) -- (8,0) ;
	\end{scope}
	
\end{tikzpicture}
\end{figure}
En la teoría de la difracción a través de una apertura circular, encontramos la integral
\begin{equation}
\Phi \sim \int_{0}^{a} \int_{0}^{2 \pi} e^{ibr \cos \theta} d \theta r dr
\label{eq:ecuacion_11_31}
\end{equation}
donde $Phi$ es la amplitud de la onda difractada. Aquí el ángulo $\theta$ es el ángulo azimutal en el plano de la apertura circular de radio $a$, y $\alpha$ es el ángulo definido por un punto en la pantalla bajo la apertura circular relativo a la normal del punto central.
\\
El parámetro $b$ está dado por
\begin{equation}
b = \dfrac{2 \pi}{\lambda} \sin \alpha
\label{eq:ecuacion_11_32}
\end{equation}
donde $\lambda$ es la longitud de onda de la onda incidente. De la ecuación (\ref{eq:ecuacion_11_30c}) tenemos que
\begin{equation}
\Phi \sim 2 \pi \int_{0}^{a} J_{0} (br) r dr 
\label{eq:ecuacion_11_33}
\end{equation}
La ecuación (\ref{eq:ecuacion_11_15}) nos permite integrar la expresión (\ref{eq:ecuacion_11_33}) inmediatamente para obtener
\begin{equation}
\Phi \sim \dfrac{2 \pi a b}{b^{2}} J_{1} (ab) \sim \dfrac{\lambda a}{\sin \alpha} J_{1} \left( \dfrac{2 \pi a}{\lambda} \sin \alpha \right)
\label{eq:ecuacion_11_34}
\end{equation}
La intensidad de la luz en el patrón de difracción es proporcional a $\Phi^{2}$ y
\begin{equation}
\Phi^{2} \sim \left[ \dfrac{J_{1} [(2 \pi a / \lambda) \sin \alpha]}{\sin \alpha} \right]^{2}
\label{eq:ecuacion_11_35}
\end{equation}
Evaluando los ceros de la función de Bessel y sus derivadas, vemos que la expresión (\ref{eq:ecuacion_11_35}) tiene un cero en
\begin{equation}
\dfrac{2 \pi a}{\lambda} \sin \alpha =  3.8317
\label{eq:ecuacion_11_36}
\end{equation}
que es lo mismo
\begin{equation}
\sin \alpha = \dfrac{3.8317 \lambda}{2 \pi a}
\label{eq:ecuacion_11_37}
\end{equation}
Para la luz verde $\lambda=5.5 \times 10^{-5}$ cm. Si $a=0.5$ cm, entonces
\begin{eqnarray}
\begin{aligned}
\alpha \simeq \sin \alpha &= 6.7 \times 10^{-5} \text{ (radianes)} \\
&\simeq 14 \text{ segundos de arco}
\end{aligned}
\label{eq:ecuacion_11_38}
\end{eqnarray}
que nos dice que la dispersión de un haz de luz es muy pequeña.
\section{Ejemplo: Cavidad resonante cilíndrica.}
En el interior de una cavidad resonante electromagnética, las ondas oscilan con una dependencia en el tiempo del tipo $e^{-i \omega t}$. Las ecuaciones de Maxwell nos conducen a
\[ \nabla \times \nabla \times E = \alpha^{2} E \]
para la parte espacial del campo eléctrico con $\alpha^{2} = \omega^{2} \varepsilon_{0} \mu_{0}$. Con $\nabla \cdot E=0$ (en el vacío, sin cargas)
\[ \nabla^{2} E + \alpha^{2} E = 0 \]
Separando las variables en coordenadas cilíndricas circualres, encontramos que la componente $z$ ($E_{z}$, en la parte espacial solamente) satisface la ecuación escalar de Helmholtz
\begin{equation}
\nabla^{2} E_{z} + \alpha^{2} E_{z} = 0
\label{eq:ecuacion_11_39}
\end{equation}
donde $\alpha^{2} =  \omega^{2} \varepsilon_{0} \mu_{0} = \omega^{2}/c^{2}$. Adicionalmente
\begin{equation}
(E_{z})_{mnk} = \sum_{m.n} J_{m} (\gamma_{mn} \rho) e^{\pm im\varphi} [ a_{mn} \sin kz + b_{mn} \cos kz ]
\label{eq:ecuacion_11_40}
\end{equation}
\begin{enumerate}
\item El parámetro $k$ es la constante de separación introducida en la división de $Z$ para la dependencia de $E_{z}(\rho, \varphi,z)$.
\item De manera análoga $m$ aparece en la separación de dependencia de $\varphi$.
\item $\gamma$ se presenta como $\alpha^{2} - k^{2}$ y está cuantizada como requisito para que $\gamma a$ sea una raíz de la función de Besser $J_{m}$-
\item Por tanto, $n$ en $\gamma_{mn}$ es la $n$-ésima raíz de $J_{m}$.
\end{enumerate}
Para las superficies en $z=0$ y $z=l$, se hace que $a_{mn}=0$ y
\begin{equation}
k = \dfrac{p \pi}{l}, \hspace{1cm} p=0,1,2,\ldots
\label{eq:ecuacion_11_41}
\end{equation}
Las ecuaciones de Maxwell entonces garantizan que los campos tangenciales $E_{\rho}$ y $E_{\varphi}$ se anulen en $z=0$ y $l$. Este es el modo de oscilación transversal del campo magnético. Tenemos entonces
\begin{eqnarray}
\begin{aligned}
\gamma^{2} &= \dfrac{\omega^{2}}{c^{2}} - k^{2} \\
&= \dfrac{\omega^{2}}{c^{2}} - \dfrac{p^{2} \pi^{2}}{l^{2}}
\end{aligned}
\label{eq:ecuacion_11_42}
\end{eqnarray}
Aquí la condición de frontera usual es $E_{z}(\rho =a)=0$. Entonces tenemos el conjunto. Por lo tanto tenemos
\begin{equation}
\gamma_{mn} = \dfrac{\alpha_{mn}}{a}
\label{eq:ecuacion_11_43}
\end{equation}
donde $\alpha_{mn}$ es el $n$-cero de $J_{m}$.
\\
El resultado de dos condiciones de frontera y la constante de separación $m^{2}$ es que la frecuencia angular de la oscilación depende de tres parámetros discretos
\begin{equation}
\omega_{mnp} = c \sqrt{\dfrac{\alpha^{2}_{mn}}{a^{2}} + \dfrac{p^{2} \pi^{2}}{l^{2}}} 
\begin{cases}
m = 0,1,2, \ldots \\
n = 1,2,3, \ldots \\
p = 0,1,2, \ldots
\end{cases}
\label{eq:ecuacion_11_44}
\end{equation}
Que son las frecuencias de resonancia permitidas para el modo TM.
\section{Ejercicio a cuenta.}
La sección diferencial de área en un experimento de dispersión nuclear está dada por $d \sigma / d \Omega = \vert f(\theta) \vert^{2}$.Una aproximación nos conduce a
\[ f(\theta) = \dfrac{-ik}{2 \pi} \int_{0}^{2\pi} \int_{0}^{R} \exp[ik \rho \sin \theta \sin \varphi] \rho d \rho d \varphi \]
Donde $\theta$ es el ángulo en el cual la partícula es dispersada. $R$ es el radio del núcleo. Demostrar que
\[ \dfrac{d \sigma}{d \Omega} = (\pi R^{2}) \dfrac{1}{\pi} \left[ \dfrac{J_{1} (k R \sin \theta)}{\sin \theta} \right]^{2} \]
\end{document}
