\documentclass[12pt]{article}
\usepackage[utf8]{inputenc}
\usepackage[spanish,es-lcroman, es-tabla]{babel}
\usepackage[autostyle,spanish=mexican]{csquotes}
\usepackage{amsmath}
\usepackage{amssymb}
\usepackage{nccmath}
\numberwithin{equation}{section}
\usepackage{amsthm}
\usepackage{graphicx}
\usepackage{epstopdf}
\DeclareGraphicsExtensions{.pdf,.png,.jpg,.eps}
\usepackage{color}
\usepackage{float}
\usepackage{multicol}
\usepackage{enumerate}
\usepackage[shortlabels]{enumitem}
\usepackage{anyfontsize}
\usepackage{anysize}
\usepackage{array}
\usepackage{multirow}
\usepackage{enumitem}
\usepackage{cancel}
\usepackage{tikz}
\usepackage{circuitikz}
\usepackage{tikz-3dplot}
\usetikzlibrary{babel}
\usetikzlibrary{shapes}
\usepackage{bm}
\usepackage{mathtools}
\usepackage{esvect}
\usepackage{hyperref}
\usepackage{relsize}
\usepackage{siunitx}
\usepackage{physics}
%\usepackage{biblatex}
\usepackage{standalone}
\usepackage{mathrsfs}
\usepackage{bigints}
\usepackage{bookmark}
\spanishdecimal{.}

\setlist[enumerate]{itemsep=0mm}

\renewcommand{\baselinestretch}{1.5}

\let\oldbibliography\thebibliography

\renewcommand{\thebibliography}[1]{\oldbibliography{#1}

\setlength{\itemsep}{0pt}}
%\marginsize{1.5cm}{1.5cm}{2cm}{2cm}


\newtheorem{defi}{{\it Definición}}[section]
\newtheorem{teo}{{\it Teorema}}[section]
\newtheorem{ejemplo}{{\it Ejemplo}}[section]
\newtheorem{propiedad}{{\it Propiedad}}[section]
\newtheorem{lema}{{\it Lema}}[section]

\usepackage{standalone}
\usepackage{enumerate}
\usepackage[left=1.5cm,top=1.5cm,right=1.5cm,bottom=1.5cm]{geometry}
\title{Tarea 2 - Matemáticas Avanzadas de la Física}
\date{ }
\begin{document}
\vspace{-4cm}
%\renewcommand\theenumii{\arabic{theenumii.enumii}}
\renewcommand\labelenumii{\theenumi.{\arabic{enumii}}}
\maketitle
\fontsize{14}{14}\selectfont
\begin{enumerate}
\item De la ley de Kirchhoff para la corriente $I$, en un circuito $RC$, se cumple la ecuación
\[ R \dfrac{dI}{dt} + \dfrac{1}{C} I = 0 \]
\begin{figure}[!h]
\centering
\includestandalone{EjemploCircuito}
\caption{Circuito RC en serie para el ejercicio.}
\end{figure}
\begin{enumerate}[label=(\alph*)]
\item Calcular $I(t)$.
\item Para una condensador de $10,000$ microfaradios cargado a una fuente de $100$ volts y que se descarga a través de una resistencia de $1 M \Omega$, calcula la corriente $I$ para $t=0$ y para $t=100$ segundos.
Nota: El voltaje inicial es $I_{0}R$ o $Q/C$ donde $Q = \int_{0}^{\infty} I(t) dt$.
\end{enumerate}
\item La fuerza gravitacional que experimenta una partícula de masa $m$ dentro de la Tierra a una distancia $r$ del centro, ($r <R \mbox{, R es el radio de la Tierra}$) es $F=-mgr/R$. Demuestra que el movimiento de la partícula colocada dentro de un tubo que atraviesa la Tierra, pasando por el centro de la misma, corresponde al de un oscilador armónico simple. Encuentra el período de oscilación de la partícula.
\item Un bloque de madera está flotando en el agua, si se empuja suavemente hacia abajo y luego se suelta el bloque, éste comenzará a oscilar. Suponemos que la cara superior e inferior del bloque son planos paralelos y que siguen siendo horizontales durante las oscilaciones, así también los lados del bloque se mantienen verticales. Demostrar que el periodo del oscilación (sin fricción) es $2 \pi \sqrt{h/g}$, donde $h$ es la altura vertical que alcanza el bloque mientras se empuja hacia abajo con respecto a la posición de equilibrio. \emph{Nota:} recuerda que la fuerza de flotación es igual al peso del agua desplazada.
\item (\textbf{2 puntos.}) Demuestra que la ecuación de Helmholtz es separable en los sistemas de coordenadas:
\begin{enumerate}[a)]
\item cilíndricas.
\item esféricas.
\end{enumerate}
\item Mostrar que la ecuación de Helmholtz
\[ \nabla^{2} \psi + k^{2} \psi = 0 \]
Es separable en coordenadas cilíndricas, si $k^{2}$ se generaliza como $k^{2} + f(\rho) + (1/\rho^{2}) g(\varphi) +  h(z)$. Nota: en clase, se trabajó el caso cuando $k^{2}$ es constante.
\item Demuestra que
\[ \nabla^{2} \psi(r,\theta,\varphi) + \left[ k^{2} + f(\rho) + \dfrac{1}{\rho^{2}} g(\theta) + \dfrac{1}{r^{2}\sin^{2} \theta} h(\varphi) \right] \psi (r,\theta,\varphi) = 0 \]
es separable (en coordenadas esféricas). Las funciones $f,g,h$ son funciones sólo de las variables que se indican, $k^{2}$ es una constante.
\item Para una esfera sólida homogénea con constante de difusión términa $K$, la ecuación de conducción de calor (sin fuentes) es
\[ \dfrac{\partial T(r,t)}{\partial t} =  K \nabla^{2} T(r,t) \]
Mediante la técnica de separación de variables, suponemos que tiene una solución de la forma
\[ T =R(r) T(t) \]
Demuestra que la ecuación radial toma la forma estándar
\[ r^{2} \dfrac{d^{2} R}{d r^{2}} + 2r \dfrac{d R}{d r} + \left[ \alpha^{2} r^{2} - n(n+1) \right] R = 0, \hspace{1cm} n = \mbox{ entero} \]
\item Resuelve los siguientes problemas de tipo Sturm-Liouville, y demuestra que los valores y funciones propias, son los que se indican:
\begin{enumerate}[label=(\alph*)]
\setlength\itemsep{1em}
\begin{fleqn}
\item  $y'' + \lambda y = 0$ con $y(0) = y'(L) = 0$ \\ en donde $\lambda_{n} = \dfrac{(2n-1)^{2} \pi^{2}}{4L^{2}}$ y $y_{n}(x) = \dfrac{(2n-1)^{2} \pi x}{2L}$ para $n \geq 1$.
\item $ y''+ \lambda y = 0 $ con $y'(0) = h y(L) + y'(L) = 0 \hspace{0.5cm} (h > 0)$ \\
en donde $\lambda_{n} =  \frac{\beta^{2}}{L^{2}}$ y $y_{n} = \cos \frac{\beta_{n} x}{L}$ para $n \geq 1$, donde $\beta_{n}$ es la $n$-ésima raíz positiva de $\tan x = \frac{hL}{x}$. Para estimar los valores de $\beta_{n}$ para $n$ grande, grafica $y = \tan x$ y $ y=\frac{hL}{x}$.
\end{fleqn}
\end{enumerate}
\item Reducir cada ecuación a una ecuación de valores propios y a otra ecuación con condiciones iniciales, y luego calcular las soluciones particulares:
\begin{enumerate}[label=(\alph*)]
\item \begin{fleqn}
\[ \dfrac{\partial^{2} u}{\partial t^{2}} - \dfrac{\partial^{2} u}{\partial x^{2}} - u = 0 \hspace{1cm} \text{para } 0 < x < 1, t>0 \]
\[ \dfrac{\partial^{2} u}{\partial t} (x,0) = 0\]
\[ u(0,t) = u(1,t) = 0\]
\item \[ \dfrac{\partial^{2} u}{\partial t^{2}} + 2 \dfrac{\partial u}{\partial t} - 4 \dfrac{\partial^{2} u}{\partial x^{2}} +  u = 0 \hspace{1cm} \text{para } 0 < x < 1, t>0 \]
\[ u(x,0) = 0\]
\[ \dfrac{\partial u}{\partial x} (0,t) = u(1,t) = 0\] 
\end{fleqn}
\end{enumerate}
\item Encuentra dos soluciones linealmente independientes en términos de la serie de Frobenius (para $x > 0$) en cada una de las ecuaciones diferenciales:
\begin{enumerate}[label=(\alph*)]
\begin{fleqn}
\item  $4x y'' + 2y' + y = 0 $
\item $ 2x y'' + 3y' - y = 0 $
\end{fleqn}
\end{enumerate}
\item Utiliza el método de Frobenius para obtener la solucion general de cada una de las siguientes ecuaciones diferenciales, para un entorno de $x = 0$:
\begin{enumerate}[label=(\alph*)]
\begin{fleqn}
\setlength\itemsep{1em}
\item  $ 2 x \dfrac{d^{2} y}{d x^{2}} + (1 - x^{2}) \dfrac{d y}{d x} - y = 0 $
\item $ x^{2} \dfrac{d^{2} y}{d x^{2}} + x \dfrac{d y}{d x} + (x^{2} - 1) y = 0 $
\end{fleqn}
\end{enumerate}
\item De la expresión
\[ \delta_{n} (x) = \dfrac{n}{\pi} \dfrac{1}{1+n^{2}x^{2}}\]
Demostrar que
\[ \int_{-\infty}^{\infty} \delta_{n} (x) d x = 1 \]
\item Una solución a la ecuación diferencial de Laguerre
\[ xy'' + (1-x) y' + ny = 0\]
para $n=0$ es $y_{1}(x)=1$. Desarrolla una segunda solución linealmente independiente.
\item A partir del estudio en mecánica cuántica del efecto Stark (en coordenadas parabólicas), nos conduce a la ecuación difencial
\[ \dfrac{d}{d \xi} \left( \xi \dfrac{d u}{d \xi} \right) + \left( \dfrac{1}{2} E \xi + \alpha - \dfrac{m^{2}}{4 \xi} - \dfrac{1}{4} F \xi^{2} \right) u = 0 \]
donde
\begin{enumerate}[label=(\roman*)]
\item $\alpha$ es la constante de separación.
\item $E$ es la energía total del sistema.
\item $F$ es una constante.
\item $Fz$ es la energía potencial que se agrega al introducir un campo eléctrico.
\end{enumerate}
Usando la raíz más grande de la ecuación indicial, desarrolla una solución en series de potencias, alrededor de $\xi=0$. Evalúa los primeros tres coeficientes en términos de $a_{0}$
\[  \begin{split}
& \text{Ecuación indicial } \hspace{1.5cm} k^{2} - \dfrac{m^{2}}{4} = 0 \\
u(\xi) &=  a_{0} \xi^{m/2} \left\lbrace 1 - \dfrac{\alpha}{m+1} \xi + \left[ \dfrac{\alpha^{2}}{2(m+1)(m+2)} - \dfrac{E}{4(m+2)} \right] \xi^{2} + \ldots \right\rbrace
\end{split} \]
Checa que la perturbación $E$ no se presenta hasta que el término $a_{3}$ se incluye.
\item Para el caso especial en donde no hay dependencia en la coordenada azimutal, del estudio del ion molecular del hidrógeno $(H2^{+})$ en mecánica cuántica, se llega a la ecuación
\[ \dfrac{d}{d \eta} \left[ (1 - \eta^{2} ) \dfrac{d u}{d \eta} \right] + \alpha u + \beta \eta^{2} u = 0 \]
Desarrolla una solución en series de potencias para $u(\eta)$. Evalúa los primeros tres coeficientes no nulos en términos de $a_{0}$
\[  \begin{split}
& \text{Ecuación indicial } \hspace{1.5cm} k(k-1) = 0 \\
u_{k=1} &=  a_{0} \eta \left\lbrace 1 - \dfrac{2- \alpha}{6} \eta^{2} + \left[ \dfrac{(2-\alpha)(12-\alpha)}{120} - \dfrac{\beta}{20} \right] \eta^{4} + \ldots \right\rbrace
\end{split} \] 

\item Inicia con 
\[ J_{0} (x) = 1 - \dfrac{x^{2}}{4} +  \dfrac{x^{4}}{64} -  \dfrac{x^{6}}{2304} + \ldots \]
Obtén la segunda solución linealmente independiente
\[ y_{2}(x) = J_{0}(x) ln(x) + \dfrac{x^{2}}{4} -  \dfrac{3 x^{4}}{128} +  \dfrac{11 x^{6}}{13284}+ \ldots \]
de la ecuación de Bessel de orden cero.
\end{enumerate}

\end{document}