\documentclass[12pt]{beamer}
\usepackage{../Estilos/BeamerMAF}
\input{../Preambulos/preambulo_Beamer_Dresden_seahorse}

\date{3 de noviembre de 2021}

\title{\large{Funciones ortonormales y operadores}}
\subtitle{Tema 3 - Bases completas y ortogonales}
\author{M. en C. Gustavo Contreras Mayén}

\begin{document}
\maketitle
\fontsize{14}{14}\selectfont
\spanishdecimal{.}

\section*{Contenido}
\frame{\tableofcontents[currentsection, hideallsubsections]}

%Ref. Romero (2013)
\section{Bases ortonormales}
\frame{\tableofcontents[currentsection, hideothersubsections]}
\subsection{Algunos conjuntos}

\begin{frame}
\frametitle{Conjuntos de funciones}
Veremos algunos ejemplos de conjuntos de funciones que forman una base ortonormal.
\\
\bigskip
Es decir, que las funciones son ortogonales y normalizadas a la unidad.
\end{frame}
\begin{frame}
\frametitle{Exponencial compleja}
Consideremos el conjunto de funciones:
\pause
\begin{align}
\Phi_{n} (\varphi) = \dfrac{1}{\sqrt{2 \, \pi}} \, e^{i n \varphi}
\label{eq:ecuacion_06_67}
\end{align}
definidas en el intervalo $[0, 2 \, \pi]$, con $n$ un número entero.
\end{frame}
\begin{frame}
\frametitle{Revisando la ortogonalidad}
De acuerdo a la definición de ortogonalidad:
\pause
\begin{eqnarray}
\begin{aligned}[b]
\braket{\Phi_{n}(\varphi)}{\Phi_{m}(\varphi)} &= \scaleint{6ex}_{\bs 0}^{2 \pi} \big[ \Phi_{n}(\varphi) \big]^{*} \, \Phi_{m}(\varphi) \dd{\varphi} = \\[0.5em] \pause
&= \dfrac{1}{2 \pi} \, \scaleint{6ex}_{\bs 0}^{2 \pi} \exp(i (m - n)) \dd{\varphi}
\end{aligned}
\label{eq:ecuacion_06_68}
\end{eqnarray}
\end{frame}
\begin{frame}
\frametitle{Caso $m = n$}
Es claro que:
\pause
\begin{eqnarray}
\begin{aligned}[b]
\dfrac{1}{2 \pi} \, \scaleint{6ex}_{\bs 0}^{2 \pi} &\exp(i (m - n)) \dd{\varphi} =  \\[0.5em] \pause
&= \dfrac{1}{2 \pi} \, \scaleint{6ex}_{\bs 0}^{2 \pi} \exp(i (n - n)) \dd{\varphi} = \\[0.5em] \pause
&= 1
\end{aligned}
\label{eq:ecuacion:06_70}
\end{eqnarray}
\end{frame}
\begin{frame}
\frametitle{Caso $n \neq m$}
Ahora resulta en:
\pause
\begin{eqnarray}
\begin{aligned}[b]
\dfrac{1}{2 \pi} \, \scaleint{6ex}_{\bs 0}^{2 \pi} &\exp(i (m - n)) \dd{\varphi} = \\[0.5em] \pause
&= \dfrac{1}{2 \pi} \, \dfrac{1}{i(n -m)} \, \exp(i (m - n)) \eval_{0}^{2 \pi} = \\[0.5em] \pause
&= \dfrac{1}{2 \pi} \, \dfrac{1}{i(n -m)} \, \big[ (-1)^{2 (m-n)} - 1 \big] = \\[0.5em] \pause
&= 0
\end{aligned}
\label{eq:ecuacion:06_69}
\end{eqnarray}
\end{frame}
\begin{frame}
\frametitle{Conclusión de la ortogonalidad}
Entonces tenemos que:
\pause
\begin{align}
\braket{\Phi_{n}(\varphi)}{\Phi_{m}(\varphi)} = \delta_{mn}
\label{eq:ecuacion_06_70}    
\end{align}
Es un conjunto ortogonal, \pause y entonces, linealmente independiente.
\end{frame}

\begin{frame}
\frametitle{Operador momento angular}
Consideremos las funciones propias del operador $\vb{L}^{2}$, el operador momento angular.
\pause
\begin{align*}
\vb{L}^{2} \, Y_{\lambda} (\theta, \varphi) =  \lambda \, Y_{\lambda} (\theta, \varphi)
\end{align*}
\end{frame}
\begin{frame}
\frametitle{Ecuación diferencial}
Que deben de satisfacer:
\pause
\begin{align*}
&\vb{L}^{2} \, Y_{\lambda} (\theta, \varphi) = - \bigg[ \dfrac{1}{\sin \theta} \, \pdv{\theta} \left( \sin \theta \, \pdv{\theta} Y_{\lambda} (\theta, \varphi) \right) + \\[0.5em]
&+ \dfrac{1}{\sin^{2} \theta} \, \pdv[2]{\varphi} \, Y_{\lambda} (\theta, \varphi) \bigg] = \lambda \, Y_{\lambda} (\theta, \varphi)
\end{align*}
\pause
En este momento no resolveremos la ecuación, pero veremos algunas de sus propiedades con respecto a la ortonormalidad.
\end{frame}
\begin{frame}
\frametitle{Solución propuesta}
Se propone la siguiente solución: $Y_{\lambda} (\theta, \varphi) = \Theta(\theta) \, \Phi(\varphi)$, de tal modo que:
\pause
\begin{align*}
&\vb{L}^{2} \, Y_{\lambda} (\theta, \varphi) =  - \bigg[ \dfrac{\Phi(\varphi)}{\sin \theta} \, \pdv{\theta} \left( \sin \theta \, \pdv{\Theta(\theta)}{\theta} Y_{\lambda} (\theta, \varphi) \right) + \\[0.5em]
&+ \dfrac{\Theta(\theta)}{\sin^{2} \theta} \, \pdv[2]{\Phi(\varphi)}{\varphi} \, Y_{\lambda} (\theta, \varphi) \bigg] = \lambda \, \Theta(\theta) \, \Phi(\varphi)
\end{align*}
\end{frame}
\begin{frame}
\frametitle{Separando la ecuación}
Se tiene que:
\pause
\begin{equation}
\begin{aligned}[b]
&\left( \dfrac{\sin^{2} \theta \, \vb{L}^{2} \, Y_{\lambda} (\theta, \varphi)}{Y_{\lambda m} (\theta, \varphi)} \right) = \\[0.5em] \pause 
&= - \bigg[ \dfrac{\sin \theta}{\Theta} \, \pdv{\theta} \left( \sin \theta \, \pdv{\Theta}{\theta} \right) + \dfrac{1}{\Phi} \, \pdv[2]{\Phi}{\varphi} \bigg] = \\[0.5em] \pause
&= \lambda \, \sin^{2} \theta
\end{aligned}
\label{eq:ecuacion_06_91}
\end{equation}
\end{frame}
\begin{frame}
\frametitle{Avanzando en la separación}
Esto implica que:
\pause
\begin{align}
\pdv{\varphi} \left( \dfrac{\sin^{2} \theta \, \vb{L}^{2} \, Y_{\lambda} (\theta, \varphi)}{Y_{\lambda m} (\theta, \varphi)} \right) = - \left( \dfrac{1}{\Phi} \, \pdv[2]{\Phi}{\varphi} \right) = 0
\end{align}
\pause
entonces:
\pause
\begin{eqnarray}
\dfrac{1}{\Phi} \, \dv[2]{\Phi}{\varphi} &=& - m^{2} \hspace{1cm} \mbox{constante} \label{eq:ecuacion_06_93} \\[0.5em] \pause
\dv[2]{\Phi}{\varphi} &=& - m^{2} \, \Phi \label{eq:ecuacion_06_94}
\end{eqnarray}
\end{frame}
\begin{frame}
\frametitle{Seguimos en la separación}
Sustituyendo la ec. (\ref{eq:ecuacion_06_93}) en la ec. (\ref{eq:ecuacion_06_91}), llegamos a:
\begin{align}
- \bigg[ \dfrac{\sin \theta}{\Theta} \, \pdv{\theta} \left( \sin \theta \, \pdv{\Theta}{\theta} \right) - m^{2} \bigg] = \lambda \, \sin^{2} \theta
\label{eq:ecuacion_06_95}
\end{align}
\pause
que se puede escribir como:
\pause
\begin{align}
\pdv{\theta} \left( \sin \theta \, \pdv{\Theta (\theta)}{\theta} \right) {+} \left( \lambda \, \sin \theta {-} \dfrac{m^{2}}{\sin \theta} \right) \, \Theta (\theta) = 0
\label{eq:ecuacion_06_96}
\end{align}
\end{frame}
\begin{frame}
\frametitle{Dependencia de la ecuación}
La ec. (\ref{eq:ecuacion_06_96}) depende de los parámetros $\lambda$ y $m$, por lo que se redefine la función $\Theta(\theta)$ como:
\begin{align*}
\Theta(\theta) = P_{\lambda}^{m} (\cos \theta).
\end{align*}
\end{frame}
\begin{frame}
\frametitle{Ecuación de tipo S-L}
La misma ec. (\ref{eq:ecuacion_06_96}) es de tipo Sturm-Liouville, con:
\pause
\begin{align}
p(\theta) = \sin \theta, \hspace{0.5cm} q(\theta) = \sin \theta, \hspace{0.5cm} \sigma(\theta) = - \dfrac{m^{2}}{\sin \theta}
\label{eq:ecuacion_06_97}
\end{align}
\pause
Además se tiene que:
\pause
\begin{align*}
p(0) = \sin 0  = p(\pi) = \sin \pi = 0
\end{align*}
las soluciones de la ec. (\ref{eq:ecuacion_06_96}) son ortonormales en el intervalo $[0, \pi]$, con función de peso $\sin \theta$.
\end{frame}
\begin{frame}
\frametitle{Ortogonalidad de las funciones}
Es decir que:
\pause
\begin{align}
\scaleint{5ex}_{\bs 0}^{\pi} \sin \theta \, P_{\lambda^{\prime}}^{m} (\cos \theta) \, P_{\lambda}^{m} (\cos \theta) = \alpha_{\lambda m} \, \delta_{\lambda^{\prime} \lambda}
\label{eq:ecuacion_06_98}
\end{align}
donde $\alpha_{\lambda m} =$ constante $> 0$.
\end{frame}
\begin{frame}
\frametitle{Solución a una EDO}
Revisemos que las funciones:
\pause
\begin{align}
\Phi_{m} (\varphi) = A_{0} \, \exp(i \, m \, \varphi)
\label{eq:ecuacion_06_99}
\end{align}
\pause
son soluciones de la ec. (\ref{eq:ecuacion_06_94}). \pause En particular si todos los $m$ son enteros, el conjunto de funciones:
\pause
\begin{align}
\Phi_{m}(\varphi) = \dfrac{\exp(i m \varphi)}{\sqrt{2 \, \pi}
}
\label{eq:ecuacion_06_100}
\end{align}
son ortonormales en el intervalo $[0, 2 \, \pi]$.
\end{frame}
\begin{frame}
\frametitle{Funciones ortonormales}
Entonces, si el conjunto de $m$ está en los enteros, las funciones:
\pause
\begin{align}
Y_{\lambda m} (\theta, \varphi) = \dfrac{1}{\sqrt{\alpha_{\lambda m}} \, \sqrt{2 \, \pi}} \, \exp(i \, m \, \varphi) \, P_{\lambda}^{m} (\cos \theta)
\label{eq:ecuacion_06_101}
\end{align}
son ortonormales.
\end{frame}
\begin{frame}
\frametitle{La ortogonalidad de las funciones}
Considerando la ortonormalidad de las funciones:
\begin{align*}
\dfrac{\exp(i \, m \, \varphi)}{\sqrt{2 \, \pi}} \hspace{1.5cm} P_{\lambda}^{m} (\cos \theta)
\end{align*}
se tiene que:
\pause
\begin{align*}
\braket{Y_{\lambda^{\prime} m^{\prime}} (\theta, \varphi)}{Y_{\lambda m} (\theta, \varphi)} = \scaleint{6ex} \, Y_{\lambda^{\prime} m^{\prime}} (\theta, \varphi) \, Y_{\lambda m} (\theta, \varphi) \dd{\Omega}
\end{align*}
\end{frame}
\begin{frame}
\frametitle{La ortogonalidad de las funciones}
\begin{eqnarray*}
\begin{aligned}
&\braket{Y_{\lambda^{\prime} m^{\prime}} (\theta, \varphi)}{Y_{\lambda m} (\theta, \varphi)} = \\[0.5em] \pause
&= \scaleint{6ex}_{\bs 0}^{2 \pi} \scaleint{6ex}_{\bs 0}^{\pi}  \sin \theta \, Y_{\lambda^{\prime} m^{\prime}}^{*} (\theta, \varphi) \, Y_{\lambda m} (\theta, \varphi) \dd{\theta} \dd{\varphi} = \\[0.5em] \pause
&= \scaleint{6ex}_{\bs 0}^{2 \pi} \scaleint{6ex}_{\bs 0}^{\pi}  \sin \theta \left( \dfrac{1}{\sqrt{\alpha_{\lambda^{\prime} m^{\prime}}} \sqrt{2 \, \pi}} \, e^{i  m \varphi} P_{\lambda}^{m} (\cos \theta) \right)^{*} \times \\[0.5em]
&\times \left( \dfrac{1}{\sqrt{\alpha_{\lambda m}} \, \sqrt{2 \, \pi}} \, e^{i m \varphi} \, P_{\lambda}^{m} (\cos \theta) \right) \dd{\theta} \dd{\varphi} =
\end{aligned}
\end{eqnarray*}
\end{frame}
\begin{frame}
\frametitle{La ortogonalidad de las funciones}
\begin{eqnarray*}
\begin{aligned}
&= \scaleint{6ex}_{\bs 0}^{2 \pi} \scaleint{6ex}_{\bs 0}^{\pi}  \sin \theta \, \left( \dfrac{1}{\sqrt{\alpha_{\lambda^{\prime} m^{\prime} } } \sqrt{2 \, \pi}} \, e^{i  m^{\prime} \varphi} P_{\lambda^{\prime}}^{m^{\prime}} (\cos \theta) \right) \times \\[0.5em]
&\times \left( \dfrac{1}{\sqrt{\alpha_{\lambda m}} \, \sqrt{2 \, \pi}} \, e^{i m \varphi} \, P_{\lambda}^{m} (\cos \theta) \right) \dd{\theta} \dd{\varphi} = 
\end{aligned}
\end{eqnarray*}
\end{frame}
\begin{frame}
\frametitle{La ortogonalidad de las funciones}
\begin{eqnarray*}
\begin{aligned}
&= \dfrac{1}{\sqrt{\alpha_{\lambda^{\prime} m^{\prime}}}  \sqrt{\alpha_{\lambda m}}} \, \bigg[ \dfrac{1}{2 \pi} \scaleint{6ex}_{\bs 0}^{2 \pi}  e^{-i m \varphi} \, e^{i m \varphi} \dd{\varphi} \bigg] \times \\[0.5em] 
&\times \scaleint{6ex}_{\bs 0}^{\pi} \sin \theta \, P_{\lambda^{\prime}}^{m^{\prime}} (\cos \theta) \, P_{\lambda}^{m} (\cos \theta) \dd{\theta} =
\end{aligned}
\end{eqnarray*}
\end{frame}
\begin{frame}
\frametitle{La ortogonalidad de las funciones}
\begin{eqnarray*}
\begin{aligned}
&= \dfrac{1}{\sqrt{\alpha_{\lambda^{\prime} m^{\prime}}}  \sqrt{\alpha_{\lambda m}}} \delta_{m m^{\prime}}
\scaleint{6ex}_{\bs 0}^{\pi} \! \sin \theta \, P_{\lambda^{\prime}}^{m^{\prime}} (\cos \theta) \, P_{\lambda}^{m} (\cos \theta) \dd{\theta} = \\[0.5em] \pause
&= \dfrac{1}{\sqrt{\alpha_{\lambda^{\prime} m}}  \sqrt{\alpha_{\lambda m}}} \delta_{m m^{\prime}}
\scaleint{6ex}_{\bs 0}^{\pi} \! \sin \theta \, P_{\lambda^{\prime}}^{m} (\cos \theta) \, P_{\lambda}^{m} (\cos \theta) \dd{\theta} = \\[0.5em] \pause
&= \dfrac{1}{\sqrt{\alpha_{\lambda^{\prime} m}}  \sqrt{\alpha_{\lambda m}}} \delta_{m m^{\prime}} \, \alpha_{\lambda m} \, \delta_{\lambda^{\prime} \lambda}
\end{aligned}
\end{eqnarray*}
\end{frame}
\begin{frame}
\frametitle{La ortogonalidad de las funciones}
\begin{eqnarray*}
\begin{aligned}    
\dfrac{1}{\sqrt{\alpha_{\lambda m}}  \sqrt{\alpha_{\lambda m}}} \, \alpha_{\lambda m} \, \delta_{\lambda^{\prime} \lambda} = \pause \delta_{m m^{\prime}} \, \delta_{\lambda \lambda^{\prime}}
\end{aligned}
\end{eqnarray*}
\pause
Es decir:
\begin{align}
\braket{Y_{\lambda^{\prime} m^{\prime}} (\theta, \varphi)}{Y_{\lambda m} (\theta, \varphi)} = \delta_{m m^{\prime}} \, \delta_{\lambda \lambda^{\prime}}
\end{align}
\end{frame}
\begin{frame}
\frametitle{Conclusión}
Aunque no sabemos aún cuál es la forma explícita de las funciones $Y_{\lambda m} (\theta, \varphi)$, pero podemos concluir que son un \textcolor{blue}{conjunto de funciones ortogonales}.
\\
\bigskip
\pause
Las funciones $Y_{\lambda m} (\theta, \varphi)$ conocidas como \textcolor{red}{\emph{armónicos esféricos}}, son de importancia en la mecánica cuántica y en la electrodinámica.
\end{frame}

\section{Operadores}
\frame{\tableofcontents[currentsection, hideothersubsections]}
\subsection{Operadores lineales}

\begin{frame}
\frametitle{Definición}
Sea $V$ un espacio vectorial, una función $O:V \to V$ es un operador lineal (\emph{transformación lineal}) si:
\pause
\begin{eqnarray}
\begin{aligned}[b]
&\forall v_{1}, v_{2} \in V, \forall \alpha, \beta \in K  \\[0.5em] \pause
&O(\alpha \, v_{1} + \beta \, v_{2}) = \alpha \, O(v_{1}) + \beta \, O(v_{2})
\end{aligned}
\label{eq:ecuacion_06_116}
\end{eqnarray}
\end{frame}
\begin{frame}
\frametitle{El operador derivada}
El operador derivada es lineal, ya que:
\pause
\begin{align*}
&\pdv{x} \bigg[ \alpha \, f_{1}(x) + \beta \, f_{2}(x) \bigg] = \\[0.5em]
&= \alpha \, \pdv{x} \, f_{1}(x) + \beta \, \pdv{x} \, f_{2}(x)
%\label{eq:ecuacion_06_117}
\end{align*}
\end{frame}
\begin{frame}
\frametitle{Operador lineal con una función}
Usando el producto por un escalar, con una función $f(x)$, se puede definir el operador lineal $O$, como:
\pause
\begin{align*}
O(v_{1}) = f(x) \, v_{1}
\end{align*}
\end{frame}
\begin{frame}
\frametitle{Operador lineal con una función}
Esto operador es lineal, ya que:
\pause
\begin{eqnarray*}
\begin{aligned}
O(\alpha \, v_{1} + \beta \, v_{2}) &= f(x) (\alpha \, v_{1} + \beta \, v_{2}) = \\[0.5em] \pause
&= \alpha \, f(x) \, v_{1} + \beta \, f(x) \, v_{2} = \\[0.5em] \pause
&= \alpha \, O(v_{1}) + \beta \, O(v_{2})
\end{aligned}
\end{eqnarray*}
\end{frame}
\begin{frame}
\frametitle{Combinación lineal de operadores}
Dadas las transformaciones lineales $O_{1}$ y $O_{2}$, cualquier combinación lineal de ellas también es una transformación lineal.
\\
\bigskip
\pause
Con los escalares $a$ y $b$, se construye la combinación lineal:
\begin{align*}
O = a \, O_{1} + b \, O_{2}
\end{align*}
\end{frame}
\begin{frame}
\frametitle{Combinación lineal de operadores}
\vspace*{-1cm}
\begin{eqnarray*}
\begin{aligned}
&O(\alpha \, v_{1} + \beta \, v_{2}) = (a \, O_{1} + b \, O_{2})(\alpha \, v_{1} + \beta \, v_{2}) = \\[0.15em] \pause
&= a \, O_{1} \, (\alpha \, v_{1} + \beta \, v_{2}) + b \, O_{2} \, (\alpha \, v_{1} + \beta \, v_{2}) = \\[0.15em] \pause
&= a \big[ \alpha \, O_{1} (v_{1}) + \beta \, O_{1}(v_{2})  \big] + b \big[ \alpha \, O_{2} (v_{1}) + \beta \, O_{2}(v_{2}) \big] = \\[0.15em] \pause
&= \alpha \big[ a \, O_{1}(v_{1}) + b \, O_{2}(v_{1}) \big] + \beta \big[ a \, O_{1}(v_{2}) + b \, O_{2}(v_{2}) \big] = \\[0.15em] \pause
&= \alpha \big[ a \, O_{1} + b \, O_{2} \big] (v_{1}) + \beta \big[ a \, O_{1} + b \, O_{2} \big] (v_{2}) = \\[0.15em] \pause
&= \alpha \, O(v_{1}) + \beta \, O(v_{2})
\end{aligned}
\end{eqnarray*}
Cualquier combinación lineal de dos operadores lineales, devuelve otro operador lineal.
\end{frame}
\begin{frame}
\frametitle{Producto de dos operadores}
Definimos el producto de dos operadores lineales como:
\pause
\begin{align*}
O = O_{1} \, O_{2}
\end{align*}
\pause
Entonces:
\pause
\begin{eqnarray*}
\begin{aligned}
O(\alpha \, v_{1} + \beta \, v_{2}) &= (O_{1} \, O_{2})(\alpha \, v_{1} + \beta \, v_{2}) = \\[0.5em] \pause
&= \ldots \\[0.5em] \pause
&= \alpha \, O(v_{1}) + \beta \, O(v_{2})
\end{aligned}
\end{eqnarray*}
\end{frame}
\begin{frame}
\frametitle{Producto de dos operadores}
Por lo tanto, el producto de dos operadores lineales, devuelve otro operador lineal.
\\
\bigskip
\pause
Revisemos algunos ejemplos de operadores lineales.
\end{frame}

\subsection{Ejemplos de operadores lineales}

\begin{frame}
\frametitle{Ejemplos de operadores lineales}
Los operadores:
\pause
\begin{align*}
\pdv{x} \hspace{1cm} \pdv{y} \hspace{1cm} \pdv{z} 
\end{align*}
son operadores lineales. \pause Esto implica que el operador Laplaciano:
\begin{align*}
\laplacian = \pdv[2]{x} + \pdv[2]{y} + \pdv[2]{z}
\end{align*}
sea un operador lineal.
\end{frame}
\begin{frame}
\frametitle{Funciones como operadores}
Cualquier función $V(x, y, z)$ como operador, es lineal. \pause Entonces el operador Hamiltoniano:
\begin{align*}
H = - \dfrac{\hbar}{2 m} \, \laplacian + V(x, y, z)
\end{align*}
\pause
es lineal, ya que es una combinación lineal de operadores lineales.
\end{frame}
\begin{frame}
\frametitle{Variables como operadores}
Las variables: \pause $x, y, z$ como operadores, son lineales. \pause Entonces los operadores:
\pause
\begin{align*}
L_{x} &= -i \left( y \pdv{z} - z \pdv{y} \right) \\[0.25em]
L_{y} &= -i \left( z \pdv{x} - x \pdv{z} \right) \\[0.25em]
L_{z} &= -i \left( x \pdv{y} - y \pdv{x} \right)
\end{align*}
\pause
Son lineales, ya que son combinaciones lineales de productos de operadores lineales.
\end{frame}
\begin{frame}
\frametitle{Operador momento angular}
Por la misma razón, el operador:
\begin{align*}
\vb{L}^{2} = L_{x}^{2} + L_{y}^{2} + L_{z}^{2}
\end{align*}
es lineal.
\end{frame}

\subsection{Operador adjunto}

\begin{frame}
\frametitle{Definición}
Dado un operador $A$ se define el operador adjunto $A^{\dagger}$ como el operador que satisface:
\pause
\begin{align*}
\braket{A \, v}{u} = \braket{v}{A^{\dagger} \, u}
\end{align*}
\\
\bigskip
\pause
Revisemos la derivada y dos propiedades importantes de los operadores adjuntos.
\end{frame}
\begin{frame}
\frametitle{El operador derivada}
Para el espacio vectorial de las funciones, el adjunto de un operador depende fuertemente del dominio, las CDF que se satisfacen y del producto escalar.
\end{frame}
\begin{frame}
\frametitle{El operador derivada}
En el espacio de las funciones suaves e integrables $\psi (x)$ definidas en $[a, b]$ y que cumplen:
\pause
\begin{align*}
\psi(a) = \psi(b) = 0
\end{align*}
\pause
se define el operador:
\pause
\begin{align*}
A = \alpha \, \pdv{x}
\end{align*}
con $\alpha$ un número complejo.
\end{frame}
\begin{frame}
\frametitle{El operador adjunto}
El adjunto de este operador (que se puede demostrar sin contratiempo) es:
\pause
\begin{align*}
\left( \alpha \, \pdv{x} \right)^{\dagger} = - \alpha^{*} \, \pdv{x}
\end{align*}
Este resultado depende necesariamente de que se cumpla:
\begin{align*}
\psi(a) = \psi(b) = 0
\end{align*}
\end{frame}
\begin{frame}
\frametitle{Propiedad de la suma}
Si $A$ y $B$ son dos operadores lineales, se tiene que:
\pause
\begin{align*}
\braket{(A + B) \, v}{u} = \braket{v}{(A + B)^{\dagger} \, u}
\end{align*}
\pause
Pero:
\pause
\begin{eqnarray*}
\begin{aligned}
\braket{(A + B) \, v}{u} &= \braket{(A \, v + B \, v) }{u} = \\[0.5em] \pause
&= \braket{A \, v}{u} + \braket{B \, v}{u} = \\[0.5em] \pause
&= \braket{v}{A^{\dagger} \, u} + \braket{v}{B^{\dagger} \, u} = \\[0.5em]  \pause
&= \braket{v}{(A^{\dagger} + B^{\dagger}) \, u}
\end{aligned}
\end{eqnarray*}
\end{frame}
\begin{frame}
\frametitle{Propiedad de la suma}
Por lo tanto:
\pause
\begin{align*}
\braket{v}{(A + B)^{\dagger} \, u} = \braket{v}{(A^{\dagger} + B^{\dagger}) \, u}
\end{align*}
\pause
Este resultado es válido para cualquier par de vectores $v$ y $u$, entonces:
\pause
\begin{align*}
(A + B)^{\dagger}  = A^{\dagger} + B^{\dagger}
\end{align*}    
\end{frame}
\begin{frame}
\frametitle{Propiedad con el producto de dos operadores}
Para el producto de dos operadores:
\begin{eqnarray*}
\begin{aligned}
\braket{(A \, B) \, v}{u} &= \braket{A (B \, v)}{u} = \\[0.5em] \pause
&= \braket{ B \, v}{A^{\dagger} \, u} = \\[0.5em] \pause
&= \braket{v}{B^{\dagger} \, A^{\dagger} \, u} = \\[0.5em] \pause
&= \braket{v}{(A \, B)^{\dagger} \, u} \\[0.5em] \pause
\Rightarrow (A \, B)^{\dagger} &= B^{\dagger} \, A^{\dagger}
\end{aligned}
\end{eqnarray*}
\end{frame}

\subsection{Operadores Hermíticos}

\begin{frame}
\frametitle{Definición}
Una clase importante de operadores son los autoadjuntos que satisfacen:
\pause
\begin{align*}
A^{\dagger} = A
\end{align*}
\pause
Estos operadores se les conoce como Hermíticos o Hermitianos.
\end{frame}
\begin{frame}
\frametitle{Propiedades de los operadores Hermíticos}
De la propiedad $(A + B)^{\dagger} = A^{\dagger} + B^{\dagger}$ se sigue que la suma de dos operadores Hermíticos, devuelve otro operador Hermítico.
\end{frame}
\begin{frame}
\frametitle{Propiedades de los operadores Hermíticos}
Con la propiedad $(A \, B)^{\dagger} = B^{\dagger} \, A^{\dagger}$, es claro que si $A$ y $B$ son operadores Hermíticos y \textbf{conmutan}, es decir $A \, B = B \, A$, se tiene que:
\pause
\begin{align*}
(A \, B)^{\dagger} = B^{\dagger} \, A^{\dagger} = B \, A = A \, B
\end{align*}
\\
\bigskip
\pause
El producto de dos operadores Hermíticos que \textbf{conmutan} es Hermítico.
\end{frame}
\begin{frame}
\frametitle{Ejemplos de operadores Hermíticos}
En el espacio de funciones, cualquier función real $f(\va{r})$ es un operador Hermítico.
\end{frame}
\begin{frame}
\frametitle{Ejemplos de operadores Hermíticos}
Del resultado:
\begin{align*}
\left( \alpha \, \pdv{x} \right)^{\dagger} = - \alpha^{*} \, \pdv{x}
\end{align*}
\pause
si $\alpha$ es complejo, por ejemplo: $\alpha = -i \hbar$, el operador: \pause
\begin{align*}
P_{x} = -i \, \hbar \, \pdv{x}
\end{align*}
es Hermítico.
\end{frame}
\begin{frame}
\frametitle{Ejemplos de operadores Hermíticos}
Como $P_{x}$ conmuta con $P_{x}$, entonces:
\pause
\begin{align*}
P_{x} \, P_{x} = P_{x}^{2}
\end{align*}
es un operador Hermítico.
\end{frame}
\begin{frame}
\frametitle{Ejemplos de operadores Hermíticos}
Si $V(x)$ es un potencial real, el operador Hamiltoniano de la mecánica cuántica:
\pause
\begin{align*}
H = \dfrac{1}{2 \, m} \, P_{x}^{2} + V(x)
\end{align*}
\pause
es Hermítico, ya que es la suma de dos operadores Hermíticos.
\end{frame}
\begin{frame}
\frametitle{Ejemplos de operadores Hermíticos}
Los operadores:
\pause
\begin{align*}
P_{x} = -i \, \hbar \, \pdv{x}, \hspace{0.3cm} P_{y} = -i \, \hbar \, \pdv{y}, \hspace{0.3cm} P_{z} = -i \, \hbar \, \pdv{z}
\end{align*}
son Hermíticos.
\end{frame}
\begin{frame}
\frametitle{Ejemplos de operadores Hermíticos}
Si $V(\va{r})$ es una función real, el operador:
\pause
\begin{align*}
H = \dfrac{1}{2 \, m} \, P^{2} +  V(\va{r}) = - \dfrac{\hbar^{2}}{2 \, m} \, \laplacian + V(\va{r})
\end{align*}
es Hermítico.
\end{frame}
\end{document}