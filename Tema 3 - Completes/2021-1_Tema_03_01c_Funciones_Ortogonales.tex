\input{../Preambulos/preambulo_presentacion_Warsaw_crane}
\title{\large{Gram-Schmidt y Completez}}
\subtitle{Ejercicios}
\author{M. en C. Gustavo Contreras Mayén}
\date{}
\institute{Facultad de Ciencias - UNAM}
\titlegraphic{\includegraphics[width=1.75cm]{../Imagenes/escudo-facultad-ciencias}\hspace*{4.75cm}~%
   \includegraphics[width=1.75cm]{../Imagenes/escudo-unam}
}
\setbeamertemplate{navigation symbols}{}
\begin{document}
\maketitle
\fontsize{14}{14}\selectfont
\spanishdecimal{.}
\section*{Contenido}
\frame[allowframebreaks]{\tableofcontents[currentsection, hideallsubsections]}
\section{Operadores}
\frame{\tableofcontents[currentsection, hideothersubsections]}
\subsection{Ejercicio}
\begin{frame}
\frametitle{Ejercicio}
Considera las siguientes funciones de onda unidimensionales que están normalizadas: $\psi_{0}(x)$ y $\psi_{1}(x)$, que cuentan con las propiedades:
\begin{align*}
\psi_{0}(-x) &= \psi_{0}(x) = \psi_{0}^{*} (x) \\
\psi_{1}(x) &= N \, \dv{\psi_{0}}{x}
\end{align*}
\end{frame}
\begin{frame}
\frametitle{Ejercicio}
Considera también la combinación lineal
\begin{align*}
\psi(x) = c_{1} \, \psi_{0}(x) + c_{2} \, \psi_{1} (x)
\end{align*}
con $\abs{c_{1}}^{2} + \abs{c_{2}}^{2} = 1$. Las constantes $N, c_{1}, c_{2}$, las consideramos conocidas.
\end{frame}
\begin{frame}
\frametitle{Ejercicio}
\setbeamercolor{item projected}{bg=blue!70!black,fg=yellow}
\setbeamertemplate{enumerate items}[circle]
\begin{enumerate}[<+->]
\item Demuestra que $\psi_{0}(x)$ y $\psi_{1}(x)$ son ortogonales y que $\psi(x)$ está normalizada.
\item Calcula los valores esperados de $x$ y $p$ en los estados $\psi_{0}(x)$, $\psi_{1}(x)$ y $\psi(x)$.
\end{enumerate}
\end{frame}
% \begin{frame}
% \frametitle{Ejercicio}
% \setbeamercolor{item projected}{bg=blue!70!black,fg=yellow}
% \setbeamertemplate{enumerate items}[circle]
% \begin{enumerate}[<+->]
% \conti
% \item Calcula el valor esperado para la energía cinética $T$ en el estado $\psi_{0}(x)$ y demuestra que
% \begin{align*}
% \bra{\psi_{0}} T^{2} \ket{\psi_{0}} = \bra{\psi_{0}} T \ket{\psi_{0}} \, \bra{\psi_{1}} T \ket{\psi_{1}}
% \end{align*}
% y que 
% \begin{align*}
% \bra{\psi_{1}} T \ket{\psi_{1}} \geq \bra{\psi} T \ket{\psi} \, \bra{\psi_{0}} T \ket{\psi_{0}}
% \end{align*}
% \end{enumerate}
% \end{frame}
\begin{frame}
\frametitle{Solución inciso 1}
Para demostrar que $\psi_{0}(x)$ y $\psi_{1}(x)$ son ortogonales, hay que calcular $\braket{\psi_{0}}{\psi_{1}}$:
\\
\bigskip
\pause
Donde:
\begin{align*}
\braket{\psi_{0}}{\psi_{1}} = N \, \int_{-\infty}^{\infty} \psi_{0}^{*} \, \dv{\psi_{0}}{x} \dd{x}
\end{align*}
\end{frame}
\begin{frame}
\frametitle{Solución iniciso 1}
De acuerdo a la propiedad de:
\begin{align*}
\psi_{0}(-x) &= \psi_{0}(x) = \psi_{0}^{*} (x)
\end{align*}
\pause
Tendremos que:
\begin{align*}
\braket{\psi_{0}}{\psi_{1}} = N \, \int_{-\infty}^{\infty} \psi_{0} \, \dv{\psi_{0}}{x} \dd{x}
\end{align*}
\end{frame}
\begin{frame}
\frametitle{Solución al inciso 1}
El integrando lo podemos simplificar, considerando la derivada de una función al cuadrado, con su respectivo factor $1/2$, así:
\begin{eqnarray*}
\braket{\psi_{0}}{\psi_{1}} &=& \dfrac{N}{2} \, \int_{-\infty}^{\infty} \dv{\psi_{0}^{2}}{x} \dd{x} = \\[0.5em] \pause
&=& \dfrac{N}{2} \, \psi_{0}^{2} (x) \eval_{-\infty}^{\infty}
\end{eqnarray*}
\pause
Como $\psi_{0}$ es una función impar:
\begin{align*}
\braket{\psi_{0}}{\psi_{1}} = 0 \hspace{1cm} \mbox{por tanto } \psi_{0}(x) \perp \psi_{1}(x)
\end{align*}
\end{frame}
\begin{frame}
\frametitle{Segunda parte del inciso 1}
Ahora hay que demostrar que $\psi(x)$ está normalizada, por lo que habrá que demostrar que: $\braket{\psi(x)}{\psi(x)} = 1$:
\\
\bigskip
\pause
Por definición:
\begin{align*}
\braket{\psi}{\psi} = \int_{-\infty}^{\infty} \psi^{*} \, \psi \dd{x}
\end{align*}
\pause
Y por la combinación lineal dada:
\begin{align*}
\psi(x) = c_{1} \, \psi_{0}(x) + c_{2} \, \psi_{1} (x)
\end{align*}
\end{frame}
\begin{frame}
\frametitle{Solución segunda parte inciso 1}
Entonces:
\begin{eqnarray*}
\braket{\psi}{\psi} &=& \int_{-\infty}^{\infty} \psi^{*} \, \psi \dd{x} \\[0.5em] \pause
&=& \int_{-\infty}^{\infty} \bigg[ c_{1} \, \psi_{0} + c_{2} \, \psi_{1} \bigg]^{*} \, \bigg[ c_{1} \, \psi_{0} + c_{2} \, \psi_{1} \bigg] \dd{x} \\[0.5em] \pause 
&=& \int_{-\infty}^{\infty} \bigg[ (c_{1}^{*} \, \psi_{0}^{*})(c_{1} \, \psi_{0}) + (c_{1}^{*} \, \psi_{0}^{*})(c_{2} \, \psi_{1}) + \\[0.5em] 
&+& (c_{2}^{*} \, \psi_{1}^{*})(c_{1} \, \psi_{0}) + (c_{2}^{*} \, \psi_{1}^{*})(c_{2} \, \psi_{1}) \bigg] \dd{x} =
\end{eqnarray*}
\end{frame}
\begin{frame}
\frametitle{Solución segunda parte inciso 1}
Ocupando la propiedad $\psi_{0}(x) = \psi_{0}^{*} (x)$, tendremos que:
\begin{eqnarray*}
\braket{\psi}{\psi} &=& \int_{-\infty}^{\infty} \bigg[ \abs{c_{1}}^{2} \, \psi_{0}^{2} + (c_{1}^{*} \, c_{2} \, \psi_{0} \, \psi_{1}) + \\[0.5em] 
&+& (c_{2}^{*} \, c_{1} \, \psi_{1}^{*} \, \psi_{0}) + \abs{c_{2}}^{2} \, \psi_{1}^{2} \bigg] \dd{x} =
\end{eqnarray*}
\pause
Que podemos ver en términos de operadores como:
\end{frame}
\begin{frame}
\frametitle{Solución segunda parte inciso 1}
Así que:
\begin{eqnarray*}
\braket{\psi}{\psi} &=& \int_{-\infty}^{\infty} \bigg[ \abs{c_{1}}^{2} \, \braket{\psi_{0}}{\psi_{0}} + c_{1}^{*} \, c_{2} \, \braket{\psi_{0}}{\psi_{1}} + \\[0.5em] 
&+& c_{2}^{*} \, c_{1} \, \braket{\psi_{0}}{\psi_{1}} + \abs{c_{2}}^{2} \, \braket{\psi_{1}}{\psi_{1}} \bigg] \dd{x}=
\end{eqnarray*}
\pause
Dada la ortogonalidad entre $\psi_{0}(x)$ y $\psi_{1}(x)$, tendremos que:
\end{frame}
\begin{frame}
\frametitle{Solución segunda parte inciso 1}
Llegamos a:
\begin{eqnarray*}
\braket{\psi}{\psi} &=& \int_{-\infty}^{\infty} \bigg[ \abs{c_{1}}^{2} \, \braket{\psi_{0}}{\psi_{0}} + \abs{c_{2}}^{2} \, \braket{\psi_{1}}{\psi_{1}} \bigg] \dd{x} = \\[0.5em] \pause
\braket{\psi}{\psi} &=&  \abs{c_{1}}^{2} \, \int_{-\infty}^{\infty} \braket{\psi_{0}}{\psi_{0}} \dd{x} + \\[0.5em]
&+& \abs{c_{2}}^{2} \, \, \int_{-\infty}^{\infty} \braket{\psi_{1}}{\psi_{1}} \dd{x} =
\end{eqnarray*}
\pause
Como $\psi_{0}(x)$ y $\psi_{1}(x)$ están normalizadas 
\end{frame}
\begin{frame}
\frametitle{Solución segunda parte inciso 1}
Entonces se tiene que:
\begin{align*}
\braket{\psi}{\psi} = \abs{c_{1}}^{2} + \abs{c_{2}}^{2}
\end{align*}
\pause
El enunciado nos dice que:
\begin{align*}
\abs{c_{1}}^{2} + \abs{c_{2}}^{2} = 1
\end{align*}
\pause
Concluimos que:
\begin{align*}
\braket{\psi}{\psi} = 1 \hspace{1cm} \mbox{Por tanto } \psi(x) \mbox{ está normalizada}
\end{align*}
\end{frame}
\begin{frame}
\frametitle{Solución inciso 2}
Se calculará el valor esperado de $x$ para los estados $\psi_{0}(x)$, $\psi_{1}(x)$  $\psi(x)$
\\
\bigskip
\pause
Por definición:
\begin{align*}
\expval{\hat{x}}_{0} = \dfrac{\expval{\hat{x}}{\psi_{0}}}{\braket{\psi_{0}}{\psi_{0}}} = 
\end{align*}
\end{frame}
\begin{frame}
\frametitle{Solución inciso 2}
Como la función $\psi_{0}$ está normalizada: $\braket{\psi_{0}}{\psi_{0}} = 1$, así:
\pause
\begin{eqnarray*}
\expval{\hat{x}}_{0} &=& \int_{-\infty}^{\infty} x \, \psi_{0}^{*} \, \psi_{0} \dd{x} = \\[0.5em] \pause
&=& \int_{-\infty}^{\infty} x \, \psi_{0}^{2} \dd{x}
\end{eqnarray*}
\pause
Como en el integrando tenemos una función impar, entonces la integral se anula, por lo que:
\begin{align*}
\expval{\hat{x}}_{0} = 0
\end{align*}
\end{frame}
\begin{frame}
\frametitle{Solución inciso 2}
Ahora para $\hat{p}$, tenemos
Por definición:
\begin{align*}
\expval{\hat{p}}_{0} = \dfrac{\expval{\hat{p}}{\psi_{0}}}{\braket{\psi_{0}}{\psi_{0}}}
\end{align*}
\pause
Por definición, el operador de momento lineal es:
\begin{align*}
\hat{p}_{x} = - i \, \hbar \, \pdv{x}
\end{align*}
\end{frame}
\begin{frame}
\frametitle{Solución inciso 2}
\fontsize{12}{12}\selectfont
Entonces:
\begin{eqnarray*}
\expval{\hat{p}}_{0} &=& \expval{\hat{p}}{\psi_{0}} = \int_{-\infty}^{\infty} \psi_{0}^{*} \left( - i \, \hbar \, \pdv{x} \psi_{0} \right) \dd{x} = \\[0.5em] \pause
&=& -i \, \hbar \, \int_{-\infty}^{\infty} \psi_{0} \ptilde{\psi}_{0} \dd{x} = \\[0.5em] \pause
&=& \dfrac{-i \, \hbar}{N} \, \int_{-\infty}^{\infty} \psi_{0} \, \psi_{1} \dd{x} = \\[0.5em] \pause
&=& \dfrac{-i \, \hbar}{N} \, \braket{\psi_{0}}{\psi_{1}} = \\[0.5em] \pause
\expval{\hat{p}}_{0} &=& 0
\end{eqnarray*}
\end{frame}
\begin{frame}
\frametitle{Solución inciso 2}
Ahora para el estado $\psi_{1}$:
\\
\bigskip
\pause
\begin{align*}
\expval{\hat{x}}_{1} = \dfrac{\expval{\hat{x}}{\psi_{1}}}{\braket{\psi_{1}}{\psi_{1}}} = 0
\end{align*}
Ya que al desarrollar la expresión, se llega a un integrando del tipo $x \, \psi_{1}^{2}$ que se anula al evaluar en $(-\infty, \infty)$ 
\end{frame}
\begin{frame}
\frametitle{Solución al inciso 2}
Ahora para 
\begin{eqnarray*}
\expval{\hat{p}}_{1} &=& \dfrac{\expval{\hat{p}}{\psi_{1}}}{\braket{\psi_{1}}{\psi_{1}}} = \\[0.5em] \pause
&=& - i \, \hbar \, \int_{-\infty}^{\infty} \psi_{1}^{*} \, \ptilde{\psi}_{1} \dd{x} = \\[0.5em] \pause
&=& - \dfrac{i \, \hbar \, N}{N^{*}} \, \int_{-\infty}^{\infty} \psi_{1} \, \ptilde{\psi}_{1} \dd{x} \\[0.5em] \pause
&=& - \dfrac{i \, \hbar \, N}{N^{*}} \, \int_{-\infty}^{\infty} \dv{\psi_{1}^{2}}{x} \dd{x} = 
\end{eqnarray*}
\end{frame}
\begin{frame}
\frametitle{Solución al inciso 2}
Entonces
\begin{eqnarray*}
\expval{\hat{p}}_{1} = - \dfrac{i \, \hbar \, N}{N^{*}} \, \psi_{1}^{2} \eval_{-\infty}^{\infty} = 0
\end{eqnarray*}
\end{frame}
\begin{frame}
\frametitle{Última parte del inciso}
Queda por resolver:
\begin{align*}
\expval{\hat{x}}_{\psi} &= \dfrac{\expval{\hat{x}}{\psi}}{\braket{\psi}{\psi}} \\[1em]
\expval{\hat{p}}_{\psi} &= \dfrac{\expval{\hat{p}}{\psi}}{\braket{\psi}{\psi}}
\end{align*}
\pause
Resuélvelos y envialos a más tardar este domingo 15 de noviembre y tendrás un punto por cada respuesta bien desarrollada.

\end{frame}
\end{document}