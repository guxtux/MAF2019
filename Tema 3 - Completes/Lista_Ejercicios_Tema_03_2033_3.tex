\documentclass[12pt]{beamer}
\usepackage{../Estilos/BeamerMAF}
\usepackage{../Estilos/ColoresLatex}
\input{../Preambulos/preambulo_Beamer_Warsaw_seahorse}

\title{\large{Evaluación Semanal Tema 3}}
\subtitle{Tema 3 - Bases completas y ortogonales}

\author{M. en C. Gustavo Contreras Mayén}

\date{}

\begin{document}
\maketitle
\fontsize{14}{14}\selectfont
\spanishdecimal{.}

\section*{Contenido}
\frame[allowframebreaks]{\tableofcontents[currentsection, hideallsubsections]}

\section{Evaluación Tema 1}
\frame{\tableofcontents[currentsection, hideothersubsections]}
\subsection{Enunciado 1}

\begin{frame}
\frametitle{Enunciado 1}
Considera el problema:
\begin{align*}
\sderivada{y} = \sin x \hspace{1.5cm} y (\pi) = 0, \hspace{0.7cm} \pderivada{y} (0) = 0
\end{align*}
\pause
\setbeamercolor{item projected}{bg=black,fg=white}
\setbeamertemplate{enumerate items}{%
\usebeamercolor[bg]{item projected}%
\raisebox{1.5pt}{\colorbox{bg}{\color{fg}\footnotesize\insertenumlabel}}%
}
\begin{enumerate}[<+->]
\item Resuelve el problema por integración directa.
\item Calcula función de Green.
\item Resuelve el problema con CDF usando la función de Green.
\end{enumerate}
\end{frame}


\subsection{Enunciado 2}

\begin{frame}
\frametitle{Ejercicio 2}
%Ref. Riley 2006 - 17.7
Considera el conjunto de funciones, $\left\{ f (x) \right\}$, de variable real $x$, definida en el intervalo $-\infty < x < \infty$, que $\to 0$ al menos tan rápidamente como $x^{-1}$ cuando $x \to \pm \infty$.
\end{frame}
\begin{frame}
\frametitle{Ejercicio 2}
Con la función de peso unitaria, determina si cada uno de los siguientes operadores lineales es autoadjunto (Hermitiano) cuando actúa sobre $\left\{ f (x) \right\}$:
\end{frame}
\begin{frame}
\frametitle{Ejercicio 2}
\setbeamercolor{item projected}{bg=blue,fg=white}
\setbeamertemplate{enumerate items}{%
\usebeamercolor[bg]{item projected}%
\raisebox{1.5pt}{\colorbox{bg}{\color{fg}\footnotesize\insertenumlabel}}%
}
\begin{enumerate}[<+->]
\item $\displaystyle \dv{x} + x$
\item $\displaystyle - i \, \dv{x} + x^{2}$
\item $\displaystyle i \, x \, \dv{x}$
\item $\displaystyle i \, \dv[3]{x}$
\end{enumerate}
\end{frame}

\subsection{Enunciado 3}

\begin{frame}
\frametitle{Ejercicio 3}
Mediante la técnica de Gram-Schmidt genera los tres primeros polinomios de Laguerre con lo siguiente:
\begin{align*}
u_{n}(x) &= x^{n} \hspace{0.6cm} n = 0, 1, 2, \ldots, \\[0.5em]
0 &\leq x < \infty, \hspace{0.6cm} \omega(x) = e^{-x}
\end{align*}
La normalización convencional es:
\begin{align*}
\int_{0}^{\infty} L_{m}(x) \, L_{n}(x) \, e^{-x} \dd{x} = \delta_{mn}
\end{align*}
\end{frame}
\end{document}