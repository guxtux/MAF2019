\documentclass[12pt]{article}
\usepackage[utf8]{inputenc}
\usepackage[spanish,es-lcroman, es-tabla]{babel}
\usepackage[autostyle,spanish=mexican]{csquotes}
\usepackage{amsmath}
\usepackage{amssymb}
\usepackage{nccmath}
\numberwithin{equation}{section}
\usepackage{amsthm}
\usepackage{graphicx}
\usepackage{epstopdf}
\DeclareGraphicsExtensions{.pdf,.png,.jpg,.eps}
\usepackage{color}
\usepackage{float}
\usepackage{multicol}
\usepackage{enumerate}
\usepackage[shortlabels]{enumitem}
\usepackage{anyfontsize}
\usepackage{anysize}
\usepackage{array}
\usepackage{multirow}
\usepackage{enumitem}
\usepackage{cancel}
\usepackage{tikz}
\usepackage{circuitikz}
\usepackage{tikz-3dplot}
\usetikzlibrary{babel}
\usetikzlibrary{shapes}
\usepackage{bm}
\usepackage{mathtools}
\usepackage{esvect}
\usepackage{hyperref}
\usepackage{relsize}
\usepackage{siunitx}
\usepackage{physics}
%\usepackage{biblatex}
\usepackage{standalone}
\usepackage{mathrsfs}
\usepackage{bigints}
\usepackage{bookmark}
\spanishdecimal{.}

\setlist[enumerate]{itemsep=0mm}

\renewcommand{\baselinestretch}{1.5}

\let\oldbibliography\thebibliography

\renewcommand{\thebibliography}[1]{\oldbibliography{#1}

\setlength{\itemsep}{0pt}}
%\marginsize{1.5cm}{1.5cm}{2cm}{2cm}


\newtheorem{defi}{{\it Definición}}[section]
\newtheorem{teo}{{\it Teorema}}[section]
\newtheorem{ejemplo}{{\it Ejemplo}}[section]
\newtheorem{propiedad}{{\it Propiedad}}[section]
\newtheorem{lema}{{\it Lema}}[section]

\usepackage{standalone}
\usepackage{enumerate}
\usepackage{hyperref}
\usepackage[left=1.5cm,top=1.5cm,right=1.5cm,bottom=1.5cm]{geometry}
\title{Problemas para Examen Tarea 2 (Temas 3 y 4) \\ \large{Matemáticas Avanzadas de la Física}}
\date{ }
\addbibresource{Tarea_Referencia.bib}
\begin{document}
\vspace{-4cm}
%\renewcommand\theenumii{\arabic{theenumii.enumii}}
\renewcommand\labelenumii{\theenumi.{\arabic{enumii}}}
\maketitle
\fontsize{14}{14}\selectfont
\begin{enumerate}
\item Demuestra que se puede escribir
\[ \delta (x - \xi) = \dfrac{2}{L} \sum_{n=1}^{\infty} \sin \left( \dfrac{n \, \pi \, \xi}{L} \right) \, \sin \left( \dfrac{n \, \pi \, x}{L} \right) \hspace{1.5cm} 0 < \xi < L  \]
\item Una representación importante de la delta de Dirac se construye considerando el límite $n \to \infty$ de la secuencia
\[ \delta_{n} = \begin{cases}
c_{n} \, (1 - x^{2})^{n} & \mbox{ para } 0 \leq \abs{x} \leq 1, \hspace{0.5cm} n = 1, 2, 3, \ldots \\
0 & \mbox{ para } \abs{x} > 1
\end{cases} \]
Determina los coeficientes $c_{n}$ tales que 
\[ \int_{-1}^{1} \delta_{n} (x) \, \dd x = 1 \]
y demuestra que\[ \lim_{n \to \infty} \int_{-1}^{1} f(x) \, \delta_{n} (x) \, \dd x = f(0) \]
%Este ejercicio lo cambió Abraham
\item Dentro del contexto de la mecánica cuántica, demuestra que el momento $\vb{p}$, es un operador Hermitiano:
\[ \vb{p} = - i \, \hbar \, \nabla \equiv i \, \dfrac{h}{2 \, \pi} \, \nabla \]
\item Como en el punto anterior, demuestra ahora que el momento angular $\vb{L}$, es un operador Hermitiano:
\[ \vb{L} = - i\, \hbar \, \vb{r} \times \nabla \equiv i \, \dfrac{h}{2 \, \pi} \, \vb{r} \times \nabla \]

Para los problemas \ref{p1-Greiner}, \ref{p2-Greiner} y \ref{p3-Greiner}, te pedimos que consultes la referencia \cite[45]{Greiner_Electro}, en donde se explica parte de la solución, tendrás que detallar TODO el proceso, sin omitir pasos y explicando lo más posible cada uno de ellos. Estos ejercicios tienen el objetivo de guiar el uso del teorema de Green para la solución de problemas en electrodinámica.

\item \label{p1-Greiner} Construye la siguiente ecuación:
\[ \int_{V} [ \varphi(\vb{r^{\prime}}) \, \Delta^{\prime} \, \psi (\vb{r^{\prime}}) -  \psi (\vb{r^{\prime}}) \, \Delta^{\prime} \, \varphi (\vb{r^{\prime}})] \dd V^{\prime} = \oint_{S} \left[ \varphi (\vb{r^\prime}) \, \pdv{\psi (\vb{r^{\prime}})}{n^{\prime}} - \psi (\vb{r^{\prime}}) \pdv{\varphi (\vb{r^{\prime}})}{n^{\prime}} \right] \, \dd a^{\prime} \]
donde $\laplacian{} = \Delta$. La expresión anterior es la representación integral del potencial, sinedo una representación más general de los teoremas de Green y de la ED de Poisson. La ecuación de partida es
\[ \phi (\vb{r}) = \int_{V} \dfrac{\rho (\vb{r^{\prime}})}{\abs{\vb{r} - \vb{r^{\prime}}}} \, \dd V^{\prime} \]
\item \label{p2-Greiner} Resuelve el problema del potencial para un punto con carga cerca de una esfera aterrizada ($\Phi = 0$ en la superficie)
\item \label{p3-Greiner} Resuelve el problema de dos semiesferas conductoras a diferentes potenciales: la semiesfera superior a un potencial $+V$ y la semiesfera inferior a un potencial $-V$.
\item Con la técnica de ortogonalización de Gram-Schmidt genera los tres primeros polinomios de Laguerre, considerando:
\[ u_{n} (x) = x^{n}, \hspace{1cm} n = 0, 1, 2, \ldots \hspace{1cm} 0 \leq x < \infty, \hspace{1cm} w(x) = e^{-x} \]
La normalización es
\[ \int_{0}^{\infty} L_{m} (x) \: L_{n} (x) \: e^{-x} \, \dd x = \delta_{mn} \]
\item De nueva cuenta, con la técnica de Gram-Schmidt, genera los tres polinomios de Hermite de menor orden, considera:
\[ u_{n} (x) = x^{n}, \hspace{1cm} n = 0, 1, 2, \ldots \hspace{1cm} -\infty < x < \infty, \hspace{1cm} w(x) = e^{-x^{2}} \]
Para este conjunto de polinomios, la normalización es
\[ \int_{-\infty}^{\infty} H_{m} (x) \: H_{n} (x) \: w(x) \, \dd x = \delta_{mn} \, 2^{m} \, m! \, \pi^{1/2} \]
\item Definiendo los operadores
\[ a_{\pm} = \dfrac{1}{\sqrt{2 \, m}} \left[ \dfrac{\hbar}{i} \, \dv{}{x} \pm i \, m \, w \, x \right] \]
\begin{enumerate}[label=\roman*)]
\item Demuestra que para el oscilador armónico, la ecuación de Schrödinger puede escribirse como
\[ a_{-} \, a_{+} = \dfrac{1}{2 \, m} \left( \left[ \dfrac{\hbar}{i} \, \dv{}{x} \right]^{2} + (m \, w \, x)^{2} \right) + \dfrac{\hbar \, w}{2} \]
\item Demuestra que
\[ a_{-} \, a_{+} = \dfrac{1}{2 \, m} \left( \left[ \dfrac{\hbar}{i} \, \dv{}{x} \right]^{2} + (m \, w \, x)^{2} \right) - \dfrac{\hbar \, w}{2} \]
Representa también la ecuación de Schrödinger. Concluye que $[a_{-}, a_{+}] = \hbar \, w$. Estos operadores son conocidos como operadores de creación y aniquilación.
\item Demuestra que
\begin{align*}
\int_{-\infty}^{\infty} \abs{a_{+} \, \psi_{n}}^{2} \, \dd x &= (n+1) \, \hbar \, w \\[1em]
\int_{-\infty}^{\infty} \abs{a_{-} \, \psi_{n}}^{2} \, \dd x &= n \, \hbar \, w
\end{align*}
\end{enumerate}
\item El valor esperado de un operador $\hat{O}$ se define como
\[ \expval{\hat{O}}{\psi} = \int_{-\infty}^{\infty} \psi^{*} (x, t) \, \hat{O} \, \psi (x, t) , \dd x  \]
Calcula $\expval{x}$, $\expval{x^{2}}$, $\expval{p}$, $\expval{p^{2}}$ y $\expval{H}$ para el oscilador armónico cuántico en el estado base.
\item En una distribución tipo Maxwell la fracción de partículas moviéndose con velocidad $v$ y $v+dv$ es
\[ \dfrac{dN}{N} = 4 \pi \left( \dfrac{m}{2 \, \pi \, k \, T} \right)^{3/2} \: \exp \left( - \dfrac{m \, v^{2}}{2 \, k \, T} \right) \, v^{2} \, dv \]
donde $N$ es el número total de partículas. El promedio o valor esperado de $v^{n}$ se define como $\expval{v^{n}} = N^{-1} \int v^{n} \, \dd N$. Demostrar que
\[ \expval{v^{n}} = \left( \dfrac{2 \, k \, T}{m} \right)^{n/2} \: \dfrac{\Gamma (\frac{n +3}{2})}{\Gamma (3/2)}\]
\item Demostrar que
\[ \int_{0}^{\infty} e^{-x^{4}} \, \dd x = \left( \dfrac{1}{4} \right) !\]
\item Comprueba las siguientes identidades de la función Beta:
\begin{enumerate}[label=\roman*)]
\setlength\itemsep{1em}
\item $B(a, b) = B(a+1, b) + B(a, b+1)$
\item $B(a, b) = \dfrac{a+b}{b} \, B(a, b+1)$ 
\item $B(a, b) = \dfrac{b-1}{a} \, B(a+1, b-1)$
\item $B(a, b) \, B(a+b, c) = B(b, c) \, B(a, b+c)$
\end{enumerate}
\item Demostrar que
\[ \int_{-1}^{1} (1-x^{2})^{1/2} \, x^{2 \, n} \, \dd x =  
\begin{cases}
\pi/2 & n = 0 \\[1em]
\pi \dfrac{(2 \, n-1)!!}{(2 \, n+2)!!} & n=1,2,3,\ldots  \end{cases}
 \]
\item Demuestra que 
\[ \Gamma(\frac{1}{2} - n) \: \Gamma(\frac{1}{2} + n) = (-1)^{n} \: \pi \]
\end{enumerate}
\vfill
\printbibliography  
\end{document}