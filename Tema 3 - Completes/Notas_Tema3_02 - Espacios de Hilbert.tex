\documentclass[12pt]{article}
\usepackage[utf8]{inputenc}
\usepackage[spanish,es-lcroman, es-tabla]{babel}
\usepackage[autostyle,spanish=mexican]{csquotes}
\usepackage{amsmath}
\usepackage{amssymb}
\usepackage{nccmath}
\numberwithin{equation}{section}
\usepackage{amsthm}
\usepackage{graphicx}
\usepackage{epstopdf}
\DeclareGraphicsExtensions{.pdf,.png,.jpg,.eps}
\usepackage{color}
\usepackage{float}
\usepackage{multicol}
\usepackage{enumerate}
\usepackage[shortlabels]{enumitem}
\usepackage{anyfontsize}
\usepackage{anysize}
\usepackage{array}
\usepackage{multirow}
\usepackage{enumitem}
\usepackage{cancel}
\usepackage{tikz}
\usepackage{circuitikz}
\usepackage{tikz-3dplot}
\usetikzlibrary{babel}
\usetikzlibrary{shapes}
\usepackage{bm}
\usepackage{mathtools}
\usepackage{esvect}
\usepackage{hyperref}
\usepackage{relsize}
\usepackage{siunitx}
\usepackage{physics}
%\usepackage{biblatex}
\usepackage{standalone}
\usepackage{mathrsfs}
\usepackage{bigints}
\usepackage{bookmark}
\spanishdecimal{.}

\setlist[enumerate]{itemsep=0mm}

\renewcommand{\baselinestretch}{1.5}

\let\oldbibliography\thebibliography

\renewcommand{\thebibliography}[1]{\oldbibliography{#1}

\setlength{\itemsep}{0pt}}
%\marginsize{1.5cm}{1.5cm}{2cm}{2cm}


\newtheorem{defi}{{\it Definición}}[section]
\newtheorem{teo}{{\it Teorema}}[section]
\newtheorem{ejemplo}{{\it Ejemplo}}[section]
\newtheorem{propiedad}{{\it Propiedad}}[section]
\newtheorem{lema}{{\it Lema}}[section]

\usepackage{standalone}
\usepackage{geometry}
\geometry{top=1.25cm, bottom=1.5cm, left=1.25cm, right=1.25cm}
%\author{M. en C. Gustavo Contreras Mayén. \texttt{curso.fisica.comp@gmail.com}}
\title{Espacios de Hilbert y notación de Dirac \\ \large {Matemáticas Avanzadas de la Física}  \vspace{-1.5\baselineskip}}
\date{}
\begin{document}
%\renewcommand\theenumii{\arabic{theenumii.enumii}}
\renewcommand\labelenumii{\theenumi.{\arabic{enumii}}}
\maketitle
\fontsize{14}{14}\selectfont
\tableofcontents
\section{Espacios de Hilbert.}
\subsection{Espacio métrico.}
\begin{defi}
El par $(X, d)$ es un \textbf{espacio métrico} si $X$ es un conjunto y $d(x, y)$ es una función real valuada, llamada métrica definida para $x, y \in X$ que satisface las condiciones:
\begin{enumerate}
\item $d(x, y) \geq 0$ y $d(x, x) = 0, \hspace{0.5cm} \forall \: x, y \in X$
\item  Si $d(x, y) = 0$, entonces $x = y$.
\item $d(x, y) = d(y, x)$
\item $d(x, y) \leq d(x, z) + d(z, y)$
\end{enumerate}
\end{defi}
Una secuencia $\{x_{n}\}$ en un espacio métrico se llama \textbf{secuencia de Cauchy} si para cada $\varepsilon > 0$ existe un $N$ tal que $d(x_{n}, x_{m}) \leq \varepsilon$ para cualquier elección de $n, m \geq N$.
\par
Un espacio métrico $(X , d)$ es \textbf{completo} si cada secuencia de Cauchy en $(X , d)$ es convergente en $(X , d)$.
\begin{defi}Sea $X$ un espacio lineal.
    
Un producto interno sobre $X$ es un mapeo que asocia a cada par de vectores $x$, y un escalar, denotado $ \braket{x}{y}$ que satisface las siguientes propiedades
\begin{enumerate}[label=\roman*.]
\item $\braket{x}{z+y} = \braket{x}{y} + \braket{x}{y} \hspace{1cm} x, y, z \in X$
\item $\braket{x}{\lambda \: y} = \lambda \: \braket{x}{y} \hspace{1cm} \lambda \in K$
\item $\braket{x}{y} = \overline{\braket{x}{y}}$
\item $\braket{x}{y} \geq 0$ y $\braket{x}{y} = 0 \rightarrow x = \bm{0} $ 
\end{enumerate}
\end{defi}
Un \textbf{espacio con producto interno} se define como un espacio lineal junto con su producto interno definido sobre $X$.
\begin{defi}
Dos vectores $x$ e $y$ son ortogonales cuando $\braket{x}{y} = 0$.
\end{defi}
\begin{defi}
Una \textbf{norma} en $X$ es $\norm{\cdot} : X \rightarrow \mathbb{R}$ tal que  
\begin{enumerate}[label=\roman*.]
\item $\norm{x} \geq 0; \hspace{1cm} \norm{x} = 0$ si y sólo si $x = \bm{0}$
\item $\norm{\lambda \: x} =  \abs{\lambda} \, \norm{x}$
\item $\norm{x + y} \leq \norm{x} + \norm{y}$
\end{enumerate}
\end{defi}
 Un espacio lineal normado es un par $(X, \norm{\cdot})$ donde $X$ es un espacio lineal y $\norm{\cdot}$ es una norma sobre $X$.
 \par
 Un espacio con producto interno tiene una estructura natural de espacio normado con la norma $\norm{x} = \sqrt{\braket{x}}$ y un espacio normado tiene la estructura natural de espacio métrico con $d(x, y) = \norm{x - y}$.

Un \textbf{espacio de Hilbert} es un espacio con producto interno, con una norma definida por el producto interno y que además es completo.
\par
Ejemplos.

\begin{enumerate}
\item Sean $X = \mathbb{C}^{n}$ si $x = (x_{1}, x_{2}, \ldots, x_{n})$ e $y = (y_{1}, y_{2}, \ldots, y_{n})$, definamos
\[ \braket{x}{y} = \sum_{k=1}^{n} \overline{x_{k}} \: y_{k} \]
$X$ es un espacio de Hilbert.
\item Sea $L^{2}(\mathbb{R})$ el conjunto de todas las funciones medibles de $\mathbb{R}$ en $\mathbb{C}$ tales que verifiquen la siguiente propiedad:

Si $f(x) \in L2 L^{2}(\mathbb{R})$,entonces:
\[ \int_{-\infty}^{\infty} \abs{f(x)}^{2} dx < \infty \]
Así $L^{2}(\mathbb{R})$ es un espacio vectorial. Si lo dotamos del producto interno
\[ \braket{f}{g} = \int_{-\infty}^{\infty} \overline{f(x)} \: g(x) |, dx \]
y la norma asociada, entoces $L^{2}(\mathbb{R})$ es un espacio de Hilbert.
\item Consideremos el conjunto de sucesiones de números complejos $(a_{n})_{n \in \mathbb{C}}$, tales que:
\[ \sum_{n=1}^{\infty} \abs{a_{n}}^{2} < \infty \]
con el producto escalar
\[ \braket{A}{B} = \sum_{n=1}^{\infty} \overline{a_{n}} \: b_{n} \]
donde $A = (a_{n})$ y $B = (b_{n})$, este conjunto es un espacio de Hilbert llamado $\ell_{2}$.
\end{enumerate}
\begin{defi} Conjunto ortonormal.

Sea $S$ un conjunto de vectores en un espacio con producto interno $X$. Entonces $S$ es un sistema o conjunto ortonormal si:
\begin{enumerate}[label=\roman*.]
\item  $\norm{x} = 1, \hspace{0.5cm} \forall \, x \in S$
\item $\braket{x}{y} = 0 \hspace{0.5cm} \forall \, x, y \in S$ con $x \neq y$
\end{enumerate}
\end{defi}
\begin{teo} Desigualdades de Bessel.
    
Sea $A$ un conjunto ortonormal de vectores contenido en un espacio con producto interno $X$. Sean $x, y \in X$. Sea ${x_{1}, x_{2}, \ldots, x_{n}, \ldots}$ una sucesión de elementos de $A$ (finita o infinita). Entonces
\begin{enumerate}[label=\roman*.]
\item $\displaystyle{\sum_{n=1}^{\infty} \abs{\braket{x}{x_{n}}}^{2} \leq \norm{x}^{2}}$
\item $\displaystyle{\sum_{n=1}^{\infty} \abs{\braket{x}{x_{n}} \, \braket{x_{n}}{y}} \leq \norm{x} \, \norm{y}}$
\end{enumerate}
\end{teo}
\begin{defi} Conjunto completo.

Un conjunto ortonormal $A \in X$ se llama \textbf{completo} si y sólo si no existe en $X$ otro conjunto ortonormal conteniendo estrictamente en $A$.

A este \textit{conjunto ortonormal completo} se le llama \textbf{base ortonormal}.
\end{defi}
\begin{defi} Espacio de Hilbert separable.

Un espacio de Hilbert es \textbf{separable} si y sólo si el cardinal de sus conjuntos ortonormales completos es finito o numerable.
\end{defi}
\begin{teo}
Sea $X$ un espacio de Hilbert de dimensión finita o infinita. Sea ${x_{1}, x_{2}, \ldots}$ un conjunto ortonormal completo en $X$. Entonces
\[ \norm{x}^{2} = \sum_{k=1}^{\infty} \abs{\braket{x_{k}}{x}} \]
\end{teo}
\begin{teo}
 Sea $A$ un conjunto ortonormal completo en un espacio de Hilbert separable y de dimensión infinita $X$. Entonces:
 \begin{enumerate}[label=\roman*.]
\item $\overline{\bqty{A}} = X$
\item Si $y \in X$ entonces $\displaystyle{y = \sum_{n=1}^{\infty} \, \braket{x_{n}}{y} \, x_{n}}$
 \end{enumerate}
\end{teo}
Ejemplos de conjuntos ortonormales completos:
\begin{enumerate}
\item Sea $X = \mathbb{C}^{n}$
\[ (1, 0, 0, \ldots, 0), \hspace{0.5cm} (0, 1, 0, \ldots, 0), \hspace{0.5cm} \ldots, (0, 0, 0, \ldots, 1) \]
\item Sea $X = \ell_{2}$
\begin{align*}
e_{1} &=  (1, 0, 0, \ldots, 0, \ldots) \\
e_{2} &=  (0, 1, 0, \ldots, 0, \ldots) \\
e_{3} &=  (0, 0, 1, \ldots, 0, \ldots) \\
&{} \vdots
\end{align*}
\item En $L^{2}(0, 2 \, \pi)$, el conjunto
\[ x_{0}(t) = \dfrac{1}{\sqrt{2 \, \pi}}, \hspace{0.5cm} x_{1}(t) = \dfrac{\cos t}{\sqrt{\pi}}, \hspace{0.5cm} x_{2}(t) = \dfrac{\sin t}{\sqrt{\pi}}, \hspace{0.5cm} x_{3}(t) = \dfrac{\cos 2 \, t}{\sqrt{\pi}}, \ldots\]
\end{enumerate}
\begin{teo}
Sea $X$ un espacio de Hilbert de dimensión finita $n$. Entonces $X$ y $\mathbb{C}^{n}$ son isométricos.
\end{teo}
\begin{teo}
Sea $X$ un espacio de Hilbert separable y de dimensión infinita. Entonces $X$ y $\ell_{2}$ son isométricos.
\end{teo}
\subsection{Espacio dual.}
\begin{defi} Funcional.

Sea $f$ una aplicación lineal de un espacio normado $X$ en $\mathbb{C}$. Si $f$ es además continuo, lo llamaremos \textit{funcional lineal continuo}, o simplemente \textit{funcional}.
\end{defi}
\begin{defi} Espacio dual.

Sea $f$ una aplicación lineal de $X \in \mathbb{C}$. Diremos que $f$ está acotada si existe una constante positiva $K$ tal que
\[ \abs{f(x)} \leq K \, \norm{x}, \hspace{1cm} \forall \, x \in X \]
El conjunto de todas las funcionales acotadas en $X$ forman un espacio vectorial llamado \textbf{espacio dual} de $X$, denotado por $X^{*}$.

Además es normado y completo con la norma
\[ \norm{f} = \inf \{ K \vert (\forall \, x \in X) \vert f(x) \vert = K \, \norm{x} \} \]
\end{defi}
\begin{teo} Teorema de representación de Riesz.

Sea $f$ un funcional acotado en un espacio de Hilbert $X$. Entonces existe uno y sólo un vector $y \in X$ tal que
\[ f(x) = \braket{y}{x}, \hspace{1cm} \forall \, x \in X \]
\end{teo}
\subsection{Operadores lineales en espacios de Hilbert.}
\begin{defi} Operador lineal.

Un operador lineal entre dos espacios normados $X$ e $Y$ es una aplicación lineal entre ambos.

En particular puede ser una aplicación de un espacio de Hilbert consigo mismo.
\end{defi}
\begin{defi} Operador lineal adjunto.

Sea $A$ un operador lineal continuo en un espacio de Hilbert $X$.
Fijemos $x \in X$ y consideremos el siguiente producto escalar $\braket{x}{A \, y}$ para todo $y \in X$.

La aplicación $f(y) = \braket{x}{A \, y}$ es un funcional lineal y continuo en $X$, el teorema de Riez nos asegura que existe, fijado $x$, un único $z \in X$ tal que
\[ f(y) = \braket{z}{y} = \braket{x}{A \, y} \hspace{0.5cm} \forall y \in X \]

Esto lo podemos hacer con cada uno de los $x \in X$, con lo que obtenemos una aplicación de $X$ en si mismo: $A^{\dagger}$, que llamaremos \textbf{operador adjunto} de $A$, y tal que $A^{\dagger} \, x = z$. De esta manera
\[ \braket{x}{A \, y} = \braket{A^{\dagger} \, x}{y}, \hspace{1cm} \forall \, x, y \in X \]
\end{defi}
\subsection*{Propiedades de operadores adjuntos.}
Sean $A$ y $B$ dos operadores acotados en un espacio de Hilbert $X$, y sea $\lambda$ un número complejo. Entonces
\begin{align*}
(A + B)^{\dagger} &= A^{\dagger} + B^{\dagger} \\
(\lambda \: A)^{\dagger} &= \overline{\lambda} \: A^{\dagger} \\
(A \: B)^{\dagger} &= B^{\dagger} \: A^{\dagger}
\end{align*}
\begin{defi} Operador autoadjunto.

Sea $X$ un espacio de Hilbert y $T$ un operador lineal continuo, se dice que $T$ es un operador \textbf{autoadjunto} si $T = T^{\dagger}$.
\end{defi}
\textbf{Ejemplo:}

Sea el espacio de Hilbert $L^{2} ([0, 1])$. Sea ahora la transformación $Q : L^{2} ([0, 1]) \rightarrow L^{2}([0, 1])$ definida de la siguiente manera
\[ (Q \: f)(x) =  x \: f(x) \hspace{1cm} \forall \, f \in L^{2}([0, 1]) \]
Esta transformación está bien definida ya que para toda $f(x) \in L^{2} ([0, 1])$ tenemos:
\[ \int_{0}^{1} \abs{(Q \: f)(x)}^{2} \, dx = \int_{0}^{1} \abs{x \: f(x)}^{2} \leq \int_{0}^{1} \abs{f(x)}^{2} \, dx < \infty \]
y por tanto $(Q \: f)(x) \in L^{2}([0, 1])$. Además es lineal.
\par
Para demostrar que $Q$ es acotado, es decir, es continuo
\[ \norm{Q \: f}^{2} = \int_{0}^{1} \abs{x \: f(x)}^{2} \, dx   \leq \int_{0}^{1} \abs{f(x)}^{2} \, dx = \norm{f}^{2}, \hspace{1cm} \Rightarrow \norm{Q \: f} \leq \norm{f} \]
lo cual nos dice que $Q$ es un operador acotado y además su norma es menor o igual a $1$.

Notemos además que para todos $f, g \in L^{2} ([0, 1])$, se tiene:
\[ \braket{g}{Q \: f} = \int_{0}^{1} \overline{g(x)} \: x \: f(x) \, dx = \int_{0}^{1} \overline{x \: g(x)} \: f(x) \, dx = \braket{Q \: g}{f} \]
es decir, $Q$ es autoadjunto.
\begin{defi} Se $A$ un operador lineal continuo.
\begin{itemize}
\item $A$ es normal si $A^{\dagger} \, A = A \, A^{\dagger}$
\item $A$ es autoadjunto si $A^{\dagger} =  A$
\item $A$ es isométrico si $\norm{A \: x} =  \norm{x}, \hspace{0.5cm} \forall x \in X$
\end{itemize}
\end{defi}
\textbf{Proposición} : Sea $U$ un operador acotado en el espacio de Hilbert $X$.
\begin{itemize}
\item Si $U$ es invertible y $U^{-1} = U^{\dagger}$, entonces $U$ es unitario.
\item $U$ es unitario si y sólo si $I = U^{\dagger} \: U = U \: U^{\dagger}$ 
\end{itemize}

\begin{defi} Proyectores.

Sea $P$ un operador continuo en el espacio de Hilbert $X$. $P$ es un proyector si $P^{2} = P$ y $P^{\dagger} =  P$.

El operador nulo y la identidad son proyectores.
\end{defi}
\textbf{Ejemplo} :

Sea $M$ un subespacio cerrado de $X$ que no coincida con $X$. 

Se sabe que $X =  M \oplus M^{\perp}$, siendo $M^{\perp} \neq \{ \bm{0} \}$. Para todo $x \in X$ existe una única descomposición $x = y + z$ con $y \in M$ y $z \in M^{\perp}$.

Definamos $P \: x = y$. $P$ es un proyector.
\begin{defi} Valor propio y vector propio.

Sea $A$ un operador con dominio $\mathcal{D} \in X$, siendo $X$ un espacio de Hilbert. Diremos que $y \in \mathcal{D}, \: y \neq 0$, es un \textbf{eigenvector (valor propio)} de $A$ si existe un número complejo $\lambda$ tal que $A \: y = \lambda \: y$ ($\lambda$ si puede ser cero)

Entonces $y$ es un \textbf{eigenvector (vector propio)} de $A$ con
eigenvalor (valor propio) $\lambda$.
\end{defi}
\begin{itemize}[label=\checkmark]
\item Diremos que $\lambda \in \mathbb{C}$ pertenece al \textbf{espectro discreto} de $A$ si es un valor propio de $A$.
\item Diremos que $\lambda \in \mathbb{C}$ pertenece al \textbf{espectro residual} de $A$ si no es un valor propio de $A$ y además el rango del operador $A - \lambda \: I$ no es denso en $X$.
\item $\lambda \in \mathbb{C}$ pertenece al \textbf{espectro continuo} de $A$ si no es un valor propio ni tampoco está en el residual de $A$, y la inversa de la aplicación $A - \lambda \: I$ no es continua. Si es continua $\lambda$ pertenece al \textbf{resolvente}.
\end{itemize}
\begin{lema}
Sea $A$ un operador autoadjunto y acotado en un espacio de Hilbert. Entonces sus eigenvalores son reales.
\end{lema}
\begin{lema}
Sea $A$ un operador autoadjunto y acotado en un espacio de Hilbert $X$. Sean $x$ e $y$ dos valores propios de $A$ con diferente valor. Entonces $x$ e $y$ son ortogonales.
\end{lema}
\begin{lema}
Si $U$ es unitario, sus eigenvalores (si existen) tienen módulo uno.
\end{lema}
En el caso de operadores no acotados, la definición de \textbf{operador adjunto} debe de tomar en cuenta los dominios.
\par
Sea $A$ un operador con dominio $\mathcal{D}$ denso en un espacio de Hilbert (de dimensión infinita y separable) $X$ Para definir el adjunto de $A$, comencemos por definir su dominio:
\[ \mathcal{D}^{*} = \{ y \in X \vert \mbox{ existe } z \in X \mbox{ tal que } \braket{z}{x} = \braket{y}{A \, x}, \forall \, x \in \mathcal{D} \} \]
De este modo, el adjunto de $A$ es
\[ A^{\dagger} \: y = z, \hspace{1cm} \forall \: y \in \mathcal{D^{*}} \]
\subsection*{Propiedades de los operadores adjuntos.}
\begin{itemize}
\item $A^{\dagger}$ es lineal.
\item $A^{\dagger}$ es siempre un operador cerrado.
\item Si $\alpha \in \mathbb{C}$ entonces $(\alpha \: A)^{\dagger} =  \overline{\alpha} \: A^{\dagger}$
\item Sean $A$ y $B$ dos operadores en $X$ con dominios $\mathcal{D}(A)$ y $\mathcal{D}(B)$. Diremos que $B$ extiende a $A$, y escribiremos $B \succ A$, si $\mathcal{D}(A) \subset \mathcal{D}(B)$ y además para todo $z \in \mathcal{D}(A)$ se verifica que $A \: z = B \: z$. La siguiente propiedad dice que
\[ \mbox{Si } B \succ  A \Rightarrow  A^{\dagger} \succ  B^{\dagger} \]
\item Si definimos 
\[ \mathcal{D} (A + B) = \mathcal{D}(A) \cap \mathcal{D}(B)\]
y
\[ \mathcal{D} (A^{\dagger} + B^{\dagger}) = \mathcal{D}(A^{\dagger}) \cap \mathcal{D}(B^{\dagger}) \]
entonces
\[ A^{\dagger} + B^{\dagger} \prec (A + B)^{\dagger}  \]
\item Si definimos $\mathcal{D}(A \, B)$
\item Si definimos $\mathcal{D} (A \, B) = \{ y \in X \vert y \in \mathcal{D}(B) \mbox{ tal que } B \: y \in \mathcal{D}(A) \}$ y si $\mathcal{D}(A \: B)$ es denso en $X$,
\[ B^{\dagger} \: A^{\dagger} \prec (A \: B)^{\dagger} \]
\end{itemize}
\begin{defi}
Un operador $A$ es llamado \textbf{simétrico (o Hermítico)} si $A \succ A^{\dagger}$, es decir
\[ \braket{y}{A \: x} = \braket{A \: y}{x}, \hspace{1cm} \forall \, x, y \in \mathcal{D}(A) \]
Un operador simétrico es autoadjunto si $A = A^{\dagger}$
\end{defi}
\textbf{Ejemplo de operadores no acotados.}
\begin{ejemplo}
Sea $X = L^{2}(\mathbb{R})$. Consideremos
\[ \mathcal{D} = \{ \phi (x) \in L^{2}(\mathcal{R}) \vert x \: \phi (x) \in X \} \]
$\mathcal{D}$ es un espacio vectorial, además se puede probar que es denso en $X$.
\par
Consideremos el operador $Q: \mathcal{D} \mapsto X$
\[ Q \: \phi(x) =  x \: \phi (x) \]
Se conoce a esta transformación como el \underline{operador de posición}. Este operador es autoadjunto en el dominio señalado.
\end{ejemplo}
\begin{ejemplo}
Definamos ahora el espacio de Schwart, $S$ como el conjunto de funciones $f(x)$ infinitamente diferenciables en todos los puntos de $\mathbb{R}$ y tales que
\[ \lim_{x \to \pm \infty} x^{n} \: \dv[m]{}{x} f(x) = 0, \hspace{1cm} \forall \: n, m = 0, 1, 2, \ldots \]
En este subespacio $Q$ y el operador
\[ P \: \psi (x) = - i \, \hbar \: \dv{x} \psi (x) \]
son \textbf{esencialmente autoadjuntos}, es decir, son operadores simétricos y su cerradura es autoadjunta.
\end{ejemplo}
\textbf{Proposición}: En $S$
\[ [Q, P] = Q \: P - P \: Q \prec i \: \hbar \: I \]
\begin{defi}
Sea $A$ un operador y sea $x \in \mathcal{D}(A) \cup \mathcal{D}(A^{2})$. Llamaremos \underline{dispersión} de $A$ en $x$ al número
\[ \Delta_{x} (A) = \sqrt{\braket{x}{(A - \braket{x}{A \: x} \: I)^{2} \: x}} = \sqrt{\braket{x}{A^{2} \: x} - (\braket{x}{A \: x})^{2}} \]
\end{defi}
\textbf{Principio de Heinserberg:} Sea $f \in S$. Entonces:
\[ (\Delta_{f} \: Q)(\Delta_{f} \: P) \geq \dfrac{\hbar}{2} \]
\section{Notación de Dirac.}
\end{document}