\documentclass[12pt]{beamer}
\usepackage{../Estilos/BeamerMAF}
\input{../Preambulos/preambulo_Beamer_Dresden_seahorse}

\date{12 de noviembre de 2021}

\title{\large{Ejercicios bases completas y ortogonales}}
\author{M. en C. Gustavo Contreras Mayén}

\begin{document}
\maketitle
\fontsize{14}{14}\selectfont
\spanishdecimal{.}

\section*{Contenido}
\frame{\tableofcontents[currentsection, hideallsubsections]}

\section{Bases completas}
\frame{\tableofcontents[currentsection, hideothersubsections]}
\subsection{Expansión en serie de funciones propias}

\begin{frame}
\frametitle{Punto de partida}
Sabemos que una función $f(x)$ se puede desarrollar en términos de una serie de funciones propias ortonormales:
\pause
\begin{align*}
f(x) = \nsum_{n=0}^{\infty} a_{n} \, \varphi_{n}
\end{align*}
\end{frame}
\begin{frame}
\frametitle{Ejercicio 1}
Vamos a demostrar que \emph{el desarrollo en series es único para un conjunto dado de funciones ortonormales} $\varphi_{n}(x)$.
\\
\bigskip
\pause
Las funciones $\varphi_{n}(x)$ como se consideran en este ejercicio son los \textcolor{blue}{vectores base} en un espacio de Hilbert de dimensión infinita.
\end{frame}
\begin{frame}
\frametitle{Resolviendo el ejercicio}
Suponemos que existen $b_{n}$, tales que:
\pause
\begin{align*}
f(x) = \nsum_{n=0}^{\infty} b_{n} \, \varphi_{n} (x)
\end{align*}
\end{frame}
\begin{frame}
\frametitle{Multiplicando e integrando}
Multiplicamos por $\varphi_{n}^{*} (x) \, \omega(x)$ para luego integrar en el intervalo $[a, b]$:
\pause
\begin{eqnarray*}
\begin{aligned}
\scaleint{6ex}_{\bs a}^{b} \varphi_{n}^{*} \, \omega \, f  \dd{x} &= \nsum_{n=0}^{\infty} b_{n} \scaleint{6ex}_{\bs a}^{b} \varphi_{n}^{*} \, \varphi_{n} \, \omega \dd{x} \\[0.5em] \pause
&= \nsum_{n=0}^{\infty} b_{n} \, \delta_{nm} \hspace{1cm} \mbox{por la ortonormalidad} \\[0.5em] \pause
&= b_{n}
\end{aligned}
\end{eqnarray*}
\end{frame}
\begin{frame}
\frametitle{Resultado obtenido}
Este resultado implica:
\pause
\begin{align*}
b_{n} = \scaleint{6ex}_{\bs a}^{b} \varphi_{n}^{*} \, \omega \, f  \dd{x}
\end{align*}
\pause
Esta expresión es la misma que para los $a_{n}$.
\\
\bigskip
\pause
Concluimos que $a_{n} = b_{n}, \forall n$, \pause por lo que la expansión de la función en una serie de funciones propias ortonormales, \textbf{es única}.
\end{frame}

\subsection{Una base completa y ortogonal}
%Ref. Zettili Problem 2.2
\begin{frame}
\frametitle{Enunciado del Ejercicio 2}
\setbeamercolor{item projected}{bg=blue!70!black,fg=yellow}
\setbeamertemplate{enumerate items}[circle]
\begin{enumerate}[<+->]
\item Encuentra una base completa y ortonormal para el espacio de las funciones trigonométricas de la forma:
\pause
\begin{align*}
\psi(\theta) = \nsum_{n=0}^{N} a_{n} \, \cos (n \, \theta)
\end{align*}
\item Calcula los vectores base con $N = 5$.
\end{enumerate}
\end{frame}
\begin{frame}
\frametitle{Aprovechando identidades}
Para resolver el primer inciso, ocupemos una identidad conocida para expresar $\psi(\theta)$:
\pause
\begin{eqnarray*}
\begin{aligned}
\psi(\theta) &= \nsum_{n=0}^{N} a_{n} \, \cos (n \, \theta) = \\[0.5em] \pause
&= \dfrac{1}{2} \nsum_{n=0}^{N} a_{n} \bigg[ e^{i n \theta} + e^{-i n \theta}\bigg] = \\[0.5em] \pause
&= \dfrac{1}{2} \bigg[ \nsum_{n=0}^{N} a_{n} \, e^{i n \theta} + \nsum_{n=-N}^{0} a_{-n} \, e^{i n \theta} \bigg] = 
\end{aligned}
\end{eqnarray*}
\end{frame}
\begin{frame}
\frametitle{Simplificando la expresión}
\begin{eqnarray*}
\begin{aligned}
= \nsum_{n=-N}^{N} C_{n} \, e^{i n \theta}
\end{aligned}
\end{eqnarray*}
\pause
donde:
\begin{align*}
C_{n} &= \dfrac{a_{n}}{2} \mbox{ para } n > 0 \\[0.5em]
C_{n} &= \dfrac{a_{-n}}{2} \mbox{ para } n < 0 \\[0.5em]
C_{0} &= a_{0}
\end{align*}
\end{frame}
\begin{frame}
\frametitle{Apoyo con un resultado}
Ya que cualquier función trigonométrica de la forma:
\begin{align*}
\psi(\theta) = \nsum_{n=0}^{N} a_{n} \, \cos (n \, \theta) 
\end{align*}
\pause
Puede expresarse en términos de las funciones:
\pause
\begin{align*}
\phi_{n}(\theta) = \dfrac{e^{i n \theta}}{\sqrt{2}}
\end{align*}
\pause
intentemos usar este conjunto de funciones $\phi_{n}(\theta)$ como una base.
\end{frame}
\begin{frame}
\frametitle{Propiedades de la base propuesta}
Como el conjunto de las $\phi_{n}(\theta)$ es completo, \pause veamos si es ortonormal.
\\
\bigskip
\pause
Evaluemos el producto escalar $\ip{\phi_{m}}{\phi_{n}}$.
\end{frame}
\begin{frame}
\frametitle{Calculando el producto escalar}
\begin{eqnarray*}
\begin{aligned}
\ip{\phi_{m}}{\phi_{n}} &= \scaleint{6ex}_{\bs - \pi}^{\pi} \phi_{m}^{*} (\theta) \, \phi_{n} (\theta) \dd{\theta} = \\[0.5em] \pause
&= \dfrac{1}{2 \, \pi} \scaleint{6ex}_{\bs - \pi}^{\pi} \exp(i (n -m )\theta) \dd{\theta} &= \\[0.5em] \pause
&= \delta_{nm}
\end{aligned}
\end{eqnarray*}
\end{frame}
\begin{frame}
\frametitle{Revisando los casos}
Llegando a este resultado, debemos de considerar dos casos:
\setbeamercolor{item projected}{bg=blue!70!black,fg=yellow}
\setbeamertemplate{enumerate items}[circle]
\begin{enumerate}[<+->]
\item $n = m$
\item $n \neq m$
\end{enumerate}
\end{frame}
\begin{frame}
\frametitle{Caso cuando $n = m$}
En el primer caso: $n = m$, es obvio que:
\pause
\begin{eqnarray*}
\begin{aligned}
\ip{\phi_{m}}{\phi_{n}} &= \dfrac{1}{2 \, \pi} \scaleint{6ex}_{\bs - \pi}^{\pi} \dd{\theta} = \\[0.5em] \pause
&= 1
\end{aligned}
\end{eqnarray*}
\end{frame}
\begin{frame}
\frametitle{Caso cuando $n \neq m$}
En el segundo caso: $n \neq m$, se tiene que:
\pause
\begin{eqnarray*}
\begin{aligned}
\ip{\phi_{m}}{\phi_{n}} &= \dfrac{1}{2 \, \pi} \scaleint{6ex}_{\bs - \pi}^{\pi} \exp(i(n - m) \theta) \dd{\theta} = \\[0.5em] \pause
&= \dfrac{1}{2 \, \pi} \, \dfrac{\exp(i(n - m) \pi) - \exp(-i(n - m) \pi)}{i (n - m)} = \\[0.5em] \pause
&= \dfrac{2 \, i \, \sin \big( \big[ n - m \big] \, \pi \big) }{2 \, i \, \pi \, (n - m)} = \\[0.5em] \pause
&= 0
\end{aligned}
\end{eqnarray*}
\end{frame}
\begin{frame}
\frametitle{Conclusión}
Por lo que el conjunto de funciones:
\pause
\begin{align*}
\phi_{n}(\theta) = \dfrac{e^{i n \theta}}{\sqrt{2}}
\end{align*}
forman una base completa y ortonormal.
\end{frame}
\begin{frame}
\frametitle{Información adicional}
Del resultado:
\begin{align*}
\nsum_{n=-N}^{N} C_{n} \, e^{i n \theta}
\end{align*}
\pause
vemos que la base tiene $2 \, N + 1$ funciones $\phi_{n}(\theta)$, \pause por lo que la dimensión de este espacio de funciones es igual a $2 \, N + 1$.
\end{frame}
\begin{frame}
\frametitle{Segundo inciso}
Una vez que ya tenemos la expresión para la base ortonormal, podemos calcular los vectores base.
\pause
\\
\bigskip
Con $N = 5$, se tiene que la dimensión del espacio es igual a $11$, \pause por lo que la base tiene $11$ vectores.
\end{frame}
\begin{frame}
\frametitle{Vectores base}
\begin{eqnarray*}
\begin{aligned}
\phi_{-5} (\theta) &= \exp(- 5 \, i \, \theta) / \sqrt{2 \, \pi} \\
\phi_{-4} (\theta) &= \exp(- 4 \, i \, \theta) / \sqrt{2 \, \pi} \\
\vdots \\
\phi_{0} (\theta) &= 1 / \sqrt{2 \, \pi} \\
\vdots \\
\phi_{4} (\theta) &= \exp(4 \, i \, \theta) / \sqrt{2 \, \pi} \\
\phi_{5} (\theta) &= \exp(5 \, i \, \theta) / \sqrt{2 \, \pi}
\end{aligned}
\end{eqnarray*}
\end{frame}

\section{Valores propios y operadores}
\frame{\tableofcontents[currentsection, hideothersubsections]}
\subsection{Valores y funciones propias}
%Ref. Zettili Problem 2.18
\begin{frame}
\frametitle{Enunciado del ejercicio 3}
\setbeamercolor{item projected}{bg=blue!70!black,fg=yellow}
\setbeamertemplate{enumerate items}[circle]
\begin{enumerate}[<+->]
\item Calcula los valores propios así como las funciones propias del operador $\hat{A} = - \dv*[2]{x}$.
\\
Limita la búsqueda de las funciones propias a aquellas funciones complejas que se anulan en todas partes excepto en la región $0 < x < a$.
\item Normaliza la función propia y encuentra la probabilidad en la región $0 < x  < \dfrac{a}{2}$.
\end{enumerate}
\end{frame}
\begin{frame}
\frametitle{Resolviendo el inciso $\mathbf{1)}$}
El problema de valores propios para $- \dv*[2]{x}$ consiste en resolver la siguiente ecuación diferencial:
\pause
\begin{align*}
- \dv[2]{\psi}{x} = \alpha \, \psi (x)
\end{align*}
\pause
y encontrar los valores propios $\alpha$ y las funciones propias $\psi(x)$.
\end{frame}
\begin{frame}
\frametitle{Solución general a la EDO}
Las solución más general para esta EDO es:
\pause
\begin{align*}
\psi(x) = A \, \exp (i \, b \, x ) + B \, \exp (-i \, b \, x )
\end{align*}
con $\alpha = b^{2}$.
\end{frame}
\begin{frame}
\frametitle{Usando las CDF}
Ocupando las condiciones de frontera de $\psi(x)$ en $x = 0$ y en $x = a$, se tiene que:
\pause
\begin{eqnarray*}
\begin{aligned}
\psi (0) &= A + B = 0 \hspace{0.3cm} \Rightarrow \hspace{0.3cm} B = - A \\[0.5em] \pause
\psi (a) &= A \, \exp (i b a) + B \, \exp(- i b a) = 0
\end{aligned}
\end{eqnarray*}
\end{frame}
\begin{frame}
\frametitle{Usando los resultados}
Al sustituir $B = - A$ en la segunda ecuación, nos lleva a:
\pause
\begin{eqnarray*}
\begin{aligned}
\psi (a) &= A \big[ \exp (i \, b \, a) - \exp(- i \, b \, a) \big] = 0 \\[0.5em] \pause
\Rightarrow  \exp (i \, b \, a) &= \exp(- i \, b \, a) \big) \\[0.5em] \pause
\Rightarrow  \exp (2 \, i \, b \, a) &= 1
\end{aligned}
\end{eqnarray*}
\end{frame}
\begin{frame}
\frametitle{Calculando los valores propios}
Por lo que:
\pause
\begin{eqnarray*}
\begin{aligned}
\sin ( 2 \, b \, a) &= 0 \\[0.5em] \pause
\cos ( 2 \, b \, a) &= 1 \\[0.5em] \pause
\Rightarrow b \, a &= n \, \pi
\end{aligned}
\end{eqnarray*}
\end{frame}
\begin{frame}
\frametitle{Valores y funciones propias}
Los valores propios son:
\pause
\begin{align*}
\alpha_{n} = \dfrac{n^{2} \, \pi^{2}}{a^{2}}
\end{align*}
\pause
Y las funciones propias son:
\pause
\begin{align*}
\psi_{n}(x) = A \, \bigg[ \exp \bigg(\dfrac{i \, n \, \pi \, x}{a} \bigg) - \exp \bigg(- \dfrac{i \, n \, \pi \, x}{a} \bigg) \bigg]
\end{align*}
\end{frame}
\begin{frame}
\frametitle{Valores y funciones propias}
Es decir que los valores propios y las funciones propias son:
\pause
\begin{align*}
\setlength{\fboxsep}{3\fboxsep}\boxed{
\alpha_{n} = \dfrac{n^{2} \, \pi^{2}}{a^{2}} \hspace{1cm} \psi_{n}(x) = C_{n} \, \sin \bigg(\dfrac{\, n \, \pi \, x}{a} \bigg) }
\end{align*}
\pause
Por lo que el espectro de valores propios del operador $\hat{A} = - \dv*[2]{x}$ \textbf{es discreto}, ya que los valores propios y las funciones propias dependen de un número discreto $n$.
\end{frame}
\begin{frame}
\frametitle{Resolviendo el inciso $\mathbf{2)}$}
La normalización de las funciones propias $\psi_{n}(x)$ será tal que:
\pause
\begin{eqnarray*}
\begin{aligned}
1 &= C_{n}^{2} \, \scaleint{6ex}_{\bs 0}^{a} \sin^{2} \bigg(\dfrac{n \, \pi \, x}{a} \bigg) \dd{x} = \\[0.5em] \pause
&= \dfrac{C_{n}^{2}}{2} \scaleint{6ex}_{\bs 0}^{a} \bigg[ 1 - \cos \bigg(\dfrac{2 \, n \, \pi \, x}{a} \bigg) \bigg] \dd{x} = \\[0.5em] \pause
&= \dfrac{C_{n}^{2}}{2} \, a
\end{aligned}
\end{eqnarray*}
\end{frame}
\begin{frame}
\frametitle{Normalizando las funciones propias}
Los que nos lleva a:
\pause
\begin{align*}
C_{n} = \sqrt{\dfrac{2}{a}}
\end{align*}
\pause
De aquí se obtienen las funciones propias normalizadas:
\pause
\begin{align*}
\setlength{\fboxsep}{3\fboxsep}\boxed{
\psi_{n}(x) = \sqrt{\dfrac{2}{a}} \, \sin \bigg(\dfrac{n \, \pi \, x}{a} \bigg) }
\end{align*}
\end{frame}
\begin{frame}
\frametitle{El valor de probabilidad}
La probabilidad en la región $0 < x < \dfrac{a}{2}$ estará dada por:
\pause
\begin{eqnarray*}
\begin{aligned}
\dfrac{2}{a} \scaleint{6ex}_{\bs 0}^{\frac{a}{2}} &\sin^{2} \bigg(\dfrac{n \pi x}{a} \bigg) \dd{x} = \pause \dfrac{1}{a} \scaleint{6ex}_{\bs 0}^{\frac{a}{2}} \bigg[ 1 {-} \cos \bigg(\dfrac{2 n \pi x}{a} \bigg) \bigg] \dd{x} = \pause \dfrac{1}{2}
\end{aligned}
\end{eqnarray*}
\pause
Este se esperaba, ya que la probabilidad total es $1$:
\begin{align*}
\scaleint{6ex}_{0}^{a} \abs{\psi_{n} (x)}^{2} \dd{x} = 1
\end{align*}
\end{frame}
\end{document}