\documentclass[hidelinks,12pt]{article}
\usepackage[left=0.25cm,top=1cm,right=0.25cm,bottom=1cm]{geometry}
%\usepackage[landscape]{geometry}
\textwidth = 20cm
\hoffset = -1cm
\usepackage[utf8]{inputenc}
\usepackage[spanish,es-tabla]{babel}
\usepackage[autostyle,spanish=mexican]{csquotes}
\usepackage[tbtags]{amsmath}
\usepackage{nccmath}
\usepackage{amsthm}
\usepackage{amssymb}
\usepackage{mathrsfs}
\usepackage{graphicx}
\usepackage{subfig}
\usepackage{standalone}
\usepackage[outdir=./Imagenes/]{epstopdf}
\usepackage{siunitx}
\usepackage{physics}
\usepackage{color}
\usepackage{float}
\usepackage{hyperref}
\usepackage{multicol}
%\usepackage{milista}
\usepackage{anyfontsize}
\usepackage{anysize}
%\usepackage{enumerate}
\usepackage[shortlabels]{enumitem}
\usepackage{capt-of}
\usepackage{bm}
\usepackage{relsize}
\usepackage{placeins}
\usepackage{empheq}
\usepackage{cancel}
\usepackage{wrapfig}
\usepackage[flushleft]{threeparttable}
\usepackage{makecell}
\usepackage{fancyhdr}
\usepackage{tikz}
\usepackage{bigints}
\usepackage{scalerel}
\usepackage{pgfplots}
\usepackage{pdflscape}
\pgfplotsset{compat=1.16}
\spanishdecimal{.}
\renewcommand{\baselinestretch}{1.5} 
\renewcommand\labelenumii{\theenumi.{\arabic{enumii}})}
\newcommand{\ptilde}[1]{\ensuremath{{#1}^{\prime}}}
\newcommand{\stilde}[1]{\ensuremath{{#1}^{\prime \prime}}}
\newcommand{\ttilde}[1]{\ensuremath{{#1}^{\prime \prime \prime}}}
\newcommand{\ntilde}[2]{\ensuremath{{#1}^{(#2)}}}

\newtheorem{defi}{{\it Definición}}[section]
\newtheorem{teo}{{\it Teorema}}[section]
\newtheorem{ejemplo}{{\it Ejemplo}}[section]
\newtheorem{propiedad}{{\it Propiedad}}[section]
\newtheorem{lema}{{\it Lema}}[section]
\newtheorem{cor}{Corolario}
\newtheorem{ejer}{Ejercicio}[section]

\newlist{milista}{enumerate}{2}
\setlist[milista,1]{label=\arabic*)}
\setlist[milista,2]{label=\arabic{milistai}.\arabic*)}
\newlength{\depthofsumsign}
\setlength{\depthofsumsign}{\depthof{$\sum$}}
\newcommand{\nsum}[1][1.4]{% only for \displaystyle
    \mathop{%
        \raisebox
            {-#1\depthofsumsign+1\depthofsumsign}
            {\scalebox
                {#1}
                {$\displaystyle\sum$}%
            }
    }
}
\def\scaleint#1{\vcenter{\hbox{\scaleto[3ex]{\displaystyle\int}{#1}}}}
\def\bs{\mkern-12mu}


\title{Ejercicios opcionales \\[0.3em]  \large{Material 3 - Completitud} \vspace{-3ex}}
\author{M. en C. Gustavo Contreras Mayén}
\date{ }

\begin{document}
\vspace{-4cm}
\maketitle
\fontsize{14}{14}\selectfont

\noindent
%Ref. De la Peña - Mecánica cuántica
\textbf{Ejercicio opcional (13). } Demuestra que se puede escribir:
\begin{align*}
\delta (x - \xi) = \dfrac{2}{L} \sum_{n=1}^{\infty} \sin \left( \dfrac{n \, \pi \, \xi}{L} \right) \, \sin \left( \dfrac{n \, \pi \, x}{L} \right) \hspace{1.5cm} 0 < \xi < L 
\end{align*}

\vspace*{1cm}
\noindent
\textbf{Ejercicio opcional (14). } Una representación importante de la delta de Dirac se construye considerando el límite $n \to \infty$ de la secuencia:
\begin{align*}
\delta_{n} = \begin{cases}
c_{n} \, (1 - x^{2})^{n} & \mbox{ para } 0 \leq \abs{x} \leq 1, \hspace{0.5cm} n = 1, 2, 3, \ldots \\
0 & \mbox{ para } \abs{x} > 1
\end{cases}
\end{align*}
\begin{enumerate}[label=\alph*)]
\item Determina los coeficientes $c_{n}$ tales que:
\begin{align*}
\scaleint{5ex}_{\bs -1}^{1} \delta_{n} (x) \dd x = 1
\end{align*}
\item Demuestra que:
\begin{align*}
\lim_{n \to \infty} \int_{-1}^{1} f(x) \, \delta_{n} (x) \, \dd x = f(0)
\end{align*}
\end{enumerate}

\end{document}