\input{../Preambulos/preambulo_presentacion_Warsaw_crane}
\title{\large{Delta de Dirac}}
\subtitle{Ejercicios}
\author{M. en C. Gustavo Contreras Mayén}
\date{}
\institute{Facultad de Ciencias - UNAM}
\titlegraphic{\includegraphics[width=1.75cm]{../Imagenes/escudo-facultad-ciencias}\hspace*{4.75cm}~%
   \includegraphics[width=1.75cm]{../Imagenes/escudo-unam}
}
\setbeamertemplate{navigation symbols}{}
\begin{document}
\maketitle
\fontsize{14}{14}\selectfont
\spanishdecimal{.}
\section*{Contenido}
\frame[allowframebreaks]{\tableofcontents[currentsection, hideallsubsections]}
\section{Introducción}
\frame{\tableofcontents[currentsection, hideothersubsections]}
\subsection{Base importante}
\begin{frame}
\frametitle{Introducción}
La delta de Dirac es un concepto que permite analizar, a través de un intervalo no muy largo de tiempo, las distintas adaptaciones conceptuales que se hacen de medios importantes de la matemática, que son utilizados de acuerdo con los intereses de ciertas disciplinas y a sus contextos concretos de aplicación.
\end{frame}
\begin{frame}
\frametitle{Formulación de Dirac}
En su obra clásica, The Principles of Quantum Mechanics, P.A.M. Dirac observa que sus investigaciones lo han llevado a
\begin{quote}
... considerar cantidades que involucran cierta clase de infinitos. Para lograr una notación precisa en el manejo de estos infinitos, introducimos una cantidad $\delta (x)$ dependiendo de un parámetro $x$, que satisface las condiciones:
\end{quote}
\end{frame}
\begin{frame}
\frametitle{Formulación de Dirac}
Que satisface las condiciones:
\begin{align*}
\int_{-\infty}^{\infty} \delta (x) \dd{x} = 1, \hspace{0.5cm} \delta(x) = 0 \hspace{0.2cm} \mbox{para } x \neq 0
\end{align*}
\end{frame}
\begin{frame}
\frametitle{Formulación de Dirac}
Dirac reconoce que el objeto que ha descrito:
\\
\bigskip
\begin{quote}
no es una función de $x$ de acuerdo con la definición matemática usual, la que requiere que una función tenga un valor definido para cada punto de su dominio, sino algo más general, lo cual podemos llamar función impropia para destacar su diferencia con la de función según la definición usual.
\end{quote}
\end{frame}
\subsection{¿Cómo estudiar la delta de Dirac?}
\begin{frame}
\frametitle{¿Cómo estudiar la delta de Dirac?}
Hay dos contextos desde los cuales se formulan las teorías, los métodos y también las propuestas de aplicación de la delta de Dirac:
\setbeamercolor{item projected}{bg=blue!70!black,fg=yellow}
\setbeamertemplate{enumerate items}[circle]
\begin{enumerate}[<+->]
\item El de la matemática pura.
\item El de la física e ingeniería.
\end{enumerate}
\end{frame}
\begin{frame}
\frametitle{Vista desde la matemática pura}
Se elabora la teoría de distribuciones que fundamenta a la delta de Dirac, y que va mucho más allá, pues involucra espacios de funciones, convergencia, operadores y teoría de la medida.
\\
\bigskip
\pause
En la teoría de distribuciones, la delta de Dirac es \emph{solo un ejemplo} de distribución.
\end{frame}
\begin{frame}
\frametitle{Teoría de las distribuciones}
El francés Laurent Schwartz, fue uno de los más respetados y prestigiosos matemáticos de nuestro tiempo.
\\
\bigskip
En el Congreso Internacional de Matemáticas celebrado en Harvard en 1950, le fue concedida la Medalla Fields por su creación de las distribuciones. 
\end{frame}
\begin{frame}
\frametitle{Vista desde la física}
La forma en como utilizan la delta de Dirac los ingenieros, físicos y profesores, pasan de las distribuciones, o funciones generalizadas a funciones que tienen nombres propios, y se reducen básicamente a las siguientes:
\end{frame}
\begin{frame}
\frametitle{Vista desde la física}
\setbeamercolor{item projected}{bg=blue!70!black,fg=yellow}
\setbeamertemplate{enumerate items}[circle]
\begin{enumerate}[<+->]
\item La \enquote{función} de Heaviside o escalón unitario.
\item La delta de Dirac, o impulso unitario.
\item La delta de Dirac periódica o tren de impulsos.
\item Las derivadas de estas \enquote{funciones}.
\end{enumerate}
\end{frame}
\subsection{Definición y propiedades}
\begin{frame}
\frametitle{Definición}
La función delta de Dirac se define como:
\begin{align*}
\delta (x) \begin{cases}
0 & \mbox{si } x \neq 0 \\
\infty & \mbox{si } x = 0
\end{cases}
\end{align*}
tal que
\begin{align*}
\int_{-\infty}^{\infty} \delta (x) \dd{x} = 1
\end{align*}
\end{frame}
\begin{frame}
\frametitle{delta de Dirac para un impulso}
Para un impulso en $x = a$
\begin{align*}
\delta (x - a) \begin{cases}
0 & \mbox{si } x \neq a \\
\infty & \mbox{si } x = a
\end{cases}
\end{align*}
tal que
\begin{align*}
\int_{-\infty}^{\infty} \delta (x - a) \dd{x} = 1
\end{align*}
\end{frame}
\begin{frame}
\frametitle{Aclaración importante}
Se debe resaltar el hecho de que la función delta de Dirac en realidad sólo tiene sentido cuando se utiliza bajo el signo integral.
\\
\bigskip
Y este es el punto central que nunca hay que perder de vista:
\end{frame}
\begin{frame}
\frametitle{Aclaración importante}
La función $\delta(x)$ de Dirac por sí sola carece de sentido matemático y físico; dicha función solo tiene sentido cuando es usada bajo el signo de integración.
\end{frame}
\begin{frame}
\frametitle{Aclaración importante}
Resulta desafortunado el hecho de que en muchos textos al discutir relaciones que tienen que ver con la función delta de Dirac se omiten las integrales, causando la falsa impresión de que una función delta de Dirac es algo que puede ser manejado algebraicamente. 
\end{frame}
\begin{frame}
\frametitle{Aclaración importante}
Considérense por ejemplo las siguientes dos relaciones tal y como aparecen en varios libros de texto:
\begin{align*}
\delta [ (x - a)(x - b)] &= \dfrac{\delta (x - a) + \delta(x - b)}{\abs{a - b}} \\[1em]
\delta (x^{2} - a^{2}) &= \dfrac{1}{2 \, \abs{a}} \, [\delta (x + a) + \delta(x - a)]
\end{align*}
\end{frame}
\begin{frame}
\frametitle{Escritura correcta}
La simbolización correcta y completa de las dos relaciones anteriores es la siguiente:
\begin{align*}
\int_{-\infty}^{\infty} \delta [ (x &- a)(x - b)] \dd{x} = \\[1em]
&= \dfrac{1}{\abs{a - b}} \, \int_{-\infty}^{\infty} \big[ \delta (x - a) + \delta(x - b) \big] \dd{x}
\end{align*}
\end{frame}
\begin{frame}
\frametitle{Escritura correcta}
Para la segunda expresión:
\begin{align*}
\int_{-\infty}^{\infty} &\delta (x^{2} - a^{2})\dd{x} = \\[1em]
&= \dfrac{1}{2 \, \abs{a}} \int_{-\infty}^{\infty} [\delta (x + a) + \delta(x - a)] \dd{x}
\end{align*}
\end{frame}
\begin{frame}
\frametitle{Algunas propiedades}
Mencionaremos algunas propiedades importantes de la delta de Dirac, cuya demostración se hace a partir de la definición propia, por lo que nos enfocaremos a mencionarlas y a revisar ejemplos.
\end{frame}
\begin{frame}
\frametitle{Propiedades de la delta de Dirac}
\begin{eqnarray*}
\int_{-\infty}^{\infty} \psi (x) \, \delta (x) \dd{x} &=& \psi(0) \\[0.5em] \pause
\int_{-\infty}^{\infty} \psi (x) \, \delta (x - a) \dd{x} &=& \psi(a)
\end{eqnarray*}
\end{frame}
\begin{frame}
\frametitle{Ejemplo 1}
Evaluar la siguiente integral:
\begin{align*}
\int_{-4}^{+7} (x^{3} - 3 \, x^{2} + 2 \, x + 1) \, \delta(x + 2) \dd{x}
\end{align*}
\pause
El único punto en el cual la integral no es igual a cero es en el punto $x = -2$, en donde la evaluación depende únicamente del valor que tenga la función $f(x = -2)$:
\begin{align*}
(-2)^{3} - 3 (-2)^{2} + 2(-2) + 1 = -23
\end{align*}
\end{frame}
\begin{frame}
\frametitle{Ejemplo 2}
Evaluar la siguiente integral
\begin{align*}
\int_{-2}^{+4} e^{2 \abs{x}-4} \, \delta(x + 5) \dd{x}
\end{align*}
\pause
En este caso la integral es igual a cero, puesto que el punto en el cual la función delta de Dirac aplica su efectividad, $x = -5$, está fuera de los límites de la integración $-2$ y $+4$.
\end{frame}
\begin{frame}
\frametitle{Derivada con la función delta}
Podemos aprovechar el resultado de la integración, ya que es tan buena que incluso se puede definir la derivada de la función delta de Dirac con respecto a $x_{0}$.
\begin{align*}
\int_{-\infty}^{\infty} f(x) \, \ptilde{\delta} (x - x_{0}) \dd{x} = - \ptilde{f}(x_{0})
\end{align*}
\end{frame}
\begin{frame}
\frametitle{Derivadas de orden superior}
Es posible obtener las derivadas de orden superior mediante la expresión:
\begin{align*}
\int_{-\infty}^{\infty} &f(x) \, \delta^{(n)} (x - x_{0}) \dd{x} = \\[0.5em]
&= \begin{cases}
(-1)^{n} \, \ntilde{f}{n} (x_{0}) & \mbox{si } a < x_{0} < b \\
0 & \mbox{de otra manera}
\end{cases}
\end{align*}   
\end{frame}
\begin{frame}
\frametitle{delta de Dirac de una función}
La delta de Dirac satisface la siguiente relación:
\begin{align*}
\delta(g(x)) = \sum_{k=1}^{n} \dfrac{\delta(x - c_{k})}{\abs{\ptilde{g}(c_{k})}}, \hspace{0.5cm} \ptilde{g}(c_{k}) \neq 0
\end{align*}
donde $\left\{ c_{k} \right\}_{k=1}^{n}$ son todas las raíces de la ecuación $g(x) = 0$.
\end{frame}
\begin{frame}
\frametitle{Derivada de la delta de Dirac de una función}
\begin{align*}
\int_{a}^{b} &f(x) \, \ptilde{\delta}(g(x)) \dd{x} = \\[0.5em]
&= \begin{cases}
- \displaystyle \sum_{k=1}^{n} \dfrac{\ptilde{f}(c_{k})}{\ptilde{g}(c_{k})} & \mbox{si } a < c_{k} < b \\
0 & \mbox{de otra manera}
\end{cases}
\end{align*}
\end{frame}
\begin{frame}
\frametitle{Ejemplo 3}
Evalúa la siguiente integral:
\begin{align*}
I \equiv \int_{-\infty}^{\infty} f(t) \, \delta(t^{2} - a^{2}) \dd{t}
\end{align*}
donde $f(t)$ es una función suave y $a$ es una constante real.
\end{frame}
\begin{frame}
\frametitle{Solución}
Identificamos que $g(t)$ como $(t^{2} - a^{2})$, \pause que tiene raíces $c_{1} = -a$ y $c_{2} = a$, con la primera derivada $\ptilde{g}(t) = 2 \, t$, por lo que tendremos:
\pause
\begin{eqnarray*}
\delta(t^{2} - a^{2}) &=& \pause \dfrac{\delta (t - c_{1})}{\abs{\ptilde{g}(c_{1})}} + \dfrac{\delta (t - c_{2})}{\abs{\ptilde{g}(c_{2})}} = \\[0.5cm] \pause
&=& \dfrac{\delta (t - (-a))}{\abs{-2a}} + \dfrac{\delta (t - a)}{\abs{2a}} = \\[0.5em] \pause
&=& \dfrac{1}{2 \abs{a}} \big[ \delta(t + a) + \delta(t - a) \big]
\end{eqnarray*}
\end{frame}
\begin{frame}
\frametitle{Solución}
Sustituyendo en la integral obtenemos:
\begin{eqnarray*}
I &=& \dfrac{1}{2 \abs{a}} \int_{-\infty}^{\infty} f(t) \, \big[ \delta(t {+} a) + \delta(t {-} a) \big] \dd{t} = \\[0.5em] \pause
&=& \dfrac{1}{2 \abs{a}} \left\{ \int_{-\infty}^{\infty} f(t) \, \delta(t {+} a) \dd{t} {+} \int_{-\infty}^{\infty} f(t) \, \delta(t {-} a) \dd{t} \right\} \\[0.5em] \pause
&=& \dfrac{1}{2 \abs{a}} \bigg[ f(-a) + f(a) \bigg]
\end{eqnarray*}
\pause
Vemos que la integral se anula (como es de esperarse) si la función $f$ es impar.
\end{frame}
\begin{frame}
\frametitle{Ejemplo 4}
Evaluar la integral
\begin{align*}
\int_{1}^{\infty} \sin t \, \delta \left( t^{2} - \dfrac{\pi}{4} \right) \dd{t}
\end{align*}
\pause
Vemos que
\begin{align*}
g(t) = t^{2} - \dfrac{\pi}{4}, \hspace{1.5cm} c_{1} = \dfrac{\pi}{2}, \hspace{0.3cm} c_{2} = - \dfrac{\pi}{2}
\end{align*}
\end{frame}
\begin{frame}
\frametitle{Solución}
Usamos solo la raíz positiva en el rango de integración. Además: $\ptilde{g}(t) =  2 \, t$, entonces: \pause
\begin{eqnarray*}
\int_{1}^{\infty} \sin t \, \delta \left( t^{2} - \dfrac{\pi}{4} \right) \dd{t} &=& \pause \dfrac{f(c_{1})}{\abs{\ptilde{g}(c_{1})}} = \\[0.5em] \pause
&=& \dfrac{\sin (\pi/2)}{\pi} = \dfrac{1}{\pi}
\end{eqnarray*}
\end{frame}
\begin{frame}
\frametitle{Solución}
En el otro caso:
\begin{align*}
\int_{-\infty}^{\infty} \sin t \, \delta \left( t^{2} - \dfrac{\pi}{4} \right) \dd{t} = 0
\end{align*}
\pause
ya que con la segunda raíz $c_{2}$ que también está en el rango de integración, pero su contribución cancela la de $c_{1}$.
\end{frame}
\begin{frame}
\frametitle{Ejemplo 5}
Evaluar la integral:
\begin{align*}
\int_{0}^{\infty} \ln z \, \delta (z^{2} - 4 ) \dd{z}
\end{align*}
\pause
Vemos que $g(z) = z^{2} - 4$, la cual tiene dos raíces $c_{1} = 2$ y $c_{2} = -2$, de las cuales solo la raíz positiva está en el rango de integración.
\end{frame}
\begin{frame}
\frametitle{Solución}
Entonces, con $\ptilde{g}(z) = 2 \, z$, tenemos:
\begin{eqnarray*}
\int_{0}^{\infty} \ln z \, \delta (z^{2} - 4 ) \dd{z} &=& \dfrac{f(c_{1})}{\abs{\ptilde{g}(c_{1})}} = \\[0.5em] \pause
&=& \dfrac{\ln (c_{1})}{\abs{2 c_{1}}} \\[0.5em] \pause
&=& \dfrac{\ln 2}{4} = \pause 0.1733
\end{eqnarray*}
\end{frame}
\subsection{Caso bidimensional}
\begin{frame}
\frametitle{Caso bidimensional}
Definiendo los puntos $P$ y $P_{0}$ con sus respectivas coordenadas cartesianas $(x, y)$ y $(x_{0}, y_{0})$ y con vectores de posición $\vb{r} = (x, y)$, $\vb{r}_{0} = (x_{0}, y_{0})$, así que:
\begin{align*}
\delta(x &- x_{0}, y - y_{0}) \equiv \delta (\vb{r} - \vb{r}_{0}) = \\[0.5em]
&=\begin{cases}
\delta(\va{0}) \equiv \delta (0, 0) = \infty & \mbox{si } \vb{r} = \vb{r}_{0} \\
0 & \mbox{de otra manera}
\end{cases}
\end{align*}
\end{frame}
\begin{frame}
\frametitle{Caso bidimensional}
La función delta de Dirac bidimensional es cero en todas partes excepto en el punto que hace que sus dos argumentos sean cero, en cuyo caso la función delta de Dirac bidimensional es infinita.
\\
\bigskip
\pause
Por lo que en coordenadas cartesianas:
\begin{align*}
\delta(\vb{r} - \vb{r}_{0}) = \delta(x - x_{0}, y - y_{0}) = \delta(x - x_{0}) \, \delta(y - y_{0})
\end{align*}
\end{frame}
\begin{frame}
\frametitle{La delta de Dirac en coord. polares}
La función delta de Dirac bidimensional en coordenadas polares se escribe como:
\begin{align*}
\delta(\vb{r} - \vb{r}_{0}) = \dfrac{1}{\rho_{0}} \delta(\rho - \rho_{0}) \delta( \varphi - \varphi_{0})
\end{align*}
\end{frame}
\begin{frame}
\frametitle{Caso de tres dimensiones}
Podemos generalizar para el caso de tres dimensiones en donde la delta de Dirac es:
\begin{align*}
&\delta (\vb{r} - \vb{r}_{0}) = \\[0.5em]
&=\begin{cases}
\delta(\va{0}) \equiv \delta (0, 0, 0) = \infty & \mbox{si } \vb{r} = \vb{r}_{0} \\
0 & \mbox{de otra manera}
\end{cases}
\end{align*}
\end{frame}
\begin{frame}
\frametitle{Caso de tres dimensiones}
La función delta de Dirac tridimensional es cero en todas partes excepto en el punto que hace que sus tres argumentos sean cero, en cuyo caso es infinito.
\pause
Por lo que en coordenadas cartesianas:
\begin{align*}
\delta(\vb{r} - \vb{r}_{0}) &= \delta(x - x_{0}, y - y_{0}, z -z_{0}) = \\[0.5em]
&= \delta(x - x_{0}) \, \delta(y - y_{0}) \, \delta(z - z_{0})
\end{align*}
\end{frame}
\begin{frame}
\frametitle{La delta de Dirac en coord. cilíndricas}
Se tiene entonces que:
\begin{align*}
\delta(\vb{r} - \vb{r}_{0}) = \dfrac{1}{\rho_{0}} \delta(\rho - \rho_{0}) \, \delta( \varphi - \varphi_{0}) \, \delta(z - z_{0})
\end{align*}
\pause
\textbf{Nota:} donde $\vb{r}$ y $\vb{r}_{0}$ deben de entenderse en coordenadas cilíndricas y no como vectores de posición en coordenadas cilíndricas.
\end{frame}
\begin{frame}
\frametitle{La delta de Dirac en coord. esféricas}
La correspondiente expresión de la delta de Dirac en coordenadas esféricas es:
\begin{align*}
\delta(\vb{r} - \vb{r}_{0}) = \dfrac{1}{r_{0}^{2} \, \sin \theta_{0}} \delta(r - r_{0}) \, \delta(\theta - \theta_{0}) \,\delta( \varphi - \varphi_{0})
\end{align*}
\pause
con $\vb{r}$ y $\vb{r}_{0}$ representando las coordenadas $(r, \theta, \varphi)$ y $(r_{0}, \theta_{0}, \varphi_{0})$ respectivamente.
\end{frame}
   
\end{document}