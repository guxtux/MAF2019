\documentclass[12pt]{article}
\usepackage[utf8]{inputenc}
\usepackage[spanish,es-lcroman, es-tabla]{babel}
\usepackage[autostyle,spanish=mexican]{csquotes}
\usepackage{amsmath}
\usepackage{amssymb}
\usepackage{nccmath}
\numberwithin{equation}{section}
\usepackage{amsthm}
\usepackage{graphicx}
\usepackage{epstopdf}
\DeclareGraphicsExtensions{.pdf,.png,.jpg,.eps}
\usepackage{color}
\usepackage{float}
\usepackage{multicol}
\usepackage{enumerate}
\usepackage[shortlabels]{enumitem}
\usepackage{anyfontsize}
\usepackage{anysize}
\usepackage{array}
\usepackage{multirow}
\usepackage{enumitem}
\usepackage{cancel}
\usepackage{tikz}
\usepackage{circuitikz}
\usepackage{tikz-3dplot}
\usetikzlibrary{babel}
\usetikzlibrary{shapes}
\usepackage{bm}
\usepackage{mathtools}
\usepackage{esvect}
\usepackage{hyperref}
\usepackage{relsize}
\usepackage{siunitx}
\usepackage{physics}
%\usepackage{biblatex}
\usepackage{standalone}
\usepackage{mathrsfs}
\usepackage{bigints}
\usepackage{bookmark}
\spanishdecimal{.}

\setlist[enumerate]{itemsep=0mm}

\renewcommand{\baselinestretch}{1.5}

\let\oldbibliography\thebibliography

\renewcommand{\thebibliography}[1]{\oldbibliography{#1}

\setlength{\itemsep}{0pt}}
%\marginsize{1.5cm}{1.5cm}{2cm}{2cm}


\newtheorem{defi}{{\it Definición}}[section]
\newtheorem{teo}{{\it Teorema}}[section]
\newtheorem{ejemplo}{{\it Ejemplo}}[section]
\newtheorem{propiedad}{{\it Propiedad}}[section]
\newtheorem{lema}{{\it Lema}}[section]

\title{Ejercicios operadores Hermiticos\\ \large {Matemáticas Avanzadas de la Física}  \vspace{-1.5\baselineskip}}
\date{}
\author{}
\begin{document}
\maketitle
\fontsize{14}{14}\selectfont
\begin{enumerate}
\item Discute si los siguientes operadores son hermiticos o no:
\begin{enumerate}
\item $\left( \hat{A} + \hat{A}^{\dagger} \right)$
\item $i \, \left( \hat{A} + \hat{A}^{\dagger} \right)$
\item $i \, \left( \hat{A} - \hat{A}^{\dagger} \right)$
\end{enumerate}
\item Calcula el Hermitiano adjunto de
\begin{align*}
f(\hat{A}) = \dfrac{(1 + i \, \hat{A} + 3 \, \hat{A}^{2})(1 - 2 \, i \, \hat{A} - 9 \, \hat{A}^{2})}{(5 + 7 \, \hat{A})}
\end{align*}
\item Demuestra que el valor esperado de un operador Hermitiano es un valor real y que para un operador antiHermitiano, el valor es es imaginario.
\end{enumerate}
\textbf{Solución:}
\begin{enumerate}
\item 
\begin{enumerate}
\item Hagamos $\hat{B} = \left( \hat{A} + \hat{A}^{\dagger} \right)$, entonces
\begin{align*}
\hat{B}^{\dagger} &= \left( \hat{A} + \hat{A}^{\dagger} \right)^{\dagger} = \\
\hat{B}^{\dagger} &= \hat{A}^{\dagger} + \hat{A} = \\
\hat{B}^{\dagger} &= \hat{B}
\end{align*}
por lo tanto, el operador $\hat{B}$ es Hermitiano independientemente de que el operador $\hat{A}$ sea Hermitiano o no.
\item Sea $\hat{C} = \left( \hat{A} + \hat{A}^{\dagger} \right)$, entonces
\begin{align*}
\left(i \, \hat{C} \right)^{\dagger} &= i^{*} \, \hat{C}^{\dagger} = \\
&= -i \, \left( \hat{A} + \hat{A}^{\dagger} \right)^{\dagger} = \\
&= -i \, \left( \hat{A}^{\dagger} + \hat{A} \right) = \\
&= -i \, \left( \hat{A} + \hat{A}^{\dagger} \right) = \\
&= - i \, \hat{C}
\end{align*}
por lo tanto el operador $\hat{C}$ es antiHermitiano.
\item Sea $\hat{D} = \left( \hat{A} - \hat{A}^{\dagger} \right)$, entonces
\begin{align*}
\left(i \, \hat{D} \right)^{\dagger} &= i^{*} \, \hat{D}^{\dagger} = \\
&= -i \, \left( \hat{A} - \hat{A}^{\dagger} \right)^{\dagger} = \\
&= -i \, \left( \hat{A}^{\dagger} - \hat{A} \right) = \\
&= -i \, (-1) \left( \hat{A} - \hat{A}^{\dagger} \right) = \\
&= i \, \left( \hat{A} - \hat{A}^{\dagger} \right) = \\
&= i \, \hat{D}
\end{align*}
por lo tanto el operador $\hat{D}$ es Hermitiano.
\end{enumerate}
\item Ya que el adjunto Hermitano de una función de operadores $f(\hat{A})$ está dado por
\begin{align*}
f^{\dagger}(\hat{A}) = f^{*}(\hat{A}^{\dagger})
\end{align*}
entonces podemos hacer que
\begin{align*}
\left( \dfrac{(1 + i \, \hat{A} + 3 \, \hat{A}^{2})(1 - 2 \, i \, \hat{A} - 9 \, \hat{A}^{2})}{(5 + 7 \, \hat{A})} \right)^{\dagger} = \\[0.5em]
= \dfrac{(1 + 2 \, i \, \hat{A^{\dagger}} - 9 \, \hat{A}^{\dagger \, 2})(1 - i \, \hat{A}{\dagger} + 3 \, \hat{A}^{{\dagger} \, 2})}{(5 + 7 \, \hat{A}^{\dagger})}
\end{align*}
\item De la definición
\begin{align*}
\hat{A} &= \hat{A}^{\dagger}  \hspace{1cm} \mbox{ o } \\
\bra{\psi} \, \hat{A} \, \ket{\phi} &= \bra{\phi} \, \hat{A} \, \ket{\psi}^{*}
\end{align*}
de manera inmediata inferimos que el valor esperador de un operador Hermitiano es real, ya que satisface la siguiente propiedad
\begin{align*}
\bra{\psi} \, \hat{A} \, \ket{\psi} &= \bra{\psi} \, \hat{A} \, \ket{\psi}^{*}
\end{align*}
es decir, si $\hat{A}^{\dagger} = \hat{A}$, entonces $\bra{\psi} \, \hat{A} \, \ket{\psi}$ es real.
\par
De manera similar, para un operador antiHermitiano, $\hat{B}^{\dagger} = - \hat{B}$, se tiene
\begin{align*}
\bra{\psi} \, \hat{B} \, \ket{\psi} = - \bra{\psi} \, \hat{B} \, \ket{\psi}^{*}
\end{align*}
lo que significa que $\bra{\psi} \, \hat{B} \, \ket{\psi}$ es un número imaginario puro.
\end{enumerate}
\end{document}