\documentclass[12pt]{beamer}
\usepackage{../Estilos/BeamerMAF}
\input{../Preambulos/preambulo_Beamer_Dresden_seahorse}

\date{26 de octubre de 2021}

\title{\large{Tema 3 - Bases completas y ortogonales}}
\subtitle{Curso MAF}
\author{M. en C. Gustavo Contreras Mayén}


\begin{document}
\maketitle
\fontsize{14}{14}\selectfont
\spanishdecimal{.}

\section*{Contenido}
\frame[allowframebreaks]{\tableofcontents[currentsection, hideallsubsections]}

\section{Introducción}
\frame{\tableofcontents[currentsection, hideothersubsections]}
\subsection{Avances en el curso}

\begin{frame}
\frametitle{Lo que hemos trabajado}
Hasta ahora hemos discutido de manera general el tipo de ecuaciones que aparecen en la física y algunas de sus propiedades, hemos trabajado con algunas de las ecuaciones diferenciales en el que estamos interesados y hemos estudiado sus singularidades.
\end{frame}
\begin{frame}
\frametitle{Lo que hemos trabajado}
Todas estas ecuaciones son ecuaciones diferenciales lineales de segundo orden, las cuales como hemos visto, admiten sólo dos soluciones linealmente independientes.
\end{frame}
\begin{frame}
\frametitle{Lo que hemos trabajado}
Nuestro objetivo en el Tema 3 es estudiar algunas propiedades adicionales de estas ecuaciones, así como algunas características generales de sus soluciones. 
\end{frame}
\begin{frame}
\frametitle{Lo que hemos trabajado}
En este tema, nos enfocaremos a resolver la ED y a \emph{desarrollar y comprender las propiedades generales de las soluciones}.
\end{frame}
\begin{frame}
\frametitle{Hacia dónde nos dirigimos}
Algunos de los conceptos que utilizaremos en este tema, los habrás revisado en los cursos de ecuaciones diferenciales y de álgebra lineal, por lo que en caso de que necesites darle un repaso, será conveniente para que no haya alguna complicación.
\end{frame}

\section{Objetivo}
\frame{\tableofcontents[currentsection, hideothersubsections]}
\subsection{Objetivos del Tema 3}

\begin{frame}
\frametitle{Objetivos}
Al concluir el Tema 3: Bases completas y ortogonales, el alumno:
\setbeamercolor{item projected}{bg=blue!70!black,fg=yellow}
\setbeamertemplate{enumerate items}[circle]
\begin{enumerate}[<+->]
\item Identificará los problemas de tipo Sturm-Liouville en una EDO2H.
\item Llevará a cabo la correspondiente transformación para definir un problema al tipo Sturm-Liouville.
\item Obtendrá las soluciones a un problema de tipo Sturm-Liouville en términos de valores propios y funciones propias.
\conti
\end{enumerate}
\end{frame}
\begin{frame}
\frametitle{Objetivos}
\setbeamercolor{item projected}{bg=blue!70!black,fg=yellow}
\setbeamertemplate{enumerate items}[circle]
\begin{enumerate}[<+->]
\conti
\item Obtendrá el operador adjunto de un problema Sturm-Liouville.
\item Verificará la condición de operador autoadjunto (Hermitiano).
\item Establecerá la condición de ortogonalidad de las funciones que son solución a un problema de tipo Sturm-Liouville.
\item Determinará la ortogonalidad de funciones que no lo son mediante el método de Gram-Schmidt.
\conti
\end{enumerate}
\end{frame}
\begin{frame}
\frametitle{Objetivos}
\setbeamercolor{item projected}{bg=blue!70!black,fg=yellow}
\setbeamertemplate{enumerate items}[circle]
\begin{enumerate}[<+->]
\conti
\item Verificará la propiedad de completes (completitud) en las funciones propias que se obtienen como solución del problema de tipo Sturm-Liouville.
\item Obtendrá la expansión de una función en términos de funciones propias ortogonales y normalizadas.
\end{enumerate}
\end{frame}

% \section{Temas a revisar}
% \frame{\tableofcontents[currentsection, hideothersubsections]}
% \subsection{Contenido}


% \begin{frame}
% \frametitle{Temas}
% \setbeamercolor{item projected}{bg=blue!70!black,fg=yellow}
% \setbeamertemplate{enumerate items}[circle]
% \begin{enumerate}[<+->]
% \item Problemas de tipo Sturm-Liouville.
% \item Operadores adjuntos y autoadjuntos (Hermitianos).
% \item Funciones propias ortogonales.
% \item Completes de una base.
% \end{enumerate}
% \end{frame}

\subsection{Temas complementarios}

\begin{frame}
\frametitle{Temas complementarios}
Considerando que se va a requerir material adicional para tener una base sólida de conocimiento, se proponen dos temas complementarios:
\setbeamercolor{item projected}{bg=blue!70!black,fg=yellow}
\setbeamertemplate{enumerate items}[circle]
\begin{enumerate}[<+->]
\item Función delta de Dirac.
\item Espacio de Hilbert.
\end{enumerate}
\end{frame}
% \begin{frame}
% \frametitle{Sobre los temas complementarios}
% El tema de la \emph{función delta de Dirac} es importante ya que vamos a apoyarnos bastante en lo que sigue del curso, por lo que recomendamos ampliamente que lo revisen y trabajen los ejercicios opcionales para repasar.
% \end{frame}

\section{Cronograma de trabajo}
\frame{\tableofcontents[currentsection, hideothersubsections]}
\subsection{Trabajo por semana}

\begin{frame}
\frametitle{Distribución de tiempos}
Para una revisión completa de los materiales de trabajo para el Tema 3, se presenta la siguiente distribución de tiempos:
\end{frame}
\begin{frame}
\frametitle{Distribución de tiempos}
\setbeamercolor{item projected}{bg=blue!70!black,fg=yellow}
\setbeamertemplate{enumerate items}[circle]
\begin{enumerate}
\item El tema de problemas tipo Sturm-Liouville se debe de revisar durante la semana 6: 26 al 29 de octubre.
\item  El tema de Operadores autoadjuntos durante la semana 7 y 8.
\item El tema de completes de una base durante la semana 8.
\end{enumerate}
\end{frame}

\subsection{Sesiones síncronas}

\begin{frame}
\frametitle{Sesiones síncronas de trabajo}
Se continuará con las sesiones de trabajo los días miércoles y viernes:
\setbeamercolor{item projected}{bg=blue!70!black,fg=yellow}
\setbeamertemplate{enumerate items}[circle]
\begin{enumerate}[<+->]
\item 26 y 29 de octubre.
\item 3 y 5 de noviembre.
\item 10 y 12 de noviembre.
\end{enumerate}
\end{frame}
\begin{frame}
\frametitle{Sesiones adicionales}
En caso de que se requieran, se continuará con sesiones adicionales de trabajo para la revisión y solución de ejercicios.
\\
\bigskip
\pause
Las reuniones se llevará a cabo bajo demanda el grupo.
\end{frame}

\section{Evaluación}
\frame{\tableofcontents[currentsection, hideothersubsections]}
\subsection{Ejercicios}

\begin{frame}
\frametitle{Ejercicios a cuenta}
La distribución de ejercicios es la siguiente:
\pause
\begin{table}
\centering
\begin{tabular}{l c c}
Material & Semanales & Opcionales \\ \hline
Sturm-Liouville & 3 & 2 \\ \hline
Operadores autoadjuntos & 3 & 2 \\ \hline
Completes de una base & 3 & 2 \\ \hline    
\end{tabular}
\end{table}
\end{frame}

\subsection{Examen Tarea}

\begin{frame}
\frametitle{Entrega del examen tarea 3}
En la sesión del 12 de noviembre se entregarán los enunciados del examen tarea 3, \pause para enviar la solución el día viernes 26 de noviembre.
\end{frame}
\begin{frame}
\frametitle{Días feriados}
De acuerdo al calendario oficial, se presentas los días feriados:
\pause
\setbeamercolor{item projected}{bg=blue!70!black,fg=yellow}
\setbeamertemplate{enumerate items}[circle]
\begin{enumerate}[<+->]
\item Lunes 1 de noviembre.
\item Martes 2 de noviembre.
\item Lunes 15 de noviembre.
\end{enumerate}
\end{frame}
\begin{frame}
\frametitle{De los días feriados}
En estos días no tendremos actividades, considerando que son días oficiales de asueto, por lo que la entrega de los ejercicios de la semana 7, se deberán de enviar junto con los de la semana 8 para el día 19 de noviembre.
\end{frame}


% \section{Punto importante}
% \frame{\tableofcontents[currentsection, hideothersubsections]}
% \subsection{Mensaje para considerar}
% \begin{frame}
% \frametitle{Avance en el curso}
% Ya hemos concluido dos de los seis temas del curso, recordando que la primera evaluación corresponde a los tres primeros temas.
% \end{frame}
% \begin{frame}
% \frametitle{Examen-Tarea}
% Se hará el envío por correo con el archivo pdf con las preguntas del examen tarea correspondiente a los Temas 1 y 2.
% \\
% \bigskip
% \pause
% Con la finalidad de que los atiendan oportunamente, los resuelvan y \textbf{\textcolor{red}{los envíen de regreso en dos semanas}}.
% \end{frame}
% \begin{frame}
% \frametitle{Acuse de recibido}
% Se solicitará el acuse de recibido, es decir, esperamos una respuesta por parte de ustedes, de haber recibido el mensaje y el archivo adjunto.
% \\
% \bigskip
% \pause
% Esto nos dará la evidencia de que recibieron lo necesario para responder parte del primer examen parcial.
% \end{frame}
% \begin{frame}
% \frametitle{Preguntas del Tema 3}
% Conforme vayamos avanzando en el Tema 3, se les proporcionarán los ejercicios del examen-tarea de este tema.
% \\
% \bigskip
% \pause
% El punto es que deben de medir bien su carga de trabajo y aplicar un buen esfuerzo para estas actividades necesarias de evaluación.
% \end{frame}
\end{document}