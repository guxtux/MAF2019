\documentclass[12pt]{beamer}
\usepackage{../Estilos/BeamerMAF}
\usepackage{../Estilos/ColoresLatex}
\input{../Preambulos/preambulo_Beamer_Dresden_seahorse}

\setbeamercolor{section in foot}{bg=ginger, fg=black}
\setbeamercolor{subsection in foot}{bg=kellygreen, fg=black}
\setbeamercolor{date in foot}{bg=goldenrod, fg=white}

\makeatletter
\setbeamertemplate{footline}
{
  \leavevmode%
  \hbox{%
  \begin{beamercolorbox}[wd=.333333\paperwidth,ht=2.25ex,dp=1ex,center]{section in foot}%
    \usebeamerfont{section in foot} \insertsection
  \end{beamercolorbox}%
  \begin{beamercolorbox}[wd=.333333\paperwidth,ht=2.25ex,dp=1ex,center]{subsection in foot}%
    \usebeamerfont{subsection in foot}  \insertsubsection
  \end{beamercolorbox}%
  \begin{beamercolorbox}[wd=.333333\paperwidth,ht=2.25ex,dp=1ex,right]{date in head/foot}%
    \usebeamerfont{date in head/foot} \insertshortdate{} \hspace*{2em}
    \insertframenumber{} / \inserttotalframenumber \hspace*{2ex} 
  \end{beamercolorbox}}%
  \vskip0pt%
}
\makeatother
% \usefonttheme{serif}
\resetcounteronoverlays{saveenumi}

\AtBeginDocument{\RenewCommandCopy\qty\SI}
\ExplSyntaxOn
\msg_redirect_name:nnn { siunitx } { physics-pkg } { none }
\ExplSyntaxOff

\numberwithin{equation}{section}

\date{\today}

\title{\large{Tema 3 - Bases completas y ortogonales}}
\subtitle{Curso MAF}
\author{M. en C. Gustavo Contreras Mayén}


\begin{document}
\maketitle
\fontsize{14}{14}\selectfont
\spanishdecimal{.}

\section*{Contenido}
\frame[allowframebreaks]{\frametitle{Contenido} \tableofcontents[currentsection, hideallsubsections]}

\section{Introducción}
\frame[allowframebreaks]{\frametitle{Temas a revisar} \tableofcontents[currentsection, hideothersubsections]}
\subsection{Avances en el curso}

\begin{frame}
\frametitle{Lo que hemos trabajado}
Hasta ahora hemos discutido de manera general el tipo de ecuaciones que aparecen en la física y algunas de sus propiedades, hemos trabajado con algunas de las ecuaciones diferenciales en el que estamos interesados y hemos estudiado sus singularidades.
\end{frame}
\begin{frame}
\frametitle{Lo que hemos trabajado}
Todas estas ecuaciones son ecuaciones diferenciales lineales de segundo orden, las cuales como hemos visto, admiten sólo dos soluciones linealmente independientes.
\end{frame}
\begin{frame}
\frametitle{Lo que hemos trabajado}
Nuestro objetivo en el Tema 3 es estudiar algunas propiedades adicionales de estas ecuaciones, así como algunas características generales de sus soluciones. 
\end{frame}
\begin{frame}
\frametitle{Lo que hemos trabajado}
En este tema, nos enfocaremos a resolver la ED y a \emph{desarrollar y comprender las propiedades generales de las soluciones}.
\end{frame}
\begin{frame}
\frametitle{Hacia dónde nos dirigimos}
Algunos de los conceptos que utilizaremos en este tema, los habrás revisado en los cursos de ecuaciones diferenciales y de álgebra lineal, por lo que en caso de que necesites darle un repaso, será conveniente para que no haya alguna complicación.
\end{frame}

\section{Objetivo}
\frame[allowframebreaks]{\frametitle{Temas a revisar} \tableofcontents[currentsection, hideothersubsections]}
\subsection{Objetivos del Tema 3}

\begin{frame}
\frametitle{Objetivos}
Para problemas de tipo Sturm-Liouville:
\setbeamercolor{item projected}{bg=babyblue,fg=black}
\setbeamertemplate{enumerate items}[square]
\begin{enumerate}[<+->]
\item Identificará los problemas de este tipo en una EDO2H.
\item Llevará a cabo la correspondiente transformación para llevarlos a la forma.
\item Obtendrá las soluciones en términos de valores propios y funciones propias.
\seti
\end{enumerate}
\end{frame}
\begin{frame}
\frametitle{Objetivos}
Operadores:
\setbeamercolor{item projected}{bg=babyblue,fg=black}
\setbeamertemplate{enumerate items}[square]
\begin{enumerate}[<+->]
\conti
\item Obtendrá el operador adjunto de un problema Sturm-Liouville.
\item Verificará la condición de operador autoadjunto (Hermitiano).
\seti
\end{enumerate}
\end{frame}
\begin{frame}
\frametitle{Objetivos}
Ortogonalidad:
\setbeamercolor{item projected}{bg=babyblue,fg=black}
\setbeamertemplate{enumerate items}[square]
\begin{enumerate}[<+->]
\conti
\item Establecerá la condición de ortogonalidad de las funciones que son solución a un problema de tipo Sturm-Liouville.
\item Con el método de Gram-Schmidt, establecerá la ortogonalidad de funciones que no lo son.
\seti
\end{enumerate}
\end{frame}
\begin{frame}
\frametitle{Objetivos}
Completitud de las funciones:
\setbeamercolor{item projected}{bg=babyblue,fg=black}
\setbeamertemplate{enumerate items}[square]
\begin{enumerate}[<+->]
\conti
\item Verificará la propiedad de completes (completitud) en las funciones propias que se obtienen como solución del problema de tipo Sturm-Liouville.
\item Obtendrá la expansión de una función en términos de funciones propias ortogonales y normalizadas.
\end{enumerate}
\end{frame}

\section{Temas a revisar}
\frame[allowframebreaks]{\frametitle{Temas a revisar} \tableofcontents[currentsection, hideothersubsections]}
\subsection{Contenido}

\begin{frame}
\frametitle{Temas}

\setbeamercolor{item projected}{bg=bananayellow,fg=ao}
\setbeamertemplate{enumerate items}{%
\usebeamercolor[bg]{item projected}%
\raisebox{1.5pt}{\colorbox{bg}{\color{fg}\footnotesize\insertenumlabel}}%
}
\begin{enumerate}[<+->]
\item Problemas de tipo Sturm-Liouville.
\item Operadores adjuntos y autoadjuntos (Hermitianos).
\item Funciones propias ortogonales.
\item Completes de una base.
\end{enumerate}
\end{frame}

% \subsection{Temas complementarios}

% \begin{frame}
% \frametitle{Temas complementarios}
% Considerando que se va a requerir material adicional para tener una base sólida de conocimiento, se propone el siguiente tema complementario:
% \setbeamercolor{item projected}{bg=babyblue,fg=black}
% \setbeamertemplate{enumerate items}[square]
% \begin{enumerate}[<+->]
% \item Espacio de Hilbert.
% \end{enumerate}
% \end{frame}
% \begin{frame}
% \frametitle{Sobre los temas complementarios}
% El tema de la \emph{función delta de Dirac} es importante ya que vamos a apoyarnos bastante en lo que sigue del curso, por lo que recomendamos ampliamente que lo revisen y trabajen los ejercicios opcionales para repasar.
% \end{frame}

% \section{Cronograma de trabajo}
% \frame{\tableofcontents[currentsection, hideothersubsections]}
% \subsection{Trabajo por semana}

% \begin{frame}
% \frametitle{Distribución de tiempos}
% Para una revisión completa de los materiales de trabajo para el Tema 3, se presenta la siguiente distribución de tiempos:
% \end{frame}
% \begin{frame}
% \frametitle{Distribución de tiempos}
% \setbeamercolor{item projected}{bg=babyblue,fg=black}
% \setbeamertemplate{enumerate items}[square]
% \begin{enumerate}
% \item Semanas 6 y 7: Problemas tipo Sturm-Liouville.
% \item Semanas 7 y 8: Operadores adjuntos y autoadjuntos.
% \item Semanas 8 y 9: Completes de una base.
% \end{enumerate}
% \end{frame}

% \subsection{Sesiones síncronas}

% \begin{frame}
% \frametitle{Sesiones síncronas de trabajo}
% Se continuará con las sesiones de trabajo los días miércoles y viernes:
% \setbeamercolor{item projected}{bg=blue!70!black,fg=yellow}
% \setbeamertemplate{enumerate items}[circle]
% \begin{enumerate}[<+->]
% \item 26 y 29 de octubre.
% \item 3 y 5 de noviembre.
% \item 10 y 12 de noviembre.
% \end{enumerate}
% \end{frame}
% \begin{frame}
% \frametitle{Sesiones adicionales}
% En caso de que se requieran, se continuará con sesiones adicionales de trabajo para la revisión y solución de ejercicios.
% \\
% \bigskip
% \pause
% Las reuniones se llevará a cabo bajo demanda el grupo.
% \end{frame}

% \section{Evaluación}
% \frame{\tableofcontents[currentsection, hideothersubsections]}
% \subsection{Ejercicios y enunciados}

% \begin{frame}
% \frametitle{Ejercicios a cuenta y enunciados}
% En la clase del día 29 de marzo se presentarán los ejercicios, así como los enunciados correspondientes del tema 3 del examen intermedio.
% \\
% \bigskip
% A continuación se detallan las fechas de entrega.
% \end{frame}
% \begin{frame}
% \frametitle{Ajuste en la fecha de entrega}
% Debido al recorrido en las actividades del curso, se modificará en esta ocasión la manera de entregar los ejercicios y el examen intermedio.
% \\
% \bigskip
% \pause
% Habrá  una primera entrega de ejercicios del Tema 3 y el Examen intermedio con enunciados que cubren hasta el contenido de ortogonalización de funciones (Gram-Schmidt)
% \end{frame}
% \begin{frame}
% \frametitle{Entrega de ejercicios}
% Los primeros ocho ejercicios para el Tema 3, se entregarán el: \pause \textbf{Martes 19 de abril a las 6 pm en Moodle}.
% \\
% \bigskip
% \pause
% Los dos ejercicios restantes, se entregarán el \pause \textbf{Martes 26 de abril a las 6 pm en Moodle}.
% \end{frame}
% \begin{frame}
% \frametitle{Examen intermedio}
% Los enunciados del  examen intermedio se indicarán en la semana 8. \pause Del Tema 3 se incluirá hasta el contenido de ortogonalización de Gram-Schmidt.
% \\
% \bigskip
% \pause
% El examen intermedio se entregará el \textbf{Martes 19 de abril a las 6 pm en Moodle}.
% \end{frame}
% \begin{frame}
% \frametitle{Ejercicio(s) pendientes} 
% El o los enunciados pendientes del tema de completes de una base (Tema 3), se entregarán el \pause \textbf{Martes 26 de abril a las 6 pm en Moodle}.
% \end{frame}

% \subsection{Examen Tarea}

% \begin{frame}
% \frametitle{Entrega del examen tarea 3}
% En la sesión del 12 de noviembre se entregarán los enunciados del examen tarea 3, \pause para enviar la solución el día viernes 26 de noviembre.
% \end{frame}
% \begin{frame}
% \frametitle{Días feriados}
% De acuerdo al calendario oficial, se presentas los días feriados:
% \pause
% \setbeamercolor{item projected}{bg=babyblue,fg=black}
% \setbeamertemplate{enumerate items}[square]
% \begin{enumerate}[<+->]
% \item Lunes 11 al viernes 15 de abril (Semana Santa)
% \end{enumerate}
% \end{frame}
% \begin{frame}
% \frametitle{De los días feriados}
% En estos días no tendremos actividades, considerando que son días oficiales de asueto.
% \\
% \bigskip
% \pause
% Regresando a las actividades el martes 19 de abril a las 3:30 pm.
% \end{frame}


% \section{Punto importante}
% \frame{\tableofcontents[currentsection, hideothersubsections]}
% \subsection{Mensaje para considerar}
% \begin{frame}
% \frametitle{Avance en el curso}
% Ya hemos concluido dos de los seis temas del curso, recordando que la primera evaluación corresponde a los tres primeros temas.
% \end{frame}
% \begin{frame}
% \frametitle{Examen-Tarea}
% Se hará el envío por correo con el archivo pdf con las preguntas del examen tarea correspondiente a los Temas 1 y 2.
% \\
% \bigskip
% \pause
% Con la finalidad de que los atiendan oportunamente, los resuelvan y \textbf{\textcolor{red}{los envíen de regreso en dos semanas}}.
% \end{frame}
% \begin{frame}
% \frametitle{Acuse de recibido}
% Se solicitará el acuse de recibido, es decir, esperamos una respuesta por parte de ustedes, de haber recibido el mensaje y el archivo adjunto.
% \\
% \bigskip
% \pause
% Esto nos dará la evidencia de que recibieron lo necesario para responder parte del primer examen parcial.
% \end{frame}
% \begin{frame}
% \frametitle{Preguntas del Tema 3}
% Conforme vayamos avanzando en el Tema 3, se les proporcionarán los ejercicios del examen-tarea de este tema.
% \\
% \bigskip
% \pause
% El punto es que deben de medir bien su carga de trabajo y aplicar un buen esfuerzo para estas actividades necesarias de evaluación.
% \end{frame}
\end{document}