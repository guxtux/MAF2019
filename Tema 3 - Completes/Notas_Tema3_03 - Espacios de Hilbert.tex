\documentclass[12pt]{article}
\usepackage[utf8]{inputenc}
\usepackage[spanish,es-lcroman, es-tabla]{babel}
\usepackage[autostyle,spanish=mexican]{csquotes}
\usepackage{amsmath}
\usepackage{amssymb}
\usepackage{nccmath}
\numberwithin{equation}{section}
\usepackage{amsthm}
\usepackage{graphicx}
\usepackage{epstopdf}
\DeclareGraphicsExtensions{.pdf,.png,.jpg,.eps}
\usepackage{color}
\usepackage{float}
\usepackage{multicol}
\usepackage{enumerate}
\usepackage[shortlabels]{enumitem}
\usepackage{anyfontsize}
\usepackage{anysize}
\usepackage{array}
\usepackage{multirow}
\usepackage{enumitem}
\usepackage{cancel}
\usepackage{tikz}
\usepackage{circuitikz}
\usepackage{tikz-3dplot}
\usetikzlibrary{babel}
\usepackage{bm}
\usepackage{mathtools}
\usepackage{esvect}
\usepackage{hyperref}
\usepackage{relsize}
\usepackage{siunitx}
\usepackage{physics}
%\usepackage{biblatex}
\usepackage{standalone}
\usepackage{mathrsfs}
\usepackage{bigints}
\usepackage{bookmark}
\spanishdecimal{.}

\setlist[enumerate]{itemsep=0mm}

\renewcommand{\baselinestretch}{1.5}

\let\oldbibliography\thebibliography

\renewcommand{\thebibliography}[1]{\oldbibliography{#1}

\setlength{\itemsep}{0pt}}
%\marginsize{1.5cm}{1.5cm}{2cm}{2cm}


\newtheorem{defi}{{\it Definición}}[section]
\newtheorem{teo}{{\it Teorema}}[section]
\newtheorem{ejemplo}{{\it Ejemplo}}[section]
\newtheorem{propiedad}{{\it Propiedad}}[section]
\newtheorem{lema}{{\it Lema}}[section]

\usepackage{standalone}
\usepackage{geometry}
\geometry{top=1.25cm, bottom=1.5cm, left=1.25cm, right=1.25cm}
%\author{M. en C. Gustavo Contreras Mayén. \texttt{curso.fisica.comp@gmail.com}}
\title{Espacios de Hilbert y notación de Dirac \\ \large {Matemáticas Avanzadas de la Física}  \vspace{-1.5\baselineskip}}
\date{}
\author{}
\begin{document}
%\renewcommand\theenumii{\arabic{theenumii.enumii}}
\renewcommand\labelenumii{\theenumi.{\arabic{enumii}}}
\maketitle
\fontsize{14}{14}\selectfont
%Referencia: Zettili - Quantum Mechanics. Concepts and Applications. 2d. Edition.
%Chapter 2. Mathematical tools of quantum mechanics.
Tratamos en este subtema con la \emph{maquinaria matemática} necesaria para estudiar la mecánica cuántica. Aunque esta revisión tiene un alcance matemático, no se intenta que sea matemáticamente completo o riguroso. Nos limitamos a los temas prácticos que son relevantes para el formalismo de la mecánica cuántica.
\par
La ecuación de Schrödinger es una de las piedras angulares de la teoría de la mecánica cuántica; Tiene la estructura de una ecuación lineal. El formalismo de la mecánica cuántica trata con operadores que son lineales y funciones de onda que pertenecen a un \textbf{espacio abstracto de Hilbert}. Las propiedades matemáticas y la estructura de los espacios de Hilbert son esenciales para una comprensión adecuada del formalismo de la mecánica cuántica.
\par
Para esto, vamos a revisar brevemente las propiedades de los espacios de Hilbert y las de los operadores lineales. Luego consideraremos la notación bra-ket de Dirac.
\par
Schrödinger y Heisenberg formularon la mecánica cuántica de dos maneras diferentes. La mecánica ondulatoria de Schrödinger y la mecánica matricial de Heisenberg son las representaciones del formalismo general de la mecánica cuántica en sistemas de base continua y discreta, respectivamente. Para esto, también examinaremos las matemáticas involucradas en la representación de kets, bras, bra-kets y operadores en bases discretas y continuas.
\section{El espacio de Hilbert y funciones de onda.}
\subsection{El espacio vectorial lineal.}
Un espacio vectorial lineal consiste de dos conjuntos de elementos y dos reglas algebraicas:
\begin{enumerate}
\item Un conjunto de vectores $\psi, \phi, \chi, \ldots$ y un conjunto de escalares $a, b, c, \ldots$
\item Una regla de suma de vectores y una regla de multiplicación por escalares.
\end{enumerate}
\textbf{(a) Regla de la suma.}

La regla se suma tiene las propiedades y estructura de un grupo abeliano:
\begin{enumerate}
\item Si $\psi$ y $\phi$ son vectores (elementos) de un espacio, su suma $\psi + \phi$, es también un vector del mismo espacio.
\item Conmutatividad: $\psi + \phi = \phi + \psi$
\item Asociatividad: $(\psi + \phi) + \chi = \psi + (\phi + \chi)$
\item La existencia de un vector cero o vector neutro: para cada vector $\psi$, debe de existir un vector cero $0$, tal que $0 + \psi = \psi + 0 = \psi$
\item La existencia de un vector simétrico o vector inverso: cada vector $\psi$ debe de tener un vector simétrico $(-\psi)$, tal que $\psi + (-\psi) = (-\psi) + \psi = 0$
\end{enumerate}
\textbf{(b) Regla de multiplicación.}

La multiplicación de vectores por escalares (que pueden ser número reales o complejos), tiene las siguientes propiedades:
\begin{enumerate}
\item El producto de un escalar por un vector, genera otro vector. En general, si $\psi$ y $\phi$ son dos vectores de ese espacio, cualquier combinación lineal $a \, \psi + b \, \phi$, es también un vector del espacio, $a$ y $b$ siendo escalares.
\item Distribución con respecto a la suma:
\begin{equation}
a (\psi + \phi) = a \, \psi + a \, \phi \hspace{1.5cm} (a + b) \, \psi = a \, \psi + b \, \psi
\label{eq:ecuacion_02_01}
\end{equation}
\item Asociación con respecto a la multiplicación de escalares:
\begin{equation}
a (b \, \psi) = (a \, b) \, \psi
\label{eq:ecuacion_02_02}
\end{equation}
\item Para cada elemento $\psi$, existe al menos un escalar unitario $I$, y un cero escalar $o$ tal que
\begin{equation}
I \, \psi = \psi \, I = \psi \hspace{1.5cm} \mbox{ y } \hspace{1cm} o \, \psi = \psi \, o = o
\label{eq:ecuacion_02_03}
\end{equation}
\end{enumerate}
\subsection{El espacio de Hilbert.}
Un espacio de Hilbert $\mathcal{H}$ es un conjunto de vectores $\psi, \phi, \chi, \ldots$ y un conjunto de escalares $a, b, c, \ldots$ \emph{el cual satisface las siguientes cuatro propiedades}:

\textbf{(1) $\mathcal{H}$ es un espacio lineal.}

Debe de cumplir con las propiedades de un espacio lineal mencionadas previamente.

\textbf{(2) $\mathcal{H}$ tiene definido un producto escalar que es estrictamente positivo.}

El producto escalar de un elemento $\psi$ con otro elemento $\phi$, es en general un número complejo descrito por $(\psi, \phi)$, donde $(\psi, \phi) =$ número complejo. \textbf{Nota: } Cuidado con el orden! Ya que el producto escalar es un número complejo, la cantidad  $(\psi, \phi)$ en general no es igual a  $(\psi, \phi) : (\phi, \psi) = \phi^{*} \, \psi$ mientras que $(\psi, \phi) = \phi^{*} \, \psi$. El producto escalar satisface las siguientes propiedades:
\begin{enumerate}[label=\roman*)]
\item El producto escalar de $\psi$ con $\phi$ es igual al complejo conjugado del producto escalar de $\phi$ con $\psi$:
\begin{equation}
(\psi, \phi) = (\phi, \psi)^{*}
\label{eq:ecuacion_02_04}
\end{equation}

\end{enumerate}
\end{document}