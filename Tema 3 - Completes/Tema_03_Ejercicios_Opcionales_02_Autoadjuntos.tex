\documentclass[12pt]{article}
\usepackage[left=0.25cm,top=1cm,right=0.25cm,bottom=1cm]{geometry}
\textwidth = 20cm
\hoffset = -1cm
\usepackage[utf8]{inputenc}
\usepackage[spanish,es-tabla]{babel}
\usepackage[autostyle,spanish=mexican]{csquotes}
\usepackage[tbtags]{amsmath}
\usepackage{nccmath}
\usepackage{amsthm}
\usepackage{amssymb}
\usepackage{graphicx}
\usepackage{standalone}
\usepackage[outdir=./]{epstopdf}
\usepackage{siunitx}
\usepackage{physics}
\usepackage{color}
\usepackage{float}
\usepackage{multicol}
%\usepackage{milista}
\usepackage{enumitem}
\usepackage{anyfontsize}
\usepackage{anysize}
\usepackage{enumitem}
\usepackage{capt-of}
\usepackage{bm}
\usepackage{relsize}
\usepackage{placeins}
\usepackage{empheq}
\usepackage{cancel}
\usepackage{wrapfig}
\spanishdecimal{.}
\renewcommand{\baselinestretch}{1.5} 
\renewcommand\labelenumii{\theenumi.{\arabic{enumii}}}
\newcommand{\ptilde}[1]{\ensuremath{{#1}^{\prime}}}
\newcommand{\stilde}[1]{\ensuremath{{#1}^{\prime \prime}}}
\newcommand{\ttilde}[1]{\ensuremath{{#1}^{\prime \prime \prime}}}
\newcommand{\ntilde}[2]{\ensuremath{{#1}^{(#2)}}}


\title{Ejercicios opcionales \\[0.3em]  \large{Operadores autoadjuntos} \vspace{-3ex}}
\author{M. en C. Gustavo Contreras Mayén}
\date{ }

\begin{document}
\vspace{-4cm}
\maketitle
\fontsize{14}{14}\selectfont

\noindent
%Ref. Riley 2006 - 17.7
\textbf{Ejercicio opcional (11).}
Considere el conjunto de funciones, $\left\{ f (x) \right\}$, de variable real $x$, definida en el intervalo $-\infty < x < \infty$, que $\to 0$ al menos tan rápidamente como $x^{-1}$ cuando $x \to \pm \infty$. Con la función de peso unitaria, determina si cada uno de los siguientes operadores lineales es autoadjunto (Hermitiano) cuando actúa sobre $\left\{ f (x) \right\}$:
\begin{enumerate}[label=\alph*)]
\item $\displaystyle \dv{x} + x$
\item $\displaystyle - i \, \dv{x} + x^{2}$
\item $\displaystyle i \, x \, \dv{x}$
\item $\displaystyle i \, \dv[3]{x}$
\end{enumerate}
\noindent
%Ref. Arfken 2006 - 10.3.6
\textbf{Ejercicio opcional (12).} Usando la ortogonalización de Gram-Schmidt construye los primeros tres polinomios de Chebyshev de tipo I, con:
\begin{align*}
u_{n} (x) = x^{n} \hspace{1cm} n = 0, 1, 2, \ldots, \hspace{1cm} -1 \leq x \leq 1, \hspace{1cm} \omega(x) = (1 - x^{2})^{-\frac{1}{2}}
\end{align*}
Considera la normalización:
\begin{align*}
\scaleint{5ex}_{\bs -1}^{1} \, T_{m}(x) \, T_{n}(x) \, \omega(x) \dd{x} = \delta_{mn} \, \begin{cases}
\pi & m = n = 0 \\
\dfrac{\pi}{2} & m = n \geq 1
\end{cases}
\end{align*}



\end{document}