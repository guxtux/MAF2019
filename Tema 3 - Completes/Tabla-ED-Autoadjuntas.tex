\documentclass[12pt,landscape]{article}
\usepackage[utf8]{inputenc}
\usepackage[letterpaper, margin=0.5cm, footskip=-0.5cm]{geometry}
%\usepackage{anysize}
%\marginsize{1cm}{1cm}{1cm}{1cm}
\usepackage[spanish,es-lcroman, es-tabla]{babel}
\usepackage[autostyle,spanish=mexican]{csquotes}
\usepackage{amsmath}
\usepackage{amssymb}
\usepackage{nccmath}
\numberwithin{equation}{section}
\usepackage{amsthm}
\usepackage{graphicx}
\usepackage[outdir=./]{epstopdf}
\DeclareGraphicsExtensions{.pdf,.png,.jpg,.eps}
\usepackage{color}
\usepackage{float}
\usepackage{fancyhdr}
\usepackage{multicol}
\usepackage{enumerate}
\usepackage[shortlabels]{enumitem}
\usepackage{anyfontsize}
\usepackage{anysize}
\usepackage{array}
\usepackage{multirow}
\usepackage{enumitem}
\usepackage{cancel}
\usepackage{nameref}
\usepackage{pdflscape}
\usepackage{makecell}
\usepackage{longtable}
\usepackage{pgfplots}
\pgfplotsset{compat=1.12}
\usepackage{tikz}
\usepackage{circuitikz}
\usepackage{tikz-3dplot}
\usepackage{caption}
\usepackage{bm}
\usepackage{mathtools}
\usepackage{esvect}
\usepackage{hyperref}
\usepackage{relsize}
\usepackage{siunitx}
\usepackage{physics}
%\usepackage[backend=biber]{biblatex}
\usepackage{standalone}
\usepackage{mathrsfs}
\usepackage{bigints}
\usepackage{bookmark}
%Quita el número de la página
\pagenumbering{gobble}
\spanishdecimal{.}
%\setlength{\voffset}{-0.75in}
\author{}
\date{ }
\usepackage[flushleft]{threeparttable}
\title{Ecuaciones Diferenciales de la Física Matemática \\ {\large Tema 3 - Completez - Curso MAF}}
\begin{document}
\maketitle
\fontsize{14}{14}\selectfont
\addtolength{\voffset}{-2cm}
\vspace{-2cm}
\begin{table}[!ht]
\centering
{\setlength\extrarowheight{1.5pt}
{\renewcommand{\arraystretch}{1.5}%
\caption{Parámetros y coeficientes de ecuaciones de la Física Matemática con valores propios.}
\begin{threeparttable}
\begin{tabular}{p{6cm} c c c c }
\hline
\makecell{Ecuación} & $p(x)$ & $q(x)$ & $\lambda$ & $w(x)$ \\ \hline
Legendre$^{a}$ & $1 - x^{2}$ & 0 & $\ell (\ell + 1)$ & 1  \\
Asociados de Legendre & $1 - x^{2}$ & 0 & $\ell (\ell + 1)$ & 1  \\
Chebychev I & $(1 - x^{2})^{1/2}$ & $0$ & $n^{2}$ & $(1 - x^{2})^{-1/2}$ \\
Chebychev II & $(1 - x^{2})^{3/2}$ & $0$ & $n^{2}$ & $(1 - x^{2})^{-1/2}$ \\
Ultraesféricos & $(1 - x^{2})^{\alpha + 1/2}$ & 0 & $n(n + 2 \, \alpha)$ & $(1 - x^{2})^{\alpha -1/2}$ \\
Bessel$^{b}$, en $0 \leq x \leq a$ & $x$ & $- \dfrac{n^{2}}{x}$ & $a^{2}$ & $x$ \\
Laguerre, en $0 \leq x < \infty$ & $x \; e^{-x}$ & $0$ & $\alpha$ & $e^{-x}$ \\
Asociados de Laguerre$^{c}$ & $x^{k + 1} \; e^{-x}$ & $0$  & $\alpha - k$ & $x^{k} \; e^{-x}$ \\
Hermite, en $0 \leq x < \infty$ & $e^{-x^{2}}$ & $0$ & $2 \alpha$ & $e^{-x^{2}}$ \\
Oscilador armónico simple & $1$ & $0$ & $n^{2}$ & $1$
\end{tabular}
%}}
\begin{tablenotes}
\small
\item $^{a} \: \ell = 0, 1, 2, \ldots, -\ell \leq m < \ell$ son enteros.
\item $^{b} \:$  La ortogonalidad de las funciones de Bessel es bastante especial, como se verá en el desarrollo del Tema 3.
\item $^{c} \: k$ es un entero no negativo.  
\end{tablenotes}
\end{threeparttable}}}
\end{table}
\newpage
\begin{table}[!ht]
\centering
{\setlength\extrarowheight{1.5pt}
{\renewcommand{\arraystretch}{1.5}%
\caption{Elección del intervalo de ortogonalidad $[a,b]$ y del factor de peso $\omega(x)$.}
\begin{threeparttable}
\begin{tabular}{p{5cm} c c c}
\hline
\makecell{Ecuación} & $a$ & $b$ & $\omega(x)$ \\ \hline
Legendre & $-1$ & $1$ & $1$ \\
Asociados de  Legendre & $-1$ & $1$ & $1$ \\
Chebychev I & $-1$ & $1$ & $(1-x^{2})^{-1/2}$ \\
Chebychev II & $-1$ & $1$ & $(1-x^{2})^{1/2}$ \\
Laguerre & $0$ & $\infty$ & $e^{-x}$ \\
Asociados de Laguerre & $0$ & $\infty$ & $x^{k} e^{-x}$ \\
Hermite & $-\infty$ & $\infty$ & $e^{-x^{2}}$ \\
Oscilador armónico & $0$ & $2 \pi$ & $1$ \\
    & $-\pi$ & $\pi$ & $1$ 
\end{tabular}
\begin{tablenotes}
\small
\item $1.$ El intervalo de ortogonalidad $[a, b]$ está determinado por las CDF.
\item $2.$ La función de peso se presenta a modo de que la EDO queda en una forma auto-adjunta.
\end{tablenotes}
\end{threeparttable}}}
\end{table}
\end{document}