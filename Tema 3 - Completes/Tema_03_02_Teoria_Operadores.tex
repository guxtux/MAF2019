\documentclass[12pt]{beamer}
\usepackage{../Estilos/BeamerMAF}
\usetheme{Dresden}
\usecolortheme{seahorse}
%\useoutertheme{default}
\setbeamercovered{invisible}
% or whatever (possibly just delete it)
\setbeamertemplate{section in toc}[sections numbered]
\setbeamertemplate{subsection in toc}[subsections numbered]
\setbeamertemplate{subsection in toc}{\leavevmode\leftskip=3.2em\rlap{\hskip-2em\inserttocsectionnumber.\inserttocsubsectionnumber}\inserttocsubsection\par}
\setbeamercolor{section in toc}{fg=blue}
\setbeamercolor{subsection in toc}{fg=blue}
\setbeamercolor{frametitle}{fg=blue}
\setbeamertemplate{caption}[numbered]

\setbeamertemplate{footline}
\beamertemplatenavigationsymbolsempty
\setbeamertemplate{headline}{}

\makeatletter
\setbeamercolor{section in foot}{bg=gray!30, fg=black!90!orange}
\setbeamercolor{subsection in foot}{bg=blue!30!yellow, fg=red}
\setbeamertemplate{footline}
{
  \leavevmode%
  \hbox{%
  \begin{beamercolorbox}[wd=.333333\paperwidth,ht=2.25ex,dp=1ex,center]{section in foot}%
    \usebeamerfont{section in foot} \insertsection
  \end{beamercolorbox}}%
  \begin{beamercolorbox}[wd=.333333\paperwidth,ht=2.25ex,dp=1ex,center]{subsection in foot}%
    \usebeamerfont{subsection in foot}  \insertsubsection
  \end{beamercolorbox}%
  \begin{beamercolorbox}[wd=.333333\paperwidth,ht=2.25ex,dp=1ex,right]{date in head/foot}%
    \usebeamerfont{date in head/foot} \insertshortdate{} \hspace*{2em}
    \insertframenumber{} / \inserttotalframenumber \hspace*{2ex} 
  \end{beamercolorbox}}%
  \vskip0pt%
\makeatother 

\makeatletter
\patchcmd{\beamer@sectionintoc}{\vskip1.5em}{\vskip0.8em}{}{}
\makeatother


\setbeamercolor{section in foot}{bg=darkspringgreen, fg=white}
\setbeamercolor{subsection in foot}{bg=persianblue, fg=white}
\setbeamercolor{date in foot}{bg=goldenrod, fg=white}

\makeatletter
\setbeamertemplate{footline}
{
\leavevmode%
\hbox{%
\begin{beamercolorbox}[wd=.333333\paperwidth,ht=2.25ex,dp=1ex,center]{section in foot}%
  \usebeamerfont{section in foot} \insertsection
\end{beamercolorbox}%
\begin{beamercolorbox}[wd=.333333\paperwidth,ht=2.25ex,dp=1ex,center]{subsection in foot}%
  \usebeamerfont{subsection in foot}  \insertsubsection
\end{beamercolorbox}%
\begin{beamercolorbox}[wd=.333333\paperwidth,ht=2.25ex,dp=1ex,right]{date in head/foot}%
  \usebeamerfont{date in head/foot} \insertshortdate{} \hspace*{1.5em}
  \insertframenumber{} / \inserttotalframenumber \hspace*{2ex} 
\end{beamercolorbox}}%
\vskip0pt%
}
\makeatother
\usefonttheme{serif}
\resetcounteronoverlays{saveenumi}

\date{29 de marzo de 2022}

\title{\large{Teoría Operadores Sturm-Liouville}}
\subtitle{Tema 3 - Bases completas y ortogonales}
\author{M. en C. Gustavo Contreras Mayén}


\begin{document}
\maketitle
\fontsize{14}{14}\selectfont
\spanishdecimal{.}

\section*{Contenido}
\frame[allowframebreaks]{\tableofcontents[currentsection, hideallsubsections]}

\section{Operadores autoadjuntos}
\frame{\tableofcontents[currentsection, hideothersubsections]}
\subsection{Introducción}

\begin{frame}
\frametitle{Relevancia de los operadores}
En el estudio de la teoría espectral de matrices, se aprende sobre el adjunto de la matriz, $\vb{A}^{\dagger}$, y el papel que juegan las matrices autoadjuntas o Hermitianas en la diagonalización.
\end{frame}
\begin{frame}
\frametitle{Nuevo concepto}
Además, se necesita el concepto del \emph{adjunto} para discutir la existencia de soluciones al problema matricial:
\begin{align*}
\vb{y} = \vb{A} \, \vb{x}
\end{align*}
\end{frame}
\begin{frame}
\frametitle{Operadores y ED}
En el mismo sentido, uno está interesado en la existencia de soluciones de la ecuación del operador $L \, u = f$ y soluciones del correspondiente problema de valores propios.
\\
\bigskip
\pause
El estudio de operadores lineales en un espacio de Hilbert es una generalización de lo que estudia en un curso de álgebra lineal.
\end{frame}
\begin{frame}
\frametitle{Retomando el operador Sturm-Liouville}
Así como se puede encontrar una base de vectores propios y diagonalizar matrices Hermitianas o autoadjuntas (o simétricas reales en el caso de matrices reales), veremos que el operador de Sturm-Liouville es \emph{autoadjunto}.
\end{frame}
\begin{frame}
\frametitle{Retomando el operador Sturm-Liouville}    
En esta parte definiremos el \emph{dominio} de un operador e introduciremos la noción de \emph{operadores adjuntos}.
\\
\bigskip
\pause
Veremos el papel que juega el adjunto en la existencia de soluciones a la ecuación del operador $L \, u = f$.
\end{frame}

\subsection{El operador adjunto}

\begin{frame}
\frametitle{Definición}
Comenzamos definiendo el adjunto de un operador: \pause el adjunto, $L^{\dagger}$, del operador $L$ satisface:
\begin{align*}
\langle u, L \, v \rangle = \langle L^{\dagger} \, u,  v \rangle
\end{align*}
para todo $v$ en el dominio de $L$ y $u$ en el dominio de $L^{\dagger}$.
\end{frame}
\begin{frame}
\frametitle{El dominio del operador}
Aquí, el dominio de un operador diferencial $L$ es el conjunto de todos $u \in L_{\sigma}^{2} (a, b)$ que satisfacen un conjunto dado de CDF homogéneas.
\\
\bigskip
\pause
Esto se comprenderá mejor con el siguiente ejemplo.
\end{frame}
\begin{frame}
\frametitle{Ejemplo de operador adjunto}
Encuentra el adjunto de:
\begin{align*}
L = a_{2}(x) \, D^{2} + a_{1}(x) \, D + a_{0}(x)
\end{align*}
con $D = \dv*{x}$.
\end{frame}
\begin{frame}
\frametitle{Resolviendo el ejemplo de operador adjunto}    
Para encontrar el adjunto, colocamos el operador dentro de una integral. \pause Consideremos el producto interior:
\pause
\begin{align*}
\langle u , L \, v \rangle = \scaleint{5ex}_{\bs a}^{b} u (a_{2} \, \sderivada{v} + a_{1} \, \pderivada{v} + a_{0} \, v) \dd{x}
\end{align*}
\end{frame}
\begin{frame}
\frametitle{Resolviendo el ejemplo}
Tenemos que \enquote{mover} el operador $L$ de $v$ \pause y determinar qué operador está actuando sobre $u$ para preservar formalmente el producto interno.
\\
\bigskip
\pause
\end{frame}
\begin{frame}
\frametitle{Resolviendo el ejemplo}    
Para un operador simple como $L = \dv*{x}$, esto se hace fácilmente mediante la integración por partes.
\\
\bigskip
\pause
Para el operador dado en el ejemplo, necesitaremos aplicar varias integraciones por partes a los términos individuales. Consideramos cada término derivado en el integrando por separado.
\end{frame}
\begin{frame}
\frametitle{Integrando el término $a_{1}$}
Para el término $a_{1} \pderivada{v}$, integramos por partes para encontrar:
\pause
\begin{eqnarray}
\begin{aligned}[b]
\scaleint{5ex}_{\bs a}^{b} \, u(x) \, &a_{1} \, \pderivada{v}(x) \dd{x} = a_{1}(x) \, u(x) \, v(x) \eval_{a}^{b} + \\[0.5em]
&- \scaleint{5ex}_{\bs a}^{b} \big[ u(x) \, a_{1} (x) \big]^{\prime} \, v(x) \dd{x}
\end{aligned}
\label{eq:ecuacion_04_17}
\end{eqnarray}
\end{frame}
\begin{frame}
\frametitle{Integrando el término $a_{2}$}
Ahora consideremos el caso para el término $a_{2} \, \sderivada{v}$, en donde será necesario hacer dos integraciones por partes:
\pause
\begin{eqnarray*}
\begin{aligned}[b]
&\scaleint{5ex}_{\bs a}^{b} \, u(x) \, a_{2} (x) \, \sderivada{v}(x) \dd{x} = a_{2}(x) \, u(x) \, \pderivada{v}(x) \eval_{a}^{b} + \\[0.5em]
&- \scaleint{5ex}_{\bs a}^{b} \big[ u(x) \, a_{2} (x) \big]^{\prime} \, \pderivada{v}(x) \dd{x} = \end{aligned}
\end{eqnarray*}
\end{frame}
\begin{frame}
\frametitle{Integrando el término $a_{2}$}
\begin{eqnarray}
\begin{aligned}[b]
&= \bigg[ a_{2}(x) \, u(x) \, \pderivada{v}(x) - \big[ a_{2}(x) \, u(x) \big]^{\prime} \, v(x) \bigg] \eval_{a}^{b} + \\[0.5em]
&+ \scaleint{5ex}_{\bs a}^{b} \, \big[ u(x) \, a_{2} (x) \big]^{\prime \prime} \, v(x) \dd{x}
\end{aligned}
\label{eq:ecuacion_04_18}
\end{eqnarray}
\end{frame}    
\begin{frame}
\frametitle{Resultado premilinar}
Combinando estos resultados, tenemos que:
\pause
\begin{eqnarray}
\begin{aligned}[b]
\langle u , L \, v \rangle &= \pause \scaleint{5ex}_{\bs a}^{b} u (a_{2} \, \sderivada{v} + a_{1} \, \pderivada{v} + a_{0} \, v) \dd{x} = \\[0.5em] \pause
&= \bigg[ a_{1}(x) \, u(x) \, v(x) + a_{2}(x) \, u(x) \, \pderivada{v}(x) + \\[0.5em]
&- \big[ a_{2}(x) \, u(x) \big]^{\prime} \, v(x) \bigg] \eval_{a}^{b} + \\[0.5em]
&+ \scaleint{5ex}_{\bs a}^{b} \, \big[ u(x) \, a_{2} (x) \big]^{\prime \prime} \, v(x) \dd{x}
\end{aligned}
\label{eq:ecuacion_04_19}
\end{eqnarray}
\end{frame}
\begin{frame}
\frametitle{Usando las CDF}
Agregando las CDF para $v$, \pause uno tiene que determinar las CDF para $u$ tales que:
\pause
\begin{align*}
&\bigg[ a_{1}(x) \, u(x) \, v(x) + a_{2}(x) \, u(x) \, \pderivada{v}(x) + \\[0.5em]
&- \big[ a_{2}(x) \, u(x) \big]^{\prime} \, v(x) \bigg] \eval_{a}^{b} = 0
\end{align*}
\end{frame}
\begin{frame}
\frametitle{Usando las CDF}
Estos nos lleva a:
\pause
\begin{align*}
\langle u , L \, v \rangle &= \scaleint{5ex}_{\bs a}^{b}  \big[ (a_{2} \, u)^{\prime \prime} - (a_{1} \, u)^{\prime} + a_{0} \, u \big] \, v \dd{x} \\[0.5em]
&\equiv \langle L^{\dagger} \, u, v \rangle 
\end{align*}
\end{frame}
\begin{frame}
\frametitle{El operador adjunto}
Por lo tanto:
\pause
\begin{align}
L^{\dagger} = a_{2}(x) \, \dv[2]{x} - a_{1}(x) \, \dv{x} + a_{0}(x)
\label{eq:ecuacion_04_20}
\end{align}
\end{frame}
\begin{frame}
\frametitle{El operador Hermitiano/Hermítico}
Cuando $L^{\dagger} = L$, \pause el operador se llama formalmente \textbf{autoadjunto}, también es conocido como \emph{operador Hermitiano}.
\\
\bigskip
\pause
Cuando el dominio de $L$ es el mismo que el dominio de $L^{\dagger}$, se utiliza el término autoadjunto.
\end{frame}
\begin{frame}
\frametitle{Ejemplo 2}
Determina $L^{\dagger}$ y su dominio para el operador:
\pause
\begin{align*}
L \, u = \dv{u}{x}
\end{align*}
donde $u$ satisface las CDF $u(0) = 2 \, u(1)$ en $[0, 1]$.
\end{frame}
\begin{frame}
\frametitle{Resolviendo el ejercicio 2}
Necesitamos encontrar el operador adjunto que satisfaga $\langle v, L \, u \rangle = \langle L^{\dagger} \, v, u \rangle$.
\\
\bigskip
\pause  
Por lo que reescribimos la integral:
\pause
\begin{eqnarray*}
\langle v, L \, u \rangle = \pause \scaleint{5ex}_{\bs 0}^{1} v \, \dv{u}{x} \dd{x} = \pause u \, v \eval_{0}^{1} - \scaleint{5ex}_{\bs 0}^{1} u \, \dv{v}{x} \dd{x} = \pause \langle L^{\dagger} \, v, u \rangle
\end{eqnarray*}
\end{frame}
\begin{frame}
\frametitle{El problema adjunto}
De aquí tenemos que el problema adjunto que consiste en un operador adjunto y la CDF asociada (o dominio de $L^{\dagger}$):
\pause
\setbeamercolor{item projected}{bg=blue!70!black,fg=yellow}
\setbeamertemplate{enumerate items}[circle]
\begin{enumerate}[<+->]
\item $L^{\dagger} = - \dv{x}$
\item $\displaystyle u \, v \eval_{0}^{1} = 0 \Rightarrow u(1) \big[ v(1) - 2 \, v(0) \big] \Rightarrow v(1) = 2 \, v(0)$
\end{enumerate}
\end{frame}

% \vspace{0.3cm}
% \noindent
% %Ref. Arfken (2006) 10.2.1
% \textbf{Ejercicio a cuenta (30).} Las funciones $\phi_{1}(x)$ y $\phi_{2}(x)$ son funciones propias del mismo operador autoadjunto (Hermitiano) pero para distintos valores propios $\lambda_{1}$ y $\lambda_{2}$. Demuestra que $\phi_{1}(x)$ y $\phi_{2}(x)$ son linealmente independientes.

\section{Identidades de Lagrange y de Green}
\frame{\tableofcontents[currentsection, hideothersubsections]}
\subsection{Identidades de apoyo}

\begin{frame}
\frametitle{Material importante}
Antes de pasar a la demostración de que los valores propios de un problema de Sturm-Liouville son reales y las funciones propias asociadas son ortogonales, \pause necesitaremos introducir dos identidades importantes.
\end{frame}
\begin{frame}
\frametitle{Con el operador Sturm-Liouville}
Para el operador de Sturm-Liouville:
\pause
\begin{align*}
\mathcal{L} = \dv{x} \left( p \, \dv{x} \right) + q
\end{align*}
\end{frame}
\begin{frame}
\frametitle{Identidad de Lagrange}
Se tienen dos identidades:
\\
\bigskip
\pause
\setbeamercolor{item projected}{bg=blue!70!black,fg=yellow}
\setbeamertemplate{enumerate items}[circle]
\begin{enumerate}[<+->]
\item \textbf{Identidad de Lagrange:} 
\begin{align*}
u \, \mathcal{L} \, v - v \, \mathcal{L} \, u = \big[ p \, (u \, \pderivada{v} - v \, \pderivada{u}) \big]^{\prime}
\end{align*}
\seti
\end{enumerate}
\end{frame}
\begin{frame}
\frametitle{Identidad de Green}
\setbeamercolor{item projected}{bg=blue!70!black,fg=yellow}
\setbeamertemplate{enumerate items}[circle]
\begin{enumerate}[<+->]    
\conti
\item \textbf{Identidad de Green:} 
\begin{align*}
\scaleint{5ex}_{\bs a}^{b} \, \big(u \, \mathcal{L} \, v - v \, \mathcal{L} \, u \big) \dd{x} = \big[ p \, (u \, \pderivada{v} - v \, \pderivada{u}) \big]\eval_{a}^{b}
\end{align*}
\end{enumerate}
\end{frame}
\begin{frame}
\frametitle{Demostrando la identidad de Lagrange}
La demostración de la identidad de Lagrange se sigue una sencilla manipulación del operador:
\pause
\begin{eqnarray*}
\begin{aligned}[b]
u  \mathcal{L} v {-} v \mathcal{L} u &= u \bigg[ \dv{x} \left( p \dv{v}{x} \right) {+} q v \bigg] {-} v \bigg[ \dv{x} \left( p \dv{u}{x} \right) {+} q u \bigg] = \\[0.5em] \pause
&= u \dv{x} \left( p \dv{v}{x} \right) - v \dv{x} \left( p \dv{u}{x} \right) = \\[0.5em]
\end{aligned}
\end{eqnarray*}
\end{frame}
\begin{frame}
\frametitle{Demostrando la identidad de Lagrange}
\begin{eqnarray}
\begin{aligned}[b]
&= u \dv{x} \left( p \dv{v}{x} \right) + p \dv{u}{x} \dv{v}{x} + \\[0.5em]
&- v \dv{x} \left( p \dv{u}{x} \right) - p \dv{u}{x} \dv{v}{x} = \\[0.5em] \pause
&= \dv{x} \bigg[ p \, u \, \dv{v}{x} - p \, v \, \dv{u}{x}  \bigg]
\end{aligned}
\label{eq:ecuacion_04_21}
\end{eqnarray}
La identidad de Green se prueba simplemente integrando la identidad de Lagrange.
\end{frame}    

\section{Ortogonalidad y eigenvalores reales}
\frame{\tableofcontents[currentsection, hideothersubsections]}
\subsection{Eigenvalores reales}

\begin{frame}
\frametitle{Avance en el contenido}
Ahora estamos listos para demostrar que:
\setbeamercolor{item projected}{bg=blue!70!black,fg=yellow}
\setbeamertemplate{enumerate items}[circle]
\begin{enumerate}[<+->]
\item  \emph{Los eigenvalores de un problema de Sturm-Liouville son reales}.
\item \emph{Las eigenfunciones correspondientes son ortogonales}.
\end{enumerate}
\end{frame}
\begin{frame}
\frametitle{Problema de eigenvalores}
Los eigenvalores del problema de tipo Sturm-Liouville son reales:
\pause
\begin{align*}
\mathcal{L} \, y = \left( x \, \pderivada{y} \right)^{\prime} + \dfrac{2}{x} \, y = - \lambda \, \sigma \, y
\end{align*}
\end{frame}
\begin{frame}
\frametitle{Demostrando el punto}
Sean las $\phi_{n}(x)$ una solución para el problema de eigenvalores asociados con $\lambda_{n}$:
\pause
\begin{align*}
\mathcal{L} \, \phi_{n} = - \lambda_{n} \, \sigma \, \phi_{n}
\end{align*}
\end{frame}
\begin{frame}
\frametitle{Usando el conjugado complejo}
Mostrando que $\overline{\lambda}_{n} = \lambda_{n}$, donde la barra significa el conjugado complejo.
\\
\bigskip
\pause
Entonces, también consideramos el conjugado complejo de esta ecuación:
\begin{align*}
\mathcal{L} \, \overline{\phi}_{n} = - \overline{\lambda}_{n} \, \sigma \, \overline{\phi}_{n}
\end{align*}
\end{frame}
\begin{frame}
\frametitle{Multiplicando por el conjugado}
Multiplicando la primera ecuación por $\overline{\phi}_{n}$, la segunda ecuación por $\phi_{n}$ y luego restando los resultados, obtenemos:
\pause
\begin{align*}
\overline{\phi}_{n} \, \mathcal{L} \, \phi_{n} - \phi_{n} \, \mathcal{L} \, \overline{\phi}_{n} = \big( \overline{\lambda}_{n} - \lambda_{n} \big) \, \sigma \, \phi_{n} \, \overline{\phi}_{n}
\end{align*}
\end{frame}
\begin{frame}
\frametitle{Integrando la expresión}
Integrando ambos lados de la expresión, se llega a:
\pause
\begin{align*}
\scaleint{5ex}_{\bs a}^{b} \bigg( \overline{\phi}_{n} \, \mathcal{L} \, \phi_{n} &- \phi_{n} \mathcal{L} \overline{\phi}_{n} \bigg) \dd{x} = \\[0.5em]
&=\big( \overline{\lambda}_{n} {-} \lambda_{n} \big) \scaleint{5ex}_{\bs a}^{b} \sigma \phi_{n} \overline{\phi}_{n} \dd{x}
\end{align*}
\end{frame}
\begin{frame}
\frametitle{Usando la identidad de Green}
Aplicando la identidad de Green en el lado izquierdo, se tiene que:
\pause
\begin{align*}
\big[ p \, (u \, \pderivada{v} - v \, \pderivada{u}) \big]\eval_{a}^{b} = \big( \overline{\lambda}_{n} - \lambda_{n} \big) \, \scaleint{5ex}_{\bs a}^{b} \sigma \, \phi_{n} \, \overline{\phi}_{n} \dd{x}
\end{align*}
\end{frame}
\begin{frame}
\frametitle{Usando condiciones homogéneas}
Usando las condiciones homogéneas:
\pause
\begin{align*}
\alpha_{1} \, y(a) + \beta_{1} \, \ptilde{y} (a) &= 0 \\[0.5em]
\alpha_{2} \, y(b) + \beta_{2} \, \ptilde{y} (b) &= 0
\end{align*}
para el operador autoadjunto, el lado izquierdo se anula.
\end{frame}
\begin{frame}
\frametitle{Resultado obtenido}
Por lo que el resultado es:
\pause
\begin{align*}
\big( \overline{\lambda}_{n} - \lambda_{n} \big) \, \scaleint{5ex}_{\bs a}^{b} \sigma \, \norm{\phi_{n}}^{2} \dd{x} = 0
\end{align*}
esta integral es no negativa.
\end{frame}
\begin{frame}
\frametitle{Sobre los eigenvalores}
Por lo que se tiene $\overline{\lambda}_{n} = \lambda_{n}$.
\\
\bigskip
\pause
 Entonces los eigenvalores son reales.
\end{frame}

\subsection{Eigenfunciones ortogonales}

\begin{frame}
\frametitle{Por demostrar}
Ahora nos interesa revisar que las funciones propias correspondientes a diferentes valores propios de un problema tipo Sturm-Liouville son ortogonales.
\pause
\begin{align*}
\dv{x} \left( p(x) \, \dv{x} \right) + q(x) \, y + \lambda \, \sigma (x) \, y = 0
\end{align*}
\end{frame}
\begin{frame}
\frametitle{Demostración de esta propiedad}
La demostración es similar al ejemplo anterior. Sea $\phi_{n}(x)$ una solución al problema de valores propios con $\lambda_{n}$:
\pause
\begin{align*}
\mathcal{L} \, \phi_{n} = - \lambda_{n} \, \sigma \, \phi_{n}
\end{align*}
\end{frame}
\begin{frame}
\frametitle{Demostración de esta propiedad}
Y sea $\phi_{m}(x)$ una solución al problema de valores propios asociado con $\lambda_{m} \neq \lambda_{n}$:
\pause
\begin{align*}
\mathcal{L} \, \phi_{m} = - \lambda_{m} \, \sigma \, \phi_{m}
\end{align*}
\end{frame}
\begin{frame}
\frametitle{Multiplicando las expresiones}
Ahora, multiplicamos la primera ecuación por $\phi_{m}$ y la segunda ecuación por $\phi_{n}$. \pause Restando estos resultados, obtenemos:
\pause
\begin{align*}
\phi_{m}\, \mathcal{L} \, \phi_{n} - \phi_{n} \, \mathcal{L} \, \phi_{m} = \big( \lambda_{m} - \lambda_{n} \big) \, \sigma \, \phi_{n} \, \phi_{m}
\end{align*}
\end{frame}
\begin{frame}
\frametitle{Integrando los resultados}
Integrando ambos lados de la ecuación, usando la identidad de Green y usando las CDF homogéneas, se tiene:
\pause
\begin{align*}
\big( \lambda_{m} - \lambda_{n} \big) \, \scaleint{5ex}_{\bs a}^{b} \sigma \, \phi_{n} \, \phi_{m} \dd{x} = 0
\end{align*}
\end{frame}
\begin{frame}
\frametitle{Considerando los eigenvalores}
Dado que los eigenvalores son distintos, podemos dividir entre $\lambda_{m} - \lambda_{n}$, llegando al resultado deseado:
\pause
\begin{align*}
\scaleint{5ex}_{\bs a}^{b} \sigma \, \phi_{n} \, \phi_{m} \dd{x} = 0
\end{align*}
Por lo tanto, las funciones propias son ortogonales con respecto a la función de peso $\sigma(x)$.
\end{frame}
\begin{frame}
\frametitle{Degeneración}
Si $N$ eigenfunciones linealmente independientes corresponden al mismo eigenvalor, se dice que este último es $N$-veces \emph{degenerado}.
\end{frame}
\begin{frame}
\frametitle{Integral no nula}
Si $\lambda_{m} = \lambda_{n}$, la integral:
\pause
\begin{align*}
\big( \lambda_{m} - \lambda_{n} \big) \, \scaleint{5ex}_{\bs a}^{b} \sigma \, \phi_{n} \, \phi_{m} \dd{x}
\end{align*}
\emph{no necesariamente} se anula.
\end{frame}
\begin{frame}
\frametitle{Consecuencia de la degeneración}
Esto implica que las eigenfunciones linealmente independientes correspondientes al mismo eigenvalor \textcolor{red}{no son automáticamente} ortogonales.
\\
\bigskip
\pause
Por lo que debe de buscarse otro método para obtener un conjunto de eigenfunciones ortogonales, \pause veremos que \textcolor{blue}{siempre} se puede lograr que sean ortogonales.
\end{frame}

\section{Ortogonalización de Gram-Schmidt}
\frame{\tableofcontents[currentsection, hideothersubsections]}
\subsection{La técnica}

\begin{frame}
\frametitle{Ortogonalizando funciones}
Este método toma un conjunto de funciones no ortogonales linealmente dependientes y literalmente construye un conjunto ortogonal de funciones en un intervalo arbitrario con respecto a una función de peso arbitraria.
\end{frame}
\begin{frame}
\frametitle{Considerando funciones reales}
Las funciones involucradas pueden ser reales o complejas, por conveniencia, asumiremos que las funciones son reales, la generalización para funciones complejas, no ofrece mayor dificultad.
\end{frame}
\begin{frame}
\frametitle{Normalización de funciones}
Veamos el caso de la normalización de funciones, que implica lo siguiente:
\pause
\begin{align*}
\scaleint{5ex}_{\bs a}^{b} \phi_{i}^{2} \, \sigma  \, \dd{x}  =  N_{i}^{2}
\end{align*}
\pause
revisemos que aún no se le ha puesto atención al valor de $N_{i}$.
\end{frame}
\begin{frame}
\frametitle{Problema de eigenvalores}
Ya que la ecuación básica:
\pause
\begin{align}
\mathcal{L} \, u (x) + \lambda \, \sigma (x) \, u (x) = 0
\label{eq:ecuacion_10_08}
\end{align}
es \textcolor{blue}{lineal} y \textcolor{blue}{homogénea}, podemos multiplicar la solución por cualquier constante, de tal manera que sigue siendo solución.
\end{frame}
\begin{frame}
\frametitle{Normalizando las funciones}
Por lo que podemos pedir que tal solución $\phi_{i}(x)$ se multiplique por $N_{i}^{-1}$ \pause y ahora la nueva $\phi_{i}$ (normalizada) satisface:
\pause
\begin{align}
\scaleint{5ex}_{\bs a}^{b} \, \phi_{i}^{2} (x) \, \sigma(x) \, \dd{x} = 1
\label{eq:ecuacion_10_39}
\end{align}
\end{frame}
\begin{frame}
\frametitle{Ocupando una delta}
En términos de una delta de Kronecker:
\pause
\begin{align}
\scaleint{5ex}_{\bs a}^{b} \, \phi_{i}(x) \, \phi_{j} (x) \, \sigma (x) \, \dd{x} = \delta_{ij}
\label{eq:ecuacion_10_40}
\end{align}
\pause
La ecuación (\ref{eq:ecuacion_10_39}) nos dice que se ha normalizado a la unidad.
\end{frame}
\begin{frame}
\frametitle{Condición de ortonormalidad}
Incluyendo la propiedad de ortogonalidad, tenemos la ecuación (\ref{eq:ecuacion_10_40}), a las funciones que la satisfacen, se dice que son \textbf{ortonormales} (ortogonales y normalizadas).
\end{frame}
\begin{frame}
\frametitle{Condición de normalización}
Cabe señalar que existen \emph{otras formas} de normalización, \pause cada una de las funciones especiales de la Física Matemática se puede normalizar de distintas formas.
\end{frame}
\begin{frame}
\frametitle{Conjunto de funciones}
Consideremos tres conjuntos de funciones:
\pause
\setbeamercolor{item projected}{bg=blue!70!black,fg=yellow}
\setbeamertemplate{enumerate items}[circle]
\begin{enumerate}[<+->]
\item Un conjunto original, linealmente independiente $u_{n}(x)$ con $n=0,1,2,\ldots$ \\
Las funciones podrían ser funciones propias degeneradas, pero no es necesario que se cumpla este punto.
\seti
\end{enumerate}
\end{frame}
\begin{frame}
\frametitle{Conjunto de funciones}
\setbeamercolor{item projected}{bg=blue!70!black,fg=yellow}
\setbeamertemplate{enumerate items}[circle]
\begin{enumerate}[<+->]  
\conti
\item Un conjunto ortogonal $\psi_{n}(x)$ que se va a construir.
\item Un conjunto de funciones $\phi_{n}(x)$ que serán normalizadas $\varphi_{n}(x)$
\end{enumerate}
\end{frame}
\begin{frame}
\frametitle{Propiedades del conjunto de funciones}
Tendremos las siguientes propiedades:
\pause
\begin{center}
{\fontsize{12}{12}\selectfont
\renewcommand{\arraystretch}{1.5}%
\begin{tabular}{p{3cm} p{3cm} p{3cm}}
\hline
\makecell{$u_{n}(x)$} & \makecell{$\psi_{n}(x)$} & \makecell{$\varphi_{n}(x)$} \\ \hline
\makecell{linealmente \\ independiente} &    \makecell{linealmente \\ independiente} & \makecell{linealmente \\ independiente} \\ \hline
\makecell{no ortogonal} & \makecell{ortogonal} & \makecell{ortogonal} \\ \hline
\makecell{no normalizada} & \makecell{no normalizada} & \makecell{normalizada \\ (ortonormal)} 
\end{tabular}
}
\end{center}
\end{frame}

\subsection{Aplicando la técnica Gram-Schmidt}

\begin{frame}
\frametitle{Definiendo la técnica}
La técnica de Gram-Schmidt consiste en tomar la n-ésima función $\psi_{n}$) para ser $u_{n}(x)$ más un combinación lineal no conocida de la función $\varphi$ previa.
\\
\bigskip
\pause
El que haya una nueva $u_{n}(x)$ nos dará la garantía de que se mantenga la independencia lineal.
\end{frame}
\begin{frame}
\frametitle{Requisito necesario}
El requisito que $\psi_{n}(x)$ sea ortogonal para cada $\varphi$ previa, proporciona los suficientes elementos para determinar cada uno de los coeficientes desconocidos.
\\
\bigskip
\pause
Así cuando ya se determinen las $\psi_{n}$, se pueden normalizar a la unidad, dejando a las  $\varphi_{n} (x)$. \pause \emph{Este procedimiento se repite} para las $\psi_{n+1}(x)$.
\end{frame}
\begin{frame}
\frametitle{Comenzando con la técnica}
Empezamos con $n = 0$, sea:
\pause
\begin{align}
\psi_{0} (x) = u_{0} (x)
\label{eq:ecuacion_10_41}
\end{align}
\pause
no nos preocupemos al no tener una $\varphi$ previa.
\end{frame}
\begin{frame}
\frametitle{Normalizando la función}
Entonces normalizamos:
\pause
\begin{align}
\varphi_{0}(x) = \dfrac{\psi_{0} (x)}{\left[ \displaystyle \scaleint{5ex} \psi_{0}^{2} \, \sigma \, \dd{x} \right]^{1/2}}
\label{eq:ecuacion_10_42}
\end{align}
\end{frame}
\begin{frame}
\frametitle{Para el siguiente valor de $n$}
Para $n = 1$, tenemos:
\pause
\begin{align}
\psi_{1} (x) = u_{1} (x) + a_{1, 0} \, \varphi_{0} (x)
\label{eq:ecuacion_10_43}
\end{align}
\pause
Que requiere que $\psi_{1} (x)$ sea ortogonal a $\varphi_{0} (x)$ (en este punto, la normalización de $\psi_{1} (x)$ es irrelevante).
\end{frame}
\begin{frame}
\frametitle{Ortogonalizando la función}
La ortogonalidad nos conduce a:
\pause
\begin{eqnarray}
\begin{aligned}[b]
\scaleint{5ex} \psi_{1} \, \varphi_{0} \, \sigma \, \dd{x} &= \pause \scaleint{5ex} u_{1} \, \varphi_{0} \, \sigma \, \dd{x} + a_{1,0} \scaleint{5ex} \varphi_{0}^{2} \, \sigma \dd{x} = \\[0.5em] \pause
&= 0
\end{aligned}
\label{eq:ecuacion_10_44}
\end{eqnarray}
\end{frame}
\begin{frame}
\frametitle{Normalizando a la unidad}
Ya que $\varphi_{0}$ se normaliza a la unidad (ec. \ref{eq:ecuacion_10_42}), tenemos:
\pause
\begin{align}
a_{1,0} = - \scaleint{5ex} u_{1} \, \varphi_{0} \, \sigma \dd{x}
\label{eq:ecuacion_10_45}
\end{align}
que deja fijo el valor de $a_{1, 0}$
\end{frame}
\begin{frame}
\frametitle{Normalizando a la unidad}
Normalizando, definimos:
\pause
\begin{align}
\varphi_{1} (x) = \dfrac{\psi_{1} (x)}{ \bigg[ \displaystyle \scaleint{5ex} \psi_{1}^{2} \, \sigma \dd{x} \bigg]^{1/2}}
\label{eq:ecuacion_10_46}
\end{align}
\end{frame}
\begin{frame}
\frametitle{Repitiendo para otros valores de $n$}
Generalizando, resulta:
\pause
\begin{align}
\varphi_{i} (x) = \dfrac{\psi_{i} (x)}{ \bigg[ \displaystyle \scaleint{5ex} \psi_{i}^{2} (x) \, \sigma (x) \dd{x} \bigg]^{1/2}}
\label{eq:ecuacion_10_47}
\end{align}
\pause
donde:
\pause
\begin{align}
\psi_{i}(x) = u_{i} + a_{1, 0} \, \phi_{0} + a_{i, 1} \, \phi_{1} + \ldots + a_{i, i-1} \, \phi_{i-1}
\label{eq:ecuacion_10_48}
\end{align}
\end{frame}
\begin{frame}
\frametitle{Los coeficientes $a_{i}$}
Los coeficientes $a_{i, j}$ están dados por:
\pause
\begin{align}
a_{i, j} = - \scaleint{5ex} u_{i} \, \varphi_{j} \, \sigma  \dd{x}
\label{eq:ecuacion_10_49}
\end{align}
Esta ecuación es para una \textcolor{blue}{normalización unitaria}.
\end{frame}
\begin{frame}
\frametitle{Normalización en general}
Para otros tipos de normalización, se tiene que:
\pause
\begin{align*}
\scaleint{5ex}_{\bs a}^{b} \left[ \varphi_{j} (x) \right]^{2} \, \sigma (x) \dd{x} =  N_{j}^{2}
\end{align*}
\end{frame}
\begin{frame}
\frametitle{Expresión para otras normalizaciones}
Entonces la ecuación (\ref{eq:ecuacion_10_47}) se reemplaza por:
\pause
\begin{align}
\varphi_{i} (x) =  N_{i} \: \dfrac{\psi_{i} (x)}{ \bigg[ \displaystyle \scaleint{5ex} \psi_{i}^{2} \, \sigma \dd{x} \bigg]^{1/2}}
\label{eq:ecuacion_10_47a}
\end{align}
\end{frame}
\begin{frame}
\frametitle{Los términos $a_{i,j}$}
Los términos $a_{i,j}$ resultan:
\pause
\begin{align}
a_{i, j} = - \dfrac{ \displaystyle \scaleint{5ex} u_{i} \, \varphi_{j} \, \sigma \dd{x}}{N_{j}^{2}}
\label{eq:ecuacion_10_49a}
\end{align}
\end{frame}
\begin{frame}
\frametitle{Conclusión del método de Gram-Schmidt}
Cabe señalar que el procedimiento de Gram-Schmidt es una manera de construir un conjunto ortogonal o ortonormal, \pause pero las funciones $\varphi_{i}(x)$ no son únicas.
\\
\bigskip
\pause
Existe un infinito de posibles conjuntos ortonormales para un intervalo dado y una función de peso dada.
\end{frame}

\subsection{Ortogonalización Gram-Schmidt Polinomios de Legendre}

\begin{frame}
\frametitle{Planteamiento del problema}
Queremos generar un conjunto ortonormal a partir de las funciones:
\pause
\begin{align*}
u_{n} (x) = x^{n}, \hspace{1.5cm} n = 0, 1, 2, \ldots
\end{align*}
En el intervalo $-1 \leq x \leq 1$ y con la función de peso: $\sigma (x) = 1$.
\end{frame}
\begin{frame}
\frametitle{Aplicando la técnica}
De acuerdo a la técnica descrita de ortogonalización de Gram-Schmidt:
\pause
\begin{align}
u_{0} = 1 \hspace{1.5cm} \varphi_{0} =  \dfrac{1}{\sqrt{2}}
\label{eq:ecuacion_10_50}
\end{align}
\end{frame}
\begin{frame}
\frametitle{Función ortogonal pero no normalizada}
Entonces:
\pause
\begin{align}
\psi_{1} (x) = x + a_{1,0} \, \dfrac{1}{\sqrt{2}}
\label{eq:ecuacion_10_51}
\end{align}
\pause
donde:
\begin{align}
a_{1, 0} = - \scaleint{5ex}_{\bs -1}^{1} \dfrac{x}{\sqrt{2}} \, \dd{x} = 0
\label{eq:ecuacion_10_52}
\end{align}
\end{frame}
% por simetría.
\begin{frame}
\frametitle{Normalizando la función ortogonal}
Normalizando $\psi_{1}$, obtenemos:
\pause
\begin{align}
\varphi_{1} (x) = \sqrt{\dfrac{3}{2}} \, x
\label{eq:ecuacion_10_53}
\end{align}
\end{frame}
\begin{frame}
\frametitle{Repitiendo el procedimiento}
Continuando el método de Gram-Schmidt, se define ahora:
\pause
\begin{align}
\psi_{2} (x) = x^{2} +  a_{2, 0} \, \dfrac{1}{\sqrt{2}} +  a_{2, 1} \, \sqrt{\dfrac{3}{2}} \, x
\label{eq:ecuacion_10_54}
\end{align}
\end{frame}
\begin{frame}
\frametitle{Valores de los términos}
Donde:
\pause
\begin{eqnarray}
a_{2, 0} &=& \pause - \scaleint{5ex}_{\bs -1}^{1} \, \dfrac{x^{2}}{\sqrt{2}} \, \dd{x} = \pause - \dfrac{\sqrt{2}}{3} \label{eq:ecuacion_10_55} \\[1em] \pause
a_{2, 1} &=& \pause - \scaleint{5ex}_{\bs -1}^{1} \, \sqrt{\dfrac{3}{2}} \, x^{3} \dd{x} = \pause 0 \label{eq:ecuacion_10_56}
\end{eqnarray}
\end{frame}
\begin{frame}
\frametitle{Función ortogonal}
Por tanto:
\pause
\begin{align}
\psi_{2} (x) = x^{2} - \dfrac{1}{3}
\label{eq:ecuacion_10_57}
\end{align}
\pause
Normalizando a la unidad, tenemos:
\begin{align}
\varphi_{2} (x) = \sqrt{\dfrac{5}{2}} \, \dfrac{1}{2} \, (3 \, x^{2} - 1)
\label{eq:ecuacion_10_58}
\end{align}
\end{frame}
\begin{frame}
\frametitle{Otra función ortogonal}
La siguiente función $\varphi_{3}(x)$ es:
\pause
\begin{align}
\varphi_{3} (x) = \sqrt{\dfrac{7}{2}} \, \dfrac{1}{2} \, (5 \, x^{3} - 3 \, x)
\label{eq:ecuacion_10_59}
\end{align}
\end{frame}
\begin{frame}
\frametitle{La $n$-ésima función}
Se puede demostrar que:
\pause
\begin{align}
\varphi_{n} (x) = \sqrt{\dfrac{2 \, n + 1}{2}} \, P_{n} (x)
\label{eq:ecuacion_10_60}
\end{align}
\pause
donde $P_{n}$ es el polinomio de orden $n$ de Legendre.
\end{frame}
\begin{frame}
\frametitle{El manejo de funciones especiales}
El uso de funciones especiales como en este caso los polinomios de Legendre $P_{n}(x)$, será algo común, ya en el Tema 5 se revisará la construcción completa de los $P_{n}(x)$, así como un conjunto de propiedades.
\end{frame}

\section{Polinomios ortogonales}
\frame{\tableofcontents[currentsection, hideothersubsections]}
\subsection{Conjunto de polinomios}

\begin{frame}
\frametitle{De la técnica Gram-Schmidt}
El ejemplo anterior se ha elegido estrictamente para ilustrar el procedimiento de Gram-Schmidt.
\\
\bigskip
\pause
Aunque tiene la ventaja de introducir los polinomios de Legendre, las funciones iniciales $u_{n} = x^{n}$ no son funciones propias degeneradas y no son soluciones de la ecuación de Legendre.
\end{frame}
\begin{frame}
\frametitle{De la técnica Gram-Schmidt}  
Las $u_{n}$ utilizadas son simplemente un conjunto de funciones que hemos reorganizado aquí para crear un conjunto ortonormal para el intervalo dado y la función de peso dada.
\end{frame}
\begin{frame}
\frametitle{De la técnica Gram-Schmidt}  
El hecho de que hayamos obtenido los polinomios de Legendre no es \enquote{magia negra} , sino una consecuencia directa \emph{de la elección de la función de peso y del intervalo}.
\end{frame}
\begin{frame}
\frametitle{Otros intervalos, otras $\sigma (x)$}  
El uso de $u_{n} = x^{n}$ pero eligiendo otros intervalos y funciones de peso, nos conduce a otros conjuntos de polinomios ortogonales. 
\end{frame}
\begin{frame}
\frametitle{Polinomios de Legendre}
\begin{table}
  \begin{tabular}{c c c}
  Intervalo & $\sigma (x)$ & Normalización estándar \\ \hline
  $-1 \leq x \leq 1$ & $1$ & $\displaystyle \scaleint{5ex}_{\bs -1}^{1} \left[ P_{n} (x) \right]^{2} \dd{x} = \dfrac{2}{2 \, n + 1} $
  \end{tabular}
\end{table}
\end{frame}
\begin{frame}
\frametitle{Polinomios de Legendre desplazados}
\begin{table}
  \begin{tabular}{c c c}
  Intervalo & $\sigma (x)$ & Normalización estándar \\ \hline
  $0 \leq x \leq 1$ & $1$ & $\displaystyle \scaleint{5ex}_{\bs -1}^{1} \left[ P_{n}^{*}(x) \right]^{2} \dd{x} = \dfrac{2}{2 \, n + 1}$  
\end{tabular}
\end{table}
\end{frame}
\begin{frame}
\frametitle{Polinomios de Chebyshev tipo I}
\begin{table}
  \begin{tabular}{c c}
  Intervalo & $\sigma (x)$ \\ \hline
  $-1 \leq x \leq 1$ & $(1 - x^{2})^{-1/2}$ \\ \hline 
\end{tabular}
\end{table}
Normalización estándar:
\begin{align*}
\scaleint{5ex}_{\bs -1}^{1} \dfrac{\left[ T_{n}(x) \right]^{2}}{(1 - x^{2})^{-1/2}} \dd{x} = \begin{cases} 
\displaystyle \frac{\pi}{2} & n \neq 0 \\
\pi & n = 0 \end{cases}
\end{align*}
\end{frame}
\begin{frame}
\frametitle{Polinomios de Chebyshev desplazados tipo I}
\begin{table}
  \begin{tabular}{c c}
  Intervalo & $\sigma (x)$ \\ \hline
  $0 \leq x \leq 1$ & $[x (1 - x)]^{-1/2}$ \\ \hline 
\end{tabular}
\end{table}
Normalización estándar:
\begin{align*}
\scaleint{5ex}_{\bs 0}^{1} \dfrac{\left[ T_{n}^{*} (x) \right]^{2}}{[x (1 - x)]^{-1/2}} \dd{x} = \begin{cases} 
\displaystyle \frac{\pi}{2} & n > 0 \\
\pi & n = 0 \end{cases}
\end{align*}
\end{frame}
\begin{frame}
\frametitle{Polinomios de Chebyshev tipo II}
\begin{table}
\begin{tabular}{c c}
Intervalo & $\sigma (x)$ \\ \hline
$-1 \leq x \leq 1$ & $(1 - x^{2})^{1/2}$ \\ \hline
\end{tabular}
\end{table}
Normalización estándar:
\begin{align*}
\scaleint{5ex}_{\bs -1}^{1} [U_{n} (x)]^{2} \, (1 - x^{2})^{1/2} \, \dd x = \frac{\pi}{2}
\end{align*}
\end{frame}
\begin{frame}
\frametitle{Polinomios de Laguerre}
\begin{table}
\begin{tabular}{c c c}
Intervalo & $\sigma (x)$ & Normalización estándar \\ \hline
$0 \leq x < \infty $ & $e^{-x}$ & $\displaystyle \scaleint{5ex}_{\bs 0}^{\infty} \left[ L_{n} (x) \right]^{2} \, e^{-x} \dd{x} =  1 $
\end{tabular}
\end{table}
\end{frame}
\begin{frame}
\frametitle{Polinomios Asociados de Laguerre}
\begin{table}
\begin{tabular}{c c}
Intervalo & $\sigma (x)$ \\ \hline
$0 \leq x < \infty $ & $x^{k} \, e^{-x}$ \\ \hline
\end{tabular}
\end{table}
Normalización estándar:
\begin{align*}
\scaleint{5ex}_{\bs 0}^{\infty} \left[ L_{n}^{k} (x) \right]^{2} \, x^{k} \, e^{-x} \dd{x} = \dfrac{(n + k)!}{n!}
\end{align*}
\end{frame}
\begin{frame}
\frametitle{Polinomios de Hermite}
\begin{table}
\begin{tabular}{c c}
Intervalo & $\sigma (x)$ \\ \hline
$- \infty < x < \infty $ & $e^{-x^{2}}$ \\ \hline
\end{tabular}
\end{table}
Normalización estándar:
\begin{align*}
\scaleint{5ex}_{\bs -\infty}^{\infty} \left[ H_{n} (x) \right]^{2} e^{-x^{2}} \dd{x} = 2^{n} \, \pi^{1/2} \, n!
\end{align*}
\end{frame}
\begin{frame}
\frametitle{Revisando la técnica de ortogonalización}
Una revisión de este proceso de ortogonalización revelará dos características arbitrarias:
\pause
\setbeamercolor{item projected}{bg=blue!70!black,fg=yellow}
\setbeamertemplate{enumerate items}[circle]
\begin{enumerate}[<+->]
\item Primero, como se enfatizó antes, no es necesario normalizar las funciones a la unidad. 
\seti
\end{enumerate}
\end{frame}
\begin{frame}
\frametitle{Revisando la técnica de ortogonalización}
En el ejemplo que acabamos de mostrar, podríamos haber requerido:
\pause
\begin{align}
\scaleint{5ex}_{\bs -1}^{1} \varphi_{n} (x) \: \varphi_{m} (x) \, \dd{x} = \dfrac{2}{2 \, n +1} \, \delta_{nm}
\label{eq:ecuacion_10_61}
\end{align}
y el conjunto resultante habrían sido el de los polinomios de Legendre.
\end{frame}
\begin{frame}
\frametitle{Revisando la técnica de ortogonalización}
\setbeamercolor{item projected}{bg=blue!70!black,fg=yellow}
\setbeamertemplate{enumerate items}[circle]
\begin{enumerate}[<+->]  
\conti
\item Segundo, el signo de $\varphi_{n} (x)$ siempre es indeterminado.
\end{enumerate}
\end{frame}
\begin{frame}
\frametitle{Revisando la técnica de ortogonalización}
En el ejemplo, elegimos el signo al requerir que el coeficiente de mayor potencia de $x$ en el polinomio sea positivo. 
\\
\bigskip
\pause
Para los polinomios de Laguerre, por otro lado, requeriríamos que el coeficiente de mayor potencia sea $(-1)^{n}/n!$
\end{frame}
\end{document}