\documentclass[12pt]{article}
\usepackage[left=0.25cm,top=1cm,right=0.25cm,bottom=1cm]{geometry}
\textwidth = 20cm
\hoffset = -1cm
\usepackage[utf8]{inputenc}
\usepackage[spanish,es-tabla]{babel}
\usepackage[autostyle,spanish=mexican]{csquotes}
\usepackage[tbtags]{amsmath}
\usepackage{nccmath}
\usepackage{amsthm}
\usepackage{amssymb}
\usepackage{graphicx}
\usepackage{standalone}
\usepackage[outdir=./]{epstopdf}
\usepackage{siunitx}
\usepackage{physics}
\usepackage{color}
\usepackage{float}
\usepackage{multicol}
%\usepackage{milista}
\usepackage{enumitem}
\usepackage{anyfontsize}
\usepackage{anysize}
\usepackage{enumitem}
\usepackage{capt-of}
\usepackage{bm}
\usepackage{relsize}
\usepackage{placeins}
\usepackage{empheq}
\usepackage{cancel}
\usepackage{wrapfig}
\spanishdecimal{.}
\renewcommand{\baselinestretch}{1.5} 
\renewcommand\labelenumii{\theenumi.{\arabic{enumii}}}
\newcommand{\ptilde}[1]{\ensuremath{{#1}^{\prime}}}
\newcommand{\stilde}[1]{\ensuremath{{#1}^{\prime \prime}}}
\newcommand{\ttilde}[1]{\ensuremath{{#1}^{\prime \prime \prime}}}
\newcommand{\ntilde}[2]{\ensuremath{{#1}^{(#2)}}}


\usepackage{apacite}
\title{Actividades pendientes del Tema 3 \\[0.3em]  \large{Matemáticas Avanzadas de la Física}\vspace{-3ex}}
\author{M. en C. Gustavo Contreras Mayén}
\date{ }
\begin{document}
\vspace{-4cm}
\maketitle
\fontsize{14}{14}\selectfont

\textbf{Indicaciones: } Deberás de resolver cada ejercicio de la manera más completa, ordenada y clara posible, anotando cada paso así como las operaciones involucradas. El puntaje de cada ejercicio es de \textbf{1 punto}.

\section{Ejercicios del Tema 3.}
%Ref. Arfken 10.4.4
\textbf{Ejercicio 9. } En lugar de la expansión de una función $F(x)$ dada por:
\begin{align*}
F(x) = \nsum_{n=0}^{\infty} a_{n} \, \varphi_{n} (x)
\end{align*}
con los coeficientes:
\begin{align*}
a_{n} = \scaleint{5ex}_{\bs a}^{b} F(x) \, \varphi_{n} (x) \, \omega (x) \dd{x}
\end{align*}
Considera la aproximación por una serie \textbf{finita}:
\begin{align*}
F(x) \approx \nsum_{n=0}^{m} c_{n} \, \varphi_{n} (x)
\end{align*}

Demuestra que el cuadrado del error medio:
\begin{align*}
\scaleint{5ex}_{\bs a}^{b} \bigg[ F(x) - \nsum_{n=0}^{m} c_{n} \, \varphi_{n} (x) \bigg]^{2} \, \omega (x) \dd{x}
\end{align*}
se minimiza cuando $c_{n} = a_{n}$. \emph{Nota: } Los valores de los coeficientes son independientes del número de términos en la serie finita. Esta independencia es una consecuencia de la ortogonalidad y no sería válida para un ajuste por mínimos cuadrados utilizando potencias de $x$.
\par
%Ref. Arfken 10.4.7
\textbf{Ejercicio 10. } Recupera la desigualdad de Schwarz de la siguiente identidad:
\begin{align*}
&\bigg[ \scaleint{5ex}_{\bs a}^{b} f(x) \, g(x) \dd{x} \bigg
]^{2} {=} \scaleint{5ex}_{\bs a}^{b} \big[ f(x) \big
]^{2} \dd{x} \, \scaleint{5ex}_{\bs a}^{b} \big[ g(x) \big
]^{2} \dd{x} + \\[0.5em]
&- \dfrac{1}{2} \, \scaleint{5ex}_{\bs a}^{b} \, \scaleint{5ex}_{\bs a}^{b} \bigg[ f(x) \, g(y) - f(y) \, g(x) \bigg
]^{2} \dd{x} \dd{y}
\end{align*}
\emph{Nota:} Cuida el signo de la expresión, recuerda que al cortar el renglón, se deja el signo $+$, en el siguiente renglón se tiene el signo $-$, por lo que el segundo término está restando el producto del primer término.

\subsection{Enunciado para el Examen.}

La representación de la función delta de Dirac:
\begin{align*}
\delta (x - t) = \nsum_{n=0}^{\infty} \varphi_{n} (x) \, \varphi_{n} (t)
\end{align*}
frecuentemente se denomina \emph{relación de cierre.}
\par
Para un conjunto ortonormal de funciones $\varphi_{n}$, demuestra que el cierre implica la completes, es decir, que la ecuación:
\begin{align*}
F (x) = \nsum_{n=0}^{\infty} a_{n} \, \varphi_{n} (x)
\end{align*}
se deriva de la expresión que define la $\delta (x - t)$.

\subsection{Entrega de las actividades.}

Los ejercicios pendientes y el enunciado del Tema 3 del examen intermedio, se deberán de entregar \textbf{el martes 26 de abril a las 6 pm} por Moodle.

\end{document}