\documentclass[12pt]{article}
\usepackage[left=0.25cm,top=1cm,right=0.25cm,bottom=1cm]{geometry}
\textwidth = 20cm
\hoffset = -1cm
\usepackage[utf8]{inputenc}
\usepackage[spanish,es-tabla]{babel}
\usepackage[autostyle,spanish=mexican]{csquotes}
\usepackage[tbtags]{amsmath}
\usepackage{nccmath}
\usepackage{amsthm}
\usepackage{amssymb}
\usepackage{graphicx}
\usepackage{standalone}
\usepackage[outdir=./]{epstopdf}
\usepackage{siunitx}
\usepackage{physics}
\usepackage{color}
\usepackage{float}
\usepackage{multicol}
%\usepackage{milista}
\usepackage{enumitem}
\usepackage{anyfontsize}
\usepackage{anysize}
\usepackage{enumitem}
\usepackage{capt-of}
\usepackage{bm}
\usepackage{relsize}
\usepackage{placeins}
\usepackage{empheq}
\usepackage{cancel}
\usepackage{wrapfig}
\spanishdecimal{.}
\renewcommand{\baselinestretch}{1.5} 
\renewcommand\labelenumii{\theenumi.{\arabic{enumii}}}
\newcommand{\ptilde}[1]{\ensuremath{{#1}^{\prime}}}
\newcommand{\stilde}[1]{\ensuremath{{#1}^{\prime \prime}}}
\newcommand{\ttilde}[1]{\ensuremath{{#1}^{\prime \prime \prime}}}
\newcommand{\ntilde}[2]{\ensuremath{{#1}^{(#2)}}}


\title{Lista de enunciados del Examen Intermedio \\ \large {Curso Matemáticas Avanzadas de la Física}\vspace{-3ex}}

\author{M. en C. Gustavo Contreras Mayén}
\date{ }

\pagestyle{fancy}
\fancyhf{}
\rhead{Curso MAF}
\lhead{\leftmark}
\rfoot{\thepage}
\setlength{\headheight}{16pt}%


\begin{document}
\maketitle
\fontsize{14}{14}\selectfont

\section{Enunciados para el examen intermedio.}

\subsection{Lista de enunciados por tema.}
A continuación se enlistan los enunciados que formarán parte del Examen Intermedio, que cubre los temas 1, 2 y 3 del curso.

\begin{enumerate}
\item \textbf{Tema 1:}
\begin{enumerate}
\item 1, 2, 4, 6, 7, 8
\end{enumerate}
\item \textbf{Tema 2:}
\begin{enumerate}
\item 1, 2, 3, 5, 6, 8
\end{enumerate}
\item \textbf{Tema 3:}
Hasta el contenido de normalización, faltaría 1 ejercicio que completa el Tema 3. 
\\
Se deben de entregar:
\begin{enumerate}
\item 1, 2, 3, 5, 8
\end{enumerate}
\end{enumerate}

\subsection{Fecha de entrega.}

La solución del examen con $17$ enunciados deberá de enviarse por Moodle el \textbf{martes 19 de abril a las 6 pm}

\subsection{Importante}

Considerando que es un examen intermedio:
\begin{enumerate}
\item No se aceptarán envíos pasada la fecha de entrega.
\item No aplica el envío posterior para ser calificado con un puntaje menor.
\item Se espera recibir un archivo ordenado por Tema y por enunciado, con el mayor detalle posible, sin dejar pasos incompletos o no claros.
\item En caso de que el peso del archivo sea mayor de $20$ MB, deberán de subir su archivo a Drive y enviar al equipo académico el enlace para descargar y calificar su entrega.
\end{enumerate}

\end{document}