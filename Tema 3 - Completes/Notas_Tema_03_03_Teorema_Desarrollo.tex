\documentclass[12pt]{beamer}
\usepackage{../Estilos/BeamerMAF}
\usepackage{../Estilos/ColoresLatex}
\input{../Preambulos/preambulo_Beamer_Warsaw_crane}

\date{}

\title{\large{Teorema del desarrollo}}
\subtitle{Tema 3 - Bases completas y ortogonales}
\author{M. en C. Gustavo Contreras Mayén}

\resetcounteronoverlays{saveenumi}

\begin{document}
\maketitle
\fontsize{14}{14}\selectfont
\spanishdecimal{.}

\section*{Contenido}
\frame{\tableofcontents[currentsection, hideallsubsections]}

\section{Teorema del desarrollo}
\frame{\tableofcontents[currentsection, hideothersubsections]}
\subsection{Introducción}

\begin{frame}
\frametitle{Introducción}
En mecánica cuántica (MQ) las cantidades físicas se representan mediante operadores Hermíticos y lineales, en general, \pause la representación de estos se realiza mediante espacios vectoriales abstractos, sin embargo, para poder realizar cálculos se requiere hacer uso de una base vectorial.
\end{frame}
\begin{frame}
\frametitle{Introducción}
El \textocolor{ao}{teorema del desarrollo} nos permite representar un operador en una base determinada \pause y al mismo tiempo, podemos hacer uso de éste, para representar estados.
\\
\bigskip
\pause
El formalismo puede ser heredado al electromagnetismo y emplearlo para resolver ecuaciones diferenciales no homogéneas.
\end{frame}

\section{Notación de Dirac}
\frame{\tableofcontents[currentsection, hideothersubsections]}
\subsection{Bras y Kets}

\begin{frame}
\frametitle{Vectores y matrices}
Consideraremos una contracción de vectores y matrices de la siguiente manera:
\pause
\begin{equation}
\begin{pmatrix}
a_{1} & a_{2} & a_{3}
\end{pmatrix}
\begin{pmatrix}
m_{11} & m_{12} & m_{13} \\
m_{21} & m_{22} & m_{23} \\
m_{31} & m_{32} & m_{33} \\
\end{pmatrix}
\begin{pmatrix}
b_{1} \\
b_{2} \\
b_{3}
\end{pmatrix}
\label{eq:ecuacion_01}
\end{equation}
\end{frame}
\begin{frame}
\frametitle{Vectores y matrices}
Por otra parte, un producto interno entre vectores lo podemos escribir como:
\pause
\begin{equation}
\begin{pmatrix}
a_{1} & a_{2} & a_{3}
\end{pmatrix}
\begin{pmatrix}
b_{1} \\
b_{2} \\
b_{3}
\end{pmatrix}
\label{eq:ecuacion_02}
\end{equation}
\end{frame}
\begin{frame}
\frametitle{Vectores y matrices}
La forma de operar las ecs. (\ref{eq:ecuacion_01}) y (\ref{eq:ecuacion_02}) es conocida, \pause en ambos casos el resultado es un escalar, \pause ahora escribamos ambas expresiones mediante notación de índices:
\end{frame}
\begin{frame}
\frametitle{Vectores y matrices}
\begin{eqnarray}
\begin{aligned}[b]
\begin{pmatrix}
a_{1} & a_{2} & a_{3}
\end{pmatrix}
\begin{pmatrix}
m_{11} & m_{12} & m_{13} \\
m_{21} & m_{22} & m_{23} \\
m_{31} & m_{32} & m_{33} \\
\end{pmatrix}
\begin{pmatrix}
b_{1} \\
b_{2} \\
b_{3}
\end{pmatrix} \\[0.5em] \pause
\Rightarrow \nsum_{ij} m_{j}^{i} \, a^{j} \, b_{i} = \pause m_{j}^{i} \, a^{j} \, b_{i} 
\end{aligned}
\label{eq:ecuacion_03}
\end{eqnarray}
\end{frame}
\begin{frame}
\frametitle{Vectores y matrices}
Para el producto interno:
\pause
\begin{eqnarray}
\begin{aligned}
\begin{pmatrix}
a_{1} & a_{2} & a_{3}
\end{pmatrix}
\begin{pmatrix}
b_{1} \\
b_{2} \\
b_{3}
\end{pmatrix} \\[0.5em] \pause
\Rightarrow \nsum_{ij} a^{j} \, b_{i} = \pause a^{j} \, b_{i}
\end{aligned}
\label{eq:ecuacion_04}
\end{eqnarray}
\end{frame}
\begin{frame}
\frametitle{Vectores y matrices}
En las Ec.(\ref{eq:ecuacion_03}) y (\ref{eq:ecuacion_04}) tenemos la suma sobre todos los índices para obtener un escalar, \pause al mismo tiempo notamos que las componentes vectoriales tienen esta representación:
\end{frame}
\begin{frame}
\frametitle{Vectores y matrices}
\begin{eqnarray}
\begin{aligned}
\begin{pmatrix}
b_{1} \\
b_{2} \\
b_{3}
\end{pmatrix}
 &\Rightarrow b_{j} \hspace{0.5cm} \mbox{vector covariante}
\\[1em] \pause
\begin{pmatrix}
a_{1} & a_{2} & a_{3}
\end{pmatrix}
 &\Rightarrow a^{i} \hspace{0.5cm} \mbox{vector contravariante}
\end{aligned}
\label{eq:ecuacion_05}
\end{eqnarray}
\end{frame}
\begin{frame}
\frametitle{Vectores y matrices}
Es decir, tenemos que los vectores son elementos de los espacios:
\pause
\begin{table}
\centering
\begin{tabular}{c | c}
Vector & Espacio \\ \hline
Fila & Vectorial \\ \hline
Renglón & Dual
\end{tabular}
\end{table}
\end{frame}
\begin{frame}
\frametitle{Vectores y matrices}
Por lo que las matrices de la ecs.(\ref{eq:ecuacion_01}) y (\ref{eq:ecuacion_03}) están formadas por una combinación entre ambos: el espacio vectorial y su espacio dual.
\end{frame}
\begin{frame}
\frametitle{Vectores y matrices}
La notación de índices permite manejar en forma compacta las ecs. (\ref{eq:ecuacion_01}) y (\ref{eq:ecuacion_02}), \pause no obstante, la información de los objetos matemáticos se da a través de sus componentes, lo que las hace dependientes de la base en la cual son escritos.
\end{frame}
\begin{frame}
\frametitle{Vectores y matrices}
El siguiente paso consiste en utilizar una notación independiente de la base en la cual estos objetos son escritos:
\pause
\begin{eqnarray}
\begin{aligned}
\begin{pmatrix}
b_{1} \\
b_{2} \\
b_{3}
\end{pmatrix} \pause
&\Rightarrow b_{j} \hspace{0.5cm} \Rightarrow \ket{b} \pause \hspace{0.5cm} \mbox{ket}
\\[0.5em] \pause
\begin{pmatrix}
a_{1} & a_{2} & a_{3}
\end{pmatrix} \pause 
&\Rightarrow a^{i} \hspace{0.5cm} \Rightarrow \bra{a} \pause \hspace{0.5cm} \mbox{bra}
\end{aligned}
\label{eq:ecuacion_06}
\end{eqnarray}
\end{frame}
\begin{frame}
\frametitle{Vectores y matrices}
En la ec.(\ref{eq:ecuacion_06}) los vectores:
\setbeamercolor{item projected}{bg=black,fg=white}
\setbeamertemplate{enumerate items}{%
\usebeamercolor[bg]{item projected}%
\raisebox{1.5pt}{\colorbox{bg}{\color{fg}\footnotesize\insertenumlabel}}%
}
\begin{enumerate}[<+->]
\item \textocolor{carmine}{Covariantes} se representan mediante un \textocolor{carmine}{ket}.
\item \textocolor{armygreen}{Contravariantes} se representan mediante un \textocolor{armygreen}{bra}
\end{enumerate}
\pause
No se hace referencia a la base en la que estos vectores son escritos.
\end{frame}
\begin{frame}
\frametitle{Vectores y matrices}
Esta asignación es la que comúnmente se hace en los textos de MQ, \pause sin embargo, tiene su correspondiente respaldo matemático en el teorema de representación de Riesz.
\end{frame}
\begin{frame}
\frametitle{Vectores y matrices}
Usando ésta notación el producto interno de la ec.(\ref{eq:ecuacion_04}) puede escribirse como:
\pause
\begin{eqnarray}
\begin{aligned}
\begin{pmatrix}
a_{1} & a_{2} & a_{3}
\end{pmatrix}
\begin{pmatrix}
b_{1} \\
b_{2} \\
b_{3}
\end{pmatrix} \\[0.5em] \pause
\Rightarrow a^{j} \, b_{i} \pause \hspace{0.5cm} \Rightarrow \braket{a}{b}
\label{eq:ecuacion_07}
\end{aligned}
\end{eqnarray}
\end{frame}
\begin{frame}
\frametitle{Producto tensorial}
Para escribir las matrices necesitamos hacer referencia al \textocolor{blue}{producto tensorial}, \pause este tipo de productos se puede aplicar entre dos tensores de distintos rangos para generar un nuevo tensor, cuyo rango es igual a la suma de los dos anteriores.
\end{frame}
\begin{frame}
\frametitle{Producto tensorial}
Un ejemplo es el siguiente:
\pause
\begin{eqnarray}
\begin{aligned}
\va{a} \otimes \va{b} = \pause
\mqty(a_{1} \\ a_{2} \\ a_{3}) \otimes \mqty(b_{1} & b_{2} & b_{3}) = \pause \mqty(
a_{1} b_{1} & a_{1} b_{2} & a_{1} b_{3} \\
a_{2} b_{1} & a_{2} b_{2} & a_{2} b_{3} \\
a_{3} b_{1} & a_{3} b_{2} & a_{3} b_{3}
)
\end{aligned}
\label{eq:ecuacion_08}
\end{eqnarray}
\end{frame}
\begin{frame}
\frametitle{Producto tensorial}
En la ec.(\ref{eq:ecuacion_08}) hemos construido un tensor de rango 2 usando dos tensores de rango 1
\end{frame}
\begin{frame}
\frametitle{Producto tensorial}
Usando la notación de Dirac reescribimos la misma ec.(\ref{eq:ecuacion_08}):
\pause
\fontsize{12}{12}\selectfont
\begin{eqnarray}
\begin{aligned}
\mqty(a_{1} \\ a_{2} \\ a_{3}) \otimes \mqty(b_{1} & b_{2} & b_{3}) = \pause \mqty(
a_{1} b_{1} & a_{1} b_{2} & a_{1} b_{3} \\
a_{2} b_{1} & a_{2} b_{2} & a_{2} b_{3} \\
a_{3} b_{1} & a_{3} b_{2} & a_{3} b_{3}
) = \pause \ket{a} \bra{b}
\end{aligned}
\label{eq:ecuacion_09}
\end{eqnarray}
\end{frame}
\begin{frame}
\frametitle{Producto tensorial}
De ese modo, se puede construir una matriz usando como base el producto tensorial de dos bases vectoriales, \pause para ello requerimos analizar un elemento extra: \pause consideramos una base vectorial discreta $\left\{ \ket{\varphi_{n}} \right\} $ y realizamos el siguiente producto $\ket{\varphi_{n}} \bra{\varphi_{n}}$:
\end{frame}
\begin{frame}
\frametitle{Producto tensorial}
Ahora tomamos la suma sobre cada producto:
\pause
\begin{align}
\nsum_{n} \ket{\varphi_{n}} \bra{\varphi_{n}}
\label{eq:ecuacion_10}
\end{align}
\end{frame}
\begin{frame}
\frametitle{Operador identidad}
Sea $\mathbf{1}$ el operador identidad, \pause cuando la Ec.(\ref{eq:ecuacion_10}) satisface la siguiente condición:
\pause
\begin{align}
\nsum_{n} \ket{\varphi_{n}} \bra{\varphi_{n}} = \mathbf{1}
\label{eq:ecuacion_11}
\end{align}
diremos que la base es completa.
\end{frame}
\begin{frame}
\frametitle{Base completa}
Sin pérdida de generalidad podemos pedir la condición:
\pause
\begin{align*}
\braket{\varphi_{n}}{\varphi_{m}} = \delta_{nm}
\end{align*}
\end{frame}
\begin{frame}
\frametitle{Base completa}
Un ejemplo de base completa, es la base canónica de $\mathbb{R}^{3}$:
\pause
\begin{eqnarray*}
&\displaystyle \nsum_{n=1}^{3}& \ket{x_{n}} \bra{x_{n}} = \pause \ket{x} \bra{x} + \ket{y} \bra{y} + \ket{z} \bra{z} = \\[0.5em] \pause
&=& \! \! \! \! \mqty(1 \\ 0 \\ 0) \mqty(1 & 0 & 0) + \mqty(0 \\ 1 \\ 0) \mqty(0 & 1 & 0) + \mqty(0 \\ 0 \\ 1) \mqty(0 & 0 & 1) =
\end{eqnarray*}
\end{frame}
\begin{frame}
\frametitle{Base completa}
\begin{eqnarray*}
\begin{aligned}
&= \mqty(
1 & 0 & 0 \\
0 & 0 & 0 \\
0 & 0 & 0
) +
\mqty(
0 & 0 & 0 \\
0 & 1 & 0 \\
0 & 0 & 0
) + 
\mqty(
0 & 0 & 0 \\
0 & 0 & 0 \\
0 & 0 & 1
) = \\[0.5em] \pause
&= \mqty(
1 & 0 & 0 \\
0 & 1 & 0 \\
0 & 0 & 1
) 
\end{aligned}
\end{eqnarray*}
\end{frame}
\begin{frame}
\frametitle{Base completa}
En este caso hemos obtenido la matriz identidad de $\mathbb{R}^{3}$.
\\
\bigskip
\pause
Usaremos este resultado para representar una matriz mediante la notación de Dirac.
\end{frame}

\section{Teorema del desarrollo}
\frame{\tableofcontents[currentsection, hideothersubsections]}
\subsection{Fundamento}

\begin{frame}
\frametitle{Introducción}
Sea $\mathbf{M}$ un operador, construiremos una representación de este objeto matemático en términos de una base finita, usamos el operador unidad de la siguiente manera:
\end{frame}
\begin{frame}
\frametitle{Introducción}
\begin{eqnarray*}
\begin{aligned}[b]
\mathbf{M} &= \underbrace{\mathbf{1 \, M \, 1} =  \left( \nsum_{m=1}^{n} \ket{\varphi_{m}} \bra{\varphi_{m}} \right) \, \mathbf{M} \, \left( \nsum_{l=1}^{n} \ket{\varphi_{l}} \bra{\varphi_{l}} \right)}_{\text{usando una base completa}} = \\[0.5em] \pause
&= \nsum_{m=1}^{n} \nsum_{l=1}^{n} \ket{\varphi_{m}} \bra{\varphi_{m}} \, \mathbf{M} \, \ket{\varphi_{l}} \bra{\varphi_{l}} = \\[0.5em] \pause
&= \nsum_{m=1}^{n} \nsum_{l=1}^{n} \underbrace{\left\{ \bra{\varphi_{m}} \, \mathbf{M} \, \ket{\varphi_{l}} \right\}}_{\text{Ver nota*}} \, \ket{\varphi_{m}} \bra{\varphi_{l}} =
\end{aligned}
\end{eqnarray*}
\end{frame}
\begin{frame}
\frametitle{Introducción}
\begin{eqnarray}
\begin{aligned}[b]
&= \nsum_{m=1}^{n} \nsum_{l=1}^{n} \underbrace{\left\{ \bra{\varphi_{m}} \, \mathbf{M} \, \ket{\varphi_{l}} \right\}}_{\text{Ver nota*}} \, \ket{\varphi_{m}} \bra{\varphi_{l}} = \\[0.5em] \pause
&= \nsum_{m=1}^{n} \nsum_{l=1}^{n} \mathbf{M}_{ml} \ket{\varphi_{m}} \bra{\varphi_{l}} = \\[0.5em] \pause
&= \mathbf{M}_{ml} \ket{\varphi_{m}} \bra{\varphi_{l}}
\end{aligned}
\label{eq:ecuacion_12}
\end{eqnarray}
\end{frame}
\begin{frame}
\frametitle{Introducción}    
\textbf{Nota*: } El término entre llaves es un número complejo correspondiente a las componentes matriciales en esta base.
\\
\bigskip
\pause
La ec.(\ref{eq:ecuacion_12}) es conocida como \textocolor{red}{el teorema del desarrollo}.
\end{frame}
\begin{frame}
\frametitle{Teorema del desarrollo}
Por otro lado, podemos usar este teorema para representar un vector cualquiera en una base arbitraria:
\pause
\begin{eqnarray}
\begin{aligned}[b]
\ket{\alpha} = \pause\mathbf{1} \, \ket{\alpha} &= \pause \nsum_{n} \ket{\varphi_{n}} \bra{\varphi_{n}} \, \ket{\alpha} = \\[0.5em] \pause
&= \nsum_{n} \ket{\varphi_{n}} \braket{\varphi_{n}}{\alpha} = \\[0.5em] \pause
&= \nsum_{n} \alpha_{n} \, \ket{\varphi_{n}}
\end{aligned}
\label{eq:ecuacion_13}
\end{eqnarray}
\end{frame}
\begin{frame}
\frametitle{Relevancia del teorema}
La relevancia del teorema radica en que los operadores de la MQ que se representan mediante operadores lineales y Hermíticos, \pause con el teorema del desarrollo se nos permite asociar una base a dicho operador y de esa manera, los problemas de mecánica cuántica son llevados a problemas de álgebra de matrices.
\end{frame}

\section{Límite continuo}
\frame{\tableofcontents[currentsection, hideothersubsections]}
\subsection{Cambio de discreto a continuo}

\begin{frame}
\frametitle{Límite continuo}
En el desarrollo anterior, nos hemos referido a una base discreta, no obstante, podemos llevar los resultados obtenidos a un límite continuo:
\pause
\begin{eqnarray}
\begin{aligned}[b]
\braket{\varphi_{n}}{\varphi_{m}} &= \delta_{nm} \hspace{0.1cm} \Rightarrow \hspace{0.1cm} \pause \braket{x}{\ptilde{x}} = \delta(x - \ptilde{x}) \\[0.5em] \pause
&= \nsum_{n} \ket{\varphi_{n}} \bra{\varphi_{n}} = \mathbf{1} \\[0.5em] \pause
&\Rightarrow \scaleint{6ex} \ket{x} \bra{x} \dd[3]{x} = \mathbf{1}
\end{aligned}
\label{eq:ecuacion_14}
\end{eqnarray}
\end{frame}
\begin{frame}
\frametitle{Producto punto}
Análogamente el producto punto entre dos vectores puede representarse de este modo:
\pause
\begin{eqnarray}
\begin{aligned}[b]
\braket{\phi}{\psi} &= \bra{\phi} \, \mathbf{1} \, \ket{\psi} = \\[0.5em] \pause
&= \scaleint{6ex} \braket{\phi}{x} \, \braket{x}{\psi} \dd[3]{x} = \\[0.5em] \pause
&= \scaleint{6ex} \phi^{*}(x) \, \psi (x) \dd[3]{x}
\end{aligned}
\label{eq:ecuacion_15}
\end{eqnarray}
\end{frame}
\begin{frame}
\frametitle{Condición particular}
Así mismo, sin pérdida de generalidad, podemos pedir la condición:
\pause
\begin{eqnarray*}
\begin{aligned}
\braket{\varphi_{n}}{\varphi_{m}} &= \delta_{nm} \\[0.5em] \pause
&= \scaleint{6ex} \phi^{*}(x) \, \psi (x) \dd[3]{x}
\end{aligned}
\end{eqnarray*}
\end{frame}
\begin{frame}
\frametitle{Desarrollo de una función}
Usaremos esto para desarrollar una función de variable continua a la que llamaremos $f (x)$, \pause para este fin, partimos de una base discreta de variable continua $\left\{ \varphi_{n}(x) \right\}$
\end{frame}
\begin{frame}
\frametitle{Desarrollo de una función}
Asumiremos que $f (x)$ tiene la siguiente estructura:
\pause
\begin{align}
f (x) = \nsum_{n} c_{n} \, \varphi_{n} (x)
\label{eq:ecuacion_16}
\end{align}
\end{frame}
\begin{frame}
\frametitle{Desarrollo de una función}
Para que la propuesta de la ec.(\ref{eq:ecuacion_16}) sea válida, necesitamos exhibir las condiciones suficientes y necesarias para construir los coeficientes $c_{n}$ de manera unívoca.
\end{frame}
\begin{frame}
\frametitle{Desarrollo de una función}
Comenzamos por tomar el siguiente producto interior de $\varphi_{m}$, con la ec.(\ref{eq:ecuacion_16}):
\pause
\begin{eqnarray}
\begin{aligned}
\scaleint{6ex} \bigg[ \varphi_{m}^{*} (x) \, f(x) \bigg] \dd{x} &= \scaleint{6ex} \bigg[ \varphi_{m}^{*} (x) \bigg] \, \nsum_{n} c_{n} \, \varphi_{n} (x) \dd{x} = \\[0.5em] \pause
&= \nsum_{n} c_{n} \,\scaleint{6ex} \bigg[ \varphi_{m}^{*} (x) \, \varphi_{n} (x) \bigg] \dd{x} = \\[0.5em] \pause
&= \nsum_{n} c_{n} \, \delta_{nm} = \\[0.5em] \pause
&= c_{n}
\end{aligned}
\label{eq:ecuacion_17}
\end{eqnarray}
\end{frame}
\begin{frame}
\frametitle{Desarrollo de una función}
De esa forma concluimos que:
\pause
\begin{align}
\scaleint{6ex} \bigg[ \varphi_{m}^{*}(x) \, f(x) \bigg] \dd{x} = c_{m}
\label{eq:ecuacion_18}
\end{align}
\end{frame}
\begin{frame}
\frametitle{Desarrollo de una función}
De la ec. (\ref{eq:ecuacion_18}) notamos que $c_{m}$ existe y es unívocamente determinado (por construcción de una integral) sí y solo sí:
\pause
\begin{align*}
\scaleint{6ex} \bigg[ \varphi_{m}^{*}(x) \, f(x) \bigg] \dd{x} 
\end{align*}
es convergente.
\end{frame}
\begin{frame}
\frametitle{Estructura de $f(x)$}
Daremos un paso más analizando la estructura de $f(x)$:
\pause
\begin{eqnarray}
\begin{aligned}
f(x) &= \nsum_{n} c_{n} \, \varphi_{n}(x) = \\[0.5em] \pause
&= \nsum_{n} \scaleint{6ex} \bigg[ \varphi_{n}^{*}(\ptilde{x}) \, f(\ptilde{x}) \bigg] \, \varphi_{n} (x) \dd{\ptilde{x}} = \\[0.5em] \pause
&=  \scaleint{6ex} f(\ptilde{x}) \, \nsum_{n} \bigg[ \varphi_{n}^{*}(\ptilde{x}) \, \varphi_{n}(x) \bigg] \dd{\ptilde{x}}
\end{aligned}
\label{eq:ecuacion_19}
\end{eqnarray}
\end{frame}
\begin{frame}
\frametitle{Estructura de $f(x)$}
Esta última expresión solo es posible si se satisface la condición:
\pause
\begin{align}
\nsum_{n} \bigg[ \varphi_{n}^{*}(\ptilde{x}) \, \varphi_{n}(x) \bigg] = \delta(\ptilde{x} - x)
\label{eq:ecuacion_20}
\end{align}
\end{frame}
\begin{frame}
\frametitle{Utilidad del límite continuo}
En el límite continuo la ec.(\ref{eq:ecuacion_19}) es la condición de que permite afirmar que la base usada es completa, está propiedad es la que permite generar una técnica de solución para resolver ecuaciones diferenciales no homogéneas.
\end{frame}

\end{document}