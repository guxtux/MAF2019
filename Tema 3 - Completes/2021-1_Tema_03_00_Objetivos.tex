\input{../Preambulos/preambulo_presentacion_Warsaw_crane}
\title{\large{Tema 3 - Bases completas y ortogonales}}
\subtitle{Objetivos}
\author{M. en C. Gustavo Contreras Mayén}
\date{\today}
\institute{Facultad de Ciencias - UNAM}
\titlegraphic{\includegraphics[width=1.75cm]{../Imagenes/escudo-facultad-ciencias}\hspace*{4.75cm}~%
   \includegraphics[width=1.75cm]{../Imagenes/escudo-unam}
}
\setbeamertemplate{navigation symbols}{}
\begin{document}
\maketitle
\fontsize{14}{14}\selectfont
\spanishdecimal{.}
\section*{Contenido}
\frame[allowframebreaks]{\tableofcontents[currentsection, hideallsubsections]}
\section{Introducción}
\frame{\tableofcontents[currentsection, hideothersubsections]}
\subsection{Objetivo del Tema}
\begin{frame}
\frametitle{Lo que hemos trabajado}
Hasta ahora hemos discutido de manera general el tipo de ecuaciones que aparecen en la física y algunas de sus propiedades, hemos trabajado con algunas de las ecuaciones diferenciales en el que estamos interesados y hemos estudiado sus singularidades.
\end{frame}
\begin{frame}
\frametitle{Lo que hemos trabajado}
Todas estas ecuaciones son ecuaciones diferenciales lineales de segundo orden, las cuales como hemos visto, admiten sólo dos soluciones linealmente independientes.
\end{frame}
\begin{frame}
\frametitle{Lo que hemos trabajado}
Nuestro objetivo en el Tema 3 es estudiar algunas propiedades adicionales de estas ecuaciones, así como algunas características generales de sus soluciones. 
\end{frame}
\begin{frame}
\frametitle{Lo que hemos trabajado}
En este tema, nos enfocaremos no a resolver la ED sino a \emph{desarrollar y comprender las propiedades generales de las soluciones}.
\end{frame}
\begin{frame}
\frametitle{Hacia dónde nos dirigimos}
Algunos de los conceptos que utilizaremos en este tema, los habrás revisado en los cursos de ecuaciones diferenciales y de álgebra lineal, por lo que en caso de que necesites darle un repaso, será conveniente para que no haya alguna complicación.
\end{frame}
\section{Temas a revisar}
\frame{\tableofcontents[currentsection, hideothersubsections]}
\subsection{Subtemas de trabajo}
\begin{frame}
\frametitle{Subtemas}
\setbeamercolor{item projected}{bg=blue!70!black,fg=yellow}
\setbeamertemplate{enumerate items}[circle]
\begin{enumerate}[<+->]
\item Ecuaciones diferenciales autoadjuntas.
\item Problemas de tipo Sturm-Liouville.
\item Operadores Hermitianos.
\item Funciones propias ortogonales.
\item Ortogonalización de Gram-Schmidt.
\item Completes de una base.
\end{enumerate}
\end{frame}
\subsection{Subtemas complementarios}
\begin{frame}
\frametitle{Temas complementarios}
Considerando que se va a requerir material adicional para tener una base sólida de conocimiento, se proponen dos temas complementarios:
\setbeamercolor{item projected}{bg=blue!70!black,fg=yellow}
\setbeamertemplate{enumerate items}[circle]
\begin{enumerate}[<+->]
\item Función delta de Dirac.
\item Espacio de Hilbert.
\end{enumerate}
\end{frame}
\begin{frame}
\frametitle{Sobre los temas complementarios}
El tema de la \emph{función delta de Dirac} es importante ya que vamos a apoyarnos bastante en lo que sigue del curso, por lo que recomendamos ampliamente que lo revisen y trabajen los ejercicios opcionales para repasar.
\end{frame}
\section{Tiempos de trabajo}
\frame{\tableofcontents[currentsection, hideothersubsections]}
\subsection{Trabajo por semana}
\begin{frame}
\frametitle{Distribución de tiempos}
Para una revisión completa de los materiales de trabajo para el Tema 3, se presenta la siguiente distribución de tiempos:
\end{frame}
\begin{frame}
\frametitle{Distribución de tiempos}
\setbeamercolor{item projected}{bg=blue!70!black,fg=yellow}
\setbeamertemplate{enumerate items}[circle]
\begin{enumerate}
\item Los subtemas Ecuaciones diferenciales autoadjuntas, Problemas de tipo Sturm-Liouville y Operadores Hermitianos.
\end{enumerate}
Se deberán de revisar en la Semana 6 (del jueves 29/10) y en la Semana 7 (del martes 3 al viernes 6 de noviembre)
\end{frame}
\begin{frame}
\frametitle{Distribución de tiempos}
\setbeamercolor{item projected}{bg=blue!70!black,fg=yellow}
\setbeamertemplate{enumerate items}[circle]
\begin{enumerate}
\item Los subtemas Funciones propias ortogonales, Ortogonalización de Gram-Schmidt y Completes de una base.
\end{enumerate}
Se deberán de revisar en la Semana 8 (del lunes 9/11 al viernes 13/11)
\end{frame}
\section{Apoyo a la calificación}
\frame{\tableofcontents[currentsection, hideothersubsections]}
\subsection{Nuevos puntajes en los ejercicios}
\begin{frame}
\frametitle{Apoyo en la puntuación (1/2)}
Se continuará con los ejercicios opcionales para cada semana, con la finalidad de incentivar el trabajo, se cambia el puntaje adicional, siendo ahora que se podrá obtener hasta $2$ puntos adicionales que se acumulará en el puntaje de la semana (pasamos de $0.5$ a $2$ puntos.)
\end{frame}
\begin{frame}
\frametitle{Apoyo en la puntuación (2/2)}
Para los dos temas complementarios se dejarán una serie de ejercicios opcionales de tal manera que podrán obtener hasta $1$ punto adicional que se acumulará en el puntaje de la semana.
\end{frame}
\begin{frame}
\frametitle{Sesiones síncronas de trabajo}
Se continuará con las sesiones de trabajo los días miércoles: 4 y 11 de noviembre a las 3 pm mediante la plataforma Zoom, considerando que ya hayan leído y revisado los materiales.
\end{frame}
\section{Punto importante}
\frame{\tableofcontents[currentsection, hideothersubsections]}
\subsection{Mensaje para considerar}
\begin{frame}
\frametitle{Avance en el curso}
Ya hemos concluido dos de los seis temas del curso, recordando que la primera evaluación corresponde a los tres primeros temas.
\end{frame}
\begin{frame}
\frametitle{Examen-Tarea}
Se hará el envío por correo con el archivo pdf con las preguntas del examen tarea correspondiente a los Temas 1 y 2.
\\
\bigskip
\pause
Con la finalidad de que los atiendan oportunamente, los resuelvan y \textbf{\textcolor{red}{los envíen de regreso en dos semanas}}.
\end{frame}
\begin{frame}
\frametitle{Acuse de recibido}
Se solicitará el acuse de recibido, es decir, esperamos una respuesta por parte de ustedes, de haber recibido el mensaje y el archivo adjunto.
\\
\bigskip
\pause
Esto nos dará la evidencia de que recibieron lo necesario para responder parte del primer examen parcial.
\end{frame}
\begin{frame}
\frametitle{Preguntas del Tema 3}
Conforme vayamos avanzando en el Tema 3, se les proporcionarán los ejercicios del examen-tarea de este tema.
\\
\bigskip
\pause
El punto es que deben de medir bien su carga de trabajo y aplicar un buen esfuerzo para estas actividades necesarias de evaluación.
\end{frame}
\end{document}