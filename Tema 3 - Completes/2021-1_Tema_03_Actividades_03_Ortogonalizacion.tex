\documentclass[12pt]{article}
\usepackage[left=0.25cm,top=1cm,right=0.25cm,bottom=1cm]{geometry}
\textwidth = 20cm
\hoffset = -1cm
\usepackage[utf8]{inputenc}
\usepackage[spanish,es-tabla]{babel}
\usepackage[autostyle,spanish=mexican]{csquotes}
\usepackage[tbtags]{amsmath}
\usepackage{nccmath}
\usepackage{amsthm}
\usepackage{amssymb}
\usepackage{graphicx}
\usepackage{standalone}
\usepackage[outdir=./]{epstopdf}
\usepackage{siunitx}
\usepackage{physics}
\usepackage{color}
\usepackage{float}
\usepackage{multicol}
%\usepackage{milista}
\usepackage{enumitem}
\usepackage{anyfontsize}
\usepackage{anysize}
\usepackage{enumitem}
\usepackage{capt-of}
\usepackage{bm}
\usepackage{relsize}
\usepackage{placeins}
\usepackage{empheq}
\usepackage{cancel}
\usepackage{wrapfig}
\spanishdecimal{.}
\renewcommand{\baselinestretch}{1.5} 
\renewcommand\labelenumii{\theenumi.{\arabic{enumii}}}
\newcommand{\ptilde}[1]{\ensuremath{{#1}^{\prime}}}
\newcommand{\stilde}[1]{\ensuremath{{#1}^{\prime \prime}}}
\newcommand{\ttilde}[1]{\ensuremath{{#1}^{\prime \prime \prime}}}
\newcommand{\ntilde}[2]{\ensuremath{{#1}^{(#2)}}}


\usepackage{apacite}
\title{Ejercicios opcionales - Ortogonalización y completez\\[0.3em]  \large{Tema 3 - Matemáticas Avanzadas de la Física}\vspace{-3ex}}
\author{M. en C. Gustavo Contreras Mayén}
\date{ }
\setlist[enumerate,1]{label=\arabic*.}
\setlist[enumerate,2]{label*=\arabic*.}
\begin{document}
\vspace{-4cm}
\maketitle
\fontsize{14}{14}\selectfont
Recuerda que en esta semana tendrás habilitado el espacio para respuestas: si la entrega se hace el próximo día domingo 15 de noviembre a las 12 del día, recibirás 2 puntos adicionales (en caso de que estén bien resueltos), en caso de que envíes los ejercicios al día 22 de noviembre a las 12 del día la puntuación máxima será de 1 punto.
\par
Te recomendamos que descargues el pdf y resuelvas cada inciso, cuando ya tengas la respuesta, anótala en la plataforma.

\begin{enumerate}
\item Demuestra que los siguientes conjuntos de funciones forman conjuntos ortogonales en los intervalos dados:
\begin{enumerate}
\item $\left\{ 1, \cos x, \cos 2 x, \cos 3 x, \ldots\right\}  \hspace{1.5cm} \mbox{ en } 0 \leq x \leq \pi$\label{inciso_01}
\item $\left\{ \sin \pi x, \sin 2 \pi x, \sin 3 \pi x, \ldots \right\} \hspace{1.5cm} \mbox{ en } -1 \leq x \leq 1$\label{inciso_02}
\item $\left\{ 1, 1 - x, 1 - 2 \, x + \dfrac{1}{2} \, x^{2} \right\} \hspace{1.5cm} \mbox{ para } 0 \leq x < \infty$\label{inciso_03}
\item Para los incisos \ref{inciso_01} y \ref{inciso_02} determina los correspondientes conjuntos de funciones ortonormales con $w(x) = 1$; para el inciso \ref{inciso_03}, usa $w(x) = e^{-x}$. Discute si cada conjunto forma o no, un conjunto completo.
\end{enumerate}
\item Mediante la técnica de Gram-Schmidt genera los tres primeros polinomios de Laguerre con lo siguiente:
\begin{align*}
u_{n}(x) = x^{n} \hspace{1cm} n = 0, 1, 2, \ldots, \hspace{1cm} 0 \leq x < \infty, \hspace{1cm} \omega(x) = e^{-x}
\end{align*}
La normalización convencional es:
\begin{align*}
\int_{0}^{\infty} L_{m}(x) \, L_{n}(x) \, e^{-x} \dd{x} = \delta_{mn}
\end{align*}
\end{enumerate}

\end{document}