\input{../Preambulos/preambulo_presentacion_Warsaw_crane}
\title{\large{Tema 3 - Bases completas y ortogonales}}
\subtitle{Ejercicios}
\author{M. en C. Gustavo Contreras Mayén}
\date{}
\institute{Facultad de Ciencias - UNAM}
\titlegraphic{\includegraphics[width=1.75cm]{../Imagenes/escudo-facultad-ciencias}\hspace*{4.75cm}~%
   \includegraphics[width=1.75cm]{../Imagenes/escudo-unam}
}
\setbeamertemplate{navigation symbols}{}
\begin{document}
\maketitle
\fontsize{14}{14}\selectfont
\spanishdecimal{.}
\section*{Contenido}
\frame[allowframebreaks]{\tableofcontents[currentsection, hideallsubsections]}
\section{Introducción}
\frame{\tableofcontents[currentsection, hideothersubsections]}
\subsection{Relevancia del trabajo previo}
\begin{frame}
\frametitle{Introducción}
En los temas anteriores del curso hemos preparado progresivamente el terreno para este Tema 3, que consideramos muy importante para el curso de MAF.
\end{frame}
\begin{frame}
\frametitle{Introducción}
\setbeamercolor{item projected}{bg=blue!70!black,fg=yellow}
\setbeamertemplate{enumerate items}[circle]
\begin{enumerate}[<+->]
\item Hemos escrito los operadores diferenciales básicos, en particular el Laplaciano, en coordenadas curvilíneas ortogonales.
\item La técnica de separación de variables nos ha permitido obtener un conjunto de EDO lineales y homogéneas cuya solución es a menudo expresable como una familia de funciones ortogonales.
\seti
\end{enumerate}
\end{frame}
\begin{frame}
\frametitle{Introducción}
\setbeamercolor{item projected}{bg=blue!70!black,fg=yellow}
\setbeamertemplate{enumerate items}[circle]
\begin{enumerate}[<+->]
\conti
\item La técnica de separación de variables no solo provee EDO, sino también constantes de separación.
\item Las soluciones y las constantes de separación están tan estrechamente ligadas a las condiciones iniciales o de frontera que no es posible estudiar la estructura de la ecuación diferencial sin hacer conjuntamente estas consideraciones.
\end{enumerate}
\end{frame}
\begin{frame}
\frametitle{Introducción}
En el Tema 3 se estudia el problema global que considera una ecuación diferencial, los valores propios (\emph{eigenvalores o autovalores)}, las funciones propias \emph{(eigenfunciones o autofunciones)}, la ortogonalidad de las soluciones y las condiciones de frontera pertinentes.
\\
\bigskip
\pause
Es este el \textbf{problema de Sturm-Liouville}.
\end{frame}
\subsection{Base conceptual}
\begin{frame}
\frametitle{Operadores}
De la teoría de ecuaciones diferenciales nos apoyaremos con la definición de un operador lineal $\mathcal{L}$ como:
\begin{align*}
\mathcal{L} = a_{2}(x) \, \stilde{y} + a_{1} \, \ptilde{y} + a_{0} \, y
\end{align*}
\pause
Mientras que el operador adjunto $\mathcal{L}^{\dagger}$ es:
\begin{align*}
\mathcal{L}^{\dagger} = \stilde{(a_{2} \, y)} + \ptilde{(a_{1} \, y)} + a_{0} \, y
\end{align*}
\end{frame}
\begin{frame}
\frametitle{Operador autoadjunto}
Decimos que un operador es autoadjunto cuando ocurre:
\begin{align*}
\mathcal{L} = \mathcal{L}^{\dagger}
\end{align*}
\pause
Se puede demostrar que un operador autoadjunto es de la forma
\begin{align*}
\mathcal{L} = \dv{x} \left[ p(x) \dv{x} \right] + \omega (x) \, q(x)
\end{align*}
\end{frame}
\begin{frame}
\frametitle{Operador autoadjunto}
Forma de Sturm-Liouville
\begin{align*}
\mathcal{L} = \dv{x} \left[ p(x) \dv{x} \right] + \omega (x) \, q(x)
\end{align*}
Donde:
\begin{itemize}[<+->]
\item $p(x)$ es una función real y derivable en un intervalo $[a, b]$.
\item $q(x)$ y $\omega (x)$ son funciones reales y continuas en $[a, b]$.
\item $\omega (x) > 0$ en $[a, b]$.
\end{itemize}
\end{frame}
\begin{frame}
\frametitle{Caso con coeficientes reales}
Para que un operador de segundo orden con coeficientes \emph{reales} sea autoadjunto, debe de ocurrir que:
\begin{align*}
\ptilde{a}_{2} = a_{1}
\end{align*}
\pause
Se debe de considerar también el caso de tener coeficientes complejos, por lo que habría que ajustar la definición.
\end{frame}  
\begin{frame}
\frametitle{Ejemplo 1}
La ecuación de Legendre
\begin{align*}
(1 - x^{2}) \, \stilde{y} - 2 \, x \, \ptilde{y} + \ell (\ell +  1) \, y = 0
\end{align*}
se tiene que:
\begin{itemize}[<+->]
\item $a_{2} = 1 -x^{2}$
\item $a_{1} = 2 \, x$
\end{itemize}
\end{frame}
\begin{frame}
\frametitle{Ejemplo 1}
Y vemos que $\ptilde{a}_{2} = a_{1}$
\\
\bigskip
\pause
Por tanto el operador
\begin{align*}
\mathcal{L} = (1 - x^{2}) \, \dv[2]{x} - 2 \, x \, \dv{x} + \ell(\ell + 1)
\end{align*}
es \textbf{autoadjunto}. \pause Por lo tanto, la ecuación de Legendre es autoadjunta.
\end{frame}
\begin{frame}
\frametitle{Ejemplo 2}
Para la ecuación de Bessel
\begin{align*}
x^{2} \, \stilde{y} + x \, \ptilde{y} + (k^{2} \, x^{2} - n^{2}) \, y = 0
\end{align*}
\pause
se tiene que:
\begin{itemize} [<+->]
\item $a_{2} = x^{2}$
\item $a_{1} = x$
\end{itemize}
\pause
por lo que $\ptilde{a}_{2} \neq a_{1}$. \pause Entonces la ecuación \emph{no es autoadjunta}.
\end{frame}
\begin{frame}
\frametitle{Ejemplo 2}
Pero si dividimos la ecuación de Bessel por $x$, se obtendrá una ecuación en donde, para los nuevos $a_{2}$ y $a_{1}$, \pause se tiene que
\begin{itemize} [<+->]
\item $a_{2} = x$
\item $a_{1} = 1$
\end{itemize}
\pause
Entonces $\ptilde{a}_{2} = a_{1}$. \pause Entonces la ecuación \emph{es autoadjunta}.
\end{frame}
\begin{frame}
\frametitle{¿Cómo hacer autoadjunta una ED?}
Podemos plantear la siguiente pregunta: ¿Hay forma de calcular un factor multiplicativo que vuelva autoadjunta a cualquier ecuación diferencial lineal?
\end{frame}
\begin{frame}
\frametitle{Teorema importante}
Hay un teorema de suma importancia en la teoría de ecuaciones diferenciales:
\\
\bigskip
\emph{Es siempre posible convertir una ecuación diferencial lineal de segundo orden no homogénea con coeficientes reales en una ecuación autoadjunta}.
\end{frame}
\begin{frame}
\frametitle{Cambiando a una ec. autoadjunta}
La ecuación diferencial
\begin{align*}
a_{2}(x) \, \stilde{y} + a_{1} \, \ptilde{y} + a_{0} \, y =  f(x)
\end{align*}
se puede expresar en la forma autoadjunta (Sturm Liouville):
\begin{align*}
\mathcal{L} = \dv{x} \left[ p(x) \dv{x} \right] + \omega (x) \, q(x) =  F(x)
\end{align*}    
\end{frame}
\begin{frame}
\frametitle{Cambiando a una ec. autoadjunta}
Donde 
\begin{align*}
p(x) &=  \exp\left( \int \dfrac{a_{1}(x)}{a_{2}(x)} \dd{x} \right) \\[0.5em]
q(x) &= p(x) \, \dfrac{a_{0}(x)}{a_{2}(x)} \\[0.5em]
F(x) &= p(x) \, \dfrac{f(x)}{a_{2}}
\end{align*}
\end{frame}
\begin{frame}
\frametitle{Ejemplo 3}
La siguiente ecuación no es autoadjunta, conviértela a autoadjunta:
\begin{align*}
x^{2} \, \stilde{y} + x \, \ptilde{y} + 2 \, y = 0
\end{align*}
\pause
Se tiene que:
\begin{align*}
a_{2}(x) &= x^{2} \\[0.5em]
a_{1}(x) &= x \\[0.5em]
a_{0}(x) &= 2
\end{align*}
\end{frame}
\begin{frame}
\frametitle{Ejemplo 3}
Por lo que:
\begin{eqnarray*}
p(x) &=&  \exp\left( \int \dfrac{x}{x^{2}} \dd{x} \right) = \pause \exp(\ln x) =  x \\[0.5em] \pause
q(x) &=& x \, \dfrac{2}{x^{2}} = \dfrac{2}{x} \\[0.5em]
F(x) &=& x \, \dfrac{0}{x^{2}} = 0
\end{eqnarray*}
\end{frame}
\begin{frame}
\frametitle{Ejemplo 3}
Así la ED autoadjunta o en la forma de Sturm Liouville es:
\begin{align*}
\dv{x} \bigg[ x \, \ptilde{y} \bigg] + \dfrac{2}{x} \, y = 0
\end{align*}
\end{frame}
\begin{frame}
\frametitle{Producto escalar}
Ahora vamos a requerir de un producto escalar en un espacio de funciones, que sean integrables en un dominio $[a ,b]$:
\begin{align*}
\braket{f}{g} = \int_{a}^{b} f^{*}(x) \, g(x) \, \omega(x) \dd{x}
\end{align*}
donde $f^{*}(x)$ es el conjugado complejo de la función, cuando se trabaja con funciones reales, se omite la marca del conjungado.
\end{frame}
\begin{frame}
\frametitle{Operador Hermítico}
Dado un operador autoadjunto y el producto escalar mencionado antes, decimos que un operador es \emph{Hermítico} si:
\begin{align*}
\braket{\mathcal{L} \, f}{g} = \braket{f}{\mathcal{L} \, g}
\end{align*}
\end{frame}
\begin{frame}
\frametitle{Condiciones de frontera}
Para que un operador sea Hermítico debemos de poner especial atención en las condiciones de frontera, ya que debe de presentarse alguna de las siguientes:
\setbeamercolor{item projected}{bg=blue!70!black,fg=yellow}
\setbeamertemplate{enumerate items}[circle]
\begin{enumerate}[<+->]
\item $p(a) = p(b)$
\seti
\end{enumerate}
\end{frame}
\begin{frame}
\frametitle{Condiciones de frontera}
\setbeamercolor{item projected}{bg=blue!70!black,fg=yellow}
\setbeamertemplate{enumerate items}[circle]
\begin{enumerate}[<+->]
\conti
\item 
\begin{align*}
\alpha_{1} \, f(a) + \alpha_{2} \, \ptilde{f} (a) &= 0 \\[0.5em]
\beta_{1} \, f(b) + \beta_{2} \, \ptilde{f} (b) &= 0
\end{align*}
\seti
\end{enumerate}
Las CDF deben de ser homogéneas y no mixtas, es decir, no se mezclan las condiciones en los extremos del dominio.
\end{frame}
\begin{frame}
\frametitle{Condiciones de frontera}
\setbeamercolor{item projected}{bg=blue!70!black,fg=yellow}
\setbeamertemplate{enumerate items}[circle]
\begin{enumerate}[<+->]
\conti
\item 
\begin{align*}
f(a) &= f(b) \\[0.5em]
\ptilde{f}(a) &= \ptilde{f}(b)
\end{align*}
Que sería el caso de un problema periódico.
\end{enumerate}
\end{frame}
\begin{frame}
\frametitle{El problema de valores propios}
Si un operador es autoadjunto y Hermítico, entonces el problema de un operador $\mathcal{L}$ tal que:
\begin{align*}
\mathcal{L} [y] + \lambda \, y = 0
\end{align*}
es el problema semejante cuando se diagonaliza una matriz: la matriz sería el operador, que correspondería un problema de valores propios del álgebra lineal.
\end{frame}
\begin{frame}
\frametitle{El problema de valores propios}
Tenemos entonces dos primeros resultados relevantes:
\begin{align*}
\mathcal{L} [u_{i}] + \lambda \, u_{i} = 0
\end{align*}
Se tiene que:
\begin{itemize}[<+->]
\item $\lambda_{i} \in \mathbb{R}$.
\item las $u_{i}$ son ortogonales: $\braket{u_{i}}{u_{j}} = 0$ si $\lambda_{i} \neq \lambda_{j}$
\end{itemize}
\end{frame}
\begin{frame}
\frametitle{Problemas de Sturm Liouville}
Problemas de este tipo surgen, usualmente, mediante separación de variables en las ecuaciones homogéneas que involucran laplacianos.
\end{frame}
\begin{frame}
\frametitle{Problemas de Sturm Liouville}
Del estudio de las ecuaciones de Legendre, Bessel, Hermite, Laguerre, etc. que son del tipo de Sturm Liouville, se sigue que existen soluciones $f(x)$ que satisfacen la ED y las CDF solo para ciertos valores $\lambda_{m}$ del parámetro $\lambda$, correspondientes a ciertas funciones $f_{m(x)}$.
\end{frame}
\begin{frame}
\frametitle{Problemas de Sturm Liouville}
Para cada valor propio $\lambda_{m}$ hay una función propia $f_{m}(x)$.
\\
\bigskip
\pause
El conjunto de funciones $\left\{ f_{m}(x) \right\}$ forma en muchos casos una base enumerable ($m$ entero) y ortogonal.
\end{frame}
\begin{frame}
\frametitle{Problemas de Sturm Liouville}
El problema asociado a valores propios, funciones propias y condiciones de frontera homogéneas se conoce como problema de Sturm Liouville.
\end{frame}
\begin{frame}
\frametitle{Ejemplo}
Consideremos la siguiente ecuación diferencial
\begin{align*}
\dv[2]{f(x)}{x} + k^{2} \, f = 0
\end{align*}
con las condiciones de frontera
\begin{align*}
f(0) = f(L) = 0
\end{align*}
\end{frame}
\begin{frame}
\frametitle{Solución general}
La solución general es:
\begin{align*}
f(x) = A \, \sin (kx) + B \, \cos (kx)
\end{align*}
\pause
Con $f(0) = 0$ tenemos que $B = 0$, por lo que:
\begin{align*}
f(x) = A \, \sin (kx)
\end{align*}
\end{frame}
\begin{frame}
\frametitle{Solución general}
Con $f(L) = 0$ se tiene que $\sin (kL) = 0$ por lo que $k = n \, \pi / L$, entonces:
\begin{align*}
f_{n} (x) = A_{n} \sin \left( \dfrac{n \, \pi \, x}{L} \right) \hspace{1cm} n = 1, 2, 3, \ldots
\end{align*}
\end{frame}
\begin{frame}
\frametitle{Solución general}
A cada valor de $n$ entero le corresponde un valor propio 
\begin{align*}
\lambda_{n} = k_{n}^{2} = \left( \dfrac{n \, \pi}{L} \right)^{2}
\end{align*}
 y una función propia $f_{n}(x)$, donde $n$ es un entero que numera la secuencia de valores propios y funciones propias.
\end{frame}
\begin{frame}
\frametitle{Interpretación física}
Este problema corresponde a la parte espacial que se obtiene de la ecuación de ondas unidimensional al realizar la separación de variables e imponer la restricción de que las soluciones sean armónicas.
\\
\bigskip
\pause
Físicamente corresponde al caso de las oscilaciones de una cuerda con extremos fijos.
\end{frame}
\end{document}