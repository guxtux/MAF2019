\documentclass[12pt]{article}
\usepackage[left=0.25cm,top=1cm,right=0.25cm,bottom=1cm]{geometry}
\textwidth = 20cm
\hoffset = -1cm
\usepackage[utf8]{inputenc}
\usepackage[spanish,es-tabla]{babel}
\usepackage[autostyle,spanish=mexican]{csquotes}
\usepackage[tbtags]{amsmath}
\usepackage{nccmath}
\usepackage{amsthm}
\usepackage{amssymb}
\usepackage{graphicx}
\usepackage{standalone}
\usepackage[outdir=./]{epstopdf}
\usepackage{siunitx}
\usepackage{physics}
\usepackage{color}
\usepackage{float}
\usepackage{multicol}
%\usepackage{milista}
\usepackage{enumitem}
\usepackage{anyfontsize}
\usepackage{anysize}
\usepackage{enumitem}
\usepackage{capt-of}
\usepackage{bm}
\usepackage{relsize}
\usepackage{placeins}
\usepackage{empheq}
\usepackage{cancel}
\usepackage{wrapfig}
\spanishdecimal{.}
\renewcommand{\baselinestretch}{1.5} 
\renewcommand\labelenumii{\theenumi.{\arabic{enumii}}}
\newcommand{\ptilde}[1]{\ensuremath{{#1}^{\prime}}}
\newcommand{\stilde}[1]{\ensuremath{{#1}^{\prime \prime}}}
\newcommand{\ttilde}[1]{\ensuremath{{#1}^{\prime \prime \prime}}}
\newcommand{\ntilde}[2]{\ensuremath{{#1}^{(#2)}}}


\title{Tema 3 - Lista de ejercicios a cuenta\\ \large{Matemáticas Avanzadas de la Física}\vspace{-3ex}}
\author{M. en C. Gustavo Contreras Mayén}
\date{ }
\begin{document}
\vspace{-4cm}
\maketitle
\fontsize{14}{14}\selectfont
\section{Presentación 1}
\begin{enumerate}
    \item Demuestra que la ecuación de Laguerre
\begin{align*}
x \, \stilde{y} + (1 - x) \, \ptilde{y} + n \, y = 0
\end{align*}
se puede escribir de una manera autoadjunta multiplicándola por la función $\exp(-x)$ y que la función de peso es $\omega(x) = \exp(-x)$.
\item Demuestra que la ecuación de Chebychev de tipo I
\begin{align*}
(1 -x^{2}) \, \stilde{y} - x \, \ptilde{y} + n^{2} \, y = 0
\end{align*}
se puede escribir en la forma autoadjunta. Determina la función de peso.
\item Demuestra que la suma de dos operadores Hermitianos es Hermitano.
\item Supongamos que el operador $\hat{Q}$ es Hermitiano y $\alpha$ es un número complejo. ¿Bajo qué condición (sobre $\alpha$) es $\alpha \, \hat{Q}$ Hermitiano?
\item ¿Cuándo el producto de dos operadores Hermitianos es Hermitiano?
\item Demuestra que el operador de posición $(\hat{x} = x)$ y el operador Hamiltoniano
\begin{align*}
\hat{H} = - \left( \dfrac{\hbar^{2}}{2 \, m} \right) \, \dv[2]{x} + V(x)
\end{align*}
es Hermitiano.
\end{enumerate}
\section{Presentación 2.}
\begin{enumerate}
\item Cuentas con los siguientes elementos:
\begin{enumerate}
\item Un conjunto de funciones $\left\{ u_{n} (x) \right\} = \left\{ x^{n} \right\}, \mbox{ con } n = 1, 2, \ldots$
\item El intervalo $(0, \infty)$
\item Una función de peso $w(x) = x \, e^{-x}$
\end{enumerate}
Con el método de Gram-Schmidt construye las primeras \textbf{tres funciones ortonormales} del conjunto $u_{n}(x)$, con ese intervalo dado y función de peso dada.
\item Demuestra que:
\begin{align*}
&\int_{-\infty}^{\infty} \left( t^{10} - t^{6} + 5 \, t^{4} - 5 \right) \, e^{-4} \dd{t} \leq \\
&\leq \sqrt{\int_{-\infty}^{\infty} \left( t^{4} - 1 \right)^{2} \, e^{-4} \dd{t}} \, \sqrt{\int_{-\infty}^{\infty} \left( t^{6} + 5 \right)^{2} \, e^{-4} \dd{t}}
\end{align*}
\item Determina las funciones que satisfacen la ecuación de valores propios
\begin{align*}
\hat{A} \, f(x) = \lambda \, f(x)
\end{align*}
cuando $\hat{A}$ es el operador que al aplicarse a una función, la eleva al cuadrado.
\end{enumerate}
\section{Ejercicios Opcionales.}
\begin{enumerate}
\item Considera las siguientes funciones de onda unidimensionales que están normalizadas: $\psi_{0}(x)$ y $\psi_{1}(x)$, que cuentan con las propiedades:
\begin{align*}
\psi_{0}(-x) &= \psi_{0}(x) = \psi_{0}^{*} (x) \\[0.5em]
\psi_{1}(x) &= N \, \dv{\psi_{0}}{x}
\end{align*}
Considera también la combinación lineal
\begin{align*}
\psi(x) = c_{1} \, \psi_{0}(x) + c_{2} \, \psi_{1} (x)
\end{align*}
con $\abs{c_{1}}^{2} + \abs{c_{2}}^{2} = 1$. Las constantes $N, c_{1}, c_{2}$, las consideramos conocidas.
\begin{enumerate}
Calcula los valores esperados de:
\item $\expval{\hat{x}}$,
\item $\expval{\hat{p}}$ 
\end{enumerate}
en el estado $\psi(x)$. (Nota: los otros dos incisos se resolvieron en la sesión de Zoom, quedando pendiente éste inciso)
\end{enumerate}
\section{Ejercicios Opcionales 1.}
\begin{enumerate}
\item Encuentra una expresión simplificada equivalente 
\begin{enumerate}
\item $\cos (t) + \sin (t) \, \delta(t)$
\item $\sin (t) + \cos (t) \, \delta(t)$
\end{enumerate}
\item Evalúa las siguientes integrales (simbólicas):
\begin{enumerate}
\item $\displaystyle\int_{-\infty}^{\infty} (t^{2} + 3 \, t + 5) \, \delta(t) \dd{t}$
\item $\displaystyle\int_{-\infty}^{\infty} \dfrac{\cos(x) \, \delta(x)}{2 \, e^{x} + 1} \dd{x}$
\item $\displaystyle\int_{-\infty}^{\infty} e^{-t^{2}}  \, \delta(t - 2) \dd{t}$
\end{enumerate}
\end{enumerate}
\section{Ejercicios Opcionales 2.}
\begin{enumerate}
\item Escribe las condiciones de ortonormalidad y completez para las siguiente bases de peso unitario:
\begin{align*}
\left\{ \varphi_{n} (x) \right\} = \left\{ \sqrt{\dfrac{2}{L}} \, \sin (\dfrac{n \pi x}{L}) \right\} \hspace{1.5cm} 0 \leq x \leq L\end{align*}
\item Considera los dos estados
\begin{align*}
\ket{\psi_{1}} &= \ket{\phi_{1}} + 4 \, i \, \ket{\phi_{2}} + 5 \, \ket{\phi_{3}} \\
\ket{\psi_{2}} &= b \, \ket{\phi_{1}} + 4 \, \ket{\phi_{2}} - 3 \, i \, \ket{\phi_{3}}
\end{align*}
donde $\ket{\phi_{1}}$, $\ket{\phi_{2}}$, $\ket{\phi_{3}}$, son ortonormales, y $b$ es una constante. Calcula el valor de $b$, para el cual, $\ket{\psi_{1}}$ y $\ket{\psi_{2}}$ son ortogonales.
\item Si $\ket{\phi_{1}}$, $\ket{\phi_{2}}$, $\ket{\phi_{3}}$, son ortonormales, demuestra que los estados
\begin{align*}
\ket{\psi} &= i \, \ket{\phi_{1}} + 3 \, i \, \ket{\phi_{2}} - \ket{\phi_{3}} \\
\ket{\chi} &= \ket{\phi_{1}} - i \, \ket{\phi_{2}} + 5 \, i \, \ket{\phi_{3}}
\end{align*}
Satisfacen:
\begin{enumerate}[label=\alph*)]
\item la desigualdad del triángulo.
\item la desigualdad de Schwarz.
\end{enumerate}
% \end{enumerate}
\item Demuestra que el conmutador de dos operadores Hermitianos es antiHermitiano.
\item Evalúa el conmutador:
\begin{align*}
[ \hat{A}, [\hat{B}, \hat{C}] \, \hat{D} ]
\end{align*}
\end{enumerate}
\section{Ejercicios Opcionales 3.}
\begin{enumerate}
\item Demuestra que los siguientes conjuntos de funciones forman conjuntos ortogonales en los intervalos dados:
\begin{enumerate}
\item $\left\{ 1, \cos x, \cos 2 x, \cos 3 x, \ldots\right\}  \hspace{1.5cm} \mbox{ en } 0 \leq x \leq \pi$\label{inciso_01}
\item $\left\{ \sin \pi x, \sin 2 \pi x, \sin 3 \pi x, \ldots \right\} \hspace{1.5cm} \mbox{ en } -1 \leq x \leq 1$\label{inciso_02}
\item $\left\{ 1, 1 - x, 1 - 2 \, x + \dfrac{1}{2} \, x^{2} \right\} \hspace{1.5cm} \mbox{ para } 0 \leq x < \infty$\label{inciso_03}
\item Para los incisos \ref{inciso_01} y \ref{inciso_02} determina los correspondientes conjuntos de funciones ortonormales con $w(x) = 1$; para el inciso \ref{inciso_03}, usa $w(x) = e^{-x}$. Discute si cada conjunto forma o no, un conjunto completo.
\end{enumerate}
\item Mediante la técnica de Gram-Schmidt genera los tres primeros polinomios de Laguerre con lo siguiente:
\begin{align*}
u_{n}(x) = x^{n} \hspace{1cm} n = 0, 1, 2, \ldots, \hspace{1cm} 0 \leq x < \infty, \hspace{1cm} \omega(x) = e^{-x}
\end{align*}
La normalización convencional es:
\begin{align*}
\int_{0}^{\infty} L_{m}(x) \, L_{n}(x) \, e^{-x} \dd{x} = \delta_{mn}
\end{align*}
\end{enumerate}
\end{document}