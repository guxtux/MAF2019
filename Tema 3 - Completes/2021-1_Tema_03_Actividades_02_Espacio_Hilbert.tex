\documentclass[12pt]{article}
\usepackage[left=0.25cm,top=1cm,right=0.25cm,bottom=1cm]{geometry}
\textwidth = 20cm
\hoffset = -1cm
\usepackage[utf8]{inputenc}
\usepackage[spanish,es-tabla]{babel}
\usepackage[autostyle,spanish=mexican]{csquotes}
\usepackage[tbtags]{amsmath}
\usepackage{nccmath}
\usepackage{amsthm}
\usepackage{amssymb}
\usepackage{graphicx}
\usepackage{standalone}
\usepackage[outdir=./]{epstopdf}
\usepackage{siunitx}
\usepackage{physics}
\usepackage{color}
\usepackage{float}
\usepackage{multicol}
%\usepackage{milista}
\usepackage{enumitem}
\usepackage{anyfontsize}
\usepackage{anysize}
\usepackage{enumitem}
\usepackage{capt-of}
\usepackage{bm}
\usepackage{relsize}
\usepackage{placeins}
\usepackage{empheq}
\usepackage{cancel}
\usepackage{wrapfig}
\spanishdecimal{.}
\renewcommand{\baselinestretch}{1.5} 
\renewcommand\labelenumii{\theenumi.{\arabic{enumii}}}
\newcommand{\ptilde}[1]{\ensuremath{{#1}^{\prime}}}
\newcommand{\stilde}[1]{\ensuremath{{#1}^{\prime \prime}}}
\newcommand{\ttilde}[1]{\ensuremath{{#1}^{\prime \prime \prime}}}
\newcommand{\ntilde}[2]{\ensuremath{{#1}^{(#2)}}}


\usepackage{apacite}
\geometry{top=1.25cm, bottom=1.5cm, left=1.25cm, right=1.25cm}
\title{Espacio de Hilbert y Notación de Dirac \\[0.3em]  \large{Tema 3 - Ejercicios opcionales}\vspace{-3ex}}
\author{M. en C. Gustavo Contreras Mayén}
\date{ }
\begin{document}
\vspace{-4cm}
\maketitle
\fontsize{14}{14}\selectfont
Recuerda que en esta semana tendrás habilitado el espacio para respuestas, el próximo día miércoles 11 de noviembre se cerrará la recepción a las 12 del día.
\par
Cada ejercicio vale \textbf{0.2 puntos}. Te recomendamos que descargues el pdf y resuelvas cada inciso, cuando ya tengas la respuesta, anótala en la plataforma.

\begin{enumerate}
\item Escribe las condiciones de ortonormalidad y completez para las siguiente bases de peso unitario:
\begin{align*}
\left\{ \varphi_{n} (x) \right\} = \left\{ \sqrt{\dfrac{2}{L}} \, \sin (\dfrac{n \pi x}{L}) \right\} \hspace{1.5cm} 0 \leq x \leq L\end{align*}
\item Considera los dos estados
\begin{align*}
\ket{\psi_{1}} &= \ket{\phi_{1}} + 4 \, i \, \ket{\phi_{2}} + 5 \, \ket{\phi_{3}} \\
\ket{\psi_{2}} &= b \, \ket{\phi_{1}} + 4 \, \ket{\phi_{2}} - 3 \, i \, \ket{\phi_{3}}
\end{align*}
donde $\ket{\phi_{1}}$, $\ket{\phi_{2}}$, $\ket{\phi_{3}}$, son ortonormales, y $b$ es una constante. Calcula el valor de $b$, para el cual, $\ket{\psi_{1}}$ y $\ket{\psi_{2}}$ son ortogonales.
\item Si $\ket{\phi_{1}}$, $\ket{\phi_{2}}$, $\ket{\phi_{3}}$, son ortonormales, demuestra que los estados
\begin{align*}
\ket{\psi} &= i \, \ket{\phi_{1}} + 3 \, i \, \ket{\phi_{2}} - \ket{\phi_{3}} \\
\ket{\chi} &= \ket{\phi_{1}} - i \, \ket{\phi_{2}} + 5 \, i \, \ket{\phi_{3}}
\end{align*}
Satisfacen:
\begin{enumerate}[label=\alph*)]
\item la desigualdad del triángulo.
\item la desigualdad de Schwarz.
\end{enumerate}
% \end{enumerate}
\item Demuestra que el conmutador de dos operadores Hermitianos es antiHermitiano.
\item Evalúa el conmutador:
\begin{align*}
[ \hat{A}, [\hat{B}, \hat{C}] \, \hat{D} ]
\end{align*}

\end{enumerate}

\end{document}