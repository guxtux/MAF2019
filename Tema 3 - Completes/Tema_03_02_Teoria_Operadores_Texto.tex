\documentclass[12pt]{article}
\usepackage[utf8]{inputenc}
\usepackage[T1]{fontenc}
\usepackage[spanish,es-lcroman]{babel}
\usepackage{amsmath}
\usepackage{amsthm}
\usepackage{amsfonts}
\usepackage{amssymb}
\usepackage{physics}
\AtBeginDocument{\RenewCommandCopy\qty\SI}
\usepackage{tikz}
\usepackage{float}
\usepackage{calc}
\usepackage[autostyle,spanish=mexican]{csquotes}
\usepackage[per-mode=symbol]{siunitx}
\usepackage{textcomp, gensymb}
\usepackage{multicol}
\usepackage{enumitem}
\usepackage{hyperref}
\usepackage{setspace}
\usepackage[left=2.00cm, right=2.00cm, top=2.00cm, 
     bottom=2.00cm]{geometry}
% \usepackage{Estilos/ColoresLatex}
\usepackage{makecell}
\usepackage{subcaption}
\usepackage[skip=10pt, indent=30pt]{parskip}
% \usepackage{scalerel}
\usepackage{scalerel}[2016-12-29]
\usepackage{biblatex}
\usepackage{cancel}

\definecolor{ao}{rgb}{0.0, 0.0, 1.0}
\definecolor{burgundy}{rgb}{0.5, 0.0, 0.13}

\hypersetup{
    colorlinks=true,
    linkcolor=ao,
    filecolor=magenta,      
    urlcolor=ao,
}

\newcommand{\ptilde}[1]{\ensuremath{{#1}^{\prime}}}
\newcommand{\stilde}[1]{\ensuremath{{#1}^{\prime \prime}}}
\newcommand{\ttilde}[1]{\ensuremath{{#1}^{\prime \prime \prime}}}
\newcommand{\ntilde}[2]{\ensuremath{{#1}^{(#2)}}}
\newcommand{\pderivada}[1]{\ensuremath{{#1}^{\prime}}}
\newcommand{\sderivada}[1]{\ensuremath{{#1}^{\prime \prime}}}
\newcommand{\tderivada}[1]{\ensuremath{{#1}^{\prime \prime \prime}}}
\newcommand{\nderivada}[2]{\ensuremath{{#1}^{(#2)}}}

\def\stretchint#1{\vcenter{\hbox{\stretchto[440]{\displaystyle\int}{#1}}}}
\def\scaleint#1{\vcenter{\hbox{\scaleto[3ex]{\displaystyle\int}{#1}}}}
\def\scaleiint#1{\vcenter{\hbox{\scaleto[6ex]{\displaystyle\iint}{#1}}}}
\def\scaleiiint#1{\vcenter{\hbox{\scaleto[6ex]{\displaystyle\iiint}{#1}}}}
\def\scaleoint#1{\vcenter{\hbox{\scaleto[3ex]{\displaystyle\oint}{#1}}}}
\def\bs{\mkern-12mu}

% \newcommand{\textbf}[2]{\textbf{\textcolor{#1}{#2}}}
\sisetup{per-mode=symbol}
\decimalpoint
\sisetup{bracket-numbers = false}
\newlength{\depthofsumsign}
\setlength{\depthofsumsign}{\depthof{$\sum$}}
\newcommand{\nsum}[1][1.4]{% only for \displaystyle
    \mathop{%
        \raisebox
            {-#1\depthofsumsign+1\depthofsumsign}
            {\scalebox
                {#1}
                {$\displaystyle\sum$}%
            }
    }
}

\AtBeginDocument{\RenewCommandCopy\qty\SI}
\ExplSyntaxOn
\msg_redirect_name:nnn { siunitx } { physics-pkg } { none }
\ExplSyntaxOff

\numberwithin{equation}{section}

\linespread{1.25}

\renewcommand{\labelenumii}{\theenumii}
\renewcommand{\theenumii}{\theenumi.\arabic{enumii}.}

\emergencystretch=1em

\title{\large{Teoría Operadores Sturm-Liouville}}
\author{M. en C. Gustavo Contreras Mayén}
\date{}


\begin{document}
\maketitle
\fontsize{14}{14}\selectfont
\spanishdecimal{.}
\tableofcontents
\newpage

\section{Operadores autoadjuntos.}
\subsection{Introducción.}

En el estudio de la teoría espectral de matrices, se aprende sobre el adjunto de la matriz, $\vb{A}^{\dagger}$, y el papel que juegan las matrices autoadjuntas o Hermitianas en la diagonalización.
\par
Además, se necesita el concepto del \textbf{adjunto} para discutir la existencia de soluciones al problema matricial:
\begin{align*}
\vb{y} = \vb{A} \, \vb{x}
\end{align*}
En el mismo sentido, uno está interesado en la existencia de soluciones de la ecuación del operador $L \, u = f$ y soluciones del correspondiente problema de valores propios. El estudio de operadores lineales en un espacio de Hilbert es una generalización de lo que estudia en un curso de álgebra lineal. Así como se puede encontrar una base de vectores propios y diagonalizar matrices Hermitianas o autoadjuntas (o simétricas reales en el caso de matrices reales), veremos que el operador de Sturm-Liouville es \textbf{autoadjunto}.
\par
En esta parte del Tema 3 definiremos el \textbf{dominio} de un operador e introduciremos la noción de \textbf{operadores adjuntos}. Veremos el papel que juega el adjunto en la existencia de soluciones a la ecuación del operador $L \, u = f$.

\subsection{El operador adjunto.}

Comenzamos definiendo el adjunto de un operador:  el adjunto, $L^{\dagger}$, del operador $L$ satisface:
\begin{align*}
\langle u, L \, v \rangle = \langle L^{\dagger} \, u,  v \rangle
\end{align*}
para todo $v$ en el dominio de $L$ y $u$ en el dominio de $L^{\dagger}$. Aquí, el dominio de un operador diferencial $L$ es el conjunto de todos $u \in L_{\sigma}^{2} (a, b)$ que satisfacen un conjunto dado de CDF homogéneas. Esto se comprenderá mejor con el siguiente ejemplo.
\par
\noindent
\textbf{Ejemplo de operador adjunto: } Encuentra el adjunto de:
\begin{align*}
L = a_{2}(x) \, D^{2} + a_{1}(x) \, D + a_{0}(x)
\end{align*}
con $D = \dv*{x}$.
Para encontrar el adjunto, colocamos el operador dentro de una integral. Consideremos el producto interior:
\begin{align*}
\langle u , L \, v \rangle = \scaleint{6ex}_{\bs a}^{b} u (a_{2} \, \sderivada{v} + a_{1} \, \pderivada{v} + a_{0} \, v) \dd{x}
\end{align*}
Tenemos que \enquote{mover} el operador $L$ de $v$  y determinar qué operador está actuando sobre $u$ para preservar formalmente el producto interno. Para un operador simple como $L = \dv*{x}$, esto se hace fácilmente mediante la integración por partes.
\par
Para el operador dado en el ejemplo, necesitaremos aplicar varias integraciones por partes a los términos individuales. Consideramos cada término derivado en el integrando por separado. Para el término $a_{1} \pderivada{v}$, integramos por partes para encontrar:
\begin{eqnarray}
\begin{aligned}[b]
\scaleint{6ex}_{\bs a}^{b} \, u(x) \, &a_{1} \, \pderivada{v}(x) \dd{x} = a_{1}(x) \, u(x) \, v(x) \eval_{a}^{b} + \\[0.5em]
&- \scaleint{6ex}_{\bs a}^{b} \big[ u(x) \, a_{1} (x) \big]^{\prime} \, v(x) \dd{x}
\end{aligned}
\label{eq:ecuacion_04_17}
\end{eqnarray}
Ahora consideremos el caso para el término $a_{2} \, \sderivada{v}$, en donde será necesario hacer dos integraciones por partes:
\begin{eqnarray}
\begin{aligned}[b]
&\scaleint{6ex}_{\bs a}^{b} \, u(x) \, a_{2} (x) \, \sderivada{v}(x) \dd{x} = a_{2}(x) \, u(x) \, \pderivada{v}(x) \eval_{a}^{b} - \scaleint{6ex}_{\bs a}^{b} \big[ u(x) \, a_{2} (x) \big]^{\prime} \, \pderivada{v}(x) \dd{x} = \\[0.5em]
&= \bigg[ a_{2}(x) \, u(x) \, \pderivada{v}(x) - \big[ a_{2}(x) \, u(x) \big]^{\prime} \, v(x) \bigg] \eval_{a}^{b} + \scaleint{6ex}_{\bs a}^{b} \, \big[ u(x) \, a_{2} (x) \big]^{\prime \prime} \, v(x) \dd{x}
\end{aligned}
\label{eq:ecuacion_04_18}
\end{eqnarray}
Combinando estos resultados, tenemos que:
\begin{eqnarray}
\begin{aligned}[b]
&\langle u , L \, v \rangle =  \scaleint{6ex}_{\bs a}^{b} u (a_{2} \, \sderivada{v} + a_{1} \, \pderivada{v} + a_{0} \, v) \dd{x} = \\[0.5em] 
&= \bigg[ a_{1}(x) \, u(x) \, v(x) + a_{2}(x) \, u(x) \, \pderivada{v}(x) - \big[ a_{2}(x) \, u(x) \big]^{\prime} \, v(x) \bigg] \eval_{a}^{b} + \\[0.5em]
&+ \scaleint{6ex}_{\bs a}^{b} \, \big[ u(x) \, a_{2} (x) \big]^{\prime \prime} \, v(x) \dd{x}
\end{aligned}
\label{eq:ecuacion_04_19}
\end{eqnarray}
Agregando las CDF para $v$,  uno tiene que determinar las CDF para $u$ tales que:
\begin{align*}
&\bigg[ a_{1}(x) \, u(x) \, v(x) + a_{2}(x) \, u(x) \, \pderivada{v}(x) - \big[ a_{2}(x) \, u(x) \big]^{\prime} \, v(x) \bigg] \eval_{a}^{b} = 0
\end{align*}
Estos nos lleva a:
\begin{align*}
\langle u , L \, v \rangle &= \scaleint{6ex}_{\bs a}^{b}  \big[ (a_{2} \, u)^{\prime \prime} - (a_{1} \, u)^{\prime} + a_{0} \, u \big] \, v \dd{x} \equiv \langle L^{\dagger} \, u, v \rangle 
\end{align*}
Por lo tanto:
\begin{align}
L^{\dagger} = a_{2}(x) \, \dv[2]{x} - a_{1}(x) \, \dv{x} + a_{0}(x)
\label{eq:ecuacion_04_20}
\end{align}

Cuando $L^{\dagger} = L$,  el operador se llama formalmente \textbf{autoadjunto}, también es conocido como \textbf{operador Hermitiano}. Cuando el dominio de $L$ es el mismo que el dominio de $L^{\dagger}$, se utiliza el término autoadjunto.
\par
\noindent
\textbf{Ejemplo 2. } Determina $L^{\dagger}$ y su dominio para el operador:
\begin{align*}
L \, u = \dv{u}{x}
\end{align*}
donde $u$ satisface las CDF $u(0) = 2 \, u(1)$ en $[0, 1]$.
\par
Necesitamos encontrar el operador adjunto que satisfaga $\langle v, L \, u \rangle = \langle L^{\dagger} \, v, u \rangle$. Por lo que reescribimos la integral:
\begin{eqnarray*}
\langle v, L \, u \rangle =  \scaleint{6ex}_{\bs 0}^{1} v \, \dv{u}{x} \dd{x} =  u \, v \eval_{0}^{1} - \scaleint{6ex}_{\bs 0}^{1} u \, \dv{v}{x} \dd{x} =  \langle L^{\dagger} \, v, u \rangle
\end{eqnarray*}
De aquí tenemos que el problema adjunto que consiste en un operador adjunto y la CDF asociada (o dominio de $L^{\dagger}$):
\begin{enumerate}
\item $L^{\dagger} = - \displaystyle \dv{x}$
\item $\displaystyle u \, v \eval_{0}^{1} = 0 \Rightarrow u (1) \big[ v(1) - 2 \, v(0) \big] \Rightarrow v (1) = 2 \, v(0)$
\end{enumerate}

\section{Identidades de Lagrange y de Green.}
\subsection{Identidades de apoyo.}

Antes de pasar a la demostración de que los valores propios de un problema de Sturm-Liouville son reales y las funciones propias asociadas son ortogonales,  necesitaremos introducir dos identidades importantes. Para el operador de Sturm-Liouville:
\begin{align*}
\mathcal{L} = \dv{x} \left( p \, \dv{x} \right) + q
\end{align*}
Se tienen dos identidades:
\begin{enumerate}
\item \textbf{Identidad de Lagrange:} 
\begin{align*}
u \, \mathcal{L} \, v - v \, \mathcal{L} \, u = \big[ p \, (u \, \pderivada{v} - v \, \pderivada{u}) \big]^{\prime}
\end{align*}
\item \textbf{Identidad de Green:} 
\begin{align*}
\scaleint{6ex}_{\bs a}^{b} \, \big(u \, \mathcal{L} \, v - v \, \mathcal{L} \, u \big) \dd{x} = \big[ p \, (u \, \pderivada{v} - v \, \pderivada{u}) \big]\eval_{a}^{b}
\end{align*}
\end{enumerate}

La demostración de la identidad de Lagrange se sigue una sencilla manipulación del operador:
\begin{eqnarray}
\begin{aligned}[b]
u  \mathcal{L} v {-} v \mathcal{L} u &= u \bigg[ \dv{x} \left( p \dv{v}{x} \right) {+} q v \bigg] {-} v \bigg[ \dv{x} \left( p \dv{u}{x} \right) {+} q u \bigg] = \\[0.5em] 
&= u \dv{x} \left( p \dv{v}{x} \right) - v \dv{x} \left( p \dv{u}{x} \right) = \\[0.5em]
&= u \dv{x} \left( p \dv{v}{x} \right) + p \dv{u}{x} \dv{v}{x} - v \dv{x} \left( p \dv{u}{x} \right) - p \dv{u}{x} \dv{v}{x} = \\[0.5em] 
&= \dv{x} \bigg[ p \, u \, \dv{v}{x} - p \, v \, \dv{u}{x}  \bigg]
\end{aligned}
\label{eq:ecuacion_04_21}
\end{eqnarray}
La identidad de Green se prueba simplemente integrando la identidad de Lagrange.
    

\section{Ortogonalidad y eigenvalores reales.}
\subsection{Eigenvalores reales.}

Ahora estamos listos para demostrar que:
\begin{enumerate}
\item \textbf{Los eigenvalores de un problema de Sturm-Liouville son reales}.
\item \textbf{Las eigenfunciones correspondientes son ortogonales}.
\end{enumerate}

Los eigenvalores del problema de tipo Sturm-Liouville son reales:
\begin{align*}
\mathcal{L} \, y = \left( x \, \pderivada{y} \right)^{\prime} + \dfrac{2}{x} \, y = - \lambda \, \sigma \, y
\end{align*}
Sean las $\phi_{n}(x)$ una solución para el problema de eigenvalores asociados con $\lambda_{n}$:
\begin{align*}
\mathcal{L} \, \phi_{n} = - \lambda_{n} \, \sigma \, \phi_{n}
\end{align*}
Mostrando que $\overline{\lambda}_{n} = \lambda_{n}$, donde la barra significa el conjugado complejo. Entonces, también consideramos el conjugado complejo de esta ecuación:
\begin{align*}
\mathcal{L} \, \overline{\phi}_{n} = - \overline{\lambda}_{n} \, \sigma \, \overline{\phi}_{n}
\end{align*}
Multiplicando la primera ecuación por $\overline{\phi}_{n}$, la segunda ecuación por $\phi_{n}$ y luego restando los resultados, obtenemos:
\begin{align*}
\overline{\phi}_{n} \, \mathcal{L} \, \phi_{n} - \phi_{n} \, \mathcal{L} \, \overline{\phi}_{n} = \big( \overline{\lambda}_{n} - \lambda_{n} \big) \, \sigma \, \phi_{n} \, \overline{\phi}_{n}
\end{align*}
Integrando ambos lados de la expresión, se llega a:
\begin{align*}
\scaleint{6ex}_{\bs a}^{b} \bigg( \overline{\phi}_{n} \, \mathcal{L} \, \phi_{n} &- \phi_{n} \mathcal{L} \overline{\phi}_{n} \bigg) \dd{x} =\big( \overline{\lambda}_{n} {-} \lambda_{n} \big) \scaleint{6ex}_{\bs a}^{b} \sigma \phi_{n} \overline{\phi}_{n} \dd{x}
\end{align*}
Aplicando la identidad de Green en el lado izquierdo, se tiene que:
\begin{align*}
\big[ p \, (u \, \pderivada{v} - v \, \pderivada{u}) \big]\eval_{a}^{b} = \big( \overline{\lambda}_{n} - \lambda_{n} \big) \, \scaleint{6ex}_{\bs a}^{b} \sigma \, \phi_{n} \, \overline{\phi}_{n} \dd{x}
\end{align*}
Usando las condiciones homogéneas:
\begin{align*}
\alpha_{1} \, y (a) + \beta_{1} \, \ptilde{y} (a) &= 0 \\[0.5em]
\alpha_{2} \, y (b) + \beta_{2} \, \ptilde{y} (b) &= 0
\end{align*}
para el operador autoadjunto, el lado izquierdo se anula. Por lo que el resultado es:
\begin{align*}
\big( \overline{\lambda}_{n} - \lambda_{n} \big) \, \scaleint{6ex}_{\bs a}^{b} \sigma \, \norm{\phi_{n}}^{2} \dd{x} = 0
\end{align*}
esta integral es no negativa. Por lo que se tiene $\overline{\lambda}_{n} = \lambda_{n}$, entonces los eigenvalores son reales.

\subsection{Eigenfunciones ortogonales.}

Ahora nos interesa revisar que las funciones propias correspondientes a diferentes valores propios de un problema tipo Sturm-Liouville son ortogonales.
\begin{align*}
\dv{x} \left( p(x) \, \dv{x} \right) \, y + q(x) \, y + \lambda \, \sigma (x) \, y = 0
\end{align*}
La demostración es similar al ejemplo anterior. Sea $\phi_{n}(x)$ una solución al problema de valores propios con $\lambda_{n}$:
\begin{align*}
\mathcal{L} \, \phi_{n} = - \lambda_{n} \, \sigma \, \phi_{n}
\end{align*}
Y sea $\phi_{m}(x)$ una solución al problema de valores propios asociado con $\lambda_{m} \neq \lambda_{n}$:
\begin{align*}
\mathcal{L} \, \phi_{m} = - \lambda_{m} \, \sigma \, \phi_{m}
\end{align*}
Ahora, multiplicamos la primera ecuación por $\phi_{m}$ y la segunda ecuación por $\phi_{n}$. Restando estos resultados, obtenemos:
\begin{align*}
\phi_{m}\, \mathcal{L} \, \phi_{n} - \phi_{n} \, \mathcal{L} \, \phi_{m} = \big( \lambda_{m} - \lambda_{n} \big) \, \sigma \, \phi_{n} \, \phi_{m}
\end{align*}
Integrando ambos lados de la ecuación, usando la identidad de Green y usando las CDF homogéneas, se tiene:
\begin{align*}
\big( \lambda_{m} - \lambda_{n} \big) \, \scaleint{6ex}_{\bs a}^{b} \sigma \, \phi_{n} \, \phi_{m} \dd{x} = 0
\end{align*}
Dado que los eigenvalores son distintos, podemos dividir entre $\lambda_{m} - \lambda_{n}$, llegando al resultado deseado:
\begin{align*}
\scaleint{6ex}_{\bs a}^{b} \sigma \, \phi_{n} \, \phi_{m} \dd{x} = 0
\end{align*}
Por lo tanto, las funciones propias son ortogonales con respecto a la función de peso $\sigma(x)$.

\subsection{Degeneración.}

Si $N$ eigenfunciones linealmente independientes corresponden al mismo eigenvalor, se dice que este último es $N$-veces \textbf{degenerado}.
\par
Si $\lambda_{m} = \lambda_{n}$, la integral:
\begin{align*}
\big( \lambda_{m} - \lambda_{n} \big) \, \scaleint{6ex}_{\bs a}^{b} \sigma \, \phi_{n} \, \phi_{m} \dd{x}
\end{align*}
\emph{no necesariamente} se anula.
\par
Esto implica que las eigenfunciones linealmente independientes correspondientes al mismo eigenvalor \textbf{no son automáticamente} ortogonales. Por lo que debe de buscarse otro método para obtener un conjunto de eigenfunciones ortogonales,  veremos que \textbf{siempre} se puede lograr que sean ortogonales.


\section{Ortogonalización de Gram-Schmidt.}
\subsection{La técnica.}

Este método toma un conjunto de funciones no ortogonales linealmente dependientes y literalmente construye un conjunto ortogonal de funciones en un intervalo arbitrario con respecto a una función de peso arbitraria.
\par
Las funciones involucradas pueden ser reales o complejas, por conveniencia, asumiremos que las funciones son reales, la generalización para funciones complejas, no ofrece mayor dificultad. Veamos el caso de la normalización de funciones, que implica lo siguiente:
\begin{align*}
\scaleint{6ex}_{\bs a}^{b} \phi_{i}^{2} \, \sigma  \, \dd{x}  =  N_{i}^{2}
\end{align*}
revisemos que aún no se le ha puesto atención al valor de $N_{i}$. Ya que la ecuación básica:
\begin{align}
\mathcal{L} \, u (x) + \lambda \, \sigma (x) \, u (x) = 0
\label{eq:ecuacion_10_08}
\end{align}
es \textbf{lineal} y \textbf{homogénea}, podemos multiplicar la solución por cualquier constante, de tal manera que sigue siendo solución. Por lo que podemos pedir que tal solución $\phi_{i}(x)$ se multiplique por $N_{i}^{-1}$  y ahora la nueva $\phi_{i}$ (normalizada) satisface:
\begin{align}
\scaleint{6ex}_{\bs a}^{b} \, \phi_{i}^{2} (x) \, \sigma(x) \, \dd{x} = 1
\label{eq:ecuacion_10_39}
\end{align}
En términos de una delta de Kronecker:
\begin{align}
\scaleint{6ex}_{\bs a}^{b} \, \phi_{i}(x) \, \phi_{j} (x) \, \sigma (x) \, \dd{x} = \delta_{ij}
\label{eq:ecuacion_10_40}
\end{align}
La ecuación (\ref{eq:ecuacion_10_39}) nos dice que se ha normalizado a la unidad. Incluyendo la propiedad de ortogonalidad, tenemos la ecuación (\ref{eq:ecuacion_10_40}), a las funciones que la satisfacen, se dice que son \textbf{ortonormales} (ortogonales y normalizadas).
\par
Cabe señalar que existen \textbf{otras formas} de normalización,  cada una de las funciones especiales de la Física Matemática se puede normalizar de distintas formas.
\par
Consideremos tres conjuntos de funciones:
\begin{enumerate}
\item Un conjunto original, linealmente independiente $u_{n}(x)$ con $n=0,1,2,\ldots$ \\
Las funciones podrían ser funciones propias degeneradas, pero no es necesario que se cumpla este punto.
\item Un conjunto ortogonal $\psi_{n}(x)$ que se va a construir.
\item Un conjunto de funciones $\varphi_{n}(x)$ que serán normalizadas. 
\end{enumerate}
Tendremos las siguientes propiedades:
\begin{center}
{
\renewcommand{\arraystretch}{1.5}%
\begin{tabular}{p{4cm} p{4cm} p{4cm}}
\hline
\makecell{$u_{n}(x)$} & \makecell{$\psi_{n}(x)$} & \makecell{$\varphi_{n}(x)$} \\ \hline
\makecell{linealmente \\ independiente} &    \makecell{linealmente \\ independiente} & \makecell{linealmente \\ independiente} \\ \hline
\makecell{no ortogonal} & \makecell{ortogonal} & \makecell{ortogonal} \\ \hline
\makecell{no normalizada} & \makecell{no normalizada} & \makecell{normalizada \\ (ortonormal)} 
\end{tabular}
}
\end{center}

\subsection{Aplicando la técnica Gram-Schmidt.}

La técnica de Gram-Schmidt consiste en tomar la n-ésima función $\psi_{n}$ para ser $u_{n}(x)$ más una combinación lineal no conocida de la función $\varphi$ previa. El que haya una nueva $u_{n}(x)$ nos dará la garantía de que se mantenga la independencia lineal.
\par
El requisito que $\psi_{n}(x)$ sea ortogonal para cada $\varphi$ previa, proporciona los suficientes elementos para determinar cada uno de los coeficientes desconocidos. Así cuando ya se determinen las $\psi_{n}$, se pueden normalizar a la unidad, dejando a las  $\varphi_{n} (x)$.  \emph{Este procedimiento se repite} para las $\psi_{n+1}(x)$.
\\[0.5em]
\textbf{Comenzando con la técnica: } Empezamos con $n = 0$, sea:
\begin{align}
\psi_{0} (x) = u_{0} (x)
\label{eq:ecuacion_10_41}
\end{align}
no nos preocupemos al no tener una $\varphi$ previa.
\par
Entonces normalizamos:
\begin{align}
\varphi_{0}(x) = \dfrac{\psi_{0} (x)}{\left[ \displaystyle \scaleint{6ex} \psi_{0}^{2} \, \sigma \, \dd{x} \right]^{1/2}}
\label{eq:ecuacion_10_42}
\end{align}
Para $n = 1$, tenemos:
\begin{align}
\psi_{1} (x) = u_{1} (x) + a_{1, 0} \, \varphi_{0} (x)
\label{eq:ecuacion_10_43}
\end{align}
Que requiere que $\psi_{1} (x)$ sea ortogonal a $\varphi_{0} (x)$ (en este punto, la normalización de $\psi_{1} (x)$ es irrelevante). La ortogonalidad nos conduce a:
\begin{eqnarray}
\begin{aligned}[b]
\scaleint{6ex} \psi_{1} \, \varphi_{0} \, \sigma \, \dd{x} &=  \scaleint{6ex} u_{1} \, \varphi_{0} \, \sigma \, \dd{x} + a_{1,0} \scaleint{6ex} \varphi_{0}^{2} \, \sigma \dd{x} = 0
\end{aligned}
\label{eq:ecuacion_10_44}
\end{eqnarray}
Ya que $\varphi_{0}$ se normaliza a la unidad (ec. \ref{eq:ecuacion_10_42}), tenemos:
\begin{align}
a_{1,0} = - \scaleint{6ex} u_{1} \, \varphi_{0} \, \sigma \dd{x}
\label{eq:ecuacion_10_45}
\end{align}
que deja fijo el valor de $a_{1, 0}$
Normalizando, definimos:
\begin{align}
\varphi_{1} (x) = \dfrac{\psi_{1} (x)}{ \bigg[ \displaystyle \scaleint{6ex} \psi_{1}^{2} \, \sigma \dd{x} \bigg]^{1/2}}
\label{eq:ecuacion_10_46}
\end{align}
Generalizando, resulta:
\begin{align}
\varphi_{i} (x) = \dfrac{\psi_{i} (x)}{ \bigg[ \displaystyle \scaleint{6ex} \psi_{i}^{2} (x) \, \sigma (x) \dd{x} \bigg]^{1/2}}
\label{eq:ecuacion_10_47}
\end{align}
donde:
\begin{align}
\psi_{i}(x) = u_{i} + a_{1, 0} \, \phi_{0} + a_{i, 1} \, \phi_{1} + \ldots + a_{i, i-1} \, \phi_{i-1}
\label{eq:ecuacion_10_48}
\end{align}
Los coeficientes $a_{i, j}$ están dados por:
\begin{align}
a_{i, j} = - \scaleint{6ex} u_{i} \, \varphi_{j} \, \sigma  \dd{x}
\label{eq:ecuacion_10_49}
\end{align}
Esta ecuación es para una \textbf{normalización unitaria}. Para otros tipos de normalización, se tiene que:
\begin{align*}
\scaleint{6ex}_{\bs a}^{b} \left[ \varphi_{j} (x) \right]^{2} \, \sigma (x) \dd{x} =  N_{j}^{2}
\end{align*}
Entonces la ecuación (\ref{eq:ecuacion_10_47}) se reemplaza por:
\begin{align}
\varphi_{i} (x) =  N_{i} \: \dfrac{\psi_{i} (x)}{ \bigg[ \displaystyle \scaleint{6ex} \psi_{i}^{2} \, \sigma \dd{x} \bigg]^{1/2}}
\label{eq:ecuacion_10_47a}
\end{align}
Los términos $a_{i,j}$ resultan:
\begin{align}
a_{i, j} = - \dfrac{ \displaystyle \scaleint{6ex} u_{i} \, \varphi_{j} \, \sigma \dd{x}}{N_{j}^{2}}
\label{eq:ecuacion_10_49a}
\end{align}
\textbf{Conclusión del método de Gram-Schmidt: } Cabe señalar que el procedimiento de Gram-Schmidt es una  manera de construir un conjunto ortogonal o ortonormal, pero las funciones $\varphi_{i}(x)$ no son únicas. Existe un infinito de posibles conjuntos ortonormales para un intervalo dado y una función de peso dada.

\subsection{Polinomios de Legendre.}

Queremos generar un conjunto ortonormal a partir de las funciones:
\begin{align*}
u_{n} (x) = x^{n}, \hspace{1.5cm} n = 0, 1, 2, \ldots
\end{align*}
En el intervalo $-1 \leq x \leq 1$ y con la función de peso: $\sigma (x) = 1$. De acuerdo a la técnica descrita de ortogonalización de Gram-Schmidt:
\begin{align}
u_{0} (x) = 1 \hspace{1.5cm} \varphi_{0} (x) =  \dfrac{1}{\sqrt{2}}
\label{eq:ecuacion_10_50}
\end{align}
Entonces:
\begin{align}
\psi_{1} (x) = x + a_{1,0} \, \dfrac{1}{\sqrt{2}}
\label{eq:ecuacion_10_51}
\end{align}
donde:
\begin{align}
a_{1, 0} = - \scaleint{6ex}_{\bs -1}^{1} \dfrac{x}{\sqrt{2}} \, \dd{x} = 0
\label{eq:ecuacion_10_52}
\end{align}
Normalizando $\psi_{1}$, obtenemos:
\begin{align}
\varphi_{1} (x) = \sqrt{\dfrac{3}{2}} \, x
\label{eq:ecuacion_10_53}
\end{align}
Continuando el método de Gram-Schmidt, se define ahora:
\begin{align}
\psi_{2} (x) = x^{2} +  a_{2, 0} \, \dfrac{1}{\sqrt{2}} +  a_{2, 1} \, \sqrt{\dfrac{3}{2}} \, x
\label{eq:ecuacion_10_54}
\end{align}
Donde:
\begin{align}
a_{2, 0} &=  - \scaleint{6ex}_{\bs -1}^{1} \, \dfrac{x^{2}}{\sqrt{2}} \, \dd{x} =  - \dfrac{\sqrt{2}}{3} \label{eq:ecuacion_10_55} \\[1em] 
a_{2, 1} &=  - \scaleint{6ex}_{\bs -1}^{1} \, \sqrt{\dfrac{3}{2}} \, x^{3} \dd{x} =  0 \label{eq:ecuacion_10_56}
\end{align}
Por tanto:
\begin{align}
\psi_{2} (x) = x^{2} - \dfrac{1}{3}
\label{eq:ecuacion_10_57}
\end{align}
Normalizando a la unidad, tenemos:
\begin{align}
\varphi_{2} (x) = \sqrt{\dfrac{5}{2}} \, \dfrac{1}{2} \, (3 \, x^{2} - 1)
\label{eq:ecuacion_10_58}
\end{align}
La siguiente función $\varphi_{3}(x)$ es:
\begin{align}
\varphi_{3} (x) = \sqrt{\dfrac{7}{2}} \, \dfrac{1}{2} \, (5 \, x^{3} - 3 \, x)
\label{eq:ecuacion_10_59}
\end{align}
Se puede demostrar que:
\begin{align}
\varphi_{n} (x) = \sqrt{\dfrac{2 \, n + 1}{2}} \, P_{n} (x)
\label{eq:ecuacion_10_60}
\end{align}
donde $P_{n}$ es el polinomio de orden $n$ de Legendre.
\par
El uso de funciones especiales como en este caso los polinomios de Legendre $P_{n}(x)$, será algo común, ya en el Tema 5 se revisará la construcción completa de los $P_{n}(x)$, así como un conjunto de propiedades.

\end{document}