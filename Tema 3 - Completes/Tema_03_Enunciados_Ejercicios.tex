\documentclass[12pt]{article}
\usepackage[left=0.25cm,top=1cm,right=0.25cm,bottom=1cm]{geometry}
\textwidth = 20cm
\hoffset = -1cm
\usepackage[utf8]{inputenc}
\usepackage[spanish,es-tabla]{babel}
\usepackage[autostyle,spanish=mexican]{csquotes}
\usepackage[tbtags]{amsmath}
\usepackage{nccmath}
\usepackage{amsthm}
\usepackage{amssymb}
\usepackage{graphicx}
\usepackage{standalone}
\usepackage[outdir=./]{epstopdf}
\usepackage{siunitx}
\usepackage{physics}
\usepackage{color}
\usepackage{float}
\usepackage{multicol}
%\usepackage{milista}
\usepackage{enumitem}
\usepackage{anyfontsize}
\usepackage{anysize}
\usepackage{enumitem}
\usepackage{capt-of}
\usepackage{bm}
\usepackage{relsize}
\usepackage{placeins}
\usepackage{empheq}
\usepackage{cancel}
\usepackage{wrapfig}
\spanishdecimal{.}
\renewcommand{\baselinestretch}{1.5} 
\renewcommand\labelenumii{\theenumi.{\arabic{enumii}}}
\newcommand{\ptilde}[1]{\ensuremath{{#1}^{\prime}}}
\newcommand{\stilde}[1]{\ensuremath{{#1}^{\prime \prime}}}
\newcommand{\ttilde}[1]{\ensuremath{{#1}^{\prime \prime \prime}}}
\newcommand{\ntilde}[2]{\ensuremath{{#1}^{(#2)}}}


\title{Ejercicios para el Tema 3 \\[0.3em]  \large{Matemáticas Avanzadas de la Física}\vspace{-3ex}}
\author{M. en C. Gustavo Contreras Mayén}
\date{ }
\begin{document}
\vspace{-4cm}
\maketitle
\fontsize{14}{14}\selectfont

\textbf{Indicaciones: } Deberás de resolver cada ejercicio de la manera más completa, ordenada y clara posible, anotando cada paso así como las operaciones involucradas. El puntaje de cada ejercicio es de \textbf{1 punto}, con excepción en donde se indica.

\begin{enumerate}
\item Para el siguiente conjunto de ecuaciones diferenciales de la física matemática:
\begin{table}[H]
\renewcommand{\arraystretch}{1.1}
\begin{tabular}{l l}
Legendre, & \multirow{4}{*}{$(1 -x^{2}) \, \sderivada{y} - 2 \, x \, \pderivada{y} + \ell (\ell + 1) \, y = 0$} \\
$-1 \leq x \leq 1$ & \\ 
Legendre recorrida, & \\
$0 \leq x \leq 1$ &  \\ \hline
Asociada de Legendre & $(1 -x^{2}) \, \sderivada{y} - 2 \, x \, \pderivada{y} + \bigg[ \ell (\ell + 1) - \dfrac{m^{2}}{1 -x^{2}} \bigg] \, y = 0$ \\ \hline
Chebyshev Tipo I, & \multirow{4}{*}{$(1 -x^{2}) \, \sderivada{y} - x \, \pderivada{y} + n^{2} \, y = 0$} \\
$-1 \leq x \leq 1$ & \\
Chebyshev recorrida Tipo I, & \\
$0 \leq x \leq 1$ & \\ \hline
Chebyshev Tipo II & $(1 -x^{2}) \, \sderivada{y} - 3 \, x \, \pderivada{y} + n(n +2) \, y = 0$ \\
Ultraesférica & $ (1 - x^{2}) \, \sderivada{y} - (2 \, \alpha + 1) \, x \, \pderivada{y} + n (n + 2 \, \alpha) \, y = 0$ \\
Bessel, $0 \leq x \leq a$ & $x^{2} \, \sderivada{y} + x \, \pderivada{y} + (x^{2} -n^{2}) \, y = 0$ \\
Laguerre, $0 < x < \infty$ & $x \, \sderivada{y} + (1 - x) \, \pderivada{y} + a \, y = 0$ \\
Hermite & $\sderivada{y} - 2 \, x \, \pderivada{y} + 2 \, \alpha \, y = 0$
\end{tabular}
\end{table}
\begin{enumerate}[label=\roman*)]
\item \textbf{(3 puntos.) } Presenta la forma de tipo Sturm-Liouville de las $10$ ecuaciones.
\item \textbf{(1 punto.) }Completa la tabla a partir de las operaciones que realizaste para llevar a la forma de tipo Sturm-Liouville.
\begin{table}[H]
\centering
\renewcommand{\arraystretch}{1.1}
\begin{tabular}{l p{3cm} p{2cm} p{1.5cm} p{1.5cm}}
Ecuación & $p (x)$ & $q (x)$ & $\lambda$ & $\omega (x)$ \\ \hline
Legendre, $-1 \leq x \leq 1$ & & & & \\
Legendre recorrida, $0 \leq x \leq 1$ & & & & \\
Asociada de Legendre & & & & \\
Chebyshev Tipo I & & & & \\
Chebyshev recorrida Tipo I & & & & \\
Chebyshev Tipo II & & & & \\
Ultraesférica & & & & \\
Bessel, $0 \leq x \leq a$ & & & & \\
Laguerre, $0 < x < \infty$ & & & & \\
Asociada de Laguerre & & & & \\
Hermite & & & & \\
\end{tabular}
\end{table}
\end{enumerate}
%Ref. Boyce 11.2
\item Para el siguiente problema:
\begin{align*}
&\sderivada{y} - \lambda \, y = 0 \\[0.5em]
&y(0) + \pderivada{y}(0) = 0, \hspace{1.5cm} y(1) = 0
\end{align*}
\begin{enumerate}[label=\alph*)]
\item Determina la forma de las funciones propias y de los valores propios distintos de cero.
\item ¿$\lambda_{1} = 0$ es valor propio?
\item Calcula el valor aproximado para $\lambda_{1}$, el valor propio distinto de cero con menor valor absoluto.
\item Calcula $\lambda_{n}$ para valores grandes de $n$.
\end{enumerate}
 %Ref. Riley 2006 - 17.7
\item Considera el conjunto de funciones, $\left\{ f (x) \right\}$, de variable real $x$, definida en el intervalo $-\infty < x < \infty$, que $\to 0$ al menos tan rápidamente como $x^{-1}$ cuando $x \to \pm \infty$. Con la función de peso unitaria, determina si cada uno de los siguientes operadores lineales es autoadjunto (Hermitiano) cuando actúa sobre $\left\{ f (x) \right\}$:
\begin{multicols}{2}
\begin{enumerate}[label=\alph*)]
\item $\displaystyle \dv{x} + x$
\item $\displaystyle - i \, \dv{x} + x^{2}$
\item $\displaystyle i \, x \, \dv{x}$
\item $\displaystyle i \, \dv[3]{x}$
\end{enumerate}
\end{multicols}
%Ref. Arfken 2006 - 10.3.6
\item Usando la ortogonalización de Gram-Schmidt construye los primeros tres polinomios de Chebyshev de tipo I, con:
\begin{align*}
u_{n} (x) = x^{n} \hspace{1cm} n = 0, 1, 2, \ldots, \hspace{1cm} -1 \leq x \leq 1, \hspace{1cm} \omega(x) = (1 - x^{2})^{-\frac{1}{2}}
\end{align*}
Considera la normalización:
\begin{align*}
\scaleint{5ex}_{\bs -1}^{1} \, T_{m}(x) \, T_{n}(x) \, \omega(x) \dd{x} = \delta_{mn} \, \begin{cases}
\pi & m = n = 0 \\
\dfrac{\pi}{2} & m = n \geq 1
\end{cases}
\end{align*}
%Ref. Riley (2006) 17.11
\item El operador diferencial $\mathcal{L}$ está definido como:
\begin{align*}
\mathcal{L} \, y = - \dv{x} \bigg[ e^{x} \, \dv{y}{x} \bigg] - \dfrac{1}{4} \, e^{x} \, y
\end{align*}
\begin{enumerate}[label=\roman*)]
\item Calcula los eigenvalores $\lambda_{n}$ para el problema:
\begin{align*}
\mathcal{L} \, y_{n} = \lambda_{n} \, e^{x} \, y_{n} \hspace{1.5cm} 0 < x < 1
\end{align*}
con las condiciones de frontera:
\begin{align*}
y (0) = 0, \hspace{1.5cm} \dv{y}{x} + \dfrac{1}{2} \, y = 0 \hspace{0.3cm} \mbox{en} \hspace{0.3cm} x = 1
\end{align*}
\item Encuentra la correspondiente $y_{n}$ no normalizada y también la función de peso $\omega (x)$ con respecto a la cual las $y_{n}$ sean ortogonales. Por lo tanto, selecciona una normalización adecuada para $y_{n}$.
\end{enumerate}
\end{enumerate}




\end{document}