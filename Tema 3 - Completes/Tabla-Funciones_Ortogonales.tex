\documentclass[12pt,landscape]{article}
\usepackage[utf8]{inputenc}
\usepackage[letterpaper, margin=0.5cm, footskip=-0.5cm]{geometry}
%\usepackage{anysize}
%\marginsize{1cm}{1cm}{1cm}{1cm}
\usepackage[spanish,es-lcroman, es-tabla]{babel}
\usepackage[autostyle,spanish=mexican]{csquotes}
\usepackage{amsmath}
\usepackage{amssymb}
\usepackage{nccmath}
\numberwithin{equation}{section}
\usepackage{amsthm}
\usepackage{graphicx}
\usepackage[outdir=./]{epstopdf}
\DeclareGraphicsExtensions{.pdf,.png,.jpg,.eps}
\usepackage{color}
\usepackage{float}
\usepackage{fancyhdr}
\usepackage{multicol}
\usepackage{enumerate}
\usepackage[shortlabels]{enumitem}
\usepackage{anyfontsize}
\usepackage{anysize}
\usepackage{array}
\usepackage{multirow}
\usepackage{enumitem}
\usepackage{cancel}
\usepackage{nameref}
\usepackage{pdflscape}
\usepackage{makecell}
\usepackage{longtable}
\usepackage{pgfplots}
\pgfplotsset{compat=1.12}
\usepackage{tikz}
\usepackage{circuitikz}
\usepackage{tikz-3dplot}
\usepackage{caption}
\usepackage{bm}
\usepackage{mathtools}
\usepackage{esvect}
\usepackage{hyperref}
\usepackage{relsize}
\usepackage{siunitx}
\usepackage{physics}
%\usepackage[backend=biber]{biblatex}
\usepackage{standalone}
\usepackage{mathrsfs}
\usepackage{bigints}
\usepackage{bookmark}
%Quita el número de la página
\pagenumbering{gobble}
\spanishdecimal{.}
%\setlength{\voffset}{-0.75in}
\author{}
\date{ }
\usepackage[flushleft]{threeparttable}
\title{Ecuaciones Diferenciales de la Física Matemática \\ {\large Tema 3 - Completez - Curso MAF}}
\begin{document}
\maketitle
\fontsize{14}{14}\selectfont
\addtolength{\voffset}{-2cm}
\vspace{-2cm}
\setcounter{table}{2}
\begin{table}[!ht]
\centering
{\setlength\extrarowheight{1.5pt}
{\renewcommand{\arraystretch}{1.5}%
\caption{Polinomios ortogonales generados por la ortogonalización de Gram-Schmidt de $u_{n}(x)= x^{n}$, con $n=0,1,2,\ldots$}
\begin{threeparttable}
\begin{tabular}{p{5cm} c c p{10cm}}
\hline
\makecell{Polinomios} & Intervalo & $w(x)$ & \makecell{Normalización estándar} \\ \hline
Legendre & $ -1 \leq x \leq 1$ & $1$ & $\displaystyle \int_{-1}^{1} \left[ P_{n}(x) \right]^{2} \dd{x} = \dfrac{2}{2 \, n + 1} $ \\
Modificados de Legendre & $ 0 \leq x \leq 1$ & $1$ & $\displaystyle \int_{-1}^{1} \left[ P_{n}^{*}(x) \right]^{2} \dd{x} = \dfrac{2}{2 \, n + 1} $ \\
Chebyshev I & $-1 \leq x \leq 1$ & $(1 - x^{2})^{-1/2}$ & $\displaystyle \int_{-1}^{1} \dfrac{\left[ T_{n}(x) \right]^{2}}{(1 - x^{2})^{-1/2}} \dd{x} = \begin{cases} 
\displaystyle \frac{\pi}{2} & n \neq 0 \\
\pi & n = 0 \end{cases} $ \\
Modificados de Chebyshev I & $0 \leq x \leq 1$ & $[x (1 - x)]^{-1/2}$ & $\displaystyle \int_{-1}^{1} \dfrac{\left[ T_{n}^{*} (x) \right]^{2}}{[x (1 - x)]^{-1/2}} \dd{x} = \begin{cases} 
\displaystyle \frac{\pi}{2} & n > 0 \\
\pi & n = 0 \end{cases} $ \\
Chebyshev II & $-1 \leq x \leq 1$ & $(1 - x^{2})^{1/2}$ & $\displaystyle\int_{-1}^{1} [U_{n} (x)]^{2} \, (1 - x^{2})^{1/2} \, \dd x = \frac{\pi}{2}$ \\
Laguerre & $0 \leq x < \infty $ & $e^{-x}$ & $\displaystyle \int_{0}^{\infty} \left[ L_{n} (x) \right]^{2} \, e^{-x} \dd{x} =  1 $ \\
Asociados de Laguerre & $0 \leq x < \infty $ & $x^{k} \, e^{-x}$ & $\displaystyle \int_{0}^{\infty} \left[ L_{n}^{k} (x) \right]^{2} \, x^{k} \, e^{-x} \dd{x} = \dfrac{(n + k)!}{n!} $ \\
Hermite & $- \infty < x < \infty $ & $e^{-x^{2}}$ & $\displaystyle \int_{-\infty}^{\infty} \left[ H_{n} (x) \right]^{2} e^{-x^{2}} \dd{x} = 2^{n} \, \pi^{1/2} \, n! $
\end{tabular}
\end{threeparttable}
}}
\end{table}
\end{document}