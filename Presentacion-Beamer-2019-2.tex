\input{pre_documento}
\input{pre_plantilla_Warsaw_default}
\input{pre_define_footers_Warsaw_default}
%\usepackage[backend=biber]{biblatex}
%\bibliography{LibrosFC.bib}
%\input{../Preambulos/pre_codigo}
%\input{../Preambulos/pre_define_footers}
\title{Matemáticas Avanzadas de la Física}
\subtitle{Semestre 2019-2}
\author[]{M. en C. Gustavo Contreras Mayén \\ M. en C. Abraham Lima Buendía}
\institute{Facultad de Ciencias - UNAM}
\titlegraphic{\includegraphics[width=2cm]{escudo-facultad-ciencias.jpg}\hspace*{4.75cm}~%
   \includegraphics[width=2cm]{escudo-unam.jpg}
}
\date{\today}
\begin{document}
\maketitle
\section*{Contenido}
\frame{\tableofcontents[currentsection, hideallsubsections]}
\fontsize{14}{14}\selectfont
\spanishdecimal{.}
\section{Presentación del curso.}
\frame{\tableofcontents[currentsection, hideothersubsections]}
\subsection{Objetivos.}
\begin{frame}
\frametitle{Objetivos del curso}
El alumno:
\begin{itemize}
\setlength{\itemsep}{0mm}
\item Reconocerá las ideas básicas del análisis de ecuaciones que involucran a funciones de varias variables.
\item Formulará aproximaciones consistentes a soluciones, con el fin de cuantificar los distintos mecanismos de la física que se involucran.
\end{itemize}
\end{frame}
\begin{frame}
\frametitle{Objetivos del curso}
El alumno:
\begin{itemize}
\setlength{\itemsep}{0mm}
\item Consultará la literatura matemática que sea relevante para los problemas de física.
\item Identificará el papel moderno que juegan las funciones especiales, como auxiliares poderosos en el análisis cualitativo de problemas en varias variables.
\end{itemize}
\end{frame}
\begin{frame}
\frametitle{Objetivo adicional}
También es nuestro objetivo demostrar al alumno que \emph{las funciones especiales y las transformadas integrales} no son solamente un tema matemático, que involucra las ramas de la geometría diferencial, las ecuaciones diferenciales y el análisis matemático.
\end{frame}
\begin{frame}
\frametitle{Objetivo adicional}
Veremos que \emph{son las técnicas de estudio fundamentales} en la electrostática, la electrodinámica, la mecánica cuántica en los límites relativista y no relativista, la dinámica de medios deformables, la hidrodinámica clásica entre otras ramas de la física.
\end{frame}
\section{Metodología de enseñanza.}
\frame{\tableofcontents[currentsection, hideothersubsections]}
\subsection{Exposición con diálogo.}
\begin{frame}
\frametitle{Exposición con diálogo}
En las clases habrá exposición con dialógo por parte de los profesores, junto con el desarrollo de ejercicios que orientarán la solución de problemas en un primer momento de tipo analítico, para luego revisar un ejercicio tomado de la física.
\end{frame}
\subsection{Tiempo para atender el curso.}
\begin{frame}
\frametitle{Tiempo para dedicarle al curso}
El curso demandará una atención por parte de ustedes con el mismo número de horas fuera de clase (consideramos que sería el mínimo de tiempo), es decir, que deberán de dedicarle fuera del salón de clase, al menos 5 horas a la semana para el desarrollo de lecturas, ejercicios, tareas y actividades.
\end{frame}
\section{Temario.}
\frame{\tableofcontents[currentsection, hideothersubsections]}
\subsection{Tema 1 - La física y la geometría.}
\begin{frame}
\frametitle{Tema 1 - La física y la geometría}
\setbeamercolor{item projected}{bg=blue!70!black,fg=yellow}
\setbeamertemplate{enumerate items}[circle]
\begin{enumerate}[<+->]
\item Métrica y teoremas integrales (Gauss y Stokes).
\item Coordenadas generalizadas.
\item Ecuaciones diferenciales de la física.
\end{enumerate}
\end{frame}
\subsection{Tema 2 - Primeras técnicas de solución.}
\begin{frame}
\frametitle{Tema 2- Primeras técnicas de solución}
\setbeamercolor{item projected}{bg=blue!70!black,fg=yellow}
\setbeamertemplate{enumerate items}[circle]
\begin{enumerate}[<+->]
\item Separación de variables.
\item Remoción de singularidades y método de Frobenius.
\item Segunda solución independiente.
\item Teorema de Green.    
\end{enumerate}
\end{frame}
\subsection{Tema 3 - Completes y ortogonalidad.}
\begin{frame}
\frametitle{Tema 3 - Completes y ortogonalidad}
\setbeamercolor{item projected}{bg=blue!70!black,fg=yellow}
\setbeamertemplate{enumerate items}[circle]
\begin{enumerate}[<+->]
\item Ecuaciones de tipo Sturm-Liouville.
\item Ortogonalización de Gram-Schimdt.
\item Completez y función delta de Dirac.
\item Desarrollo de una función en una base propia.    
\end{enumerate}
\end{frame}
\subsection{Tema 4 - Función Gamma y Función Beta.}
\begin{frame}
\frametitle{Tema 4 - Función Gamma y Función Beta}
\setbeamercolor{item projected}{bg=blue!70!black,fg=yellow}
\setbeamertemplate{enumerate items}[circle]
\begin{enumerate}[<+->]
\item Doble factorial
\item Función Gamma
\item Función Beta
\item Relación entre las funciones Gamma y Beta
\end{enumerate}
\end{frame}
\subsection{Tema 5 - Separación de variables en coordenadas esféricas.}
\begin{frame}
\frametitle{Tema 5 - Separación de variables en coordenadas esféricas}
\setbeamercolor{item projected}{bg=blue!70!black,fg=yellow}
\setbeamertemplate{enumerate items}[circle]
\begin{enumerate}[<+->]
\item Ecuación de Legendre.
\item Ecuación asociada de Legendre.
\item Armónicos esféricos, teorema de adición y momento angular.
\end{enumerate}
\end{frame}
\subsection{Tema 6 - Funciones especiales.}
\begin{frame}
\frametitle{Tema 6 - Funciones especiales}
\setbeamercolor{item projected}{bg=blue!70!black,fg=yellow}
\setbeamertemplate{enumerate items}[circle]
\begin{enumerate}[<+->]
\item Funciones de Bessel (propagación de ondas cilíndricas y esféricas)
\item Funciones de Hermite (oscilador cuántico)
\item Funciones de Laguerre (átomo de hidrógeno)
\item Funciones de Chebychev (tipos I y II)
\item Funciones hipergeométricas
\item Funciones de Gegenbauer    
\end{enumerate}
\end{frame}
\subsection{Tema 7 - Transformadas integrales.}
\begin{frame}
\frametitle{Tema 7 - Transformadas integrales}
\setbeamercolor{item projected}{bg=blue!70!black,fg=yellow}
\setbeamertemplate{enumerate items}[circle]
\begin{enumerate}[<+->]
\item Transformada de Fourier.
\item Transformada de Laplace.
\item Aplicaciones.
\item Transformada discreta usada en cómputo científico.
\end{enumerate}
\end{frame}
\section{Evaluación.}
\frame{\tableofcontents[currentsection, hideothersubsections]}
\subsection{Ejercicios semanales en clase.}
\begin{frame}
\frametitle{Ejercicios semanales en clase}
Durante las clases se presentarán ejercicios que se deberán de resolver a modo de tarea, para que: puedan trabajarlos oportunamente, hacer consultas, resolver dudas, etc.
\\
\bigskip
\pause
Al concluir la semana se tendrá un conjunto de ejercicios que deberán de ser entregados al siguiente viernes al iniciar la clase.
\end{frame}
\begin{frame}
\frametitle{Muy importante}
Se considerarán como ejercicios a cuenta de calificación, aquellos ejercicios resueltos que representen al menos el $50\%$ del total.
\\
\bigskip
\pause
Por ejemplo, si se dejan 6 ejercicios para entregar, se espera que al menos entreguen 3 ejercicios para que se revisen y se tomen en cuenta para la calificación, si se entregan menos de tres ejercicios, sólo se revisarán y no contarán para el porcentaje de calificación.
\end{frame}
\subsection{Exámenes parciales.}
\begin{frame}
\frametitle{Exámenes parciales}
Habrá 4 exámenes parciales distribuidos de la siguiente manera:
\begin{itemize}
\item \emph{Primer parcial:} Considera los temas 1 y 2.
\item \emph{Segundo parcial:} Los temas 3 y 4
\item \emph{Tercer parcial:} Los temas 5 y 6.
\item \emph{Cuarto parcial:} El tema 7.
\end{itemize}
\end{frame}
\begin{frame}
\frametitle{Exámenes parciales}
Se entregará el listado de problemas para cada examen con suficiente tiempo.
\\
\bigskip
La solución del examen se entrega de manera individual, no se recibirán los exámenes resueltos en CD, USB, o por correo, sólo a mano.
\end{frame}
\begin{frame}
\frametitle{Exámenes parciales}
El examen debe de entregarse con el $100\%$ de los ejercicios resueltos, sólo de esta manera se tomará en cuenta para la calificación, de lo contrario, sólo se revisarán los ejercicios resueltos sin que aporten puntuación.
\end{frame}
\subsection{Elementos para la calificación.}
\begin{frame}
\frametitle{Elementos para la calificación}
El porcentaje para cada elemento de evaluación es el siguiente:
\begin{itemize}
\setlength{\itemsep}{0mm}
\item Ejercicios semanales: $\mathbf{40\%}$.
\item Exámenes parciales: $\mathbf{60\%}$.
\end{itemize}
\end{frame}
\subsection{Consideraciones importantes.}
\begin{frame}
\frametitle{Consideraciones importantes}
\begin{itemize}
\setlength{\itemsep}{0mm}
\item La entrega de tareas se hará en el salón de clase, teniendo 20 minutos como límite de entrega; no se recibirán tareas por medios electrónicos.
\item Considerando el esquema de trabajo para el curso (1 hora diaria), no habrá exámenes de reposición.
\end{itemize}
\end{frame}
\begin{frame}
\frametitle{Consideraciones importantes}
\begin{itemize}
\setlength{\itemsep}{0mm}
\item En caso de que un examen parcial (o más exámenes) tenga(n) una calificación menor a $6$ (seis), el alumno es candidato para presentar el examen final.
\item Para presentar examen final del curso el alumno debió de haber presentado y entregado los cuatro exámenes parciales del curso.
\end{itemize}
\end{frame}
\begin{frame}
\frametitle{Consideraciones importantes}
\begin{itemize}
\setlength{\itemsep}{0mm}
\item En caso de no haber entregado alguna tarea de ejercicios y/o no haber presentado un examen del curso, no se tiene derecho a presentar el examen final, por lo que la calificación final que se asentará en el acta, será \textbf{NP (No presentó)}.
\end{itemize}
\end{frame}
\begin{frame}
\frametitle{Consideraciones importantes}
\begin{itemize}
\setlength{\itemsep}{0mm}
\item En caso de haber presentado al menos un examen y/o haber entregado una tarea de ejercicios, y posteriormente no se tenga registro de otra entrega, se entenderá que abandonaron el curso, por lo que no se tiene derecho para presentar el examen final, y la calificación que se asentará en el acta del curso, será $\mathbf{5}$ \textbf{(cinco)}.
\end{itemize}
\end{frame}
\begin{frame}
\frametitle{Consideraciones importantes}
\begin{itemize}
\setlength{\itemsep}{0mm}
\item En caso de presentar la primera ronda del examen final y la calificación obtenida sea no aprobatoria ($<6.0$), se puede presentar una segunda y última ronda del examen final.
\item La calificación obtenida en el examen final, es la que se asentará en actas.
\item Si el alumno no presenta el primer examen final, tendrá como calificación final en acta $\mathbf{5}$ \textbf{(cinco)}. 
\end{itemize}
\end{frame}
\end{document}