\documentclass[12pt]{article}
\usepackage[utf8]{inputenc}
\usepackage[T1]{fontenc}
\usepackage[spanish,es-lcroman]{babel}
\usepackage{amsmath}
\usepackage{amsthm}
\usepackage{amsfonts}
\usepackage{amssymb}
\usepackage{physics}
\AtBeginDocument{\RenewCommandCopy\qty\SI}
\usepackage{tikz}
\usepackage{float}
\usepackage{calc}
\usepackage[autostyle,spanish=mexican]{csquotes}
\usepackage[per-mode=symbol]{siunitx}
\usepackage{textcomp, gensymb}
\usepackage{multicol}
\usepackage{enumitem}
\usepackage{hyperref}
\usepackage{setspace}
\usepackage[left=2.00cm, right=2.00cm, top=2.00cm, 
     bottom=2.00cm]{geometry}
% \usepackage{Estilos/ColoresLatex}
\usepackage{makecell}
\usepackage{subcaption}
\usepackage[skip=10pt, indent=30pt]{parskip}
% \usepackage{scalerel}
\usepackage{scalerel}[2016-12-29]
\usepackage{biblatex}
\usepackage{cancel}

\definecolor{ao}{rgb}{0.0, 0.0, 1.0}
\definecolor{burgundy}{rgb}{0.5, 0.0, 0.13}

\hypersetup{
    colorlinks=true,
    linkcolor=ao,
    filecolor=magenta,      
    urlcolor=ao,
}

\newcommand{\ptilde}[1]{\ensuremath{{#1}^{\prime}}}
\newcommand{\stilde}[1]{\ensuremath{{#1}^{\prime \prime}}}
\newcommand{\ttilde}[1]{\ensuremath{{#1}^{\prime \prime \prime}}}
\newcommand{\ntilde}[2]{\ensuremath{{#1}^{(#2)}}}
\newcommand{\pderivada}[1]{\ensuremath{{#1}^{\prime}}}
\newcommand{\sderivada}[1]{\ensuremath{{#1}^{\prime \prime}}}
\newcommand{\tderivada}[1]{\ensuremath{{#1}^{\prime \prime \prime}}}
\newcommand{\nderivada}[2]{\ensuremath{{#1}^{(#2)}}}

\def\stretchint#1{\vcenter{\hbox{\stretchto[440]{\displaystyle\int}{#1}}}}
\def\scaleint#1{\vcenter{\hbox{\scaleto[3ex]{\displaystyle\int}{#1}}}}
\def\scaleiint#1{\vcenter{\hbox{\scaleto[6ex]{\displaystyle\iint}{#1}}}}
\def\scaleiiint#1{\vcenter{\hbox{\scaleto[6ex]{\displaystyle\iiint}{#1}}}}
\def\scaleoint#1{\vcenter{\hbox{\scaleto[3ex]{\displaystyle\oint}{#1}}}}
\def\bs{\mkern-12mu}

\newcommand{\textocolor}[2]{\textbf{\textcolor{#1}{#2}}}
\sisetup{per-mode=symbol}
\decimalpoint
\sisetup{bracket-numbers = false}
\newlength{\depthofsumsign}
\setlength{\depthofsumsign}{\depthof{$\sum$}}
\newcommand{\nsum}[1][1.4]{% only for \displaystyle
    \mathop{%
        \raisebox
            {-#1\depthofsumsign+1\depthofsumsign}
            {\scalebox
                {#1}
                {$\displaystyle\sum$}%
            }
    }
}

\AtBeginDocument{\RenewCommandCopy\qty\SI}
\ExplSyntaxOn
\msg_redirect_name:nnn { siunitx } { physics-pkg } { none }
\ExplSyntaxOff

\numberwithin{equation}{section}

\linespread{1.15}

\renewcommand{\labelenumii}{\theenumii}
\renewcommand{\theenumii}{\theenumi.\arabic{enumii}.}

\emergencystretch=1em

\title{Examen Extraordinario\\ \large{Matemáticas Avanzadas de la Física} \vspace{-50pt}}
\date{}
\begin{document}
\newgeometry{margin=1.5cm}
\renewcommand\labelenumii{\theenumi.{\arabic{enumii})}}
\maketitle
\fontsize{14}{14}\selectfont
\begin{enumerate}
%Referencia: Kirkwood - Mathematical physics with partial differential equations
%Cap. 1. Problem 3
\item Calcula el volumen en un sistema de coordenadas esféricas oblatas de una esfera de radio $r$. Considera que $a = 1$.
%Referencia: Chow - Mathematical methods for physicists. Problem 10.8
%Referencia: Tamvakis - Problems and solutions in quantum mechanics. Problem 1.4
\item Considera la siguiente superposición de ondas planas:
\begin{align*}
\psi_{k, \delta k} (x) \equiv \dfrac{1}{2 \sqrt{\pi \delta k}} \int_{k -\delta k}^{k + \delta k} e^{i \, q \, x} \dd{q}
\end{align*}
donde suponemos que el parámetro $k$ es mucho más pequeño que el número de onda $k$, es decir: $\delta k \ll k$.

Demuestra que
\begin{enumerate}
\item Las funciones de onda $\psi_{k, \delta k} (x)$ están normalizadas y son ortogonales unas de otras.
\item Para una partícula libre calcula el valor esperado para el momento y la energía en cada estado. 
\end{enumerate}
%Referencia: Zettili - Quantum Mechanics. Section 6.5 Solved problems. Problem 6.4
\item Un electrón se encuentra dentro de un pozo de potencial infinito esférico:
\begin{align*}
V(r) = \begin{cases}
0 & r < a \\
+\infty & r > a
\end{cases}
\end{align*}
\begin{enumerate}
\item Usando la ecuación radial de Schrödinger, calcula las autoenergías ligadas y las correspondientes funciones de onda radiales normalizadas, para el caso donde el momento angular orbital del electrón es cero, es decir, cuando $l = 0$.
\item Demuestra que el estado de energía más bajo para $l = 7$, está por arriba del segundo estado de energía más bajo para $l = 0$.
\end{enumerate}
\item Un análisis con la mecánica cuántica del efecto Stark (en coordenadas parabólicas), nos lleva a la siguiente ecuación diferencial
\begin{align*}
\dv{\xi} \left( \xi \, \dv{u}{\xi}  \right) + \left( \dfrac{1}{2} \, E \, \xi + L - \dfrac{m^{2}}{4 \, \xi} - \dfrac{1}{4} \, F \, \xi^{2} \right) \, u = 0
\end{align*}
Donde $F$ es una medida de la perturbación en la energía provocada por un campo eléctrico externo. Encuentra las funciones de onda sin perturbaciones $(F=0)$, en términos de los polinomios asociados de Laguerre.
\item Demuestra que el momento angular $\vb{L}$, es un operador Hermitiano:
\begin{align*}
\vb{L} = - i\, \hbar \, \vb{r} \cp \nabla \equiv i \, \dfrac{h}{2 \, \pi} \, \vb{r} \cp \nabla
\end{align*}
\item Un análisis para los patrones de radiación de las antenas para un sistema con una apertura circular, involucra la ecuación
\begin{align*}
g(u) = \int_{0}^{1} f(r) \, J_{0} (u \, r) \, r \dd{r}
\end{align*}
Si $f(r) = 1 - r^{2}$, demuestra que
\begin{align*}
g(u) = \dfrac{2}{u^{2}} \, J_{2}(u)
\end{align*}
\item La ecuación unidimensional de onda de Schrödinger es
\begin{align*}
- \dfrac{\hbar^{2}}{2 \, m} \; \dv[2]{\psi(x)}{x} +  V(x) \, \psi(x) = E \, \psi (x)
\end{align*}
Para el caso especial de que $V(x)$ es una función analítica de $x$, demuestra que la correspondiente ecuación de onda para el momento es
\begin{align*}
V \left( i \, \hbar \dv{p} \right) g(p) + \dfrac{p^{2}}{2 \, m} \, g(p) =  E \, g(p)
\end{align*}
Recupera esta ecuación de onda para el momento usando la transformada de Fourier y su inversa. No utilices la sustitución directa $\displaystyle{x \to i \, \hbar (\dv*{p})}$.
\end{enumerate}
\end{document}