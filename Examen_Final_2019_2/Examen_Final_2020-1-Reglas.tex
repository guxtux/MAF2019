\documentclass[12pt]{article}
\usepackage[letterpaper]{geometry}
%\textwidth = 345.0pt
%\hoffset = -3cm
\usepackage[utf8]{inputenc}
\usepackage[spanish,es-tabla]{babel}
\usepackage[autostyle,spanish=mexican]{csquotes}
\usepackage{amsmath}
\usepackage{standalone}
\usepackage{tikz}   
\usepackage{tikz-3dplot}
\usepackage{nccmath}
\usepackage{amsthm}
\usepackage{amssymb}
\usepackage{graphicx}
\usepackage{comment}
\usepackage{siunitx}
\usepackage{physics}
\usepackage{color}
\usepackage{float}
\usepackage{multicol}
%\usepackage{milista}
\usepackage{enumitem}
\usepackage{anyfontsize}
\usepackage{anysize}
\marginsize{1cm}{1cm}{2cm}{2cm}
\usepackage{enumitem}
\usepackage{capt-of}
\usepackage{bm}
\usepackage{relsize}
\newlist{milista}{enumerate}{2}
\setlist[milista,1]{label=\arabic*)}
\setlist[milista,2]{label=\arabic{milistai}.\arabic*)}
\spanishdecimal{.}
\renewcommand{\baselinestretch}{1.5}
\author{ }
\usepackage{etoolbox}
\AtBeginEnvironment{tikzpicture}{\shorthandoff{>}\shorthandoff{<}}{}{}
\author{}
\title{Examen Final \\ \large{Matemáticas Avanzadas de la Física} \vspace{-50pt}}
\date{}
\begin{document}
\setlength{\parskip}{5mm}
\newgeometry{margin=1.5cm}
\renewcommand\labelenumii{\theenumi.{\arabic{enumii})}}
\maketitle
\fontsize{14}{14}\selectfont
El examen final deberá de entregarse el día \textbf{lunes 2 de diciembre}, en el cubículo $303$ del Departamento de Física, siendo las $15:00$ pm la hora límite de entrega.
\par
El examen debe de entregarse con \textbf{la solución de todos los problemas indicados}, haciendo una completa y desarrollada respuesta, en caso de no reducir algebraicamente expresiones, dejar integrales sin resolver, pasos no detallados, el examen ya no se considerará como completo. Recuerden que están presentando un examen del curso de Matemáticas Avanzadas de la Física del sexto semestre.
\par
Se les cita el día \textbf{martes 3 de diciembre a las 12 horas} en el Laboratorio de Biofísica y Física Médica, para revisar su examen presentado y la calificación obtenida.
\par
La calificación que obtengan del examen final, en caso de ser aprobatoria, será la que se incluya en el acta final del curso. En caso de que su calificación sea no aprobatoria, tendrán derecho a presentar un segundo y último examen final, que se les entregaría el mismo día martes 3 de diciembre.
\par
Si no se cuenta con su examen resuelto el día de entrega, no podrán presentar el segundo examen final y la calificación que se asentará en el acta, será $\mathbf{5}$ (cinco)
\par
Hagan un esfuerzo para acreditar el curso, ya que cuentan con todo lo necesario para ello. 
\begin{figure}[H]
\includegraphics[scale=0.4]{./Imagenes/pulgar_arriba.png}
\end{figure}
!¡ Suerte ¡!
\end{document}