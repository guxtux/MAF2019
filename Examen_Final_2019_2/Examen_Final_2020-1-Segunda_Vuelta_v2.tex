\documentclass[12pt]{article}
\usepackage[letterpaper]{geometry}
%\textwidth = 345.0pt
%\hoffset = -3cm
\usepackage[utf8]{inputenc}
\usepackage[spanish,es-tabla]{babel}
\usepackage[autostyle,spanish=mexican]{csquotes}
\usepackage{amsmath}
\usepackage{nccmath}
\usepackage{amsthm}
\usepackage{amssymb}
\usepackage{graphicx}
\usepackage{comment}
\usepackage{siunitx}
\usepackage{physics}
\usepackage{color}
\usepackage{float}
\usepackage{multicol}
%\usepackage{milista}
\usepackage{enumitem}
\usepackage{anyfontsize}
\usepackage{anysize}
\marginsize{1cm}{1cm}{1cm}{1cm}
\usepackage{enumitem}
\usepackage{capt-of}
\usepackage{bm}
\usepackage{relsize}
\newlist{milista}{enumerate}{2}
\setlist[milista,1]{label=\arabic*)}
\setlist[milista,2]{label=\arabic{milistai}.\arabic*)}
\spanishdecimal{.}
\renewcommand{\baselinestretch}{1.5}
\author{ }
\author{}
\title{Examen Final - Segunda Vuelta  \\ \large{Matemáticas Avanzadas de la Física}\vspace{-50pt}}
\date{ }
\begin{document}
\newgeometry{margin=1.5cm}
\renewcommand\labelenumii{\theenumi.{\arabic{enumii})}}
\maketitle
\fontsize{14}{14}\selectfont
\begin{enumerate}
%Referencia: Tamvakis - Problems and solutions in quantum mechanics. Problem 1.4
\item Considera la siguiente superposición de ondas planas:
\begin{align*}
\psi_{k, \delta k} (x) \equiv \dfrac{1}{2 \sqrt{\pi \delta k}} \int_{k -\delta k}^{k + \delta k} e^{i \, q \, x} \dd{q}
\end{align*}
donde suponemos que el parámetro $k$ es mucho más pequeño que el número de onda $k$, es decir: $\delta k \ll k$.

Demuestra que las funciones de onda $\psi_{k, \delta k} (x)$ están normalizadas y son ortogonales unas de otras.
%Referencia: Zettili - Quantum Mechanics. Section 6.5 Solved problems. Problem 6.1
\item Considera una partícula (sin spin) de masa $m$ que se está moviendo en un potencial tridimensional
\begin{align*}
V(x, y, z) = \begin{cases}
\dfrac{1}{2} m \, \omega^{2} \, z^{2} & 0 < x < a, 0 < y < a \\
\infty & \mbox{en cualquier otro punto}
\end{cases}
\end{align*}
\begin{enumerate}
\item Calcula la energía total y la función de onda completa para esta partícula.
\item Suponiendo que $\hbar \omega > 3 \, \pi^{2} \hbar^{2} / (2 \, m \, a^{2})$, calcula las energías y las correspondientes degeneraciones para el estado base y el primer estado excitado.
\end{enumerate}
%Referencia: Zettili - Quantum Mechanics. Section 6.5 Solved problems. Problem 6.4
\item Un electrón se encuentra dentro de un pozo de potencial infinito esférico:
\begin{align*}
V(r) = \begin{cases}
0 & r < a \\
+\infty & r > a
\end{cases}
\end{align*}
\begin{enumerate}
\item Usando la ecuación radial de Schödinger, calcula las autoenergías ligadas y las correspondientes funciones de onda radiales normalizadas, para el caso donde el momento angular orbital del electrón es cero, es decir, cuando $l = 0$.
\item Demuestra que el estado de energía más bajo para $l = 7$, está por arriba del segundo estado de energía más bajo para $l = 0$.
\end{enumerate}
\item Debes de resolver el primer inciso para luego ocupar el resultado en el segundo inciso.
\begin{enumerate}
\item Usando el teorema del desarrollo, escribe la forma general para una función de Green que sea solución a una ecuación de valores propios.
\item Calcula la función de Green de una partícula en un pozo infinito de ancho $a$ unidimensional.
\end{enumerate}
\item Una buena aproximación para la interacción entre dos nucleones puede describirse mediante un potencial mesónico atractivo con $A$ negativo:
\begin{align*}
V = \dfrac{A \, \exp(- a \, x)}{x}
\end{align*}
Desarrolla una solución en serie para la ecuación de onda de Schrödinger
\begin{align*}
\dfrac{\hbar^{2}}{2 \, m} \, \dv[2]{\psi}{x} + (E - V) \, \psi = 0
\end{align*}
con los primeros tres coeficientes no nulos.
\end{enumerate}
\end{document}