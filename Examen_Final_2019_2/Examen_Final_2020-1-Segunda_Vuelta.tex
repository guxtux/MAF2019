\documentclass[12pt]{article}
\usepackage[letterpaper]{geometry}
%\textwidth = 345.0pt
%\hoffset = -3cm
\usepackage[utf8]{inputenc}
\usepackage[spanish,es-tabla]{babel}
\usepackage[autostyle,spanish=mexican]{csquotes}
\usepackage{amsmath}
\usepackage{nccmath}
\usepackage{amsthm}
\usepackage{amssymb}
\usepackage{graphicx}
\usepackage{comment}
\usepackage{siunitx}
\usepackage{physics}
\usepackage{color}
\usepackage{float}
\usepackage{multicol}
%\usepackage{milista}
\usepackage{enumitem}
\usepackage{anyfontsize}
\usepackage{anysize}
\marginsize{1cm}{1cm}{1cm}{1cm}
\usepackage{enumitem}
\usepackage{capt-of}
\usepackage{bm}
\usepackage{relsize}
\newlist{milista}{enumerate}{2}
\setlist[milista,1]{label=\arabic*)}
\setlist[milista,2]{label=\arabic{milistai}.\arabic*)}
\spanishdecimal{.}
\renewcommand{\baselinestretch}{1.5}
\author{ }
\author{}
\title{Examen Final - Segunda vuelta\\ \large{Matemáticas Avanzadas de la Física} \vspace{-50pt}}
\date{}
\begin{document}
\newgeometry{margin=1.5cm}
\renewcommand\labelenumii{\theenumi.{\arabic{enumii})}}
\maketitle
\fontsize{14}{14}\selectfont
\begin{enumerate}
%Referencia: Kirkwood - Mathematical physics with partial differential equations
%Cap. 1. Problem 3
\item Calcula el volumen en un sistema de coordenadas esféricas oblatas de una esfera de radio $r$. Considera que $a = 1$.
%Referencia: Chow - Mathematical methods for physicists. Problem 10.8
\item Calcula la función de Green que satisface la ecuación:
\begin{align*}
\dv[2]{G}{x} = \delta(x - x^{\prime})
\end{align*}
con las condiciones de frontera: $G = 0$ cuando $x = 0$ y $G$ permanece acotada mientras que $x \to \infty$.
%Referencia: Tamvakis - Problems and solutions in quantum mechanics. Problem 1.4
\item Considera la siguiente superposición de ondas planas:
\begin{align*}
\psi_{k, \delta k} (x) \equiv \dfrac{1}{2 \sqrt{\pi \delta k}} \int_{k -\delta k}^{k + \delta k} e^{i \, q \, x} \dd{q}
\end{align*}
donde suponemos que el parámetro $k$ es mucho más pequeño que el número de onda $k$, es decir: $\delta k \ll k$.

Demuestra que
\begin{enumerate}
\item Las funciones de onda $\psi_{k, \delta k} (x)$ están normalizadas y son ortogonales unas de otras.
\item Para una partícula libre calcula el valor esperado para el momento y la energía en cada estado. 
\end{enumerate}
%Referencia: Zettili - Quantum Mechanics. Section 6.5 Solved problems. Problem 6.1
\item Considera una partícula (sin spin) de masa $m$ que se está moviendo en un potencial tridimensional
\begin{align*}
V(x, y, z) = \begin{cases}
\dfrac{1}{2} m \, \omega^{2} \, z^{2} & 0 < x < a, 0 < y < a \\
\infty & \mbox{en cualquier otro punto}
\end{cases}
\end{align*}
\begin{enumerate}
\item Calcula la energía total y la función de onda completa para esta partícula.
\item Suponiendo que $\hbar \omega > 3 \, \pi^{2} \hbar^{2} / (2 \, m \, a^{2})$, calcula las energías y las correspondientes degeneraciones para el estado base y el primer estado excitado.
\end{enumerate}
%Referencia: Zettili - Quantum Mechanics. Section 6.5 Solved problems. Problem 6.4
\item Un electrón se encuentra dentro de un pozo de potencial infinito esférico:
\begin{align*}
V(r) = \begin{cases}
0 & r < a \\
+\infty & r > a
\end{cases}
\end{align*}
\begin{enumerate}
\item Usando la ecuación radial de Schödinger, calcula las autoenergías ligadas y las correspondientes funciones de onda radiales normalizadas, para el caso donde el momento angular orbital del electrón es cero, es decir, cuando $l = 0$.
\item Demuestra que el estado de energía más bajo para $l = 7$, está por arriba del segundo estado de energía más bajo para $l = 0$.
\end{enumerate}
%Referencia: Davis - Integral transforms and their applications. Problem 7.2
\item Para una función $f(x)$, con la transformada de Fourier $F(p)$, se definen las cantidades:

\begin{minipage}{0.4\linewidth}
\begin{align*}
\expval{x^{n}} &= \int_{-\infty}^{\infty} x^{n} \, \abs{f(x)}^{2} \dd{x} \\[0.5em]
(\Delta x)^{2} &= \expval{x^{2}} - \expval{x}^{2}
\end{align*}
\end{minipage} \hspace{0.3cm}
\begin{minipage}{0.4\linewidth}
\begin{align*}
\expval{p^{n}} &= \dfrac{1}{2 \pi} \int_{-\infty}^{\infty} p^{n} \, \abs{F(p)}^{2} \dd{p} \\[0.5em]
(\Delta p)^{2} &= \expval{p^{2}} - \expval{p}^{2}
\end{align*}
\end{minipage}

Considera la siguiente desigualdad
\begin{align*}
\int_{-\infty}^{\infty} \abs{x \, f(x) - \expval{x} \, f(x)  + \alpha \left\{ f^{\prime}(x) + i \expval{p} \, f(x) \right\}}^{2} \dd{x} \geq 0
\end{align*}
donde $\alpha$ es un número real arbitrario. Demuestra que
\begin{align*}
(\Delta x) (\Delta p) \geq \dfrac{1}{2}
\end{align*}
\end{enumerate}
\end{document}