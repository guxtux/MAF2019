\documentclass[12pt]{article}
\usepackage[letterpaper]{geometry}
%\textwidth = 345.0pt
%\hoffset = -3cm
\usepackage[utf8]{inputenc}
\usepackage[spanish,es-tabla]{babel}
\usepackage[autostyle,spanish=mexican]{csquotes}
\usepackage{amsmath}
\usepackage{standalone}
\usepackage{tikz}   
\usepackage{tikz-3dplot}
\usepackage{nccmath}
\usepackage{amsthm}
\usepackage{amssymb}
\usepackage{graphicx}
\usepackage{comment}
\usepackage{siunitx}
\usepackage{physics}
\usepackage{color}
\usepackage{float}
\usepackage{multicol}
%\usepackage{milista}
\usepackage{enumitem}
\usepackage{anyfontsize}
\usepackage{anysize}
\marginsize{1cm}{1cm}{2cm}{2cm}
\usepackage{enumitem}
\usepackage{capt-of}
\usepackage{bm}
\usepackage{relsize}
\newlist{milista}{enumerate}{2}
\setlist[milista,1]{label=\arabic*)}
\setlist[milista,2]{label=\arabic{milistai}.\arabic*)}
\spanishdecimal{.}
\renewcommand{\baselinestretch}{1.5}
\author{ }
\usepackage{etoolbox}
\AtBeginEnvironment{tikzpicture}{\shorthandoff{>}\shorthandoff{<}}{}{}
\author{}
\title{Examen Final B  \\ \large{Matemáticas Avanzadas de la Física}} \vspace{-1.5\baselineskip}
\date{ }
\begin{document}
\vspace{-4cm}
\renewcommand\labelenumii{\theenumi.{\arabic{enumii})}}
\maketitle
\fontsize{14}{14}\selectfont
\begin{enumerate}
\item Un análisis con la mecánica cuántica del efecto Stark (en coordenadas parabólicas), nos lleva a la siguiente ecuación diferencial
\begin{align*}
\dv{\xi} \left( \xi \, \dv{u}{\xi}  \right) + \left( \dfrac{1}{2} \, E \, \xi + L - \dfrac{m^{2}}{4 \, \xi} - \dfrac{1}{4} \, F \, \xi^{2} \right) \, u = 0
\end{align*}
Donde $F$ es una medida de la perturbación en la energía provocada por un campo eléctrico externo. Encuentra las funciones de onda sin perturbaciones $(F=0)$, en términos de los polinomios asociados de Laguerre.
\item \begin{enumerate}
\item Usando el teorema del desarrollo, escribe la forma general para una función de Green que sea solución a una ecuación de valores propios.
\item Usando el resultado anterior, calcula la función de Green de una partícula en un pozo cuadrado o infinito de ancho $a$ unidimensional.
\end{enumerate}
\item Un análisis para los patrones de radiación de las antenas para un sistema con una apertura circular, involucra la ecuación
\begin{align*}
g(u) = \int_{0}^{1} f(r) \, J_{0} (u \, r) \, r \dd{r}
\end{align*}
Si $f(r) = 1 - r^{2}$, demuestra que
\begin{align*}
g(u) = \dfrac{2}{u^{2}} \, J_{2}(u)
\end{align*}
\item \begin{enumerate}
\item Obtén la expansión de Jacobi-Anger
\begin{align*}
\exp(i \, z \, \cos \theta) = \sum_{m=-\infty}^{\infty} i^{m} \, J_{m}(z) \, \exp(i \, m \, \theta)
\end{align*}
Esta es una expansión para una onda plana en términos de una serie de ondas cilíndricas.
\item Demuestra que
\begin{align*}
\cos x = J_{0} (x) + 2 \, \sum_{n=1}^{\infty} (-1)^{n} \, J_{2n} (x)
\end{align*}
\end{enumerate}
\item Demuestra que
\begin{align*}
\dfrac{\ket{\psi}\bra{\psi}}{\braket{\psi}{\psi}}
\end{align*} es un operador de proyección, independientemente de que $\ket{\psi}$ está normalizado o no.
\end{enumerate}
\end{document}