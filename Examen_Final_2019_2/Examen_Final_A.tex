\documentclass[12pt]{article}
\usepackage[letterpaper]{geometry}
%\textwidth = 345.0pt
%\hoffset = -3cm
\usepackage[utf8]{inputenc}
\usepackage[spanish,es-tabla]{babel}
\usepackage[autostyle,spanish=mexican]{csquotes}
\usepackage{amsmath}
\usepackage{standalone}
\usepackage{tikz}   
\usepackage{tikz-3dplot}
\usepackage{nccmath}
\usepackage{amsthm}
\usepackage{amssymb}
\usepackage{graphicx}
\usepackage{comment}
\usepackage{siunitx}
\usepackage{physics}
\usepackage{color}
\usepackage{float}
\usepackage{multicol}
%\usepackage{milista}
\usepackage{enumitem}
\usepackage{anyfontsize}
\usepackage{anysize}
\marginsize{1cm}{1cm}{2cm}{2cm}
\usepackage{enumitem}
\usepackage{capt-of}
\usepackage{bm}
\usepackage{relsize}
\newlist{milista}{enumerate}{2}
\setlist[milista,1]{label=\arabic*)}
\setlist[milista,2]{label=\arabic{milistai}.\arabic*)}
\spanishdecimal{.}
\renewcommand{\baselinestretch}{1.5}
\author{ }
\usepackage{etoolbox}
\AtBeginEnvironment{tikzpicture}{\shorthandoff{>}\shorthandoff{<}}{}{}
\author{}
\title{Examen Final A  \\ \large{Matemáticas Avanzadas de la Física}} \vspace{-1.5\baselineskip}
\date{ }
\begin{document}
\vspace{-4cm}
\renewcommand\labelenumii{\theenumi.{\arabic{enumii})}}
\maketitle
\fontsize{14}{14}\selectfont
\begin{enumerate}
\item Dos hemisferios sólidos conductores de calor de radio $a$, separados por un hueco aislante muy pequeño, forman una esfera. Cada mitad de la esfera está en contacto - por fuera - con dos baños de calor (infinitos) a temperaturas $T_{0}$ y $-T_{0}$. Encuentra la distribución de temperatura $T (r, \theta, \phi)$ en el interior de la esfera.
\begin{figure}[H]
    \centering
    \includestandalone{Figuras/esfera_2}
    \caption{Dos semiesferas separadas infinitesimalmente a temperaturas opuestas.}
    \label{fig:figura2}
\end{figure}
\item Calcula el potencial electrostático para un arreglo de cargas como se muestra en la siguiente figura. Este arreglo corresponde a un cuadrúpolo eléctrico lineal.
\item Demuestra que
\begin{align*}
P_{2n}^{1} (0) &= 0 \\
P_{2n+1}^{1} (0) &= (-1)^{n} \, \dfrac{(2n + 1)!}{(2^{2} \, n!)^{2}} = (-1)^{2} \, \dfrac{(2n + 1)!!}{(2n)!!}
\end{align*}
Utilizando cada uno de los siguientes métodos:
\begin{enumerate}
\item Usando la relación de recurrencia.
\item Expandiendo la función generatriz.
\item Con la fórmula de Rodrigues.
\end{enumerate}
\item Demuestra que
\begin{align*}
\dfrac{1}{\pi} \, \lim_{s \to 0} \mathcal{L} \left\{ \cos x t \right\} = \delta (x)
\end{align*}
\end{enumerate}
\end{document}