\documentclass[12pt]{article}
\usepackage[letterpaper]{geometry}
%\textwidth = 345.0pt
%\hoffset = -3cm
\usepackage[utf8]{inputenc}
\usepackage[spanish,es-tabla]{babel}
\usepackage[autostyle,spanish=mexican]{csquotes}
\usepackage{amsmath}
\usepackage{nccmath}
\usepackage{amsthm}
\usepackage{amssymb}
\usepackage{graphicx}
\usepackage{comment}
\usepackage{siunitx}
\usepackage{physics}
\usepackage{color}
\usepackage{float}
\usepackage{multicol}
%\usepackage{milista}
\usepackage{enumitem}
\usepackage{anyfontsize}
\usepackage{anysize}
\marginsize{1cm}{1cm}{1cm}{1cm}
\usepackage{enumitem}
\usepackage{capt-of}
\usepackage{bm}
\usepackage{relsize}
\newlist{milista}{enumerate}{2}
\setlist[milista,1]{label=\arabic*)}
\setlist[milista,2]{label=\arabic{milistai}.\arabic*)}
\spanishdecimal{.}
\renewcommand{\baselinestretch}{1.5}
\author{ }
\author{}
\title{Examen Final C  \\ \large{Matemáticas Avanzadas de la Física}} \vspace{-1.5\baselineskip}
\date{ }
\begin{document}
\vspace{-4cm}
\renewcommand\labelenumii{\theenumi.{\arabic{enumii})}}
\maketitle
\fontsize{14}{14}\selectfont
\begin{enumerate}
    \item Las coordenadas cilíndricas elípticas se construyen partiendo de una familia de elipses confocales
    \begin{align*}
    \dfrac{x^{2}}{A^{2}} +  \dfrac{y^{2}}{A^{2} - a^{2}} =  1 \hspace{1cm} A \geq a
    \end{align*}
    La familia de curvas ortogonales a las elipses en cada punto es una familia de hipérbolas confocales de la forma
    \begin{align*}
    \dfrac{x^{2}}{C^{2}} - \dfrac{y^{2}}{a^{2} - C^{2}} = 1 \hspace{1cm} C \leq a
    \end{align*}
    Las \textbf{coordenadas cilíndricas elípticas} $(\xi, \eta, z)$ se obtienen haciendo $A = a \, \cosh \xi$ y $C = a \, \sen \eta$.
    \par
    Las superficies coordenadas son:
    \begin{enumerate}[label=\alph*)]
    \item Cilíndricos elípticos $(\xi = \mbox{ constante})$.
    \item Cilíndros hiperbólicos $(\eta = \mbox{ constante})$.
    \item Planos perpendiculares al eje $z$.
    \end{enumerate}
    Las correspondientes reglas de transformación son:
    \begin{align*}
    x &= a \, \cosh \xi \, \cos \eta \\
    y &= a \, \sinh \xi \, \sin \eta \\
    z &= z
    \end{align*}
    con $\xi: 0 \to \infty, \eta: 0 \to 2 \, \pi, z: -\infty \to \infty$
    \begin{enumerate}
    \item Escribe los símbolos de Christoffel de este sistema coordenado.
    \item Calcula $h_{\xi}, h_{\eta}, h_{z}$
    \item Escribe los operadores $\grad{\varphi}, \div{\vb{A}}, \curl \vb{A}, \laplacian{\varphi}$
    \end{enumerate}
\item La sección diferencial en un experimento de dispersión nuclear está dada por $\dv*{\sigma}{\Omega} = \abs{f(\theta)^{2}}$. Un tratamiento aproximado nos lleva a
\begin{align*}
f(\theta) = \dfrac{- i \, k}{2 \, \pi} \int_{0}^{2 \pi} \int_{0}^{R} \exp(i \, k \, \rho \, \sin \theta \, \sin \varphi ) \,  \rho \, \dd{rho} \dd{\varphi}
\end{align*}
Donde $\theta$ es un ángulo en donde la partícula se dispersa. $R$ es el radio nuclear. Demuestra que
\begin{align*}
\dv{\sigma}{\Omega} = (\pi \, R^{2}) \, \dfrac{1}{\pi} \left[ \dfrac{J_{1} (k \, R \, \sin \theta)}{\sin \theta} \right]^{2}
\end{align*}
\item \begin{enumerate}
\item Obtén la expansión de Jacobi-Anger
\begin{align*}
\exp(i \, z \, \cos \theta) = \sum_{m=-\infty}^{\infty} i^{m} \, J_{m}(z) \, \exp(i \, m \, \theta)
\end{align*}
Esta es una expansión para una onda plana en términos de una serie de ondas cilíndricas.
\item Demuestra que
\begin{align*}
\sin x = 2 \, \sum_{n=0}^{\infty} (-1)^{n} \, J_{2n+} (x)
\end{align*}
\end{enumerate}
\item Demuestra que si $\hat{A}$ es un operador de proyección, entonces el operador $1 - \hat{A}$ es también un operador de proyección.
\item Calcula $\expval{x}$, $\expval{p}$, $\expval{x^{2}}$, $\expval{p^{2}}$ y $\expval{T}$ para el $n$ estado estacionario del oscilador armónico cuántico. Comprueba que se cumple el principio de incertidumbre.
\end{enumerate}
\end{document}