\documentclass[12pt]{article}
\usepackage[letterpaper]{geometry}
%\textwidth = 345.0pt
%\hoffset = -3cm
\usepackage[utf8]{inputenc}
\usepackage[spanish,es-tabla]{babel}
\usepackage[autostyle,spanish=mexican]{csquotes}
\usepackage{amsmath}
\usepackage{standalone}
\usepackage{tikz}   
\usepackage{tikz-3dplot}
\usepackage{nccmath}
\usepackage{amsthm}
\usepackage{amssymb}
\usepackage{graphicx}
\usepackage{comment}
\usepackage{siunitx}
\usepackage{physics}
\usepackage{color}
\usepackage{float}
\usepackage{multicol}
%\usepackage{milista}
\usepackage{enumitem}
\usepackage{anyfontsize}
\usepackage{anysize}
\marginsize{1cm}{1cm}{2cm}{2cm}
\usepackage{enumitem}
\usepackage{capt-of}
\usepackage{bm}
\usepackage{relsize}
\newlist{milista}{enumerate}{2}
\setlist[milista,1]{label=\arabic*)}
\setlist[milista,2]{label=\arabic{milistai}.\arabic*)}
\spanishdecimal{.}
\renewcommand{\baselinestretch}{1.5}
\author{ }
\usepackage{etoolbox}
\AtBeginEnvironment{tikzpicture}{\shorthandoff{>}\shorthandoff{<}}{}{}
\author{}
\title{Examen Final  \\ \large{Matemáticas Avanzadas de la Física}\vspace{-50pt}}
\date{ }
\begin{document}
\newgeometry{margin=1.5cm}
\renewcommand\labelenumii{\theenumi.{\arabic{enumii})}}
\maketitle
\fontsize{14}{14}\selectfont
\begin{enumerate}
    % \item Las coordenadas cilíndricas elípticas se construyen partiendo de una familia de elipses confocales
    % \begin{align*}
    % \dfrac{x^{2}}{A^{2}} +  \dfrac{y^{2}}{A^{2} - a^{2}} =  1 \hspace{1cm} A \geq a
    % \end{align*}
    % La familia de curvas ortogonales a las elipses en cada punto es una familia de hipérbolas confocales de la forma
    % \begin{align*}
    % \dfrac{x^{2}}{C^{2}} - \dfrac{y^{2}}{a^{2} - C^{2}} = 1 \hspace{1cm} C \leq a
    % \end{align*}
    % Las \textbf{coordenadas cilíndricas elípticas} $(\xi, \eta, z)$ se obtienen haciendo $A = a \, \cosh \xi$ y $C = a \, \sen \eta$.
    % \par
    % Las superficies coordenadas son:
    % \begin{enumerate}[label=\alph*)]
    % \item Cilíndricos elípticos $(\xi = \mbox{ constante})$.
    % \item Cilíndros hiperbólicos $(\eta = \mbox{ constante})$.
    % \item Planos perpendiculares al eje $z$.
    % \end{enumerate}
    % Las correspondientes reglas de transformación son:
    % \begin{align*}
    % x &= a \, \cosh \xi \, \cos \eta \\
    % y &= a \, \sinh \xi \, \sin \eta \\
    % z &= z
    % \end{align*}
    % con $\xi: 0 \to \infty, \eta: 0 \to 2 \, \pi, z: -\infty \to \infty$
    % \begin{enumerate}
    % \item Escribe los símbolos de Christoffel de este sistema coordenado.
    % \item Calcula $h_{\xi}, h_{\eta}, h_{z}$
    % \item Escribe los operadores $\grad{\varphi}, \div{\vb{A}}, \curl \vb{A}, \laplacian{\varphi}$
    % \end{enumerate}
    \item Una esfera conductora de calor de radio $a$ está compuesta por dos hemisferios con un espacio infinitesimal aislante entre ellos, como se muestra en la figura (\ref{fig:figura2}). Las mitades superior e inferior de la esfera están en contacto con baños térmicos de temperaturas $+ T_{1}$ y $-T_{1}$, respectivamente. La esfera de radio $a$ está dentro de otra esfera conductora de calor de radio $b$ con una temperatura $T_{2}$. Encuentra la temperatura en los puntos:
    \begin{enumerate}
    \item Dentro de la esfera interior,
    \item En la región entre las dos esferas y
    \item Por fuera de la esfera exterior.
    \end{enumerate} 
    \begin{figure}[!ht]
        \centering
        \includestandalone[scale=0.7]{Figuras/esfera1}
        %\includestandalone{esfera1}
        \caption{Los hemisferios de la esfera interior se encuentran a diferentes temperatura.}
        \label{fig:figura2}
    \end{figure}
% \item La sección diferencial en un experimento de dispersión nuclear está dada por 
% \begin{align*}
% \dv*{\sigma}{\Omega} = \abs{f(\theta)}^{2}
% \end{align*}
% Un tratamiento aproximado nos lleva a
% \begin{align*}
% f(\theta) = \dfrac{- i \, k}{2 \, \pi} \int_{0}^{2 \pi} \int_{0}^{R} \exp(i \, k \, \rho \, \sin \theta \, \sin \varphi ) \,  \rho \, \dd{\rho} \dd{\varphi}
% \end{align*}
% Donde $\theta$ es un ángulo en donde la partícula se dispersa. $R$ es el radio nuclear. Demuestra que
% \begin{align*}
% \dv{\sigma}{\Omega} = (\pi \, R^{2}) \, \dfrac{1}{\pi} \left[ \dfrac{J_{1} (k \, R \, \sin \theta)}{\sin \theta} \right]^{2}
% \end{align*}
% \item En este problema hay que resolver el primer inciso, para luego utilizar el resultado en el segundo inciso. 
% \begin{enumerate}
% \item Obtén la expansión de Jacobi-Anger
% \begin{align*}
% \exp(i \, z \, \cos \theta) = \sum_{m=-\infty}^{\infty} i^{m} \, J_{m}(z) \, \exp(i \, m \, \theta)
% \end{align*}
% Esta es una expansión para una onda plana en términos de una serie de ondas cilíndricas.
% \item Demuestra que
% \begin{align*}
% \sin x = 2 \, \sum_{n=0}^{\infty} (-1)^{n} \, J_{2n+} (x)
% \end{align*}
% \end{enumerate}
% \item Calcula $\expval{x}$, $\expval{p}$, $\expval{x^{2}}$, $\expval{p^{2}}$ y $\expval{T}$ para el $n$ estado estacionario del oscilador armónico cuántico. Comprueba que se cumple el principio de incertidumbre.
\item Debes de resolver el primer inciso para luego ocupar el resultado en el segundo inciso.
\begin{enumerate}
\item Usando el teorema del desarrollo, escribe la forma general para una función de Green que sea solución a una ecuación de valores propios.
\item Calcula la función de Green de una partícula en un pozo infinito de ancho $a$ unidimensional.
\end{enumerate}
\item Una buena aproximación para la interacción entre dos nucleones puede describirse mediante un potencial mesónico atractivo con $A$ negativo:
\begin{align*}
V = \dfrac{A \, \exp(- a \, x)}{x}
\end{align*}
Desarrolla una solución en serie para la ecuación de onda de Schrödinger
\begin{align*}
\dfrac{\hbar^{2}}{2 \, m} \, \dv[2]{\psi}{x} + (E - V) \, \psi = 0
\end{align*}
con los primeros tres coeficientes no nulos.
\item La ecuación unidimensional de onda de Schrödinger es
\[ - \dfrac{\hbar^{2}}{2 \, m} \; \dv[2]{\psi(x)}{x} +  V(x) \, \psi(x) = E \, \psi (x) \]
Para el caso especial de que $V(x)$ es una función analítica de $x$, demuestra que la correspondiente ecuación de onda para el momento es
\[ V \left( i \, \hbar \dv{p} \right) g(p) + \dfrac{p^{2}}{2 \, m} \, g(p) =  E \, g(p)  \]
Recupera esta ecuación de onda para el momento usando la transformada de Fourier y su inversa. No utilices la sustitución directa $\displaystyle{x \to i \, \hbar (\dv{p})}$.
\end{enumerate}
\end{document}