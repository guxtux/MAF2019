\documentclass[12pt]{article}
\usepackage[letterpaper]{geometry}
%\textwidth = 345.0pt
%\hoffset = -3cm
\usepackage[utf8]{inputenc}
\usepackage[spanish,es-tabla]{babel}
\usepackage[autostyle,spanish=mexican]{csquotes}
\usepackage{amsmath}
\usepackage{nccmath}
\usepackage{amsthm}
\usepackage{amssymb}
\usepackage{graphicx}
\usepackage{comment}
\usepackage{siunitx}
\usepackage{physics}
\usepackage{color}
\usepackage{float}
\usepackage{multicol}
%\usepackage{milista}
\usepackage{enumitem}
\usepackage{anyfontsize}
\usepackage{anysize}
\marginsize{1cm}{1cm}{1cm}{1cm}
\usepackage{enumitem}
\usepackage{capt-of}
\usepackage{bm}
\usepackage{relsize}
\newlist{milista}{enumerate}{2}
\setlist[milista,1]{label=\arabic*)}
\setlist[milista,2]{label=\arabic{milistai}.\arabic*)}
\spanishdecimal{.}
\renewcommand{\baselinestretch}{1.5}
\author{ }
\author{}
\title{Examen Final \\ \large{Matemáticas Avanzadas de la Física}\vspace{-50pt}}
\date{ }
\begin{document}
\newgeometry{margin=1.5cm}
\renewcommand\labelenumii{\theenumi.{\arabic{enumii})}}
\maketitle
\fontsize{14}{14}\selectfont
\begin{enumerate}
\item Un análisis con la mecánica cuántica del efecto Stark (en coordenadas parabólicas), nos lleva a la siguiente ecuación diferencial
\begin{align*}
\dv{\xi} \left( \xi \, \dv{u}{\xi}  \right) + \left( \dfrac{1}{2} \, E \, \xi + L - \dfrac{m^{2}}{4 \, \xi} - \dfrac{1}{4} \, F \, \xi^{2} \right) \, u = 0
\end{align*}
Donde $F$ es una medida de la perturbación en la energía provocada por un campo eléctrico externo. Encuentra las funciones de onda sin perturbaciones $(F=0)$, en términos de los polinomios asociados de Laguerre.
% \item \begin{enumerate}
% \item Usando el teorema del desarrollo, escribe la forma general para una función de Green que sea solución a una ecuación de valores propios.
% \item Usando el resultado anterior, calcula la función de Green de una partícula en un pozo cuadrado o infinito de ancho $a$ unidimensional.
% \end{enumerate}
\item Demuestra ahora que el momento angular $\vb{L}$, es un operador Hermitiano:
\[ \vb{L} = - i\, \hbar \, \vb{r} \times \nabla \equiv i \, \dfrac{h}{2 \, \pi} \, \vb{r} \times \nabla \]
\item Un análisis para los patrones de radiación de las antenas para un sistema con una apertura circular, involucra la ecuación
\begin{align*}
g(u) = \int_{0}^{1} f(r) \, J_{0} (u \, r) \, r \dd{r}
\end{align*}
Si $f(r) = 1 - r^{2}$, demuestra que
\begin{align*}
g(u) = \dfrac{2}{u^{2}} \, J_{2}(u)
\end{align*}
% \item \begin{enumerate}
% \item Obtén la expansión de Jacobi-Anger
% \begin{align*}
% \exp(i \, z \, \cos \theta) = \sum_{m=-\infty}^{\infty} i^{m} \, J_{m}(z) \, \exp(i \, m \, \theta)
% \end{align*}
% Esta es una expansión para una onda plana en términos de una serie de ondas cilíndricas.
% \item Demuestra que
% \begin{align*}
% \cos x = J_{0} (x) + 2 \, \sum_{n=1}^{\infty} (-1)^{n} \, J_{2n} (x)
% \end{align*}
% \end{enumerate}
% \item Demuestra que
% \begin{align*}
% \dfrac{\ket{\psi}\bra{\psi}}{\braket{\psi}{\psi}}
% \end{align*} es un operador de proyección, independientemente de que $\ket{\psi}$ está normalizado o no.
\item La ecuación unidimensional de onda de Schrödinger es
\[ - \dfrac{\hbar^{2}}{2 \, m} \; \dv[2]{\psi(x)}{x} +  V(x) \, \psi(x) = E \, \psi (x) \]
Para el caso especial de que $V(x)$ es una función analítica de $x$, demuestra que la correspondiente ecuación de onda para el momento es
\[ V \left( i \, \hbar \dv{p} \right) g(p) + \dfrac{p^{2}}{2 \, m} \, g(p) =  E \, g(p)  \]
Recupera esta ecuación de onda para el momento usando la transformada de Fourier y su inversa. No utilices la sustitución directa $\displaystyle{x \to i \, \hbar (\dv{p})}$.
\end{enumerate}
\end{document}