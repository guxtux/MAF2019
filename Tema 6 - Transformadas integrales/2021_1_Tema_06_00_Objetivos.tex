\documentclass[10pt]{beamer}
\usetheme[
%%% option passed to the outer theme
%    progressstyle=fixedCircCnt,   % fixedCircCnt, movingCircCnt (moving is deault)
  ]{Feather}
  
% If you want to change the colors of the various elements in the theme, edit and uncomment the following lines

% Change the bar colors:
%\setbeamercolor{Feather}{fg=red!20,bg=red}

% Change the color of the structural elements:
%\setbeamercolor{structure}{fg=red}

% Change the frame title text color:
%\setbeamercolor{frametitle}{fg=blue}

% Change the normal text color background:
%\setbeamercolor{normal text}{fg=black,bg=gray!10}

%-------------------------------------------------------
% INCLUDE PACKAGES
%-------------------------------------------------------

\usepackage[utf8]{inputenc}
\usepackage[spanish]{babel}
\usepackage[T1]{fontenc}
\usepackage{helvet}
\usepackage{multirow}

%-------------------------------------------------------
% DEFFINING AND REDEFINING COMMANDS
%-------------------------------------------------------

% colored hyperlinks
\newcommand{\chref}[2]{
  \href{#1}{{\usebeamercolor[bg]{Feather}#2}}
}

%-------------------------------------------------------
% INFORMATION IN THE TITLE PAGE
%-------------------------------------------------------

\title[] % [] is optional - is placed on the bottom of the sidebar on every slide
{ % is placed on the title page
      \textbf{Tema 6 - Transformadas integrales \\ \medskip
      \large{Matemáticas Avanzadas de la Física}}
}

\subtitle[Transformadas integrales]
{
%      \textbf{v. 1.0.0}
}

\author[M. en C. Gustavo Contreras Mayén]
{      M. en C. Gustavo Contreras Mayén \\
      {\ttfamily gux7avo@ciencias.unam.mx}
}

\institute[]
{
      Departmento de Física \\
      Facultad de Ciencias, UNAM \\
  
  %there must be an empty line above this line - otherwise some unwanted space is added between the university and the country (I do not know why;( )
}

\date{\today}

%-------------------------------------------------------
% THE BODY OF THE PRESENTATION
%-------------------------------------------------------

\begin{document}

%-------------------------------------------------------
% THE TITLEPAGE
%-------------------------------------------------------

{\1% % this is the name of the PDF file for the background
\begin{frame}[plain,noframenumbering] % the plain option removes the header from the title page, noframenumbering removes the numbering of this frame only
  \titlepage % call the title page information from above
\end{frame}}


\begin{frame}{Contenido}{}
\tableofcontents
\end{frame}

%-------------------------------------------------------
\section{Objetivos}
%-------------------------------------------------------
\begin{frame}{Objetivos}
%-------------------------------------------------------
\begin{itemize}
\item<1-> El alumno identificará la naturaleza de las transformadas integrales, así como los distintos tipos que existe en la física matemática.
\item <2-> Aplicará la transformada de Fourier para resolver distintos tipos de problemas con ecuaciones diferenciales parciales.
\item <3-> Reconocerá y utilizará la transformada de Laplace en ejercicios con ecuaciones diferenciales ordinarias y parciales.
\end{itemize}
\end{frame}
%-------------------------------------------------------
\section{Introducción a las transformadas integrales}
\subsection{Marco teórico}

\begin{frame}{Introducción a las transformadas integrales}{Marco teórico}
Se revisará de manera general y con un punto de vista de aplicación, el concepto de transformada integral, así como los distintos tipos que se suelen encontrar en la Física Matemática.
\end{frame}
% %-------------------------------------------------------
\subsection{Tipos de transformadas integrales}

\begin{frame}{Introducción a las transformadas integrales}{Tipos de transformadas integrales}
Una vez revisada la definición de transformada integral, se presenta una lista con una serie de transformadas integrales, las de más uso en ciencia e ingeniería, así como transformadas especiales que se ocupan a partir de funciones especiales.
\\
\bigskip
\pause
Cada una de estas transformadas tiene su correspondiente transformada inversa.
\end{frame}
%--------------------------------------------------------------
\section{Transformadas Integrales}
\subsection{Transformada de Fourier}

\begin{frame}{Transformadas integrales}{La Transformada de Fourier}
Se hace una revisión sobre la Transformada de Fourier así como la Transformada inversa.
\\
\bigskip
\pause
Como herramienta necesaria se recomienda hacer una revisión sobre el tema de series de Fourier, ya que se ocuparán algunos resultados de ese tema.
\end{frame}
\begin{frame}{Transformadas integrales}{La Transformada de Fourier}
Se discutirán una serie de propiedades de la Transformada de Fourier, en donde para algunos casos, se revisará la demostración de esas propiedades, dejando otras para revisión por parte del alumno, como tarea moral.
\\
\bigskip
\pause
Ocupando la definición de las transformadas es posible obtener el resultado, siguiendo una serie de pasos sencillos.
\end{frame}
%--------------------------------------------------------------
\subsection{Transformada de Laplace}

\begin{frame}{Transformadas integrales}{La Transformada de Laplace}
Se hace una revisión sobre la Transformada de Laplace así como la Transformada inversa, destacando que esta transformada es la de mayor utilidad en física matemática.
\\
\bigskip
\pause
También se presentarán un conjunto de propiedades de la Transformada de Laplace para ocuparla como herramienta de solución a problemas con ecuaciones diferenciales.
\end{frame}
%--------------------------------------------------------------
\section{Calendarización}
\subsection{Semanas de trabajo}

\begin{frame}{Calendarización}{Semanas de trabajo}
Tendremos las dos últimas semanas de trabajo del semestre:
\setbeamercolor{item projected}{bg=blue!70!black,fg=yellow}
\setbeamertemplate{enumerate items}[circle]
\begin{enumerate}[<+->]
\item Del 18 al 22 de enero de 2021.
\item Del 25 al 29 de enero de 2021.
\end{enumerate}
\end{frame}
\begin{frame}{Calendarización}{Semana 15}
Se dejarán los materiales de trabajo:
\begin{itemize}
\item Introducción a las transformadas integrales.
\item Transformada de Fourier.
\item Ejercicios
\end{itemize}
\end{frame}
\begin{frame}{Calendarización}{Semana 16}
Se dejarán los materiales de trabajo:
\begin{itemize}
\item Transformada de Laplace.
\item Ejercicios
\end{itemize}
\end{frame}
%--------------------------------------------------------------
\subsection{Sesiones síncronas}

\begin{frame}{Calendarización}{Sesiones Zoom}
Se han programado las siguientes sesiones de trabajo síncronas en Zoom a las 3 pm:
\setbeamercolor{item projected}{bg=blue!70!black,fg=yellow}
\setbeamertemplate{enumerate items}[circle]
\begin{enumerate}[<+->]
\item Lunes 18 de enero de 2021.
\item Viernes 22 de enero de 2021.
\item Miércoles 27 de enero de 2021.
\item Viernes 29 de enero de 2021.
\end{enumerate}
\end{frame}
%--------------------------------------------------------------
\section{Evaluación}
\subsection{Última parte del segundo parcial}
\begin{frame}{Evaluación}{Tema 6}
Se dejará una tarea examen que corresponde a este tema, con lo que se conforma el segundo examen parcial del curso.
\\
\bigskip
\pause
Se adelantarán los ejercicios sobre la transformada de Laplace, así como los contenidos de la semana 16, para que quien decida avanzar, logré terminar a tiempo el examen tarea.
\end{frame}
\end{document}