\documentclass[hidelinks,12pt]{article}
\usepackage[left=0.25cm,top=1cm,right=0.25cm,bottom=1cm]{geometry}
%\usepackage[landscape]{geometry}
\textwidth = 20cm
\hoffset = -1cm
\usepackage[utf8]{inputenc}
\usepackage[spanish,es-tabla]{babel}
\usepackage[autostyle,spanish=mexican]{csquotes}
\usepackage[tbtags]{amsmath}
\usepackage{nccmath}
\usepackage{amsthm}
\usepackage{amssymb}
\usepackage{mathrsfs}
\usepackage{graphicx}
\usepackage{subfig}
\usepackage{standalone}
\usepackage[outdir=./Imagenes/]{epstopdf}
\usepackage{siunitx}
\usepackage{physics}
\usepackage{color}
\usepackage{float}
\usepackage{hyperref}
\usepackage{multicol}
%\usepackage{milista}
\usepackage{anyfontsize}
\usepackage{anysize}
%\usepackage{enumerate}
\usepackage[shortlabels]{enumitem}
\usepackage{capt-of}
\usepackage{bm}
\usepackage{relsize}
\usepackage{placeins}
\usepackage{empheq}
\usepackage{cancel}
\usepackage{wrapfig}
\usepackage[flushleft]{threeparttable}
\usepackage{makecell}
\usepackage{fancyhdr}
\usepackage{tikz}
\usepackage{bigints}
\usepackage{scalerel}
\usepackage{pgfplots}
\usepackage{pdflscape}
\pgfplotsset{compat=1.16}
\spanishdecimal{.}
\renewcommand{\baselinestretch}{1.5} 
\renewcommand\labelenumii{\theenumi.{\arabic{enumii}})}
\newcommand{\ptilde}[1]{\ensuremath{{#1}^{\prime}}}
\newcommand{\stilde}[1]{\ensuremath{{#1}^{\prime \prime}}}
\newcommand{\ttilde}[1]{\ensuremath{{#1}^{\prime \prime \prime}}}
\newcommand{\ntilde}[2]{\ensuremath{{#1}^{(#2)}}}

\newtheorem{defi}{{\it Definición}}[section]
\newtheorem{teo}{{\it Teorema}}[section]
\newtheorem{ejemplo}{{\it Ejemplo}}[section]
\newtheorem{propiedad}{{\it Propiedad}}[section]
\newtheorem{lema}{{\it Lema}}[section]
\newtheorem{cor}{Corolario}
\newtheorem{ejer}{Ejercicio}[section]

\newlist{milista}{enumerate}{2}
\setlist[milista,1]{label=\arabic*)}
\setlist[milista,2]{label=\arabic{milistai}.\arabic*)}
\newlength{\depthofsumsign}
\setlength{\depthofsumsign}{\depthof{$\sum$}}
\newcommand{\nsum}[1][1.4]{% only for \displaystyle
    \mathop{%
        \raisebox
            {-#1\depthofsumsign+1\depthofsumsign}
            {\scalebox
                {#1}
                {$\displaystyle\sum$}%
            }
    }
}
\def\scaleint#1{\vcenter{\hbox{\scaleto[3ex]{\displaystyle\int}{#1}}}}
\def\bs{\mkern-12mu}


%\usepackage{showframe}
\title{Ejercicios Transformadas de Fourier \\ \large {Tema 6 - Transformadas integrales} \vspace{-3ex}}
\author{M. en C. Gustavo Contreras Mayén}
\date{ }
\begin{document}
\vspace{-4cm}
\maketitle
\fontsize{14}{14}\selectfont
\tableofcontents
\newpage

\section{Temperatura en una placa semiinfinita.}

La temperatura constante de una placa semiinfinita está dada por la siguiente ecuación diferencial, condiciones iniciales y de frontera:
\begin{align*}
&\pdv[2]{u}{x} + \pdv[2]{u}{y} = 0 \hspace{1cm} 0 < x < \pi, \hspace{0.3cm} y > 0 \\[0.5cm]
&u(0, y) = 0 \hspace{1cm} u(\pi, u) = e^{-y} \hspace{0.3cm} y > 0 \\[0.5cm]
&\pdv{u}{y} \eval_{y=0} = 0 \hspace{1cm} 0 < x < \pi
\end{align*}
\textbf{Encuentra el valor de temperatura de $u(x,y)$}.
\\[0.5em]
\texttt{Solución. } El dominio de la variable y la condición prescrita en $y = 0$, indican que se puede aplicar la transformada coseno de Fourier al problema, así:
\begin{align*}
F_{c} \big[u(x,y)\big] = \int_{0}^{\infty} u(x, y) \, \cos \alpha \, y \dd{y} = U(x, \alpha)
\end{align*}
Como la transformada coseno de Fourier de la derivada de una función es:
\begin{align*}
F_{c} \big[\stilde{y}(x)] = - \alpha^{2} \, F[\alpha] - \ptilde{f} (0)
\end{align*}
se tiene que:
\begin{align*}
F_{c} \left[\pdv[2]{u}{x} \right] + F_{c} \left[\pdv[2]{u}{y} \right] = F_{c} [0]
\end{align*}
por lo tanto:
\begin{align*}
\dv[2] - \alpha^{2} \, U(x, \alpha) - u_{y} (x, 0) = 0 \hspace{0.3cm} \Rightarrow \hspace{0.3cm} \dv[2]{U}{x} - \alpha^{2} \, U = 0
\end{align*}
Puesto que el dominio de $x$ es un intervalo finito, es preferible escribir la solución a la EDO como:
\begin{align}
U(x, \alpha) = c_{1} \, \cosh (\alpha \, x) + c_{2} \, \senh (\alpha \, x)
\label{eq:ecuacion_016}
\end{align}
Ahora bien:
\begin{align*}
F_{c} \big[ u(0, y)\big] &= F_{c} [0] \\[0.5em]
F_{c} \big[ u(\pi, y)\big] &= F_{c} \big[ e^{-y} \big]
\end{align*}
son equivalentes a
\begin{align*}
U(0, \alpha) &= 0 \\[0.5em]
U(\pi, \alpha) &= \dfrac{1}{1 +  \alpha^{2}}
\end{align*}
respectivamente.

Cuando se aplican estas últimas condiciones, la solución(\ref{eq:ecuacion_016}) nos devuelve:
\begin{align*}
c_{1} &= 0 \\[0.5em]
c_{2} &= \dfrac{1}{(1 + \alpha^{2}) \, \sinh \alpha \pi}
\end{align*}
Por lo tanto
\begin{align*}
U(x, \alpha) = \dfrac{\senh \alpha \, x}{1 + \alpha^{2}) \, \sinh \alpha \pi}
\end{align*}
de modo que al ocupar la transformada coseno inversa de Fourier:
\begin{align*}
F_{c}^{-1} \big[F(\alpha)\big] = \dfrac{2}{\pi} \int_{0}^{\infty} F[\alpha] \, \cos \alpha \, x \dd{x}
\end{align*}
obtenemos el siguiente resultado:
\begin{align*}
u(x, y) = \dfrac{2}{\pi} \int_{0}^{\infty}  \dfrac{\senh \alpha \, x}{1 + \alpha^{2}) \, \sinh \alpha \pi} \, \cos \alpha \, x \dd{x}
\end{align*}



\end{document}