\documentclass[hidelinks,12pt]{article}
\usepackage[left=0.25cm,top=1cm,right=0.25cm,bottom=1cm]{geometry}
%\usepackage[landscape]{geometry}
\textwidth = 20cm
\hoffset = -1cm
\usepackage[utf8]{inputenc}
\usepackage[spanish,es-tabla]{babel}
\usepackage[autostyle,spanish=mexican]{csquotes}
\usepackage[tbtags]{amsmath}
\usepackage{nccmath}
\usepackage{amsthm}
\usepackage{amssymb}
\usepackage{mathrsfs}
\usepackage{graphicx}
\usepackage{subfig}
\usepackage{standalone}
\usepackage[outdir=./Imagenes/]{epstopdf}
\usepackage{siunitx}
\usepackage{physics}
\usepackage{color}
\usepackage{float}
\usepackage{hyperref}
\usepackage{multicol}
%\usepackage{milista}
\usepackage{anyfontsize}
\usepackage{anysize}
%\usepackage{enumerate}
\usepackage[shortlabels]{enumitem}
\usepackage{capt-of}
\usepackage{bm}
\usepackage{relsize}
\usepackage{placeins}
\usepackage{empheq}
\usepackage{cancel}
\usepackage{wrapfig}
\usepackage[flushleft]{threeparttable}
\usepackage{makecell}
\usepackage{fancyhdr}
\usepackage{tikz}
\usepackage{bigints}
\usepackage{scalerel}
\usepackage{pgfplots}
\usepackage{pdflscape}
\pgfplotsset{compat=1.16}
\spanishdecimal{.}
\renewcommand{\baselinestretch}{1.5} 
\renewcommand\labelenumii{\theenumi.{\arabic{enumii}})}
\newcommand{\ptilde}[1]{\ensuremath{{#1}^{\prime}}}
\newcommand{\stilde}[1]{\ensuremath{{#1}^{\prime \prime}}}
\newcommand{\ttilde}[1]{\ensuremath{{#1}^{\prime \prime \prime}}}
\newcommand{\ntilde}[2]{\ensuremath{{#1}^{(#2)}}}

\newtheorem{defi}{{\it Definición}}[section]
\newtheorem{teo}{{\it Teorema}}[section]
\newtheorem{ejemplo}{{\it Ejemplo}}[section]
\newtheorem{propiedad}{{\it Propiedad}}[section]
\newtheorem{lema}{{\it Lema}}[section]
\newtheorem{cor}{Corolario}
\newtheorem{ejer}{Ejercicio}[section]

\newlist{milista}{enumerate}{2}
\setlist[milista,1]{label=\arabic*)}
\setlist[milista,2]{label=\arabic{milistai}.\arabic*)}
\newlength{\depthofsumsign}
\setlength{\depthofsumsign}{\depthof{$\sum$}}
\newcommand{\nsum}[1][1.4]{% only for \displaystyle
    \mathop{%
        \raisebox
            {-#1\depthofsumsign+1\depthofsumsign}
            {\scalebox
                {#1}
                {$\displaystyle\sum$}%
            }
    }
}
\def\scaleint#1{\vcenter{\hbox{\scaleto[3ex]{\displaystyle\int}{#1}}}}
\def\bs{\mkern-12mu}


%\usepackage{showframe}
\title{Tarea - Examen \\ \large {Tema 6 - Transformadas integrales} \vspace{-3ex}}
\author{M. en C. Gustavo Contreras Mayén}
\date{ }
\begin{document}
\vspace{-4cm}
\maketitle
\fontsize{14}{14}\selectfont
\section{Transformadas de Fourier.}
\begin{enumerate}
\item Calcula las transformadas seno y coseno de Fourier de la función definida por:
\begin{align*}
f(x) = \begin{cases}
\sin x, & 0 < x < a \\
0, & x > a
\end{cases}
\end{align*}
\item Calcula la transformada de Fourier de la función:
\begin{align*}
f(x) = \begin{cases}
e^{-a x}, & x > 0, a > 0 \\
-e^{a x}, & x < 0, a > 0
\end{cases}
\end{align*}
% \item Utiliza la transformada de Fourier para determinar el desplazamiento $y (x, t)$ de una cuerda vibrante infinita, dado que la cuerda está inicialmente en reposo y que el desplazamiento inicial es $f (x)$, $-\inf< x < \infty$.
% \par
% La ecuación de onda unidimensional de una cuerda vibrante está dada por:
% \begin{align*}
% \pdv[2]{y}{t} + c^{2} \, \pdv{y}{x} = 0 \hspace{1cm} -\infty < x < \infty, \hspace{0.3cm} t > 0
% \end{align*}
% Demuestra que la solución requerida también se puede poner en el forma:
% \begin{align}
% y(x, t) = \dfrac{1}{2} \, \big[ f(x + c \, t) + f(x - c \, t)\big]
% \end{align}
\item La temperatura $u (x, t)$ de una varilla semiinfinita
$0 \leq x < \infty$ satisface el ecuación diferencial parcial
\begin{align*}
\pdv{u}{t} = \kappa \, \pdv[2]{u}{x}
\end{align*}
sujeta a las condiciones
\begin{align*}
u(x, 0) = 0, \hspace{1cm} \pdv{u}{x} = \lambda \hspace{0.3cm} \mbox{una constante, cuando } \hspace{0.2cm} x = 0, \hspace{0.2cm} t > 0
\end{align*}
Determina la temperatura $u(x, t)$ en la varilla.
\end{enumerate}

\section{Transformada de Laplace.}

\begin{enumerate}
\item Evalúa $L \big[\sin a \, t/ t \big]$. ¿Existe la $L \big[\cos a \, t/ t \big]$
\item La función \textbf{Seno integral} se define como:
\begin{align*}
Si(t) = \int_{0}^{t} \dfrac{\sin u }{u} \dd{u}
\end{align*}
mientras que la función \textbf{Coseno integral} queda definida por:
\begin{align*}
Ci(t) = - \int_{t}^{\infty} \dfrac{\cos u }{u} \dd{u}
\end{align*}
Calcula la transformada de Laplace de $Si(t)$ y de $Ci(t)$
\item Una función periódica $f(t)$ de período $2 \, \pi$ presenta una discontinuidad finita en $t = \pi$, está dada por:
\begin{align*}
f(t) = \begin{cases}
\sin t, & 0 \leq t < \pi \\
\cos t, & \pi < t \leq 2 \, \pi
\end{cases}
\end{align*} 
Evalúa la transformada de Laplace.

\end{enumerate}
\end{document}