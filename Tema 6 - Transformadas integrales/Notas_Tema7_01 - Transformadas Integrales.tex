\RequirePackage[l2tabu, orthodox]{nag}
\documentclass[12pt]{article}
\usepackage[utf8]{inputenc}
\usepackage[spanish,es-lcroman, es-tabla]{babel}
\usepackage[autostyle,spanish=mexican]{csquotes}
\usepackage{amsmath}
\usepackage{amssymb}
\usepackage{nccmath}
\numberwithin{equation}{section}
\usepackage{amsthm}
\usepackage{graphicx}
\usepackage{epstopdf}
\DeclareGraphicsExtensions{.pdf,.png,.jpg,.eps}
\usepackage{color}
\usepackage{float}
\usepackage{multicol}
\usepackage{enumerate}
\usepackage[shortlabels]{enumitem}
\usepackage{anyfontsize}
\usepackage{anysize}
\usepackage{array}
\usepackage{multirow}
\usepackage{enumitem}
\usepackage{cancel}
\usepackage{tikz}
\usepackage{circuitikz}
\usepackage{tikz-3dplot}
\usetikzlibrary{babel}
\usetikzlibrary{shapes}
\usepackage{bm}
\usepackage{mathtools}
\usepackage{esvect}
\usepackage{hyperref}
\usepackage{relsize}
\usepackage{siunitx}
\usepackage{physics}
%\usepackage{biblatex}
\usepackage{standalone}
\usepackage{mathrsfs}
\usepackage{bigints}
\usepackage{bookmark}
\spanishdecimal{.}

\setlist[enumerate]{itemsep=0mm}

\renewcommand{\baselinestretch}{1.5}

\let\oldbibliography\thebibliography

\renewcommand{\thebibliography}[1]{\oldbibliography{#1}

\setlength{\itemsep}{0pt}}
%\marginsize{1.5cm}{1.5cm}{2cm}{2cm}


\newtheorem{defi}{{\it Definición}}[section]
\newtheorem{teo}{{\it Teorema}}[section]
\newtheorem{ejemplo}{{\it Ejemplo}}[section]
\newtheorem{propiedad}{{\it Propiedad}}[section]
\newtheorem{lema}{{\it Lema}}[section]

%\usepackage{standalone}
%\usepackage{mathrsfs}
%\usepackage{bigints}
%\newtheorem{defi}{{\textit{Definición}}}[section]
%\newtheorem{teo}{{\textit{Teorema}}}[section]
\newcommand{\saltosin}{\nonumber \\}
%\newcommand{\comillado}[1]{``#1''}
%\spanishdecimal{.}
%\usepackage{enumerate}
%\author{M. en C. Gustavo Contreras Mayén. \texttt{curso.fisica.comp@gmail.com}}
\title{Transformadas Integrales \\ {\large Matemáticas Avanzadas de la Física}}
\date{ }
\begin{document}
%\renewcommand\theenumii{\arabic{theenumii.enumii}}
%\renewcommand\labelenumii{\theenumi.{\arab{enumii}}}
\maketitle
\fontsize{14}{14}\selectfont
%Referencia Arfken - Mathematical methods for physicists. Cap. 15 Integral Transforms -djvu
\section{Transformadas Integrales.}
Frecuentemente en física uno se encuentra con pares de funciones que se encuentran relacionadas por una expresión como la siguiente
\begin{equation}
g (\alpha) = \int_{a}^{b} f(t) \, K(\alpha, t) \, \dd t
\label{eq:ecuacion_15_01}
\end{equation}
La función $g (\alpha)$ es llamada la transformada integral de $f(t)$ por el núcleo (o kernel)  $K (\alpha,t)$. La operación se puede describir como un mapeo de una función $f(t)$ en el espacio-$t$ a otra función $g (\alpha)$ en el espacio-$\alpha$. Dos ejemplos de esta interpretación en física son las relaciones entre: el tiempo y la frecuencia en Electrodinámica Clásica y Mecánica Cuántica, y la relación entre el espacio de configuraciones y el espacio de momentos en Mecánica Cuántica.
\par
Cuando los límites de integración $a$ y $b$ son finitos, decimos que $g(\alpha)$ es es la transformada finita de $f(t)$. Existen varios tipos de transformadas integrales que aparecen frecuentemente en física, cada una de ellas está asociada a un núcleo diferente. De entre ellas las diferentes posibilidades podemos mencionar los núcleos siguientes:
\begin{enumerate}
\item \emph{Transformada de Fourier.}
\begin{equation}
g (\omega) = \dfrac{1}{\sqrt{2 \, \pi}} \int_{-\infty}^{\infty} f(t) \, e^{i \omega t} \, \dd t
\label{eq:ecuacion_07_01}
\end{equation}
\item \emph{Transformada de Laplace.}
\begin{equation}
g (\alpha)= \int_{0}^{\infty} f(t) \; \exp(-\alpha t) \, \dd t
\label{eq:ecuacion_7_02}
\end{equation}
\item \emph{Transformadas de Fourier seno y coseno.}
\begin{equation}
g (\alpha)= \int_{0}^{\infty} f(t) \; \substack{ \textstyle \sin \\[0.5em] \textstyle \cos} \; \alpha \, t \,  \dd t
\label{eq:ecuacion_7_03}
\end{equation}
\item \emph{Transformada de Fourier compleja.}
\begin{equation}
g (\alpha)= \int_{-\infty}^{\infty} f(t) \; \exp(i \alpha t) \, \dd t
\label{eq:ecuacion_7_04}
\end{equation}
\item \emph{Transformada de Hankel.}
\begin{equation}
g (\alpha)= \int_{0}^{\infty} f(t) \; t \; J_{n} (\alpha \, t) \, \dd t
\label{eq:ecuacion_7_05}
\end{equation}
donde $J_{n}(\alpha t)$ es la función de Bessel de primera clase de orden $n$.
\item \emph{Transformada de Mellin.}
\begin{equation}
g (\alpha)= \int_{0}^{\infty} f(t) \; t^{\alpha-1} \, \dd t
\label{eq:ecuacion_7_06}
\end{equation}
\end{enumerate}
Como se verá más adelante, aplicar una transformada integral a una EDP es para excluir temporalmente una variable independiente que se ha elegido y dejar de solución de una EDP en una variable menos. La solución de esta ecuación será una función de $\alpha$ y las variables restantes. Cuando se ha obtenido esta solución, tiene que ser \enquote{invertida} para recuperar la variable \enquote{perdida}: Así, si $t$ es la variable eliminada y $g (\alpha)$ es una de las transformaciones dadas anteriormente, obtenemos primero las ecuaciones auxiliares que dan $g$ en términos de $\alpha$ y las variables independientes restantes, resolvemos para $g$ y luego inviertimos para obtener $g(\alpha)$.
\par
El proceso de inversión significa, en efecto, la solución de una de las ecuaciones integrales (\ref{eq:ecuacion_7_02}) $\ldots$ (\ref{eq:ecuacion_7_06}), $g (\alpha)$ que se supone conocido y $f(t)$ que se encuentran, como se puede ver en la figura (\ref{fig:figura_01}). Tales soluciones son conocidas y pueden ser obtenidos formalmente del teorema de la integral de Fourier.
\begin{figure}[H]
\centering
\includestandalone{Figuras/esquema_transformadas}
\caption{Esquema de las transformadas integrales.}
\label{fig:figura_01}
\end{figure}
\subsection{Linealidad de las transformadas integrales.}
Todas estas transformadas integrales son lineales, por lo que satisfacen las siguientes propiedades:
\begin{align}
\begin{aligned}
\int_{a}^{b} [ c_{1} \, f_{1} (t) + c_{2} \, f_{2}(t)] \; K(\alpha, t) \, \dd t &= c_{1} \, \int_{a}^{b} f_{1} (t) \; K(\alpha, t) \, \dd t + \\
&+ c_{2} \, \int_{a}^{b} f_{2} (t) \; K(\alpha, t) \, \dd t
\end{aligned}
\label{eq:ecuacion_15_08} 
\end{align}
y además
\begin{equation}
\int_{a}^{b}  c \; f (t) \; K(\alpha, t) \, \dd t =  c \; \int_{a}^{b} f (t) \; K(\alpha, t) \, \dd t
\label{eq:ecuacion_15_09}
\end{equation}
donde $c_{1}$ y $c_{2}$ son constantes y $f_{1}(t)$ y $f_{2}(t)$ son funciones para las cuales la operación transformada está definda.
\par
Representando la transformada integral lineal por el operador $\mathcal{L}$, obtenemos
\begin{equation}
g (\alpha) = \mathcal{L} \, f(t)
\label{eq:ecuacion_15_10}
\end{equation}
Uno espera que exista el operador inverso $\mathcal{L}^{-1}$, de manera tal que
\begin{equation}
f(t) = \mathcal{L}^{-1}  \; g (\alpha)
\label{eq:ecuacion_15_11}
\end{equation}
En general, el mayor problema con el uso de las transformadas integrales, es determinar el operador inverso. Sin embargo, para los dos tipos de transformaciones: de Fourier y de Laplace, obtener el inverso es relativamente sencillo.
\section{La transformada de Fourier.}
Podemos representar una función en series de Fourier, siempre y cuando la función cumpla con lo siguiente:
\begin{enumerate}[label=\alph*)]
\item La función está acotada en el intervalo $[0, 2 \, \pi]$ o $[-L, L]$ (intervalo finito).
\item La función está definida en el intervalo $(- \infty, \infty)$, sólo \textbf{si la función es periódica}.
\end{enumerate}
Nos podemos plantear la siguiente pregunta: ¿Qué pasa si tenemos una función no periódica en el intervalo infinito $(-\infty, \infty)$?
\par
Respuesta: \emph{Existe una representación integral}.
\par
Sabemos que si $f(x)$ es una función continua a pedazos con discontinuidades finitas en el intervalo $[-L, L]$, podemos hacer una expansión en series de Fourier:
\begin{equation}
f(x) = \dfrac{a_{0}}{2} + \sum_{n=1}^{\infty} \left[ a_{n} \cos \left( \dfrac{n \, \pi \, x}{L} \right) + b_{n} \sin \left( \dfrac{n \, \pi \, x}{L} \right) \right]
\label{eq:8_11}
\end{equation}
donde los coeficientes $a_{n}$ y $b_{n}$ son:
\begin{align}
a_{n} &= \dfrac{1}{L} \int_{-L}^{L} f(t) \, \cos \left( \dfrac{n \, \pi \, t}{L} \right) \, \dd t \label{eq:15_12} \\[0.5em]
b_{n} &= \dfrac{1}{L} \int_{-L}^{L} f(t) \, \sin \left( \dfrac{n \pi \, t}{L} \right) \, \dd t
\label{eq:ecuacion_15_13}
\end{align}
Con lo cual podemos re-escribir a $f(x)$ como una serie de Fourier
\begin{align}
\begin{aligned}
f(x) &= \dfrac{1}{2 \, L} \int_{-L}^{L} f(t) \, \dd t + \dfrac{1}{L} \sum_{k=1}^{\infty} \cos \left( \dfrac{n \, \pi \, x}{L} \right) \, \int_{-L}^{L} f(t) \, \cos \left( \dfrac{n \, \pi \, t}{L} \right) \, \dd t + \\
&+ \dfrac{1}{L} \sum_{n=1}^{\infty}  \sin \left( \dfrac{n \, \pi \, x}{L} \right) \int_{-L}^{L} f(t) \, \sin \left( \dfrac{n \, \pi \, t}{L} \right) \, \dd t
\label{eq:ecuacion_15_14}
\end{aligned}
\end{align}
o de la forma
\begin{equation}
f(x) = \dfrac{1}{2 \, L} \int_{-L}^{L} f(t) \, \dd t + \dfrac{1}{L} \sum_{n=1}^{\infty} \, \int_{-L}^{L} f(t) \, \cos \left( \dfrac{n \, \pi}{L} \right) (t - x) \, \dd t
\label{eq:ecuacion_15_15}
\end{equation}
Hagamos que el parámetro $L$ tienda a infinito, cambiando el intervalo finito $[-L, L]$ al intervalo infinito $(-\infty, \infty)$. Establecemos las siguientes relaciones:
\[ \dfrac{n \, \pi}{L} = \omega, \hspace{1cm} \dfrac{\pi}{L} = \Delta \omega, \hspace{1cm} \mbox{con } L \to \infty \]
Veamos la validez de esta afirmación. Consideremos la partición de los $\mathbb{R}^{+}$ dada por
\begin{equation}
\omega_{0} = 0 < \omega_{1} = \dfrac{\pi}{L} < \omega_{2} = \dfrac{2 \, \pi}{L} < \ldots < \omega_{n} = \dfrac{n \, \pi}{L}
\label{eq:ecuacion_08_16}
\end{equation}
Entonces tenemos que
\begin{equation}
f(x) \rightarrow \dfrac{1}{\pi} \sum_{n=1}^{\infty} \Delta \omega \; \int_{-\infty}^{\infty} f(t) \, \cos \omega (t - x) \, \dd t
\label{eq:ecuacion_15_16}
\end{equation}
re-emplazando la suma infinita por la intergral sobre $\omega$:
\begin{equation}
f(x) = \dfrac{1}{\pi} \int_{0}^{\infty} \, \dd \omega \; \int_{-\infty}^{\infty} f(t) \, \cos \omega (t - x) \, \dd t
\label{eq:ecuacion_15_17}
\end{equation}
 El primer término (que corresponde a $a_{0}$) se anula, y suponemos que la integral $\displaystyle{\int_{-\infty}^{\infty} f(t) \, \dd t}$ existe.
\par
Debemos de enfatizar que el resultado obtenido en la ec. (\ref{eq:ecuacion_15_17}) sólo es formal, no pretende ser un resultado riguroso; se tomará como la \emph{integral de Fourier}, considerando que la función $f(x)$ cumple con las siguientes condiciones:
\begin{enumerate}
\item Es continua (continua por partes)
\item Es diferenciable (diferenciable por partes)
\item La integral $\displaystyle{\int_{-\infty}^{\infty} \abs{f(x)} \, \dd x}$ es finita. 
\end{enumerate}
\subsection{Forma exponencial de la integral de Fourier.}
La integral de Fourier (ec. \ref{eq:ecuacion_15_17}) se puede escribir de forma exponencial, para obtener esta representación, revisemos que
\begin{equation}
f(x) =  \dfrac{1}{2 \, \pi} \int_{- \infty}^{\infty} \,\dd \omega \; \int_{-\infty}^{\infty} f(t) \; \cos \omega (t - x) \, \dd t
\label{eq:ecuacion_15_18}
\end{equation}
mientras que
\begin{equation}
\dfrac{1}{2 \, \pi} \int_{-\infty}^{\infty} \, \dd \omega \; \int_{-\infty}^{\infty} f(t) \; \sin \omega (t - x) \, \dd t = 0
\label{eq:ecuacion_15_19}
\end{equation}
veamos que $ \cos \omega (t - x)$ es una función par de $\omega$ y $\sin \omega (t - x)$ es una función impar de $\omega$. Sumando las ecuaciones (\ref{eq:ecuacion_15_18}) y (\ref{eq:ecuacion_15_19}) (con un factor $i$), obtenemos el \emph{teorema integral de Fourier}
\begin{equation}
\setlength{\fboxsep}{3\fboxsep}\boxed{
f(x) = \dfrac{1}{2 \, \pi} \int_{-\infty}^{\infty} e^{-i \omega x} \, \dd \omega \; \int_{-\infty}^{\infty} f(t) \, e^{i \omega t} \, \dd t }
\label{eq:ecuacion_15_20}
\end{equation}
Hasta ahora $\omega$ es una variable matemática auxiliar. En muchos problemas de la física, $\omega$ es una frecuencia angular. En este caso podemos interpretar la integral de Fourier como una representación de $f(x)$ en términos de una distribución de ondas sinusoidales infinitamente largas de frecuencia angular $\omega$, en la cual esta frecuencia es una \textbf{variable continua}.
\subsection{Obtención mediante la función delta de Dirac.}
Nótese que si cambiamos el orden de integración de la ec. (\ref{eq:ecuacion_15_20}), la función $f(x)$ se escribe de la forma
\begin{align}
\setlength{\fboxsep}{3\fboxsep}\boxed{f(x)= \int_{-\infty}^{\infty} f(t) \underbrace{ \left[ \dfrac{1}{2 \, \pi} \int_{-\infty}^{\infty} e^{i \omega(t - x)} \, \dd \omega \right] }_{\mbox{debe de ser una delta de Dirac}} \, \dd t} 
\label{eq:ecuacion_15_20a}
\end{align}
En apariencia, la cantidad entre corchetes parece ser una función delta de Dirac $\delta (t - x)$. Podríamos tomar la ec. (\ref{eq:ecuacion_15_20a}) como una representación de la delta de Dirac, lo que nos daría una pista para obtener el teorema integral de Fourier.
\par
De una de las propiedades de la función delta de Dirac
\[ f(0) = \int_{-\infty}^{\infty} f(x) \, \delta(x) \, \dd x \]
cambiando la singularidad de $t = 0$ a $t = x$
\begin{equation}
f(x) = \lim_{n \to \infty} \int_{-\infty}^{\infty} f(t) \, \delta_{n} (t - x) \, \dd t
\label{eq:ecuacion_15_21a} 
\end{equation}
donde $\delta_{n}(t - x)$ es una secuencia que define la distribución $\delta(t - x)$. Revisemos que la ec. (\ref{eq:ecuacion_15_21a}) supone que $f(t)$ es continua en $t = x$. Tomamos $\delta_{n} (t - x)$  de la expresión
\[ \delta_{n} = \dfrac{\sin n \, x}{\pi \, x} \int_{-n}^{n} e^{i x t} \, \dd t \]
para hacer
\begin{equation}
\delta_{n} (t - x) = \dfrac{\sin n (t - x)}{\pi (t - x)} = \dfrac{1}{2 \, \pi} \int_{-n}^{n} e^{i \omega (t - x)} \, \dd \omega
\label{eq:ecuacion_15_21b}
\end{equation}
al sustituir en la ecuación (\ref{eq:ecuacion_15_21a}), tenemos
\begin{equation}
f(x) = \lim_{n \to \infty} \dfrac{1}{2 \, \pi} \int_{-\infty}^{\infty} f(t) \, \int_{-n}^{n} e^{i \omega (t - x)} \, \dd \omega \, \dd t
\label{eq:ecuacion_15_21c}
\end{equation}
Intercambiando el orden de integración y tomando el límite cuando $n \to \infty$, obtenemos la ec. (\ref{eq:ecuacion_15_20}), el teorema integral de Fourier.
\par
Con el entendimiento de que pertenece bajo el signo integral, como en la ec. (\ref{eq:ecuacion_15_21a}), la definición
\begin{equation}
\setlength{\fboxsep}{3\fboxsep}\boxed{\delta(t - x) = \dfrac{1}{2 \, \pi} \int_{-\infty}^{\infty} e^{i \omega (t - x)} \, dd \omega}
\label{eq:ecuacion_15_21d}
\end{equation}
nos proporciona una representación bastante útil de la función delta de Dirac.
\subsection{Transformada de Fourier. Teorema de inversión.}
\begin{defi}{Transformada de Fourier.}

Denotamos la transformada de Fourier de la función $f(t)$ mediante $g(\omega)$ y se define por
\begin{equation}
g (\omega) \equiv \dfrac{1}{\sqrt{2 \, \pi}} \int_{-\infty}^{\infty} f(t) \; e^{i \omega t} \, \dd t
\label{eq:ecuacion_15_22}
\end{equation}
\end{defi}
\subsection*{Transformada exponencial.}
Pero se ha demostrado que la ec. (\ref{eq:ecuacion_15_20}) tiene una relación inversa 
% \begin{equation}
% f(x) = \dfrac{1}{\pi} \int_{-\infty}^{\infty} e^{-i \omega x} dt \int_{-\infty}^{\infty} f(t) \; e^{i \omega t} dt
% \label{eq:8_28}
% \end{equation}
% Este resultado nos permite establecer el siguiente teorema:
% \begin{teo}{Teorema de inversión.}
\begin{equation}
f(t) = \dfrac{1}{\sqrt{2 \, \pi}} \int_{-\infty}^{\infty} g(\omega) \; e^{-i \omega t} \, \dd \omega
\label{eq:ecuacion_15_23}
\end{equation}
%\end{teo}
Nótese que las ecs. (\ref{eq:ecuacion_15_22}) y (\ref{eq:ecuacion_15_23}), no son del todo simétricas, con excepción del signo de $i$.
\par
Adicionalmente consideremos dos puntos: 
\begin{enumerate}
\item  El factor $1/\sqrt{2 \, \pi}$ es opcional, pero no necesario, algunos autores prefieren usar el factor completo $1/2 \, \pi$ en alguna de las ecuaciones (\ref{eq:ecuacion_15_22}) o (\ref{eq:ecuacion_15_23}).
\item La integral de Fourier (ec. \ref{eq:ecuacion_15_20}), tiene mucha atención en la literatura matemática, enfocándose de manera particular en la Transformada de Fourier y su inversa, son ecuaciones que tienen una relevancia en la física.
\end{enumerate}
Cuando nos desplazamos al espacio tridimensional con el par de transformadas, obtenemos
\begin{subequations}
\begin{align}
g (\vb{k}) &= \dfrac{1}{(2 \, \pi)^{3/2}} \int f (\vb{r}) \; e^{i \vb{k \cdot r}} \; \dd[3] r \label{eq:ecuacion_15_23a} \\
f (\vb{r}) &= \dfrac{1}{(2 \, \pi)^{3/2}} \int g (\vb{k}) \; e^{-i \vb{k \cdot r}} \; \dd[3] k \label{eq:ecuacion_15_23b}
\end{align}
\end{subequations}
Las integrales se evalúan sobre todo el espacio. Para verificar esto, se sigue inmediatamente mediante la sustitución de la parte izquierda de una ecuación en el integrando de la otra ecuación y usando la función delta tridimensional. La ecuación (\ref{eq:ecuacion_15_23b}) puede ser interpretada como una expansión de una función $f(\vb{r})$ en un continuo de las funciones propias de onda plana, $g(\vb{k})$ se convierte en la amplitud de la onda, $exp(-i \vb{k \cdot r})$
\subsection{Transformada coseno de Fourier.}
Si $f(x)$ es par o impar, las transformadas pueden expresarse de una forma diferente. Consideremos inicialmente una función par $f_{c}$ con $f_{c}(x) = f_{c}(-x)$. Escribiendo la exponencial de la ec. (\ref{eq:ecuacion_15_22}) en su forma trigonométrica, se tiene que
\begin{align}
\begin{aligned}
g_{c}(\omega) &= \dfrac{1}{\sqrt{2 \, \pi}} \int_{-\infty}^{\infty} f_{c} (t) \; (\cos \omega t + i \sin \omega t) \, \dd t \\
&= \sqrt{\dfrac{2}{\pi}} \int_{0}^{\infty} f_{c} (t) \; \cos \omega t \, \dd t 
\label{eq:ecuacion_15_24}
\end{aligned}
\end{align}
la dependencia de $\sin \omega t$ se anula en la integración del intervalo simétrico $(-\infty, \infty)$. De manera similar, como $\cos \omega t$ es par, las ecuaciones (\ref{eq:ecuacion_15_23}) se convierten en
\begin{equation}
f_{c} (x) = \sqrt{\dfrac{2}{\pi}} \int_{0}^{\infty} g_{c} (\omega) \cos \omega x \, \dd \omega 
\label{eq:ecuacion_15_25}
\end{equation}
Las ecuaciones (\ref{eq:ecuacion_15_24}) y (\ref{eq:ecuacion_15_25}) se conocen como las \emph{transformadas coseno de Fourier}.
\subsection{Tranformada seno de Fourier.}
Las correspondientes pares de la transformada seno de Fourier, se obtienen suponiendo que $f_{s}(x) = - f_{s}(-x)$ es impar, aplicando los mismos argumentos de simetría, las ecuaciones son
\begin{align}
g_{s} (\omega) &=  \sqrt{\dfrac{2}{\pi}} \int_{0}^{\infty} f_{s} (t) \; \sin \omega t \, dd t \label{eq:ecuacion_15_26} \\
f_{s} (x) &= \sqrt{\dfrac{2}{\pi}} \int_{0}^{\infty} g_{s} (\omega) \; \sin \omega x \, \dd \omega  \label{eq:ecuacion_15_27}
\end{align}
De la última ecuación podemos dar la siguiente interpretación física: $f(x)$ está describiendo una serie continua de ondas sinusoidales. La amplitud de $\sin \omega x$ está dada por $\sqrt{2 / \pi} \, g_{s} (\omega)$ en donde $g_{s}(x)$ es la transformada seno de Fourier de $f(x)$.
\par
Toma en cuenta que la transformada coseno de Fourier y la transformada seno de Fourier implican sólo valores positivos (y cero) en los argumentos. Utilizamos la paridad de $f (x)$ para establecer las transformadas; pero una vez que se establecen, el comportamiento de las funciones $f$ y $g$ con un argumento negativo es irrelevante. En efecto, las propias ecuaciones de transformación imponen una paridad definida: \textbf{par para la transformada coseno de Fourier} e \textbf{impar para la transformada seno de Fourier}.
\subsection*{Ejemplo. Tren finito de ondas.}
Una aplicación importante de la transformada de Fourier es la solución de un pulso finito de ondas senoidales.
\begin{equation}
f(t) = \begin{cases}
\sin \omega_{0} t, & \abs{t} < \dfrac{N \, \pi}{\omega_{0}} \\[0.5em]
0, & \abs{t} > \dfrac{N \, \pi}{\omega_{0}}
\end{cases}
\label{eq_ecuacion_15_28}
\end{equation}
Que corresponde a $N$ ciclos del tren de ondas inicial, como se ve en la figura (\ref{fig:tren_finito}).
\begin{figure}[H]
\centering
\includestandalone{Figuras/Figura_01_Fourier}
\caption{Tren finito de ondas.}
\label{fig:tren_finito}
\end{figure}
Como $f(t)$ es impar, podemos usar la transformada seno de Fourier (ec. \ref{eq:ecuacion_15_26}), para obtener
\begin{equation}
g_{s} (\omega) = \sqrt{\dfrac{2}{\pi}} \int_{0}^{N \pi / \omega_{0}} \sin \omega_{0} t \; \sin \omega t \, \dd t
\label{eq:ecuacion_15_29}
\end{equation}
Integrando, encontramos la amplitud de la función
\begin{equation}
g_{s} (\omega) = \sqrt{\dfrac{2}{\pi}} \left[ \dfrac{\sin[(\omega_{0} - \omega)(N  \, \pi / \omega_{0})]}{2 \, (\omega_{0} - \omega)} - \dfrac{\sin[(\omega_{0} + \omega)(N \, \pi / \omega_{0})]}{2 \, (\omega_{0} + \omega)} \right]
\label{eq:ecuacion_15_30}
\end{equation}
Es de considerable interés ver cómo $g_{s} (\omega)$ depende de la frecuencia. Para valores grandes de $\omega_{0}$ y $\omega \simeq \omega_{0}$, sólo el primer término será de importancia debido a los denominadores, esto se puede ver en la Fig. (\ref{fig:TF_tren_finito}). Esta es la curva de la amplitud para un patrón de difracción de una sola rendija.
\begin{figure}[H]
    \centering
    \includestandalone[scale=0.7]{Figuras/Figura_02_Fourier}
    \caption{Transformada de Fourier de un tren finito de ondas.}
    \label{fig:TF_tren_finito}
\end{figure}
Los ceros se encuentran en
\begin{equation}
\dfrac{ \omega_{0} - \omega}{\omega} = \dfrac{\Delta \omega}{\omega_{0}} = \pm \dfrac{1}{N}, \pm \dfrac{2}{N}, \hspace{1cm} \text{y así}
\label{eq:ecuacion_15_31}
\end{equation}
Para valores grandes de $N$, $g_{s}(\omega)$ puede interpretarse como una distribución de Dirac. Considerando que las contribuciones fuera del máximo central son pequeñas para este caso, podemos tomar
\begin{equation}
\Delta \omega = \dfrac{\omega_{0}}{N}
\label{eq:ecuacion_15_32}
\end{equation}
como una buena medida de la distribución de la frecuencia de nuestro pulso de onda. Directamente se nota que, si $N$ es grande (un pulso largo) la distribución de frecuencias será pequeña. Por otra parte, si nuestro pulso es corto, $N$ pequeño, la distribución de frecuencias será más ancha y los máximos secundarios son más importantes.
\subsection*{Ejemplo. La función paso.}
Considera para un valor fijo de $a$ la función \emph{paso} o función \emph{pulso rectangular} $p_{a}(t)$ de altura $1$ y duración $a$, definida por
\begin{equation}
p_{a}(t) = \begin{cases}
1 & \text{ para } \abs{t} \leq \dfrac{a}{2} \\
0 & \text{ para cualquier otro valor} \end{cases}
\label{eq:ecuacion_06_10_Beerends}
\end{equation}
Se puede ver de la figura (\ref{fig:figura_funcionpaso}) que $p_{a}(t)$ es integrable.
\begin{figure}[H]
\centering
\includestandalone{Figuras/funcion_paso}
\caption{Función paso rectangular.}
\label{fig:figura_funcionpaso}
\end{figure}
Para $\omega \neq 0$ se tiene
\begin{align*}
(g \, p_{a}) (\omega) &= \int_{-\infty}^{\infty} p_{a}(t) \; e^{-i \omega t} \, \dd t = \int_{-a/2}^{a/2} e^{-i \omega t} \, \dd t = \left[ \dfrac{- e^{- i \omega t}}{i \, \omega} \right]_{-a/2}^{a/2} \\[0.5em]
&= \dfrac{e^{ia\omega/2} - e^{-i a \omega/2}}{i \, \omega} = \dfrac{2 \, \sin (a \omega/2)}{\omega}
\end{align*}
mientras que para $\omega = 0$, se tiene
\begin{equation*}
(g \, p_{a})(0) = \int_{-\infty}^{\infty} p_{a}(t) \, \dd t = \int_{-a/2}^{a/2} \, \dd t = a
\end{equation*}
Es bien sabido que el límite 
\[ \lim_{x \to 0} \dfrac{\sin x}{x} = 1 \]
entonces obtenemos 
\[ \lim_{\omega \to 0}  (g \, p_{a}) \, (\omega) = \lim_{\omega \to 0} \dfrac{2 \, \sin (a \, \omega /2)}{\omega} = a \]
A pesar de que $p_{a}(t)$ en sí, no es continua, vemos que
\begin{equation}
(g \, p_{a})(\omega) = \dfrac{2 \sin (a \omega/2)}{\omega}
\label{eq:ecuacion_06_11_Beerends}
\end{equation}
es continua en $\mathbb{R}$.
\begin{figure}[H]
\centering
\includestandalone{Figuras/T_funcionpaso}
\caption{Transformada de Fourier de la función paso rectangular.}
\label{fig:figura_Tfuncionpaso}
\end{figure}
\section{La Transformada de Fourier de la derivada.}
Para apoyarnos en la solución de una ED utilizando la transformada de Fourier y su inversa, necesitamos revisar cómo es la TF de una derivada. 
\par
Usando la forma exponencial de la TF de $f(x)$:
\begin{equation}
g(\omega) = \dfrac{1}{\sqrt{2 \pi}} \int_{-\infty}^{\infty} f(x) \; e^{i \omega x} \, \dd x
\label{eq:ecuacion_15_37}
\end{equation}
y para $\displaystyle \dv{f(x)}{x}$
\begin{equation}
g_{1}(\omega) = \dfrac{1}{\sqrt{2 \pi}} \int_{-\infty}^{\infty} \dv{f(x)}{x} \; e^{i \omega x} \, \dd x
\label{eq:ecuacion_15_38}
\end{equation}
Integrando la ecuación (\ref{eq:ecuacion_15_38}), se obtiene
\begin{equation}
g_{1}(\omega) = \dfrac{e^{i \omega x}}{\sqrt{2 \pi}} \; f(x) \eval_{-\infty}^{\infty} - \dfrac{i \, \omega}{\sqrt{2 \pi}} \int_{-\infty}^{\infty} f(x) \; e^{i \omega x} \, \dd x
\label{eq:ecuacion_15_39}
\end{equation}
Si $f(x)$ se anula mientras $x \to \pm \infty$, se tiene que
\begin{equation}
g_{1}(\omega) = - i \, \omega \, g(\omega)
\label{eq:ecuacion_15_40}
\end{equation}
esto es, la transformada de la derivada es $(-i \, \omega)$ veces la transformada de la función original. Es posible generalizar entonces que la derivada de orden $n$ sea
\begin{equation}
g_{n} (\omega) = (- i \, \omega)^{n} \, g(\omega)
\label{eq:ecuacion_15_41}
\end{equation}
garantizando que la integración por partes se anula mientras $x \to \pm \infty$.
\par
Esta es la potencia de la transformada de Fourier, y la razón por la cual es muy útil para resolver EDO y EDP. Se cambió la diferenciación por una multiplicación en el espacio-$\omega$.
\subsection*{Ejemplo. La ecuación de onda.}
Aprovecharemos la ventaja de la transformada de la deriva para manejar ecuaciones diferenciales parciales. Consideremos una cuerda infinita que vibra, la amplitud $y$ de las vibraciones es pequeña, y satisface la ecuación de onda
\begin{equation}
\pdv[2]{y}{x} = \dfrac{1}{v^{2}} \, \pdv[2]{y}{t}
\label{eq:ecuacion_15_42}
\end{equation}
con la condición inicial
\begin{equation}
y(x, 0) = f(x)
\label{eq:ecuacion_15_43}
\end{equation}
donde $f$ está localizada, es decir, se acerca a cero en para valores grandes de $x$.
\par
Aplicar la TF en $x$, significa multiplicar por $e^{i \omega x}$ y luego integrar sobre $x$, así que
\begin{equation}
\int_{-\infty}^{\infty} \pdv[2]{y(x, t)}{x} \; e^{i \omega x} \, \dd x = \dfrac{1}{v^{2}} \, \int_{-\infty}^{\infty} \pdv[2]{y (x, t)}{t} \; e^{i \omega x} \, \dd x
\label{eq:ecuacion_15_44}
\end{equation}
o que es lo mismo
\begin{equation}
(-i \, \alpha)^{2} \; Y(\alpha, t) = \dfrac{1}{v^{2}} \; \pdv[2]{Y(\alpha, t)}{t}
\label{eq:ecuacion_15_45}
\end{equation}
donde se ha utilizado
\begin{equation}
Y(\alpha, t) = \dfrac{1}{\sqrt{2 \pi}} \int_{-\infty}^{\infty} y(x, t) \; e^{i \alpha x} \, \dd x
\label{eq:ecuacion_15_46}
\end{equation}
y la ec. (\ref{eq:ecuacion_15_41}) para la segunda derivada. Nótese que parte del integrando en la ec. (\ref{eq:ecuacion_15_39}) se anula: la onda aún no se ha desplazado a $\pm \infty$ porque se está propagando hacia adelante en el tiempo, y no hay una fuente en el infinito porque $f(\pm \infty) = 0$.
\par
Dado que no se tienen derivadas con respecto a $\alpha$, la ec. (\ref{eq:ecuacion_15_45}) es un EDO, de hecho la ecuación del oscilador armónico. La transformación de una EDP a una EDO es un logro significativo. Resolvemos la ec. (\ref{eq:ecuacion_15_45}) sujeta a las condiciones iniciales. En $t=0$, aplicando las ecuaciones (\ref{eq:ecuacion_15_43}) y (\ref{eq:ecuacion_15_46}), se simplifica a
\begin{equation}
Y(\alpha, 0) = \dfrac{1}{\sqrt{2 \pi}} \int_{-\infty}^{\infty} f(x) \; e^{i \alpha x} \, \dd x =  F(\alpha)
\label{eq:ecuacion_15_47}
\end{equation}
La solución general de la ec. (\ref{eq:ecuacion_15_45}) es de una forma exponencial
\begin{equation}
Y(\alpha, t) = F(\alpha) \; \exp(\pm  i \, v \, \alpha \, t)
\label{eq:ecuacion_15_48}
\end{equation}
Usando la fórmula de inversión (ec. \ref{eq:ecuacion_15_23}), se tiene
\begin{equation}
y(x,t) = \dfrac{1}{\sqrt{2 \pi}} \int_{-\infty}^{\infty} Y(\alpha, t) \; e^{-i \alpha x} \, \dd \alpha
\label{eq:ecuacion_15_49}
\end{equation}
y, por la ecuación (\ref{eq:ecuacion_15_48})
\begin{equation}
y(x,t) = \dfrac{1}{\sqrt{2 \pi}} \int_{-\infty}^{\infty} F(\alpha) \; \exp(- i \, \alpha (x \mp v \, t)) \, \dd \alpha
\label{eq:ecuacion_15_50}
\end{equation}
Ya que $f(x)$ es la TF inversa de $F(\alpha)$
\begin{equation}
y(x, t) = f (x \pm v \, t)
\label{eq:ecuacion_15_51}
\end{equation}
que corresponde a ondas que avanzan en las direcciones $+x$ y $-x$ respectivamente.
\subsection*{Ejemplo. Ecuación de calor.}
Para ejemplificar la transformación de una EDP a una EDO, usemos la TF a la EDP de flujo de calor:
\[ \pdv{\psi}{t} = a^{2} \; \pdv[2]{\psi}{x} \]
donde la solución $\psi(x,t)$ es la temperatura en el espacio como función del tiempo. Tomando la TF en ambos lados de la ecuación, nótese que $\omega$ es la variable conjugada compleja para $x$ ya que $t$ es el tiempo en la EDP de flujo de calor, donde
\[ \Psi(\omega, t) = \dfrac{1}{\sqrt{2 \pi}} \int_{-\infty}^{\infty} \psi(x, t) \; e^{i \omega x} \, \dd x \]
Esto nos lleva a una EDO para la TF de $\Psi$ de $\psi$ en la variable temporal $t$
\[ \pdv{\Psi(\omega, t)}{t} = -a^{2} \, \omega^{2} \, \Psi(\omega, t) \] 
Al integrar, obtenemos
\[ \ln \Psi = -a^{2} \, \omega^{2} \, t + \ln C \hspace{1.5cm} \text{ o } \hspace{1.5cm} \Psi = C \; \exp(-a^{2} \, \omega^{2} \, t)\]
donde la constante de integración $C$ dependerá de $\omega$ y en general queda determinada por las condiciones iniciales. De hecho $C = \Psi(\omega, 0)$ es la distribución espacial inicial de $\Psi$, que está dada por la transformación (en $x$) de la distribución inicial de $\psi$, llamada $\psi(x,0)$.
\par
Dejando la solución en la transformada inversa de Fourier, se obtiene
\[ \psi(x, t) = \dfrac{1}{\sqrt{2 \pi}} \int_{-\infty}^{\infty} C(\omega) \; \exp(-i \, \omega \, x) \; \exp(-a^{2} \, \omega^{2} \, t) \, \dd \omega \]
Por razones de simplicidad, tomamos $C$ independiente de $\omega$ (suponiendo una distribución inicial de temperatura como una función delta), al integrar completando el cuadrado en $\omega$ y realizando los cambios de variable ($a^{2} \to a^{2} \, t$, $\omega \to x$, $t \to -\omega$. Nos encontramos con una solución particular de la EDP de flujo de calor
\[ \psi(x,t) = \dfrac{C}{a \sqrt{2 t}} \; \exp \left( - \dfrac{x^{2}}{4 \, a^{2} \, t} \right) \]
\section{Teorema de convolución.}
Consideremos dos funciones $f(x)$ y $g(x)$ cada una con su respectiva TF $F(t)$ y $G(t)$, se define la operación
\begin{equation}
f * g \equiv \dfrac{1}{\sqrt{2 \pi}} \int_{-\infty}^{\infty} g(y) \; f(x - y) \, \dd y
\label{eq:ecuacion_15_52}
\end{equation}
como la \textbf{convolución} de las dos funciones $f$ y $g$ en el intervalo $(-\infty, \infty)$. 
\\
Al incluir las TF en la ec. (\ref{eq:ecuacion_15_52}), para luego intercambiar el orden de integración y tranformar $g(y)$. 
\begin{align}
\int_{-\infty}^{\infty} g(y) \; f(x - y) \, \dd y &= \dfrac{1}{\sqrt{2 \pi}} \int_{-\infty}^{\infty} F(t) \; e^{-it(x - y)} \, \dd{t} \dd{y} \saltosin
&= \dfrac{1}{\sqrt{2 \pi}} \int_{-\infty}^{\infty} F(t) \left[ \int_{-\infty}^{\infty} g(y) \; e^{ity} \, \dd y \right] \, e^{-itx} \, \dd t \saltosin
&= \int_{-\infty}^{\infty} F(t) \; G(t) \; e^{-itx} \, \dd t \label{eq:ecuacion_15_53}
\end{align}
Este resultado se puede interpretar como sigue: la TF inversa de un producto de TF, es la convolución de las funciones originales $f * g$
\par
Para el caso especial $x = 0$, se tiene que
\begin{equation}
\int_{-\infty}^{\infty} F(t) \; G(t) dt = \int_{-\infty}^{\infty} f(-y) \; g(y) \, \dd y
\label{eq:ecuacion_15_54}
\end{equation}
El signo negativo en $-y$ sugiere que se intentaron los cambios.
\section{Relación de Parseval.}
Resultados análogos a las ecs. (\ref{eq:ecuacion_15_53}) y (\ref{eq:ecuacion_15_54}) se obtienen de las transformadas seno y coseno de Fourier.
\par
La ec. (\ref{eq:ecuacion_15_54}) y las respectivas convoluciones seno y coseno se les llama \emph{relaciones de Parseval}, en analogía al teorema de Parseval para las series de Fourier.
\par
La relación de Parseval es
\begin{equation}
\int_{-\infty}^{\infty} F(\omega) \; G^{*} (\omega) \, \dd \omega = \int_{-\infty}^{\infty} f(t) \; g^{*}(t) \, \dd t
\label{eq:ecuacion_15_55}
\end{equation}
Esta relación se obtiene utilizando la representación de la delta de Dirac ec. (\ref{eq:ecuacion_15_21d}), se tiene:
\begin{equation}
\int_{-\infty}^{\infty} f(t) \, g^{*}(t) \, \dd t = \int_{-\infty}^{\infty} \dfrac{1}{\sqrt{2 \, \pi}} \int_{-\infty}^{\infty} F(\omega) \, e^{-i \omega t} \, \dd \omega \cdot \dfrac{1}{\sqrt{2 \, \pi}} \int_{-\infty}^{\infty} G^{*}(x) \, e^{i x t} \, \dd{x} \dd{t}
\label{eq:ecuacion_15_56}
\end{equation}
hay que mantener especial atención en la transforma del conjugado complejo de $G^{*}(x)$ a $g^{*}(t)$. Integrando primero con respecto a $t$ y usando la ec. (\ref{eq:ecuacion_15_21d}), se obtiene
\begin{align}
\int_{-\infty}^{\infty} f(t) \, g^{*}(t) \, \dd t &= \int_{-\infty}^{\infty} F(\omega) \, \int_{-\infty}^{\infty} G^{*} (x) \, \delta (x - \omega) \, \dd{x} \dd{\omega} \nonumber \\
&= \int_{-\infty}^{\infty} F(\omega) \, G^{*} (\omega) \,\dd \omega
\label{eq:ecuacion_15_57}
\end{align}
que es la relación de Parseval.
\par
Si $f(t) = g(t)$, entonces las integrales en la relación de Parseval son la normalización de las integrales. La ec. (\ref{eq:ecuacion_15_57}) garantiza que si una función $f(t)$ es normalizada a la unidad, entonces su transformada $F(\omega)$ también es normalizada a la unidad, esto es muy importante en la mecánica cuántica.
\section{Representación del momento.}
En dinámica avanzada y en mecánica cuántica, el momento lineal y la posición espacial se producen en igualdad de condiciones. En esta sección revisaremos la distribución espacial habitual y derivaremos la distribución de momento correspondiente. Para el caso unidimensional, nuestra función de onda $\psi(x)$ tiene las siguientes propiedades:
\begin{enumerate}
\item Sea $\psi^{*}(x) \, \psi(x)$ la densidad de probabilidad de encontrar una partícula entre $x$ y $x + \dd{x}$,
\item La probabilidad unitaria es
\begin{equation}
\int_{-\infty}^{\infty} \psi^{*}(x) \, \psi(x) \, \dd x = 1
\label{eq:ecuacion_15_58}
\end{equation}
\item De manera adicional
\begin{equation}
\expval{x} = \int_{-\infty}^{\infty} \psi^{*} (x) \, \psi (x) \, \dd x
\label{eq:ecuacion_15_59}   
\end{equation}
que representa el \textbf{promedio} de la posición de la partícula en el eje-$x$, se le conoce como \emph{valor esperado (expectation value)}.
\end{enumerate}
Queremos una función $g(p)$ que nos proporcione la misma información sobre el momento:
\begin{enumerate}
\item Sea $g^{*}(p) \, g(p)$ la densidad de probabilidad de que la partícula tenga un momento entre $p$ y $p + \dd{p}$,
\item La probabilidad unitaria es
\begin{equation}
\int_{-\infty}^{\infty} g^{*}(p) \, g(p) \, \dd p = 1
\label{eq:ecuacion_15_60}
\end{equation}
\item Y que el valor esperado para el momento sea
\begin{equation}
\expval{p} = \int_{-\infty}^{\infty} g^{*} (p) \, p \, g(p) \, \dd p
\label{eq:ecuacion_15_61}   
\end{equation}
\end{enumerate}
Como veremos, esta función viene dada por la transformada de Fourier de nuestra función espacial $\psi(x)$. Específicamente:
\begin{align}
g (p) &= \dfrac{1}{\sqrt{2 \, \pi \, \hbar}} \int_{-\infty}^{\infty} \psi (x) \, e^{-i p x / \hbar} \, \dd x \label{eq:ecuacion_15_62} \\
g^{*} (p) &= \dfrac{1}{\sqrt{2 \, \pi \, \hbar}} \int_{-\infty}^{\infty} \psi^{*} (x) \, e^{i p x / \hbar} \, \dd x \label{eq:ecuacion_15_63}
\end{align}
La correspondiente función de momento en 3D es
\begin{align*}
g (\vb{p}) = \dfrac{1}{(2 \, \pi \, \hbar)^{3/2}} \iiint \limits_{-\infty}^{\infty} \psi (\vb{r}) \, \exp(-i \vb{r} \cdot \vb{p} / \hbar) \, \dd[3]{x}
\end{align*}
Para verificar las ecs. (\ref{eq:ecuacion_15_62}) y (\ref{eq:ecuacion_15_63}), veremos primero las propiedades $2$ y $3$.
\par
Propiedad $2$: La normalización se satisface en automático de la relación de Parseval: si la función espacial $\psi (x)$ se normaliza a la unidad, la función de momento $g(p)$ también se normaliza a la unidad.
\par
Para revisar la propiedad $3$, debemos de mostrar que
\begin{equation}
\expval{p} = \int_{-\infty}^{\infty} g^{*} (p) \, p \, g(p) \, \dd p = \int_{-\infty}^{\infty} \psi^{*} (x) \, \dfrac{\hbar}{i} \, \dv{x} \psi (x) \, \dd x
\label{eq:ecuacion_15_64}
\end{equation}
donde $(\hbar/i)(\dv{x})$ es el operador de momento angular en el espacio. Sustituimos las funciones de momento por las TF espaciales, y la primera integral es
\begin{equation}
\dfrac{1}{2 \,\pi \, \hbar} \iiint \limits_{-\infty}^{\infty} p \, \exp(-i p (x - x^{\prime})/ \hbar) \, \psi^{*}(x^{\prime}) \, \psi (x) \dd{p} \dd{x^{\prime}} \dd{x}
\label{eq:ecuacion_15_65}
\end{equation}
Ahora podemos utilizar la identidad de onda plana
\begin{equation}
p \, \exp(-i p (x - x^{\prime})/ \hbar) = \dv{x} \left[ - \dfrac{\hbar}{i} \, \exp(-i p (x - x^{\prime})/ \hbar) \right]
\label{eq:ecuacion_15_66}
\end{equation}
donde $p$ es una constante, no un operador. Al sustituir en la ec. (\ref{eq:ecuacion_15_65}) para luego integrar por partes, mantiendo $x^{\prime}$ y $p$ constantes, obtenemos el siguiente resultado
\begin{equation}
\expval{p} = \iint \limits_{-\infty}^{\infty} \left[ \dfrac{1}{2 \,\pi \, \hbar} \int_{-\infty}^{\infty} \exp(-i p (x - x^{\prime})/ \hbar) \, \dd p \right] \cdot \psi (x^{\prime}) \, \dfrac{\hbar}{i} \, \dv{x} \psi (x) \dd{x^{\prime}} \dd{x}
\label{eq:ecuacion_15_67}
\end{equation}
Hemos supuesto que $\psi (x)$ se anula cuando $x \to \pm \infty$, por lo que se elimina la parte que se integra. Usando la función delta de Dirac de la ec. (\ref{eq:ecuacion_15_21c}), la ec. (\ref{eq:ecuacion_15_67}) se reduce a la ec. (\ref{eq:ecuacion_15_64}) para verificar la representación del momento.
\subsection*{Ejemplo. El átomo de hidrógeno.}
El estado base del átomo de hidrógeno puede describirse por la función de onda espacial
\begin{equation}
\psi (\vb{r}) = \left( \dfrac{1}{\pi \, a_{0}^{3}} \right)^{1/3} \, e^{-r/a_{0}}
\label{eq:ecuacion_15_68}
\end{equation}
donde $a_{0}$ es el radio de Bohr: $4 \, \pi \varepsilon_{0} \, \hbar^{2} / m \, e^{2}$, ahora se tiene una función de onda en 3D.
\par
La correspondiente transformada de la ec. (\ref{eq:ecuacion_15_62}) es
\begin{equation}
g (\vb{p}) = \dfrac{1}{(2 \, \pi \, \hbar)^{3/2}} \int \psi (\vb{r}) \, \exp(-i \vb{p} \cdot \vb{r} / \hbar) \, \dd[3]{r}
\label{eq:ecuacion_15_69}
\end{equation}
Sustituyendo la ec. (\ref{eq:ecuacion_15_68}) en la ec. (\ref{eq:ecuacion_15_69}), además de usar el siguiente resultado
\begin{equation}
\int \exp(-a r + i \vb{b} \cdot \vb{r}) \, \dd[3]{r} = \dfrac{8 \, \pi \, a}{(a^{2} + b^{2})^{2}}
\label{eq:ecuacion_15_70}
\end{equation}
obtenemos la función de onda del momento hidrogenoide
\begin{equation}
g (\vb{p}) = \dfrac{2^{3/2}}{\pi} \, \dfrac{a_{0}^{3} \, \hbar^{5/2}}{(a_{0}^{2} \, p^{2} + \hbar^{2})^{2}}
\label{eq:ecuacion_15_71}
\end{equation}
Estas funciones de momento se han encontrado útiles en problemas como la dispersión de Compton a partir de electrones atómicos, la distribución de longitud de onda de la radiación dispersada, dependiendo de la distribución de momento de los electrones objetivo.
\par
La relación entre la representación espacial ordinaria y la representación del momento se puede aclarar considerando las relaciones básicas de conmutación de la mecánica cuántica. Pasamos de un hamiltoniano clásico a la ecuación de onda de Schrodinger al exigir que el momento $p$ y la posición $x$ \textbf{no conmuten}. En su lugar, requerimos que
\begin{equation}
\qty[p, x] \equiv p \, x - x \, p = - i \, \hbar
\label{eq:ecuacion_15_72}
\end{equation}
Para el caso multidimensional, la ec (\ref{eq:ecuacion_15_72}) sería la siguiente
\begin{equation}
\qty[p_{i}, x_{j}] = - i \, \hbar \, \delta_{i j}
\label{eq:ecuacion_15_73}
\end{equation}
La representación espacial de Schrödinger se obtiene utilizando
\[ x \to x : \hspace{1cm} p_{i} \to - i \, \hbar \, \pdv{x_{i}} \]
re-emplazando el momento por la derivada parcial espacial. Vemos que
\begin{equation}
\qty[p, x] \, \psi (x) = - i \, \hbar \, \psi(x)
\label{eq:ecuacion_15_74}
\end{equation}
Sin embargo, la ec. (\ref{eq:ecuacion_15_72}) de igual forma se satisface usando
\[p \to p : \hspace{1cm} x_{j} \to - i \, \hbar \, \pdv{p_{j}} \]
Que es a representación del momento. Entonces
\begin{equation}
\qty[p, x] \, g (p) = - i \, \hbar \, g (p)
\label{eq:ecuacion_15_75}
\end{equation}
Por lo tanto, la representación $(x)$ no es única; $(p)$ es una posibilidad alterna.
\par
En general, la representación de Schrödinger $(x)$ que conduce a la ecuación de onda de Schrödinger es más conveniente porque la energía potencial $V$ generalmente se da en función de la posición $V (x, y, z)$. La representación de momento $(p)$ generalmente conduce a una ecuación integral.
\end{document}
