% \documentclass[hidelinks,12pt]{article}
\usepackage[left=0.25cm,top=1cm,right=0.25cm,bottom=1cm]{geometry}
%\usepackage[landscape]{geometry}
\textwidth = 20cm
\hoffset = -1cm
\usepackage[utf8]{inputenc}
\usepackage[spanish,es-tabla]{babel}
\usepackage[autostyle,spanish=mexican]{csquotes}
\usepackage[tbtags]{amsmath}
\usepackage{nccmath}
\usepackage{amsthm}
\usepackage{amssymb}
\usepackage{mathrsfs}
\usepackage{graphicx}
\usepackage{subfig}
\usepackage{standalone}
\usepackage[outdir=./Imagenes/]{epstopdf}
\usepackage{siunitx}
\usepackage{physics}
\usepackage{color}
\usepackage{float}
\usepackage{hyperref}
\usepackage{multicol}
%\usepackage{milista}
\usepackage{anyfontsize}
\usepackage{anysize}
%\usepackage{enumerate}
\usepackage[shortlabels]{enumitem}
\usepackage{capt-of}
\usepackage{bm}
\usepackage{relsize}
\usepackage{placeins}
\usepackage{empheq}
\usepackage{cancel}
\usepackage{wrapfig}
\usepackage[flushleft]{threeparttable}
\usepackage{makecell}
\usepackage{fancyhdr}
\usepackage{tikz}
\usepackage{bigints}
\usepackage{scalerel}
\usepackage{pgfplots}
\usepackage{pdflscape}
\pgfplotsset{compat=1.16}
\spanishdecimal{.}
\renewcommand{\baselinestretch}{1.5} 
\renewcommand\labelenumii{\theenumi.{\arabic{enumii}})}
\newcommand{\ptilde}[1]{\ensuremath{{#1}^{\prime}}}
\newcommand{\stilde}[1]{\ensuremath{{#1}^{\prime \prime}}}
\newcommand{\ttilde}[1]{\ensuremath{{#1}^{\prime \prime \prime}}}
\newcommand{\ntilde}[2]{\ensuremath{{#1}^{(#2)}}}

\newtheorem{defi}{{\it Definición}}[section]
\newtheorem{teo}{{\it Teorema}}[section]
\newtheorem{ejemplo}{{\it Ejemplo}}[section]
\newtheorem{propiedad}{{\it Propiedad}}[section]
\newtheorem{lema}{{\it Lema}}[section]
\newtheorem{cor}{Corolario}
\newtheorem{ejer}{Ejercicio}[section]

\newlist{milista}{enumerate}{2}
\setlist[milista,1]{label=\arabic*)}
\setlist[milista,2]{label=\arabic{milistai}.\arabic*)}
\newlength{\depthofsumsign}
\setlength{\depthofsumsign}{\depthof{$\sum$}}
\newcommand{\nsum}[1][1.4]{% only for \displaystyle
    \mathop{%
        \raisebox
            {-#1\depthofsumsign+1\depthofsumsign}
            {\scalebox
                {#1}
                {$\displaystyle\sum$}%
            }
    }
}
\def\scaleint#1{\vcenter{\hbox{\scaleto[3ex]{\displaystyle\int}{#1}}}}
\def\bs{\mkern-12mu}


% %\usepackage{showframe}
% \title{Transformadas de Laplace \\ \large {Tema 6 - Transformadas integrales} \vspace{-3ex}}
% \author{M. en C. Gustavo Contreras Mayén}
% \date{ }
% \begin{document}
% \vspace{-4cm}
% \maketitle
% \fontsize{14}{14}\selectfont
% \tableofcontents
% \newpage

\documentclass[12pt]{beamer}
\usepackage{../Estilos/BeamerMAF}
\usepackage[absolute, overlay]{textpos}
\usepackage{../Estilos/ColoresLatex}
%Sección para el tema de beamer, con el theme, usercolortheme y sección de footers
\usetheme{Antibes}
\usecolortheme{beaver}
%\useoutertheme{default}
\setbeamercovered{invisible}
% or whatever (possibly just delete it)
\setbeamertemplate{section in toc}[sections numbered]
\setbeamertemplate{subsection in toc}[subsections numbered]
\setbeamertemplate{subsection in toc}{\leavevmode\leftskip=3.2em\rlap{\hskip-2em\inserttocsectionnumber.\inserttocsubsectionnumber}\inserttocsubsection\par}
\setbeamercolor{section in toc}{fg=blue}
\setbeamercolor{subsection in toc}{fg=blue}
%\setbeamercolor{frametitle}{fg=blue}
\setbeamertemplate{caption}[numbered]

\setbeamertemplate{footline}
\beamertemplatenavigationsymbolsempty
\setbeamertemplate{headline}{}


\makeatletter
\setbeamercolor{secºtion in foot}{bg=gray!30, fg=black!90!orange}
\setbeamercolor{subsection in foot}{bg=blue!30!yellow, fg=red}
\setbeamercolor{date in foot}{bg=black, fg=white}
\setbeamertemplate{footline}
{
  \leavevmode%
  \hbox{%
  \begin{beamercolorbox}[wd=.333333\paperwidth,ht=2.25ex,dp=1ex,center]{section in foot}%
    \usebeamerfont{section in foot} \insertsection
  \end{beamercolorbox}%
  \begin{beamercolorbox}[wd=.333333\paperwidth,ht=2.25ex,dp=1ex,center]{subsection in foot}%
    \usebeamerfont{subsection in foot}  \insertsubsection
  \end{beamercolorbox}%
  \begin{beamercolorbox}[wd=.333333\paperwidth,ht=2.25ex,dp=1ex,right]{date in head/foot}%
    \usebeamerfont{date in head/foot} \insertshortdate{} \hspace*{2em}
    \insertframenumber{} / \inserttotalframenumber \hspace*{2ex} 
  \end{beamercolorbox}}%
  \vskip0pt%
}







\setbeamercolor{section in foot}{bg=amethyst, fg=white}
\setbeamercolor{subsection in foot}{bg=almond, fg=black}

\makeatletter
\setbeamertemplate{footline}
{
\leavevmode%
\hbox{%
\begin{beamercolorbox}[wd=.333333\paperwidth,ht=2.25ex,dp=1ex,center]{section in foot}%
  \usebeamerfont{section in foot} \insertsection
\end{beamercolorbox}%
\begin{beamercolorbox}[wd=.333333\paperwidth,ht=2.25ex,dp=1ex,center]{subsection in foot}%
  \usebeamerfont{subsection in foot}  \insertsubsection
\end{beamercolorbox}%
\begin{beamercolorbox}[wd=.333333\paperwidth,ht=2.25ex,dp=1ex,right]{date in head/foot}%
  \usebeamerfont{date in head/foot} \insertshortdate{} \hspace*{1.5em}
  \insertframenumber{} / \inserttotalframenumber \hspace*{2ex} 
\end{beamercolorbox}}%
\vskip0pt%
}
\makeatother
\usefonttheme{serif}
\setbeamercolor{frametitle}{bg=champagne}
\resetcounteronoverlays{saveenumi}

\date{}

\title{\large{Tema 6 - Transformada de Laplace}}
\subtitle{Matemáticas Avanzadas de la Física}
\author{M. en C. Gustavo Contreras Mayén}

\AtBeginDocument{\RenewCommandCopy\qty\SI}

\begin{document}
\maketitle
\fontsize{14}{14}\selectfont
\spanishdecimal{.}

\section*{Contenido}
\frame[allowframebreaks]{\frametitle{Temas a revisar} \tableofcontents[currentsection, hideallsubsections]}

%Ref. Patra (2018) Chap. 3 The Laplace Transfom
\section{La Transformada de Laplace}
\frame[allowframebreaks]{\tableofcontents[currentsection, hideothersubsections]}
\subsection{Introducción}

\begin{frame}
\frametitle{Existencia de la integral}
En el material de trabajo sobre la transformada de Fourier, se revisó que si la integral:
\pause
\begin{align*}
\scaleint{6ex}_{\bs -\infty}^{\infty} \abs{f (t)} \dd{t}
\end{align*}
no converge, \pause la TF $F (\xi)$ no existe para todo valor real de $\xi$.
\end{frame}
\begin{frame}
\frametitle{Existencia de la integral}
Por ejemplo: si $f (t) = \sin \omega \, t$ que es un real, \pause la $F (\xi)$ no existe.
\\
\bigskip
\pause
Pero tales situaciones surgen ocasionalmente en la práctica.
\end{frame}
\begin{frame}
\frametitle{Existencia de la integral}
Para manejar esta situación, consideramos una nueva función $f_{1} (t)$ conectada a $f (t)$ definida por:
\pause
\begin{align*}
f_{1} (t) =  \exp(- \gamma \, t) \, f (t) \, H (t)
\end{align*}
donde $\gamma$ es una constante arbitraria real positiva y $H (t)$ es la función de paso unitario de Heaviside.
\end{frame}
\begin{frame}
\frametitle{Existencia de la integral}
Claramente se tiene que $f_{1} (t) \in A_{1} (R) \, \forall \, t \in R$ y por lo tanto, la TF de $f_{1} (t)$ existe, ya que:
\pause
\begin{align*}
\scaleint{6ex}_{\bs -\infty}^{+\infty} f_{1} \dd{t} = \scaleint{6ex}_{\bs 0}^{\infty} \exp(-\gamma \, t) \, f (t) \, \dd{t}
\end{align*}
es convergente.
\end{frame}
\begin{frame}
\frametitle{Existencia de la integral}
De hecho, en este caso, del teorema de la integral de Fourier:
\pause
\begin{align*}
f_{1} (t) &= \dfrac{1}{2 \, \pi} \scaleint{6ex}_{\bs -\infty}^{\infty} \exp(-i \, \xi \, t) \dd{\xi} \times \\[0.5em]
&\times \scaleint{6ex}_{\bs -\infty}^{+\infty} f_{1} (u) \, \exp(i \, \xi \, u) \dd{u}
\end{align*}
\end{frame}
\begin{frame}
\frametitle{Existencia de la integral}
Implica que:
\pause
\begin{align*}
&f (t) = \dfrac{1}{2 \, \pi} \scaleint{6ex}_{\bs -\infty}^{\infty} \exp(-i \, \xi \, t) \dd{\xi} \cdot \exp{\gamma \, t} \times \\[0.5em]
&\times \scaleint{6ex}_{\bs 0}^{+\infty} f (u) \, \exp(- (\gamma - i \, \xi) \, u) \dd{u}
\end{align*}
\end{frame}
\begin{frame}
\frametitle{Reescribiendo la expresión}
Si escribimos $p = - i \, \xi$, la relación anterior puede expresarse como:
\pause
\begin{eqnarray}
\begin{aligned}[b]
f (t) &= \dfrac{1}{2 \, \pi \, i} \scaleint{6ex}_{\bs \gamma -i \, \infty}^{\gamma + i \, \infty} e^{p \, t} \bigg[ \scaleint{6ex}_{\bs 0}^{+\infty} f (u) \, e^{- p \, u} \dd{u} \bigg] \dd{p} = \\[0.5em] \pause
&= \dfrac{1}{2 \, \pi \, i} \scaleint{6ex}_{\bs \gamma -i \, \infty}^{\gamma + i \, \infty} \exp(p \, t) \, \overline{f} (p) \dd{p} 
\end{aligned}
\label{eq:ecuacion_03_01}
\end{eqnarray}
\end{frame}
\begin{frame}
\frametitle{Reescribiendo la expresión}
Donde:
\pause
\begin{align}
\begin{aligned}
\overline{f} (p) = \scaleint{6ex}_{\bs 0}^{+\infty} f (u) \, \exp(- p \, u) \dd{u}, \\[0.5em]
\Re{p} = \gamma > 0
\end{aligned}
\label{eq:ecuacion_03_02}
\end{align}
\pause
Las ecs. (\ref{eq:ecuacion_03_01}) y (\ref{eq:ecuacion_03_02}) constituyen una transformada con $K (p, u) = \exp(- p \, u)$ como núcleo o kernel.
\end{frame}

\subsection{Definiciones}

\begin{frame}
\frametitle{La Tranformada de Laplace}
Definimos la \textocolor{ao}{transformada de Laplace} (TL) de una función continua en tramos de variable real $t$, definida en un semieje $t \geq 0$ como:
\pause
\begin{eqnarray}
\begin{aligned}
L \big[ f (t); t \to p \big] &= \pause L \big[ f (t) \big] = \pause f (p) = \pause \overline{f} (p) = \\[0.5em] \pause
&= \scaleint{6ex}_{\bs 0}^{\infty} f (t) \, \exp(-p \, t) \dd{t}
\end{aligned}
\label{eq:ecuacion_03_03}
\end{eqnarray}
\end{frame}
\begin{frame}
\frametitle{La Tranformada inversa de Laplace}
Definimos la \textocolor{armygreen}{transformada de Laplace inversa} como:
\pause
\begin{eqnarray}
\begin{aligned}
&f (t) = L^{-1} \big[ \overline{f}(t); p \to t \big] = \pause L^{-1} \big[ f (p) \big] = \\[0.5em] \pause
&= \dfrac{1}{2 \pi \, i} \scaleint{6ex}_{\bs \gamma-i \infty}^{\gamma+\infty} \exp(p \, t) \, \overline{f} (p) \, \dd{p}
\end{aligned}
\label{eq:ecuacion_03_04}
\end{eqnarray}
donde $\gamma = \Re{p} > 0$.
\end{frame}
\begin{frame}
\frametitle{Anotaciones relevantes}
\setbeamercolor{item projected}{bg=black,fg=white}
\setbeamertemplate{enumerate items}{%
\usebeamercolor[bg]{item projected}%
\raisebox{1.5pt}{\colorbox{bg}{\color{fg}\footnotesize\insertenumlabel}}%
}
\begin{enumerate}[<+->]
\item Algunos autores usan la variable $s$ en lugar de $p$.
\item La función $f (t)$ debe de ser de orden exponencial (concepto que ya definimos) para la existencia de la TL.
\seti
\end{enumerate}
\end{frame}
\begin{frame}
\frametitle{Anotaciones relevantes}
\setbeamercolor{item projected}{bg=black,fg=white}
\setbeamertemplate{enumerate items}{%
\usebeamercolor[bg]{item projected}%
\raisebox{1.5pt}{\colorbox{bg}{\color{fg}\footnotesize\insertenumlabel}}%
}
\begin{enumerate}[<+->]
\conti
\item En términos de notación de operadores, la ec. (\ref{eq:ecuacion_03_03}) se expresa como:
\pause
\begin{align}
L \big[ f (t); t \to p \big] = \overline{f} (p) 
\label{eq:ecuacion_03_05}
\end{align}
\end{enumerate}
\end{frame}
\begin{frame}
\frametitle{Anotaciones relevantes}
Y la relación en la ec. (\ref{eq:ecuacion_03_04}) se expresa como:
\pause
\begin{align}
L^{-1} \big[ \overline{f}(p); p \to t  \big] = f (t)
\label{eq:ecuacion_03_06}
\end{align}
\end{frame}
\begin{frame}
\frametitle{Anotaciones relevantes}
Por lo que:
\pause
\begin{align*}
L^{-1} \big[ L(f (t))] \big] = f (t) &= I \big[ f (t) \big] \\[0.5em]
\Rightarrow \hspace{0.4cm} L^{-1} \, L \equiv I
\end{align*}
\end{frame}
\begin{frame}
\frametitle{Anotaciones relevantes}
También se tiene que:
\pause
\begin{eqnarray*}
\begin{aligned}
L \big[ L^{-1} (\overline{f}(p))] \big] = \overline{f}(p) &= \pause I \big[ \overline{f}(p) \big] \\[0.5em] \pause
\Rightarrow \hspace{0.4cm} L \, L^{-1} \equiv I
\end{aligned}
\end{eqnarray*}
\end{frame}
\begin{frame}
\frametitle{Anotaciones relevantes}
Esto significa que $L^{-1} \, L \equiv L \, L^{-1} \equiv I$, \pause demostrando entonces que esos operadores $L$ y $L^{-1}$ son conmutativos.
\end{frame}

\subsection{Condiciones suficientes TL}

\begin{frame}
\frametitle{Teorema}
Si $f (t)$ es una función de algún orden exponencial para un valor de $t$ grande, y es continua en tramos sobre el intervalo $0 \leq t \leq \infty$, entonces la TL de $f (t)$ existe.
\end{frame}
\begin{frame}
\frametitle{Función de orden exponencial}
Se dice que la función $f (t)$ es de \textocolor{bole}{orden exponencial} conforme $t \to + \infty$ si existen constantes no negativas $M$, $\sigma$ y $t_{0}$ tales que:
\pause
\begin{align}
\abs{f (t)} \leq M \; e^{\sigma \, t} \hspace{1cm} \mbox{para } t \geq t_{0}
\label{eq:023}
\end{align}
\end{frame}
\begin{frame}
\frametitle{Función de orden exponencial}
Así, una función es de orden exponencial siempre que su incremento (conforme $t \to + \infty$) no sea más rápido que un múltiplo constante de alguna función exponencial con un exponente lineal.
\end{frame}
\begin{frame}
\frametitle{Función de orden exponencial}
Los valores particulares de $M$, $\sigma$ y $t_{0}$ no son tan importantes; lo importante es que algunos de esos valores existan de tal manera que la condición en (\ref{eq:023}) se satisfaga.
\end{frame}
\begin{frame}
\frametitle{Función de orden exponencial}
La condición en (\ref{eq:023}) simplemente dice que $f (t) / e^{\sigma t}$ se encuentra entre $-M$ y $M$, y es por tanto acotada en su valor para $t$ suficientemente grande.
\\
\bigskip
\pause
En particular, esto se cumple (con $\sigma = 0$) si $f (t)$ en sí misma está acotada.
\end{frame}
\begin{frame}
\frametitle{Función de orden exponencial}
Por tanto, toda función acotada -tal como $\cos k \, t$ o $\sin k \, t$ - es de orden exponencial.
\end{frame}

% \noindent \textbf{Demostración: } Sea $f (t)$ de orden exponencial $\sigma$ tal que:
% \begin{align*}
% \abs{f (t)} < M \, e^{\sigma t} \hspace{1cm} \mbox{para} t \geq t_{0}
% \end{align*}
% Entonces se tenemos:
% \begin{align*}
% L \big[ f (t); t \to p \big] &= \scaleint{6ex}_{\bs 0}^{\infty} \exp(-p \, t) \, f (t) \dd{t} = \\[0.5em]
% &= \scaleint{6ex}_{\bs 0}^{t_{0}} \exp(-p \, t) \, f (t) \dd{t} + \scaleint{6ex}_{\bs t_{0}}^{\infty} \exp(-p \, t) \, f (t) \dd{t} = \\[0.5em]
% &= I_{1} + I_{2}
% \end{align*}
% Ya que $f (t)$ es una función continua en tramos en cada intervalo finito $0 \leq t \leq t_{0}$, la integral $I_{1}$ existe y es convergente. También se tiene:
% \begin{align*}
% \abs{I_{2}} &= \abs{\scaleint{6ex}_{\bs t_{0}}^{\infty} e^{- p t} \, f (t) \dd{t}} \leq \\[0.5em]
% &\leq \scaleint{6ex}_{\bs t_{0}}^{\infty} e^{- p t} \, \abs{f (t)} \dd{t} < \\[0.5em]
% &< M \, \scaleint{6ex}_{\bs t_{0}}^{\infty} \exp(-(p - \sigma) \, t) \dd{t} = \dfrac{M \, \exp(-(p - \sigma) \, t_{0})}{(p - \sigma)}, \hspace{1cm} \mbox{si  } p > \sigma
% \end{align*}
% Por lo que $\abs{I_{2}}$ es finito para todo $t_{0} > 0$ y $p > \sigma$, y de aquí resulta que $I_{2}$ es convergente. Por tanto $L\abs{f (t)}$ existe para toda $p > \sigma$.

% \begin{frame}
% \frametitle{Existencia de la TL}
% Aunque las condiciones establecidas en el teorema anterior son suficientes para la existencia de la TL, \pause no lo son.
% \end{frame}
% \begin{frame}
% \frametitle{Existencia de la TL}
% Esto significa que incluso si una función no satisface las condiciones anteriores, la TL de esa función puede existir o no. 
% \end{frame}
% \begin{frame}
% \frametitle{Existencia de la TL}
% Consideremos el siguiente ejemplo con $f (t) = 1 / \sqrt{t}$:
% \\
% \bigskip
% \pause
% De la función se tiene que $f (t) \to \infty$ mientras que $t \to 0$, por lo que $f (t)$ no es una función continua en tramos para cada intervalo finito para $t \geq 0$.
% \par
% Ahora veamos:
% \begin{align*}
% L \bigg[ \dfrac{1}{\sqrt{t}} \bigg] &= \scaleint{6ex}_{\bs 0}^{\infty} e^{-p t} \, \dfrac{1}{\sqrt{t}} \dd{t} = \\[0.5em]
% &= \dfrac{2}{\sqrt{p}} \scaleint{6ex}_{\bs 0}^{\infty} e^{x^{2}} \dd{x} = \\[0.5em]
% &= \sqrt{\dfrac{\pi}{p}} \hspace{1cm} p > 0
% \end{align*}
% Esto prueba la existencia de la TL de $f (t)$.

\section{Propiedades de la TL}
\frame[allowframebreaks]{\tableofcontents[currentsection, hideothersubsections]}
\subsection{Linealidad}

\begin{frame}
\frametitle{Linalidad de la TL}
Si $L \big[  f_{1} (t); t \to p  \big]$ y $L \big[  f_{2} (t); t \to p  \big]$ existen ambas y con las constantes $c_{1}$ y $c_{2}$, se tiene entonces:
\pause
\begin{eqnarray*}
\begin{aligned}
&L \big[  c_{1} \, f_{1} (t) + c_{2} \, f_{2} (t) ; t \to p  \big] = \\[0.5em] \pause
&=  c_{1} \, L \big[  f_{1} (t); t \to p  \big] + c_{2} \, L c \big[  f_{2} (t); t \to p  \big]
\end{aligned}
\end{eqnarray*}
\end{frame}

\subsection{Primer teorema de desplazamiento}

\begin{frame}
\frametitle{Teorema de desplazamiento}
Si la TL de $f (t)$ es $\overline{f} (p)$, \pause entonces la TL de $\exp(a \, t)$ es $\overline{f} (p - a)$.
\end{frame}
\begin{frame}
\frametitle{Demostración del teorema}
Como tenemos:
\pause
\begin{eqnarray*}
\begin{aligned}
L \big[  f (t); t \to p  \big] &= \scaleint{6ex}_{\bs 0}^{\infty} f (t) \, \exp{-p \, t} \dd{t} = \\[0.5em] \pause
&= \overline{f} (p)
\end{aligned}
\end{eqnarray*}
\end{frame}
\begin{frame}
\frametitle{Demostración del teorema}
Entonces:
\pause
\begin{eqnarray}
\begin{aligned}[b]
&L \big[  \exp (a \, t) \, f (t); t \to p  \big] = \\[0.5em] \pause
&= \scaleint{6ex}_{\bs 0}^{\infty} \exp(a \, t) \, f (t) \, \exp (-p \, t) \dd{t} = \\[0.5em] \pause
&= \scaleint{6ex}_{\bs 0}^{\infty} f (t) \, \exp(-(p - a) \, t) \dd{t} = \\[0.5em] \pause
&= \overline{f} (p - a)
\end{aligned}
\label{eq:ecuacion_03_07}
\end{eqnarray}
\end{frame}

\subsection{Segundo teorema de desplazamiento}

\begin{frame}
\frametitle{Segundo teorema de desplazamiento}
Si la TL de $f (t)$ es $\overline{f} (p)$, \pause entonces la TL de $f (t - a) \, H (t - a)$ es \pause $\exp(-a \, p) \, \overline{f} (p)$.
\end{frame}
\begin{frame}
\frametitle{Demostración del teorema}
Como sabemos que:
\pause
\begin{eqnarray*}
\begin{aligned}
L \big[  f (t); t \to p  \big] &= \scaleint{6ex}_{\bs 0}^{\infty} f (t) \, \exp(-p \, t) \dd{t} = \\[0.5em] \pause
&= \overline{f} (p)
\end{aligned}
\end{eqnarray*}
\end{frame}
\begin{frame}
\frametitle{Demostración del teorema}
Entonces:
\pause
\begin{eqnarray}
\begin{aligned}
&L \big[  f (t - a) \, H (t - a); t \to p  \big] = \\[0.5em] \pause
&= \scaleint{6ex}_{\bs 0}^{\infty} f (t - a) \, H (t - a) \, \exp(-p \, t) \dd{t} = \\[0.5em] \pause
&= \scaleint{6ex}_{\bs a}^{\infty} f (t - a) \, \exp(-p \, x) \cdot \exp(-p \, a) \dd{x} = \\[0.5em] \pause
&= \exp(-p \, a) \, \overline{f} (p) 
\end{aligned}
\label{eq:ecuacion_03_08}
\end{eqnarray}
\end{frame}

\subsection{Cambio de escala}

\begin{frame}
\frametitle{Teorema cambio de escala}
Si:
\pause
\begin{align*}
L \big[  f (t); t \to p  \big] &= \overline{f} (p)
\end{align*}
\pause
Entonces:
\pause
\begin{align*}
L \big[  f (a \, t); t \to p  \big] &= \dfrac{1}{a}\overline{f} \left(\dfrac{p}{a}\right)
\end{align*}
\end{frame}
\begin{frame}
\frametitle{Demostración del teorema}
La TL de $f (t)$ es:
\pause
\begin{eqnarray*}
\begin{aligned}
L \big[  f (t); t \to p  \big] &= \scaleint{6ex}_{\bs 0}^{\infty} f (t) \, \exp(-p \, t) \dd{t} = \\[0.5em] \pause
&= \overline{f} (p)
\end{aligned}
\end{eqnarray*}
\end{frame}
\begin{frame}
\frametitle{Demostración del teorema}
Entonces:
\pause
\begin{eqnarray}
\begin{aligned}[b]
&L \big[  f(a \, t); t \to p  \big] = \\[0.5em] \pause
&= \scaleint{6ex}_{\bs 0}^{\infty} f (a \, t) \, \exp(-p \, t) \dd{t} = \\[0.5em] \pause 
&= \dfrac{1}{a} \scaleint{6ex}_{\bs 0}^{\infty} f (x) \, \exp(- (p/a) \, t) \dd{x} = \\[0.5em] \pause
&= \dfrac{1}{a}\overline{f} \left(\dfrac{p}{a}\right)
\end{aligned}
\label{eq:ecuacion_03_09}
\end{eqnarray}
\end{frame}

\subsection{Ejemplos}

\begin{frame}
\frametitle{Ejemplo 1}
Calculemos la TL de algunas funciones sencillas a partir de la definición.
\pause
\begin{eqnarray*}
\begin{aligned}
&L \big [H (t); t \to p  \big] = \scaleint{6ex}_{\bs 0}^{\infty} \exp(- p \, t) \, H (t) \dd{t} = \\[0.5em] \pause
&= \scaleint{6ex}_{\bs 0}^{\infty} \exp(- p \, t) \dd{t} = \\[0.5em] \pause
&= \dfrac{1}{p}
\end{aligned}
\end{eqnarray*}
\end{frame}
\begin{frame}
\frametitle{Ejemplo 1}
De aquí que:
\pause
\begin{eqnarray*}
\begin{aligned}
&L \big[  H(t - a); t \to p  \big] = \scaleint{6ex}_{\bs 0}^{\infty} \exp(- p \, t) \, H (t - a) \dd{t} = \\[0.5em] \pause
&= \scaleint{6ex}_{\bs a}^{\infty} \exp(- p \, t) \dd{t} = \\[0.5em] \pause
&= \dfrac{\exp(-a \, p)}{p}
\end{aligned}
\end{eqnarray*}
\end{frame}
\begin{frame}
\frametitle{Ejemplo 2}
\begin{eqnarray*}
\begin{aligned}
&L \big[  t^{\nu}; t \to p  \big] &= \scaleint{6ex}_{\bs 0}^{\infty} t^{\nu} \, \exp(- p \, t) \dd{t} = \\[0.5em] \pause
&= \scaleint{6ex}_{\bs 0}^{\infty} u^{\nu} \, \exp(-u) \dd{u} \hspace{1cm} \mbox{cuando  } u = p \, t \\[0.5em] \pause
&= p^{-\nu - 1} \, \Gamma (\nu + 1)
\end{aligned}
\end{eqnarray*}
\pause
la cual existe cuando $\Re{\nu} > - 1$
\end{frame}
\begin{frame}
\frametitle{Ejemplo 3}
\begin{eqnarray*}
\begin{aligned}
&L \big[  e^{a t}; t \to p  \big] = \scaleint{6ex}_{\bs 0}^{\infty} e^{-p t} \, e^{a \, t} \dd{t} = \\[0.5em] \pause
&= \dfrac{1}{p - a} \hspace{1cm} p > a
\end{aligned}
\end{eqnarray*}
\end{frame}
\begin{frame}
\frametitle{Ejemplo 4}
\begin{eqnarray*}
\begin{aligned}
&L \big[  \sin (a t); t \to p  \big] = \scaleint{6ex}_{\bs 0}^{\infty} e^{-p t} \, \sin (a \, t) \dd{t} = \\[0.5em] \pause
&= \scaleint{6ex}_{\bs 0}^{\infty} e^{-p t} \,\left[ \dfrac{e^{a i t} - e^{- a i t}}{2 \, i} \right] \dd{t} = \\[0.5em] \pause
&= \left[ \scaleint{6ex}_{\bs 0}^{\infty}  \dfrac{e^{-(p - a i) t} - e^{- (p + a i) t}}{2 \, i} \right] \dd{t} = 
\end{aligned}
\end{eqnarray*}
\end{frame}
\begin{frame}
\frametitle{Ejemplo 4}
Por la propiedad de linealidad:
\pause
\begin{eqnarray*}
\begin{aligned}    
&= \dfrac{1}{2 \, i} \left[ \dfrac{1}{p {-} a \, i} {-} \dfrac{1}{p {+} a \, i} \right] = \\[0.5em] \pause
&= \dfrac{a}{p^{2} + a^{2}}, \hspace{1cm} p > 0
\end{aligned}
\end{eqnarray*}
\end{frame}
\begin{frame}
\frametitle{Ejemplo 5}
\begin{eqnarray*}
\begin{aligned}
&L \big[  \cos (a t); t \to p  \big] = \scaleint{6ex}_{\bs 0}^{\infty} e^{-p t} \,\dfrac{e^{a i t} + e^{- a i t}}{2} \dd{t} = \\[0.5em] \pause
&= \dfrac{1}{2} \left[ \scaleint{6ex}_{\bs 0}^{\infty} e^{-(p - a i) t} \dd{t} \right] + \dfrac{1}{2} \left[ \scaleint{6ex}_{\bs 0}^{\infty} e^{-(p + a i) t} \dd{t} \right] = \\[0.5em] \pause
&\mbox{por la propiedad de linealidad  } \\[0.5em] \pause
&= \dfrac{1}{2} \left[ \dfrac{1}{p - a \, i} + \dfrac{1}{p + a \, i}\right] = \pause \dfrac{p}{p^{2} + a^{2}}, \hspace{1cm} p > 0
\end{aligned}
\end{eqnarray*}
\end{frame}
\begin{frame}
\frametitle{Ejemplo 6}
Evalúa la TL de la función que se indica:
\pause
\begin{align*}
L \big[  f (t)  \big] \hspace{0.5cm} \mbox{donde} \hspace{0.5em} f (t) = \begin{cases}
t/a & 0 < t < a \\
1 & t > a
\end{cases}
\end{align*}
\end{frame}
\begin{frame}
\frametitle{Solución al Ejemplo 6}
Solución:
\pause
\begin{align*}
&L \big[  f (t)  \big] = \scaleint{6ex}_{\bs 0}^{\infty} e^{-p t} \, f (t) \dd{t}  = \\[0.5em]
&= \scaleint{6ex}_{\bs 0}^{a} e^{-p t} \cdot \dfrac{t}{a} \dd{t} + \scaleint{6ex}_{\bs 0}^{a} e^{-p t} \dd{t} = \\[0.5em]
&= \dfrac{1 - e^{- a p}}{a \, p^{2}}
\end{align*}
\end{frame}
\begin{frame}
\frametitle{Ejemplo 7}
Evalúa $L \bigg[\dfrac{1}{\sqrt{\pi \, t}}\bigg]$
\end{frame}
\begin{frame}
\frametitle{Solución al Ejemplo 7}
Solución:
\pause
\begin{eqnarray*}
\begin{aligned}
&L \bigg[ \dfrac{1}{\sqrt{\pi \, t}} \bigg] = \scaleint{6ex}_{\bs 0}^{\infty} e^{- p t} \, \dfrac{1}{\sqrt{\pi \, t}} \dd{t} = \\[0.5em] \pause
&= \dfrac{\sqrt{p}}{\sqrt{\pi}} \scaleint{6ex}_{\bs 0}^{\infty} e^{-x} \, x^{1/2 - 1} \dfrac{\dd{x}}{p} = \\[0.5em] \pause
&= \dfrac{1}{\sqrt{\pi} \, p} \Gamma \left( \dfrac{1}{2} \right) = \\[0.5em] \pause
&= \dfrac{\sqrt{\pi}}{\sqrt{\pi \, p}} = \dfrac{1}{\sqrt{p}}
\end{aligned}
\end{eqnarray*}
\end{frame}
\begin{frame}
\frametitle{Ejemplo 8}
Evaluar $L \big[  t^{\nu} \, e^{-a t}  \big]$, si $\Re{\nu + 1} > 0$
\\
\bigskip
\pause
Solución: Se sabe que:
\pause
\begin{align*}
L \big[  t^{\nu}; t \to p  \big] = \dfrac{\Gamma (\nu + 1)}{p^{\nu+1}}
\end{align*}
\end{frame}
\begin{frame}
\frametitle{Solución al Ejemplo 8}
Entonces, ocupando el teorema del desplazamiento, obtenemos:
\pause
\begin{align*}
L \big[  t^{\nu} \, e^{-a t}; t \to p  \big] = \dfrac{\Gamma (\nu + 1)}{(p + a)^{\nu+1}}
\end{align*}
\end{frame}
\begin{frame}
\frametitle{Ejemplo 9}
Evalúa $L \big[ \sin (t - a) \, H (t- a); t \to p  \big]$.
\\
\bigskip
\pause
Para luego evaluar: $L \big[  e^{(t-a) k} \, \sin (t - a) \, H(t- a); t \to p  \big]$
\end{frame}
\begin{frame}
\frametitle{Solución al Ejemplo 9}
Sabemos que:
\pause
\begin{align*}
L \big[  \sin t; t \to p  \big] = \dfrac{1}{p^{2} + 1}
\end{align*}
\pause
Por lo que ocupando el segundo teorema del desplazamiento, se obtiene:
\pause
\begin{align*}
L \big[  \sin (t - a) \, H (t- a)  \big] = \dfrac{e^{-p a}}{(1 + p^{2})}
\end{align*}
\end{frame}
\begin{frame}
\frametitle{Solución al Ejemplo 9}
Por lo que:
\pause
\begin{align*}
L \big[  e^{(t-a) k} \sin (t {-} a) \, H (t {-} a); t \to p \big] = \dfrac{e^{-(p - k)}}{\big[  1 {+} (p {-} k^{2})  \big]}
\end{align*}
después de haber utilizado el primer teorema del desplazamiento.
\end{frame}

\subsection{La TL de la derivada de una función}

\begin{frame}
\frametitle{Teorema}
Sea $f (t)$ una función continua de $t \geq 0$ y es de orden exponencial para un valor de $t$ grande y si $\pderivada{f}(t)$ es una función continua a tramos para $t \geq 0$.
\end{frame}
\begin{frame}
\frametitle{Teorema}
Entonces la TL de la derivada $\pderivada{f}(t)$ existe cuando $p > \sigma$ y está dada por:
\pause
\begin{align*}
L \big[  \pderivada{f} (t)  \big] = p \, L \big[  f (t)  \big] - f(0)
\end{align*}
\end{frame}
\begin{frame}
\frametitle{Demostración del Teorema}
Por definición de la TL:
\pause
\begin{eqnarray}
\begin{aligned}[b]
&L \big[  \pderivada{f}(t) \big] = \scaleint{6ex}_{\bs 0}^{\infty} e^{-p t} \, \pderivada{f} (t) \dd{t} = \\[0.5em] \pause
&= \big[ e^{-p t} \cdot f (t) \big] \eval_{0}^{\infty} - \scaleint{6ex}_{\bs 0}^{\infty} (-p) \, e^{-p t} \ f (t) \dd{t} = \\[0.5em] \pause
&= \lim_{t \to \infty} e^{-p t} \, f (t) - f(0) + p \, L \big[ f (t) \big]
\end{aligned}
\label{eq:ecuacion_03_10}
\end{eqnarray}
\end{frame}
\begin{frame}
\frametitle{Demostración del Teorema}
Como $f (t)$ es de orden exponencial $\sigma$ para $t \to \infty$, se tiene:
\pause
\begin{align*}
\abs{f (t)} \leq M \, e^{-\sigma t}, \hspace{0.5em} t \geq 0
\end{align*}
\pause
Por lo que:
\pause
\begin{align*}
\abs{f (t) \, e^{-p t}} = e^{-p t} \, \abs{f (t)} \leq M \, e^{-p t} \cdot e^{\sigma t} = M \, e^{-(p - \sigma) t}
\end{align*}
\end{frame}
\begin{frame}
\frametitle{Demostración del Teorema}
Así pues:
\pause
\begin{align*}
\lim_{t \to \infty} e^{-p t} \, f (t) = 0 \hspace{0.5cm} \mbox{mientras  } p - \sigma > 0
\end{align*}
\pause
Entonces, de la ec. (\ref{eq:ecuacion_03_10}), llegamos a:
\pause
\begin{align}
L \big[  \pderivada{f} (t)  \big] = p \, L \big[  f (t)  \big] - f (0)
\label{eq:ecuacion_03_11}
\end{align}
\end{frame}
\begin{frame}
\frametitle{Corolario 1 al Teorema}
Si $\sderivada{f} (t)$ existe para $t \geq 0$ y es una función continua en tramos, \pause entonces siguiendo el mismo procedimiento anterior, es posible extender el resultado del teorema como:
\pause
\begin{eqnarray}
\begin{aligned}[b]
&L \big[  \sderivada{f} (t)  \big] = p \, L \big[  \pderivada{f} (t)  \big] - \pderivada{f} (0) = \\[0.5em] \pause
&= p \, \big[  p \left\{ L (f (t)) \right\} - f (0)  \big] - \pderivada{f} (0) = \\[0.5em] \pause
&= p^{2} \, L \big[  f (t)  \big] - p \, f (0) - \pderivada{f} (0)
\end{aligned}
\label{eq:ecuacion_03_12}
\end{eqnarray}
\end{frame}
\begin{frame}
\frametitle{Corolario 2 al Teorema}
En general, si $\nderivada{f}{n} (t)$ existe para $t \geq 0$ y es una función continua en tramos de $t$, entonces:
\pause
\begin{eqnarray}
\begin{aligned}[b]
&L \big[  \nderivada{f}{n} (t)  \big] = p^{n} \, L \big[  f (t)  \big] - p^{n-1} \, f (0) + \\[0.5em] \pause
&- p^{n-2} \, \pderivada{f}(0) - \ldots - \ntilde{f}{n-1} (0)
\end{aligned}
\label{eq:ecuacion_03_13}
\end{eqnarray}
\end{frame}

\subsection{TL de la integral de una función}

\begin{frame}
\frametitle{La transformada de la integral}
Si la TL de una función $f (t)$ es $\overline{f} (p)$. Entonces la transformada de la integral:
\pause
\begin{align}
\scaleint{6ex}_{\bs 0}^{t} f (\tau) \dd{\tau} = \dfrac{\overline{f} (p)}{p}
\label{eq:ecuacion_03_16}
\end{align}
\end{frame}

\subsection{La TL de una función periódica}

\begin{frame}
\frametitle{Una función periódica}
Sea $f (t)$ una función con período $\tau$, tal que:
\pause
\begin{align*}
f (t + n \, \tau) = f (t) \hspace{1cm} n = 1, 2, 3, \ldots
\end{align*}
\end{frame}
\begin{frame}
\frametitle{Una función periódica}
Si $f (t)$ es una función continua en tramos para $t > 0$, entonces:
\pause
\begin{align}
L \big[  f (t)  \big] = \dfrac{1}{1 - e^{-p \tau}} \scaleint{6ex}_{\bs 0}^{\tau} e^{-p t} \, f (t) \dd{t}
\end{align}
\end{frame}

\subsection{Convolución de dos funciones}


\begin{frame}
\frametitle{Definición de convolución}
Sean $f (t)$ y $g (t)$ dos funciones continuas por partes y son de algún orden exponencial para $t$ grande y para todo $t \geq 0$.
\end{frame}
\begin{frame}
\frametitle{Definición de convolución}
Entonces la \textocolor{cobalt}{convolución} de estas funciones se denota por $f * g (t)$ y se define por:
\pause
\begin{align}
f * g (t) = \scaleint{6ex}_{\bs 0}^{t} f (u) \, g (t - u) \dd{u}
\label{03_55}
\end{align}
Esta relación también se le conoce como \emph{faltung} de $f (t)$ y $g (t)$.
\end{frame}
\begin{frame}
\frametitle{Propiedades de la convolución}
Por definición:
\pause
\begin{eqnarray}
\begin{aligned}
f * g (t) &= \scaleint{6ex}_{\bs 0}^{t} f (u) \, g (t - u) \dd{u} = \\[0.5em] \pause
&= \scaleint{6ex}_{\bs 0}^{t} f (t - \tau) \, g (\tau) \dd{\tau} = \\[0.5em] \pause
&= g * f (t)
\end{aligned}
\label{eq:ecuacion_03_56}
\end{eqnarray}
\pause
Por lo tanto, la convolución de dos funciones satisface la ley conmutativa.
\end{frame}
\begin{frame}
\frametitle{Propiedades de la convolución}
Satisface también:
\pause
\setbeamercolor{item projected}{bg=byzantine,fg=white}
\setbeamertemplate{enumerate items}{%
\usebeamercolor[bg]{item projected}%
\raisebox{1.5pt}{\colorbox{bg}{\color{fg}\footnotesize\insertenumlabel}}%
}
\begin{enumerate}[<+->]
\item La ley distributiva:
\begin{align}
f * \big[  g + h  \big](t) = f * g (t) + f * h (t)
\label{eq:ecuacion_03_57}
\end{align}
\item La ley asociativa:
\begin{align}
\big[  f * (g * h)  \big](t) = \big[  (f * g) * h  \big] (t)
\label{eq:ecuacion_03_58}
\end{align}
\end{enumerate}
\end{frame}
\begin{frame}
\frametitle{Evaluando la convolución}
Ahora podemos enfocarnos a evaluar la TL de la convolución de dos funciones $f (t)$ y $g (t)$:
\pause
\begin{eqnarray*}
\begin{aligned}
&L \big[  (f * g) (t); t \to p  \big] =  \\[0.5em] \pause
&= L \left[ \scaleint{6ex}_{\bs 0}^{t} f (\tau) \, g (t - \tau) \dd{\tau}; t \to p \right] = \\[0.5em] \pause
&= \scaleint{6ex}_{\bs 0}^{\infty} e^{-p t} \, \left[ \scaleint{6ex}_{\bs 0}^{t} f (\tau) \, g (t - \tau) \dd{\tau}\right] \, \dd{t} =
\end{aligned}
\end{eqnarray*}
\end{frame}
\begin{frame}
\frametitle{Evaluando la convolución}
\begin{eqnarray}
\begin{aligned}[b]
&= \scaleint{6ex}_{\bs 0}^{\infty} f (\tau) \, \left[ \scaleint{6ex}_{\bs \tau}^{\infty} e^{-p t} \, g (t - \tau) \dd{t}\right] \, \dd{\tau} = \\[0.5em] \pause
&= \scaleint{6ex}_{\bs 0}^{\infty} f(\tau) \, e^{- p t} \, \left[ \scaleint{6ex}_{\bs 0}^{\infty} e^{-p \eta} \, g (\eta) \dd{\eta} \right] \, \dd{\tau} = \\[0.5em] \pause
&= \overline{f} (p) \, \overline{g} (p)
\end{aligned}
\label{eq:ecuacion_03_59}
\end{eqnarray}
\end{frame}
\begin{frame}
\frametitle{Evaluando la convolución}
En particular se tiene:
\pause
\begin{align*}
L \big[  f * f (t); t \to p  \big] = \big[  \overline{f} (p)  \big]^{2}
\end{align*}
\end{frame}

\section{La transformada inversa de Laplace}
\frame[allowframebreaks]{\tableofcontents[currentsection, hideothersubsections]}
\subsection{Introducción}

\begin{frame}
\frametitle{Tipo de función}
Si $f (t)$ pertenece a una clase $A$ (lo que significa que es una función continua por partes sobre $0 \leq t < \infty$ y es de algún orden exponencial), \pause entonces su Transformada de Laplace $\overline{f} (p)$ existe.
\end{frame}
\begin{frame}
\frametitle{Tipo de función}
Esto se denota simbólicamente por:
\pause
\begin{align}
L \big[  f (t); t \to p  \big] = \scaleint{6ex}_{\bs 0}^{\infty} e^{-p t} \, f (t) \dd{t} \equiv \overline{f} (p)
\label{ec:ecuacion_04_01}
\end{align}
\end{frame}
\begin{frame}
\frametitle{La transformada inversa}    
Entonces, a su vez $f (t)$, la función de objeto es la TL inversa de la función de imagen $\overline{f} (p)$ y está dada simbólicamente por:
\pause
\begin{align}
L^{-1} \big[  \overline{f} (t); p \to t  \big] \equiv f (t)
\label{ec:ecuacion_04_02}
\end{align}
\end{frame}
\begin{frame}
\frametitle{La transformada inversa}    
Así por la ec. (\ref{ec:ecuacion_04_01}), se puede evaluar $\overline{f} (p)$ para una $f (t)$ dada, ya que $\overline{f} (p)$ existe.
\end{frame}
\begin{frame}
\frametitle{La transformada inversa}    
Ahora consideraremos el problema inverso: \pause derivar información de la $\overline{f} (p)$ prescrita que nos permite obtener la función original $f (t)$ a través de alguna fórmula, llamada \textocolor{darkviolet}{fórmula de inversión de Laplace}.
\end{frame}
\begin{frame}
\frametitle{La transformada inversa}    
Antes de responder a este problema básico planteado anteriormente, \pause en primer lugar presentaremos el enfoque heurístico para determinar la transformada inversa de Laplace de alguna $\overline{f} (p)$.
\end{frame}
\begin{frame}
\frametitle{La transformada inversa}    
También, según sea necesario, definimos una nueva función, \pause llamada \textocolor{burgundy}{función nula}, en esta conexión junto con un teorema vinculado, conocido como \textocolor{auburn}{teorema de Lerch}.
\end{frame}
\begin{frame}
\frametitle{La función nula}
\textocolor{byzantium}{Función nula.} \pause Una función $N (t)$ que satisface la condición:
\pause
\begin{align*}
\scaleint{6ex}_{\bs 0}^{\tau} N (\tau) \dd{\tau} = 0 \hspace{0.5cm} \forall \, t > 0
\end{align*}
se denomina función nula.
\end{frame}
\begin{frame}
\frametitle{Teorema de Lerch}
Si $L \big[ f_{1} (t); t \to p \big], \Re {p} > c_{1}$ y $L \big[ f_{2} (t); t \to p \big], \Re {p} > c_{2}$ existen ambas, y si:
\pause
\begin{align*}
L \big[ f_{1} (t); t \to p \big] = L \big[ f_{2} (t); t \to p \big]
\end{align*}
Para $\Re{p} > c = \max{c_{1},c_{2}}$.
\end{frame}
\begin{frame}
\frametitle{Teorema de Lerch}
Entonces:
\pause
\begin{align*}
f_{2} (t) - f_{1} (t) = N_{t}
\end{align*}
\pause
Además, si $f_{1} (t)$ y $f_{2} (t)$ son continuas en toda una línea real, entonces $f_{1} (t) = f_{2} (t), \hspace{0.3cm} t > 0$.
\end{frame}

\subsection{Propiedades del TLI}

% \subsection{Linealidad de la transformada inversa.}
\begin{frame}
\frametitle{Linealidad}
Si:
\pause
\begin{align*}
\overline{f}_{1} (p) = L \big[ f_{1} (t); t \to p \big] \hspace{0.35cm} \mbox{y} \hspace{0.35cm} \overline{f}_{2} (p) = L \big[ f_{2} (t); t \to p \big]
\end{align*}
y $c{1}$ y $c_{2}$ dos constantes arbitrarias.
\end{frame}
\begin{frame}
\frametitle{Linealidad}
Entonces se tiene que:
\pause
\begin{eqnarray*}
\begin{aligned}
&L^{-1} \, \big[  c_{1} \, \overline{f}_{1} + c_{2} \, \overline{f}_{2} (p); p \to t \big] = \\[0.5em] \pause
&= c_{1} \, L^{-1} \, \big[  \overline{f}_{1}; p \to t \big] + c_{2} \, L^{-1} \, \big[  \overline{f}_{2}; p \to t \big] = \\[0.5em] \pause
&= c_{1} \, f_{1} (t) + c_{2} \, f_{2} (t)
\end{aligned}
\end{eqnarray*}
\end{frame}

\subsection{La TLI de algunas funciones}

\begin{frame}
\frametitle{Apoyo con resultados}
Antes de encontrar la TLI de funciones elementales que son un poco complejas por naturaleza, a continuación exponemos algunos resultados obtenidos previamente.
\\
\bigskip
\pause
Estos resultados pueden considerarse fórmulas para otras aplicaciones.
\end{frame}
\begin{frame}
\frametitle{Resultados de apoyo}
\setbeamercolor{item projected}{bg=carmine,fg=white}
\setbeamertemplate{enumerate items}{%
\usebeamercolor[bg]{item projected}%
\raisebox{1.5pt}{\colorbox{bg}{\color{fg}\footnotesize\insertenumlabel}}%
}
\begin{enumerate}[<+->]
\item
\begin{align*}
L \big[ H (t); t \to p \big] = \dfrac{1}{p} \\
\hspace{0.3cm} \Rightarrow \hspace{0.3cm} L^{-1} \bigg[ \dfrac{1}{p}; p \to t \bigg] = H (t)
\end{align*}
\item
\begin{align*}
L \big[ t^{\nu}; t \to p \big] = \dfrac{\Gamma(\nu + 1)}{p^{\nu + 1}} \\
\hspace{0.25cm} \Rightarrow \hspace{0.25cm} L^{-1} \bigg[ \dfrac{\Gamma(\nu + 1)}{p^{\nu + 1}}; p \to t \bigg] = t^{\nu}, \hspace{0.1cm} \Re{\nu} > - 1
\end{align*}
\seti
\end{enumerate}
\end{frame}
\begin{frame}
\frametitle{Resultados de apoyo}
\setbeamercolor{item projected}{bg=carmine,fg=white}
\setbeamertemplate{enumerate items}{%
\usebeamercolor[bg]{item projected}%
\raisebox{1.5pt}{\colorbox{bg}{\color{fg}\footnotesize\insertenumlabel}}%
}
\begin{enumerate}[<+->]
\conti
\item
\begin{align*}
 L \big[ e^{a t}; t \to p \big] = \dfrac{1}{p - a} \\[0.5em] 
\hspace{0.3cm} \Rightarrow \hspace{0.3cm} L^{-1} \bigg[ \dfrac{1}{p - a}; p \to t \bigg] = \\[0.5em]
&= e^{a t}, \hspace{0.2cm} \Re{p} > a
\end{align*}
\seti
\end{enumerate}
\end{frame}
\begin{frame}
\frametitle{Resultados de apoyo}
\setbeamercolor{item projected}{bg=carmine,fg=white}
\setbeamertemplate{enumerate items}{%
\usebeamercolor[bg]{item projected}%
\raisebox{1.5pt}{\colorbox{bg}{\color{fg}\footnotesize\insertenumlabel}}%
}
\begin{enumerate}[<+->]
\conti
\item
\begin{align*}
&L \big[ \sin (a \, t); t \to p \big] = \dfrac{a}{p^{2} + a^{2}} \\[0.5em] 
&\Rightarrow \hspace{0.2cm} L^{-1} \bigg[ \dfrac{a}{p^{2} + a^{2}}; p \to t \bigg] = \\[0.5em]
&= \sin (a \, t), \hspace{0.1cm} p > 0
\end{align*}
\seti
\end{enumerate}
\end{frame}
\begin{frame}
\frametitle{Resultados de apoyo}
\setbeamercolor{item projected}{bg=carmine,fg=white}
\setbeamertemplate{enumerate items}{%
\usebeamercolor[bg]{item projected}%
\raisebox{1.5pt}{\colorbox{bg}{\color{fg}\footnotesize\insertenumlabel}}%
}
\begin{enumerate}[<+->]
\conti
\item 
\begin{align*}
L \big[ \cos (a \, t); t \to p \big] = \dfrac{p}{p^{2} + a^{2}} \\[0.5em] 
\hspace{0.3cm} \Rightarrow \hspace{0.3cm} L^{-1} \bigg[ \dfrac{p}{p^{2} + a^{2}}; p \to t \bigg] {=} \cos (a \, t), \hspace{0.1cm} p > 0
\end{align*}
\end{enumerate}
\end{frame}



\begin{frame}
\frametitle{Manejando la inversión}
Ahora dirigiremos nuestra atención al discutir algunas reglas de manipulación de la inversión de Laplace de algunas combinaciones de funciones elementales de $p$ a través de los siguientes ejemplos.
\end{frame}
\begin{frame}
\frametitle{Ejemplo 1}
Evaluar:
\pause
\begin{align*}
L^{-1} \bigg[ \dfrac{1}{(p + 1)(p^{2} + 1)} ; p \to t \bigg]
\end{align*}
\end{frame}
\begin{frame}
\frametitle{Solución al Ejemplo 1}
Resolviendo mediante fracciones parciales siguiendo el método tradicional, se tiene que:
\pause
\begin{eqnarray*}
\begin{aligned}
\dfrac{1}{(p^{2} + 1)(p + 1)} \equiv \pause \dfrac{1}{2} \, \dfrac{1}{p + 1} - \dfrac{1}{2} \, \dfrac{p}{p^{2} + 1} + \dfrac{1}{2} \, \dfrac{1}{p^{2} + 1}
\end{aligned}
\end{eqnarray*}
\end{frame}
\begin{frame}
\frametitle{Solución al Ejemplo 1}
\begin{eqnarray*}
\begin{aligned}
&L^{-1} \bigg[ \dfrac{1}{(p + 1)(p^{2} + 1)} \bigg] = \\[0.5em] \pause
&= \dfrac{1}{2} \, L^{-1} \bigg[ \dfrac{1}{p + 1}\bigg] {-} \dfrac{1}{2} \, L^{-1} \bigg[\dfrac{p}{p^{2} + 1} \bigg] {+} \dfrac{1}{2} L^{-1} \, \bigg[\dfrac{1}{p^{2} + 1} \bigg] = \\[0.5em] \pause
&= \dfrac{1}{2} \, \big[  e^{-t} - \cos t + \sin t  \big]
\end{aligned}
\end{eqnarray*}
\end{frame}
\begin{frame}
\frametitle{Ejemplo 2}
Evaluar:
\pause
\begin{align*}
L^{-1} \left[ \dfrac{6 \, p^{2} + 22 \, p + 18}{p^{3} + 6 \, p^{2} + 11 \, p + 6} \right]
\end{align*}
\end{frame}
\begin{frame}
\frametitle{Solución al Ejemplo 2}
Tenemos entonces que:
\pause
\begin{eqnarray*}
\begin{aligned}
&L^{-1} \left[ \dfrac{6 \, p^{2} + 22 \, p + 18}{p^{3} + 6 \, p^{2} + 11 \, p + 6} \right] = \\[0.5em] \pause
&= L^{-1} \left[ \dfrac{6 \, p^{2} + 22 \, p + 18}{(p + 1)(p + 2)(p + 3)} \right] = \\[0.5em] \pause
&= L^{-1} \left[ \dfrac{1}{p + 1} + \dfrac{2}{p + 2} + \dfrac{3}{p + 3} \right] = \\[0.5em] \pause
&= e^{t} + 2 \, e^{2 t} + 3 \, e^{- 3 t}
\end{aligned}
\end{eqnarray*}
\end{frame}

\section{Expansión fracciones parciales}
\frame{\tableofcontents[currentsection, hideothersubsections]}
\subsection{El método de expansión}

\begin{frame}
\frametitle{Definiendo el método}
Sean $f (p)$ y $g (p)$ dos polinomios en $p$ tales que el grado de $f (p)$ es menor que el de $g (p)$.
\\
\bigskip
\pause
Entonces $f (p) / g (p)$ se puede expresar como:
\pause
\begin{align*}
\dfrac{f (p)}{g (p)} = \nsum_{r=1}^{n} \dfrac{A_{r}}{p - a_{r}} 
\end{align*}
donde  $p - a_{r}$  son los factores de $g (p)$.
\end{frame}
\begin{frame}
\frametitle{Definiendo el método}
Donde:
\pause
\begin{align*}
A_{r} &= \lim_{p \to a_{r}} \dfrac{(p - a_{r}) \, f (p)}{g (p)} = \dfrac{f(a_{r})}{\pderivada{g} (a_{r})} \\[0.5em]
&\mbox{para  } r = 1, 2, \ldots, n
\end{align*}
\end{frame}
\begin{frame}
\frametitle{Ejemplo 3}
Evalúa:
\pause
\begin{align*}
L^{-1} \left[ \dfrac{p + 5}{(p + 1)(p^{2} + 1)} \right]
\end{align*}
\end{frame}
\begin{frame}
\frametitle{Solución al Ejemplo 3}
Al expresar en términos de fracciones parciales, se tiene:
\pause
\begin{eqnarray*}
\begin{aligned}
&\left[ \dfrac{p + 5}{(p + 1)(p^{2} + 1)} \right] \equiv \dfrac{A_{1}}{p + 1} + \dfrac{A_{2}}{p + 1} + \dfrac{A_{3}}{p - i} = \\[0.5em] \pause
&= \dfrac{2}{p -1} + \dfrac{\dfrac{i - 5}{2(1 + i)}}{p + i} + \dfrac{\dfrac{i + 5}{2(i - 1)}}{p - i} = \\[0.5em] \pause
&= \dfrac{2}{p + 1} + \dfrac{-2 \, p}{p^{2 + 1}} + \dfrac{3}{p^2 + 1}
\end{aligned}
\end{eqnarray*}
\end{frame}
\begin{frame}
\frametitle{Solución al Ejemplo 3}
Por lo tanto:
\pause
\begin{align*}
L^{-1} \left[ \dfrac{p + 5}{(p + 1)(p^{2} + 1)} \right] = 2 \, e^{-t} - 2 \, \cos t +  3 \, \sin t
\end{align*}
\end{frame}
\begin{frame}
\frametitle{Ejemplo 4}
Evalúa:
\pause
\begin{align*}
L^{-1} \left[ \dfrac{4 \, p + 5}{(p - 1)^{2} \, (p + 2)} \right]
\end{align*}
\end{frame}
\begin{frame}
\frametitle{Solución al Ejemplo 4}
Por el método de fracciones parciales, se tiene:
\pause
\begin{align*}
\dfrac{4 \, p + 5}{(p - 1)^{2} \, (p + 2)} \equiv \dfrac{A}{p - 1} + \dfrac{B}{(p - 1)^{2}} + \dfrac{C}{p + 2}
\end{align*}
\end{frame}
\begin{frame}
\frametitle{Solución al Ejemplo 4}
Entonces:
\pause
\begin{align*}
4 \, p + 5 \equiv A \, (p - 1)(p + 2) + B \, (p + 2) + C \, (p - 1)^{2}
\end{align*}
\end{frame}
\begin{frame}
\frametitle{Solución al Ejemplo 4}
De esta identidad tenemos:
\pause
\begin{eqnarray*}
\begin{aligned}
&\mbox{haciendo que } p = 1 \hspace{0.3cm} \pause \Rightarrow \pause 9 = 3 \, B \hspace{0.2cm} \pause \Rightarrow \hspace{0.2cm} B = 3 \\[0.5em] \pause
&\mbox{haciendo que } p = -2 \hspace{0.3cm} \pause \Rightarrow - 3 = 9 \, C \hspace{0.2cm} \pause \Rightarrow \hspace{0.2cm} C = -\dfrac{1}{3} \\[0.5em] \pause
&\mbox{igualando el término independiente de p}  \\[0.5em] \pause
&\Rightarrow \hspace{0.3cm} 5 = -2 \, A {+}  2 \, B {+} C \\[0.5em] \pause
&\Rightarrow \hspace{0.3cm} A = \dfrac{1}{3}
\end{aligned}
\end{eqnarray*}
\end{frame}
\begin{frame}
\frametitle{Solución al Ejemplo 4}
Por lo tanto:
\pause
\begin{eqnarray*}
\begin{aligned}
&L^{-1} \left[ \dfrac{4 \, p + 5}{(p - 1)^{2} \, (p + 2)} \right] = \\[0.5em] \pause
&= \dfrac{1}{3} \, e^{t} +  3 \, L^{-1} \left[\dfrac{1}{(p - 1)^{2}} \right] - \dfrac{1}{3} \, e^{-2 t} = \\[0.5em] \pause
&= \dfrac{1}{3} \, e^{t} - \dfrac{1}{3} \, e^{- 2 t} + 3 \, t \, e^{-t}
\end{aligned}
\end{eqnarray*}
\end{frame}

\section{Aplicaciones TL}
\frame{\tableofcontents[currentsection, hideothersubsections]}
\subsection{Solución de EDO}

\begin{frame}
\frametitle{Solución a EDO con coeficientes constantes}
Supongamos que deseamos resolver una EDO lineal de orden $n$:
\pause
\begin{align}
\dv[n]{y}{t} + c_{1} \, \dv[n-1]{y}{t} + \ldots + c_{n-1} \, \dv{y}{t} + c_{n} \, y = 0
\label{eq:ecuacion_04_54}
\end{align}
donde $c_{i} = 1, 2, \ldots, n$ son constantes dadas.
\end{frame}
\begin{frame}
\frametitle{Solución a EDO con coeficientes constantes}
Sujetas a las condiciones iniciales:
\pause
\begin{align*}
y(0) = k_{1}, \, \pderivada{y}(0) = k_{2} , \ldots, \ntilde{y}{n-1}(0)= k_{n}
\end{align*}
\end{frame}
\begin{frame}
\frametitle{Solución a EDO con coeficientes constantes}
Hacemos lo siguiente:
\setbeamercolor{item projected}{bg=carnelian,fg=white}
\setbeamertemplate{enumerate items}{%
\usebeamercolor[bg]{item projected}%
\raisebox{1.5pt}{\colorbox{bg}{\color{fg}\footnotesize\insertenumlabel}}%
}
\begin{enumerate}[<+->]
\item Tomamos la TL en ambos lados de la EDO.
\item Utilizamos los resultados y propiedades revisados previamente.
\item Así como las condiciones iniciales dadas en la EDO.
\end{enumerate}
\end{frame}
\begin{frame}
\frametitle{Solución a EDO con coeficientes constantes}
Se tendrá que:
\pause
\begin{eqnarray*}
\begin{aligned}
&\big[ \overline{y}(p) \, p^{n} - p^{n-1} \, k_{1} - \ldots - p \, k_{n-1} - k_{n}  \big] + \\[0.5em] 
&+ c_{1} \, \big[ \overline{y}(p) \, p^{n-1} - p^{n-2} \, k_{1} - \ldots - k_{n-1} - k_{n}  \big] + \\[0.5em] 
&+ \ldots + c_{n-1} \big[ p \, \overline{y}(p) - k_{1}  \big] + c_{n} \, \overline{y}(p) = \\[0.5em] \pause
&= \overline{f}(p)
\end{aligned}
\end{eqnarray*}
\end{frame}
\begin{frame}
\frametitle{Solución a EDO con coeficientes constantes}
Por lo tanto:
\pause
\begin{eqnarray*}
\begin{aligned}
&\overline{y}(p) \big[ p^{n} + c_{1} \, p^{n-1} \, + \ldots + c_{n}  \big] = \overline{f}(p) + \\[0.5em] \pause
&+ k_{1} \, \big[ p^{n-1} + c_{1} \, p^{n-2} + \ldots + c_{n-1} \, p \big] + \\[0.5em] \pause
&+ k_{2} \, \big[ p^{n-2} + c_{1} \, p^{n-3} + \ldots  \big] +  \ldots + \\[0.5em] \pause
&+ k_{n-1} \big[ p + c_{1}  \big] + k_{n} \hspace{0.5cm} \Rightarrow
\end{aligned}
\end{eqnarray*}
\end{frame}
\begin{frame}
\frametitle{Solución a EDO con coeficientes constantes}
Lo que implica:
\begin{eqnarray*}
\begin{aligned}
&\overline{y}(p) \big[ p^{n} + c_{1} \, p^{n-1} + \cdots + c_{n}  \big] = \overline{f}(p) + \\[0.5em] 
&+ k_{1} \big[ p^{n-1} + c_{1} \, p^{n-2} + \cdots c_{n-1} \, p \big] + \\[0.5em]
&+ k_{2} \big[ p^{n-2} + c_{1} \, p^{n-3} + \cdots \big] + \cdots + k_{n-1} \big[  p + c_{1} \big] + k_{n} 
\end{aligned}
\end{eqnarray*}
\end{frame}
\begin{frame}
\frametitle{Solución a EDO con coeficientes constantes}
Lo que implica:
\begin{eqnarray}
\begin{aligned}
&\Rightarrow \hspace{0.2cm} \overline{y}(p) = \dfrac{\big[ \overline{f}(p) + \overline{g}(p)  \big]}{\big[ p^{n}+ c_{1} \, p^{n-1} + \ldots + c_{n} \big]} \\[0.5em] \pause
&\overline{y}(p) = \dfrac{\overline{f}(p)}{h(p)} + \dfrac{\overline{g}(p)}{h(p)} \\[0.5em]
&\mbox{con } \overline{f} (p) = L \big[ f (t) \big]
\end{aligned}
\label{eq:ecuacion_04_55}
\end{eqnarray}
donde $\overline{g}(p)$ es un polinomio de grado $n-1$ en $p$ y $h(p)$ es también un polinomio de grado $n$ en $p$.
\end{frame}
\begin{frame}
\frametitle{Solución a EDO con coeficientes constantes}
Por lo tanto, la solución de la EDO en las condiciones iniciales dadas se puede obtener después de tomar la TLI de la última ecuación (\ref{eq:ecuacion_04_55}).
\end{frame}
\begin{frame}
\frametitle{Solución a EDO con coeficientes constantes}
Dado que el lado derecho es una función conocida, tenemos:
\pause
\begin{align}
y(t) = L^{-1} \left[ \dfrac{\overline{f}(p)}{h(p)} \right] + L^{-1} \left[ \dfrac{\overline{g}(p)}{h(p)} \right]
\label{eq:ecuacion_04_56}
\end{align}
\end{frame}
\begin{frame}
\frametitle{Solución a EDO con coeficientes constantes}
Si la ec. (\ref{eq:ecuacion_04_54}) es homogénea, \pause entonces $f (t) \equiv 0$ y, por tanto, la solución de la EDO homogénea con condiciones iniciales dadas es:
\pause
\begin{align}
y(t) = L^{-1} \left[ \dfrac{\overline{g}(p)}{h(p)} \right]
\label{eq:ecuacion_04_57}
\end{align}
\end{frame}

\subsection*{Ejemplo con una EDO}

\begin{frame}
\frametitle{Ejemplo con una EDO}
Resuelve usando la TL el problema de valores iniciales:
\pause
\begin{align*}
\tderivada{y} (t) +  2 \, \sderivada{y} (t) - \pderivada{y}(t) - 2 \, y(t) = 0
\end{align*}
con las condiciones iniciales: $y(0) = \pderivada{y}(0) = 0$ y $\sderivada{y} = 6$
\end{frame}
\begin{frame}
\frametitle{Solución a la EDO}
Aplicando la TL en ambos lados de la EDO con las condiciones iniciales dadas, se tiene que:
\pause
\begin{align*}
&\big[ p^{3} \, \overline{y}(p) - p^{2} \, y(0) - p \, \pderivada{y}(0) - \sderivada{y}(0)  \big] + \\[0.5em]
&+ 2 \, \big[ p^{2} \, \overline{y}(p) - p \, y(0) - \pderivada{y}(0)  \big] + \\[0.5em]
&- \big[ p \, \overline{y}(p) - y(0)  \big] - 2 \, \overline{y}(p) = 0 \\[1em]
\end{align*}
\end{frame}
\begin{frame}
\frametitle{Solución a la EDO}    
\begin{eqnarray*}
\begin{aligned}
&\Rightarrow \overline{y}(p) = \dfrac{6}{\big[ p^{3} - 2 \, p^{2} - p -2  \big]} \\[0.5em] \pause
&\Rightarrow \overline{y}(p) = \dfrac{6}{(p - 1)(p + 1)(p +2)} \\[0.5em] \pause
&\equiv \dfrac{1}{p - 1} - \dfrac{3}{p + 1} + \dfrac{2}{p + 2}  
\end{aligned}
\end{eqnarray*}
\end{frame}
\begin{frame}
\frametitle{Solución a la EDO}
Entonces al aplicar la TLI en ambos lados de la expresión, se obtiene:
\pause
\begin{align*}
y (t) = e^{t} - 3 \, e^{-t} + 2 \, e^{- 2 t}
\end{align*}
\end{frame}
\begin{frame}
\frametitle{Ejemplo 2 con otra EDO}
Resuelve el siguiente problema de valores iniciales, definido por:
\pause
\begin{align*}
\left[ \dv[2]{t} + n^{2}\right] \, x (t) = a \, \sin (n \, t +  \alpha) \hspace{0.5cm} x(0) = \pderivada{x}(0) = 0
\end{align*}
\end{frame}
\begin{frame}
\frametitle{Solución a la EDO}
Aplicando la TL en ambos lados de la EDO con las condiciones iniciales dadas, se tiene que:
\pause
\begin{eqnarray*}
\begin{aligned}
&(p^{2} + n^{2}) \, \overline{x}(p) = a \, \cos \alpha \, \dfrac{n}{p^{2} + n^{2}} +  a \, \sin \alpha \, \dfrac{p}{p^{2} + n^{2}} \\[0.5em] \pause
&\Rightarrow \overline{x}(p) = a \, n \, \cos \alpha \, \dfrac{1}{(p^{2} + n^{2})^{2}} + a \, n \, \sin \alpha \, \dfrac{p}{(p^{2} + n^{2})^{2}}
\end{aligned}
\end{eqnarray*}
\end{frame}
\begin{frame}
\frametitle{Solución a la EDO}
Aplicando ahora la TLI, se llega a:
\pause
\begin{eqnarray*}
\begin{aligned}
x (t) &= a \, n \, \cos \alpha \, \dfrac{1}{2 \, n^{3}} \, \big[ \sin n \, t - n \, t \, \cos n \, t  \big] + \\[0.5em] 
&+ a \, \sin \alpha \, \dfrac{t}{2 \, n} \, \sin n \, t = \\[0.5em] \pause
&= \dfrac{a \big[ \sin n \, t \, \cos \alpha - n \, t \, \cos (n \, t +  \alpha)]}{2 \, n^{2}}
\end{aligned}
\end{eqnarray*}
\end{frame}
\begin{frame}
\frametitle{Ejemplo 3 con otra EDO}
Se aplica un voltaje $E \, e^{- a t}$ al tiempo $t = 0$ a un circuito de inductancia $L$ y resistencia $R$ conectadas en serie, donde $a$, $E$, $L$, y $R$ son constantes.
\\
\bigskip
\pause
\textocolor{red}{Determina la corriente en cualquier tiempo, es decir: $i(t)$.}
\end{frame}
\begin{frame}
\frametitle{Solución a la EDO}
En un circuito eléctrico en donde se tiene un voltaje $E(t)$, una resistencia $R$ e inductancia $L$, la corriente $i(t)$ en el circuito está dada por:
\pause
\begin{align*}
L \, \dv{i}{t} + R \, i(t) =  E(t)
\end{align*}
\pause
En este caso: $E(t) = E \, e^{-a t}$ y para el $t = 0$, $i(0) = 0$.
\end{frame}
\begin{frame}
\frametitle{Solución a la EDO}
Al aplicar la TL a la ecuación de la corriente, con las condiciones dadas, se tiene que:
\pause
\begin{eqnarray*}
\begin{aligned}
&L \big[  p \, \overline{i}(p) - 0  \big] + R \, \overline{i} (p) = E \, \dfrac{1}{p + a} \\[0.5em] \pause
&\Rightarrow \overline{i}(p) = \dfrac{E}{(L \, p + R)(p + a)} 
\end{aligned}
\end{eqnarray*}
\end{frame}
\begin{frame}
\frametitle{Solución a la EDO}
\begin{eqnarray*}
\begin{aligned}
&\equiv \dfrac{E}{L \, \left(a - \dfrac{R}{L} \right)} \, \left[ \dfrac{1}{p + \dfrac{R}{L}}  - \dfrac{1}{p + a}\right] \\[0.5em] \pause
&= \dfrac{E}{R - a \, L} \left[ \dfrac{1}{p + a} - \dfrac{1}{p + \dfrac{R}{L}}\right]
\end{aligned}
\end{eqnarray*}
\end{frame}
\begin{frame}
\frametitle{Solución a la EDO}
Aplicando la TLI:
\begin{align*}
i (t) = \dfrac{E}{R - a \, L} \, \big[  e^{-a t} - e^{-R t /L} \big]
\end{align*}
\end{frame}

\subsection{EDO simultáneas}

\begin{frame}
\frametitle{Sistema de EDO con coeficientes constantes}
La EDO que involucra más de una variable dependiente pero con una sola variable independiente da lugar a ecuaciones simultáneas.
\\
\bigskip
\pause
El procedimiento para resolver tales ecuaciones simultáneas es casi el mismo que se discutió previamente. 
\end{frame}
\begin{frame}
\frametitle{Solución al sistema de EDO}
Aquí también, tenemos que tomar la TL de las ecuaciones simultáneas para reducirlas al número correspondiente de ecuaciones algebraicas que luego pueden resolverse para las variables dependientes TL.
\end{frame}
\begin{frame}
\frametitle{Solución al sistema de EDO}
Finalmente invirtiendo estas relaciones podemos recuperar las variables dependientes formando las soluciones requeridas.
\end{frame}
\begin{frame}
\frametitle{Ejemplo sistema de EDO}
Resolvamos el siguiente problema de valores iniciales definido por las EDO simultáneas:
\pause
\begin{align*}
\dv{x}{t} - y &= e^{t} \hspace{1.5cm} x(0) =  1 \\[0.5em]
\dv{y}{t} + x &= \sin t \hspace{1.5cm} y(0) =  0
\end{align*}
\end{frame}
\begin{frame}
\frametitle{Solución al sistema de EDO}
Aplicando la TL de las dos EDO con las condiciones iniciales indicadas, se tiene que:
\pause
\begin{eqnarray*}
\begin{aligned}
p \, \overline{x} (p) - 1 - \overline{y} (p) = \dfrac{1}{p -1} \\[0.5em] \pause
p \, \overline{y} (p) + \overline{x} (p) = \dfrac{1}{p^{2} -1}
\end{aligned}
\end{eqnarray*}
\end{frame}
\begin{frame}
\frametitle{Solución al sistema de EDO}
Resolviendo éstas dos ecuaciones para $\overline{x}(p)$ y $\overline{y}(p)$, llegamos a:
\pause
\begin{align*}
\overline{x}(p) = \dfrac{p}{p^{2} + 1} + \dfrac{p}{(p - 1)(p^{2} + 1)} + \dfrac{1}{(p^{2} + 1)^{2}}
\end{align*}
\end{frame}
\begin{frame}
\frametitle{Solución al sistema de EDO}
Usando la TLI, se tiene que:
\pause
\begin{align*}
x (t) = \dfrac{1}{2} \big[  \cos t + 2 \, \sin t +  e^{t} - t \, \cos t  \big]
\end{align*}
\pause
De la misma manera se llega a:
\pause
\begin{align*}
\overline{y} (p) = - 1 - \dfrac{1}{p - 1} + p \, \overline{x} (p)
\end{align*}
\end{frame}
\begin{frame}
\frametitle{Solución al sistema de EDO}
Que al aplicar la TLI y simplificando la expresión, tenemos que:
\pause
\begin{align*}
y (t)= \dfrac{1}{2} \big[  t \, \sin t - e^{t} + \cos t - \sin t  \big]
\end{align*}
\end{frame}

% \section{Ejercicios a cuenta.}

% \noindent
% \textbf{Ejercicio a cuenta (60). } Calcula la TL de la siguiente función:
% \begin{align*}
% f (t) = \begin{cases}
% 0, & 0 < t < 1 \\
% (t - 1)^{2}, & t > 1
% \end{cases}
% \end{align*}
% \\[0.5em]
% \noindent
% \textbf{Ejercicio a cuenta (61). } Una función periódica $f (t)$ de período $2 \, \pi$ tiene una discontinuidad finita en $t = \pi$, la función está dada por:
% \begin{align*}
% f (t) = \begin{cases}
% \sin t, & 0 \leq t \leq \pi \\
% \cos t, & \pi < t \leq 2 \pi
% \end{cases}
% \end{align*}
% Calcula su TL.
% \\[0.5em]
% \noindent
% \textbf{Ejercicio a cuenta (62). } Ocupando la TL resuelve la siguiente ecuación diferencial:
% \begin{align*}
% \dv[2]{x}{t} + 9 \, x = \cos 2 \, t
% \end{align*}
% sujeta a las siguientes condiciones $x (0) =  1$, $x \bigg( \dfrac{\pi}{2} \bigg) = -1$
\end{document}