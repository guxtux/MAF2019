\documentclass[hidelinks,12pt]{article}
\usepackage[left=0.25cm,top=1cm,right=0.25cm,bottom=1cm]{geometry}
%\usepackage[landscape]{geometry}
\textwidth = 20cm
\hoffset = -1cm
\usepackage[utf8]{inputenc}
\usepackage[spanish,es-tabla]{babel}
\usepackage[autostyle,spanish=mexican]{csquotes}
\usepackage[tbtags]{amsmath}
\usepackage{nccmath}
\usepackage{amsthm}
\usepackage{amssymb}
\usepackage{mathrsfs}
\usepackage{graphicx}
\usepackage{subfig}
\usepackage{standalone}
\usepackage[outdir=./Imagenes/]{epstopdf}
\usepackage{siunitx}
\usepackage{physics}
\usepackage{color}
\usepackage{float}
\usepackage{hyperref}
\usepackage{multicol}
%\usepackage{milista}
\usepackage{anyfontsize}
\usepackage{anysize}
%\usepackage{enumerate}
\usepackage[shortlabels]{enumitem}
\usepackage{capt-of}
\usepackage{bm}
\usepackage{relsize}
\usepackage{placeins}
\usepackage{empheq}
\usepackage{cancel}
\usepackage{wrapfig}
\usepackage[flushleft]{threeparttable}
\usepackage{makecell}
\usepackage{fancyhdr}
\usepackage{tikz}
\usepackage{bigints}
\usepackage{scalerel}
\usepackage{pgfplots}
\usepackage{pdflscape}
\pgfplotsset{compat=1.16}
\spanishdecimal{.}
\renewcommand{\baselinestretch}{1.5} 
\renewcommand\labelenumii{\theenumi.{\arabic{enumii}})}
\newcommand{\ptilde}[1]{\ensuremath{{#1}^{\prime}}}
\newcommand{\stilde}[1]{\ensuremath{{#1}^{\prime \prime}}}
\newcommand{\ttilde}[1]{\ensuremath{{#1}^{\prime \prime \prime}}}
\newcommand{\ntilde}[2]{\ensuremath{{#1}^{(#2)}}}

\newtheorem{defi}{{\it Definición}}[section]
\newtheorem{teo}{{\it Teorema}}[section]
\newtheorem{ejemplo}{{\it Ejemplo}}[section]
\newtheorem{propiedad}{{\it Propiedad}}[section]
\newtheorem{lema}{{\it Lema}}[section]
\newtheorem{cor}{Corolario}
\newtheorem{ejer}{Ejercicio}[section]

\newlist{milista}{enumerate}{2}
\setlist[milista,1]{label=\arabic*)}
\setlist[milista,2]{label=\arabic{milistai}.\arabic*)}
\newlength{\depthofsumsign}
\setlength{\depthofsumsign}{\depthof{$\sum$}}
\newcommand{\nsum}[1][1.4]{% only for \displaystyle
    \mathop{%
        \raisebox
            {-#1\depthofsumsign+1\depthofsumsign}
            {\scalebox
                {#1}
                {$\displaystyle\sum$}%
            }
    }
}
\def\scaleint#1{\vcenter{\hbox{\scaleto[3ex]{\displaystyle\int}{#1}}}}
\def\bs{\mkern-12mu}


%\usepackage{showframe}
\title{Transformadas de Laplace \\ \large {Tema 6 - Transformadas integrales} \vspace{-3ex}}
\author{M. en C. Gustavo Contreras Mayén}
\date{ }
\begin{document}
\vspace{-4cm}
\maketitle
\fontsize{14}{14}\selectfont
\tableofcontents
\newpage

\section{Introducción.}

En el material de trabajo sobre la transformada de Fourier, se revisó que si la integral
\begin{align*}
\int_{-\infty}^{\infty} \abs{f(t)} \dd{t}
\end{align*}
no converge, la transformada de Fourier $F(\xi)$ no existe para todo valor real de $\xi$. Por ejemplo: si $f(t) = \sin \omega \, t$ que es un real, la $F(\xi)$ no existe. Pero tales situaciones surgen ocasionalmente en la práctica. Para manejar esta situación, consideramos una nueva función $f_{1} (t)$ conectada a $f (t)$ definida por:
\begin{align*}
f_{1}(t) =  \exp(- \gamma \, t) \, f(t) \, H(t)
\end{align*}
donde $\gamma$ es una constante arbitraria real positiva y $H(t)$ es la función de paso unitario de Heaviside. Claramente se tiene que $f_{1}(t) \in A_{1}(R) \, \forall \, t \in R$ y por lo tanto, la transformada de Fourier de $f_{1}(t)$ existe, ya que:
\begin{align*}
\int_{-\infty}^{+\infty} f_{1} \dd{t} = \int_{0}^{\infty} \exp(-\gamma \, t) \, f(t) \, \dd{t}
\end{align*}
es convergente. De hecho, en este caso, del teorema de la integral de Fourier:
\begin{align*}
f_{1} (t) = \dfrac{1}{2 \, \pi} \int_{-\infty}^{\infty} \exp(-i \, \xi \, t) \dd{\xi} \, \int_{-\infty}^{+\infty} f_{1}(u) \, \exp(i \, \xi \, u) \dd{u}
\end{align*}
implica que:
\begin{align*}
f (t) = \dfrac{1}{2 \, \pi} \int_{-\infty}^{\infty} \exp(-i \, \xi \, t) \dd{\xi} \cdot \exp{\gamma \, t} \int_{0}^{+\infty} f(u) \, \exp(- (\gamma - i \, \xi) \, u) \dd{u}
\end{align*}
Si escribimos $p = - i \, \xi$, la relación anterior puede expresarse como:
\begin{align}
\begin{aligned}[b]
f (t) &= \dfrac{1}{2 \, \pi \, i} \int_{\gamma -i \, \infty}^{\gamma + i \, \infty} \exp(p \, t) \bigg[ \int_{0}^{+\infty} f(u) \, \exp(- p \, u) \dd{u} \bigg] \dd{p} = \\[0.5em]
&= \dfrac{1}{2 \, \pi \, i} \int_{\gamma -i \, \infty}^{\gamma + i \, \infty} \exp(p \, t) \, \overline{f} (p) \dd{p} 
\end{aligned}
\label{eq:ecuacion_03_01}
\end{align}
donde:
\begin{align}
\overline{f} (p) = \int_{0}^{+\infty} f(u) \, \exp(- p \, u) \dd{u}, \hspace{1cm} \Re{p} = \gamma > 0
\label{eq:ecuacion_03_02}
\end{align}
Las ecs. (\ref{eq:ecuacion_03_01}) y (\ref{eq:ecuacion_03_02}) constituyen una transformada con $K(p, u) = \exp(- p \, u)$ como núcleo o kernel.

\section{Definiciones.}

Definimos la transformada de Laplace de una función continua en tramos de variable real $t$, definida en un semieje $t \geq 0$ como:
\begin{align}
L \big[f(t); t \to p\big] = L \big[f(t)\big] = F(p) = \overline{f}(p) = \int_{0}^{\infty} f(t) \, \exp(-p \, t) \dd{t}
\label{eq:ecuacion_03_03}
\end{align}
y la transformada inversa de Laplace se define como:
\begin{align}
f(t) = L^{-1} \big[\overline{f}(t); p \to t\big] = L^{-1} \big[F(p)\big] = \dfrac{1}{2 \pi \, i} \int_{\gamma-i \infty}^{\gamma+\infty} \exp(p \, t) \, \overline{f} (p) \, \dd{p}
\label{eq:ecuacion_03_04}
\end{align}
donde $\gamma = \Re{p} > 0$.
\\[1em]
\noindent \textbf{Nota 1: } Algunos autores usan la variable $s$ en lugar de $p$.
\\
\textbf{Nota 2: } La función $f(t)$ debe de ser de orden exponencial\footnote{Este concepto se define en el siguiente nuumeral.} para la existencia de la transformada de Laplace.
\\
\textbf{Nota 3: } En términos de notación de operadores, la ec. (\ref{eq:ecuacion_03_03}) se expresa como:
\begin{align}
L \big[ f(t); t \to p] = \overline{f} (p) \big]
\label{eq:ecuacion_03_05}
\end{align}
y la relación en la ec. (\ref{eq:ecuacion_03_04}) se expresa como:
\begin{align}
L^{-1} \big[\overline{f}(p); p \to t \big] = f (t)
\label{eq:ecuacion_03_06}
\end{align}
Por lo que:
\begin{align*}
L^{-1} \big[L(f(t))]\big] = f(t) &= I \big[f(t)\big] \\[0.5em]
\Rightarrow \hspace{0.4cm} L^{-1} \, L \equiv I
\end{align*}
También se tiene:
\begin{align*}
L \big[L^{-1} (\overline{f}(p))]\big] = \overline{f}(p) &= I \big[\overline{f}(p)\big] \\[0.5em]
\Rightarrow \hspace{0.4cm} L \, L^{-1} \equiv I
\end{align*}
Esto significa que $L^{-1} \, L \equiv L \, L^{-1} \equiv I$, demostrando entonces que esos operadores $L$ y $L^{-1}$ son conmutativos.

\subsection{Condiciones suficientes para la existencia de la transformada de Laplace.}

\noindent \textbf{Teorema: } Si $f(t)$ es una función de algún orden exponencial para un valor de $t$ grande, y es continua en tramos sobre el intervalo $0 \leq t \leq \infty$, entonces la transformada de Laplace de $f(t)$ existe.
\\[0.5em]
\textbf{Definición: } Se dice que la función $f(t)$ es de \textbf{orden exponencial} conforme $t \to + \infty$ si existen constantes no negativas $M$, $\sigma$ y $t_{0}$ tales que
\begin{align}
\abs{f(t)} \leq M \; e^{\sigma \, t} \hspace{1cm} \mbox{para } t \geq t_{0}
\label{eq:023}
\end{align}
Así, una función es de orden exponencial siempre que su incremento (conforme $t \to + \infty$) no sea más rápido que un múltiplo constante de alguna función exponencial con un exponente lineal. Los valores particulares de $M$, $\sigma$ y $t_{0}$ no son tan importantes; lo importante es que algunos de esos valores existan de tal manera que la condición en (\ref{eq:023}) se satisfaga.
\par
La condición en (\ref{eq:023}) simplemente dice que $f(t) / e^{\sigma t}$ se encuentra entre $-M$ y $M$, y es por tanto acotada en su valor para $t$ suficientemente grande. En particular, esto se cumple (con $\sigma = 0$) si $f(t)$ en sí misma está acotada. Por tanto, toda función acotada -tal como $\cos k \, t$ o $\sin k \, t$ - es de orden exponencial.
\par
\noindent \textbf{Demostración: } Sea $f(t)$ de orden exponencial $\sigma$ tal que:
\begin{align*}
\abs{f(t)} < M \, e^{\sigma t} \hspace{1cm} \mbox{para} t \geq t_{0}
\end{align*}
Entonces se tenemos:
\begin{align*}
L \big[f(t); t \to p\big] &= \int_{0}^{\infty} \exp(-p \, t) \, f(t) \dd{t} = \\[0.5em]
&= \int_{0}^{t_{0}} \exp(-p \, t) \, f(t) \dd{t} + \int_{t_{0}}^{\infty} \exp(-p \, t) \, f(t) \dd{t} = \\[0.5em]
&= I_{1} + I_{2}
\end{align*}
Ya que $f(t)$ es una función continua en tramos en cada intervalo finito $0 \leq t \leq t_{0}$, la integral $I_{1}$ existe y es convergente. También se tiene:
\begin{align*}
\abs{I_{2}} &= \abs{\int_{t_{0}}^{\infty} e^{- p t} \, f(t) \dd{t}} \leq \\[0.5em]
&\leq \int_{t_{0}}^{\infty} e^{- p t} \, \abs{f(t)} \dd{t} < \\[0.5em]
&< M \, \int_{t_{0}}^{\infty} \exp(-(p - \sigma) \, t) \dd{t} = \dfrac{M \, \exp(-(p - \sigma) \, t_{0})}{(p - \sigma)}, \hspace{1cm} \mbox{si  } p > \sigma
\end{align*}
Por lo que $\abs{I_{2}}$ es finito para todo $t_{0} > 0$ y $p > \sigma$, y de aquí resulta que $I_{2}$ es convergente. Por tanto $L\abs{f(t)}$ existe para toda $p > \sigma$.
Aunque las condiciones establecidas en el teorema anterior son suficientes para la existencia de la transformada de Laplace, no lo son. Esto significa que incluso si una función no satisface las condiciones anteriores, la transformada de Laplace de esa función puede existir o no. 
\par
Consideremos el siguiente ejemplo con $f(t) = 1 / \sqrt{t}$:
\par
De la función se tiene que $f(t) \to \infty$ mientras que $t \to 0$, por lo que $f(t)$ no es una función continua en tramos para cada intervalo finito para $t \geq 0$.
\par
Ahora veamos:
\begin{align*}
L \big[\dfrac{1}{\sqrt{t}}\big] &= \int_{0}^{\infty} e^{-p t} \, \dfrac{1}{\sqrt{t}} \dd{t} = \\[0.5em]
&= \dfrac{2}{\sqrt{p}} \int_{0}^{\infty} e^{x^{2}} \dd{x} = \\[0.5em]
&= \sqrt{\dfrac{\pi}{p}} \hspace{1cm} p > 0
\end{align*}
Esto prueba la existencia de la transformada de Laplace de $f(t)$.

\section{Propiedades.}

\subsection{Propiedad de linealidad.}

Si $L \big[f_{1}(t); t \to p\big]$ y $L \big[f_{2}(t); t \to p\big]$ existen ambas y con las constantes $c_{1}$ y $c_{2}$, se tiene entonces:
\begin{align*}
L \big[c_{1} \, f_{1}(t) + c_{2} \, f_{2}(t) ; t \to p\big] = c_{1} \, L \big[f_{1}(t); t \to p\big] + c_{2} \, L c\big[f_{2}(t); t \to p\big]
\end{align*}

\subsection{Primer teorema de desplazamiento.}

\noindent \textbf{Teorema: } Si la transformada de Laplace de $f(t)$ es $\overline{f}(p)$, entonces la transformada de Laplace de $\exp(a \, t)$ es $\overline{f}(p - a)$.
\\[0.5em]
\textbf{Demostración: } Como tenemos:
\begin{align*}
L \big[f(t); t \to p\big] &= \int_{0}^{\infty} f(t) \, \exp{-p \, t} \dd{t} = \\[0.5em]+
&= \overline{f} (p)
\end{align*}
Entonces:
\begin{align}
\begin{aligned}[b]
L \big[\exp (a \, t) \, f(t); t \to p\big] &= \int_{0}^{\infty} \exp(a \, t) \, f(t) \, \exp(-p \, t) \dd{t} = \\[0.5em]
&= \int_{0}^{\infty} f(t) \, \exp(-(p - a) \, t) \dd{t} = \\[0.5em]
&= \overline{f} (p - a)
\end{aligned}
\label{eq:ecuacion_03_07}
\end{align}

\subsection{Segundo teorema de desplazamiento.}

\noindent \textbf{Teorema: } Si la transformada de Laplace de $f(t)$ es $\overline{f}(p)$, entonces la transformada de Laplace de $f(t - a) \, H(t - a)$ es $\exp(-a \, p) \, \overline{f}(p)$.
\\[0.5em]
\textbf{Demostración: } 
Como sabemos que:
\begin{align*}
L \big[f(t); t \to p\big] &= \int_{0}^{\infty} f(t) \, \exp(-p \, t) \dd{t} = \\[0.5em]
&= \overline{f} (p)
\end{align*}
Entonces:
\begin{align}
\begin{aligned}
L \big[f(t - a) \, H (t - a); t \to p\big] &= \int_{0}^{\infty} f(t - a) \, H(t - a) \, \exp(-p \, t) \dd{t} = \\[0.5em]
&= \int_{a}^{\infty} f(t - a) \, \exp(-p \, x) \cdot \exp(-p \, a) \dd{x} = \\[0.5em]
&= \exp(-p \, a) \, \overline{f} (p) 
\end{aligned}
\label{eq:ecuacion_03_08}
\end{align}

\subsection{Propiedad de cambio de escala.}

\noindent \textbf{Teorema: } Si
\begin{align*}
L \big[f(t); t \to p\big] &= \overline{f} (p)
\end{align*}
Entonces
\begin{align*}
L \big[f(a \, t); t \to p\big] &= \dfrac{1}{a}\overline{f} \left(\dfrac{p}{a}\right)
\end{align*}
\\[0.5em]
\textbf{Demostración: } La transformada de Laplace de $f(t)$ es:
\begin{align*}
L \big[f(t); t \to p\big] &= \int_{0}^{\infty} f(t) \, \exp(-p \, t) \dd{t} = \\[0.5em]
&= \overline{f} (p)
\end{align*}
Entonces:
\begin{align}
\begin{aligned}
L \big[f(a \, t); t \to p\big] &= \int_{0}^{\infty} f(a \, t) \, \exp(-p \, t) \dd{t} = \\[0.5em]
&= \dfrac{1}{a} \int_{0}^{\infty} f(x) \, \exp(- (p/a) \, t) \dd{x} = \\[0.5em]
&= \dfrac{1}{a}\overline{f} \left(\dfrac{p}{a}\right)
\end{aligned}
\label{eq:ecuacion_03_09}
\end{align}

\subsection*{Ejemplos.}

Calculemos la transformada de Laplace de algunas funciones sencillas a partir de la definición.

\begin{ejemplo}
\begin{align*}
L \big[H(t); t \to p\big] &= \int_{0}^{\infty} \exp(- p \, t) \, H(t) \dd{t} = \\[0.5em]
&= \int_{0}^{\infty} \exp(- p \, t) \dd{t} = \\[0.5em]
&= \dfrac{1}{p}
\end{align*}
De aquí que:
\begin{align*}
L \big[H(t - a); t \to p\big] &= \int_{0}^{\infty} \exp(- p \, t) \, H(t - a) \dd{t} = \\[0.5em]
&= \int_{a}^{\infty} \exp(- p \, t) \dd{t} = \\[0.5em]
&= \dfrac{\exp(-a \, p)}{p}
\end{align*}
\end{ejemplo}
\begin{ejemplo}
\begin{align*}
L \big[t^{\nu}; t \to p\big] &= \int_{0}^{\infty} t^{\nu} \, \exp(- p \, t) \dd{t} = \\[0.5em]
&= \int_{0}^{\infty} u^{\nu} \, \exp(-u) \dd{u} \hspace{1cm} \mbox{cuando  } u = p \, t \\[0.5em]
&= p^{-\nu - 1} \, \Gamma (\nu + 1) \hspace{1cm} \mbox{la cual existe cuando  } \Re{\nu} > - 1
\end{align*}
\end{ejemplo}
\begin{ejemplo}
\begin{align*}
L \big[e^{a t}; t \to p\big] &= \int_{0}^{\infty} e^{-p t} \, e^{a \, t} \dd{t} = \\[0.5em]
&= \dfrac{1}{p - a} \hspace{1cm} p > a
\end{align*}
\end{ejemplo}
\begin{ejemplo}
\begin{align*}
L \big[\sin (a t); t \to p\big] &= \int_{0}^{\infty} e^{-p t} \, \sin (a \, t) \dd{t} = \\[0.5em]
&= \int_{0}^{\infty} e^{-p t} \,\left[ \dfrac{e^{a i t} - e^{- a i t}}{2 \, i} \right] \dd{t} = \\[0.5em]
&= \left[ \int_{0}^{\infty}  \dfrac{e^{-(p - a i) t} - e^{- (p + a i) t}}{2 \, i} \right] \dd{t} = \\[0.5em]
&= \dfrac{1}{2 \, i} \left[ \dfrac{1}{p {-} a \, i} {-} \dfrac{1}{p {+} a \, i} \right] = \hspace{0.25cm} \mbox{por la propiedad de linealidad} \\[0.5em]
&= \dfrac{a}{p^{2} + a^{2}}, \hspace{1cm} p > 0
\end{align*}
\end{ejemplo}
\begin{ejemplo}
\begin{align*}
L \big[\cos (a t); t \to p\big] &= \int_{0}^{\infty} e^{-p t} \,\dfrac{e^{a i t} + e^{- a i t}}{2} \dd{t} = \\[0.5em]
&= \dfrac{1}{2} \left[ \int_{0}^{\infty} e^{-(p - a i) t} \dd{t} \right] + \dfrac{1}{2} \left[ \int_{0}^{\infty} e^{-(p + a i) t} \dd{t} \right] = \\[0.5em]
&\mbox{por la propiedad de linealidad  } \\[0.5em]
&= \dfrac{1}{2} \left[ \dfrac{1}{p - a \, i} + \dfrac{1}{p + a \, i}\right] = \\[0.5em]
&= \dfrac{p}{p^{2} + a^{2}}, \hspace{1cm} p > 0
\end{align*}
\end{ejemplo}
Evalúa la transformada de Laplace de la función que se indica:
\begin{ejemplo}
\begin{align*}
L\big[f(t)\big] \hspace{0.5cm} \mbox{donde} \hspace{0.5em} f(t) = \begin{cases}
t/a & 0 < t < a \\
1 & t > a
\end{cases}
\end{align*}

Solución:
\begin{align*}
L\big[f(t)\big] = \int_{0}^{\infty} e^{-p t} \, f(t) \dd{t}  &= \int_{0}^{a} e^{-p t} \cdot \dfrac{t}{a} \dd{t} + \int_{0}^{a} e^{-p t} \dd{t} = \\[0.5em]
&=\dfrac{1 - e^{- a p}}{a \, p^{2}}
\end{align*}
\end{ejemplo}
\begin{ejemplo}
Evalúa $L \bigg[\dfrac{1}{\sqrt{\pi \, t}}\bigg]$

Solución:
\begin{align*}
L \bigg[\dfrac{1}{\sqrt{\pi \, t}}\bigg] &= \int_{0}^{\infty} e^{- p t} \, \dfrac{1}{\sqrt{\pi \, t}} \dd{t} = \\[0.5em]
&= \dfrac{\sqrt{p}}{\sqrt{\pi}} \int_{0}^{\infty} e^{-x} \, x^{1/2 - 1} \dfrac{\dd{x}}{p} = \\[0.5em]
&= \dfrac{1}{\sqrt{\pi} \, p} \Gamma \left( \dfrac{1}{2} \right) = \\[0.5em]
&= \dfrac{\sqrt{\pi}}{\sqrt{\pi \, p}} = \dfrac{1}{\sqrt{\pi}}
\end{align*}
\end{ejemplo}
\begin{ejemplo}
Evaluar $L \big[t^{\nu} \, e^{-a t} \big]$, si $\Re{\nu + 1} > 0$

Solución: Se sabe que:
\begin{align*}
L \big[t^{\nu}; t \to p\big] = \dfrac{\Gamma (\nu + 1)}{p^{\nu+1}}
\end{align*}
Entonces, ocupando el teorema del desplazamiento, obtenemos:
\begin{align*}
L \big[t^{\nu} \, e^{-a t}; t \to p\big] = \dfrac{\Gamma (\nu + 1)}{(p + a)^{\nu+1}}
\end{align*}
\end{ejemplo}
\begin{ejemplo}
Evalúa $L \big[\sin (t - a) \, H(t- a); t \to p\big]$. \\
Para luego evaluar: $L \big[e^{(t-a) k} \, \sin (t - a) \, H(t- a); t \to p\big]$

Solución: Sabemos que:
\begin{align*}
L \big[\sin t; t \to p\big] = \dfrac{1}{p^{2} + 1}
\end{align*}
Por lo que ocupando el segundo teorema del desplazamiento, se obtiene:
\begin{align*}
L \big[\sin (t - a) \, H(t- a)\big] = \dfrac{e^{-p a}}{(1 + p^{2})}
\end{align*}
por lo que:
\begin{align*}
L \big[e^{(t-a) k} \, \sin (t - a) \, H(t- a); t \to p\big] &= \dfrac{e^{-(p - k)}}{\big[1 + (p - k^{2})\big]}
\end{align*}
después de haber utilizado el primer teorema del desplazamiento.
\end{ejemplo}

\subsection{La transformada de Laplace de la derivada de una función.}

\noindent \textbf{Teorema: } Sea $f(t)$ una función continua de $t \geq 0$ y es de orden exponencial para un valor de $t$ grande y si $\ptilde{f}(t)$ es una función continua a tramos para $t \geq 0$, entonces la transformada de Laplace de la derivada $\ptilde{f}(t)$ existe cuando $p > \sigma$ y está dada por:
\begin{align*}
L \big[\ptilde{f}(t)\big] = p \, L \big[f(t)\big] - f(0)
\end{align*}

\noindent \textbf{Dmostración:} Por definición de la transformada de Laplace:
\begin{align}
\begin{aligned}[b]
L \big[\ptilde{f}(t)\big] &= \int_{0}^{\infty} e^{-p t} \, \ptilde{f} (t) \dd{t} = \\[0.5em]
&= \big[e^{-p t} \cdot f(t)\big] \eval_{0}^{\infty} - \int_{0}^{\infty} (-p) \, e^{-p t} \, \ptilde{f} (t) \dd{t} = \\[0.5em]
&= \lim_{t \to \infty} e^{-p t} \, f(t) - f(0) + p \, L \big[f(t)\big]
\end{aligned}
\label{eq:ecuacion_03_10}
\end{align}
Como $f(t)$ es de orden exponencial $\sigma$ para $t \to \infty$, se tiene
\begin{align*}
\abs{f(t)} \leq M \, e^{-\sigma t}, \hspace{0.5em} t \geq 0
\end{align*}
Por lo que
\begin{align*}
\abs{f(t) \, e^{-p t}} = e^{-p t} \, \abs{f(t)} \leq M \, e^{-p t} \cdot e^{\sigma t} = M \, e^{-(p - \sigma) t}
\end{align*}
Así pues:
\begin{align*}
\lim_{t \to \infty} e^{-p t} \, f(t) = 0 \hspace{0.5cm} \mbox{mientras  } p - \sigma > 0
\end{align*}
Entonces, de la ec. (\ref{eq:ecuacion_03_10}), llegamos a:
\begin{align}
L \big[\ptilde{f}(t)\big] = p \, L \big[f(t)\big] - f(0)
\label{eq:ecuacion_03_11}
\end{align}

\noindent \textbf{Corolario 1: } Si $\stilde{f}(t)$ existe para $t \geq 0$ y es una función continua en tramos, entonces siguiendo el mismo procedimiento anterior, es posible extender el resultado del teorema como:
\begin{align}
\begin{aligned}[b]
L \big[\stilde{f}(t)\big] &= p \, L \big[\ptilde{f}(t)\big] - \ptilde{f}(0) = \\[0.5em]
&= p \, \big[p \left\{ L (f(t)) \right\} - f(0) \big] - \ptilde{f} (0) = \\[0.5em]
&= p^{2} \, L \big[f(t)\big] - p \, f(0) - \ptilde{f}(0)
\end{aligned}
\label{eq:ecuacion_03_12}
\end{align}

\noindent \textbf{Corolario 2:} En general, si $\ntilde{f}{n}(t)$ existe para $t \geq 0$ y es una función continua en tramos de $t$, entonces:
\begin{align}
L \big[\ntilde{f}{n}(t)\big] = p^{n} \, L \big[f(t)\big] - p^{n-1} \, f(0) - p^{n-2} \, \ptilde{f}(0) - \ldots - \ntilde{f}{n-1} (0)
\label{eq:ecuacion_03_13}
\end{align}

\subsection{Transoformada de Laplace de la integral de una función.}

Si la trasnformada de Laplace de una función $f(t)$ es $\overline{f}(p$). Entonces la transformada de la integral:
\begin{align}
\int_{0}^{t} f(\tau) \dd{\tau} = \dfrac{\overline{f}(p)}{p}
\label{eq:ecuacion_03_16}
\end{align}

\subsection{La Transformada de Laplace de una función periódica.}

Sea $f(t)$ una función con período $\tau$, tal que 
\begin{align*}
f(t + n \, \tau) = f(t) \hspace{1cm} n = 1, 2, 3, \ldots
\end{align*}
Si $f(t)$ es una función continua en tramos para $t > 0$, entonces:
\begin{align}
L \big[f(t)\big] = \dfrac{1}{1 - e^{-p \tau}} \int_{0}^{\tau} e^{-p t} \, f(t) \dd{t}
\end{align}

\subsection{Convolución de fos funciones.}

Sean $f (t)$ y $g (t)$ dos funciones continuas por partes y son de algún orden exponencial para $t$ grande y para todo $t \geq 0$. Entonces la \emph{convolución}\footnote{Recuerda la observación sobre la traducción del término en inglés \emph{convolution} que se discutió en el material de trabajo: Trasnformada de Fourier.} de estas funciones se denota por $f * g (t)$ y se define por:
\begin{align}
f * g(t) = \int_{0}^{t} f(u) \, g(t - u) \dd{u}
\label{03_55}
\end{align}
Esta relación también se le conoce como \emph{faltung} de $f(t)$ y $g(t)$. Por definición:
\begin{align}
\begin{aligned}
f * g(t) &= \int_{0}^{t} f(u) \, g(t - u) \dd{u} = \\[0.5em]
&= \int_{0}^{t} f(t - \tau) \, g(\tau) \dd{\tau} = \\[0.5em]
&= g * f(t)
\end{aligned}
\label{eq:ecuacion_03_56}
\end{align}
Por lo tanto, la convolución de dos funciones satisface la ley conmutativa.

Satisface también:
\begin{enumerate}
\item La ley distributiva
\begin{align}
f * \big[g + h\big](t) = f * g(t) + f * h(t)
\label{eq:ecuacion_03_57}
\end{align}
\item La ley asociativa
\begin{align}
\big[f * (g * h)\big](t) = \big[(f * g) * h\big](t)
\label{eq:ecuacion_03_58}
\end{align}
\end{enumerate}
Ahora podemos enfocarnos a evaluar la transoformada de Laplace de la convolución de dos funciones $f(t)$ y $g(t)$:
\begin{align}
\begin{aligned}[b]
L \big[(f * g)(t); t \to p\big] &= L \left[ \int_{0}^{t} f(\tau) \, g(t - \tau) \dd{\tau}; t \to p \right] = \\[0.5em]
&= \int_{0}^{\infty} e^{-p t} \, \left[ \int_{0}^{t} f(\tau) \, g(t - \tau) \dd{\tau}\right] \, \dd{t} = \\[0.5em]
&= \int_{0}^{\infty} f(\tau) \, \left[ \int_{\tau}^{\infty} e^{-p t} \, g(t - \tau) \dd{t}\right] \, \dd{\tau} = \\[0.5em]
&= \int_{0}^{\infty} f(\tau) \, e^{- p t} \, \left[ \int_{0}^{\infty} e^{-p \eta} \, g(\eta) \dd{\eta} \right] \, \dd{\tau} = \\[0.5em]
&= \overline{f}(p) \, \overline{g}(p)
\end{aligned}
\label{eq:ecuacion_03_59}
\end{align}
En particular se tiene:
\begin{align*}
L \big[f * f(t); t \to p\big] = \big[\overline{f}(p)\big]^{2}
\end{align*}

\section{La transformada inversa de Laplace.}

\subsection{Introducción.}
\end{document}