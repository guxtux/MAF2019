\documentclass[12pt]{article}
\usepackage[left=0.25cm,top=1cm,right=0.25cm,bottom=1cm]{geometry}
\textwidth = 20cm
\hoffset = -1cm
\usepackage[utf8]{inputenc}
\usepackage[spanish,es-tabla]{babel}
\usepackage[autostyle,spanish=mexican]{csquotes}
\usepackage[tbtags]{amsmath}
\usepackage{nccmath}
\usepackage{amsthm}
\usepackage{amssymb}
\usepackage{graphicx}
\usepackage{standalone}
\usepackage[outdir=./]{epstopdf}
\usepackage{siunitx}
\usepackage{physics}
\usepackage{color}
\usepackage{float}
\usepackage{multicol}
%\usepackage{milista}
\usepackage{enumitem}
\usepackage{anyfontsize}
\usepackage{anysize}
\usepackage{enumitem}
\usepackage{capt-of}
\usepackage{bm}
\usepackage{relsize}
\usepackage{placeins}
\usepackage{empheq}
\usepackage{cancel}
\usepackage{wrapfig}
\spanishdecimal{.}
\renewcommand{\baselinestretch}{1.5} 
\renewcommand\labelenumii{\theenumi.{\arabic{enumii}}}
\newcommand{\ptilde}[1]{\ensuremath{{#1}^{\prime}}}
\newcommand{\stilde}[1]{\ensuremath{{#1}^{\prime \prime}}}
\newcommand{\ttilde}[1]{\ensuremath{{#1}^{\prime \prime \prime}}}
\newcommand{\ntilde}[2]{\ensuremath{{#1}^{(#2)}}}


\title{Enunciados del Tema 6 para el Segundo Examen \\[0.3em]  \large{Matemáticas Avanzadas de la Física}\vspace{-3ex}}
\author{M. en C. Gustavo Contreras Mayén}
\date{ }
\begin{document}
\vspace{-4cm}
\maketitle
\fontsize{14}{14}\selectfont

\textbf{Indicaciones: } Deberás de resolver cada ejercicio de la manera más completa, ordenada y clara posible, anotando cada paso así como las operaciones involucradas. El puntaje de cada ejercicio es de \textbf{1 punto}, con excepción en donde se indica.

\begin{enumerate}

\item Calcula la transformada de Laplace de la siguiente función:
\begin{align*}
f (t) = \begin{cases}
0, & 0 < t < 1 \\
(t - 1)^{2}, & t > 1
\end{cases}
\end{align*}
%Ref. Example 4.20 p. 152
% \item Una función periódica $f (t)$ de período $2 \, \pi$ tiene una discontinuidad finita en $t = \pi$, la función está dada por:
% \begin{align*}
% f (t) = \begin{cases}
% \sin t, & 0 \leq t \leq \pi \\
% \cos t, & \pi < t \leq 2 \pi
% \end{cases}
% \end{align*}
% Calcula su transformada de Laplace.
%Exercises (1) p. 264
\item Encuentra la transformada inversa de Laplace de:
\begin{enumerate}[label=\alph*)]
\item $\dfrac{p}{(p + 1)^{2} (p^{2} + 1)}$
\item $\dfrac{p^{2} + 6}{(p^{2} + 1) (p^{2} + 4)}$
\end{enumerate}
%Ref. Example 4.32 p. 170
\item Ocupando la transformada de Laplace resuelve la siguiente ecuación diferencial:
\begin{align*}
\dv[2]{x}{t} + 9 \, x = \cos 2 \, t
\end{align*}
sujeta a las siguientes condiciones $x (0) =  1$, $x \bigg( \dfrac{\pi}{2} \bigg) = -1$
\end{enumerate}
\end{document}