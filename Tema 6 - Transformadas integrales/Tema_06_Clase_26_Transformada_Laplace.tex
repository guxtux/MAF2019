\documentclass[12pt]{beamer}
\usepackage{../Estilos/BeamerMAF}
\usetheme{Copenhagen}
\usecolortheme{wolverine}
%\useoutertheme{default}
\setbeamercovered{invisible}
% or whatever (possibly just delete it)
\setbeamertemplate{section in toc}[sections numbered]
\setbeamertemplate{subsection in toc}[subsections numbered]
\setbeamertemplate{subsection in toc}{\leavevmode\leftskip=3.2em\rlap{\hskip-2em\inserttocsectionnumber.\inserttocsubsectionnumber}\inserttocsubsection\par}
% \setbeamercolor{section in toc}{fg=blue}
% \setbeamercolor{subsection in toc}{fg=blue}
% \setbeamercolor{frametitle}{fg=blue}
\setbeamertemplate{caption}[numbered]

\setbeamertemplate{footline}
\beamertemplatenavigationsymbolsempty
\setbeamertemplate{headline}{}


\makeatletter
% \setbeamercolor{section in foot}{bg=gray!30, fg=black!90!orange}
% \setbeamercolor{subsection in foot}{bg=blue!30}
% \setbeamercolor{date in foot}{bg=black}
\setbeamertemplate{footline}
{
  \leavevmode%
  \hbox{%
  \begin{beamercolorbox}[wd=.333333\paperwidth,ht=2.25ex,dp=1ex,center]{section in foot}%
    \usebeamerfont{section in foot} \insertsection
  \end{beamercolorbox}%
  \begin{beamercolorbox}[wd=.333333\paperwidth,ht=2.25ex,dp=1ex,center]{subsection in foot}%
    \usebeamerfont{subsection in foot}  \insertsubsection
  \end{beamercolorbox}%
  \begin{beamercolorbox}[wd=.333333\paperwidth,ht=2.25ex,dp=1ex,right]{date in head/foot}%
    \usebeamerfont{date in head/foot} \insertshortdate{} \hspace*{2em}
    \insertframenumber{} / \inserttotalframenumber \hspace*{2ex} 
  \end{beamercolorbox}}%
  \vskip0pt%
}
\makeatother

\makeatletter
\patchcmd{\beamer@sectionintoc}{\vskip1.5em}{\vskip0.8em}{}{}
\makeatother

% %\newlength{\depthofsumsign}
% \setlength{\depthofsumsign}{\depthof{$\sum$}}
% \newcommand{\nsum}[1][1.4]{% only for \displaystyle
%     \mathop{%
%         \raisebox
%             {-#1\depthofsumsign+1\depthofsumsign}
%             {\scalebox
%                 {#1}
%                 {$\displaystyle\sum$}%
%             }
%     }
% }
% \def\scaleint#1{\vcenter{\hbox{\scaleto[3ex]{\displaystyle\int}{#1}}}}
% \def\scaleoint#1{\vcenter{\hbox{\scaleto[3ex]{\displaystyle\oint}{#1}}}}
% \def\bs{\mkern-12mu}


\makeatletter
\setbeamercolor{section in foot}{bg=cadetblue!20}
\setbeamercolor{subsection in foot}{bg=OliveGreen!30}
\makeatother

\date{14 de enero de 2022}

\title{\large{Transformada de Laplace}}
\author{M. en C. Gustavo Contreras Mayén}

\begin{document}
\maketitle
\fontsize{14}{14}\selectfont
\spanishdecimal{.}

\section*{Contenido}
\frame[allowframebreaks]{\tableofcontents[currentsection, hideallsubsections]}


%Referencia. Debnath - Chap. 2 Laplace Transform 3.9
\section{Definiciones}
\frame{\tableofcontents[currentsection, hideothersubsections]}
\subsection{la TF y su inversa}

\begin{frame}
\frametitle{La Transformada de Laplace}
Definimos la transformada de Laplace de una función continua en tramos de variable real $t$, definida en un semieje $t \geq 0$ como:
\pause
\begin{eqnarray}
\begin{aligned}
L \big[f(t); t \to p\big] &= \pause L \big[f(t)\big] = \pause F(p) = \pause \overline{f}(p) = \\[0.5em] \pause
&= \scaleint{6ex}_{\bs 0}^{\infty} f(t) \, \exp(-p \, t) \dd{t}
\end{aligned}
\label{eq:ecuacion_03_03}
\end{eqnarray}
\end{frame}
\begin{frame}
\frametitle{La TL inversa}
La transformada inversa de Laplace se define como:
\pause
\begin{eqnarray}
\begin{aligned}
f(t) &= \pause L^{-1} \big[\overline{f}(t); p \to t\big] = \pause L^{-1} \big[F(p)\big] = \\[0.5em] \pause
&= \dfrac{1}{2 \pi \, i} \scaleint{6ex}_{\bs \gamma-i \infty}^{\gamma+\infty} \exp(p \, t) \, \overline{f} (p) \, \dd{p}
\end{aligned}
\label{eq:ecuacion_03_04}
\end{eqnarray}
donde $\gamma = \Re{p} > 0$.
\end{frame}

\section{Uso de las propiedades de la TL}
\frame{\tableofcontents[currentsection, hideothersubsections]}
\subsection{Ejemplo 1}

\begin{frame}
\frametitle{EValuación de la TL}
Evalúa la TL:
\pause
\begin{align*}
L \big[ \sin (t - a) \, H (t - a); t \to p  \big]
\end{align*}
\pause
Y con ese resultado, evalúa:
\pause
\begin{align*}
L \big[ \exp\big[ (t - a ) k \big] \sin (t - a) \, H (t - a); t \to p  \big]
\end{align*}
\end{frame}
\begin{frame}
\frametitle{Solución al Ejemplo 1}
Para resolver este ejemplo, debemos de ocupar el resultado de la TL:
\begin{eqnarray*}
\begin{aligned}
L \big[ \sin t ; t \to p  \big] &= \scaleint{6ex}_{\bs 0}^{\infty} \sin t \, \exp(-p \, t) \dd{t} = \\[0.5em] \pause
&= \dfrac{1}{p^{2} + 1}
\end{aligned}
\end{eqnarray*}
\end{frame}
\begin{frame}
\frametitle{Ocupando una propiedad de la TL}
Por el segundo teorema de desplazamiento de la TL:
\\
\bigskip
\pause
\noindent \textbf{Teorema: } Si la transformada de Laplace de $f(t)$ es $\overline{f}(p)$, entonces la transformada de Laplace de $f(t - a) \, H(t - a)$ es $\exp(-a \, p) \, \overline{f}(p)$.
\end{frame}
\begin{frame}
\frametitle{Ocupando el segundo teorema de desplazamiento}
Entonces se tiene que:
\begin{align*}
L \big[ \exp\big[ (t - a ) k \big] \sin (t - a) \, H (t - a) \big] = \dfrac{\exp (- p a)}{(1 + p^{2})}
\end{align*}
\end{frame}
\begin{frame}
\frametitle{Otra propiedad de la TL}
Ahora ocuparemos el primer teorema de desplazamiento para la TL:
\\
\bigskip
\pause
\noindent \textbf{Teorema: } Si la transformada de Laplace de $f(t)$ es $\overline{f}(p)$, entonces la transformada de Laplace de $\exp(a \, t)$ es $\overline{f}(p - a)$.
\end{frame}
\begin{frame}
\frametitle{Resultado del ejercicio}
Al ocupar el primer teorema del desplazamiento, llegamos al resultado:
\pause
\begin{eqnarray*}
\begin{aligned}
&L \big[ \exp\big[k (t - a) \big] \, \sin (t - a) \, H (t - a); t \to p  \big] = \\[0.5em]
&= \dfrac{\exp\big[ -(p - k) \big]}{\big[ 1 + (p - k)^{2} \big]}
\end{aligned}
\end{eqnarray*}
\end{frame}

\subsection{Ejemplo 2}

\begin{frame}
\frametitle{Enunciado del ejemplo}
Evalúa:
\pause
\begin{align*}
L \big[ \sinh bt; t \to p \big]
\end{align*}
\pause
Para luego obtener:
\pause
\begin{align*}
L \big[ \exp (a t) \, \sinh bt; t \to p \big]
\end{align*}    
\end{frame}
\begin{frame}
\frametitle{Solución}
Nuevamente nos apoyamos en un resultado de la TL de la función $\sinh t$:
\pause
\begin{eqnarray*}
\begin{aligned}
L \big[ \sinh t ; t \to p \big] &= \scaleint{6ex}_{\bs 0}^{\infty} \sinh t \, \exp(-p \, t) \dd{t} = \\[0.5em] \pause
&= \dfrac{1}{p^{2} - 1}
\end{aligned}
\end{eqnarray*}
\end{frame}
\begin{frame}
\frametitle{Usando otra propiedad de la TL}
Ocuparemos la propiedad de cambio de escala de la TL:
\\
\bigskip
\pause
\noindent \textbf{Teorema: } Si:
\begin{align*}
L \big[f(t); t \to p\big] &= \overline{f} (p)
\end{align*}
Entonces
\begin{align*}
L \big[f(a \, t); t \to p\big] &= \dfrac{1}{a}\overline{f} \left(\dfrac{p}{a}\right)
\end{align*}
\end{frame}
\begin{frame}
\frametitle{Resultado de la TL}
Luego de ocupar la propiedad de cambio de escala, tenemos que la TL es:
\pause
\begin{align*}
L \big[ \sinh b t ; t \to p \big] = \dfrac{b}{p^{2} - b^{2}}
\end{align*}
\end{frame}
\begin{frame}
\frametitle{Respuesta al ejemplo}
El resultado final del ejercicio es:
\pause
\begin{align*}
L \big[ \exp (a t) \, \sinh bt; t \to p \big] = \dfrac{b}{(p - a)^{2} + b^{2}}
\end{align*}
\end{frame}

\subsection{Ejercicio 3}

\begin{frame}
\frametitle{Enunciado del ejercicio}
Evalúa la TL:
\pause
\begin{align*}
L \big[ \exp (a t) \, \cos bt; t \to p \big]
\end{align*}
\pause
Con ese resultado, obtén:
\pause
\begin{align*}
L \big[ \exp\big[ a(t - c) \, \cos b (t - c) \, H (t - c); t \to p \big]
\end{align*}
\end{frame}
\begin{frame}
\frametitle{Resultado de utilidad}
Como hemos visto en los ejercicios previos, cuando se tiene una función que involucra a otras \enquote{funciones conocidas}, de las que es necesario contar con su TL, \pause es conveniente ocupar las TL de esas funciones ya sea en una lista o tabla.
\end{frame}
\begin{frame}
\frametitle{Resultado de utilidad}
En este caso, se requiere conocer el valor de la TL de:
\pause
\begin{align*}
L \big[ \cos b t; t \to p \big] = \dfrac{p}{p^{2} + b^{2}}
\end{align*}    
\end{frame}
\begin{frame}
\frametitle{Ocupando las respectivas propiedades}
Así mismo, se procede a ocupar las propiedades de la TL que sean las pertinentes, de esta manera se simplifica el trabajo para obtener la solución.
\end{frame}
\begin{frame}
\frametitle{Avanzando en la solución}
Al ocupar el teorema de desplazamiento:
\pause
\begin{align*}
L \big[ \exp(a t) \, \cos b t; t \to p \big] = \dfrac{p - a}{(p - a)^{2} + b^{2}}
\end{align*}
\end{frame}
\begin{frame}
\frametitle{Solución al ejercicio}
Al ocupar ahora otra de las propiedades de la TL, se llega al resultado final del ejercicio:
\pause
\begin{eqnarray*}
\begin{aligned}
&L \big[ \exp\big[ a(t - c) \, \cos b (t - c) \, H (t - c); t \to p \big] = \\[0.5em]
&= \exp (- p c) \, \dfrac{(p - a)}{(p - a)^{2} + b^{2}}
\end{aligned}
\end{eqnarray*}
\end{frame}

\subsection{Ejercicio 4}

\begin{frame}
\frametitle{Enunciado del ejercicio 4}
Evalúa las siguientes TL:
\pause
\setbeamercolor{item projected}{bg=blue!70!black,fg=yellow}
\setbeamertemplate{enumerate items}[circle]
\begin{enumerate}[<+->]
\item $L \big[ t \, H(t - 1); t \to p \big]$
\item $L \big[ H(t - 1) \, \cos a t; t \to p \big]$
\end{enumerate}
\end{frame}
\begin{frame}
\frametitle{Resultado para ambos incisos}
De la definición de:
\pause
\begin{eqnarray*}
\begin{aligned}
&L \big[ f(t) \, H (t - a); t \to p \big] = \scaleint{6ex}_{\bs a}^{\infty} \exp(- p t) \, f(t) \dd{t} = \\[0.5em] \pause
&=  \scaleint{6ex}_{\bs 0}^{\infty} \exp\big[ - p (u {+} a) \big] \, f(u {+} a) \dd{u} = \\[0.5em] \pause
&= \exp(- p a) \scaleint{6ex}_{\bs 0}^{\infty} \exp(- p u) \, f(u {+} a) \dd{u} = \\[0.5em] \pause
&= \exp(- p a) \, L \big[ f (t {+} a); t \to p \big]
\end{aligned}
\end{eqnarray*}
\end{frame}
\begin{frame}
\frametitle{Ocupando el resultado anterior}
Para resolver los dos incisos del ejercicio, ocuparemos el resultado que se obtuvo:
\pause
\begin{align*}
L \big[ f(t) \, H (t{-} a) \big] = \exp(- p \, a) \, L \big[ f (t {+} a); t \to p \big]    
\end{align*}
\end{frame}
\begin{frame}
\frametitle{Resolviendo el inciso i)}
Tenemos entonces que:
\pause
\begin{eqnarray*}
\begin{aligned}
&L \big[ t \, H(t {-} 1); t \to p \big] = \scaleint{6ex}_{1}^{\infty} \exp(- p t) \, t \dd{t} = \hspace{1cm} a = 1 \\[0.5em]
&= e^{-p} \scaleint{6ex}_{0}^{\infty} \exp(- p u) \, (t + 1) \dd{t} = \\[0.5em] \pause
&= e^{-p} \bigg[ \dfrac{1}{p^{2}} + \dfrac{1}{p} \bigg] = \pause \dfrac{p + 1}{p^{2}} \, e^{-p}
\end{aligned}
\end{eqnarray*}
\end{frame}
\begin{frame}
\frametitle{Resolviendo el inciso ii)}
Tenemos ahora que:
\pause
\begin{eqnarray*}
\begin{aligned}
&L \big[ H(t {-} 1) \, \cos a t \big] = \exp(- p) \, L \big[ \cos a (t + 1) \big] = \\[0.5em] \pause
&= e^{-p} \, L \big[ \cos a t \cdot \cos a - \sin a t \cdot \sin a \big] = \\[0.5em] \pause
&= \dfrac{e^{-p}}{p^{2} + 1} \, \big[ p \cos a - a \sin a \big]
\end{aligned}
\end{eqnarray*}
\end{frame}

\subsection{Otra propiedad}

\begin{frame}
\frametitle{Una propiedad importante}
Dentro de la lista de propiedades de la TL que se mencionan en las notas de trabajo, hay que agregar la siguiente:
\\
\bigskip
\pause
\noindent \textbf{Teorema: } Si la TL de una función continua en partes $f(t)$ es $\overline{f}(p)$, entonces la TL de:
\pause
\begin{align*}
L \big[ t^{n} \, f(t); t \to p \big] = (-1)^{n} \, \dv[n]{\overline{f}(p)}{p}
\end{align*}
con $n$ un entero positivo.
\end{frame}

\end{document}