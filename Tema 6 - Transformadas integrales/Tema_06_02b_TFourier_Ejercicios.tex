\documentclass[12pt]{beamer}
\usepackage{../Estilos/BeamerMAF}
\usepackage[absolute, overlay]{textpos}
\usepackage{../Estilos/ColoresLatex}
\input{../Preambulos/preambulo_Beamer_Antibes_beaver}

\setbeamercolor{section in foot}{bg=amethyst, fg=white}
\setbeamercolor{subsection in foot}{bg=almond, fg=black}

\makeatletter
\setbeamertemplate{footline}
{
\leavevmode%
\hbox{%
\begin{beamercolorbox}[wd=.333333\paperwidth,ht=2.25ex,dp=1ex,center]{section in foot}%
  \usebeamerfont{section in foot} \insertsection
\end{beamercolorbox}%
\begin{beamercolorbox}[wd=.333333\paperwidth,ht=2.25ex,dp=1ex,center]{subsection in foot}%
  \usebeamerfont{subsection in foot}  \insertsubsection
\end{beamercolorbox}%
\begin{beamercolorbox}[wd=.333333\paperwidth,ht=2.25ex,dp=1ex,right]{date in head/foot}%
  \usebeamerfont{date in head/foot} \insertshortdate{} \hspace*{1.5em}
  \insertframenumber{} / \inserttotalframenumber \hspace*{2ex} 
\end{beamercolorbox}}%
\vskip0pt%
}
\makeatother
\usefonttheme{serif}
\setbeamercolor{frametitle}{bg=champagne}
\resetcounteronoverlays{saveenumi}

\date{31 de mayo de 2022}

\title{\large{Transformada de Fourier - Ejercicios}}
\subtitle{Matemáticas Avanzadas de la Física}
\author{M. en C. Gustavo Contreras Mayén}

\begin{document}
\maketitle
\fontsize{14}{14}\selectfont
\spanishdecimal{.}

\section*{Contenido}
\frame[allowframebreaks]{\frametitle{Temas a revisar} \tableofcontents[currentsection, hideallsubsections]}

\section{La Transformada de Fourier}
\frame[allowframebreaks]{\frametitle{Contenido}\tableofcontents[currentsection, hideothersubsections]}
\subsection{Ejercicios}

%Ref. Patra (2018) . Example 1.4
\begin{frame}
\frametitle{Enunciado}
Calcula la función cuya Transformada coseno de Fourier es:
\pause
\begin{align*}
\sqrt{\dfrac{2}{\pi}} \, \dfrac{\sin a \, \xi}{\xi}
\end{align*}
\end{frame}
\begin{frame}
\frametitle{Solución}
Sabemos que la transformada coseno de Fourier es:
\pause
\begin{align*}
F_{c} \big[ f (x); x \to \xi \big] = \sqrt{\dfrac{2}{\pi}} \, \dfrac{\sin a \, \xi}{\xi}
\end{align*}
\end{frame}
\begin{frame}
\frametitle{Usando la TCIF}
La Transformada Coseno Inversa de Fourier (TCIF), está dada por la expresión:
\pause
\begin{align*}
f (x) = \sqrt{\dfrac{2}{\pi}} \scaleint{6ex}_{\bs 0}^{\infty} F_{c} (\xi) \, \cos \xi \, x \dd{\xi}
\end{align*}
\end{frame}
\begin{frame}
\frametitle{Usando la TCIF}
Entonces por la definición anterior:
\pause
\begin{align*}
f (x) = \dfrac{2}{\pi} \scaleint{6ex}_{\bs 0}^{\infty} \dfrac{\sin a \, \xi}{\xi} \, \cos \xi \, x \dd{\xi}
\end{align*}
\pause
Ahora tendremos que hacer uso extensivo de nuestras habilidades en el álgebra e integración para resolver la expresión.
\end{frame}
\begin{frame}
\frametitle{Identidad trigonométrica}
Usamos una identidad trigonométrica del producto:
\pause
\begin{align*}
\sin x \, \cos y = \dfrac{\sin (x + y) + \sin (x - y)}{2}
\end{align*}
\pause
Tenemos entonces que:
\begin{eqnarray*}
\begin{aligned}
\sin a \, \xi \, \cos \xi \, x &= \dfrac{\sin \big[ a \xi + \xi x \big] + \sin \big[ a \xi - \xi x \big]}{2} = \\[0.5em] \pause
&= \dfrac{\sin \big[ \xi (a + x) \big] + \sin \big[ \xi (a - x) \big]}{2}
\end{aligned}
\end{eqnarray*}
\end{frame}
\begin{frame}
\frametitle{Ajuste en la expresión}
Ahora la TCIF es:
\pause
\begin{eqnarray*}
\begin{aligned}
f (x) &= \dfrac{1}{\pi} \scaleint{6ex}_{\bs 0}^{\infty} \dfrac{1}{\xi} \, \sin \big[ \xi (a + x) \big] + \sin \big[ \xi (a - x) \big] \dd{\xi} = \\[0.5em] \pause
&= \dfrac{1}{\pi} \bigg[ \scaleint{6ex}_{\bs 0}^{\infty} \dfrac{\sin \big[ \xi (a + x) \big]}{\xi} \dd{x} + \\[0.5em] 
&+ \scaleint{6ex}_{\bs 0}^{\infty} \dfrac{\sin \big[ \xi (a - x) \big]}{\xi} \dd{x} \bigg] =
\end{aligned}
\end{eqnarray*}
\end{frame}


\end{document}