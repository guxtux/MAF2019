\documentclass[12pt]{article}
\usepackage[left=0.25cm,top=1cm,right=0.25cm,bottom=1cm]{geometry}
\textwidth = 20cm
\hoffset = -1cm
\usepackage[utf8]{inputenc}
\usepackage[spanish,es-tabla]{babel}
\usepackage[autostyle,spanish=mexican]{csquotes}
\usepackage[tbtags]{amsmath}
\usepackage{nccmath}
\usepackage{amsthm}
\usepackage{amssymb}
\usepackage{graphicx}
\usepackage{standalone}
\usepackage[outdir=./]{epstopdf}
\usepackage{siunitx}
\usepackage{physics}
\usepackage{color}
\usepackage{float}
\usepackage{multicol}
%\usepackage{milista}
\usepackage{enumitem}
\usepackage{anyfontsize}
\usepackage{anysize}
\usepackage{enumitem}
\usepackage{capt-of}
\usepackage{bm}
\usepackage{relsize}
\usepackage{placeins}
\usepackage{empheq}
\usepackage{cancel}
\usepackage{wrapfig}
\spanishdecimal{.}
\renewcommand{\baselinestretch}{1.5} 
\renewcommand\labelenumii{\theenumi.{\arabic{enumii}}}
\newcommand{\ptilde}[1]{\ensuremath{{#1}^{\prime}}}
\newcommand{\stilde}[1]{\ensuremath{{#1}^{\prime \prime}}}
\newcommand{\ttilde}[1]{\ensuremath{{#1}^{\prime \prime \prime}}}
\newcommand{\ntilde}[2]{\ensuremath{{#1}^{(#2)}}}


\title{Enunciados del Tema 6 para el Segundo Examen \\[0.3em]  \large{Matemáticas Avanzadas de la Física}\vspace{-3ex}}
\author{M. en C. Gustavo Contreras Mayén}
\date{ }
\begin{document}
\vspace{-4cm}
\maketitle
\fontsize{14}{14}\selectfont

\textbf{Indicaciones: } Deberás de resolver cada ejercicio de la manera más completa, ordenada y clara posible, anotando cada paso así como las operaciones involucradas. El puntaje de cada ejercicio es de \textbf{1 punto}, con excepción en donde se indica.

\begin{enumerate}
%Ref. Patra Example 1.12
\item Con las siguientes funciones:
\begin{align*}
g (x) = e^{-a x} \hspace{2cm} f (x) \begin{cases}
1, & 0 < x < b \\
0, & x > b
\end{cases} 
\end{align*}
y con la relación de Parseval pertinente, demuestra que:
\begin{align*}
\scaleint{6ex}_{\bs 0}^{\infty} \dfrac{\sin a t}{t (a^{2} + t^{2})} \dd{t} = \dfrac{\pi}{2} \, \dfrac{1 - \exp(-a^{2})}{a^{2}}
\end{align*}

%Ref. Patra Example 1.38
\item La temperatura $u (x, t)$ en una barra semiinfinita $0 \leq x < \infty$ satisface la siguiente EDP:
\begin{align*}
\pdv{u}{t} =\kappa \, \pdv[2]{u}{x}
\end{align*}
sujeta a las siguientes condiciones:
\begin{align*}
u (x, 0) &= 0, \hspace{0.5cm} x \geq 0 \\[0.5em]
\pdv{u}{x} &= - \lambda \hspace{0.5cm} \mbox{una constante, cuando \quad} x = 0, \hspace{0.2cm} t > 0  
\end{align*}
Calcula la temperatura para valores $x > 0$ y $t > 0$.

%Ref. Patra Example 1.36
\item Resuelve la siguiente ecuación diferencial parcial:
\begin{align*}
\pdv{u}{t} = 2 \, \pdv[2]{u}{x}
\end{align*}
sujeta a las siguientes condiciones:
\begin{align*}
u (0, t) &= 0 \\[0.5em]
u (x, 0) &= e^{-x}, \hspace{1.5cm} x > 0
\end{align*}
y $u(x, t)$ está acotada cuando $x > 0$ y $t > 0$.
\end{enumerate}
\end{document}