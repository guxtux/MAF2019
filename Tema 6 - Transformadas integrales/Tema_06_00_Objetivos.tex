\documentclass[10pt]{beamer}
\usetheme[
%%% option passed to the outer theme
%    progressstyle=fixedCircCnt,   % fixedCircCnt, movingCircCnt (moving is deault)
  ]{Feather}
  
% If you want to change the colors of the various elements in the theme, edit and uncomment the following lines

% Change the bar colors:
%\setbeamercolor{Feather}{fg=red!20,bg=red}

% Change the color of the structural elements:
%\setbeamercolor{structure}{fg=red}

% Change the frame title text color:
%\setbeamercolor{frametitle}{fg=blue}

% Change the normal text color background:
%\setbeamercolor{normal text}{fg=black,bg=gray!10}

%-------------------------------------------------------
% INCLUDE PACKAGES
%-------------------------------------------------------

\usepackage[utf8]{inputenc}
\usepackage[spanish]{babel}
\usepackage[T1]{fontenc}
\usepackage{helvet}
\usepackage{multirow}

%-------------------------------------------------------
% DEFFINING AND REDEFINING COMMANDS
%-------------------------------------------------------

% colored hyperlinks
\newcommand{\chref}[2]{
  \href{#1}{{\usebeamercolor[bg]{Feather}#2}}
}

%-------------------------------------------------------
% INFORMATION IN THE TITLE PAGE
%-------------------------------------------------------

\title[] % [] is optional - is placed on the bottom of the sidebar on every slide
{ % is placed on the title page
      \textbf{Tema 6 - Transformadas integrales \\ \medskip
      \large{Matemáticas Avanzadas de la Física}}
}

\subtitle[Transformadas integrales]
{
%      \textbf{v. 1.0.0}
}

\author[M. en C. Gustavo Contreras Mayén]
{      M. en C. Gustavo Contreras Mayén \\
      {\ttfamily gux7avo@ciencias.unam.mx}
}

\institute[]
{
      Departmento de Física \\
      Facultad de Ciencias, UNAM \\
}

\date{\today}

%-------------------------------------------------------
% THE BODY OF THE PRESENTATION
%-------------------------------------------------------

\begin{document}

%-------------------------------------------------------
% THE TITLEPAGE
%-------------------------------------------------------

{\1% % this is the name of the PDF file for the background
\begin{frame}[plain,noframenumbering] % the plain option removes the header from the title page, noframenumbering removes the numbering of this frame only
  \titlepage % call the title page information from above
\end{frame}}


\begin{frame}{Contenido}{}
\tableofcontents
\end{frame}

%-------------------------------------------------------
\section{Objetivos}
%-------------------------------------------------------
\begin{frame}{Objetivos}
%-------------------------------------------------------
\begin{itemize}
\item<1-> El alumno identificará la naturaleza de las transformadas integrales, así como los distintos tipos que existe en la física matemática.
\item <2-> Aplicará la transformada de Fourier para resolver distintos tipos de problemas con ecuaciones diferenciales parciales.
\item <3-> Reconocerá y utilizará la transformada de Laplace en ejercicios con ecuaciones diferenciales ordinarias y parciales.
\item <4-> Establecerá el uso de la transformada discreta de Fourier y de su evaluación numérica.
\end{itemize}
\end{frame}
%-------------------------------------------------------
\section{Introducción a las transformadas integrales}
\subsection{Marco teórico}

\begin{frame}{Introducción a las transformadas integrales}{Marco teórico}
Se revisará de manera general y con un punto de vista de aplicación, el concepto de transformada integral, así como los distintos tipos que se suelen encontrar en la Física Matemática.
\end{frame}
% %-------------------------------------------------------
\subsection{Tipos de transformadas integrales}

\begin{frame}{Introducción a las transformadas integrales}{Tipos de transformadas integrales}
Una vez revisada la definición de transformada integral, se presenta una lista con una serie de transformadas integrales, las de más uso en ciencia e ingeniería, así como transformadas especiales que se ocupan a partir de funciones especiales.
\\
\bigskip
\pause
Cada una de estas transformadas tiene su correspondiente transformada inversa.
\end{frame}
%--------------------------------------------------------------
\section{Transformadas Integrales}
\subsection{Transformada de Fourier}

\begin{frame}{Transformadas integrales}{La Transformada de Fourier}
Se hace una revisión sobre la Transformada de Fourier así como la Transformada inversa.
\\
\bigskip
\pause
Como herramienta necesaria se recomienda hacer una revisión sobre el tema de series de Fourier, ya que se ocuparán algunos resultados de ese tema.
\end{frame}
\begin{frame}{Transformadas integrales}{La Transformada de Fourier}
Se discutirán una serie de propiedades de la Transformada de Fourier, en donde para algunos casos, se revisará la demostración de esas propiedades, dejando otras para revisión por parte del alumno, como tarea moral.
\\
\bigskip
\pause
Ocupando la definición de las transformadas es posible obtener el resultado, siguiendo una serie de pasos sencillos.
\end{frame}
%--------------------------------------------------------------
\subsection{Transformada de Laplace}

\begin{frame}{Transformadas integrales}{La Transformada de Laplace}
Se hace una revisión sobre la Transformada de Laplace así como la Transformada inversa, destacando que esta transformada es la de mayor utilidad en física matemática.
\\
\bigskip
\pause
También se presentarán un conjunto de propiedades de la Transformada de Laplace para ocuparla como herramienta de solución a problemas con ecuaciones diferenciales.
\end{frame}
%--------------------------------------------------------------
\subsection{Transformada Discreta y Rápida de Fourier}

\begin{frame}{Transformadas integrales}{La Transformada Discreta y Rápida de Fourier}
Cuando se tiene que la función de estudio en la transformada no es continua, sino que se presenta de manera discreta, se requiere hacer un ajuste en la transformada integral, así como en el algoritmo para calcular los valores.
\\
\bigskip
\pause
Se presentarán dos técnicas de evaluación de la Trasformada Discreta de Fourier.
\end{frame}

%--------------------------------------------------------------
\section{Material de trabajo}
\subsection{Documentos de revisión}

\begin{frame}{Lecturas de revisión}{Material de trabajo}
Para lograr los objetivos del Tema 6, se tendrán que revisar los siguientes materiales de trabajo que estarán disponibles en la plataforma Moodle.
\\
\bigskip
\pause
\setbeamercolor{item projected}{bg=blue!70!black,fg=yellow}
\setbeamertemplate{enumerate items}[circle]
\begin{enumerate}[<+->]
\item Introducción a las transformadas integrales.
\item Transformada de Fourier.
\item Transformada de Laplace.
\item Transformada discreta y rápida de Fourier.
\end{enumerate}
\end{frame}

%--------------------------------------------------------------
\section{Evaluación}
\subsection{Ejercicios semanales}

\begin{frame}
\frametitle{Ejercicios}
La distribución de ejercicios semanales y opcionales es la siguiente:
\pause
\renewcommand{\arraystretch}{1.15}
\begin{table}
\centering
\begin{tabular}{l c c}
Material & Semanales & Opcionales \\ \hline
T. Fourier & 3 & 1 \\ \hline
T. Laplace  & 3 & 1 \\ \hline
DFT y FFT & 2 & 1 \\ \hline    
\end{tabular}
\end{table}
\end{frame}
\begin{frame}
\frametitle{Ajuste en la entrega de ejercicios}
Los ejercicios a cuenta y opcionales estarán disponibles tanto en el material de trabajo como en Moodle, siendo necesario el ajuste en la entrega de los mismos, cambiando la fecha para el siguiente miércoles y no el viernes.
\\
\bigskip
\pause
Por lo que los ejercicios del tema de Transformada de Fourier se entregarán el día 19 de enero, mientras que los ejercicios de la transformada de Laplace, el día 26 de enero a las 6 pm, como hora límite. 
\end{frame}
\begin{frame}
\frametitle{Últimos ejercicios}
Los ejercicios de los temas DFT y FFT se entregarán el lunes 31 de enero a las 6 pm como hora límite.
\\
\bigskip
\pause
Con este ajuste nos será posible contar con las evaluaciones y obtener los correspondientes promedios, para que así se pueda revisar de manera individual la situación del grupo.
\end{frame}
\begin{frame}
\frametitle{Tarea Examen}
La tarea examen del Tema 6, se entregará el día viernes 14 de enero de 2022 durante la sesión en Zoom.
\\
\bigskip
\pause
Se deberá de entregar el examen resuelto el día lunes 31 de enero de 2022 a más tardar a las 6 pm.
\end{frame}

%--------------------------------------------------------------
\section{Calendarización}
\subsection{Semanas de trabajo}

\begin{frame}{Calendarización}{Semanas de trabajo}
Tendremos las tres últimas semanas de trabajo del semestre:
\setbeamercolor{item projected}{bg=blue!70!black,fg=yellow}
\setbeamertemplate{enumerate items}[circle]
\begin{enumerate}[<+->]
\item Transformada de Fourier. Del 10 al 14 de enero de 2022.
\item Transformada de Laplace. Del 17 al 21 de enero de 2022.
\item DFT y FFT. Del 24 al 28 de enero de 2022.
\end{enumerate}
\end{frame}
%--------------------------------------------------------------

\subsection{Sesiones síncronas}

\begin{frame}{Calendarización}{Sesiones Zoom}
Se han programado las siguientes sesiones de trabajo síncronas en Zoom a las 3 pm:
\setbeamercolor{item projected}{bg=blue!70!black,fg=yellow}
\setbeamertemplate{enumerate items}[circle]
\begin{enumerate}[<+->]
\item Lunes 10 y 14 de enero de 2022.
\item Miércoles 19 y viernes 21 de enero de 2022.
\item Miércoles 26 y viernes 28 de enero de 2022.
\end{enumerate}
\end{frame}
%--------------------------------------------------------------
\end{document}