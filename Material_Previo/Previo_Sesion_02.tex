\documentclass[12pt]{beamer}
\usepackage{../Estilos/BeamerMAF}
\input{../Preambulos/preambulo_Beamer_Cambridge_beaver}
\date{}
\title{Ejercicios}
\author{M. en C. Gustavo Contreras Mayén}

\begin{document}
\maketitle
\fontsize{14}{14}\selectfont
\spanishdecimal{.}

\section*{Contenido}
\frame{\tableofcontents[currentsection, hideallsubsections]}

\section{Vectores no ortogonales}
\frame{\tableofcontents[currentsection, hideothersubsections]}
\subsection{Vectores no coplanares}

\begin{frame}
\frametitle{Primer ejercicio}
¿Cuáles de los siguientes conjuntos de vectores son no coplanares?
\\
\bigskip
\begin{minipage}{0.4\textwidth}
a) \begin{align*}
\va{b}_{1} &= 2 \, \vu{i} + \vu{j} + 4 \, \vu{k} \\[0.5em]
\va{b}_{2} &= \vu{i} \hspace{1.1cm} + 3 \, \vu{k} \\[0.5em]
\va{b}_{3} &= -3 \, \vu{i} - 4 \, \vu{j} - \vu{k}
\end{align*}
\end{minipage}
\hspace{0.3cm}
\pause
\begin{minipage}{0.4\textwidth}
b) \begin{align*}
\va{b}_{1} &= \vu{i} + 2 \, \vu{j} + 2 \, \vu{k} \\[0.5em]
\va{b}_{2} &= - 3 \, \vu{i} - 4 \, \vu{j} + 2 \, \vu{k} \\[0.5em]
\va{b}_{3} &= - 2 \, \vu{i} - \vu{j} - \vu{k}
\end{align*}
\end{minipage}
\end{frame}
\begin{frame}
\frametitle{Vectores no coplanares}
Un conjunto de tres vectores $\left\{ \va{b}_{1}, \va{b}_{2}, \va{b}_{3} \right\}$ serán no coplanares si se satisface la siguiente condición:
\pause
\begin{align*}
\va{b}_{1} \cdot \va{b}_{2} \times \va{b}_{3} \neq 0
\end{align*}
\pause
Por lo que habrá que revisar esta condición en los conjuntos de vectores dados en los incisos a) y b).
\end{frame}
\begin{frame}
\frametitle{Inciso a)}
Calculamos entonces:
\begin{eqnarray*}
\va{b}_{1} \cdot \va{b}_{2} \times \va{b}_{3} = \pause
\mqty|
2 & 1 & 4 \\
1 & 0 & 3 \\
-3 & -4 & -1 | = \pause 0
\end{eqnarray*}
\pause
Por lo tanto, el conjunto de tres vectores $\left\{ \va{b}_{1}, \va{b}_{2}, \va{b}_{3} \right\}$ son coplanares. \pause Esto implica que no podremos obtener un conjunto de vectores recíprocos.
\end{frame}
\begin{frame}
\frametitle{Inciso b)}
Calculamos entonces:
\begin{eqnarray*}
\va{b}_{1} \cdot \va{b}_{2} \times \va{b}_{3} = \pause
\mqty|
1 & 2 & 2 \\
-3 & -4 & 2 \\
2 & -1 & -1 | = \pause 30
\end{eqnarray*}
\pause
Por lo tanto, el conjunto de tres vectores $\left\{ \va{b}_{1}, \va{b}_{2}, \va{b}_{3} \right\}$ son no coplanares. \pause Esto implica que podremos obtener un conjunto de vectores recíprocos.
\end{frame}
\begin{frame}
\frametitle{Segundo ejercicio}
Halla el conjunto recíproco de vectores no coplanares del ejercicio anterior.
\\
\bigskip
\pause
Recordemos que el conjunto recíproco de vectores no coplanares, está dada por las expresiones:
\begin{eqnarray*}
\va{b}_{1}^{*} = \pause \dfrac{\va{b}_{2} \times \va{b}_{3}}{\va{b}_{1} \cdot \va{b}_{2} \times \va{b}_{3}} = \pause
\dfrac{1}{30} \, \mqty|
\vu{i} & \vu{j} & \vu{k} \\
-3 & -4 & 2 \\
2 & -1 & -1| = \pause \dfrac{1}{30} \, [6 \, \vu{i} + \vu{j} + 11 \, \vu{k}]
\end{eqnarray*}
\end{frame}
\begin{frame}
\frametitle{Conjunto recíproco}
\vspace*{-1.5cm}
\begin{eqnarray*}
\va{b}_{2}^{*} = \pause \dfrac{\va{b}_{3} \times \va{b}_{1}}{\va{b}_{1} \cdot \va{b}_{2} \times \va{b}_{3}} = \pause
\dfrac{1}{30} \, \mqty|
\vu{i} & \vu{j} & \vu{k} \\
-2 & -1 & -1 \\
1 & 2 & 2| = \pause \dfrac{1}{30} \, [- 5 \, \vu{j} + 5 \, \vu{k}]
\end{eqnarray*}
\pause
\begin{eqnarray*}
\va{b}_{3}^{*} = \pause \dfrac{\va{b}_{1} \times \va{b}_{2}}{\va{b}_{1} \cdot \va{b}_{2} \times \va{b}_{3}} = \pause
\dfrac{1}{30} \, \mqty|
\vu{i} & \vu{j} & \vu{k} \\
1 & 2 & 2 \\
-3 & -4 & 2| = \pause \dfrac{1}{30} \, [12 \, \vu{i} - 8 \, \vu{j} + 2 \, \vu{k}]
\end{eqnarray*}
\end{frame}
\begin{frame}
\frametitle{Ejercicio 3}
Expresar los vectores:
\begin{align*}
\va{A} &= 2 \, \vu{i} - 4 \, \vu{j} + 3 \, \vu{k} \\[0.5em]
\va{B} &= -3 \, \vu{i} +2 \, \vu{j} - \vu{k}
\end{align*}
\pause
a) en función de una suma lineal de los vectores no coplanares del ejercicio 1, y b) en función de una suma lineal de sus vectores recíprocos.
\end{frame}
\begin{frame}
\frametitle{Lo que debemos de expresar}
El vector $\va{A}$ expresado en términos de los vectores no coplanares es de la forma:
\begin{align*}
\va{A} = \alpha_{1}^{*} \, \va{b}_{1} + \alpha_{2}^{*} \, \va{b}_{2} + \alpha_{3}^{*} \, \va{b}_{3}
\end{align*}
\pause
donde los coeficientes $\alpha_{i}^{*}$ son:
\pause
\begin{align*}
\alpha_{i}^{*} = \va{A} \cdot \va{b}_{i}^{*}
\end{align*}
\end{frame}
\begin{frame}
\frametitle{Los coeficientes $\alpha_{i}^{*}$}
Entonces:
\begin{eqnarray*}
\alpha_{1}^{*} &=& \pause \va{A} \cdot \va{b}_{1}^{*} = \\[0.5em] \pause
&=& \bigg( 2 \, \vu{i} - 4 \, \vu{j} + 3 \, \vu{k} \bigg) \cdot \bigg( \dfrac{1}{30} \, [6 \, \vu{i} + \vu{j} + 11 \, \vu{k}] \bigg) = \\[0.5em] \pause
&=& \dfrac{41}{30}
\end{eqnarray*}
\end{frame}
\begin{frame}
\frametitle{Los coeficientes $\alpha_{i}^{*}$}
Entonces:
\begin{eqnarray*}
\alpha_{2}^{*} &=& \pause \va{A} \cdot \va{b}_{2}^{*} =  \pause \dfrac{7}{6} \\[1em] \pause
\alpha_{3}^{*} &=& \pause \va{A} \cdot \va{b}_{3}^{*} =  \pause \dfrac{31}{15}
\end{eqnarray*}
\end{frame}
\begin{frame}
\frametitle{La expresión solicitada}
Por lo que podemos escribir:
\begin{eqnarray*}
\va{A} &=&  \alpha_{1}^{*} \, \va{b}_{1} + \alpha_{2}^{*} \, \va{b}_{2} + \alpha_{3}^{*} \, \va{b}_{3} = \\[0.5em] \pause
&=&  \dfrac{41}{30} \, \va{b}_{1} + \dfrac{7}{6} \, \va{b}_{2} + \dfrac{31}{15} \, \va{b}_{3}
\end{eqnarray*}
\end{frame}
\begin{frame}
\frametitle{Inciso b)}
Para expresar al vector $\va{A}$ en términos del conjunto recíproco de vectores, tenemos que:
\pause
\begin{align*}
\va{A} &=  \alpha_{1} \, \va{b}_{1}^{*} + \alpha_{2} \, \va{b}_{2}^{*} + \alpha_{3} \, \va{b}_{3}^{*}
\end{align*}
\pause
donde los coeficientes son:
\pause
\begin{align*}
\alpha_{i} = \va{A} \cdot \va{b}_{i}
\end{align*}
\end{frame}
\begin{frame}
\frametitle{Los coeficientes $\alpha_{i}$}
Entonces los coeficientes se calculan:
\begin{eqnarray*}
\alpha_{1} &=& \va{A} \cdot \va{b}_{1} = \pause \bigg( 2 \, \vu{i} - 4 \, \vu{j} + 3 \, \vu{k} \bigg) \cdot \bigg( \vu{i} + 2 \, \vu{j} + 2 \, \vu{k} \bigg) = \pause 0 \\[0.5em] \pause
\alpha_{2} &=& \va{A} \cdot \va{b}_{2} = \pause 16 \\[0.5em] \pause
\alpha_{3} &=& \va{A} \cdot \va{b}_{3} = \pause 5
\end{eqnarray*}
\end{frame}
\begin{frame}
\frametitle{El vector $\va{A}$}
Entonces el vector $\va{A}$ en término del conjunto recíproco de vectores se presenta como:
\pause
\begin{align*}
\va{A} = 16 \, \va{b}_{2}^{*} + 5 \, \va{b}_{3}^{*}
\end{align*}
\end{frame}
\begin{frame}
\frametitle{Inciso b) vector $\va{B}$ }
Con el vector $\va{B} = -3 \, \vu{i} +2 \, \vu{j} - \vu{k}$, seguimos el mismo procedimiento: \pause en términos de los vectores no coplanares: \pause
\begin{align*}
\va{B} = \alpha_{1}^{*} \, \va{b}_{1} + \alpha_{2}^{*} \, \va{b}_{2} + \alpha_{3}^{*} \, \va{b}_{3}
\end{align*}
\pause
donde $\alpha_{i}^{*} = \va{B} \cdot \va{b}_{i}^{*}$
\end{frame}
\begin{frame}
\frametitle{Los coeficientes $\alpha_{i}^{*}$}
Tenemos que:
\begin{eqnarray*}
\alpha_{1}^{*} = \va{B} \cdot \va{b}_{1}^{*} = \pause - \dfrac{9}{10} \\[0.5em] \pause
\alpha_{2}^{*} = \va{B} \cdot \va{b}_{2}^{*} = \pause - \dfrac{1}{2} \\[0.5em] \pause
\alpha_{3}^{*} = \va{B} \cdot \va{b}_{3}^{*} = \pause - \dfrac{9}{5}
\end{eqnarray*}
\end{frame}
\begin{frame}
\frametitle{El vector $\va{B}$}
Entonces el vector $\va{B}$ en términos de los vectores no coplanares se escribe como:
\begin{align*}
\va{B} = - \dfrac{9}{10} \, \va{b}_{1} - \dfrac{1}{2} \va{b}_{2} -\dfrac{9}{5} \, \va{b}_{3}
\end{align*}
\end{frame}
\begin{frame}
\frametitle{En términos del conjunto recíproco}
Ahora en términos del conjunto recíproco de vectores:
\begin{align*}
\va{B} = \alpha_{1} \, \va{b}_{1}^{*} + \alpha_{2} \, \va{b}_{2}^{*} + \alpha_{3} \, \va{b}_{3}^{*}
\end{align*}
\pause
donde los coeficientes son:
\begin{align*}
\alpha_{i} = \va{B} \cdot \va{b}_{i}
\end{align*}
\end{frame}
\begin{frame}
\frametitle{Los coeficientes $\alpha_{i}$}
Encontramos que:
\begin{eqnarray*}
\alpha_{1} &=& \va{B} \cdot \va{b}_{1} = \pause -1 \\[0.5em] \pause
\alpha_{2} &=& \va{B} \cdot \va{b}_{2} = \pause -1 \\[0.5em] \pause
\alpha_{3} &=& \va{B} \cdot \va{b}_{3} = \pause -7
\end{eqnarray*}
\end{frame}
\begin{frame}
\frametitle{Resultado}
El vector $\va{B}$ se escribe como:
\pause
\begin{align*}
\va{B} = - \va{b}_{1}^{*} - \va{b}_{2}^{*} - 7 \, \va{b}_{3}^{*} 
\end{align*}
\end{frame}

\section{Notación de índices}
\frame{\tableofcontents[currentsection, hideothersubsections]}
\subsection{Triple producto escalar}

\begin{frame}
\frametitle{Ejercicio 1}
Mediante notación de índices demuestra que:
\begin{align*}
\va{A} \cdot \bigg( \va{B} \times \va{C} \bigg) = \va{B} \cdot \bigg( \va{C} \times \va{A} \bigg)
\end{align*}
\pause
Un camino largo para demostrar esta identidad, es realizar las correspondientes operaciones tanto del lado izquierdo como del lado derecho de la igualdad, y encontrar que son las mismas.
\end{frame}
\begin{frame}
\frametitle{Usando la notación de índices}
Veremos que con la notación de índices, será más rápido y fácil demostrar la identidad.
\end{frame}
\begin{frame}
\frametitle{Usando la notación de índices}
Hagamos que:
\begin{eqnarray*}
\va{B} \times \va{C} = \va{D} = \pause D_{i} \, \vu{e}_{i}
\end{eqnarray*}
\pause
donde:
\begin{align*}
D_{i} = e_{ijk} \, B_{j} \, C_{k}
\end{align*}
los índices tienen el rango de $1, 2, 3$.
\end{frame}
\begin{frame}
\frametitle{Usando la notación de índices}
Del mismo modo, hacemos que:
\begin{eqnarray*}
\va{C} \times \va{A} = \va{F} = \pause F_{i} \, \vu{e}_{i}
\end{eqnarray*}
\pause
donde:
\begin{align*}
F_{i} = e_{ijk} \, C_{j} \, A_{k}
\end{align*}
los índices tienen el rango de $1, 2, 3$.
\end{frame}
\begin{frame}
\frametitle{Demostrando la identidad}
Para demostrar la identidad, revisemos que:
\begin{eqnarray*}
\va{A} \cdot \bigg( \va{B} \times \va{C} \bigg) &=& \pause \va{A} \cdot \va{D} = \pause A_{i} \, D_{i} = \\[0.45em] \pause
&=& A_{i} \, e_{ijk} \, B_{j} \, C_{k} = \\[0.45em] \pause
&=& B_{j} \, \bigg( e_{ijk} \, A_{i} \, C_{k} \bigg) = \\[0.45em] \pause
&=& B_{j} \, \bigg( e_{jki} \, C_{k} \, A_{i} \bigg)
\end{eqnarray*}
\pause
\begin{tikzpicture}[overlay, block1/.style={
    rectangle,
    draw=blue,
    thick,
    fill=blue!20,
    text width=8em,
    align=center,
    rounded corners,
    minimum height=2em
  },]
\draw (3, 1) node [block1] {ya que $e_{ijk} = e_{jki}$};
\end{tikzpicture}
\end{frame}
\begin{frame}
\frametitle{Ocupando la otra expresión}
De la expresión $F_{i} = e_{ijk} \, C_{j} \, A_{k}$, \pause permutando los símbolos podemos obtener la siguiente expresión equivalente:
\pause
\begin{align*}
F_{j} = e_{jki} \, C_{k} \, A_{i}
\end{align*}
\end{frame}
\begin{frame}
\frametitle{Llegando al resultado}
Entonces podemos escribir que:
\pause
\begin{eqnarray*}
\va{A} \cdot \bigg( \va{B} \times \va{C} \bigg) &=& \pause B_{j} \, F_{j} = \\[0.5em] \pause
&=& \va{B} \cdot \va{F} = \\[0.5em] \pause
&=& \va{B} \cdot \bigg( \va{C} \times \va{A} \bigg) 
\end{eqnarray*}
que es lo que se quería demostrar.
\end{frame}

\subsection{Productos vectoriales}

\begin{frame}
\frametitle{Ejercicio 2}
Demuestra que:
\begin{align*}
\bigg( \va{A} \times \va{B} \bigg) \times &\bigg( \va{C} \times \va{D} \bigg) = \\[1em] 
&= \va{C} \bigg( \va{D} \cdot \va{A} \times \va{B} \bigg) - \va{D} \bigg( \va{C} \cdot \va{A} \times \va{B} \bigg)
\end{align*}
\end{frame}
\begin{frame}
\frametitle{Simplificando con la notación de índices}
El demostrar este ejercicio \enquote{a mano} es más laborioso que el anterior, por lo que nos conviene utilizar la notación de índices y algunas identidades que se indican en las notas de trabajo.
\end{frame}
\begin{frame}
\frametitle{Definiendo productos vectoriales}
Comenzamos haciendo que:
\begin{align*}
\va{F} &= \va{A} \times \va{B} = F_{i} \, \vu{e}_{i} \\[1em]
\va{E} &= \va{C} \times \va{D} = E_{i} \, \vu{e}_{i}
\end{align*}
\end{frame}
\begin{frame}
\frametitle{Componentes}
Los vectores anteriores tienen las componentes:
\begin{align*}
F_{i} &= e_{ijk} \, A_{j} \, B_{k} \\[1em] 
E_{i} &= e_{mnp} \, C_{n} \, D_{p}
\end{align*}
donde todos los índices tienen el rango $1, 2, 3$.
\end{frame}
\begin{frame}
\frametitle{Otro producto vectorial}
Definimos el vector $\va{G}$ como:
\pause
\begin{align*}
\va{G} = \va{F} \times \va{E} = G_{i} \, \vu{e}_{i}
\end{align*}
\pause
que tiene las componentes:
\pause
\begin{eqnarray*}
G_{q} = e_{qim} \, F_{i} \, E_{m} = \pause e_{qim} \, e_{ijk} \, e_{mnp} \, A_{j} \, B_{k} \, C_{n} \, D_{p}
\end{eqnarray*}
\end{frame}
\begin{frame}
\frametitle{Usando una identidad}
De la identidad $e_{qim} = e_{mqi}$, podemos expresar:
\pause
\begin{align*}
G_{q} = (e_{mqi} \, e_{mnp}) \, e_{ijk} \, A_{j} \, B_{k} \, C_{n} \, D_{p}
\end{align*}
\pause
que ahora tiene la forma en donde podemos ocupar la identidad $e-\delta$, sobre el término del paréntesis.
\end{frame}
\begin{frame}
\frametitle{Expresión que se obtiene}
Se tiene que:
\pause
\begin{eqnarray*}
\!\!
G_{q} &=& (\delta_{qn} \, \delta_{ip} - \delta_{qp} \, \delta_{in}) \, e_{ijk} \, A_{j} \, B_{k} \, C_{n} \, D_{p} = \\[0.5em] \pause
&=& e_{ijk} \, \bigg( \delta_{qn} \, \delta_{ip} \, A_{j} \, B_{k} \, C_{n} \, D_{p} - \delta_{qp} \, \delta_{in} \, A_{j} \, B_{k} \, C_{n} \, D_{p} \bigg) = \\[0.5em] \pause
&=& e_{ijk} \, \bigg[ (D_{p} \, \delta_{ip}) (C_{n} \, \delta_{qn}) \, A_{j} \, B_{k} - (D_{p} \, \delta_{qp}) (C_{n} \, \delta_{in}) \, A_{j} \, B_{k} \bigg] = \\[0.5em] \pause
&=& e_{ijk} \bigg[ D_{i} \, C_{q} \, A_{j} \, B_{k} {-} D_{q} \, C_{i} \, A_{j} \, B_{k} \bigg] =
\end{eqnarray*}
\end{frame}
\begin{frame}
\frametitle{Resultado}
\begin{align*}
= C_{q} \, \bigg[ D_{i} \, e_{ijk} \, A_{j} \, B_{k} \bigg] - D_{q} \, \bigg[ C_{i} \, e_{ijk} \, A_{j} \, B_{k} \bigg]
\end{align*}
\pause
Que son las componentes del vector:
\pause
\begin{align*}
\va{C} \bigg( \va{D} \cdot \va{A} \times \va{B} \bigg) - \va{D} \bigg( \va{C} \cdot \va{A} \times \va{B} \bigg) \qed
\end{align*}
\end{frame}
\end{document}