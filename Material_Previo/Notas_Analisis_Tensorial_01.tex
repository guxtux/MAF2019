\documentclass[14pt]{extarticle}
\usepackage[utf8]{inputenc}
\usepackage[T1]{fontenc}
\usepackage[spanish,es-lcroman]{babel}
\usepackage{amsmath}
\usepackage{amsthm}
\usepackage{physics}
\AtBeginDocument{\RenewCommandCopy\qty\SI}
\usepackage{tikz}
\usepackage{float}
\usepackage{calc}
\usepackage[autostyle,spanish=mexican]{csquotes}
\usepackage[per-mode=symbol]{siunitx}
\usepackage{textcomp, gensymb}
\usepackage{multicol}
\usepackage{enumitem}
\usepackage{setspace}
\usepackage[left=2.00cm, right=2.00cm, top=2.00cm, 
     bottom=2.00cm]{geometry}
% \usepackage{Estilos/ColoresLatex}
\usepackage{makecell}
\usepackage{subcaption}
\usepackage[skip=10pt, indent=30pt]{parskip}
% \usepackage{scalerel}
\usepackage{scalerel}[2016-12-29]
\def\stretchint#1{\vcenter{\hbox{\stretchto[440]{\displaystyle\int}{#1}}}}
\def\scaleint#1{\vcenter{\hbox{\scaleto[3ex]{\displaystyle\int}{#1}}}}
\def\scaleiint#1{\vcenter{\hbox{\scaleto[6ex]{\displaystyle\iint}{#1}}}}
\def\scaleiiint#1{\vcenter{\hbox{\scaleto[6ex]{\displaystyle\iiint}{#1}}}}
\def\scaleoint#1{\vcenter{\hbox{\scaleto[3ex]{\displaystyle\oint}{#1}}}}
\def\bs{\mkern-12mu}

\newcommand{\textocolor}[2]{\textbf{\textcolor{#1}{#2}}}
\sisetup{per-mode=symbol}
\decimalpoint
\sisetup{bracket-numbers = false}
\newlength{\depthofsumsign}
\setlength{\depthofsumsign}{\depthof{$\sum$}}
\newcommand{\nsum}[1][1.4]{% only for \displaystyle
    \mathop{%
        \raisebox
            {-#1\depthofsumsign+1\depthofsumsign}
            {\scalebox
                {#1}
                {$\displaystyle\sum$}%
            }
    }
}

\ExplSyntaxOn
\msg_redirect_name:nnn { siunitx } { physics-pkg } { none }
\ExplSyntaxOff

% \documentclass[hidelinks,12pt]{book}
%\usepackage[left=0.25cm,top=1cm,right=0.25cm,bottom=1cm]{geometry}
\usepackage{geometry}
% \usepackage[landscape]{geometry}
% \textwidth = 20cm
% \hoffset = -1cm
\usepackage[utf8]{inputenc}
\usepackage[spanish,es-tabla, es-lcroman]{babel}
\usepackage[autostyle,spanish=mexican]{csquotes}
\usepackage[tbtags]{amsmath}
\usepackage{nccmath}
\usepackage{amsthm}
\usepackage{amssymb}
\usepackage{mathrsfs}
\usepackage{graphicx}
\usepackage{subfig}
\usepackage{caption}
%\usepackage{subcaption}
\usepackage{standalone}
\graphicspath{{Imagenes/}{../Imagenes/}}
\usepackage[outdir=./Imagenes/]{epstopdf}
\usepackage{siunitx}
\usepackage{physics}
\usepackage{color}
\usepackage{float}
\usepackage{hyperref}
\usepackage{multicol}
\usepackage{multirow}
%\usepackage{milista}
\usepackage{anyfontsize}
\usepackage{anysize}
%\usepackage{enumerate}
\usepackage[shortlabels]{enumitem}
\usepackage{capt-of}
\usepackage{bm}
\usepackage{mdframed}
\usepackage{relsize}
\usepackage{placeins}
\usepackage{empheq}
\usepackage{cancel}
\usepackage{pdfpages}
\usepackage{wrapfig}
\usepackage[flushleft]{threeparttable}
\usepackage{makecell}
\usepackage{fancyhdr}
\usepackage{tikz}
\usepackage{bigints}
\usepackage{tcolorbox}
\tcbuselibrary{breakable}
\usepackage{scalerel}
\usepackage{pgfplots}
\usepackage{pdflscape}
\usepackage{enumitem}
\pgfplotsset{compat=1.16}
\spanishdecimal{.}
\renewcommand{\baselinestretch}{1.5}
\renewcommand{\labelenumii}{\arabic{enumi}.\arabic{enumii}}
\renewcommand{\labelenumiii}{\arabic{enumi}.\arabic{enumii}.\arabic{enumiii}}

\newcommand{\ptilde}[1]{\ensuremath{{#1}^{\prime}}}
\newcommand{\stilde}[1]{\ensuremath{{#1}^{\prime \prime}}}
\newcommand{\ttilde}[1]{\ensuremath{{#1}^{\prime \prime \prime}}}
\newcommand{\ntilde}[2]{\ensuremath{{#1}^{(#2)}}}
\newcommand{\pderivada}[1]{\ensuremath{{#1}^{\prime}}}
\newcommand{\sderivada}[1]{\ensuremath{{#1}^{\prime \prime}}}
\newcommand{\tderivada}[1]{\ensuremath{{#1}^{\prime \prime \prime}}}
\newcommand{\nderivada}[2]{\ensuremath{{#1}^{(#2)}}}


\newtheorem{defi}{{\it Definición}}[section]
\newtheorem{teo}{{\it Teorema}}[section]
\newtheorem{ejemplo}{{\it Ejemplo}}[section]
\newtheorem{propiedad}{{\it Propiedad}}[section]
\newtheorem{lema}{{\it Lema}}[section]
\newtheorem{cor}{Corolario}
\newtheorem{ejer}{Ejercicio}[section]

\newlist{milista}{enumerate}{2}
\setlist[milista,1]{label=\arabic*)}
\setlist[milista,2]{label=\arabic{milistai}.\arabic*)}
\newlength{\depthofsumsign}
\setlength{\depthofsumsign}{\depthof{$\sum$}}
\newcommand{\nsum}[1][1.4]{% only for \displaystyle
    \mathop{%
        \raisebox
            {-#1\depthofsumsign+1\depthofsumsign}
            {\scalebox
                {#1}
                {$\displaystyle\sum$}%
            }
    }
}
\def\scaleint#1{\vcenter{\hbox{\scaleto[3ex]{\displaystyle\int}{#1}}}}
\def\scaleoint#1{\vcenter{\hbox{\scaleto[3ex]{\displaystyle\oint}{#1}}}}
\def\scaleiiint#1{\vcenter{\hbox{\scaleto[3ex]{\displaystyle\iiint}{#1}}}}
\def\bs{\mkern-12mu}

\newcommand{\Cancel}[2][black]{{\color{#1}\cancel{\color{black}#2}}}

\AtBeginDocument{\RenewCommandCopy\qty\SI}
% \usepackage{apacite}

\title{Análisis tensorial \\ \large{Material de consulta previo}\vspace{-3ex}}
\author{M. en C. Gustavo Contreras Mayén}
\date{ }

\numberwithin{equation}{section}

\linespread{1.25}

\begin{document}

\vspace{-4cm}
\maketitle
\fontsize{14}{14}\selectfont
\tableofcontents
\newpage

%Ref. Arfken - Tensor Analysis. Cap. 4 (2013)
\section{Introducción.}

Los tensores son de gran importancia en muchas áreas de la física, incluyendo la relatividad general y la teoría electromagnética. Una de las fuentes más prolíficas de cantidades tensoriales es el sólido anisotrópico. Aquí, las propiedades elásticas, ópticas, eléctricas y magnéticas pueden implicar a los tensores. Como ilustración introductoria, consideremos el flujo de una corriente eléctrica. Podemos establecer la ley de Ohm en la forma usual:
\begin{align}
\vb{J} = \sigma \ \vb{E}
\label{eq:ecuacion_03_01}
\end{align}
con la densidad de corriente $\vb{J}$ y el campo eléctrico $\vb{E}$, ambas cantidades vectoriales. Si se tiene un medio isotrópico, $\sigma$, la conductividad, es un escalar, y para el componente $x$, por ejemplo:
\begin{align}
J_{1} = \sigma \, E_{1}
\label{eq:ecuacion_03_02}
\end{align}
Sin embargo, si nuestro medio es anisotrópico, igual que en muchos cristales, o un plasma en presencia de un campo magnético, la densidad de corriente en la dirección $x$ puede depender de los campos eléctricos en las direcciones $y$ y $z$ así como del campo en la dirección $x$. Suponiendo una relación lineal, debemos sustituir la ec. (\ref{eq:ecuacion_03_02}) con
\begin{align}
J_{1} = \sigma_{11} \, E_{1} + \sigma_{12} \, E_{2} + \sigma_{13} \, E_{3}
\label{eq:ecuacion_03_03}
\end{align}
y en general:
\begin{align}
J_{i} = \nsum_{k} \sigma_{ik} \, E_{k}
\label{eq:ecuacion_03_04}
\end{align}
Para el espacio ordinario tridimensional, la conductividad escalar $\sigma$ ha dado lugar a un conjunto de nueve elementos:
\begin{align}
\sigma \rightarrow \begin{pmatrix}
\sigma_{11} & \sigma_{12} & \sigma_{13} \\
\sigma_{21} & \sigma_{22} & \sigma_{23} \\
\sigma_{31} & \sigma_{32} & \sigma_{33}
\end{pmatrix}
\label{eq:ecuacion_03_05}
\end{align}
Este arreglo de nueve elementos forma de hecho un tensor.
\par
Una cantidad que permanece sin cambiar bajo las rotaciones de un sistema coordenado, es decir una cantidad invariante, se denomina \textbf{escalar}. Una cantidad cuyos componentes transformados como son aquellos de la distancia de un punto desde un origen seleccionado se denomina \textbf{vector}. 
\par
Si llamamos a los escalares como \textbf{tensores de rango 0} y a los vectores como \textbf{tensores de rango 1}, identificamos un tensor de rango $n$ en un espacio de dimensión $d$ como un objeto con las siguientes propiedades:
\begin{itemize}
\item Tiene componentes etiquetados por $n$ índices, a cada índice se le asignan valores de 1 a $d$, y por lo tanto tiene un total de $d^{n}$ componentes.
\item Los componentes se transforman de una manera específica bajo transformaciones de coordenadas.
\end{itemize}
El comportamiento bajo la transformación de coordenadas es de importancia central para el análisis de tensores y se ajusta tanto a la forma en que los matemáticos definen los espacios lineales como a la noción de los físicos de que los observables físicos no deben depender de la elección de los marcos de coordenadas.

\section{Tensores covariantes y contravariantes.}

Consideremos una trasformación rotacional de un vector:
\begin{align*}
\vb{A} = A_{1} \, \vu{e}_{1} + A_{2} \, \vu{e}_{2} + A_{3} \, \vu{e}_{3}
\end{align*}
de un sistema cartesiano definido por $\vu{e}_{i} \, (i = 1, 2, 3)$ a un sistema coordenado rotado definido por $\vu{e}_{i}^{\prime}$; el mismo vector $\vb{A}$ se representa entonces como:
\begin{align*}
\vb{A}^{\prime} = A_{1}^{\prime} \, \vu{e}_{1}^{\prime} + A_{2}^{\prime} \, \vu{e}_{2}^{\prime} + A_{3}^{\prime} \, \vu{e}_{3}^{\prime}
\end{align*}
Las componentes de $\vb{A}$ y $\vb{A}^{\prime}$ se relacionan por:
\begin{align}
A_{i}^{\prime} = \nsum_{j} \left( \vu{e}_{i}^{\prime} \cdot \vu{e}_{j} \right) \, A_{j}
\label{eq:ecuacion_04_01}
\end{align}
donde los coeficientes $\left( \vu{e}_{i}^{\prime} \cdot \vu{e}_{j} \right)$ son las proyecciones de $\vu{e}_{i}^{\prime}$ en las direcciones $\vu{e}_{j}$. Dado que $\vu{e}_{i}$ y $\vu{e}_{j}$ están linealmente relacionados, podemos escribir:
\begin{align}
A_{i}^{\prime} = \nsum \pdv{x_{i}^{\prime}}{x_{j}} \, A_{j}
\label{eq:ecuacion_04_02}
\end{align}
La fórmula de la ecuación (\ref{eq:ecuacion_04_02}) corresponde a la aplicación de la regla de la cadena para convertir el conjunto $A_{j}$ en el conjunto $A_{i}^{\prime}$, y es válida para $A_{j}$ y $A_{i}^{\prime}$ de magnitud arbitraria porque ambos vectores dependen linealmente de sus componentes.
\par
También hemos observado anteriormente que el gradiente de un escalar $\phi$ tiene en las coordenadas cartesianas no rotadas las componentes:
\begin{align*}
\left( \nabla \phi \right)_{j} = \left( \pdv{\phi}{x_{j}} \right) \, \vu{e}_{j}
\end{align*}
lo que significa que en un sistema rotado tendríamos:
\begin{align}
\left( \nabla \phi \right)_{i}^{\prime} \equiv \pdv{\phi}{x_{i}^{\prime}} = \nsum_{j} \pdv{x_{j}}{x_{i}^{\prime}} \pdv{\phi}{x_{j}}
\label{eq:ecuacion_04_03}
\end{align}
mostrando que el gradiente tiene una ley de transformación que difiere de la de la ecuación (\ref{eq:ecuacion_04_02}) en que $\pdv*{x_{i}^{\prime}}{x_{j}}$ ha sido reemplazado por $\pdv*{x_{j}}{x_{i}^{\prime}}$.
\par
Recordando que estas dos expresiones, si se escriben en detalle, corresponden, respectivamente, a $\left( \pdv*{x_{i}^{\prime}}{x_{j}} \right)_{x_{k}}$ y $\left( \pdv*{x_{j}}{x_{i}^{\prime}} \right)_{x_{k}^{\prime}}$, donde $k$ se extiende sobre los valores de índice distintos del que ya está en el denominador, y notando también que (en coordenadas cartesianas) son dos formas diferentes de calcular la misma cantidad (la magnitud y el signo de la proyección de uno de estos vectores unitarios sobre el otro), vemos que es legítimo identificar tanto a $\vb{A}$ como a $\grad{\phi}$ como vectores.
\par
Sin embargo, como un lector atento se puede notar por la inserción repetida de la palabra \enquote{cartesiano}, las derivadas parciales en las ecuaciones (\ref{eq:ecuacion_04_02}) y (\ref{eq:ecuacion_04_03}) solo están garantizadas como iguales en sistemas de coordenadas cartesianos, y dado que a veces es necesario usar sistemas no cartesianos, se hace necesario distinguir estas dos reglas de transformación diferentes. Las cantidades que se transforman de acuerdo con la ecuación (\ref{eq:ecuacion_04_02}) se denominan \textbf{vectores contravariantes}, mientras que las que se transforman de acuerdo con la ecuación (\ref{eq:ecuacion_04_03}) se denominan \textbf{covariantes}. Cuando pueden estar en juego sistemas no cartesianos, es habitual distinguir estas propiedades de transformación escribiendo el índice de un vector contravariante como un superíndice y el de un vector covariante como un subíndice. Esto significa, entre otras cosas, que los componentes del vector de posición $\vb{r}$, que sea contravariante, deberá de escribirse $\left( x^{1}, x^{2}, x^{3} \right)$. Resumiendo:
\begin{align}
\left( A^{\prime} \right)^{i} &= \nsum_{j} \pdv{(x^{\prime})^{i}}{x^{j}} \, A^{j} \hspace{1cm} \vb{A}, \text{vector contravariante}
\label{eq:ecuacion_04_04} \\
A_{i}^{\prime} &= \nsum_{j} \pdv{x^{j}}{(x^{\prime})^{i}} \, A_{j} \hspace{1cm} \vb{A}, \text{vector covariante}
\label{eq:ecuacion_04_05}
\end{align}

Es útil observar que la aparición de subíndices y superíndices es sistemática; el índice libre (es decir, no sumado) $i$ aparece como superíndice en ambos lados de la ec. (\ref{eq:ecuacion_04_04}), mientras que aparece como subíndice en ambos lados de la ec. (\ref{eq:ecuacion_04_05}), si interpretamos un índice superior en el denominador como equivalente a un índice inferior. El índice sumado aparece una vez como superior y una vez como inferior (tratando nuevamente un índice superior en el denominador como un índice inferior). Una abreviatura que se usa con frecuencia (la \textbf{convención de Einstein}) es omitir el signo de suma en fórmulas como las ecs. (\ref{eq:ecuacion_04_04}) y (\ref{eq:ecuacion_04_05}) y entender que cuando el mismo símbolo aparece como índice superior e inferior en la misma expresión, debe sumarse.

\section{Tensores de rango 2.}

Ahora definamos los tensores contravariantes, mixtos y covariantes de rango 2 mediante las siguientes ecuaciones para sus componentes bajo transformaciones de coordenadas:
\begin{align}
\begin{aligned}
\left( A^{\prime} \right)^{ij} &= \nsum_{kl} \pdv{\left( x^{\prime} \right)^{i}}{x^{k}} \pdv{\left( x^{\prime} \right)^{j}}{x^{l}} \, A^{kl} \\[0.5em]
\left( B^{\prime} \right)_{j}^{i} &= \nsum_{kl} \pdv{\left( x^{\prime} \right)^{i}}{x^{k}} \pdv{x^{l}}{\left( x^{\prime} \right)^{j}} \, B_{l}^{k} \\[0.5em]
\left( C^{\prime} \right)_{ij} &= \nsum_{kl} \pdv{x^{k}}{\left( x^{\prime} \right)^{i}} \pdv{x^{l}}{\left( x^{\prime} \right)^{j}} \, C_{kl}
\end{aligned}
\label{eq:ecuacion_04_06}
\end{align}

Claramente, el rango va como el número de derivadas parciales (o cosenos directores) en la definición: $0$ para un escalar, $1$ para un vector, $2$ para un tensor de segundo rango, y así sucesivamente. Cada índice (subíndice o superíndice) varía sobre el número de dimensiones del espacio. El número de índices (igual al rango del tensor) no está limitado por la dimensionalidad del espacio. Vemos que $A^{kl}$ es contravariante con respecto a ambos índices, $C_{kl}$ es covariante con respecto a ambos índices y $B_{l}^{k}$ se transforma contravariantemente con respecto al índice $k$ pero covariantemente con respecto al índice $l$. Una vez más, si estamos usando coordenadas cartesianas, las tres formas de los tensores de segundo rango, contravariante, mixta y covariante son las mismas.
\par
Al igual que con los componentes de un vector, las leyes de transformación para los componentes de un tensor, ec. (\ref{eq:ecuacion_04_06}), hacen que sus propiedades físicamente relevantes sean independientes de la elección del marco de referencia. Esto es lo que hace que el análisis tensorial sea importante en física. La independencia relativa al marco de referencia (invariancia) es ideal para expresar e investigar leyes físicas universales.
\par
El tensor de segundo rango $A$ (con componentes $A^{kl}$) se puede representar convenientemente escribiendo sus componentes en una matriz cuadrada ($3 \times 3$ si estamos en el espacio tridimensional (3D)):
\begin{align}
A = \begin{pmatrix}
A^{11} & A^{12} & A^{13} \\
A^{21} & A^{22} & A^{23} \\
A^{31} & A^{32} & A^{33}
\end{pmatrix}
\label{eq:ecuacion_04_07}
\end{align}
Esto no significa que cualquier matriz cuadrada de números o funciones forme un tensor. La condición esencial es que los componentes se transformen de acuerdo con la ec. (\ref{eq:ecuacion_04_06}).
\par
Podemos considerar cada una de las ecs. (\ref{eq:ecuacion_04_06}) como una ecuación matricial. Para $A$, toma la forma:
\begin{align}
\left( A^{\prime} \right)^{ij} = \nsum_{kl} S_{ik} \, A^{kl} \, \left( S^{T} \right)_{lj} \hspace{1cm} \text{o} \hspace{1cm} A^{\prime} = S A S^{T}
\label{eq:ecuacion_04_08}
\end{align}
una construcción que se conoce como \textbf{transformación de similitud}.


\section{Suma y resta de tensores.}

La suma y resta de tensores se define en términos de los elementos individuales, al igual que para los vectores. Si:
\begin{align}
A + B = C
\label{eq:ecuacion_04_09}
\end{align}
entonces, tomando como ejemplo $A$, $B$ y $C$ como tensores contravariantes de rango $2$:
\begin{align}
A^{ij} + B^{ij} = C^{ij}
\label{eq:ecuacion_04_10}
\end{align}
En general, por supuesto, $A$ y $B$ deben ser tensores del mismo rango (tanto de contravarianza como de covarianza) y en el mismo espacio.

\section{Simetría.}

El orden en que aparecen los índices en nuestra descripción de un tensor es importante. En general, $A^{mn}$ es independiente de $A^{nm}$, pero hay algunos casos de especial interés. Si, para todos los $m$ y $n$:
\begin{align}
A^{mn} = A^{nm} \hspace{1cm} A \text{ es simétrico}
\label{eq:ecuacion_04_11}
\end{align}
Si, por otro lado:
\begin{align}
A^{mn} = -A^{nm} \hspace{1cm} A \text{ es antisimétrico}
\label{eq:ecuacion_04_12}
\end{align}
Claramente, cada tensor (de segundo rango) se puede resolver en partes simétricas y antisimétricas mediante la identidad:
\begin{align}
A^{mn} = \dfrac{1}{2} \, \left( A^{mn} + A^{nm} \right) + \dfrac{1}{2} \, \left( A^{mn} - A^{nm} \right)
\label{eq:ecuacion_04_13}
\end{align}
El primer término a la derecha es un tensor simétrico, el segundo, un tensor antisimétrico.

\section{Tensores isotrópicos.}

Para ilustrar algunas de las técnicas de análisis tensorial, demostremos que el ahora conocido delta de Kronecker, $\delta_{kl}$, es en realidad un tensor mixto de rango $2$, $\delta_{l}^{k}$. La pregunta es: ¿la $\delta_{l}^{k}$ se transforma según la ec. (\ref{eq:ecuacion_04_06})?, que es nuestro criterio para llamarlo tensor. Si $\delta_{l}^{k}$ es el tensor mixto correspondiente a esta notación, debe satisfacer (usando la convención de suma, lo que significa que los índices $k$ y $l$ deben sumarse):
\begin{align*}
\left( \delta^{\prime} \right)_{j}^{i} = \pdv{\left( x^{\prime} \right)^{i}}{x^{k}} \pdv{x^{l}}{\left( x^{\prime} \right)^{j}} \, \delta_{l}^{k} = \pdv{\left( x^{\prime} \right)^{i}}{x^{k}} \pdv{x^{k}}{\left( x^{\prime} \right)^{j}}
\end{align*}
donde hemos realizado la suma sobre $l$ y hemos utilizado la definición del delta de Kronecker. A continuación:
\begin{align*}
\pdv{\left( x^{\prime} \right)^{i}}{x^{k}} \pdv{x^{k}}{\left( x^{\prime} \right)^{j}} = \pdv{\left( x^{\prime} \right)^{i}}{\left( x^{\prime} \right)^{j}}
\end{align*}
donde hemos identificado la suma sobre $k$ del lado izquierdo como una instancia de la regla de la cadena para la diferenciación. Sin embargo, $\left( x^{\prime} \right)^{i}$ y $\left( x^{\prime} \right)^{j}$ son coordenadas independientes y, por lo tanto, la variación de una con respecto a la otra debe ser cero si son diferentes, y será la unidad si coinciden; es decir:
\begin{align}
\pdv{\left( x^{\prime} \right)^{i}}{\left( x^{\prime} \right)^{j}} = \left( \delta^{\prime} \right)_{j}^{i}
\label{eq:ecuacion_04_14}
\end{align}
demostrando que la $\delta_{l}^{k}$ son de hecho los componentes de un tensor mixto de segundo rango. Nótese que este resultado es independiente del número de dimensiones de nuestro espacio.
\par
El delta de Kronecker tiene otra propiedad interesante. Tiene los mismos componentes en todos nuestros sistemas de coordenadas rotados y por lo tanto se llama \textbf{isotrópico}. No existe ningún tensor (vector) isotrópico de primer rango.

\section{Contracción.}

Al trabajar con vectores, formamos un producto escalar sumando los productos de los componentes correspondientes:
\begin{align*}
\vb{A} \vdot \vb{B} = \nsum_{i} A_{i} \, B_{i}
\end{align*}
La generalización de esta expresión en el análisis tensorial es un proceso conocido como \textbf{contracción}.
\par
Se establecen dos índices, uno covariante y el otro contravariante, iguales entre sí y luego (como lo implica la convención de suma) sumamos sobre este índice repetido. Por ejemplo, contraigamos el tensor mixto de segundo rango $B_{j}^{i}$ estableciendo $j$ en $i$ y luego sumando sobre $i$. Para ver qué sucede, observemos la fórmula de transformación que convierte $B$ en $B^{\prime}$. Usando la convención de suma:
\begin{align*}
\left( B^{\prime} \right)_{i}^{i} = \pdv{\left( x^{\prime} \right)^{i}}{x^{k}} \pdv{x^{l}}{\left( x^{\prime} \right)^{i}} \, B_{l}^{k} = \pdv{x^{l}}{x^{k}} \, B_{l}^{k}
\end{align*}
donde reconocimos la suma sobre $i$ como una instancia de la regla de la cadena para la diferenciación. Entonces, como las $x^{i}$ son independientes, podemos usar la ec. (\ref{eq:ecuacion_04_14}) para llegar a:
\begin{align}
\left( B^{\prime} \right)_{i}^{i} = \delta_{k}^{l} \, B_{l}^{k} = B_{k}^{k}
\label{eq:ecuacion_04_16}
\end{align}
Recordando que el índice repetido ($i$ o $k$) se suma, vemos que el $B$ contraído es invariante bajo transformación y por lo tanto es un escalar. En general, la operación de contracción reduce el rango de un tensor en $2$.
\end{document}