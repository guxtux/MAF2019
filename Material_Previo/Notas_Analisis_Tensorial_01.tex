\documentclass[12pt]{article}
\usepackage[left=0.25cm,top=1cm,right=0.25cm,bottom=1cm]{geometry}
\textwidth = 20cm
\hoffset = -1cm
\usepackage[utf8]{inputenc}
\usepackage[spanish,es-tabla]{babel}
\usepackage[autostyle,spanish=mexican]{csquotes}
\usepackage[tbtags]{amsmath}
\usepackage{nccmath}
\usepackage{amsthm}
\usepackage{amssymb}
\usepackage{graphicx}
\usepackage{standalone}
\usepackage[outdir=./]{epstopdf}
\usepackage{siunitx}
\usepackage{physics}
\usepackage{color}
\usepackage{float}
\usepackage{multicol}
%\usepackage{milista}
\usepackage{enumitem}
\usepackage{anyfontsize}
\usepackage{anysize}
\usepackage{enumitem}
\usepackage{capt-of}
\usepackage{bm}
\usepackage{relsize}
\usepackage{placeins}
\usepackage{empheq}
\usepackage{cancel}
\usepackage{wrapfig}
\spanishdecimal{.}
\renewcommand{\baselinestretch}{1.5} 
\renewcommand\labelenumii{\theenumi.{\arabic{enumii}}}
\newcommand{\ptilde}[1]{\ensuremath{{#1}^{\prime}}}
\newcommand{\stilde}[1]{\ensuremath{{#1}^{\prime \prime}}}
\newcommand{\ttilde}[1]{\ensuremath{{#1}^{\prime \prime \prime}}}
\newcommand{\ntilde}[2]{\ensuremath{{#1}^{(#2)}}}


% \usepackage{apacite}
\title{Análisis tensorial \\ \large{Material de consulta previo}\vspace{-3ex}}
\author{M. en C. Gustavo Contreras Mayén}
\date{ }

\numberwithin{equation}{section}

\begin{document}

\vspace{-4cm}
\maketitle
\fontsize{14}{14}\selectfont
\tableofcontents
\newpage

%Ref. Arfken - Tensor Analysis. Cap. 4 (2013)
\section{Introducción.}

Los tensores son de gran importancia en muchas áreas de la física, incluyendo la relatividad general y la teoría electromagnética. Una de las fuentes más prolíficas de cantidades tensoriales es el sólido anisotrópico. Aquí, las propiedades elásticas, ópticas, eléctricas y magnéticas pueden implicar a los tensores. Como ilustración introductoria, consideremos el flujo de una corriente eléctrica. Podemos establecer la ley de Ohm en la forma usual:
\begin{align}
\vb{J} = \sigma \ \vb{E}
\label{eq:ecuacion_03_01}
\end{align}
con la densidad de corriente $\vb{J}$ y el campo eléctrico $\vb{E}$, ambas cantidades vectoriales. Si se tiene un medio isotrópico, $\sigma$, la conductividad, es un escalar, y para el componente $x$, por ejemplo:
\begin{align}
J_{1} = \sigma \, E_{1}
\label{eq:ecuacion_03_02}
\end{align}
Sin embargo, si nuestro medio es anisotrópico, igual que en muchos cristales, o un plasma en presencia de un campo magnético, la densidad de corriente en la dirección $x$ puede depender de los campos eléctricos en las direcciones $y$ y $z$ así como del campo en la dirección $x$. Suponiendo una relación lineal, debemos sustituir la ec. (\ref{eq:ecuacion_03_02}) con
\begin{align}
J_{1} = \sigma_{11} \, E_{1} + \sigma_{12} \, E_{2} + \sigma_{13} \, E_{3}
\label{eq:ecuacion_03_03}
\end{align}
y en general:
\begin{align}
J_{i} = \nsum_{k} \sigma_{ik} \, E_{k}
\label{eq:ecuacion_03_04}
\end{align}
Para el espacio ordinario tridimensional, la conductividad escalar $\sigma$ ha dado lugar a un conjunto de nueve elementos:
\begin{align}
\sigma \rightarrow \begin{pmatrix}
\sigma_{11} & \sigma_{12} & \sigma_{13} \\
\sigma_{21} & \sigma_{22} & \sigma_{23} \\
\sigma_{31} & \sigma_{32} & \sigma_{33}
\end{pmatrix}
\label{eq:ecuacion_03_05}
\end{align}
Este arreglo de nueve elementos forma de hecho un tensor.
\par
Una cantidad que permanece sin cambiar bajo las rotaciones de un sistema coordenado, es decir una cantidad invariante, se denomina \textbf{escalar}. Una cantidad cuyos componentes transformados como son aquellos de la distancia de un punto desde un origen seleccionado se denomina \textbf{vector}. 
\par
Si llamamos a los escalares como \textbf{tensores de rango 0} y a los vectores como \textbf{tensores de rango 1}, identificamos un tensor de rango $n$ en un espacio de dimensión $d$ como un objeto con las siguientes propiedades:
\begin{itemize}
\item Tiene componentes etiquetados por $n$ índices, a cada índice se le asignan valores de 1 a $d$, y por lo tanto tiene un total de $d^{n}$ componentes.
\item Los componentes se transforman de una manera específica bajo transformaciones de coordenadas.
\end{itemize}
El comportamiento bajo la transformación de coordenadas es de importancia central para el análisis de tensores y se ajusta tanto a la forma en que los matemáticos definen los espacios lineales como a la noción de los físicos de que los observables físicos no deben depender de la elección de los marcos de coordenadas.

\section{Tensores covariantes y contravariantes.}

Consideremos una trasformación rotacional de un vector:
\begin{align*}
\vb{A} = A_{1} \, \vu{e}_{1} + A_{2} \, \vu{e}_{2} + A_{3} \, \vu{e}_{3}
\end{align*}
de un sistema cartesiano definido por $\vu{e}_{i} \, (i = 1, 2, 3)$ a un sistema coordenado rotado definido por $\vu{e}_{i}^{\prime}$; el mismo vector $\vb{A}$ se representa entonces como:
\begin{align*}
\vb{A}^{\prime} = A_{1}^{\prime} \, \vu{e}_{1}^{\prime} + A_{2}^{\prime} \, \vu{e}_{2}^{\prime} + A_{3}^{\prime} \, \vu{e}_{3}^{\prime}
\end{align*}
Las componentes de $\vb{A}$ y $\vb{A}^{\prime}$ se relacionan por:
\begin{align}
A_{i}^{\prime} = \nsum_{j} \left( \vu{e}_{i}^{\prime} \cdot \vu{e}_{j} \right) \, A_{j}
\label{eq:ecuacion_04_01}
\end{align}
donde los coeficientes $\left( \vu{e}_{i}^{\prime} \cdot \vu{e}_{j} \right)$ son las proyecciones de $\vu{e}_{i}^{\prime}$ en las direcciones $\vu{e}_{j}$. Dado que $\vu{e}_{i}$ y $\vu{e}_{j}$ están linealmente relacionados, podemos escribir:
\begin{align}
A_{i}^{\prime} = \nsum \pdv{x_{i}^{\prime}}{x_{j}} \, A_{j}
\label{eq:ecuacion_04_02}
\end{align}
La fórmula de la ecuación (\ref{eq:ecuacion_04_02}) corresponde a la aplicación de la regla de la cadena para convertir el conjunto $A_{j}$ en el conjunto $A_{i}^{\prime}$, y es válida para $A_{j}$ y $A_{i}^{\prime}$ de magnitud arbitraria porque ambos vectores dependen linealmente de sus componentes.
\par
También hemos observado anteriormente que el gradiente de un escalar $\phi$ tiene en las coordenadas cartesianas no rotadas las componentes:
\begin{align*}
\left( \nabla \phi \right)_{j} = \left( \pdv{\phi}{x_{j}} \right) \, \vu{e}_{j}
\end{align*}
lo que significa que en un sistema rotado tendríamos:
\begin{align}
\left( \nabla \phi \right)_{i}^{\prime} \equiv \pdv{\phi}{x_{i}^{\prime}} = \nsum_{j} \pdv{x_{j}}{x_{i}^{\prime}} \pdv{\phi}{x_{j}}
\label{eq:ecuacion_04_03}
\end{align}
mostrando que el gradiente tiene una ley de transformación que difiere de la de la ecuación (\ref{eq:ecuacion_04_02}) en que $\pdv*{x_{i}^{\prime}}{x_{j}}$ ha sido reemplazado por $\pdv*{x_{j}}{x_{i}^{\prime}}$.
\par
Recordando que estas dos expresiones, si se escriben en detalle, corresponden, respectivamente, a $\left( \pdv*{x_{i}^{\prime}}{x_{j}} \right)_{x_{k}}$ y $\left( \pdv*{x_{j}}{x_{i}^{\prime}} \right)_{x_{k}^{\prime}}$, donde $k$ se extiende sobre los valores de índice distintos del que ya está en el denominador, y notando también que (en coordenadas cartesianas) son dos formas diferentes de calcular la misma cantidad (la magnitud y el signo de la proyección de uno de estos vectores unitarios sobre el otro), vemos que era legítimo identificar tanto a $\vb{A}$ como a $grad{\phi}$ como vectores.
\end{document}