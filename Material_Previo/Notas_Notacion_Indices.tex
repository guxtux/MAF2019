\documentclass[14pt]{extarticle}
\usepackage[utf8]{inputenc}
\usepackage[T1]{fontenc}
\usepackage[spanish,es-lcroman]{babel}
\usepackage{amsmath}
\usepackage{amsthm}
\usepackage{physics}
\AtBeginDocument{\RenewCommandCopy\qty\SI}
\usepackage{tikz}
\usepackage{float}
\usepackage{calc}
\usepackage[autostyle,spanish=mexican]{csquotes}
\usepackage[per-mode=symbol]{siunitx}
\usepackage{textcomp, gensymb}
\usepackage{multicol}
\usepackage{enumitem}
\usepackage{setspace}
\usepackage[left=2.00cm, right=2.00cm, top=2.00cm, 
     bottom=2.00cm]{geometry}
% \usepackage{Estilos/ColoresLatex}
\usepackage{makecell}
\usepackage{subcaption}
\usepackage[skip=10pt, indent=30pt]{parskip}
% \usepackage{scalerel}
\usepackage{scalerel}[2016-12-29]

\def\stretchint#1{\vcenter{\hbox{\stretchto[440]{\displaystyle\int}{#1}}}}
\def\scaleint#1{\vcenter{\hbox{\scaleto[3ex]{\displaystyle\int}{#1}}}}
\def\scaleiint#1{\vcenter{\hbox{\scaleto[6ex]{\displaystyle\iint}{#1}}}}
\def\scaleiiint#1{\vcenter{\hbox{\scaleto[6ex]{\displaystyle\iiint}{#1}}}}
\def\scaleoint#1{\vcenter{\hbox{\scaleto[3ex]{\displaystyle\oint}{#1}}}}
\def\bs{\mkern-12mu}

\newcommand{\textocolor}[2]{\textbf{\textcolor{#1}{#2}}}
\sisetup{per-mode=symbol}
\decimalpoint
\sisetup{bracket-numbers = false}
\newlength{\depthofsumsign}
\setlength{\depthofsumsign}{\depthof{$\sum$}}
\newcommand{\nsum}[1][1.4]{% only for \displaystyle
    \mathop{%
        \raisebox
            {-#1\depthofsumsign+1\depthofsumsign}
            {\scalebox
                {#1}
                {$\displaystyle\sum$}%
            }
    }
}

\ExplSyntaxOn
\msg_redirect_name:nnn { siunitx } { physics-pkg } { none }
\ExplSyntaxOff

% \documentclass[hidelinks,12pt]{book}
%\usepackage[left=0.25cm,top=1cm,right=0.25cm,bottom=1cm]{geometry}
\usepackage{geometry}
% \usepackage[landscape]{geometry}
% \textwidth = 20cm
% \hoffset = -1cm
\usepackage[utf8]{inputenc}
\usepackage[spanish,es-tabla, es-lcroman]{babel}
\usepackage[autostyle,spanish=mexican]{csquotes}
\usepackage[tbtags]{amsmath}
\usepackage{nccmath}
\usepackage{amsthm}
\usepackage{amssymb}
\usepackage{mathrsfs}
\usepackage{graphicx}
\usepackage{subfig}
\usepackage{caption}
%\usepackage{subcaption}
\usepackage{standalone}
\graphicspath{{Imagenes/}{../Imagenes/}}
\usepackage[outdir=./Imagenes/]{epstopdf}
\usepackage{siunitx}
\usepackage{physics}
\usepackage{color}
\usepackage{float}
\usepackage{hyperref}
\usepackage{multicol}
\usepackage{multirow}
%\usepackage{milista}
\usepackage{anyfontsize}
\usepackage{anysize}
%\usepackage{enumerate}
\usepackage[shortlabels]{enumitem}
\usepackage{capt-of}
\usepackage{bm}
\usepackage{mdframed}
\usepackage{relsize}
\usepackage{placeins}
\usepackage{empheq}
\usepackage{cancel}
\usepackage{pdfpages}
\usepackage{wrapfig}
\usepackage[flushleft]{threeparttable}
\usepackage{makecell}
\usepackage{fancyhdr}
\usepackage{tikz}
\usepackage{bigints}
\usepackage{tcolorbox}
\tcbuselibrary{breakable}
\usepackage{scalerel}
\usepackage{pgfplots}
\usepackage{pdflscape}
\usepackage{enumitem}
\pgfplotsset{compat=1.16}
\spanishdecimal{.}
\renewcommand{\baselinestretch}{1.5}
\renewcommand{\labelenumii}{\arabic{enumi}.\arabic{enumii}}
\renewcommand{\labelenumiii}{\arabic{enumi}.\arabic{enumii}.\arabic{enumiii}}

\newcommand{\ptilde}[1]{\ensuremath{{#1}^{\prime}}}
\newcommand{\stilde}[1]{\ensuremath{{#1}^{\prime \prime}}}
\newcommand{\ttilde}[1]{\ensuremath{{#1}^{\prime \prime \prime}}}
\newcommand{\ntilde}[2]{\ensuremath{{#1}^{(#2)}}}
\newcommand{\pderivada}[1]{\ensuremath{{#1}^{\prime}}}
\newcommand{\sderivada}[1]{\ensuremath{{#1}^{\prime \prime}}}
\newcommand{\tderivada}[1]{\ensuremath{{#1}^{\prime \prime \prime}}}
\newcommand{\nderivada}[2]{\ensuremath{{#1}^{(#2)}}}


\newtheorem{defi}{{\it Definición}}[section]
\newtheorem{teo}{{\it Teorema}}[section]
\newtheorem{ejemplo}{{\it Ejemplo}}[section]
\newtheorem{propiedad}{{\it Propiedad}}[section]
\newtheorem{lema}{{\it Lema}}[section]
\newtheorem{cor}{Corolario}
\newtheorem{ejer}{Ejercicio}[section]

\newlist{milista}{enumerate}{2}
\setlist[milista,1]{label=\arabic*)}
\setlist[milista,2]{label=\arabic{milistai}.\arabic*)}
\newlength{\depthofsumsign}
\setlength{\depthofsumsign}{\depthof{$\sum$}}
\newcommand{\nsum}[1][1.4]{% only for \displaystyle
    \mathop{%
        \raisebox
            {-#1\depthofsumsign+1\depthofsumsign}
            {\scalebox
                {#1}
                {$\displaystyle\sum$}%
            }
    }
}
\def\scaleint#1{\vcenter{\hbox{\scaleto[3ex]{\displaystyle\int}{#1}}}}
\def\scaleoint#1{\vcenter{\hbox{\scaleto[3ex]{\displaystyle\oint}{#1}}}}
\def\scaleiiint#1{\vcenter{\hbox{\scaleto[3ex]{\displaystyle\iiint}{#1}}}}
\def\bs{\mkern-12mu}

\newcommand{\Cancel}[2][black]{{\color{#1}\cancel{\color{black}#2}}}

\AtBeginDocument{\RenewCommandCopy\qty\SI}
% \usepackage{apacite}

\title{Notación de índices \\ \large{Material de consulta previo}\vspace{-3ex}}
\author{M. en C. Gustavo Contreras Mayén}
\date{ }

\numberwithin{equation}{section}

\linespread{1.25}

\begin{document}

\vspace{-4cm}
\maketitle
\fontsize{14}{14}\selectfont

\section{El símbolo de Levi-Civita}

Para manejar el producto cruz o la combinación de producto punto y cruz de vectores, es conveniente usar el símbolo de Levi-Civita.
\par
Consideramos el producto cruz de dos vectores $\vb{a}$ con $\vb{b}$:
\begin{align}
\vb{c} = \vb{a} \cross \vb{b}
\label{eq:ecuacion_01}
\end{align}
Las componentes cartesianas del vector $\vb{c}$ se pueden escribir como:
\begin{align}
c_{i} = \nsum_{jk} \epsilon_{ijk} \, a_{j} \, b_{k}
\label{eq:ecuacion_02}
\end{align}
o de la forma:
\begin{align}
c_{k} = \nsum_{ij} a_{i} \, b_{j} \, \epsilon_{ijk}
\label{eq:ecuacion_03}
\end{align}
donde el \textbf{símbolo de Levi-Civita} está definido como:
\begin{align}
\epsilon_{xyz} &= \epsilon_{yzx} = \epsilon_{zxy} = 1 \label{eq:ecuacion_04} \\[0.5em]
\epsilon_{xzy} &= \epsilon_{yxz} = \epsilon_{zyx} = - 1 \label{eq:ecuacion_05}
\end{align}
y todos los demás componentes son cero. Por lo tanto, todos los índices deben ser diferentes para que $\epsilon_{ijk}$ sea distinto de cero. Como puede verse en la definición, el intercambio de dos índices cualesquiera cambia el signo de $\epsilon$.
\par
Un rotacional $\curl{a}$ también se puede escribir usando el símbolo de Levi-Civita de la siguiente manera:
\begin{align*}
\left( \curl{a} \right)_{i} = \nsum_{jk} \, \epsilon_{ijk} \, \partial_{j} b_{k}
\end{align*}
donde:
\begin{align*}
\partial_{x} = \pdv{x}, \hspace{0.5cm} \partial_{y} = \pdv{y}, \hspace{0.5cm} \partial_{z} = \pdv{z}
\end{align*}

Verifiquemos que el símbolo de Levi-Civita reproduce la definición de producto cruz:
\begin{align}
c_{x} = \epsilon_{xyz} \, a_{y} b_{z} + \epsilon_{xzy} \, a_{z} b_{y}
\label{eq:ecuacion_06}
\end{align}
Estos son los únicos términos distintos de cero en la suma de los índices $j$ y $k$ en la ec. (\ref{eq:ecuacion_02}). Como $\epsilon_{xyz} = 1$ y $\epsilon_{xzy} = - 1$ encontramos que:
\begin{align}
c_{x} = a_{y} b_{z} - a_{z} b_{y}
\label{eq:ecuacion_07}
\end{align}
De manera similar:
\begin{align}
c_{y} &= \epsilon_{yzx} \, a_{z} b_{x} + \epsilon_{yxz} \, a_{x} b_{z} = a_{z} b_{x} - a_{x} b_{z} \label{eq:ecuacion_08} \\[0.5em]
c_{z} &= \epsilon_{zxy} \, a_{x} b_{y} + \epsilon_{zxy} \, a_{y} b_{x} = a_{x} b_{y} - a_{y} b_{x} \label{eq:ecuacion_09}
\end{align}

Para manejar el álgebra vectorial usando el símbolo de Levi-Civita, la siguiente fórmula es bastante útil:
\begin{align}
\nsum_{k} \epsilon_{ijk} \, \epsilon_{klm} = \nsum_{k} \epsilon_{ijk} \, \epsilon_{lmk} = \delta_{il} \delta_{jm} - \delta_{im} \delta_{jl}
\label{eq:ecuacion_10}
\end{align}
Esta fórmula puede entenderse al darse cuenta de que $\epsilon_{ijk}$ es cero a menos que todos los índices sean diferentes. Por ejemplo, si $i = x$ y $j = y$ sólo las dos posibilidades siguientes dan un resultado distinto de cero:
\begin{align}
\epsilon_{xyz} \, \epsilon_{zxy} = \epsilon_{xyz} \, \epsilon_{xyz} = 1 \hspace{1.5cm} \text{cuando } l = i = x \hspace{0.3cm} \text{y} \hspace{0.3cm} m = j = y
\label{eq:ecuacion_11}
\end{align}
o cuando:
\begin{align}
\epsilon_{xyz} \, \epsilon_{zyx} = \epsilon_{xyz} \, \epsilon_{yxz} = - 1 \hspace{1.5cm} \text{cuando } l = j = y \hspace{0.3cm} \text{y} \hspace{0.3cm} m = i = x
\label{eq:ecuacion_12}
\end{align}

Se obtienen resultados similares para otras elecciones de $i$ y $j$ y por lo tanto obtenemos el resultado general en la ec. (\ref{eq:ecuacion_10}). En algunos libros, el símbolo de suma a veces se omite con la convención de que los índices repetidos se suman:
\begin{align}
c_{i} = \epsilon_{ijk} \, a_{j} b_{k}
\label{eq:ecuacion_13}
\end{align}
entonces:
\begin{align}
\epsilon_{ijk} \, \epsilon_{klm} = \epsilon_{ijk} \, \epsilon_{lmk} = \delta_{il} \delta_{jm} - \delta_{im} \delta_{jl}
\label{eq:ecuacion_14}
\end{align}

\section{Ejemplos.}

\subsection{Ejemplo 1. Triple producto vectorial.}

Resuelve el siguiente triple producto vectorial:
\begin{align}
\vb{r} = \vb{a} \cross \left( \vb{b} \cross \vb{c} \right)
\label{eq:ecuacion_15}
\end{align}
Con la notación de índices se tiene que:
\begin{align}
\begin{aligned}[b]
r_{i} &= \nsum_{jk} \epsilon_{ijk} \, a_{j} \left( \vb{b} \cross \vb{c} \right)_{k} = \\[0.5em]
&= \nsum_{jk} \epsilon_{ijk} \, a_{j} \, \nsum_{lm} \epsilon_{klm} \, b_{l} c_{m} = \\[0.5em]
&= \nsum_{jlm} \left( \nsum_{k} \epsilon_{ijk} \, \epsilon_{lmn} \right) a_{j} b_{l} c_{m}
\end{aligned}
\label{eq:ecuacion_16}
\end{align}
Ahora usamos la ec. (\ref{eq:ecuacion_10}) para obtener:
\begin{align}
\begin{aligned}
r_{i} &= \nsum_{jlm} \left( \delta_{il} \delta_{jm} - \delta_{im} \delta_{jl} \right) a_{j} b_{l} c_{m} = \\[0.5em]
&= \nsum_{j} \left( a_{j} b_{i} c_{j} - a_{j} b_{j} c_{i} \right)
\end{aligned}
\label{eq:ecuacion_17}
\end{align}
Como:
\begin{align*}
\nsum_{j} a_{j} c_{j} = \vb{a} \vdot \vb{c} \hspace{2cm} \nsum_{j} a_{j} b_{j} = \vb{a} \vdot \vb{b}
\end{align*}
encontramos el resultado en términos de otra identidad vectorial:
\begin{align*}
\vb{r} = \vb{b} \left( \vb{a} \vdot \vb{c} \right) - \left( \vb{a} \vdot \vb{b} \right) \vb{c}
\end{align*}

\subsection{Ejemplo 2. Producto punto y producto cruz.}

Resuelve:
\begin{align*}
V = \vb{a} \vdot \left( \vb{b} \cross \vb{c} \right)
\end{align*}
Nuevamente ocupamos la notación de índices:
\begin{align*}
V &= \nsum_{i} a_{i} \left( \vb{b} \times \vb{c} \right)_{i} = \\[0.5em]
&= \nsum_{i} a_{i} \, \nsum_{jk} \epsilon_{ijk} \, b_{j} c_{k} = \\[0.5em]
&= \nsum_{k} \left( \nsum_{ij} \epsilon_{ijk} \, a_{i} b_{j} \right) c_{k} = \\[0.5em]
&= \nsum_{k} \left( \vb{a} \cross \vb{b} \right)_{k} \, c_{k} = \\[0.5em]
&= \left( \vb{a} \cross \vb{b} \right) \vdot \vb{c}
\end{align*}
Vemos que en la tercera línea del desarrollo anterior, podemos escribir:
\begin{align*}
V &= \nsum_{j} \left( \nsum_{ik} \epsilon_{ijk} \, a_{i} c_{k} \right) b_{j} = \\[0.5em]
&= \nsum_{j} \left( - \nsum_{ik} \epsilon_{jik} \, a_{i} c_{k} \right) b_{j} = \\[0.5em]
&= - \nsum_{j} \left( \vb{a} \cross \vb{c} \right)_{j} \, b_{j} = \\[0.5em]
&= - \left( \vb{a} \cross \vb{c} \right) \vdot \vb{b}
\end{align*}
Por lo que se tiene:
\begin{align*}
\vb{a} \vdot \left( \vb{b} \cross \vb{c} \right) = \left( \vb{a} \cross \vb{c}  \right) \vdot \vb{c} = - \left( \vb{a} \cross \vb{c} \right) \vdot \vb{b}
\end{align*}

\subsection{Ejemplo 3. Rotacional de un producto cruz.}

Resuelve con notación de índices:
\begin{align*}
\vb{r} = \curl{\left( \vb{a} \cross \vb{b} \right)}
\end{align*}
Ocupando la expresión para el rotacional con notación de índices:
\begin{align*}
r_{i} &= \nsum_{jk} \epsilon_{ijk} \, \partial_{j} \left( \vb{a} \cross \vb{b} \right)_{k} = \\[0.5em]
&= \nsum_{jk} \epsilon_{ijk} \, \partial_{j} \left( \nsum_{lm} \epsilon_{klm} \, a_{l} b_{m} \right) = \\[0.5em]
&= \nsum_{jlm} \left( \nsum_{k} \epsilon_{ijk} \epsilon_{klm} \right) \, \partial_{j} \left( a_{l} b_{m} \right)
\end{align*}
Usando la ec. (\ref{eq:ecuacion_10}), se tiene que:
\begin{align*}
r_{i} &= \nsum_{jlm} \left( \delta_{ik} \delta_{jm} - \delta_{im} \delta_jl \right) \left( b_{m} \, \partial_{j} a_{l} + a_{l} \, \partial_{j} b_{m} \right) = \\[0.5em]
&= \nsum_{j} \left( b_{j} \, \partial_{j} a_{i} + a_{i} \, \partial_{j} b_{j} - b_{i} \partial_{j} a_{j} - a_{j} \, \partial_{j} b_{i} \right)
\end{align*}
Dado que:
\begin{align*}
\nsum_{j} b_{j} \, \partial_{j} a_{i} &= \left( \vb{b} \vdot \nabla \right) \, a_{i} \hspace{1.5cm} \nsum_{j} a_{i} \, \partial_{j} b_{j} =  a_{i} \, \left( \nabla \vdot \vb{b}\right) \\[0.5em]
\nsum_{j} b_{i} \, \partial_{j} a_{j} &=  b_{i} \, \left( \nabla \vdot \vb{a}\right) \hspace{1.5cm} \nsum_{j} a_{j} \, \partial_{j} b_{i} = \left( \vb{a} \vdot \nabla \right) \, b_{i}
\end{align*}
se obtiene:
\begin{align*}
\curl{\left( \vb{a} \cross \vb{b} \right)} = \left( \vb{b} \vdot \nabla \right) \, \vb{a} + \vb{a} \left( \div{\vb{b}} \right) - \vb{b} \left( \div{\vb{a}} \right) - \left( \vb{a} \vdot \nabla \right) \, \vb{b}
\end{align*}

\subsection{Ejemplo 4. Divergencia de un rotacional.}

Resuelve con notación de índices:
\begin{align*}
V = \divergence{\vb{a} \cross \vb{b}}
\end{align*}

Usando las definiciones presentadas anteriormente:
\begin{eqnarray*}
\begin{aligned}
V &= \nsum_{i} \partial_{i} \left( \vb{a} \cross \vb{b} \right)_{i} = \\[0.5em]
&= \nsum_{i} \partial_{i} \, \nsum_{jk} \epsilon_{ijk} a_{j} b_{k} = \\[0.5em]
&= \nsum_{ijk} \partial_{i} \left( a_{j} b_{k} \right) = \\[0.5em]
&= \nsum_{ijk} \epsilon_{ijk} \left( b_{k} \, \partial_{i} \, a_{j} + a_{j} \, \partial_{i} \, b_{k} \right)
\end{aligned}
\end{eqnarray*}
El primer término es:
\begin{align*}
\nsum_{ijk} \epsilon_{ijk} \, b_{k} \, \partial_{i} \, a_{j} &= \nsum_{k} b_{k} \left( \nsum_{ij} \epsilon_{kij} \, \partial_{i} \, a_{j} \right) = \\[0.5em]
&= \nsum_{k} b_{k} \left( \curl{\vb{a}} \right)_{k} = \\[0.5em]
&= \vb{b} \vdot \left( \curl{\vb{a}} \right)
\end{align*}
el segundo término es:
\begin{align*}
\nsum_{ijk} \epsilon_{ijk} \, a_{j} \, \partial_{i} \, b_{k} &= \nsum_{j} a_{j} \left( - \nsum_{ik} \epsilon_{jik} \, \partial_{i} \, b_{k} \right) = \\[0.5em]
&= \nsum_{j} a_{j} \left( \curl{\vb{b}} \right)_{j} = \\[0.5em]
&= - \vb{a} \vdot \left( \curl{\vb{b}} \right)
\end{align*}
Por lo tanto:
\begin{align*}
V = \divergence{\vb{a} \cross \vb{b}} = \vb{b} \vdot \left( \curl{\vb{a}} \right) - \vb{a} \vdot \left( \curl{\vb{b}} \right)
\end{align*}

\subsection{Identidades.}

Demuestra las siguientes identidades:
\begin{align*}
\vb{r} = \nabla \cross \left( \grad{\phi} \right) = 0 \hspace{1.2cm} \text{y} \hspace{1.2cm}
\div{\curl{\vb{a}}} = 0
\end{align*}
La primera identidad se demuestra de la siguiente manera:
\begin{align*}
r_{i} &= \nsum_{jk} \epsilon_{ijk} \, \partial_{j} \left( \grad{\phi} \right)_{k} = \\[0.5em]
&= \nsum_{jk} \epsilon_{ijk} \, \partial_{j} \, \partial_{k} \, \phi
\end{align*}
Ya que:
\begin{align*}
\partial_{j} \, \partial_{k} \, \phi = \partial_{k} \, \partial_{j} \, \phi
\end{align*}
tenemos para la componente $x$:
\begin{align*}
r_{x} = \left( \epsilon_{xyz}  + \epsilon_{xzy} \right) \partial_{y} \, \partial_{z} \, \phi = 0
\end{align*}
ya que $\epsilon_{xzy} = - \epsilon_{xyz}$. De manera similar para las otras componentes de $\vb{r}$.
\par
Para la segunda identidad:
\begin{align*}
\div{\curl{\vb{a}}} &= \nsum_{i} \partial_{i} \left( \curl{\vb{a}} \right)_{i} = \\[0.5em]
&= \nsum_{i} \partial_{i} \nsum_{jk} \epsilon_{ijk} \, \partial_{j} \, a_{k} = \\[0.5em]
&= \nsum_{ijk} \epsilon_{ijk} \, \partial_{i} \, \partial_{j} \, a_{k}
\end{align*}
Para $k = x$ y usando $\partial_{i} \, \partial_{j} \, a_{k} = \partial_{j} \, \partial_{i} a_{k} $, se tiene en el lado derecho:
\begin{align*}
\left( \epsilon_{yzx} + \epsilon_{zyx} \right) \, \partial_{y} \, \partial_{z} \, a_{x} = 0
\end{align*}
Por lo tanto:
\begin{align*}
\div{\curl{\vb{a}}} = 0
\end{align*}
\end{document}