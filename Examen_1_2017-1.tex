\documentclass[12pt]{article}
\usepackage[utf8]{inputenc}
\usepackage[spanish,es-lcroman, es-tabla]{babel}
\usepackage[autostyle,spanish=mexican]{csquotes}
\usepackage{amsmath}
\usepackage{amssymb}
\usepackage{nccmath}
\numberwithin{equation}{section}
\usepackage{amsthm}
\usepackage{graphicx}
\usepackage{epstopdf}
\DeclareGraphicsExtensions{.pdf,.png,.jpg,.eps}
\usepackage{color}
\usepackage{float}
\usepackage{multicol}
\usepackage{enumerate}
\usepackage[shortlabels]{enumitem}
\usepackage{anyfontsize}
\usepackage{anysize}
\usepackage{array}
\usepackage{multirow}
\usepackage{enumitem}
\usepackage{cancel}
\usepackage{tikz}
\usepackage{circuitikz}
\usepackage{tikz-3dplot}
\usetikzlibrary{babel}
\usetikzlibrary{shapes}
\usepackage{bm}
\usepackage{mathtools}
\usepackage{esvect}
\usepackage{hyperref}
\usepackage{relsize}
\usepackage{siunitx}
\usepackage{physics}
%\usepackage{biblatex}
\usepackage{standalone}
\usepackage{mathrsfs}
\usepackage{bigints}
\usepackage{bookmark}
\spanishdecimal{.}

\setlist[enumerate]{itemsep=0mm}

\renewcommand{\baselinestretch}{1.5}

\let\oldbibliography\thebibliography

\renewcommand{\thebibliography}[1]{\oldbibliography{#1}

\setlength{\itemsep}{0pt}}
%\marginsize{1.5cm}{1.5cm}{2cm}{2cm}


\newtheorem{defi}{{\it Definición}}[section]
\newtheorem{teo}{{\it Teorema}}[section]
\newtheorem{ejemplo}{{\it Ejemplo}}[section]
\newtheorem{propiedad}{{\it Propiedad}}[section]
\newtheorem{lema}{{\it Lema}}[section]

\usepackage{geometry}
\geometry{top=1.25cm, bottom=1.5cm, left=1.25cm, right=1.25cm}
\usepackage{enumerate}
%\usepackage[shortlabels]{enumitem}
\usepackage{pifont}
\renewcommand{\labelitemi}{\ding{43}}
%\author{M. en C. Gustavo Contreras Mayén. \texttt{curso.fisica.comp@gmail.com}}
\title{{Examen Parcial 1} \\ {\large Matemáticas Avanzadas de la Física} \vspace{-1.5\baselineskip}}
\date{ }
\let\olditemize\itemize
\def\itemize{\olditemize\itemsep=0pt}
\begin{document}
%\vspace{-3cm}
%\renewcommand\theenumii{\arabic{theenumii.enumii}}
\renewcommand\labelenumii{\theenumi.{\arabic{enumii}}}
\maketitle
\fontsize{14}{14}\selectfont
\textbf{Indicaciones:}
\begin{itemize}
\item Responde lo más claro posible cada una de las preguntas.
\item Este primer examen cubre los dos primeros temas del curso.
\end{itemize}
\begin{enumerate}
\item En las diferentes ecuaciones diferenciales que se presentaron al inicio del curso, se observa el hecho de que sean ecuaciones diferenciales lineales, ¿qué representa desde el punto de vista físico que sean ecuaciones lineales? 
\item Dada una variedad diferenciable tal que se tiene un sistema de cartas como función de
coordenadas generalizadas,
\[x_{i}(q_{1}, q_{2}, q_{3}) \]
 explica cómo se obtiene la nueva base de vectores, no necesariamente unitarios, para la descripción de esta variedad; a partir de su respuesta anterior indique cómo obtener los factores de escala , como se escriben los elementos de línea, área y volumen.
\item Señala qué condiciones debe cumplir una ecuación diferencial parcial de forma que pueda usarse el método de separación de variables, explica en qué consiste dicha técnica.
\item Para una ecuación diferencial ordinaria de segundo orden, menciona cuáles son las singularidades de
ésta, bajo qué condiciones pueden ser removidas, trata tanto el caso de un punto finito en el dominio de la ecuación así como un punto en infinito.
\item Describe cómo remover singularidades, tanto en un punto singular ordinario como en infinito,
detalla el método de Frobenius.
\end{enumerate}
\end{document}