\documentclass[12pt]{article}
\usepackage{termcal}
\usepackage[utf8]{inputenc}
\usepackage[spanish]{babel}
\usepackage{xcolor}
\renewcommand{\calprintdate}{%
     \ifnewmonth\framebox{\arabic{month}/\arabic{date}}%
     \else\arabic{date}%
     \fi}
\definecolor{awesome}{rgb}{1.0, 0.13, 0.32}
\definecolor{blush}{rgb}{0.87, 0.36, 0.51}
\definecolor{capri}{rgb}{0.0, 0.75, 1.0}
\definecolor{darkcyan}{rgb}{0.0, 0.55, 0.55}
\definecolor{darktangerine}{rgb}{1.0, 0.66, 0.07}

\author{M. en C. Gustavo Contreras Mayén. \texttt{curso.fisica.comp@gmail.com}\\
Fís. Abraham Lima Buendía. \texttt{abraham3081@ciencias.unam.mx}}
\title{Clases para el curso de MAF \\ \large{Semestre 2017-1}}
\begin{document}
\date{}
\maketitle
\begin{calendar}{8/8/16}{18}
\setlength{\calboxdepth}{.3in}
% Description of the Week.
\skipday %[Monday]{\classday} % Monday
%\calday[Lunes]{\classday}
\calday[Martes]{\classday}
\skipday  % Wednesday
%\calday[Miércoles]{\classday}
\calday[Jueves]{\classday} % Thursday (unnumbered)
\skipday
%\calday[Viernes]{\classday}
%\calday[Friday]{\classday} % Friday
\skipday\skipday % weekend (no class)

% Holidays
\options{9/15/16}{\noclassday}
\caltext{9/15/16}{\color{red}{Festivo\\Día de la Independencia}}
\options{11/1/16}{\noclassday}
\caltext{11/1/16}{\color{red}{Festivo\\Día de Muertos}}

%Temas del curso
\caltext{8/9/16}{\begin{enumerate}
\item Presentación del curso de MAF.
\item Resumen de cómo se trabajará y a dónde queremos llevarlos al final del curso.
\end{enumerate}}

\caltext{8/11/16}{\color{blue}{\textbf{Tema 1. La física y la geometría.}}}
\caltext{8/18/16}{\color{darkcyan}{\textbf{Ejercicios Tema 1.}}}

\caltext{8/23/16}{\color{blue}{\textbf{Tema 2. Técnicas de solución.}}}
\caltext{9/1/16}{\color{darkcyan}{\textbf{Ejercicios Tema 2.}}}

\caltext{9/8/16}{\color{blue}{\textbf{Tema 3. Completez y ortogonalidad.}}}
\caltext{9/22/16}{\color{blue}{\textbf{Tema 4. Funciones Gamma y Beta.}}}
\caltext{9/29/16}{\color{darkcyan}{\textbf{Ejercicios Temas 3 y 4.}}}

\caltext{10/4/16}{\color{blue}{\textbf{Tema 5. Separación de varibales en coordenadas esféricas.}}}
\caltext{10/13/16}{\color{blue}{\textbf{Tema 6. Funciones especiales.}}}
\caltext{11/3/16}{\color{darkcyan}{\textbf{Ejercicios Temas 5 y 6.}}}

\caltext{11/8/16}{\color{blue}{\textbf{Tema 7. Transformadas integrales.}}}
\caltext{11/24/16}{\color{darkcyan}{\textbf{Ejercicios Tema 7.}}}


% Exams
\caltext{9/6/16}{\color{awesome}{\textbf{Examen Clase I}}}
\caltext{10/6/16}{\color{awesome}{\textbf{Entrega Tarea - Examen  II}}}
\caltext{11/10/16}{\color{awesome}{\textbf{Entrega Tarea - Examen III}}}
%ya no se considera como día de clase, solo para entrega de la tarea examen
\options{11/29/16}{\noclassday}
\caltext{11/29/16}{\color{awesome}{\textbf{Entrega Tarea - Examen IV}}}


\options{12/1/16}{\noclassday}
\caltext{12/1/16}{Semana 1 de Finales}
\options{12/6/16}{\noclassday}
\caltext{12/6/16}{Semana 2 de Finales}
\options{12/8/16}{\noclassday}
\caltext{12/8/16}{Semana 2 de Finales}

% Text on consecutive days
% the subject of the 9th lecture
%\caltexton{9}{\S1.2,1.3\\Sorting and disporting}
% the subject of the 10th lecture
%\caltextnext{\S1.3,1,4\\Assembling and dissembling}
%\caltextnext{}
% the subject of the 12th lecture
%\caltextnext{\S9.9\\Tending and rending}
\end{calendar}
\end{document}