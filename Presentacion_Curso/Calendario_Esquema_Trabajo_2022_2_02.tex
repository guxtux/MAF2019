\documentclass[12pt]{article}
\usepackage[utf8]{inputenc}
\usepackage[spanish]{babel}
\usepackage[letterpaper, landscape, margin=1in]{geometry}
\usepackage{xcolor}
\usepackage{termcal}
\usepackage{tikz}
\renewcommand{\calprintclass}{}
\renewcommand{\calprintdate}{%
     \ifnewmonth\framebox{\arabic{month}/\arabic{date}}%
     \else\arabic{date}%
     \fi}

\definecolor{awesome}{rgb}{1.0, 0.13, 0.32}
\definecolor{blush}{rgb}{0.87, 0.36, 0.51}
\definecolor{capri}{rgb}{0.0, 0.75, 1.0}
\definecolor{darkcyan}{rgb}{0.0, 0.55, 0.55}
\definecolor{darktangerine}{rgb}{1.0, 0.66, 0.07}
\definecolor{cadmiumgreen}{rgb}{0.0, 0.42, 0.24}
\definecolor{darkslateblue}{rgb}{0.28, 0.24, 0.55}

\newcommand{\marcoconferencia}{\fcolorbox{darkcyan}{darkcyan}{\textbf{\textcolor{white}{Videoconferencia}}}}
\newcommand{\marcotarea}[1]{\fcolorbox{red}{white}{\textbf{\color{black}{Tarea Semana #1}}}}
\newcommand{\marcotema}[1]{\color{blue}{\textbf{#1}}}
\newcommand{\marcoclase}{\fcolorbox{blush}{blush}{\textbf{\textcolor{white}{Clase presencial}}}}

\title{}
\author{}
\begin{document}
\date{}
\maketitle
\vspace*{-4cm}
\begin{calendar}{3/7/22}{2}
\setlength{\calwidth}{1.05\textwidth} 
\setlength{\calboxdepth}{1.5in}
% Description of the Week.

\calday[Lunes]{\classday}
\calday[Martes]{\classday}
\calday[Miércoles]{\classday}
\calday[Jueves]{\classday}
\calday[Viernes]{\classday}
\skipday\skipday % weekend (no class)


%Temas del curso

\caltext{3/7/22}{\textbf{Semana 4}}
\caltext{3/8/22}{\marcoconferencia}
\caltext{3/10/22}{\marcoconferencia}

\caltext{3/14/22}{\textbf{Semana 5}}
\caltext{3/15/22}{\marcoconferencia \\[5em] \marcoclase}
\caltext{3/17/22}{\marcoconferencia \\[5em] \marcoclase}
\end{calendar}


\begin{tikzpicture}[overlay]
     
\draw [fill, color=capri, text=black, text width = 21cm] (-0.5, 5.5) rectangle (23.1, 6.8) node[pos=0.5] {El alumno consulta los materiales de trabajo, los complementarios en la plataforma, así como de los videos, con los que podrá resolver los ejercicios del tema.};

\draw (7, 7.6) node [text width=4cm, centered] {\textbf{\textcolor{red}{Entrega \\ Ejercicios Tema 1}}};

\draw [fill, color=capri, text=black, text width = 21cm] (-0.5, 2.6) rectangle (23.1, 3.8) node[pos=0.5] {El alumno consulta los materiales de trabajo, los complementarios en la plataforma, así como de los videos, con los que podrá resolver los ejercicios del tema.};
    
\end{tikzpicture}
\end{document}