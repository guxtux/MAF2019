\documentclass[12pt]{article}
\usepackage[left=0.3cm,top=2cm,right=0.3cm,bottom=2cm]{geometry}
\usepackage[utf8]{inputenc}
\usepackage[spanish,es-tabla]{babel}
\usepackage{amsmath}
\usepackage{amsthm}
\usepackage{graphicx}
\usepackage{color}
\usepackage{float}
\usepackage{multicol}
\usepackage{enumerate}
\usepackage{anyfontsize}
\usepackage{anysize}
\usepackage{enumitem}
\usepackage{capt-of}
\spanishdecimal{.}
\setlist[enumerate]{itemsep=0mm}
\renewcommand{\baselinestretch}{1.2}
\let\oldbibliography\thebibliography
\renewcommand{\thebibliography}[1]{\oldbibliography{#1}
\setlength{\itemsep}{0pt}}
%\marginsize{1.5cm}{1.5cm}{0cm}{2cm}
\author{M. en C. Gustavo Contreras Mayén. \texttt{curso.fisica.comp@gmail.com}\\
Fís. Abraham Lima Buendía. \texttt{abraham3081@ciencias.unam.mx}}
\title{Matemáticas Avanzadas de la Física \\ {\large Semestre 2017-1 Grupo 8183}}
\date{ }
\begin{document}
\vspace{-4cm}
%\renewcommand\theenumii{\arabic{theenumii.enumii}}
\renewcommand\labelenumii{\theenumi.{\arabic{enumii}}}
\maketitle
\fontsize{14}{14}\selectfont
\section{Objetivos.}
Dentro el curso el alumno:
\begin{itemize}
\setlength{\itemsep}{0mm}
\item Reconocerá las ideas básicas del análisis de ecuaciones que involucran a funciones de varias variables.
\item Formulará aproximaciones consistentes a soluciones, con el fin de cuantificar los distintos mecanismos de la física que se involucran.
\item Consultará la literatura matemática que sea relevante para los problemas de física.
\item Identificará el papel moderno que juegan las funciones especiales, como auxiliares poderosos en el análisis cualitativo de problemas en varias variables.
\end{itemize}
También es nuestro objetivo demostrar al alumno que las funciones especiales y las transformadas integrales no son solamente un tema matemático, que involucra las ramas de la geometría diferencial, las ecuaciones diferenciales y el análisis matemático, sino que son las técnicas de estudio fundamentales en la electrostática, la electrodinámica, la mecánica cuántica en los límites relativista y  no relativista, la dinámica de medios deformables, la hidrodinámica clásica entre otras ramas de la física.
\\
\\
\textbf{Punto importante: } Considerando que MAF es una asignatura de sexto semestre, se espera que hayan cursado y acreditado: Cálculo I a Cálculo IV, Ecuaciones Diferenciales Ordinarias I, Algebra lineal I, Variable Compleja I, la llamada Física Clásica (Física contemporánea, Mecánica vectorial, Fenómenos colectivos, Electromagnetismo, Óptica e Introducción a la Mecánica cuántica)
\\
\\
\textbf{Lugar: }Aula P109.
\\
\textbf{Horario: } Martes de 18:00 a 20:00 y Jueves de 18 a 21:00 horas.
\section{Metodología de enseñanza.}
En las clases habrá exposición con dialógo por parte de los profesores, junto con el desarrollo de ejercicios que orientarán la solución de problemas en un primer momento de tipo analítico, para luego revisar un ejercicio tomado de la física.
\\
\\
Se dejarán lecturas complementarias, consulta de referencias bibliográficas, con la finalidad de que consulten y repasen ciertos temas para consolidar la base de conocimiento necesaria para los contenidos del curso, en algunos casos será necesario apoyarse con la solución numérica de ejercicios usando algún lenguaje de programación (\texttt{python}) o con paquetería (\texttt{Wolfram}).
\\
\\
El curso demandará una atención por parte de ustedes con el mismo número de horas fuera de clase (consideramos que sería el mínimo de tiempo), es decir, que deberán de dedicarle fuera del salón de clase, al menos 5 horas a la semana para el desarrollo de lecturas, ejercicios, tareas y actividades.
\section{Temario}
\begin{enumerate}
\item La física y la geometría.
\begin{enumerate}
\item Coordenadas generalizadas.
\item Métrica y teoremas integrales (Gauss y Stokes).
\item Ecuaciones diferenciales de la física.
\end{enumerate}
\item Primeras técnicas de solución.
\begin{enumerate}
\item Separación de variables.
\item Remoción de singularidades y método de Frobenius.
\item Segunda solución independiente.
\item Teorema de Green.
\end{enumerate}
\item Completez y ortogonalidad de una base.
\begin{enumerate}
\item Ecuaciones de tipo Sturm-Liouville.
\item Ortogonalización de Gram-Schimdt.
\item Completez y función delta de Dirac.
\item Desarrollo de una función en una base propia.
\end{enumerate}
\item Función Gamma($\Gamma$) y función Beta ($\beta$)
\item Separación de variables en coordenadas esféricas.
\begin{enumerate}
\item Ecuación de Legendre.
\item Ecuación asociada de Legendre.
\item Armónicos esféricos, teorema de adición y momento angular.
\end{enumerate}
\item Funciones especiales.
\begin{enumerate}
\item Funciones de Bessel (propagación de ondas cilíndricas y esféricas)
\item Funciones de Hermite (oscilador cuántico)
\item Funciones de Laguerre (átomo de hidrógeno)
\item Funciones de Chebychev (tipos I y II)
\item Funciones hipergeométricas
\end{enumerate}
\item Transformadas integrales.
\begin{enumerate}
\item Transformada de Fourier.
\item Transformada de Laplace.
\end{enumerate}
\end{enumerate}
\section{Evaluación.}
La evaluación del curso contempla la entrega de tareas y de la solución de exámenes.
\subsection{Tareas.}
El total de tareas en el curso es de 4, por lo que se contemplan tareas con problemas que abarquen dos temas. \\
\textbf{Importante: } Durante las clases se mencionarán los problemas que se dejarán de tarea, para que puedan trabajarlos oportunamente, hacer consultas, resolver dudas, etc. de tal manera que una semana antes de señalar la fecha de entrega, tendrán prácticamente la lista de problemas que conformarán la tarea. A manera de referencia, se les entregará la lista de problemas (que ustedes ya deberían de tener anotados al asistir a clases). La fecha de entrega no se modifica, las tareas se entregarán a mano, no aceptaremos tareas por correo, ni por usb.
\\
Es responsabilidad de cada alumno, saber en qué momento inicia la solución de los problemas de la tarea, como se mencionó anteriormente, en las tareas se trabajarán problemas de dos temas, por lo que el número de problemas por tarea, ronda los 20 problemas.
\\
Habrá ejercicios que se resolverán durante las clases, considerando que por el tiempo que tendremos, en algunos casos, los ejercicios no podrán resolverse de manera completa, por lo que se les dejarán de tarea para que completen lo faltante, con esto queremos mencionar que es importante que asistan a clase para enterarse de los ejercicios a cuenta de la tarea.
\subsection{Exámenes.}
Habrá 4 exámenes parciales de tipo conceptual, en correspondencia con las tareas, distribuidos de la siguiente manera:
\begin{itemize}
\item \emph{Primer parcial:} Considera los temas 1, 2.
\item \emph{Segundo parcial:} Los temas 3 y 4
\item \emph{Tercer parcial:} Los temas 5 y 6.
\item \emph{Cuarto parcial:} El tema 7.
\end{itemize}
El peso para cada elemento de evaluación es el siguiente:
\begin{itemize}
\setlength{\itemsep}{0mm}
\item En total 4 tareas: \hspace{2cm} 60 \%.
\item En total 4 exámenes parciales: 40\%.
\end{itemize}
\textbf{Consideraciones importantes:}
\begin{itemize}
\setlength{\itemsep}{0mm}
\item La entrega de tareas se hará en el salón de clase, teniendo como límite de entrega: las 6:20 pm; no se recibirán tareas por medios electrónicos.
\item \underline{En caso de no acreditar un sólo examen parcial} de los cuatro programados (es decir, la calificación es $<6.0$), se podrá presentar una sola reposición.
\item Si la calificación de dos exámenes parciales es no aprobatoria, y como sólo se puede reponer un examen parcial, el alumno es candidato para presentar el examen final, considera lo siguiente:
\item Para presentar examen final del curso, deben de cumplirse de manera simultánea los siguientes criterios:
\begin{itemize}
\item Haber presentado los cuatro exámenes del curso: ya sea que dos o hasta los cuatro exámenes tengan calificación de no acreditado.
\item Haber entregado las cuatro tareas.
\end{itemize}  
\item En caso de no haber entregado alguna tarea y no haber presentado un examen del curso, no se tiene derecho a  presentar el examen final, por lo que la calificación final será NP (No presentó)
\item En caso de haber presentado al menos un examen y/o haber entregado una tarea, se entiende que abandonaron el curso, no se tiene derecho para presentar el examen final, y la calificación que se asentará en el acta del curso, será 5 (cinco).
\item En caso de presentar la primera ronda del examen final y la calificación sea no aprobatoria ($<6.0$), se puede presentar una segunda y última ronda del examen final, la calificación obtenida en el examen, es la que se asentará en actas.
\end{itemize}
\section{Bibliografía.}
A continuación se indican algunos textos que serán útiles para consulta dentro del curso, de manera adicional, se proporcionarán algunos artículos científicos para revisión de ejemplos aplicados. Se habilitará un compendio de materiales de trabajo mediante una carpeta en Dropbox, tanto de materiales básicos como complementarios, tales como artículos, capítulos de libros, notas, etc. Para contar con la liga de acceso, deberán de enviar un mensaje de correo a Abraham para que les proporcione la liga de Dropbox.
\nocite{*}
\renewcommand{\refname}{Bibliografía básica.}
\bibliography{LibrosCurso2015-2}\bibliographystyle{unsrt}
\section{Fechas importantes.}
\begin{itemize}
\item Lunes 8 de agosto: Inicio del semestre 2017-1.
\item Martes 9 de agosto: Primera clase del curso.
\item Jueves 15 de septiembre: Festivo.
\item Martes 1 de noviembre: Festivo.
\item Viernes 25 de noviembre: Termina el semestre 2017-1.
\item Lunes 28 de noviembre al viernes 1 de diciembre: Primera semana de finales.
\item Lunes 6 al viernes 9 de diciembre: Segunda semana de finales.
\end{itemize}
\end{document}