\documentclass[14pt]{extarticle}
\usepackage[utf8]{inputenc}
\usepackage[T1]{fontenc}
\usepackage[spanish,es-lcroman]{babel}
\usepackage{amsmath}
\usepackage{amsthm}
\usepackage{physics}
\AtBeginDocument{\RenewCommandCopy\qty\SI}
\usepackage{tikz}
\usepackage{float}
\usepackage{calc}
\usepackage[autostyle,spanish=mexican]{csquotes}
\usepackage[per-mode=symbol]{siunitx}
\usepackage{textcomp, gensymb}
\usepackage{multicol}
\usepackage{enumitem}
\usepackage{hyperref}
\usepackage{setspace}
\usepackage[left=2.00cm, right=2.00cm, top=2.00cm, 
     bottom=2.00cm]{geometry}
% \usepackage{Estilos/ColoresLatex}
\usepackage{makecell}
\usepackage{subcaption}
\usepackage[skip=10pt, indent=30pt]{parskip}
% \usepackage{scalerel}
\usepackage{scalerel}[2016-12-29]

\definecolor{ao}{rgb}{0.0, 0.0, 1.0}
\definecolor{burgundy}{rgb}{0.5, 0.0, 0.13}

\hypersetup{
    colorlinks=true,
    linkcolor=ao,
    filecolor=magenta,      
    urlcolor=ao,
}


\def\stretchint#1{\vcenter{\hbox{\stretchto[440]{\displaystyle\int}{#1}}}}
\def\scaleint#1{\vcenter{\hbox{\scaleto[3ex]{\displaystyle\int}{#1}}}}
\def\scaleiint#1{\vcenter{\hbox{\scaleto[6ex]{\displaystyle\iint}{#1}}}}
\def\scaleiiint#1{\vcenter{\hbox{\scaleto[6ex]{\displaystyle\iiint}{#1}}}}
\def\scaleoint#1{\vcenter{\hbox{\scaleto[3ex]{\displaystyle\oint}{#1}}}}
\def\bs{\mkern-12mu}

\newcommand{\textocolor}[2]{\textbf{\textcolor{#1}{#2}}}
\sisetup{per-mode=symbol}
\decimalpoint
\sisetup{bracket-numbers = false}
\newlength{\depthofsumsign}
\setlength{\depthofsumsign}{\depthof{$\sum$}}
\newcommand{\nsum}[1][1.4]{% only for \displaystyle
    \mathop{%
        \raisebox
            {-#1\depthofsumsign+1\depthofsumsign}
            {\scalebox
                {#1}
                {$\displaystyle\sum$}%
            }
    }
}

\ExplSyntaxOn
\msg_redirect_name:nnn { siunitx } { physics-pkg } { none }
\ExplSyntaxOff

\numberwithin{equation}{section}

\linespread{1.15}

% \documentclass[hidelinks,12pt]{book}
%\usepackage[left=0.25cm,top=1cm,right=0.25cm,bottom=1cm]{geometry}
\usepackage{geometry}
% \usepackage[landscape]{geometry}
% \textwidth = 20cm
% \hoffset = -1cm
\usepackage[utf8]{inputenc}
\usepackage[spanish,es-tabla, es-lcroman]{babel}
\usepackage[autostyle,spanish=mexican]{csquotes}
\usepackage[tbtags]{amsmath}
\usepackage{nccmath}
\usepackage{amsthm}
\usepackage{amssymb}
\usepackage{mathrsfs}
\usepackage{graphicx}
\usepackage{subfig}
\usepackage{caption}
%\usepackage{subcaption}
\usepackage{standalone}
\graphicspath{{Imagenes/}{../Imagenes/}}
\usepackage[outdir=./Imagenes/]{epstopdf}
\usepackage{siunitx}
\usepackage{physics}
\usepackage{color}
\usepackage{float}
\usepackage{hyperref}
\usepackage{multicol}
\usepackage{multirow}
%\usepackage{milista}
\usepackage{anyfontsize}
\usepackage{anysize}
%\usepackage{enumerate}
\usepackage[shortlabels]{enumitem}
\usepackage{capt-of}
\usepackage{bm}
\usepackage{mdframed}
\usepackage{relsize}
\usepackage{placeins}
\usepackage{empheq}
\usepackage{cancel}
\usepackage{pdfpages}
\usepackage{wrapfig}
\usepackage[flushleft]{threeparttable}
\usepackage{makecell}
\usepackage{fancyhdr}
\usepackage{tikz}
\usepackage{bigints}
\usepackage{tcolorbox}
\tcbuselibrary{breakable}
\usepackage{scalerel}
\usepackage{pgfplots}
\usepackage{pdflscape}
\usepackage{enumitem}
\pgfplotsset{compat=1.16}
\spanishdecimal{.}
\renewcommand{\baselinestretch}{1.5}
\renewcommand{\labelenumii}{\arabic{enumi}.\arabic{enumii}}
\renewcommand{\labelenumiii}{\arabic{enumi}.\arabic{enumii}.\arabic{enumiii}}

\newcommand{\ptilde}[1]{\ensuremath{{#1}^{\prime}}}
\newcommand{\stilde}[1]{\ensuremath{{#1}^{\prime \prime}}}
\newcommand{\ttilde}[1]{\ensuremath{{#1}^{\prime \prime \prime}}}
\newcommand{\ntilde}[2]{\ensuremath{{#1}^{(#2)}}}
\newcommand{\pderivada}[1]{\ensuremath{{#1}^{\prime}}}
\newcommand{\sderivada}[1]{\ensuremath{{#1}^{\prime \prime}}}
\newcommand{\tderivada}[1]{\ensuremath{{#1}^{\prime \prime \prime}}}
\newcommand{\nderivada}[2]{\ensuremath{{#1}^{(#2)}}}


\newtheorem{defi}{{\it Definición}}[section]
\newtheorem{teo}{{\it Teorema}}[section]
\newtheorem{ejemplo}{{\it Ejemplo}}[section]
\newtheorem{propiedad}{{\it Propiedad}}[section]
\newtheorem{lema}{{\it Lema}}[section]
\newtheorem{cor}{Corolario}
\newtheorem{ejer}{Ejercicio}[section]

\newlist{milista}{enumerate}{2}
\setlist[milista,1]{label=\arabic*)}
\setlist[milista,2]{label=\arabic{milistai}.\arabic*)}
\newlength{\depthofsumsign}
\setlength{\depthofsumsign}{\depthof{$\sum$}}
\newcommand{\nsum}[1][1.4]{% only for \displaystyle
    \mathop{%
        \raisebox
            {-#1\depthofsumsign+1\depthofsumsign}
            {\scalebox
                {#1}
                {$\displaystyle\sum$}%
            }
    }
}
\def\scaleint#1{\vcenter{\hbox{\scaleto[3ex]{\displaystyle\int}{#1}}}}
\def\scaleoint#1{\vcenter{\hbox{\scaleto[3ex]{\displaystyle\oint}{#1}}}}
\def\scaleiiint#1{\vcenter{\hbox{\scaleto[3ex]{\displaystyle\iiint}{#1}}}}
\def\bs{\mkern-12mu}

\newcommand{\Cancel}[2][black]{{\color{#1}\cancel{\color{black}#2}}}

\AtBeginDocument{\RenewCommandCopy\qty\SI}
% \usepackage{apacite}

\title{Presentación Curso MAF \\ \large{Semestre 2025-1}\vspace{-3ex}}
\author{}
\date{ }

\begin{document}

\vspace{-4cm}
\maketitle

\section{Equipo académico.}

M. en C. Gustavo Contreras Mayén. \hspace{0.3cm} \href{mailto:gux7avo@ciencias.unam.mx}{gux7avo@ciencias.unam.mx}

M. en C. Abraham Lima Buendía. \hspace{0.3cm} \href{mailto:abraham3081@ciencias.unam.mx}{abraham3081@ciencias.unam.mx}

\section{Objetivos del curso.}

En la página de la Facultad, el programa de la asignatura: Matemáticas Avanzadas de la Física se puede consultar 
\href{https://www.fciencias.unam.mx/sites/default/files/temario/610.pdf}{aquí}, y contiene los siguientes objetivos:
\begin{enumerate}
\item Reconocerá las ideas básicas del análisis de ecuaciones que involucran a funciones de varias variables.
\item Formulará aproximaciones consistentes a soluciones, con el fin de cuantificar los distintos mecanismos de la física que se involucran.
\item Consultará la literatura matemática que sea relevante para los problemas de física.
\item Identificará el papel moderno que juegan las funciones especiales, como auxiliares poderosos en el análisis cualitativo de problemas en varias variables.
\end{enumerate}

También es nuestro objetivo demostrar al alumno que \textbf{las funciones especiales y las transformadas integrales} no son solamente un tema matemático que involucra las ramas de la geometría diferencial, las ecuaciones diferenciales y el análisis matemático.

Veremos que \textbf{son las técnicas de estudio fundamentales} en la electrostática, la electrodinámica, la mecánica cuántica, la dinámica de medios deformables, la hidrodinámica clásica entre otras ramas de la física.

El curso de MAF les brindará un manejo más fluido y consistente para lo que van a cursar en el sexto semestre y los tres semestres que les restan para concluir la carrera.

\section{Evaluación.}

El esquema de evaluación para el curso se conforma de dos elementos:
\begin{enumerate}[label=\alph*)]
\item \textbf{Exámenes.}

Tendrán un peso del \num{60} \% de la calificación. Se aplicarán tres exámenes parciales durante el curso.
\item \textbf{Tareas.}

Le corresponderán el \num{40} \% de la calificación. Por cada tema del curso se dejará una tarea para entregar.
\end{enumerate}
En la primera sesión del curso se extenderá cada elemento de evaluación, precisando las características de los mismos.

\newpage

\section{Temario.}

El temario que se trabajará en el curso es el siguiente:

\begin{enumerate}[label=Tema \arabic*., leftmargin=3\parindent]
\item La física y la geometría.
\item Primeras técnicas de solución.
\item Bases completas y ortogonales.
\item Funciones especiales.
\item El átomo de hidrógeno.
\item Transformadas integrales.
\end{enumerate}
En la primera sesión del curso se presentará de manera detallada el contenido de cada tema.

\section{De la asignatura.}

Al ser MAF una asignatura de sexto semestre, se considerará que el alumno ha cubierto el correspondiente avance de la carrera; se requiere de la formación y conocimiento del quinto semestre tanto de la física como de la matemática.
 
\section{Bibliografía.}

Se ocuparán las referencias bibliográficas básicas y complementarias que se indican en el programa del curso, más otros materiales específicos que se darán a conocer en la revisión de los temas.

\section{Primera sesión.}

Se les extiende la más cordial invitación para asistir el día martes 6 de agosto a las 16 horas en el salón, en donde se extenderá este syllabus del curso. En caso de algún comentario, te pedimos que nos envíes un correo y con mucho gusto te daremos respuesta lo más pronto posible.
\end{document}