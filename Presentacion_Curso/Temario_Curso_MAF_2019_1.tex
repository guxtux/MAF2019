\documentclass[12pt]{article}
\usepackage[left=0.3cm,top=2cm,right=0.3cm,bottom=2cm]{geometry}
\usepackage[utf8]{inputenc}
\usepackage[spanish,es-tabla]{babel}
\usepackage{amsmath}
\usepackage{amsthm}
\usepackage{graphicx}
\usepackage{color}
\usepackage{float}
\usepackage{multicol}
\usepackage{enumerate}
\usepackage{anyfontsize}
\usepackage{anysize}
\usepackage{natbib}
\usepackage{enumitem}
\usepackage{capt-of}
\spanishdecimal{.}
\setlist[enumerate]{itemsep=0mm}
\renewcommand{\baselinestretch}{1.2}
\let\oldbibliography\thebibliography
\renewcommand{\thebibliography}[1]{\oldbibliography{#1}
\setlength{\itemsep}{0pt}}
\marginsize{1.5cm}{1.5cm}{1cm}{2cm}
\author{M. en C. Gustavo Contreras Mayén. \texttt{curso.fisica.comp@gmail.com}\\
M. en C. Abraham Lima Buendía. \texttt{abraham3081@ciencias.unam.mx}}
\title{Matemáticas Avanzadas de la Física \\ {\large Semestre 2019-1}}
\date{ }
\makeatletter
\renewcommand{\@biblabel}[1]{}
\renewenvironment{thebibliography}[1]
{\section*{\refname}%
	\@mkboth{\MakeUppercase\refname}{\MakeUppercase\refname}%
	\list{}%
	{\labelwidth=0pt
		\labelsep=0pt
		\leftmargin1.5em
		\itemindent=-1.5em
		\advance\leftmargin\labelsep
		\@openbib@code
	}%
	\sloppy
	\clubpenalty4000
	\@clubpenalty \clubpenalty
	\widowpenalty4000%
	\sfcode`\.\@m}
\makeatother
\usepackage{breakcites}	
\begin{document}
\vspace{-4cm}
%\renewcommand\theenumii{\arabic{theenumii.enumii}}
\renewcommand\labelenumii{\theenumi.{\arabic{enumii}}}
\maketitle
\fontsize{14}{14}\selectfont
\textbf{Lugar: } Pendiente.
\par
\textbf{Horario: } Lunes a viernes de 17 a 18 pm.
\section{Objetivos.}
El alumno:
\begin{itemize}
\setlength{\itemsep}{0mm}
\item Reconocerá las ideas básicas del análisis de ecuaciones que involucran a funciones de varias variables.
\item Formulará aproximaciones consistentes a soluciones, con el fin de cuantificar los distintos mecanismos de la física que se involucran.
\item Consultará la literatura matemática que sea relevante para los problemas de física.
\item Identificará el papel moderno que juegan las funciones especiales, como auxiliares poderosos en el análisis cualitativo de problemas en varias variables.
\end{itemize}
También es nuestro objetivo demostrar al alumno que \emph{las funciones especiales y las transformadas integrales} no son solamente un tema matemático, que involucra las ramas de la geometría diferencial, las ecuaciones diferenciales y el análisis matemático, sino que \emph{son las técnicas de estudio fundamentales} en la electrostática, la electrodinámica, la mecánica cuántica en los límites relativista y  no relativista, la dinámica de medios deformables, la hidrodinámica clásica entre otras ramas de la física.
%\textbf{Punto importante: } Considerando que MAF es una asignatura de sexto semestre, se espera que hayan cursado y acreditado: de Cálculo I a Cálculo IV, Ecuaciones Diferenciales Ordinarias I, Algebra lineal I, Variable Compleja I, la llamada Física Clásica (Física contemporánea, Mecánica vectorial, Fenómenos colectivos, Electromagnetismo, Óptica e Introducción a la Mecánica cuántica)
\section{Metodología de enseñanza.}

En las clases habrá exposición con dialógo por parte de los profesores, junto con el desarrollo de ejercicios que orientarán la solución de problemas en un primer momento de tipo analítico, para luego revisar un ejercicio tomado de la física.

%Se dejarán lecturas complementarias, consulta de referencias bibliográficas, con la finalidad de que consulten y repasen ciertos temas para consolidar la base de conocimiento necesaria para los contenidos del curso, en algunos casos será necesario apoyarse con la solución numérica de ejercicios usando algún lenguaje de programación (\texttt{python}) o con paquetería (\texttt{Wolfram}).
\par
El curso demandará una atención por parte de ustedes con el mismo número de horas fuera de clase (consideramos que sería el mínimo de tiempo), es decir, que deberán de dedicarle fuera del salón de clase, al menos 5 horas a la semana para el desarrollo de lecturas, ejercicios, tareas y actividades.
\section{Temario}
\begin{enumerate}
\item La física y la geometría.
\begin{enumerate}
\item Coordenadas generalizadas.
\item Métrica y teoremas integrales (Gauss y Stokes).
\item Ecuaciones diferenciales de la física.
\end{enumerate}
\item Primeras técnicas de solución.
\begin{enumerate}
\item Separación de variables.
\item Remoción de singularidades y método de Frobenius.
\item Segunda solución independiente.
\item Teorema de Green.
\end{enumerate}
\item Completez y ortogonalidad de una base.
\begin{enumerate}
\item Ecuaciones de tipo Sturm-Liouville.
\item Ortogonalización de Gram-Schimdt.
\item Completez y función delta de Dirac.
\item Desarrollo de una función en una base propia.
\end{enumerate}
\item Función Gamma ($\Gamma$) y función Beta ($\beta$)
\item Separación de variables en coordenadas esféricas.
\begin{enumerate}
\item Ecuación de Legendre.
\item Ecuación asociada de Legendre.
\item Armónicos esféricos, teorema de adición y momento angular.
\end{enumerate}
\item Funciones especiales.
\begin{enumerate}
\item Funciones de Bessel (propagación de ondas cilíndricas y esféricas)
\item Funciones de Hermite (oscilador cuántico)
\item Funciones de Laguerre (átomo de hidrógeno)
\item Funciones de Chebychev (tipos I y II)
\item Funciones hipergeométricas
\end{enumerate}
\item Transformadas integrales.
\begin{enumerate}
\item Transformada de Fourier.
\item Transformada de Laplace.
\end{enumerate}
\end{enumerate}
\section{Evaluación.}
\subsection{Ejercicios semanales.}
Durante las clases se mencionarán ejercicios que se dejarán a modo de tarea, para que puedan trabajarlos oportunamente, hacer consultas, resolver dudas, etc. Al concluir la semana se tendrá un conjunto de ejercicios que deberán de ser entregados al siguiente viernes.

\textbf{Muy importante:} Se considerarán como ejercicios a cuenta de calificación, aquellos ejercicios resueltos que se entreguen y semana al menos el $50\%$ del total, por ejemplo, si se dejan 6 ejercicios para entregar, se espera que al menos entreguen 3 ejercicios para que se revisen y se tomen en cuenta para la calificación, si se entregan menos de tres ejercicios, sólo se revisarán y no contarán para el porcentaje de calificación.
\subsection{Exámenes.}
Habrá 4 exámenes parciales distribuidos de la siguiente manera:
\begin{itemize}
\item \emph{Primer parcial:} Considera los temas 1, 2.
\item \emph{Segundo parcial:} Los temas 3 y 4
\item \emph{Tercer parcial:} Los temas 5 y 6.
\item \emph{Cuarto parcial:} El tema 7.
\end{itemize}

Se entregarán con suficiente tiempo el listado de problemas para cada examen, se entregará su solución de manera individual, no se recibirán los exámenes resueltos en CD, USB, o por correo, sólo a mano.
\par
También se considera que entreguen al menos el $50\%$ de los problemas del examen para que se consideren a cuenta.
\par
El peso para cada elemento de evaluación es el siguiente:
\begin{itemize}
\setlength{\itemsep}{0mm}
\item Ejercicios semanales: $\mathbf{40\%}$.
\item Exámenes parciales: $\mathbf{60\%}$.
\end{itemize}
\textbf{Consideraciones importantes:}
\begin{itemize}
\setlength{\itemsep}{0mm}
\item La entrega de tareas se hará en el salón de clase, teniendo 20 minutos como límite de entrega; no se recibirán tareas por medios electrónicos.
\item Considerando el esquema de trabajo para el curso (1 hora diaria), no habrá exámenes de reposición.
\item En caso de que un examen parcial (o más exámenes) tenga(n) una calificación menor a $6$ (seis), el alumno es candidato para presentar el examen final.
\item Para presentar examen final del curso el alumno debió de haber presentado y entregado los cuatro exámenes parciales del curso.
\item En caso de no haber entregado alguna tarea de ejercicios y no haber presentado un examen del curso, no se tiene derecho a presentar el examen final, por lo que la calificación final que se asentará en el acta, será \textbf{NP (No presentó)}.
\item En caso de haber presentado al menos un examen y/o haber entregado una tarea de ejercicios, se entiende que abandonaron el curso, y no se tiene derecho para presentar el examen final, y la calificación que se asentará en el acta del curso, será $\mathbf{5}$ \textbf{(cinco)}.
\item En caso de presentar la primera ronda del examen final y la calificación sea no aprobatoria ($<6.0$), se puede presentar una segunda y última ronda del examen final, la calificación obtenida en el examen, es la que se asentará en actas.
\end{itemize}
\section{Bibliografía.}
A continuación se indican algunos textos que serán útiles para consulta dentro del curso, de manera adicional, se proporcionarán algunos artículos científicos para revisión de ejemplos aplicados. Se habilitará un compendio de materiales de trabajo mediante una carpeta en Dropbox, tanto de materiales básicos como complementarios, tales como artículos, capítulos de libros, notas, etc. Para contar con la liga de acceso, deberán de enviar un mensaje de correo a Abraham para que les proporcione la liga de Dropbox.

\renewcommand{\refname}{Bibliografía básica.}
\nocite{*}
\bibliographystyle{plain}
\bibliography{LibrosCurso2015-2}

\section{Fechas importantes .}
\begin{itemize}
\item Lunes 6 de agosto. Inicio del semestre 2019-1.
\item Jueves 1 de noviembre. Día de Muertos - Feriado.
\item Viernes 2 de noviembre. Día de Muertos - Feriado.
\item Viernes 23 de noviembre, Fin de Semestre.
\item Del 26 al 30 de noviembre, primera semana de finales.
\item Del 3 al 7 de diciembre,  segunda semana de finales.
\end{itemize}
\end{document}