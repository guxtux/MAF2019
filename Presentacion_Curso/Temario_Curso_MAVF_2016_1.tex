\documentclass[12pt]{article}
\usepackage[left=0.5cm,top=2cm,right=0.5cm,bottom=2cm]{geometry}
\usepackage[utf8]{inputenc}
\usepackage[spanish,es-tabla]{babel}
\usepackage{amsmath}
\usepackage{amsthm}
\usepackage{graphicx}
\usepackage{color}
\usepackage{float}
\usepackage{multicol}
\usepackage{enumerate}
\usepackage{anyfontsize}
\usepackage{anysize}
\usepackage{enumitem}
\usepackage{capt-of}
\spanishdecimal{.}
\setlist[enumerate]{itemsep=0mm}
\renewcommand{\baselinestretch}{1.2}
\let\oldbibliography\thebibliography
\renewcommand{\thebibliography}[1]{\oldbibliography{#1}
\setlength{\itemsep}{0pt}}
%\marginsize{1.5cm}{1.5cm}{0cm}{2cm}
\author{M. en C. Gustavo Contreras Mayén. \texttt{curso.fisica.comp@gmail.com}\\
Fís. Abraham Lima Buendía. \texttt{abraham3081@ciencias.unam.mx}}
\title{Matemáticas Avanzadas de la Física \\ {\large Semestre 2016-1 Grupo 8198}}
\date{ }
\begin{document}
\vspace{-4cm}
%\renewcommand\theenumii{\arabic{theenumii.enumii}}
\renewcommand\labelenumii{\theenumi.{\arabic{enumii}}}
\maketitle
\fontsize{12}{12}\selectfont
\section{Objetivos.}
Dentro el curso el alumno:
\begin{itemize}
\setlength{\itemsep}{0mm}
\item Reconocerá las ideas básicas del análisis de ecuaciones que involucran a funciones de varias variables.
\item Formulará aproximaciones consistentes a soluciones, con el fin de cuantificar los distintos mecanismos de la física que se involucran.
\item Consultará la literatura matemática que sea relevante para los problemas de física.
\item Identificará el papel moderno que juegan las funciones especiales, como auxiliares poderosos en el análisis cualitativo de problemas en varias variables.
\end{itemize}
También es nuestro objetivo demostrar al alumno que las funciones especiales y las transformadas integrales no son solamente un tema matemático, que involucra las ramas de la geometría diferencial, las ecuaciones diferenciales y el análisis matemático, sino que son las técnicas de estudio fundamentales en la electrostática, la electrodinámica, la mecánica cuántica en los límites relativista y  no relativista, la dinámica de medios deformables, la hidrodinámica clásica entre otras ramas de la física.
\\
\\
\textbf{Punto importante: } Considerando que MAF es una asignatura de sexto semestre, se espera que hayan cursado y acreditado: Cálculo I a Cálculo IV, Ecuaciones Diferenciales Ordinarias I, Algebra lineal I, Variable Compleja I, la llamada Física Clásica (Física contemporánea, Mecánica vectorial, Fenómenos colectivos, Electromagnetismo, Óptica e Introducción a la Mecánica cuántica)
\\
\\
\textbf{Lugar: }Aula P115.
\\
\textbf{Horario: } Martes y Jueves de 14 a 16:30 horas.
\section{Metodología de enseñanza.}
En las clases habrá exposición con dialógo por parte de los profesores, junto con el desarrollo de ejercicios que orientarán la solución de problemas en un primer momento de tipo analítico, para luego revisar un ejercicio tomado de la física.
\\
\\
Se dejarán lecturas complementarias, consulta de referencias bibliográficas, y en algunos casos, la solución numérica de ejercicios usando algún lenguaje de programación o paquetería.
\\
\\
El curso demandará como mínimo el mismo número de horas fuera de clase, es decir, que deberán de dedicarle al menos 5 horas a la semana para el desarrollo de lecturas, ejercicios, tareas y actividades.
\section{Temario}
\begin{enumerate}
\item La física y la geometría.
\item Primeras técnicas de solución.
\item Completez y ortogonalidad de una base.
\item Función Gamma($\Gamma$)y función Beta ($\beta$)
\item Separación de variables en coordenadas esféricas.
\item Funciones especiales.
\item Transformadas integrales.
\end{enumerate}
\section{Evaluación.}
La evaluación del curso contempla la entrega de tareas y de la solución de exámenes.
\subsection{Tareas.}
El total de tareas en el curso es de 7, una por cada tema del curso. La evaluación de las tareas se realizará en dos tiempos.
\\
\textbf{Primer tiempo:} Cada tarea tendrá un conjunto de 20 problemas, que se entregarán de manera anticipada con el suficiente tiempo para resolver la totalidad de los problemas; la siguiente tabla (ver tabla \ref{tab:relacion}) muestra la relación entre el número de problemas entregados (y suponemos que bien resueltos) con la calificación que podrían obtener.
\begin{table}[H]
\captionof{table}{Relación de problemas entregados y calificación}
\centering
\begin{tabular}{c | l}
\hline \hline 
Problemas & Calificación \\ [1ex] \hline
$20$ & $10$ (diez) \\ \hline
$19-17$ & $9$ (nueve) \\ \hline
$16-14$ & $8$ (ocho) \\ \hline
$13-11$  & $7$ (siete) \\ \hline
$10$ & $6$ (seis) \\ [1ex] \hline
\end{tabular}
\label{tab:relacion}
\end{table} 
\textbf{Segundo tiempo:} Cada alumno presentará la solución de uno de los problemas (se elegirá al azar), de tal manera que el alumno argumentará su solución en términos de la matemática y física involucrada en el problema, así como su interpretación. Contará con 15 minutos para realizar su exposición. En caso de que en la defensa de la solución no sea consistente (confusión, equivocaciones, evidencia de que el alumno copió la solución del ejercicio) se penalizará restando 2 puntos a la calificación obtenida de la tarea.
\subsection{Exámenes.}
Habrá 3 exámenes parciales de tipo conceptual, distribuidos de la siguiente manera:
\begin{itemize}
\item \emph{Primer parcial:} Considera los temas 1, 2, 3 y 4.
\item \emph{Segunda parcial:} Los temas 5 y 6.
\item \emph{Tercer parcial:} El tema 7.
\end{itemize}
El peso para cada elemento de evaluación es el siguiente:
\begin{itemize}
\setlength{\itemsep}{0mm}
\item En total 7 tareas (una por cada tema): \hspace{1cm} 70 \%.
\item En total 3 exámenes parciales: \hspace{2.5cm} 30\%.
\end{itemize}
\textbf{Consideraciones importantes:}
\begin{itemize}
\setlength{\itemsep}{0mm}
\item La entrega de tareas se hará en el salón de clase, teniendo como límite de entrega: las 2:20 pm; no se recibirán tareas por medios electrónicos.
\item En caso de no acreditar un examen parcial de los tres programados ( calificación $<6.0$), se podrá presentar una sola reposición.
\item Para presentar examen final del curso, deben de cumplirse de manera simultánea los siguientes criterios:
\begin{itemize}
\item Haber presentado los tres exámenes del curso y de éstos: ya sea que dos o los tres exámenes no estén acreditados.
\item Haber entregado todas las tareas.
\end{itemize}  
En caso de que falte alguna tarea y/o un examen del curso, no se cuenta con la oportunidad de presentar el examen final, por lo que el promedio final queda en 5 (cinco).
\\
En caso de presentar el examen final y la calificación sea no aprobatoria ($<6.0$), se puede presentar una segunda y última ronda del examen final, la calificación obtenida en el examen, es la que se asentará en actas.
\end{itemize}
\section{Bibliografía.}
A continuación se indican algunos textos que serán útiles para consulta dentro del curso, de manera adicional, se proporcionarán algunos artículos científicos para revisión de ejemplos aplicados. Se habilitará un compendio de materiales de trabajo mediante una carpeta en Dropbox, tanto de materiales básicos como complementarios, tales como artículos, capítulos de libros, notas, etc. Para contar con la liga de acceso, deberán de enviar un mensaje de correo a Abraham para que les proporcione la liga de Dropbox.
\nocite{*}
\renewcommand{\refname}{Bibliografía básica.}
\bibliography{LibrosCurso2015-2}\bibliographystyle{unsrt}
\section{Fechas importantes.}
\begin{itemize}
\item Lunes 10 de agosto: Inicio del semestre 2016-1.
\item Martes 15 de septiembre, día feriado.
\item Viernes 27 de noviembre: Fin del semestre 2016-1.
\item Del lunes 30 de noviembre  al viernes 4 de diciembre: Primera semana de finales.
\item Del lunes 7 al viernes 11 de diciembre: Segunda semana de finales.
\end{itemize}
\end{document}