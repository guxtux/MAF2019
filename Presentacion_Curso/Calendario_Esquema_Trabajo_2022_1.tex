\documentclass[12pt]{article}
\usepackage[utf8]{inputenc}
\usepackage[spanish]{babel}
\usepackage[letterpaper, landscape, margin=1in]{geometry}
\usepackage{xcolor}
\usepackage{termcal}
\usepackage{tikz}
\renewcommand{\calprintclass}{}
\renewcommand{\calprintdate}{%
     \ifnewmonth\framebox{\arabic{month}/\arabic{date}}%
     \else\arabic{date}%
     \fi}
\definecolor{awesome}{rgb}{1.0, 0.13, 0.32}
\definecolor{blush}{rgb}{0.87, 0.36, 0.51}
\definecolor{capri}{rgb}{0.0, 0.75, 1.0}
\definecolor{darkcyan}{rgb}{0.0, 0.55, 0.55}
\definecolor{darktangerine}{rgb}{1.0, 0.66, 0.07}
\definecolor{cadmiumgreen}{rgb}{0.0, 0.42, 0.24}
\definecolor{darkslateblue}{rgb}{0.28, 0.24, 0.55}

%\author{M. en C. Gustavo Contreras Mayén. \texttt{gux7avo@ciencias.unam.mx}\\
%M. en C. Abraham Lima Buendía. \texttt{abraham3081@ciencias.unam.mx}}
\title{}
\author{}
\begin{document}
\date{}
\maketitle
\vspace*{-4cm}
\begin{calendar}{9/20/21}{3}
\setlength{\calwidth}{1.05\textwidth} 
\setlength{\calboxdepth}{1.5in}
% Description of the Week.

\calday[Lunes]{\classday}
\calday[Martes]{\classday}
\calday[Miércoles]{\classday}
\calday[Jueves]{\classday}
\calday[Viernes]{\classday}
\skipday\skipday % weekend (no class)


% Holidays
\options{11/2/20}{\noclassday}
\caltext{11/2/20}{\color{red}{Día Feriado}}
\options{11/16/20}{\noclassday}
\caltext{11/16/20}{\color{red}{Día Feriado}}
\options{2/1/21}{\noclassday}
\caltext{2/1/21}{\color{red}{Día Feriado}}

%Temas del curso
\caltext{9/20/21}{\color{blue}{\textbf{Videoconferencia: Inicio del Tema}}}
\caltext{10/9/20}{\color{red}\textbf{Entrega de ejercicios Semana 1}}
%\textbf{Sesión con videoconferencia:} Se presentará el tema, el alcance del mismo, así como la actividades que el alumno deberá de atender durante la semana. 
% \caltext{9/22/20}{\color{darkslateblue}{\textbf{}}}
\end{calendar}
\begin{tikzpicture}[overlay]
    \node at (0.7, 11) {\textbf{Semana 1}};
     
    \draw [fill, color=yellow, text=black, text width = 14.5cm] (4.5, 9.7) rectangle (23.1, 11.7) node[pos=0.5] {El alumno consulta los materiales de trabajo, los complementarios en la plataforma, así como de los videos y tendrá los ejercicios que deberá de entregar el viernes de la siguiente semana.};

    \draw [fill, color=darkcyan, text=white] (18.7, 12.3) rectangle (22.9, 13) node[pos=0.5] {Videoconferencia};

    \node at (0.7, 8.5) {\textbf{Semana 2}};

    \draw [fill, color=darkcyan, text=white] (9.2, 8) rectangle (13.4, 8.7) node[pos=0.5] {Videoconferencia};

    \draw [fill, color=darkcyan, text=white] (18.7, 8) rectangle (22.9, 8.7) node[pos=0.5] {Videoconferencia};

    \draw [fill, color=red, text=white] (18.7, 7) rectangle (22.9, 7.7) node[pos=0.5] {Entrega Tarea S1};

    \draw [fill, color=yellow, text=black, text width = 18cm] (-0.5, 5.3) rectangle (23.1, 6.75) node[pos=0.5] {El alumno consulta los materiales de trabajo, los complementarios en la plataforma, así como de los videos y tendrá los ejercicios que deberá de entregar el viernes de la siguiente semana.};

    \node at (0.7, 3.75) {\textbf{Semana 3}};

    \draw [fill, color=darkcyan, text=white] (9.2, 3.6) rectangle (13.4, 4.3) node[pos=0.5] {Videoconferencia};

    \draw [fill, color=darkcyan, text=white] (18.7, 3.6) rectangle (22.9, 4.3) node[pos=0.5] {Videoconferencia};

    \draw [fill, color=red, text=white] (18.7, 2.6) rectangle (22.9, 3.3) node[pos=0.5] {Entrega Tarea S2};

    \draw [fill, color=yellow, text=black, text width = 18cm] (-0.5, 0.9) rectangle (23.1, 2.3) node[pos=0.5] {El alumno consulta los materiales de trabajo, los complementarios en la plataforma, así como de los videos y tendrá los ejercicios que deberá de entregar el viernes de la siguiente semana.};
     
\end{tikzpicture}
\end{document}