\documentclass[hidelinks,12pt]{article}
\usepackage[left=0.25cm,top=1cm,right=0.25cm,bottom=1cm]{geometry}
%\usepackage[landscape]{geometry}
\textwidth = 20cm
\hoffset = -1cm
\usepackage[utf8]{inputenc}
\usepackage[spanish,es-tabla]{babel}
\usepackage[autostyle,spanish=mexican]{csquotes}
\usepackage[tbtags]{amsmath}
\usepackage{nccmath}
\usepackage{amsthm}
\usepackage{amssymb}
\usepackage{mathrsfs}
\usepackage{graphicx}
\usepackage{subfig}
\usepackage{standalone}
\usepackage[outdir=./Imagenes/]{epstopdf}
\usepackage{siunitx}
\usepackage{physics}
\usepackage{color}
\usepackage{float}
\usepackage{hyperref}
\usepackage{multicol}
%\usepackage{milista}
\usepackage{anyfontsize}
\usepackage{anysize}
%\usepackage{enumerate}
\usepackage[shortlabels]{enumitem}
\usepackage{capt-of}
\usepackage{bm}
\usepackage{relsize}
\usepackage{placeins}
\usepackage{empheq}
\usepackage{cancel}
\usepackage{wrapfig}
\usepackage[flushleft]{threeparttable}
\usepackage{makecell}
\usepackage{fancyhdr}
\usepackage{tikz}
\usepackage{bigints}
\usepackage{scalerel}
\usepackage{pgfplots}
\usepackage{pdflscape}
\pgfplotsset{compat=1.16}
\spanishdecimal{.}
\renewcommand{\baselinestretch}{1.5} 
\renewcommand\labelenumii{\theenumi.{\arabic{enumii}})}
\newcommand{\ptilde}[1]{\ensuremath{{#1}^{\prime}}}
\newcommand{\stilde}[1]{\ensuremath{{#1}^{\prime \prime}}}
\newcommand{\ttilde}[1]{\ensuremath{{#1}^{\prime \prime \prime}}}
\newcommand{\ntilde}[2]{\ensuremath{{#1}^{(#2)}}}

\newtheorem{defi}{{\it Definición}}[section]
\newtheorem{teo}{{\it Teorema}}[section]
\newtheorem{ejemplo}{{\it Ejemplo}}[section]
\newtheorem{propiedad}{{\it Propiedad}}[section]
\newtheorem{lema}{{\it Lema}}[section]
\newtheorem{cor}{Corolario}
\newtheorem{ejer}{Ejercicio}[section]

\newlist{milista}{enumerate}{2}
\setlist[milista,1]{label=\arabic*)}
\setlist[milista,2]{label=\arabic{milistai}.\arabic*)}
\newlength{\depthofsumsign}
\setlength{\depthofsumsign}{\depthof{$\sum$}}
\newcommand{\nsum}[1][1.4]{% only for \displaystyle
    \mathop{%
        \raisebox
            {-#1\depthofsumsign+1\depthofsumsign}
            {\scalebox
                {#1}
                {$\displaystyle\sum$}%
            }
    }
}
\def\scaleint#1{\vcenter{\hbox{\scaleto[3ex]{\displaystyle\int}{#1}}}}
\def\bs{\mkern-12mu}


\usepackage{apacite}
\title{Matemáticas Avanzadas de la Física \\ Syllabus del curso  \vspace{-3ex}}
\author{M. en C. Gustavo Contreras Mayén}
\date{ }

\begin{document}
\vspace{-4cm}
\maketitle
\fontsize{14}{14}\selectfont
\tableofcontents
\newpage

\section{Presentación del curso}

\subsection{Objetivos}

En la página de la Facultad, el programa de la asignatura: Matemáticas Avanzadas de la Física se puede consultar \href{https://www.fciencias.unam.mx/sites/default/files/temario/610.pdf}{(aquí)}, y contiene los siguientes objetivos:
\par
\noindent
En donde el alumno:
\begin{itemize}
\setlength{\itemsep}{0mm}
\item Reconocerá las ideas básicas del análisis de ecuaciones que involucran a funciones de varias variables.
\item Formulará aproximaciones consistentes a soluciones, con el fin de cuantificar los distintos mecanismos de la física que se involucran.
\end{itemize}

Además:
\begin{itemize}
\setlength{\itemsep}{0mm}
\item Consultará la literatura matemática que sea relevante para los problemas de física.
\item Identificará el papel moderno que juegan las funciones especiales, como auxiliares poderosos en el análisis cualitativo de problemas en varias variables.
\end{itemize}

También es nuestro objetivo demostrar al alumno que \emph{las funciones especiales y las transformadas integrales} no son solamente un tema matemático, que involucra las ramas de la geometría diferencial, las ecuaciones diferenciales y el análisis matemático.
\par
Veremos que \emph{son las técnicas de estudio fundamentales} en la electrostática, la electrodinámica, la mecánica cuántica en los límites relativista y no relativista, la dinámica de medios deformables, la hidrodinámica clásica entre otras ramas de la física.
\par
MAF les brindará un manejo más fluido y consistente para lo que van a cursar en el sexto semestre y los tres semestres que les restan para concluir la carrera.
\par
Es una asignatura con bastante relevancia para la formación del físico.

\section{Metodología de enseñanza.}

\subsection{Semestre presencial}

En acuerdo con el programa de horarios el intersemestral 2023-3 tendrá una modalidad presencial.
\par
Las sesiones se llevarán a cabo los días establecidos: \textbf{lunes, martes, miércoles y jueves} de 11 am a las 16:30 pm en el aula que se indique.

\section{Evaluación}

\subsection{Elementos para la calificación.}

El porcentaje para cada elemento de evaluación es el siguiente:
\begin{itemize}
\setlength{\itemsep}{0mm}
\item Evaluación semanal: $\mathbf{80\%}$.
\item Ejercicios: $\mathbf{20\%}$.
\end{itemize}

\subsection{Ejercicios.}

Se presentarán 15 ejercicios a resolver durante el intersemestral. Los enunciados de lo ejercicios se entregarán durante cada clase.  Al concluir la semana, se deberá de enviar la solución vía correo electrónico.
\par
Cada ejercicio aporta un punto, siempre y cuando esté bien resuelto. En caso de que que el ejercicio no se haya resuelto debidamente, se otorgará una parte proporcional del punto. Recomendamos ampliamente la solución de todos los ejercicios.

\subsection{Evaluación semanal.}

Dado el formato del intersemestral, se tendrán tres evaluaciones durante el curso.

\begin{enumerate}
\item Semana 1: Temas 1 y 2.
\item Semana 2: Temas 3 y 4.
\item Semana 3: Temas 5 y 6.
\end{enumerate}

Los enunciados de cada evaluación se entregarán a media semana, de esta manera se tendrá el suficiente tiempo para la solución y entrega del mismo, que sería para las semanas 1 y 2, para el lunes de la siguiente semana. En promedio se incluirán 5 preguntas por cada semana.
\par
En la semana 3 se tendrá que hacer un ajuste para la entrega de la evaluación por la conclusión del intersemestral.
\par
Durante la calificación de los ejercicios de las evaluaciones y ejercicios  \emph{se revisa y se evalúa el proceso de resolución de un problema, es decir, será necesario detallar cada paso en la solución}. Los ejercicios así como las evaluaciones tendrán que ser resueltos a mano, en la solución se deberá de detallar cada paso que se realice. No se recibirán soluciones que indiquen comprobaciones hechas en \emph{Mathematica, MatLab, Maple, etc.}
\par 
¿Cómo entregar los Ejercicios y Evaluaciones? \\
Si cuentan con un escáner, se deberá de digitalizar cada una de las hojas que hayan ocupado en un archivo pdf y enviarlo mediante la plataforma.
\par
En caso de no contar con un escáner para la digitalización de las soluciones, se podrá enviar un archivo con las imágenes de la solución, tomadas con una tableta o el celular.
\par
Se les pedirá encarecidamente, ser lo más claros en la escritura, numeración de las hojas, el orden y limpieza en sus soluciones, para que así recibamos un archivo que nos permita evaluar su trabajo.

\section{Temario}

\subsection{La física y la geometría}

\begin{enumerate}[label=4.1.\arabic*]
\item Sistemas de coordenadas curvilíneas ortogonales.
\item Operadores diferenciales en coordenadas curvilíneas.
\item Funciones Gamma y Beta.
\end{enumerate}

\subsection{Primeras técnicas de solución}

\begin{enumerate}[label=4.2.\arabic*]
\item Técnica de separación de variables.
\item Método de Frobenius y remoción de singularidades.
\item Segunda solución linealmente independiente.
\item Función de Green.
\end{enumerate}

\subsection{Bases completas y ortogonales}

\begin{enumerate}[label=4.3.\arabic*]
\item La delta de Dirac. 
\item Ecuaciones de tipo Sturm-Liouville.
\item Ortogonalización de Gram-Schimdt.
\item Completes de las funciones propias.
\end{enumerate}

\subsection{Funciones especiales}

\begin{enumerate}[label=4.4.\arabic*]
\item Funciones de Bessel. (Propagación de ondas cilíndricas)
\item Funciones de Hermite. (Oscilador armónico cuántico)
\item Funciones de Chebyshev (Interpolación numérica)
\item Funciones hipergeométricas: ordinaria y confluente.
\item Funciones de Gegenbauer.
\end{enumerate}

\subsection{Sep. variables en coord. esféricas}

\begin{enumerate}[label=4.5.\arabic*]
\item Ec. asociada de Laguerre. y la Ec. ordinaria de Laguerre.
\item Armónicos esféricos.
\item Teorema de adición.
\end{enumerate}

\subsection{Transformadas integrales}

\begin{enumerate}[label=4.6.\arabic*]
\item Transformada de Fourier.
\item Transformada de Laplace.
\item Transformada discreta de Fourier.
\item Transformada rápida de Fourier.
\end{enumerate}

\end{document}