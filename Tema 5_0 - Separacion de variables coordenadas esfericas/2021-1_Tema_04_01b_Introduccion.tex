\input{../Preambulos/preambulo_presentacion_Dresden_seahorse}
\title{\large{Polinomios de Legendre}}
\subtitle{Construcción alterna - 2a. parte}
\author{M. en C. Gustavo Contreras Mayén}
\date{}
\institute{Facultad de Ciencias - UNAM}
\titlegraphic{\includegraphics[width=1.75cm]{../Imagenes/escudo-facultad-ciencias}\hspace*{4.75cm}~%
   \includegraphics[width=1.75cm]{../Imagenes/escudo-unam}
}
\setbeamertemplate{navigation symbols}{}
\begin{document}
\maketitle
\fontsize{14}{14}\selectfont
\spanishdecimal{.}
\section*{Contenido}
\frame[allowframebreaks]{\tableofcontents[currentsection, hideallsubsections]}
\section{Función generatriz}
\frame{\tableofcontents[currentsection, hideothersubsections]}
\subsection{Derivando con respecto a \texorpdfstring{$t$}{t}}
\begin{frame}
\frametitle{La función generatriz}
A partir del estudio del desarrollo multipolar, presentamos la función generatriz de los Polinomios de Legendre:
\begin{align*}
g(x, t) = \dfrac{1}{\left[ 1 - 2 \, t \, x + t^{2} \right]^{1/2} } = \sum_{\ell=0}^{\infty} t^{\ell} \, P_{\ell} (x)
\end{align*}
\end{frame}
\begin{frame}
\frametitle{Cálculo de los polinomios}
Y hemos llegado a una expresión que nos permite calcular los Polinomios de Legendre de orden $\ell$.
\\
\bigskip
\pause
Estudiaremos ahora a la derivada parcial de la función generatriz.
\end{frame}
\begin{frame}
\frametitle{Derivada parcial}
La derivada parcial con respecto a t, de la función generatriz es:
\begin{eqnarray*}
\pdv{t} g(x, t) = \pause \dfrac{x - t}{( 1 - 2 x \, t + t^{2})^{3/2}} = \sum_{n=0}^{\infty} \ell \, P_{\ell} (x) t^{\ell -1}
\end{eqnarray*}
\end{frame}
\begin{frame}
\frametitle{Extendiendo el desarrollo}
Así tenemos que:
\begin{align*}
&\Rightarrow \left\{ \dfrac{x - t}{( 1 - 2 x \, t + t^{2})^{1/2}} \right\} \left\{ \dfrac{1}{ 1 - 2 x \, t + t^{2}} \right\} = \\[0.5em]
&= \sum_{n=0}^{\infty} \ell \, P_{\ell} (x) \, t^{\ell -1}
\end{align*}
\pause
Al usar de nuevo la definición de los Polinomios de Legendre:
\end{frame}
\begin{frame}
\frametitle{Extendiendo el desarrollo}
Ahorrando espacio, escribiremos $P_{\ell}$. Entonces:
\begin{align*}
(t {-} x) \sum_{\ell=0}^{\infty} P_{\ell} \, t^{\ell} + ( 1 {-} 2 x t {+} t^{2}) \, \sum_{\ell=1}^{\infty} \ell \, P_{\ell} \, t^{\ell-1} = 0
\end{align*}
\pause
Ordenando términos de menor a mayor potencia:
\begin{align*}
(1 {-} 2 x t {+} t^{2}) \, \sum_{\ell=1}^{\infty} \ell \, P_{\ell} \, t^{\ell-1} {+} (t {-} x) \sum_{\ell=0}^{\infty} P_{\ell} \, t^{\ell} = 0
\end{align*}
\end{frame}
\begin{frame}
\frametitle{Extendiendo las sumas}
Ya que cada una de las sumas tiene factores previos, hay que separarlos y revisar lo que ocurre con cada una de las sumas:
\end{frame}
\begin{frame}
\frametitle{Extendiendo las sumas}
\begin{align*}
&\sum_{\ell=1}^{\infty} \ell \, P_{\ell} \, t^{\ell-1} - 2 \, x \, \sum_{\ell=1}^{\infty} \ell \, P_{\ell} \, t^{\ell} - x \, \sum_{\ell=0}^{\infty} P_{\ell} \, t^{\ell} + \\[0.5em]
&+ \sum_{\ell=1}^{\infty} \ell \, P_{\ell} \, t^{\ell+1} + \sum_{\ell=0}^{\infty} P_{\ell} \, t^{\ell+1} = 0
\end{align*}
\end{frame}
\begin{frame}
\frametitle{Ajustando índices y potencias}
Ya que tenemos sumas cuyos índices son distintos, además de la potencia de $t$, por lo que habrá que uniformizar tanto el índice de las sumas y la potencia de la variable $t$.
\\
\bigskip
\pause
Dejemos el índice de las sumas comience en $\ell = 2$, tal que las potencias queden en $t^{\ell}$. 
\end{frame}
\begin{frame}
\frametitle{Ajustando cada suma}
Primera suma:
\begin{align*}
\sum_{\ell=1}^{\infty} \ell \, P_{\ell} \, t^{\ell-1} = P_{1} +  2 \, P_{2} \, t + \sum_{\ell=2}^{\infty} (\ell + 1) \, P_{\ell+1} \, t^{\ell}
\end{align*}
\pause
Segunda suma:
\begin{align*}
- 2 \, x \, \sum_{\ell=1}^{\infty} \ell \, P_{\ell} \, t^{\ell} = - 2 \, x \, P_{1} \, t \,  \sum_{\ell=2}^{\infty} P_{\ell} \, t^{\ell}
\end{align*}
\end{frame}
\begin{frame}
\frametitle{Ajustando cada suma}
Tercera suma:
\begin{align*}
- x \, \sum_{\ell=0}^{\infty} P_{\ell} \, t^{\ell} = - x \, P_{0} - x \, P_{1} \, t - \sum_{\ell=2}^{\infty} x \, P_{\ell} \, t^{\ell}
\end{align*}
\pause
Cuarta suma:
\begin{align*}
\sum_{\ell=1}^{\infty} \ell \, P_{\ell} \, t^{\ell+1} = \sum_{\ell=2} (\ell - 1) P_{\ell-1} \, t^{\ell}
\end{align*}
\end{frame}
\begin{frame}
\frametitle{Ajustando cada suma}
Quinta suma:
\begin{align*}
\sum_{\ell=0}^{\infty} P_{\ell} \, t^{\ell+1} = P_{0} \, t + \sum_{\ell=2}^{\infty} P_{\ell-1} \, t^{\ell}
\end{align*}
\pause
Arreglamos todos los términos de la suma, comenzando con los términos independientes de $t$, luego los que se multiplican por $t$ y al final, los que se multiplican por $t^{\ell}$.
\end{frame}
\begin{frame}
\frametitle{Todos los términos}
\begin{align*}
&\bigg[ P_{1} - x \, P_{0} \bigg] + \bigg[ 2 \, P_{2} - 3 \, x \, P_{1} + P_{0} \bigg] \, t + \\[0.5em]
&+ \sum_{\ell=2}^{\infty} \bigg[ (\ell + 1) \, P_{\ell+1} {-} (2 \, \ell + 1) \, x \, P_{\ell} {+} \ell \, P_{\ell-1} \bigg] \, t^{\ell} = 0
\end{align*}
\pause
Como sabemos que al tener una serie infinita, todos los coeficientes de la serie deben de anularse, por lo que:
\end{frame}
\begin{frame}
\frametitle{Todos los términos}
Sabemos de antemano que $P_{0} = 1$ y $P_{1} = x$, entonces:
\pause
\begin{align*}
&\bigg[ P_{1} - x \, P_{0} \bigg] + \bigg[ 2 \, P_{2} - 3 \, x \, P_{1} + P_{0} \bigg] \, t + \\[0.5em]
&+ \sum_{\ell=2}^{\infty} \bigg[ (\ell + 1) \, P_{\ell+1} {-} (2 \, \ell + 1) \, x \, P_{\ell} {+} \ell \, P_{\ell-1} \bigg] \, t^{\ell} = 0
\end{align*}
\begin{tikzpicture}[overlay]
\draw[fill=red, opacity=0.3] (0.2, 2.25) rectangle (2.6, 3.6); \pause
\draw[fill=yellow, opacity=0.3] (3.3, 2.25) rectangle (7.6, 3.6); \pause
\draw[fill=blue, opacity=0.3] (1.5, 0.5) rectangle (9.3, 2);
\end{tikzpicture}
\end{frame}
\begin{frame}
\frametitle{Regla de recurrencia}
Entonces tendremos en particular que:
\begin{align*}
(\ell + 1) \, P_{\ell+1} (x) - (2 \, \ell + 1) \, x \, P_{\ell} (x) + \ell \, P_{\ell-1} (x) = 0
\end{align*}
Esta relación es válida para $\forall \, \ell$, entonces podremos construir el polinomio de Legendre del grado deseado, a partir de $P_{0}(x)$ y de $P_{1}(x)$.
\end{frame}
\subsection{Derivando con respecto a \texorpdfstring{$x$}{x}}
\begin{frame}
\frametitle{Derivando con respecto a $x$}
Ahora tomamos la derivada parcial con respecto a $x$ de la función generatriz:
\pause
\begin{align*}
\pdv{x} g(x, t) = \dfrac{t}{(1 - 2 \, x \, t + t^{2})^{3/2}} = \sum_{\ell=0}^{\infty} \ptilde{P}_{\ell} (x) \, t^{\ell}
\end{align*}
\pause
Nuevamente, ocupamos la definición de la función generatriz, para obtener así:
\end{frame}
\begin{frame}
\frametitle{Los términos obtenidos:}
\begin{align*}
\dfrac{t}{(1 - 2 \, x \, t + t^{2})^{1/2}} = (1 - 2 \, x \, t + t^{2}) \, \sum_{\ell=0}^{\infty} \ptilde{P}_{\ell} (x) \, t^{\ell}
\end{align*}
\pause
Entonces:
\begin{align*}
t \, \sum_{\ell=0}^{\infty} P_{\ell} (x) \, t^{\ell} = (1 - 2 \, x \, t + t^{2}) \, \sum_{\ell=0}^{\infty} \ptilde{P}_{\ell} (x) \, t^{\ell}
\end{align*}
\end{frame}
\begin{frame}
\frametitle{Separando la suma}
Extendemos la suma que tenemos del lado derecho de la igualdad:
\pause
\begin{align*}
&t \, \sum_{\ell=0}^{\infty} P_{\ell} (x) \, t^{\ell} = \sum_{\ell=0}^{\infty} \ptilde{P}_{\ell} (x) \, t^{\ell} - 2 \, x \, \sum_{\ell=0}^{\infty} \ptilde{P}_{\ell} (x) \, t^{\ell+1} + \\[0.5em]
&+ \sum_{\ell=0}^{\infty} \ptilde{P}_{\ell} (x) \, t^{\ell+2}
\end{align*}
\end{frame}
\begin{frame}
\frametitle{Simplificando la expresión}
Como sabemos que $P_{0}(x) = 1 \, \Rightarrow \, \ptilde{P}_{0}(x)= 0$, así:
\pause
\begin{align*}
&\sum_{\ell=0}^{\infty} P_{\ell} (x) \, t^{\ell+1} = \sum_{\ell=1}^{\infty} \ptilde{P}_{\ell} (x) \, t^{\ell} - 2 \, x \, \sum_{\ell=1}^{\infty} \ptilde{P}_{\ell} (x) \, t^{\ell+1} + \\[0.5em]
&+ \sum_{\ell=1}^{\infty} \ptilde{P}_{\ell} (x) \, t^{\ell+2}
\end{align*}
\pause
Procedemos a ajustar los índices de las sumas:
\end{frame}
\begin{frame}
\frametitle{Ajustando los índices de las sumas}
\begin{align*}
&P_{0}(x) \, t \, \sum_{\ell=1}^{\infty} P_{\ell} (x) \, t^{\ell+1} = \ptilde{P}_{1} (x) \, t {+} \sum_{\ell=1}^{\infty} \ptilde{P}_{\ell+1} (x) \, t^{\ell+1} + \\[0.5em]
&- 2 \, x \, \sum_{\ell=1}^{\infty} \ptilde{P}_{\ell} (x) \, t^{\ell+1} + \sum_{\ell=1}^{\infty} \ptilde{P}_{\ell-1} (x) \, t^{\ell+1}
\end{align*}
\pause
Una vez dejando los índices de las sumas con el mismo valor, así como la potencia de $t$, agrupamos los términos:
\end{frame}
\begin{frame}
\frametitle{Agrupando los términos}
Tomemos en cuenta que $\ptilde{P}_{1}(x) = P_{0}(x) = 1$, así:
\pause
\begin{align*}
\sum_{\ell=1}^{\infty} t^{n+1} \bigg[ \ptilde{P}_{\ell+1} (x) {+} \ptilde{P}_{\ell-1} (x) {-} 2 x \, \ptilde{P}_{\ell} (x) {+} P_{\ell} (x) \bigg] = 0
\end{align*}
\end{frame}
\begin{frame}
\frametitle{Resultado importante}
Por lo que:
\begin{align*}
\setlength{\fboxsep}{3\fboxsep}\boxed{
\ptilde{P}_{\ell+1} (x) + \ptilde{P}_{\ell-1} (x) = 2 \, x \, \ptilde{P}_{\ell} (x) + P_{\ell} (x)
}
\end{align*}
\pause
Que es válida $\forall \, \ell$.
\end{frame}
\begin{frame}
\frametitle{Observación}
La expresión que se obtuvo relaciona la derivadas de los polinomios de Legendre de grados vecinos, vemos que incluye a $\ptilde{P}_{\ell}(x)$.
\\
\bigskip
\pause
Nos interesa tener una expresión que dependa solo de órdenes previos:
\end{frame}
\subsection{Expresión para la derivada}
\begin{frame}
\frametitle{Nueva expresión}
Partimos entonces de la siguiente igualdad:
\begin{align*}
(\ell + 1) \, P_{\ell+1} (x) - (2 \, \ell + 1) \, x \, P_{\ell} (x) + \ell \, P_{\ell-1} (x) = 0
\end{align*}
\pause
Que ahora derivamos con respecto a $x$:
\end{frame}
\begin{frame}
\frametitle{Derivando la igualdad}
Entonces:
\begin{align}
\begin{aligned}
&(\ell + 1) \ptilde{P}_{\ell+1} (x) - (2 \, \ell + 1) P_{\ell} (x) + \\[0.5em]
&- (2 \, \ell +  1) \, x \ptilde{P}_{\ell} (x) + \ell \, \ptilde{P}_{\ell-1} (x) = 0
\end{aligned}
\label{eq;ecuacion_a}
\end{align}
\end{frame}
\begin{frame}
\frametitle{Ocupando un resultado}
Usamos ahora la igualdad encontrada al derivar la función generatriz:
\begin{align}
\begin{aligned}
&\dfrac{(2 \, \ell + 1)}{2} \, \ptilde{P}_{\ell+1} (x) + \dfrac{(2 \, \ell + 1)}{2} \, \ptilde{P}_{\ell-1} (x) + \\[0.5em]
&- (2 \, \ell + 1) \, x \, \ptilde{P}_{\ell} (x) - \dfrac{(2 \, \ell + 1)}{2} \, \ptilde{P}_{\ell} (x) = 0
\end{aligned}
\label{eq:ecuacion_b}
\end{align}
\pause
Como siguiente paso: restamos la ec. (\ref{eq;ecuacion_a}) de (\ref{eq:ecuacion_b}):
\end{frame}
\begin{frame}
\frametitle{Restando de las expresiones}
\begin{align*}
&\left[ \ell + \dfrac{1}{2} {-} \ell {-} 1 \right] \, \ptilde{P}_{\ell+1} (x) + \left[ 2 \, \ell {+} 1 {-} \ell {-} \dfrac{1}{2} \right] \, P_{\ell} (x) + \\[0.5em]
&+ \left[ \ell + \dfrac{1}{2} - \ell \right] \, \ptilde{P}_{\ell-1} (x) = 0
\end{align*}
\pause
Reduciendo la expresión:
\begin{align*}
\left[ - \dfrac{1}{2} \right] \, \ptilde{P}_{\ell+1} (x) {+} \left[ \ell + \dfrac{1}{2} \right] \, P_{\ell} (x) {+} \left[ \dfrac{1}{2} \right] \, \ptilde{P}_{\ell-1} (x) = 0
\end{align*}
\end{frame}
\begin{frame}
\frametitle{Nueva expresión}
La nueva expresión queda como:
\begin{align*}
\setlength{\fboxsep}{3\fboxsep}\boxed{
(2 \, \ell + 1) \, P_{\ell} (x) =    \ptilde{P}_{\ell+1} (x) - \ptilde{P}_{\ell-1} (x) = 0
}
\end{align*}
\pause
Esta expresión requiere solo de órdenes posterior y anterior de la derivada de $P_{\ell}(x)$.
\end{frame}
\subsection{Relaciones de recurrencia}
\begin{frame}
\frametitle{Creando relaciones de recurrencia}
Con un poco de álgebra se pueden encontrar relaciones similares:
\begin{align*}
\ptilde{P}_{\ell+1} &= (\ell + 1) \, P_{\ell}(x) + x \, \ptilde{P}_{\ell} \\[0.5em]
\ptilde{P}_{\ell-1} &= - \ell \, P_{\ell}(x) + x \, \ptilde{P}_{\ell} \\[0.5em]
(1 - x^{2}) P_\ell (x) &= \ell \, P_{\ell-1} (x) - \ell \, x \, P_{\ell} (x)
\end{align*}
\end{frame}
\subsection{Polinomio ordinario de Legendre}
\begin{frame}
\frametitle{Obteniendo el polinomio ordinario}
Usando la última expresión de las relaciones de recurrencia que mostramos:
\begin{align*}
(1 - x^{2}) \ptilde{P}_\ell (x) &= \ell \, P_{\ell-1} (x) - \ell \, x \, P_{\ell} (x)
\end{align*}
\pause
La derivamos con respecto a $x$, para obtener:
\begin{align*}
&(1 {-} x^{2}) \stilde{P}_\ell (x) {-} 2 \, x \, \ptilde{P}_\ell (x) = \ell \, \ptilde{P}_{\ell-1} (x) - \ell \, x \, P_{\ell} (x) + \\[0.5em]
&- \ell \, x \, \ptilde{P}_{\ell}(x)
\end{align*}
\end{frame}
\begin{frame}
\frametitle{Obteniendo el polinomio ordinario}
Ocuparemos el siguiente resultado \pause (¿lo podrías obtener?):
\begin{align*}
\ell \, \ptilde{P}_{\ell} (x) = - \ell^{2} \, P_{\ell} (x) +  n \, x \, \ptilde{P}_{\ell} (x)
\end{align*}
para eliminar el término $\ptilde{P}_{\ell-1} (x)$
\end{frame}
\begin{frame}
\frametitle{Obteniendo el polinomio ordinario}
\begin{eqnarray*}
&(&1 {-} x^{2}) \stilde{P}_\ell (x) {-} 2 \, x \, \ptilde{P}_\ell (x) = \ell^{2} \, \ptilde{P}_{\ell} (x) + \cancel{\ell \, x \, \ptilde{P_{\ell} (x)}} + \\[0.5em]
&-& \ell \, P_{\ell}(x) - \cancel{\ell \, x \, \ptilde{P}_{\ell}(x)} = \\[0.5em] \pause
&=& \ell (\ell + 1) \, P_{\ell} (x)
\end{eqnarray*}
Por tanto:
\end{frame}
\begin{frame}
\frametitle{El polinomio ordinario}
Tenemos que:
\begin{align*}
\setlength{\fboxsep}{3\fboxsep}\boxed{
(1 {-} x^{2}) \stilde{P}_\ell (x) {-} 2 \, x \, \ptilde{P}_\ell (x) + \ell (\ell + 1) \, P_{\ell} (x) = 0
}
\end{align*}
\pause
A esta ecuación se le conoce en la literatura de física matemática como el \emph{Polinomio Ordinario de Legendre}.
\\
\bigskip
\pause
Si resolvemos esta EDO2H mediante el método de Frobenius, obtendremos los mismos resultados.
\end{frame}
\subsection{Paridad de los polinomios}
\begin{frame}
\frametitle{Paridad de los polinomios}
Otra característica importante es la \emph{paridad} de los polinomios de Legendre.
\\
\bigskip
\pause
Para ello debemos de estudiar el cambio de variables:
\begin{align*}
x \hspace{0.2cm} \rightarrow \hspace{0.2cm} - x \hspace{1.5cm} t \hspace{0.2cm} \rightarrow \hspace{0.2cm} - t
\end{align*}
\end{frame}
\begin{frame}
\frametitle{Aplicando el cambio de variable}
Hacemos el cambio de variable en la función generatriz
\begin{eqnarray*}
g(x, t) &=& \dfrac{1}{(1 - 2 \, x \, t + t^{2})^{1/2}} = \sum_{\ell=0}^{\infty} P_{\ell} (x) \, t^{\ell} = \\[0.5em] \pause
&=& \dfrac{1}{(1 {-} 2 \, (-x) \, (-t) {+} (-t)^{2})^{1/2}} = \\[0.5em] \pause
&=& \sum_{\ell=0}^{\infty} P_{\ell} (-x) \, (-t)^{\ell} = \pause \sum_{\ell=0}^{\infty} (-1)^{\ell} \, P_{\ell} (-x) \, t^{\ell} = 
\end{eqnarray*}
\end{frame}
\begin{frame}
\frametitle{Paridad de los polinomios}
Entonces la relación de paridad de los polinomios de Legendre está dada por la expresión:
\begin{align*}
\setlength{\fboxsep}{3\fboxsep}\boxed{
P_{\ell}(x) = (-1)^{\ell} \, P_\ell(-x)
}
\end{align*}
\end{frame}
\end{document}