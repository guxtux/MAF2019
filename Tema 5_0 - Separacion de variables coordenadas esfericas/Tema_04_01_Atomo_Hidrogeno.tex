\documentclass[hidelinks,12pt]{article}
\usepackage[left=0.25cm,top=1cm,right=0.25cm,bottom=1cm]{geometry}
%\usepackage[landscape]{geometry}
\textwidth = 20cm
\hoffset = -1cm
\usepackage[utf8]{inputenc}
\usepackage[spanish,es-tabla]{babel}
\usepackage[autostyle,spanish=mexican]{csquotes}
\usepackage[tbtags]{amsmath}
\usepackage{nccmath}
\usepackage{amsthm}
\usepackage{amssymb}
\usepackage{mathrsfs}
\usepackage{graphicx}
\usepackage{subfig}
\usepackage{standalone}
\usepackage[outdir=./Imagenes/]{epstopdf}
\usepackage{siunitx}
\usepackage{physics}
\usepackage{color}
\usepackage{float}
\usepackage{hyperref}
\usepackage{multicol}
%\usepackage{milista}
\usepackage{anyfontsize}
\usepackage{anysize}
%\usepackage{enumerate}
\usepackage[shortlabels]{enumitem}
\usepackage{capt-of}
\usepackage{bm}
\usepackage{relsize}
\usepackage{placeins}
\usepackage{empheq}
\usepackage{cancel}
\usepackage{wrapfig}
\usepackage[flushleft]{threeparttable}
\usepackage{makecell}
\usepackage{fancyhdr}
\usepackage{tikz}
\usepackage{bigints}
\usepackage{scalerel}
\usepackage{pgfplots}
\usepackage{pdflscape}
\pgfplotsset{compat=1.16}
\spanishdecimal{.}
\renewcommand{\baselinestretch}{1.5} 
\renewcommand\labelenumii{\theenumi.{\arabic{enumii}})}
\newcommand{\ptilde}[1]{\ensuremath{{#1}^{\prime}}}
\newcommand{\stilde}[1]{\ensuremath{{#1}^{\prime \prime}}}
\newcommand{\ttilde}[1]{\ensuremath{{#1}^{\prime \prime \prime}}}
\newcommand{\ntilde}[2]{\ensuremath{{#1}^{(#2)}}}

\newtheorem{defi}{{\it Definición}}[section]
\newtheorem{teo}{{\it Teorema}}[section]
\newtheorem{ejemplo}{{\it Ejemplo}}[section]
\newtheorem{propiedad}{{\it Propiedad}}[section]
\newtheorem{lema}{{\it Lema}}[section]
\newtheorem{cor}{Corolario}
\newtheorem{ejer}{Ejercicio}[section]

\newlist{milista}{enumerate}{2}
\setlist[milista,1]{label=\arabic*)}
\setlist[milista,2]{label=\arabic{milistai}.\arabic*)}
\newlength{\depthofsumsign}
\setlength{\depthofsumsign}{\depthof{$\sum$}}
\newcommand{\nsum}[1][1.4]{% only for \displaystyle
    \mathop{%
        \raisebox
            {-#1\depthofsumsign+1\depthofsumsign}
            {\scalebox
                {#1}
                {$\displaystyle\sum$}%
            }
    }
}
\def\scaleint#1{\vcenter{\hbox{\scaleto[3ex]{\displaystyle\int}{#1}}}}
\def\bs{\mkern-12mu}


\title{El átomo de hidrógeno \\ \large {Tema 4 - Separación de variables en coord. esféricas}\vspace{-3ex}}

\author{M. en C. Gustavo Contreras Mayén}
\date{ }

\pagestyle{fancy}
\fancyhf{}
\rhead{Curso MAF}
\lhead{\leftmark}
\rfoot{\thepage}
\setlength{\headheight}{16pt}%


\begin{document}
\maketitle
\fontsize{14}{14}\selectfont
\tableofcontents
\newpage


\section{Una partícula en un potencial central.}
\subsection{Ecuación de Schrödinger.}

%Ref. Schaum's (1998) Quantum mechanics.
El Hamiltoniano de una partícula de masa $M$ dentro de un potencial central $V (r)$ es:
\begin{align}
H = \dfrac{\vb{p}^{2}}{2 M} + V (r) = - \dfrac{\hbar^{2}}{2 M} \, \laplacian + V (r)
\label{eq:ecuacion_08_01}
\end{align}
donde el Laplaciano $\laplacian$ en coordenadas esféricas es:
\begin{align}
\laplacian = \dfrac{1}{r} \pdv[2]{r} + \dfrac{1}{r^{2}} \bigg[ \pdv[2]{\theta} + \dfrac{1}{\tan \theta} \, \pdv{\theta} + \dfrac{1}{\sin^{2} \theta} \, \pdv[2]{\phi} \bigg]
\label{eq:ecuacion_08_02}
\end{align}

Considerando que el momento angular\footnote{Será conveniente que revises el material complementario sobre el momento angular} se puede expresar como:
\begin{align}
\vb{L}^{2} = - \hbar^{2} \, \bigg[ \pdv[2]{\theta} + \dfrac{1}{\tan \theta} \, \pdv{\theta} + \dfrac{1}{\sin^{2} \theta} \, \pdv[2]{\phi}  \bigg]
\label{eq:ecuacion_08_01_06}
\end{align}
el Hamiltoniano $H$ lo escribimos:
\begin{align}
H = - \dfrac{\hbar^{2}}{2 M} \, \dfrac{1}{r} \, \pdv[2]{r} + \dfrac{1}{2 M r^{2}} \, \vb{L}^{2} + V (r)
\label{eq:ecuacion_08_03}
\end{align}


%Las tres componentes del operador $\vb{L}$ conmutan con $\vb{L}^{2}$, y



% El potencial en el átomo de hidrógeno es el potencial de interacción de tipo Coulomb entre el núcleo y el electrón.
% \par
% Este es un potencial radial, es decir, depende solamente de la distancia al núcleo $(r)$:
% \begin{align}
% V = V(r) = - \dfrac{k \, Z \, e^{2}}{r}
% \label{eq:ecuacion_01}
% \end{align}
% donde $Z$ el número atómico (en este caso $Z=1$), $e$ es la carga del electrón y $k$ es la constante de Coulomb.
% \par
% Por lo tanto el Hamiltoniano cuántico (el operador correspondiente a la energía total de sistema) se escribe como:
% \begin{align}
% H = - \dfrac{\hbar^{2}}{2 \, m} \, \laplacian + V(r)
% \label{eq:ecuacion_02} 
% \end{align}

% El sistema de coordenadas esféricas es el más adecuado para el problema: la ecuación de Schrödinger será más fácil de resolver en este sistema.
% \par
% Como ya sabemos expresar el Laplaciano en este sistema, haremos uso de esa expresión:
% \begin{align*}
% \laplacian = \dfrac{1}{r^{2}} \pdv{r} \left( r^{2} \pdv{\phi}{r} \right) {+} \dfrac{1}{r^{2} \sin \theta} \pdv{\theta} \left( \sin \theta \pdv{\phi}{\theta} \right) {+} \dfrac{1}{r^{2} \sin^{2} \theta} \pdv[2]{\phi}{\phi} 
% \end{align*}
% La expresión para el Laplaciano es complicada así que buscaremos una expresión más adecuada para resolver la ecuación de Schrödinger más fácilmente.

% \subsection{Momento angular.}

% La teoría del momento angular en mecánica cuántica es de gran importancia tanto por el número como por la variedad de sus consecuencias.
% \par
% A partir de la espectroscopía rotacional, que depende del momento angular de las moléculas, se consigue información acerca de las dimensiones y formas de moléculas.
% \par
% Utilizando los espectros de resonancia magnética nuclear y de resonancia paramagnética electrónica, cuyo origen es el momento angular de espín de núcleos y electrones, se consigue información sobre la estructura y configuración de moléculas.
% \par
% El momento angular orbital de los electrones en los átomos define las forma de los orbitales atómicos los cuales, a su vez, determinan la orientación de los enlaces y la estereoquímica de las moléculas. El momento angular de un sistema es muy importante, cuando \emph{es una constante de movimiento}, es decir, cuando se conserva, porque en este caso sirve para clasificar los niveles de energía del sistema.
% \par
% En mecánica cuántica los operadores de momento angular orbital son:
% \begin{align}
% \begin{aligned}
% \hat{L}_{x} &=& - i \, \hbar \, \left( y \, \pdv{z} - z \, \pdv{y} \right) \\[0.5em] 
% \hat{L}_{y} &=& - i \, \hbar \, \left( z \, \pdv{x} - x \, \pdv{z} \right) \\[0.5em] 
% \hat{L}_{z} &=& - i \, \hbar \, \left( x \, \pdv{y} - y \, \pdv{x} \right)
% \end{aligned}
% \label{eq:ecuacion_01_03a}
% \end{align}

% El cuadrado del operador momento angular es tal que:
% \begin{align}
% \hat{L}^{2} = \hat{L} \cdot \hat{L} = \hat{L}_{x}^{2} + \hat{L}_{y}^{2} + \hat{L}_{z}^{2}
% \label{eq:ecuacion_01_03b}
% \end{align}

% Para aplicar estos operadores sobre funciones del tipo $\psi(r, \theta, \phi)$ es necesario expresarlos en coordenadas polares. Utilizando las relaciones:
% \begin{align*}
% r^{2} &= x^{2} + y^{2} +z^{2} \\
% \cos \theta &= \dfrac{z}{\sqrt{x^{2} + y^{2} +z^{2}}} \\
% \tan \phi &= \dfrac{y}{x}
% \end{align*}
 
% Para luego aplicar las derivadas parciales $\pdv*{x}$, $\pdv*{y}$ y $\pdv*{z}$, se tiene:
% \begin{align}
% \begin{aligned}
% \hat{L}_{x} &= + i\, \hbar \, \left( \sin \phi \,\pdv{\theta} + \cot \theta\, \cos \phi \, \pdv{\phi} \right) \\[0.5em] 
% \hat{L}_{y} &= - i\, \hbar \, \left( \cos \phi \,\pdv{\theta} - \cot \theta\, \sin \phi \, \pdv{\phi} \right) \\[0.5em] 
% \hat{L}_{z} &= - i\, \hbar \, \pdv{\phi}
% \end{aligned}
% \label{eq:ecuacion_01_04a}
% \end{align}

% El cuadrado del operador momento angular es:
% \begin{align}
% \hat{L}^{2} = - \hbar^{2} \left( \dfrac{1}{\sin \theta} \pdv{\theta} \, \sin \theta \, \pdv{\theta} + \dfrac{1}{\sin^{2} \theta} \, \pdv[2]{\phi} \right)
% \label{eq:ecuacion_01_04b}
% \end{align}

% Es importante notar que solo se utiliza el operador $\hat{L}^{2}$ o sus componentes, pero nunca el operador $\hat{L}$ directamente, ya que el momento angular es un vector $\va{L}$ y no un escalar.

% \subsection{Constante de movimiento.}

% La condición para que el operador $\hat{O}$ represente una \emph{constante de movimiento} de un sistema es que se cumpla la relación:
% \begin{align}
% \hat{O} \, \hat{H} = \hat{H} \, \hat{O}
% \label{eq:ecuacion_01_05}
% \end{align}
% donde $\hat{H}$ es el Hamiltoniano del sistema.
% \par
% La relación anterior implica que el conmutador:
% \begin{align}
% [\hat{O}, \hat{H}] = \hat{O} \hat{H} - \hat{H} \, \hat{O}
% \label{eq:ecuacion_01_06}
% \end{align}
% vale cero.
% \par
% En efecto, cuando dos operadores conmutan, existe un conjunto de funciones que son funciones propias de los dos operadores simultáneamente. Es decir, que la misma función $\psi$ que caracteriza el estado del sistema con energía $E$:
% \begin{align*}
% \hat{H} \, \psi = E \, \psi
% \end{align*}
% también caracteriza el estado del sistema con propiedad $\hat{O}$ igual a $0$:
% \begin{align*}
% \hat{O} \, \psi = 0 \, \psi
% \end{align*}

% Dicho de otra manera, cuando el sistema se encuentra en el estado caracterizado por $\psi$, su energía es $E$ y su propiedad $\hat{O}$ es $o$. Ambos valores $E$ y $o$ son constantes mientras el sistema permanezca en el mismo estado $\psi$.
% \par
% En los casos en los que $\psi$ sea degenerada, siempre será posible construir una combinación lineal de las funciones propias correspondientes a $E$ tal que sea también función propia de $\hat{O}$.

\subsection{Reglas de conmutación.}

Las reglas de conmutación entre los operadores de momento angular y sus componentes pueden ser deducidas fácilmente utilizando las expresiones en coordenadas cartesianas y algunas identidades de los conmutadores como:
\begin{align*}
[ \hat{A} + \hat{B}, \hat{C}] &= [\hat{A}, \hat{C}] + [\hat{B} + \hat{C}] \\[0.5em]
[ \hat{A}^{2} , \hat{B}] &= [\hat{A}, \hat{B}] \, \hat{A} +  \hat{A} \, [\hat{A} , \hat{B}]
\end{align*}

Se cumple entonces que:
\begin{align}
\begin{aligned}
[ \hat{L}_{x}, \hat{L}_{y} ] &= i \, \hbar \, \hat{L}_{z} \\[0.5em]
[ \hat{L}_{y}, \hat{L}_{z} ] &= i \, \hbar \, \hat{L}_{x} \\[0.5em]
[ \hat{L}_{z}, \hat{L}_{x} ] &= i \, \hbar \, \hat{L}_{y} \\[0.5em]
[\hat{L}^{2}, \hat{L}_{x}] = [\hat{L}^{2}&, \hat{L}_{y}] = [\hat{L}^{2}, \hat{L}_{z}] = 0
\end{aligned}
\label{eq:ecuacion_01_07}
\end{align}

Entonces: $\hat{L}^{2}$ conmuta con cualquiera de sus componentes, pero las componentes no conmutan entre sí. Las propiedades de conmutación entre los operadores de momento angular orbital y el Hamiltoniano dependen del sistema y deben ser determinadas para cada problema.
\par
% Frecuentemente $\hat{L}^{2}$ y $\hat{L}_{z}$ conmutan con $\hat{H}$ y en estos casos el módulo del momento angular y la componente sobre el eje $z$ del momento angular son constantes de movimiento.
% \par
% Frecuentemente $\hat{L}^{2}$ y $\hat{L}_{z}$ conmutan con $\hat{H}$ y en estos casos el módulo del momento angular y la componente sobre el eje $z$ del momento angular son constantes de movimiento.
% \par
% Por ejemplo, en el caso de átomos hidrogenoides $\hat{H}$ y $\hat{L}_{z}$ conmutan, donde:
% \begin{align*}
% \hat{H} &= - \dfrac{\hbar^{2}}{2 \mu} \left[ \dfrac{1}{r^{2}} \pdv{r} \left( r^{2} \pdv{\phi}{r} \right) {+} \dfrac{1}{r^{2} \sin \theta} \pdv{\theta} \left( \sin \theta \pdv{\phi}{\theta} \right) {+} \right. \\[0.5em]
% &+ \left. \dfrac{1}{r^{2} \sin^{2} \theta} \pdv[2]{\phi}{\phi} \right] - \dfrac{Z \, e^{2}}{r} \\[1em]
% \hat{L}_{z} &= - i \, \hbar \, \pdv{\phi}
% \end{align*}

% Entonces:
% \begin{align*}
% \hat{H} \cdot \hat{L}_{z} &= + \dfrac{i \, \hbar^{3}}{2 \mu} \left\{ \left[ \dfrac{1}{r^{2}} \pdv{r} \left( r^{2} \pdv{\phi}{r} \right) {+} \dfrac{1}{r^{2} \sin \theta} \pdv{\theta} \left( \sin \theta \pdv{\phi}{\theta} \right) {+} \right. \right. \\[0.5em]
% &+ \left. \left. \dfrac{1}{r^{2} \sin^{2} \theta} \pdv[2]{\phi}{\phi} + \dfrac{Z \, e^{2}}{r} \right] \pdv{\phi} + \dfrac{1}{r^{2} \sin^{2} \theta} \pdv[3]{\phi}{\phi} \right\}
% \end{align*}

% Mientras que:
% \begin{align*}
% \hat{L}_{z} \cdot \hat{H}  &= + \dfrac{i \, \hbar^{3}}{2 \mu} \left\{ \pdv{\phi} \left[ \dfrac{1}{r^{2}} \pdv{r} \left( r^{2} \pdv{\phi}{r} \right) {+} \right. \right. \\[0.5em]
% &+ \dfrac{1}{r^{2} \sin \theta} \pdv{\theta} \left( \sin \theta \pdv{\phi}{\theta} \right) {+} \\[0.5em]
% &+ \left. \left. \dfrac{1}{r^{2} \sin^{2} \theta} \pdv[2]{\phi}{\phi} + \dfrac{Z \, e^{2}}{r} \right] + \dfrac{1}{r^{2} \sin^{2} \theta} \pdv[3]{\phi}{\phi} \right\}
% \end{align*}

% Sabiendo que:
% \begin{align*}
% \pdv{\phi} \, \pdv{r} = \pdv{r} \, \pdv{\phi} \\[0.5em]
% \pdv{\phi} \, \pdv{\theta} = \pdv{\theta} \, \pdv{\phi}
% \end{align*}

% Es decir, las dos expresiones son iguales, por lo que:
% \begin{align*}
% \big[ \hat{H}, \hat{L}_{z} \big] = \hat{H} \, \hat{L}_{z} - \hat{L}_{z} \, \hat{H} = 0
% \end{align*}

% En coordenadas cartesianas $\hat{L}^{2}$ depende de tres coordenadas $(x, y, z)$; en coordenadas esféricas, $\hat{L}^{2}$ depende solo de dos $(\theta, \phi)$.
% \par
% En coordenadas cartesianas una de las variables no es independiente; en coordenadas esféricas, $\hat{L}^{2}$ solo depende de los ángulos, y no de la distancia $r$.
% \par
% Los observables correspondientes a los operadores $\hat{L}_{x}$, $\hat{L}_{y}$ y $\hat{L}_{z}$, son totalmente equivalentes, lo único que cambia es su orientación con respecto al sistema de referencia. Por esta razón siempre se usa $\hat{L}_{z}$, ya que la expresión matemática de su operador es mucho más simple, depende de solo un ángulos.
% \par
% Nos apoyaremos en un resultado de la teoría de los operadores y conmutadores: : Si $\hat{A}$ y $\hat{B}$ conmutan, es decir, si  $[\hat{A}, \hat{B}] = 0$, entonces existe una solución común $\psi$  para el par de ecuaciones diferenciales correspondientes a las ecuaciones de valores propios de estos operadores, siendo $\psi$ la función propia mientras que $a$ y $b$ son los valores propios correspondientes:
% \begin{align*}
% \hat{A} \, \psi &= a \, \psi \\[0.5em]
% \hat{B} \, \psi &= b \, \psi
% \end{align*}

% Ahora bien, utilizando ese resultado y el hecho de que $[\hat{L}^{2}, \hat{L}_{z}] = 0$ podemos buscar una solución común, que escribimos como $Y(\theta, \phi)$, al par de las ecuaciones diferenciales:
% \begin{align*}
% \hat{L}_{z} \, Y(\theta, \phi) &= b \, Y(\theta, \phi) \\[0.5em]
% \hat{L}^{2} \, Y(\theta, \phi) &= c \, Y(\theta, \phi)
% \end{align*}

%Ref. Schaum's
Tendremos la siguiente ecuación de eigenvalores:
\begin{align}
\bigg[ - \dfrac{\hbar^{2}}{2 \mu} \, \dfrac{1}{r} \, \pdv[2]{r} (r) + \dfrac{\vb{L}^{2}}{2 \mu r^{2}} + V (r) \bigg] \, \psi (r, \theta, \phi) = E \, \psi (r, \theta, \phi) 
\label{eq:ecuacion_08_01_02}
\end{align}

Ya se revisó que las tres componentes de $\vb{L}$ conmutan con $\vb{L}^{2}$, por lo que también conmutan con $H$:
\begin{align}
\comm{H}{L_{x}} = \comm{H}{L_{y}} = \comm{H}{L_{z}} = 0
\label{eq:ecuacion_08_04}
\end{align}

Los tres observables $H$, $\vb{L}^{2}$ y $L_{z}$ conmutan, por lo que podemos buscar funciones $\psi (r, \theta, \phi)$, que también sean eigenfunciones de $\vb{L}^{2}$ y $L_{z}$.
\par
La componente $L_{z}$ del momento angular es:
\begin{align*}
L_{z} = - i \, \hbar , \pdv{\phi}
\end{align*}

Ahora podremos resolver el siguiente sistema de ecuaciones diferenciales de eigenvalores:
\begin{align}
H \, \psi (r, \theta, \phi) &= E \, \psi (r, \theta, \phi) \label{eq:ecuacion_08_05} \\[0.5em]
\vb{L}^{2} \, \psi (r, \theta, \phi) &= \ell (\ell + 1) \, \hbar^{2} \, \psi (r, \theta, \phi) \label{eq:ecuacion_08_06} \\[0.5em]
L_{z} \, \psi (r, \theta, \phi) &= m \, \hbar \, \psi (r, \theta, \phi) \label{eq:ecuacion_08_07}
\end{align}
y determinar aquellos estados que son eigenfunciones de $H$, $\vb{L}^{2}$ y $L_{z}$.
\par
Con la técnica de separación de variables, tenemos que: una solución estaría dada por el producto de:
\begin{align}
\psi (r, \theta, \phi) = R (r) \, Y (\theta, \phi)
\label{eq:ecuacion_08_08}
\end{align}




%Ref. Ghatak (2004) 9.3
\subsection{Problema de valores propios.}

Sin pérdida de generalidad, podemos expresar nuestro problema de valores propios para $\hat{L}^{2}$ como:
\begin{align}
\hat{L}^{2} \, Y(\theta, \phi) = \lambda \, \hbar^{2} \, Y(\theta, \phi)
\label{eq:ecuacion_027}
\end{align}
donde $\lambda \, \hbar^{2}$ representan los valores propios de $\hat{L}^{2}$, y $Y(\theta, \phi)$ corresponde a las funciones propias. 
\par
Veremos que $\lambda$ toma valores $\ell (\ell + 1)$ con $\ell = 0, 1, 2, \ldots$ y las correspondientes funciones propias son los \emph{armónicos esféricos}.
\par
Para cada valor de $\ell$, habrá un orden $(2 \, \ell + 1)$ de degeneración, es decir, habrá $(2 \, \ell + 1)$ funciones propias que corresponden al mismo valor propio $\ell (\ell + 1) \, \hbar^{2}$. El operador $\hat{L}^{2}$ lo sustituimos en la ec. (\ref{eq:ecuacion_027}), así que:
\begin{align}
\dfrac{1}{\sin \theta} \pdv{\theta} \, \sin \theta \, \pdv{Y}{\theta} + \dfrac{1}{\sin^{2} \theta} \, \pdv[2]{Y}{\phi} + \lambda \, Y(\theta, \phi) = 0
\end{align}
Para resolver esta ecuación, usamos la técnica de separación de variables. Proponemos una solución de la forma:
\begin{align}
Y(\theta, \phi) = \Theta(\theta) \, \Phi(\phi)
\label{eq:ecuacion_029}
\end{align}
Que susituimos en la expresión anterior, para luego multiplicar por:
\begin{align*}
\dfrac{\sin^{2} \theta}{Y(\theta, \phi)}
\end{align*}

Entonces obtendremos:
\begin{align}
\dfrac{\sin^{2} \theta}{\Theta} \left[ \dfrac{1}{\sin \theta} \pdv{\theta} \, \sin \theta \, \pdv{\Theta}{\theta} + \lambda \, \Theta (\theta) \right] = -  \dfrac{1}{\Phi} \, \dv[2]{\Phi}{\phi} = m^{2}
\label{eq:ecuacion_030}
\end{align}

De hecho, las variables se han separado y hemos establecido cada lado igual a una constante positiva $m^{2}$, cuya razón quedará clara en breve. La ec. (\ref{eq:ecuacion_030}) nos da:
\begin{align*}
\dv[2]{\Phi}{\phi} + m^{2} \Phi (\phi) = 0
\end{align*}
cuya solución está dada por:
\begin{align*}
\Phi(\phi) \sim e^{i m \phi}
\end{align*}

Para que la función de onda sea univaluada, debe de ocurrir que:
\begin{align}
\Phi(\phi +  2 \, \pi) = \Phi(\phi)
\label{eq:ecuacion_031}
\end{align}
o equivalentemente:
\begin{align*}
e^{2 \pi m i} = 1
\end{align*}
Obteniendo entonces que:
\begin{align*}
m = 0, \pm 1, \pm 2, \ldots
\end{align*}

En este paso se justifica que no podríamos haber establecido una constante positiva (o compleja) porque entonces la función de onda no habría sido de un solo valor.
\par
Al identificar las funciones con un subíndice $m$, tenemos:
\begin{align}
\Phi_{m}(\phi) = \dfrac{1}{\sqrt{2 \, \pi}} \, e^{i m \phi} \hspace{1cm} m = \pm 1, \pm 2, \ldots
\label{eq:ecuacion_032}
\end{align}

Donde el factor $\dfrac{1}{\sqrt{2 \, \pi}}$ asegura que:
\begin{align*}
\int_{0}^{2 \pi} \abs{\Phi_{m}(\phi)}^{2} \dd{\phi} = 1
\end{align*}
que es la condición de normalización. Entonces se tendrá que:
\begin{align}
\int_{0}^{2 \pi} \Phi_{\ptilde{m}}^{*}(\phi) \, \Phi_{m}(\phi) \dd{\phi} = \delta_{m \ptilde{m}}
\label{eq:ecuacion_033}
\end{align}
representa la condición de ortonormalización para $\Phi_{m}(\phi)$.
\par
Para la segunda ecuación $\Theta (\theta)$ (ec. \ref{eq:ecuacion_030}), tendremos que:
\begin{align}
\dfrac{1}{\sin \theta} \dv{\theta} \left( \sin \theta \, \dv{\Theta}{\theta} \right) + \left( \lambda - \dfrac{m^{2}}{\sin^{2} \theta} \right) \, \Theta (\theta) = 0
\label{eq:ecuacion_034}
\end{align}

Hacemos el siguiente cambio de variable: $\cos \theta = \mu$ y $\Theta(\theta) = F(\mu)$, para obtener:

\begin{align}
\dv{\mu} \left[ (1 - \mu^{2}) \, \dv{F}{\mu} \right] + \left[ \lambda - \dfrac{m^{2}}{1 - \mu^{2}} \right] \, F(\mu) = 0
\label{eq:ecuacion_035}
\end{align}
Hay que considerar dos casos: $m = 0$ y $m \neq 0$. 
\par
Con $m = 0$, la ec. (\ref{eq:ecuacion_035}) se reduce a:
\begin{align}
(1 - \mu^{2}) \, \dv[2]{F}{\mu} - 2  \, \mu \, \dv{F}{\mu} + \lambda \, F(\mu) = 0
\label{eq:ecuacion_036}
\end{align}

El método de \enquote{fuerza bruta} para obtener es resolver directamente la ec. (\ref{eq:ecuacion_035}); esto lo veremos posteriormente. Sin embargo, la forma más sencilla y elegante de obtener las soluciones es mediante el uso de operadores de escalera del momento angular, que también revisaremos esa solución.
\par
Hemos llegado a plantear una ecuación diferencial para la parte angular, nos falta considerar la parte radial. Para la solución completa de cualquier problema de una partícula con un potencial radial, se tiene que la función de onda es un producto de un factor radial y un armónico esférico.
\end{document}