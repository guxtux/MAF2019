\documentclass[hidelinks,12pt]{article}
\usepackage[left=0.25cm,top=1cm,right=0.25cm,bottom=1cm]{geometry}
%\usepackage[landscape]{geometry}
\textwidth = 20cm
\hoffset = -1cm
\usepackage[utf8]{inputenc}
\usepackage[spanish,es-tabla]{babel}
\usepackage[autostyle,spanish=mexican]{csquotes}
\usepackage[tbtags]{amsmath}
\usepackage{nccmath}
\usepackage{amsthm}
\usepackage{amssymb}
\usepackage{mathrsfs}
\usepackage{graphicx}
\usepackage{subfig}
\usepackage{standalone}
\usepackage[outdir=./Imagenes/]{epstopdf}
\usepackage{siunitx}
\usepackage{physics}
\usepackage{color}
\usepackage{float}
\usepackage{hyperref}
\usepackage{multicol}
%\usepackage{milista}
\usepackage{anyfontsize}
\usepackage{anysize}
%\usepackage{enumerate}
\usepackage[shortlabels]{enumitem}
\usepackage{capt-of}
\usepackage{bm}
\usepackage{relsize}
\usepackage{placeins}
\usepackage{empheq}
\usepackage{cancel}
\usepackage{wrapfig}
\usepackage[flushleft]{threeparttable}
\usepackage{makecell}
\usepackage{fancyhdr}
\usepackage{tikz}
\usepackage{bigints}
\usepackage{scalerel}
\usepackage{pgfplots}
\usepackage{pdflscape}
\pgfplotsset{compat=1.16}
\spanishdecimal{.}
\renewcommand{\baselinestretch}{1.5} 
\renewcommand\labelenumii{\theenumi.{\arabic{enumii}})}
\newcommand{\ptilde}[1]{\ensuremath{{#1}^{\prime}}}
\newcommand{\stilde}[1]{\ensuremath{{#1}^{\prime \prime}}}
\newcommand{\ttilde}[1]{\ensuremath{{#1}^{\prime \prime \prime}}}
\newcommand{\ntilde}[2]{\ensuremath{{#1}^{(#2)}}}

\newtheorem{defi}{{\it Definición}}[section]
\newtheorem{teo}{{\it Teorema}}[section]
\newtheorem{ejemplo}{{\it Ejemplo}}[section]
\newtheorem{propiedad}{{\it Propiedad}}[section]
\newtheorem{lema}{{\it Lema}}[section]
\newtheorem{cor}{Corolario}
\newtheorem{ejer}{Ejercicio}[section]

\newlist{milista}{enumerate}{2}
\setlist[milista,1]{label=\arabic*)}
\setlist[milista,2]{label=\arabic{milistai}.\arabic*)}
\newlength{\depthofsumsign}
\setlength{\depthofsumsign}{\depthof{$\sum$}}
\newcommand{\nsum}[1][1.4]{% only for \displaystyle
    \mathop{%
        \raisebox
            {-#1\depthofsumsign+1\depthofsumsign}
            {\scalebox
                {#1}
                {$\displaystyle\sum$}%
            }
    }
}
\def\scaleint#1{\vcenter{\hbox{\scaleto[3ex]{\displaystyle\int}{#1}}}}
\def\bs{\mkern-12mu}



\title{El grupo de rotaciones y los armónicos esféricos \\ {\large Tema 4}\vspace{-3ex}}
\author{M. en C. Gustavo Contreras Mayén}
\date{ }

\pagestyle{fancy}
\fancyhf{}
\rhead{Curso MAF}
\lhead{\leftmark}
\rfoot{\thepage}
\setlength{\headheight}{16pt}%

\def\changemargin#1#2{\list{}{\rightmargin#2\leftmargin#1}\item[]}
\let\endchangemargin=\endlist 


\begin{document}
\maketitle
\fontsize{14}{14}\selectfont
\tableofcontents
\newpage

%Ref. Romero (2013) Funciones especiales. Cap. 10

En este tema estudiaremos los polinomios asociados de Legendre, los polinomios ordinarios de Legendre, y los armónicos esféricos, quienes son importantes para resolver la ecuación de Laplace y diversos problemas de electrodinámica y mecánica cuántica. Normalmente estas funciones se obtienen resolviendo ecuaciones diferenciales. Sin embargo, también es posible obtenerlas usando el grupo de rotaciones. Usaremos este último método debido a que nos introduce a las aplicación de la teoría de grupos en la física. Este tema se puede ver como una invitación al estudio de las aplicaciones de la teoría de grupos.

\section{Transformación de coordenadas lineales.}

Sea $f$ una función de la variable $x$, con el cambio de variable:
\begin{align}
\pderivada{x} = \alpha \, x \hspace{1.5cm} \mbox{con } \alpha = \mbox{ constante}
\label{eq:ecuacion_10_001}
\end{align}
al diferenciar con respecto a $x$ tenemos que:
\begin{align}
\dv{f}{\pderivada{x}} = \dv{x}{\pderivada{x}} \, \dv{f}{x} = \dfrac{1}{\alpha} \, \dv{f}{x}
\label{eq:ecuacion_10_002}   
\end{align}
Entonces tenemos que si la variable $x$ transforma con $\alpha$, el operador derivada transforma con $\dfrac{1}{\alpha}$, es decir:
\begin{align}
\begin{aligned}
x \rightarrow \pderivada{x} &= \alpha \, x \\[0.5em]
\dv{x} \rightarrow \dv{\pderivada{x}} &= \dfrac{1}{\alpha} \, \dv{f}{x}
\end{aligned}
\label{eq:ecuacion_10_003}
\end{align}

Veamos ahora que pasa en dos dimensiones. Consideremos la matriz de $2 \times 2$ con entradas constantes:
\begin{align}
\Lambda = \mqty(a_{1} & a_{2} \\ a_{3} & a_{4})
\label{eq:ecuacion_10_004}   
\end{align}
cuya matriz inversa es:
\begin{align}
\Lambda^{-1} = \dfrac{1}{\abs{\Lambda}} \, \mqty(a_{4} & - a_{2} \\ - a_{3} & a_{1}) \hspace{1.5cm} \abs{\Lambda} = a_{1} \, a_{4} - a_{2} \, a_{3}
\label{eq:ecuacion_10_005}   
\end{align}
También definamos los vectores columna:
\begin{align}
X = \mqty(x^{1} \\ x^{2}) \hspace{1.5cm} \grad = \mqty(\displaystyle \pdv{x^{1}} \\[0.5em] \displaystyle \pdv{x^{2}})
\label{eq:ecuacion_10_006}   
\end{align}
Entonces podemos hacer una transformación lineal de coordenadas de la forma: $\pderivada{X} = \Lambda \, X$, es decir:
\begin{align}
\mqty(x^{\prime \, 1} \\ x^{\prime \, 2}) =
\mqty(a_{1} & a_{2} \\ a_{3} & a_{4}) \, \mqty(x^{1} \\ x^{2})
\label{eq:ecuacion_10_007}   
\end{align}
La cual tiene la transformación inversa: $X = \Lambda^{-1} \, \pderivada{X}$
\begin{align}
\mqty(x^{1} \\ x^{2}) = \dfrac{1}{\abs{\Lambda}} \, 
\mqty(a_{4} & - a_{2} \\ - a_{3} & a_{1}) \, \mqty(x^{\prime \, 1} \\ x^{\prime \, 2})
\label{eq:ecuacion_10_008}   
\end{align}
es decir:
\begin{align}
x^{1} &= \dfrac{a_{4} \, x^{\prime \, 1} - a_{2} \, x^{\prime \, 2}}{\abs{\Lambda}} \label{eq:ecuacion_10_009} \\[0.5em]
x^{2} &= \dfrac{-a_{3} \, x^{\prime \, 1} + a_{1} \, x^{\prime \, 2}}{\abs{\Lambda}} \label{eq:ecuacion_10_010}
\end{align}
Además, al diferenciar por la regla de la cadena se tiene que:
\begin{align}
\begin{aligned}
\pdv{x^{\prime \, 1}} &= \pdv{x^{1}}{x^{\prime \, 1}} \, \pdv{x^{1}} + \pdv{x^{2}}{x^{\prime \, 1}} \, \pdv{x^{2}} \\[0.5em]
\pdv{x^{\prime \, 2}} &= \pdv{x^{1}}{x^{\prime \, 2}} \, \pdv{x^{1}} + \pdv{x^{2}}{x^{\prime \, 2}} \, \pdv{x^{2}}
\end{aligned}
\label{eq:ecuacion_10_011}
\end{align}
de la regla de transformación ec. (\ref{eq:ecuacion_10_010}) se encuentra que:
\begin{align}
\begin{aligned}
\pdv{x^{\prime \ 1}} &= \dfrac{1}{\abs{\Lambda}} \, \left( a_{4} \, \pdv{x^{1}} - a_{3} \, \pdv{x^{2}} \right) \\[1em]
\pdv{x^{\prime \ 2}} &= \dfrac{1}{\abs{\Lambda}} \, \left( - a_{2} \, \pdv{x^{1}} + a_{1} \, \pdv{x^{2}} \right)
\end{aligned}
\label{eq:ecuacion_10_012}
\end{align}
que se puede expresar como:
\begin{align}
\mqty(\displaystyle \pdv{x^{\prime \, 1}} \\[0.75em] \displaystyle \pdv{x^{\prime \, 2}}) = \dfrac{1}{\abs{\Lambda}} \, \mqty(a_{4} & - a_{3} \\ - a_{2} & a_{1}) \, \mqty(\displaystyle \pdv{x^{1}} \\[0.75em] \displaystyle \pdv{x^{2}})
\label{eq:ecuacion_10_013}   
\end{align}
De esta ecuación podemos ver que la matriz involucrada en la transformación de las derivadas parciales es la transpuesta de la matriz inversa de $\Lambda$, es decir $(\Lambda^{-1})^{T}$. Otra forma de escribir esta ecuación es:
\begin{align}
\grad^{\prime} = (\Lambda^{-1})^{T} \, \grad
\label{eq:ecuacion_10_014}
\end{align}
por lo que:
\begin{align}
\grad = (\Lambda)^{T} \, \grad^{\prime}
\label{eq:ecuacion_10_015}
\end{align}

Este resultado se puede generalizar para más dimensiones. En efecto, en general una transformación de coordenadas se escribe como:
\begin{align}
x^{\prime \, i} = \Lambda_{ij} \, x^{j} \hspace{1.5cm} x^{i} = (\Lambda^{-1})_{ij} \, x^{\prime \, j}
\label{eq:ecuacion_10_016}
\end{align}
Por la regla de la cadena se tiene que:
\begin{align}
\begin{aligned}[b]
\pdv{x^{\prime \, i}} &= \pdv{x^{j}}{x^{\prime \, i}} \, \pdv{x^{j}} = \\[0.5em]
&= \pdv{(\Lambda^{-1})_{jk} \, x^{\prime \, k}}{x^{\prime \, i}} \, \pdv{x^{j}} = \\[0.5em]
&= (\Lambda^{-1})_{jk} \, \pdv{x^{\prime , k}}{x^{\prime \, i}} \, \pdv{x^{j}} =  \\[0.5em]
&= (\Lambda^{-1})_{jk} \, \delta_{ki} \, \pdv{x^{j}} =  \\[0.5em]
&= (\Lambda^{-1})_{ji} \, \pdv{x^{j}} = \\[0.5em]
&= \bigg[ \left( \Lambda^{-1} \right)^{T} \bigg]_{ij} \, \pdv{x^{j}}
\end{aligned}
\label{eq:ecuacion_10_017}
\end{align}
Por lo tanto, para cualquier dimensión se cumple:
\begin{align}
\grad^{\prime} = (\Lambda^{-1})^{T} \, \grad
\label{eq:ecuacion_10_018}
\end{align}
Esta ley de transformación será de gran utilidad para obtener las simetrías de la ecuación de Laplace.

\section{Laplaciano y elemento de línea.}

Supongamos que la matriz $\tilde{\eta}$ de tamaño $n \times n$, satisface:
\begin{align}
\tilde{\eta} \, \tilde{\eta} = I
\label{eq:ecuacion_10_019}
\end{align}
donde la matriz $I$ es la matriz identidad de $n \times n$. Entonces podemos definir un elemento de línea como:
\begin{align}
\dd{s^{2}} = \dd{X^{T}} \, \tilde{\eta} \dd{X}, \hspace{1.5cm} \dd{X} = \mqty(\dd{x^{1}} \\ \dd{x^{2}} \\ \vdots \\ \dd{x^{n}})
\label{eq:ecuacion_10_020}
\end{align}
A la matriz $\tilde{\eta}$ se le llama \emph{métrica}, un ejemplo en dos dimensiones de este tipo de matrices son:
\begin{align}
\mqty(\pmat{0}) \hspace{1.5cm} \mqty(-1 & 0 \\ 0 & 1)
\label{eq:ecuacion_10_021}
\end{align}

Con la matriz $\tilde{\eta}$ el \enquote{Laplaciano} se define como:
\begin{align}
\laplacian = \nabla^{T} \, \tilde{\eta} \, \nabla
\label{eq:ecuacion_10_022}
\end{align}
Notablemente, el elemento de línea está íntimamente relacionado con el Laplaciano, en particular tienen las mismas simetrías. Esta relación es importante y se da también para espacios no euclidianos. Veamos como se da esta relación.
\par
Bajo una transformación lineal de coordenadas se tiene:
\begin{align}
\begin{aligned}[b]
\dd{s^{\prime \, 2}} &= \dd{X^{\prime \, T}} \tilde{\eta} \dd{X^{\prime}} = \\[0.5em]
&= \left(\Lambda \dd{X} \right)^{T} \, \tilde{\eta} \, \Lambda \, \dd{X} = \\[0.5em]
&= \dd{X^{T}} \left( \Lambda^{T} \, \tilde{\eta} \, \Lambda \right) \dd{X}
\end{aligned}
\label{eq:ecuacion_10_023}
\end{align}

Las transformaciones $\Lambda$ que dejan invariante al elemento de línea, deben de cumplir $\dd{s^{2}} = \dd{s^{\prime \, 2}}$. Al igualar la ec. (\ref{eq:ecuacion_10_020}) con la ec. (\ref{eq:ecuacion_10_023}) se llega a la condición: 
\begin{align}
\Lambda^{T} \, \tilde{\eta} \, \Lambda = \tilde{\eta}
\label{eq:ecuacion_10_024}
\end{align}
Además, considerando que bajo una transformación lineal de coordenadas el gradiente transforma como eq. (\ref{eq:ecuacion_10_018}), se obtiene:
\begin{align}
\begin{aligned}[b]
\nabla^{\prime \, 2} &= \nabla^{\prime \, T} \, \tilde{\eta} \, \nabla^{\prime} = \\[0.5em]
&= \bigg[ \left(\Lambda^{-1} \right)^{T} \, \nabla \bigg]^{T} \, \tilde{\eta} \, \left( \Lambda^{-1} \right)^{T} \, \nabla = \\[0.5em]
&= \nabla^{T} \, \left[ \left( \left( \Lambda^{-1} \right)^{T} \right)^{T} \, \tilde{\eta} \, \left( \Lambda^{-1} \right)^{T} \right] \, \nabla = \\[0.5em]
&= \nabla^{T} \, \left( \Lambda^{-1} \, \tilde{\eta} \, \left( \Lambda^{-1} \right)^{T} \right) \, \nabla
\end{aligned}
\label{eq:ecuacion_10_025}
\end{align}
Las transformaciones que dejan invariante al Laplaciano deben de cumplir:
\begin{align*}
\laplacian = \nabla^{\prime \, 2}
\end{align*}
entonces igualando la ec. (\ref{eq:ecuacion_10_022}) con la ec. (\ref{eq:ecuacion_10_025}), se tiene la condición:
\begin{align}
\Lambda^{-1} \, \tilde{\eta} \, \left( \Lambda^{-1} \right)^{T} = \tilde{\eta}
\label{eq:ecuacion_10_026}
\end{align}
Como $(\tilde{\eta})^{-1} = \tilde{\eta}$, la condición dada por la ec. (\ref{eq:ecuacion_10_026}), tiene la forma:
\begin{align}
\begin{aligned}[b]
\tilde{\eta} &= \left(\tilde{\eta} \right)^{-1} \\[0.5em]
&= \bigg[ \Lambda^{-1} \, \tilde{\eta} \, \left( \Lambda^{-1} \right)^{T} \bigg]^{-1} = \\[0.5em]
&= \bigg[ \left( \Lambda^{-1} \right)^{T} \bigg]^{-1} \, \tilde{\eta} \, \left( \Lambda^{-1} \right)^{-1} = \\[0.5em]
&= \Lambda^{T} \, \tilde{\eta} \, \Lambda
\end{aligned}
\label{eq:ecuacion_10_027}
\end{align}
que coincide con la eq. (\ref{eq:ecuacion_10_024}). Por lo tanto, las transformaciones lineales, $\Lambda$, que dejan invariante al elemento de línea eq. (\ref{eq:ecuacion_10_020}) también dejan invariante al Laplaciano eq. (\ref{eq:ecuacion_10_022}), claramente la afirmación inversa también es correcta.

\newpage
\section{Grupo de transformaciones.}

% Antes de continuar recordemos lo que es un grupo. Sea $G$ un conjunto con una operación\footnote{Con la finalidad de mejorar la lectura de la operación, se muestra un símbolo más remarcado $\bullet$ en comparación del símbolo $\cdot$, que representa la misma operación.} $\bullet \, : G \times G \rightarrow G$. El par $(G, \bullet)$ es un grupo si cumple los siguientes axiomas:
% \begin{enumerate}[label=\arabic*)]
% \item Axioma de cerradura:
% \begin{align}
% g_{1} \in G, \, g_{2} \in G \hspace{0.7cm} \Rightarrow \hspace{0.7cm} g_{1} \bullet g_{2} \in G
% \label{eq:ecuacion_10_028}
% \end{align}
% \item Axioma de asociatividad:
% \begin{align}
% g_{1} \in G, g_{2} \in G, g_{3} \in G \hspace{0.5cm} \Rightarrow \hspace{0.5cm} g_{1} \bullet (g_{2} \bullet g_{3}) = (g_{1} \bullet g_{2}) \bullet g_{3}
% \label{eq:ecuacion_10_029}
% \end{align}
% \item Axioma del neutro:
% \begin{align}
% \exists \, e \in G, \hspace{1cm} g_{1} \in G \hspace{1cm} \Rightarrow \hspace{1cm} g_{1} \bullet e = e \bullet g_{1} = g_{1}
% \label{eq:ecuacion_10_030}
% \end{align}
% \item Axioma del inverso:
% \begin{align}
% \forall \, g_{1} \in G, \exists \, g_{1}^{-1} \in G, \hspace{1cm} g_{1} \bullet g_{1}^{-1} = g_{-1} \bullet g_{1} = e
% \label{eq:ecuacion_10_031}
% \end{align}
% \end{enumerate}
Definamos a $T$ como el conjunto de transformaciones $\Lambda$ que dejan invariante al Laplaciano. Estas transformaciones cumplen:
\begin{align}
\Lambda^{T} \, \tilde{\eta} \, \Lambda = \tilde{\eta} \hspace{1.5cm} \tilde{\eta}^{2} = I
\label{eq:ecuacion_10_032}
\end{align}
Ahora probaremos que $T$ es un grupo\footnote{Recordemos que ya hemos definido las cuatro propiedades que se deben de cumplir en un conjunto de elementos, de tal manera que la satisfacerse, forman un grupo.}.
\par
Supongamos que $\Lambda_{1} \in T$ y $\Lambda_{2} \in T$, entonces cumplen:
\begin{align}
\Lambda_{1}^{T} \, \tilde{\eta} \, \Lambda_{1} = \tilde{\eta}, \hspace{1.5cm} \Lambda_{2}^{T} \, \tilde{\eta} \, \Lambda_{2} = \tilde{\eta}
\label{eq:ecuacion_10_033}
\end{align}
de donde:
\begin{align}
(\Lambda_{1} \, \Lambda_{2})^{T} \, \tilde{\eta} \, (\Lambda_{1} \, \Lambda_{2}) = \Lambda_{2}^{T} \, \Lambda_{1}^{T} \, \tilde{\eta} \, \Lambda_{1} \, \Lambda_{2} = \Lambda_{2}^{T} \, \tilde{\eta} \, \Lambda_{2} = \tilde{\eta}
\label{eq:ecuacion_10_034}
\end{align}
esto implica que $\Lambda_{1} \, \Lambda_{2} \in T$, es decir:
\begin{align}
\Lambda_{1} \in T, \Lambda_{2} \in T \hspace{1cm} \Rightarrow \hspace{1cm} \Lambda_{1} \, \Lambda_{2} \in T
\label{eq:ecuacion_10_035}
\end{align}
Por lo tanto, se cumple el axioma de la cerradura.
\par
El producto de matrices es asociativo, en particular el producto de las matrices que satisfacen la eq. (\ref{eq:ecuacion_10_032}). Además, la identidad $I$ satisface la eq. (\ref{eq:ecuacion_10_032}), es decir, $I \in T$. Así se cumplen el axioma de la asociatividad y el del elemento
neutro.
\par
Se tiene que $\tilde{\eta}^{2} = I$, si $\Lambda$ está en $T$ entonces se cumple $\Lambda^{T} \, \tilde{\eta} \, \Lambda \, \tilde{\eta} = I$. De donde:
\begin{align}
\tilde{\eta} \, \Lambda \, \tilde{\eta} = \left( \Lambda^{T} \right)^{-1} = \left( \Lambda^{-1} \right)^{T}
\label{eq:ecuacion_10_036}
\end{align}
por lo tanto:
\begin{align}
\left( \Lambda^{-1} \right)^{T}\, \tilde{\eta} \, \Lambda^{-1} = ( \tilde{\eta} \, \Lambda \, \tilde{\eta} ) \, \tilde{\eta} \, \Lambda^{-1} = \tilde{\eta} \, \Lambda \, \Lambda^{-1} = \tilde{\eta}
\label{eq:ecuacion_10_037}
\end{align}
Así, cuando $\Lambda$ está en $T$, también $\Lambda^{-1}$ está en $T$. Esto nos indica que se cumple el axioma del inverso.
\par
En consecuencia el conjunto de matrices que satisface la eq. (\ref{eq:ecuacion_10_032}) es un grupo. Es decir, el conjunto de transformaciones que dejan invariante al elemento de línea eq. (\ref{eq:ecuacion_10_020}) forma un grupo, que es el mismo grupo que deja invariante al Laplaciano eq. (\ref{eq:ecuacion_10_022}).

\section{El grupo de rotaciones.}

Sea $\va{x} = (x_{1}, \ldots, x_{n})$ un vector en $\mathbb{R}^{n}$ y definamos la forma cuadrática $l^{2} = x_{1}^{2} + x_{2}^{2} + \ldots + x_{n}^{2}$, la cual representa la distancia de $\va{x}$ al origen. Nota que si definimos la matriz columna:
\begin{align}
X = \mqty(x_{1} \\ x_{2} \\ \vdots \\ x_{n})
\label{eq:ecuacion_10_038}
\end{align}
y la matriz renglón:
\begin{align}
X^{T} = \mqty(x_{1} & x_{2} & \ldots & x_{n})
\label{eq:ecuacion_10_039}
\end{align}
la distancia se puede escribir como:
\begin{align}
l^{2} = X^{T} \, X = X^{T} \, I \, X 
\label{eq:ecuacion_10_040}
\end{align}
Ahora, si $\Lambda$ es una matriz de $n \times n$ y se hace la transformación de coordenadas:
\begin{align}
X^{\prime} = \Lambda \, X
\label{eq:ecuacion_10_041}
\end{align}
se tiene la distancia:
\begin{align}
l^{2} = X^{\prime \, T} \, I , X^{\prime} = X^{T} \, \left( \Lambda^{T} \, I \, \Lambda \right) \, X 
\label{eq:ecuacion_10_042}
\end{align}
Por lo tanto, si $\Lambda$ es tal que deja la distancia invariante, es decir $l^{2} = l^{\prime \, 2}$, debe cumplir que:
\begin{align}
\Lambda^{T} \, I \, \Lambda = I
\label{eq:ecuacion_10_043}
\end{align}

Otra forma de expresar esta igualdad es $\Lambda^{T} = \Lambda^{-1}$. Claramente las matrices que cumplen la ec. (\ref{eq:ecuacion_10_043}) forman un grupo, a este grupo de matrices se le llama $O(n)$.
\par
Recordemos que para cualquier matriz $A$ se cumple que: $\det A = \det A^{T}$. Entonces, las matrices que satisfacen la eq. (\ref{eq:ecuacion_10_043}) deben cumplir $(\det \Lambda)^{2} = 1$, es decir $\det \Lambda = \pm 1$. El subconjunto de matrices $\Lambda$ que cumplen $\det \Lambda = - 1$ no forman un grupo, por ejemplo, la identidad no está en ese subconjunto. Sin embargo, las matrices $\Lambda$ que cumplen $\det \Lambda = 1$ sí forman un grupo, este es el grupo\footnote{La letra S, indica el carácter especial (\emph{special} en inglés), del subgrupo de matrices ortogonales} $SO(n)$.
\par
Notemos que la matriz de $n \times n$:
\begin{align}
\Lambda = \mqty(\Lambda_{11} & \Lambda_{12} & \ldots & \Lambda_{1n} \\ \Lambda_{21} & \Lambda_{22} & \ldots & \Lambda_{2n} \\ \vdots & \vdots & \ddots & \vdots \\ \Lambda_{n1} & \Lambda_{n2} & \ldots & \Lambda_{nn})
\label{eq:ecuacion_10_44}
\end{align}
se puede formar con los siguientes vectores columna:
\begin{align}
\va{C}_{1} = \mqty(\Lambda_{11} \\ \Lambda_{21} \\ \vdots \\ \Lambda_{n1}), \hspace{0.5cm} \va{C}_{2} = \mqty(\Lambda_{12} \\ \Lambda_{22} \\ \vdots \\ \Lambda_{n2}), \hspace{0.5cm} \ldots, \va{C}_{n} = \mqty(\Lambda_{1n} \\ \Lambda_{2n} \\ \vdots \\ \Lambda_{nn})
\label{eq:ecuacion_10_045}
\end{align}
Claramente, para la matriz transpuesta $\Lambda^{T}$, estos vectores representan los renglones. Por lo tanto, por la condición dada en la ec. (\ref{eq:ecuacion_10_043}), se puede escribir como:
\begin{align}
\Lambda^{T} \, \Lambda &= \mqty(\Lambda_{11} & \Lambda_{21} & \ldots & \Lambda_{n1} \\ \Lambda_{12} & \Lambda_{22} & \ldots & \Lambda_{nn} \\ \vdots & \vdots & \ddots & \vdots \\ \Lambda_{1n} & \Lambda_{2n} & \ldots & \Lambda_{nn}) \, \mqty(\Lambda_{11} & \Lambda_{12} & \ldots & \Lambda_{1n} \\ \Lambda_{21} & \Lambda_{22} & \ldots & \Lambda_{2n} \\ \vdots & \vdots & \ddots & \vdots \\ \Lambda_{n1} & \Lambda_{n2} & \ldots & \Lambda_{nn}) \nonumber \\[1em]
&= \mqty(\va{C}_{1} \cdot \va{C}_{1} & \va{C}_{1} \cdot \va{C}_{2} & \ldots & \va{C}_{1} \cdot \va{C}_{n} \\ \va{C}_{2} \cdot \va{C}_{1} & \va{C}_{2} \cdot \va{C}_{2} & \ldots & \va{C}_{n} \cdot \va{C}_{2} \\ \vdots & \vdots & \ddots & \vdots \\ \va{C}_{n} \cdot \va{C}_{1} & \va{C}_{n1} \cdot \va{C}_{2} & \ldots & \va{C}_{n} \cdot \va{C}_{n}) = I
\label{eq:ecuacion_10_046}
\end{align}
Otra forma de expresar esta igualdad es:
\begin{align}
\va{C}_{i} \cdot \va{C}_{j} = \delta_{ij}
\label{eq:ecuacion_10_047}
\end{align}
es decir si una matriz satisface la eq. (\ref{eq:ecuacion_10_043}), tiene sus columnas ortonormales. Ahora, si $\Lambda$ satisface la condición (\ref{eq:ecuacion_10_043}), entonces $\Lambda^{-1}$ también la satisface. Por lo tanto, $\Lambda^{-1}$ tiene sus columnas ortonormales entre sí. Pero se debe cumplir $\Lambda^{-1} = \Lambda^{T}$, entonces las columnas de $\Lambda^{T}$ son ortonormales entre sí.
\par
Considerando que las columnas de $\Lambda^{T}$ son los renglones de $\Lambda$, podemos ver que los renglones de $\Lambda$ son ortonormales entre sí. En conclusión, si $\Lambda$ satisface la eq. (\ref{eq:ecuacion_10_043}), sus columnas y renglones son ortonormales entre sí.
\par
Una matriz de $n \times n$ tiene $n^{2}$ parámetros libres, pero si satisface la ec. (\ref{eq:ecuacion_10_046}) no todos sus parámetros son libres. De la ec. (\ref{eq:ecuacion_10_046}) se puede ver que $\Lambda^{T} \, \Lambda$ es una matriz simétrica, por lo que la ec. (\ref{eq:ecuacion_10_046}) sólo tiene $\dfrac{n (n+1)}{2}$ ecuaciones independientes. Así, los parámetros libres de una matriz que satisface la ec. (\ref{eq:ecuacion_10_046}) son:
\begin{align*}
n^{2} - \dfrac{n (n+1)}{2} = \dfrac{n (n-1)}{2}
\end{align*}
Para el caso $n = 3$ hay tres parámetros libres, para ésta dimensión cualquier matriz se puede escribir como:
\begin{align}
\Lambda = \mqty(a_{1} & b_{1} & c_{1} \\ a_{2} & b_{2} & c_{2} \\ a_{3} & b_{3} & c_{3})
\label{eq:ecuacion_10_048}
\end{align}
Si ésta matriz satisface la ec. (\ref{eq:ecuacion_10_046}), debe cumplir:
\begin{align}
\begin{aligned}
\va{a} \vdot \va{a} = \va{b} \vdot \va{b} = \va{c} \vdot \va{c} =  1 \\[0.5em]
\va{a} \vdot \va{b} = \va{a} \vdot \va{c} = \va{b} \vdot \va{c} =  0
\end{aligned}
\label{eq:ecuacion_10_049}
\end{align}
esto nos dice que la punta de los vectores $\va{a}, \va{b},\va{c}$ están en una esfera y que son ortonormales entre sí. Un ejemplo de este tipo de matrices son:
\begin{align}
\Lambda_{x} (\theta) = \mqty(1 & 0 & 0 \\ 0 & \cos \theta & \sin \theta \\ 0 & -\sin \theta & \cos \theta) \label{eq:ecuacion_10_050} \\[1em]
\Lambda_{y} (\psi) = \mqty(\cos \psi & 0 & \sin \psi \\ 0 & 1 & 0 \\ - \sin \psi & 0 & \cos \psi) \label{eq:ecuacion_10_051} \\[1em]
\Lambda_{z} (\phi) = \mqty(\cos \phi & \sin \phi & 0 \\ - \sin \phi & \cos \phi & 0 \\ 0 & 0 & 1) \label{eq:ecuacion_10_052}
\end{align}
Cada una de estas matrices representa una rotación:
\begin{itemize}
\item $\Lambda_{x} (\theta)$ representa una rotación sobre el eje $x$.
\item $\Lambda_{y} (\psi)$ representa una rotación sobre el eje $y$.
\item $\Lambda_{z} (\phi)$ representa una rotación sobre el eje $z$
\end{itemize}
Por lo tanto, las rotaciones dejan invariante la distancia y al Laplaciano. Nótese que las tres matrices son linealmente independientes.
\par
Existen otras formas de reparametrizar una matriz de rotación, por ejemplo la base unitaria en coordenadas esféricas:
\begin{align*}
\vu{e}_{r} &= \cos \phi \sin \theta \, \vu{i} + \sin \phi \sin \theta \, \vu{j} + \cos \theta \, \vu{k} \\
\vu{e}_{\theta} &= \cos \phi \cos \theta \, \vu{i} + \sin \phi \cos \theta \, \vu{j} - \sin \theta \, \vu{k}
\end{align*}
satisfacen la condición dada por la eq. (\ref{eq:ecuacion_10_049}), pero no son la solución más general, pues sólo dependen de dos parámetros mientras que la solución general de la eq. (\ref{eq:ecuacion_10_049}) depende de tres. Pero estos vectores nos sirven para obtener la solución general. Propondremos a $\va{c}$ como:
\begin{align}
\va{c} = (\sin \psi \sin \theta, \cos \psi \sin \theta, \cos \theta)
\label{eq:ecuacion_10_053}
\end{align}
Los vectores $\va{b}$ y $\va{c}$ deben ser tales que si $\phi = 0$ se cumple:
\begin{align}
\va{b}\eval_{\phi=0} &= \vu{e}_{\theta} = (\sin \psi \cos \theta, \cos \psi \cos \theta, - \sin \theta) \label{eq:ecuacion_10_054} \\[1em]
\eval{\va{c}}_{\phi=0} &= \vu{e}_{\psi} = (\cos \psi, -\sin \psi, 0) \label{eq:ecuacion_10_055}
\end{align}
Un par de vectores que satisfacen estas condiciones son:
\begin{changemargin}{-0.25cm}{-0.25cm} 
\begin{align*}
\va{a} &= (\sin \phi \sin \psi \cos \theta + \cos \phi \cos \psi, - \sin \phi \cos \psi \cos \theta - \cos \phi \sin \psi, \sin \phi \sin \theta) \\[1em]
\va{b} &= (\sin \phi \cos \psi + \cos \theta \sin \psi \cos \theta, -\sin \phi \sin \psi + \cos \phi \cos \psi \cos \theta, - \cos \phi \sin \theta)
\end{align*}
\end{changemargin}
Se puede demostrar que los vectores $\va{a}, \va{b}$ y $\va{c}$ cumplen con la ec. (\ref{eq:ecuacion_10_049}). De esta manera $\Lambda$ se puede escribir como\footnote{Con el fin de abreviar, $c \, \psi = \cos \psi$, así también $s \, \phi = \sin \phi$, etc.}:
\begin{align}
\begin{aligned}
&\Lambda (\phi, \theta, \psi) = \\[0.5em]
&= \mqty(
c \, \phi \, c \, \psi - s \, \phi \, s \, \psi \, c \, \theta & s \, \theta \, c \, \phi + c \, \phi \, s \, \psi \, c \, \theta & s \, \psi \, s \, \theta \\
- c \, \phi \, s \, \theta \, - s \, \phi \, c \, \psi c \, \theta & - s \, \phi \, s \, \psi + c \, \phi \, c \, \psi \, c \, \theta & c \, \psi \, s \, \theta \\
s \, \phi \, s \, \theta & - c \, \phi \, s \, \theta & c \, \theta)
\end{aligned}
\label{eq:ecuacion_10_056}
\end{align}
Claramente aún hay cierta arbitrariedad, pues podemos cambiar el lugar de los vectores $\va{a}, \va{b}, \va{c}$, también podemos cambiar renglones por columnas y se seguirá cumpliendo la eq. (\ref{eq:ecuacion_10_043}). La ventaja de escribir $\Lambda$ de la forma (\ref{eq:ecuacion_10_056}) es que se puede expresar como el producto de tres matrices:
\begin{align}
\Lambda (\phi, \theta, \psi) = \Lambda_{1} (\phi) \, \Lambda_{2} (\theta) \, \Lambda_{3} (\psi) 
\label{eq:ecuacion_10_057}
\end{align}
con:
\begin{align}
\Lambda_{1} (\phi) = \mqty(\cos \phi & \sin \phi & 0 \\ - \sin \phi & \cos \phi & 0 \\ 0 & 0 & 1) \label{eq:ecuacion_10_058} \\[1em]
\Lambda_{2} (\theta) = \mqty(1 & 0 & 0 \\ 0 & \cos \theta & \sin \theta \\ 0 & - \sin \theta & \cos \theta) \label{eq:ecuacion_10_059} \\[1em]
\Lambda_{3} (\psi) = \mqty(\cos \psi & \sin \psi & 0 \\ - \sin \psi & \cos \psi & 0 \\ 0 & 0 & 1) \label{eq:ecuacion_10_060}
\end{align}
Esta descomposición es muy útil para el estudio del movimiento del cuerpo rígido. A los parámetros $\theta, \psi, \phi$ se les llama \emph{ángulos de Euler}.
\par
Las rotaciones no son las únicas transformaciones de $O(3)$. Por ejemplo, las matrices:
\begin{align}
\mqty(-1 & 0 & 0 \\ 0 & 1 & 0 \\ 0 & 0 & 1) \hspace{1cm} \mqty(1 & 0 & 0 \\ 0 & -1 & 0 \\ 0 & 0 & -1) \hspace{1cm} \mqty(-1 & 0 & 0 \\ 0 & -1 & 0 \\ 0 & 0 & -1)
\label{eq:ecuacion_10_061}
\end{align}
cumplen la ec. (\ref{eq:ecuacion_10_049}) pero no son de rotación, tampoco son de $SO(3)$. Las rotaciones representan las transformaciones de $O(3)$ que se pueden conectar con la matriz unidad $I$. Pues al hacer $\phi = 0, \theta = 0, \psi = 0$, se obtiene la unidad $I$.

\section{Transformaciones infinitesimales.}

Las transformaciones de $SO(n)$ que están infinitesimalmente cercanas a la unidad, es decir, aquellas que cumplen:
\begin{align}
\Lambda \approx I + \varepsilon \, M \hspace{1.5cm} \varepsilon \ll 1
\label{eq:ecuacion_10_062}
\end{align}
son particularmente importantes. Para este tipo de transformaciones la condición dada por la eq. (\ref{eq:ecuacion_10_043}) implica que:
\begin{align}
I = \Lambda^{T} \, \Lambda &\approx (I + \varepsilon \, M)^{T} \, (I + \varepsilon \, M) \label{eq:ecuacion_10_063} \\[0.5em]
&= (I + \varepsilon \, M^{T}) \, (I + \varepsilon \, M)  \approx I + \varepsilon (M^{T} + M) \label{eq:ecuacion_10_064}
\end{align}
Por lo tanto:
\begin{align}
M = - M^{T}
\label{eq:ecuacion_10_065}
\end{align}
es decir, $M$ debe ser antisimétrica. Notemos que una matriz antisimétrica de $n \times n$ sólo puede tener:
\begin{align*}
\dfrac{n(n-1)}{2}
\end{align*}
parámetros libres, este número de grados de libertad coincide con los parámetros libres del grupo $SO(n)$. Para el caso particular $n = 3$, cualquier matriz antisimétrica se puede escribir como:
\begin{align}
M = \mqty(0 & -\alpha_{3} & \alpha_{2} \\ \alpha_{3} & 0 & -\alpha_{1} \\ -\alpha_{2} & \alpha_{1} & 0)
\label{eq:ecuacion_10_066}
\end{align}
Definamos los vectores $\va{r} = (x, y ,z)$, $\delta \, \va{\alpha} = \varepsilon (\alpha_{1}, \alpha_{2}, \alpha_{3})$, entonces:
\begin{align}
\begin{aligned}[b]
\delta \, X &= \varepsilon \, M \, X = \\[0.5em]
&= \varepsilon \, \mqty(0 & -\alpha_{3} & \alpha_{2} \\ \alpha_{3} & 0 & -\alpha_{1} \\ -\alpha_{2} & \alpha_{1} & 0) \mqty(x \\ y \\ z) = \\[0.5em]
&= \varepsilon \mqty(\alpha_{2} \, z - \alpha_{3} \, y \\ \alpha_{3} \, x - \alpha_{1} \, z \\ \alpha_{1} \, x - \alpha_{2} \, y) \\[0.5em]
&= \delta \, \va{\alpha} \cp \va{r}
\end{aligned}
\label{eq:ecuacion_10_067}
\end{align}
Por lo tanto, una rotación infinitesimal está dada por:
\begin{align}
X^{\prime} = \Lambda \, X \approx ( I + \varepsilon \, M) \, X = (I \, M + \varepsilon \, M \, X)
\label{eq:ecuacion_10_068}
\end{align}
es decir:
\begin{align}
\va{r}^{\, \prime} = \va{r} + \delta \, \va{\alpha} \cp \va{r}
\label{eq:ecuacion_10_069}
\end{align}
Estas rotaciones son importantes porque definen al resto de las transformaciones de $SO(3)$, para ver esto primero notemos que:
\begin{align}
\begin{aligned}
M &= \mqty(0 & -\alpha_{3} & \alpha_{2} \\ \alpha_{3} & 0 & -\alpha_{1} \\ -\alpha_{2} & \alpha_{1} & 0) = \alpha_{1} \, m_{1} + \alpha_{2} \, m_{2} + \alpha_{3} \, m_{3} = \\[0.5em]
&= \va{\alpha} \cdot \va{m}
\end{aligned}
\label{eq:ecuacion_10_070}
\end{align}
donde:
\begin{align*}
m_{1} = \mqty(0 & 0 & 0 \\ 0 & 0 & -1 \\ 0 & 1 & 0), \hspace{1cm} m_{2} = \mqty(0 & 0 & 1 \\ 0 & 0 & 0 \\ -1 & 0 & 0), \hspace{1cm} m_{3} = \mqty(0 & -1 & 0 \\ 1 & 0 & 0 \\ 0 & 0 & 0)
\end{align*}
Se puede observar que la matriz (\ref{eq:ecuacion_10_070}) no se modifica si hacemos el cambio $\va{\alpha} \rightarrow - i \, \va{\alpha}$ y $\va{m} \rightarrow - i \, \va{m} = \va{M}$, es decir:
\begin{align*}
M_{1} = i \, \mqty(0 & 0 & 0 \\ 0 & 0 & -1 \\ 0 & 1 & 0), \hspace{1cm} M_{2} = i \, \mqty(0 & 0 & 1 \\ 0 & 0 & 0 \\ -1 & 0 & 0), \hspace{1cm} M_{3} = i \, \mqty(0 & -1 & 0 \\ 1 & 0 & 0 \\ 0 & 0 & 0)
\end{align*}
La ventaja de ocupar la matrices $M_{i}$ es que son Hermíticas, es decir, \break \hfill $M_{i}^{\dagger} = M_{i}$. Esto implica que sus valores propios son reales, por lo tanto las matrices
$\va{M} = (M_{1} , M_{2} , M_{3})$ pueden representar cantidades físicas.
\par
Una transformación finita se debe hacer como producto infinito de transformaciones infinitesimales. Por ejemplo, ocupando:
\begin{align}
\begin{aligned}[b]
X^{\prime} &\approx \bigg[ \Lambda \left( \dfrac{\va{\alpha}}{N} \right) \bigg]^{N} \, X = \\[0.5em]
&= \big[ I - i \, \delta \, \va{\alpha} \cdot \va{M} \big]^{N} \, X = \\[0.5em]
&= \bigg[ I - i \, \dfrac{1}{N} \, \va{\alpha} \cdot \va{M} \bigg]^{N} \, X
\end{aligned}
\label{eq:ecuacion_10_071}
\end{align}
y considerando el siguiente resultado:
\begin{align}
\lim_{N \to \infty} \left(  I + \dfrac{\beta \, x}{N} \right)^{N} = e^{\beta \, x}
\label{eq:ecuacion_10_072}
\end{align}
se tiene que:
\begin{align}
\lim_{N \to \infty} \left(  I - i \, \dfrac{1}{N} \, \va{\alpha} \cdot \va{M} \right)^{N} = e^{- i \, \va{\alpha} \cdot \va{M}}
\label{eq:ecuacion_10_073}
\end{align}
Por lo tanto, una transformación finita, está dada por:
\begin{align}
X^{\prime} = e^{- i \, \va{\alpha} \cdot \va{M}} \, X
\label{eq:ecuacion_10_074}
\end{align}
Así, cualquier transformación infinitesimal conectada con la identidad tiene la forma:
\begin{align}
\Lambda (\va{\alpha}) = e^{- i \, \va{\alpha} \cdot \va{M}}
\label{eq:ecuacion_10_075}
\end{align}
En este sentido se dice que $M_{1}, M_{2}, M_{3}$ son \emph{los generadores} del grupo $SO(3)$.
\par
Ahora veamos de forma explícita la expresión dada por la ec.(\ref{eq:ecuacion_10_075}) para algunos casos particulares. Primero notemos que si $n \geq 1$, se tiene:
\begin{align}
M_{1}^{2n} = \mqty(0 & 0 & 0 \\ 0 & 1 & 0 \\ 0 & 0 & 1) = T_{1}, \hspace{1cm} M_{1}^{2n + 1} = M_{1}  \label{eq:ecuacion_10_076} \\[1em]
M_{2}^{2n} = \mqty(1 & 0 & 0 \\ 0 & 0 & 0 \\ 0 & 0 & 1) = T_{2}, \hspace{1cm} M_{2}^{2n + 1} = M_{2}  \label{eq:ecuacion_10_077} \\[1em]
M_{3}^{2n} = \mqty(1 & 0 & 0 \\ 0 & 1 & 0 \\ 0 & 0 & 0) = T_{3}, \hspace{1cm} M_{3}^{2n + 1} = M_{3}  \label{eq:ecuacion_10_078}
\end{align}
Entonces, considerando estos resultados en conjunto con las series de $\cos \beta$ y $\sin \beta$, se tiene:
\begin{align}
\begin{aligned}
e^{i \, \beta \, M_{i}} &= \nsum_{n \geq 0} \dfrac{1}{n!} \, (- i \, \beta \, M_{i})^{n} = \\[0.5em]
&= \nsum_{n \geq 0} \dfrac{1}{(2 n)!} \, (- i \, \beta)^{2n} \, (M_{i})^{2n} + \nsum_{n \geq 0} \dfrac{1}{(2 n +1)!} \, (- i \, \beta)^{2n+1} \, (M_{i})^{2n+1} = \\[0.5em]
&= I + T_{i} \, \nsum_{n \geq 1} \dfrac{1}{(2 n)!} \, (- i \, \beta)^{2n} + M_{i} \, \nsum_{n \geq 0} \dfrac{1}{(2 n +1)!} \, (- i \, \beta)^{2n+1} = \\[0.5em]
&= I - T{i} + T_{i} + \nsum_{n \geq 1} \dfrac{(-1)^{n}}{(2 n)!} \, \beta^{2n} - i \, M \, \nsum_{n \geq 0} \dfrac{(-1)^{n}}{(2 n +1)!} \, \beta^{2n+1} = \\[0.5em]
&= I - T{i} + T_{i} \, \nsum_{n \geq 0} \dfrac{(-1)^{n} \beta^{2 n}}{(2 n)!} - i \, M_{i} \, \sin \beta =\\[0.5em]
&= I - T{i} + T_{i} \, \cos \beta - i \, M_{i} \, \sin \beta
\end{aligned}
\label{eq:ecuacion_10_079}
\end{align}
De donde:
\begin{align}
\begin{aligned}
e^{-i \, \alpha_{1} \, M_{1}} = \mqty(1 & 0 & 0 \\ 0 & \cos \alpha_{1} & - \sin \alpha_{1} \\ 0 & \sin \alpha_{1} & \cos \alpha_{1}) \\[0.5em]
e^{-i \, \alpha_{2} \, M_{2}} = \mqty(\cos \alpha_{2} & 0 & \sin \alpha_{2} \\ 0 & 1 & 0 \\ - \sin \alpha_{2} & 0 & \cos \alpha_{2}) \\[0.5em]
e^{-i \, \alpha_{3} \, M_{3}} = \mqty(\cos \alpha_{3} & - \sin \alpha_{3} & 0 \\ \sin \alpha_{3} & \cos \alpha_{3} & 0 \\ 0 & 0 & 1)
\end{aligned}
\label{eq:ecuacion_10_080}
\end{align}
Así, $e^{-i \, \alpha_{1} \, M_{1}}$ representa una rotación sobre el eje $x$, $e^{-i \, \alpha_{2} \, M_{2}}$ representa una rotación sobre el eje $y$, mientras que $e^{-i \, \alpha_{3} \, M_{3}}$ representa una rotación sobre el eje $z$.
\par
Veamos qué reglas de conmutación\footnote{Ya conocemos el concepto de conmutador $[A, B] = A \, B - B \, A$.} cumplen los generadores de las rotaciones $M_{1}, M_{2}, M_{3}$. Primero notemos que:
\begin{align}
\begin{aligned}
M_{1} \, M_{2} = - \mqty(0 & 0 & 0 \\ 1 & 0 & 0 \\ 0 & 0 & 0) \\[0.5em] 
M_{1} \, M_{3} = - \mqty(0 & 0 & 0 \\ 0 & 0 & 0 \\ 1 & 0 & 0) \\[0.5em]
M_{2} \, M_{3} = - \mqty(0 & 0 & 0 \\ 0 & 0 & 0 \\ 0 & 1 & 0)
\end{aligned}
\qquad
\begin{aligned}
M_{2} \, M_{1} = - \mqty(0 & 1 & 0 \\ 0 & 0 & 0 \\ 0 & 0 & 0) \\[0.5em]
M_{3} \, M_{1} = - \mqty(0 & 0 & 1 \\ 0 & 0 & 0 \\ 0 & 0 & 0) \\[0.5em]
M_{3} \, M_{2} = - \mqty(0 & 0 & 0 \\ 0 & 0 & 1 \\ 0 & 0 & 0)
\end{aligned}
\label{eq:ecuacion_10_081}
\end{align}
por lo que:
\begin{align}
\comm{M_{1}}{M_{2}} = i \, M_{3}, \hspace{1cm} \comm{M_{3}}{M_{1}} = i \, M_{2}, \hspace{1cm} \comm{M_{2}}{M_{3}} = i \, M_{1} \label{eq:ecuacion_10_082}
\end{align}
Estas reglas de conmutación se pueden escribir como:
\begin{align}
\comm{M_{i}}{M_{j}} = i \, \varepsilon_{ijk} \, M_{k}
\label{eq:ecuacion_10_083}
\end{align}
Por el hecho de que el conmutador de dos generadores de rotación nos de otro generador de rotación, se dice que \emph{estos generadores forman un álgebra de Lie}, el álgebra de Lie de $SO(3)$. Nótese que el conmutador entre dos generadores no es cero, dos generadores no se pueden diagonalizar simultáneamente. Ahora definamos:
\begin{align}
M^{2} = M_{1}^{2} + M_{2}^{2} + M_{3}^{2}
\label{eq:ecuacion_10_084}
\end{align}
de donde:
\begin{align}
M^{2} = 2 \, I
\label{eq:ecuacion_10_085}
\end{align}
por lo tanto:
\begin{align}
\comm{M^{2}}{M_{i}} = 0
\label{eq:ecuacion_10_086}
\end{align}
Así, los valores propios de $M^{2}$ se pueden obtener al mismo tiempo que los generadores $M_{i}$.
\par
Los valores propios de los generadores $M_{i}$ se pueden calcular directamente y están dados por:
\begin{align}
M_{1} \, V = \lambda \, V \hspace{0.5cm} \Rightarrow \hspace{0.5cm} \lambda = \pm 1 \hspace{1cm} V = a \dfrac{1}{\sqrt{2}} \mqty(0 \\ 1 \\ \pm i) \label{eq:ecuacion_10_087} \\[0.5em]
M_{2} \, V = \lambda \, V \hspace{0.5cm} \Rightarrow \hspace{0.5cm} \lambda = \pm 1 \hspace{1cm} V = a \dfrac{1}{\sqrt{2}} \mqty(1 \\ 0 \\ \pm i) \label{eq:ecuacion_10_088} \\[0.5em]
M_{3} \, V = \lambda \, V \hspace{0.5cm} \Rightarrow \hspace{0.5cm} \lambda = \pm 1 \hspace{1cm} V = a \dfrac{1}{\sqrt{2}} \mqty(1 \\ \pm i \\ 0) \label{eq:ecuacion_10_089}
\end{align}
Si $a = 1$, en cada caso se tienen vectores propios ortonormales. Estos resultados no son difíciles de obtener, posteriormente veremos que los generadores de las rotaciones en el espacio de las funciones de $\va{x}$ están relacionados con los operadores diferenciales.
\par
Con el símbolo $\epsilon_{ijk}$ las matrices $\va{M} = (M_{1}, M_{2}, M_{3})$ se pueden escribir de forma más económica. En efecto, en componentes tenemos:
\begin{align}
(M_{1})_{ij} = i \, \epsilon_{i 1 j} \hspace{1cm} (M_{2})_{ij} = i \, \epsilon_{i 2 j} \hspace{1cm} (M_{3})_{ij} = i \, \epsilon_{i 3 j} \label{eq:ecuacion_10_090}
\end{align}
Por ejemplo:
\begin{align}
(M_{1})_{ij} = i \, \epsilon_{i 1 j} = i \, \mqty(\epsilon_{111} & \epsilon_{112} & \epsilon_{113} \\ \epsilon_{211} & \epsilon_{212} & \epsilon_{213} \\ \epsilon_{311} & \epsilon_{312} & \epsilon_{313}) = i \, \mqty(0 & 0 & 0 \\ 0 & 0 & -1 \\ 0 & 1 & 0)
\label{eq:ecuacion_10_091}
\end{align}
se puede demostrar que las demás igualdades se cumplen.
\par
Así, las componentes de cualquier matriz antisimétrica se pueden escribir como:
\begin{align}
M_{ik} = - i \, \alpha_{j} (M_{j})_{ik} = \alpha_{j} \, \epsilon_{ijk}
\label{eq:ecuacion_10_092}
\end{align}
Ahora, si $\alpha_{i}$ es infinitesimal, por ejemplo $\delta \, \alpha_{i} = \alpha_{i} / N$ con $N$ grande, una transformación infinitesimal en el espacio de $\va{x}$ está dada por:
\begin{align}
X^{\prime} = \Lambda \, X = (I - i \, \delta \, \va{\alpha} \cdot \va{M}) \, X
\label{eq:ecuacion_10_093}
\end{align}
En componentes se tiene:
\begin{align}
\begin{aligned}[b]
x_{i}^{\prime} &\approx \Lambda_{ik} \, x_{k} = \\[0.5em]
&= (I - i \, \delta \, \alpha_{j} \, M_{j})_{ik} \, x_{k} = \\[0.5em]
&= (\delta_{ik} - i \, \delta \, \alpha_{j} \, (M_{j})_{ik}) \, x_{k} = \\[0.5em]
&= (\delta_{ik} + \delta \, \alpha_{j} \, \epsilon_{ijk}) \, x_{k} = \\[0.5em]
&= x_{i} + \epsilon_{ijk} \, \delta \, \alpha_{j} \, x_{k} = \\[0.5em]
&= x_{i} + (\delta \, \va{\alpha} \cp \va{x})_{i}
\end{aligned}
\label{eq:ecuacion_10_094}
\end{align}
es decir:
\begin{align}
x_{i}^{\prime} &\approx x_{i} + (\delta \, \va{\alpha} \cp \va{x})_{i} \label{eq:ecuacion_10_095}
\end{align}

\section{Armónicos esféricos.}

Hasta el momento nos hemos enfocado en las trasformaciones que pasan del espacio $\va{r}$ al $\va{r}^{\, \prime}$. Ahora veamos que pasa con las funciones que actúan en estos espacios.
\par
Supongamos que tenemos una función, $F$, de $\mathbb{R}^{3} \rightarrow \mathbb{R}$. Entonces al evaluar esta función en un punto $\va{r}^{\, \prime}$ y ocupando la transformación infinitesimal dada por la eq. (\ref{eq:ecuacion_10_069}), se encuentra que:
\begin{align}
F(\va{r}^{\, \prime}) \approx F (\va{r} + \delta \, \va{\alpha} \cp \va{r}) = F(\va{r}) + (\delta \, \va{\alpha} \cp \va{r}) \cdot \va{\nabla} F(\va{r})
\label{eq:ecuacion_10_096}
\end{align}
Usando la propiedad cíclica del triple producto escalar, se tiene que:
\begin{align*}
(\delta \, \va{\alpha} \cp \va{r}) \cdot \va{\nabla} F(\va{r}) &= \big( \va{\nabla} F(\va{r}) \cp \delta \, \va{\alpha} \big) \cdot \va{r} \\[0.5em]
&= \big( \va{r} \cp \va{\nabla} F(\va{r}) \big) \cdot \delta \, \va{\alpha} \\[0.5em]
&= i \, \big( \delta \, \va{\alpha} \cdot \va{L}) \, F(\va{r} \big) 
\end{align*}
con:
\begin{align}
\va{L} = - i \, \va{r} \cp \va{\nabla}
\label{eq:ecuacion_10_097}
\end{align}
Entonces tenemos que:
\begin{align}
F(\va{r}^{\, \prime}) \approx \big( 1 + i \, \delta \va{\alpha} \cdot \va{L} \big) \, F (\va{r})
\label{eq:ecuacion_10_098}
\end{align}
Por lo tanto, \textbf{el operador $\va{L}$ es el generador de las rotaciones en el espacio de las funciones}. Anteriormente se vio que este operador es Hermítico, por lo que sus valores propios son reales.
\par
Veamos qué forma tiene una rotación finita. Al igual que en caso del espacio $\va{r}$ tomaremos $\delta \, \va{\alpha} = \va{\alpha} / N$ y consideraremos que para tener una transformación finita debemos hacer el producto de un número infinito de transformaciones infinitesimales. Entonces:
\begin{align}
F(\va{r}^{\, \prime}) \approx \bigg( 1 + \dfrac{\va{\alpha}}{N} \cdot \va{L} \bigg)^{N} \, F(\va{r})
\label{eq:ecuacion_10_099}
\end{align}
Al ocupar el resultado obtenido en la ec. (\ref{eq:ecuacion_10_072}), se encuentra que:
\begin{align}
\lim_{N \to \infty} \left( I + i \, \dfrac{1}{N} \, \va{\alpha} \cdot \va{L} \right)^{N} = U(\va{\alpha}) = e^{i \, \va{\alpha} \cdot \va{L}}
\label{eq:ecuacion_10_100}
\end{align}
Por lo tanto, una transformación finita está dada por:
\begin{align}
F(\va{r}^{\, \prime}) = e^{i \, \va{\alpha} \cdot \va{L}} \, F(\va{r})
\label{eq:ecuacion_10_101}
\end{align}
Nótese que el operador $U(\va{\alpha})$ satisface:
\begin{align}
U(\va{\alpha})^{\dagger} = U(- \va{\alpha}) = U^{-1} (\va{\alpha})
\label{eq:ecuacion_10_102}
\end{align}
Cuando un operador $A$ cumple:
\begin{align*}
A^{\dagger} = A^{-1}
\end{align*}
se dice que es \emph{un operador unitario}. Así, el operador $U(\va{\alpha})$ es unitario.

\subsection{Reglas de conmutación del momento angular.}

Anteriormente vimos que los generadores $M_{i}$ satisfacen las reglas de conmutación dadas por la ec. (\ref{eq:ecuacion_10_083}). Veamos qué reglas de conmutación cumplen los operadores $L_{i}$. Primero notemos que definiendo $\va{p} = - i \, \va{\nabla}$, se tiene que $\va{L} = \va{r} \cp \va{p}$. En componentes se encuentra que $L_{i} = \epsilon_{ijk} \, x_{j} \, p_{k}$. De donde
\begin{align}
\begin{aligned}[b]
\comm{L_{i}}{L_{j}} &= \comm{\epsilon_{ilm} \, x_{l} \, p_{m}}{\epsilon_{jrs} \, x_{r} \, p_{s}} = \\[0.5em]
&= \epsilon_{ilm} \, \epsilon_{jrs} \, \comm{x_{l} \, p_{m}}{x_{r} \, p_{s}} = \\[0.5em]
&= \epsilon_{ilm} \, \epsilon_{jrs} \, (x_{l} \, \comm{p_{m}}{x_{r}} \, p_{s} + x_{r} \, \comm{x_{l}}{p_{s}} \, p_{m}) = \\[0.5em]
&= \epsilon_{ilm} \, \epsilon_{jrs} \, (- i \, x_{l} \, p_{s} \, \delta_{mr} + i \, x_{r} \, p_{m} \, \delta_{ls}) = \\[0.5em]
&= i \, (- \epsilon_{ilr} \, \epsilon_{jrs} \, x_{l} \, p_{s} + \epsilon_{ism} \, \epsilon_{jrs} \, x_{r} \, p_{m}) = \\[0.5em]
&= i \, (\epsilon_{ilr} \, \epsilon_{rjs} \, x_{l} \, p_{s} - \epsilon_{ims} \, \epsilon_{sjr} \, x_{r} \, p_{m}) = \\[0.5em]
&= i \, (\epsilon_{iar} \, \epsilon_{rjb} - \, x_{a} \, p_{b} - \epsilon_{ibs} \, \epsilon_{sja} \, x_{r} \, p_{b}) = \\[0.5em]
&= i \, (\epsilon_{iar} \, \epsilon_{rjb} - \epsilon_{ibs} \, \epsilon_{sja}) \, x_{a} \, p_{b}
\end{aligned}
\label{eq:ecuacion_10_103}
\end{align}
Ahora veamos que:
\begin{align}
\begin{aligned}[b]
\epsilon_{iar} \, \epsilon_{rjb} - \epsilon_{ibs} \, \epsilon_{sja} &= (\delta_{ij} \, \delta_{ab} - \delta_{ib} \, \delta_{aj}) - (\delta_{ij} \, \delta_{ba} - \delta_{ia} \, \delta_{bj}) = \\[0.5em]
&= \delta_{ia} \, \delta_{bj} - \delta_{ib} \, \delta_{aj} = \\[0.5em]
&= \epsilon_{ijk} \, \epsilon_{kab}
\end{aligned}
\label{eq:ecuacion_10_104}
\end{align}
Ocupando este resultado, tenemos que:
\begin{align*}
\comm{L_{i}}{L_{j}} = i \, \epsilon_{ijk} \, \epsilon_{kab} \, x_{a} \, p_{b}
\end{align*}
es decir:
\begin{align}
\comm{L_{i}}{L_{j}} = i \, \epsilon_{ijk} \, L_{k}
\label{eq:ecuacion_10_105}
\end{align}
Se puede observar que son las mismas reglas de conmutación que cumple $M_{i}$ ec. (\ref{eq:ecuacion_10_083}). A éstas reglas de conmutación se les llama \emph{el álgebra de Lie} de $SO(3)$.
\par
Anteriormente vimos que la matriz $M^{2} = M_{1}^{2} + M_{2}^{2} + M_{3}^{2}$ conmuta con todas las matrices $M_{i}$. Para los operadores $L_{i}$ el operador equivalente a $M^{2}$ es:
\begin{align}
L^{2} = L_{x}^{2} + L_{y}^{2} + L_{z}^{2} = L_{j} , L_{j} 
\label{eq:ecuacion_10_106}
\end{align}
El cual cumple con:
\begin{align}
\begin{aligned}[b]
\comm{L^{2}}{L_{i}} &= \comm{L_{j} \, L_{j}}{L_{i}} = \\[0.5em]
&= L_{j} \, \comm{L_{j}}{L_{i}} + \comm{L_{j}}{L_{i}} \, L_{j} \\[0.5em]
&= i \, \epsilon_{jil} \, L_{j} \, L_{l} + i \, \epsilon_{jil} \, L_{l} \, L_{j} \\[0.5em]
&= - i \, (\epsilon_{ijl} \, L_{j} \, L_{l} + i + \epsilon_{ijl} \, L_{l} \, L_{j})
\end{aligned}
\label{eq:ecuacion_10_107}
\end{align}
ahora, al renombrar los índices se encuentra que:
\begin{align}
\begin{aligned}[b]
\epsilon_{ijl} \, L_{j} \, L_{l} &= \nsum_{j=1}^{3} \nsum_{l=1}^{3} \epsilon_{ijl} \, L_{j} \, L_{l} = \\[0.5em]
&= \nsum_{l=1}^{3} \nsum_{j=1}^{3} \epsilon_{ilj} \, L_{l} \, L_{j} \\[0.5em]
&= \epsilon_{ilj} \, L_{l} \, L_{j}
\end{aligned}
\label{eq:ecuacion_10_108}
\end{align}
al ocupar esta igualdad en la ec. (\ref{eq:ecuacion_10_107}), se tiene que:
\begin{align}
\begin{aligned}
\comm{L^{2}}{L_{i}} &= - i \, (\epsilon_{ilj} \, L_{j} \, L_{l} + \epsilon_{ijl} \, L_{l} \, L_{j}) = \\[0.5em]
&= - i \, (- \epsilon_{ijl} \, L_{l} \, L_{j} + \epsilon_{ijl} \, L_{l} \, L_{j}) = \\[0.5em]
&= 0
\end{aligned}
\label{eq:ecuacion_10_109}
\end{align}
Por lo tanto, $L^{2}$ conmuta con cualquier $L_{i}$ . Al operador $L^{2}$ se le llama \emph{el Casimir} del álgebra de Lie de $SO(3)$. Como $L^{2}$ conmuta con cualquier $L_{i}$, este operador comparte vectores propios con estos tres operadores.

\subsection{Ecuación de valores propios de \texorpdfstring{$L^{2}$.}{L2}}

Para obtener los vectores y valores propios de $M_{i}$ y $M^{2}$ resolvimos un problema de álgebra lineal, mas para obtener los vectores propios de $L^{2}$ y por ejemplo, los de $L_{z}$, se deben plantear las ecuaciones:
\begin{align}
\begin{aligned}[b]
L^{2} \, Y_{\lambda m} &= \lambda \, Y_{\lambda m} \\[0.5em]
L_{z} \, Y_{\lambda m} &= m \, Y_{\lambda m}
\end{aligned}
\label{eq:ecuacion_10_110}
\end{align}
Estas son dos ecuaciones diferenciales. En efecto considerando las expresiones de $L^{2}$ y $L_{z}$ en coordenadas esféricas, se encuentra que:
\begin{align}
\begin{aligned}[b]
L^{2} \, Y_{\lambda m} (\theta, \varphi) &= - \left[ \dfrac{1}{\sin \theta} \, \pdv{\theta} \left( \sin \theta \, \pdv{Y_{lm} (\theta, \varphi)}{\theta} \right) + \dfrac{1}{\sin^{2} \theta} \pdv[2]{Y_{lm} (\theta, \varphi)}{\varphi}  \right] = \\[0.5em]
&= \lambda \, Y_{\lambda m} (\theta, \varphi)
\end{aligned}
\label{eq:ecuacion_10_111}
\end{align}
\begin{align}
L_{z} \, Y_{\lambda m} (\theta, \varphi) &= - i \, \pdv{Y_{\lambda m} (\theta, \varphi)}{\varphi} = m \, Y_{\lambda m} (\theta, \varphi) \label{eq:ecuacion_10_112}
\end{align}
De la segunda ecuación, es claro que $Y_{\lambda m} (\theta, \varphi)$ es de la forma:
\begin{align}
Y_{\lambda m} (\theta, \varphi) = \alpha_{\lambda m} \, e^{i m \varphi} \, P_{\lambda}^{m} (\theta)
\label{eq:ecuacion_10_113}
\end{align}
Si queremos que la función $Y_{\lambda m} (\theta, \varphi) $ no sea multivaluada, debemos pedir que $Y_{\lambda m} (\theta, \varphi + 2 \pi) = Y_{\lambda m} (\theta, \varphi)$. Esto implica $e^{i m \phi} = e^{i m (\phi + 2\pi})$, lo cual se cumple sólo si:
\begin{align}
m = 0, \pm 1, \pm 2, \pm 3, \ldots
\label{eq:ecuacion_10_114}
\end{align}
Por lo tanto, $m$ debe de ser un entero. Posteriormente veremos los posibles valores de $\lambda$.
\par
Sustituyendo la ec. (\ref{eq:ecuacion_10_113}) en la ec. (\ref{eq:ecuacion_10_111}), se encuentra que:
\begin{align}
\dfrac{1}{\sin \theta} \, \pdv{\theta} \left( \sin \theta \, \pdv{P_{\lambda}^{m} (\theta)}{\theta} \right) - \dfrac{m^{2}}{\sin^{2} \theta} \pdv[2]{P_{\lambda}^{m} (\theta)}{\theta} = - \lambda \, P_{\lambda}^{m} (\theta) \label{eq:ecuacion_10_115}
\end{align}
Con el cambio de variable $U = \cos \theta$, se tiene:
\begin{align}
\sin \theta = \sqrt{1 - u^{2}}, \hspace{1.5cm} \partial_{\theta} = - \sqrt{1 - u^{2}} \, \partial_{u}
\label{eq:ecuacion_10_116}
\end{align}
Por lo que la ec. (\ref{eq:ecuacion_10_115}) toma la forma:
\begin{align}
\dv{u} \left( (1 - u^{2}) \, \dv{P_{\lambda}^{m} (u)}{u} \right) + \left( \lambda - \dfrac{m^{2}}{1 - u^{2}} \right) \, P_{\lambda}^{m} (u) = 0
\label{eq:ecuacion_10_117}
\end{align}
Esta es la llamada \emph{ecuación asociada de Legendre}.
 Para el caso $m = 0$, se define $P_{\lambda}^{0} = P_{\lambda} (u)$, que debe satisfacer:
\begin{align}
\dv{u} \bigg[ (1 - u^{2}) \, \dv{P_{\lambda} (u)}{u} \bigg] + \lambda \, P_{\lambda} (u) = 0
\label{eq:ecuacion_10_118}
\end{align}
que es la llamada \emph{ecuación ordinaria de Legendre}. En lo que sigue, estudiando la estructura del grupo de rotaciones, obtendremos las soluciones de estas ecuaciones. Primero se verá la ortonormalidad de las soluciones de la ec. (\ref{eq:ecuacion_10_115}).

\subsection{Relaciones de ortonormalidad.}

Supongamos que $m$ y $m^{\prime}$ son enteros y que $m \neq m^{\prime}$, entonces
\begin{align}
\begin{aligned}[b]
\scaleint{6ex}_{\bs 0}^{2 \pi} \left( e^{i \, m^{\prime} \, \varphi} \right)^{*} \, e^{i \, m \, \varphi} \dd{\varphi} &= \scaleint{6ex}_{\bs 0}^{2 \pi} e^{-i \, m^{\prime} \, \varphi} \, e^{i \, m \, \varphi} \dd{\varphi} = \\[0.5em]
&= \scaleint{6ex}_{\bs 0}^{2 \pi} e^{-i \, (m - m^{\prime} )\, \varphi} \dd{\varphi} = \\[0.5em]
&= \dfrac{e^{-i \, (m - m^{\prime} )\, \varphi}}{i \, (m - m^{\prime})} \eval_{0}^{2 \pi} = \\[0.5em]
&= \dfrac{e^{-i \, (m - m^{\prime} )\, 2 \, \pi} - 1}{i \, (m - m^{\prime})} = 0 
\end{aligned}
\label{eq:ecuacion_10_119}
\end{align}
Si $m = m^{\prime}$, se encuentra que:
\begin{align}
\begin{aligned}[b]
\scaleint{6ex}_{\bs 0}^{2 \pi} \left( e^{i \, m^{\prime} \, \varphi} \right)^{*} \, e^{i \, m \, \varphi} \dd{\varphi} &= \scaleint{6ex}_{\bs 0}^{2 \pi} e^{-i \, m^{\prime} \, \varphi} \, e^{i \, m \, \varphi} \dd{\varphi} = \\[0.5em]
&= \scaleint{6ex}_{\bs 0}^{2 \pi} \dd{\varphi} = 2 \, \pi
\end{aligned}
\label{eq:ecuacion_10_120}
\end{align}
De donde, si $m$ y $m^{\prime}$ son enteros, se tiene que:
\begin{align}
\scaleint{6ex}_{\bs 0}^{2 \pi} \left( e^{i \, m^{\prime} \, \varphi} \right)^{*} \, e^{i \, m \, \varphi} \dd{\varphi} &= \scaleint{6ex}_{\bs 0}^{2 \pi} e^{-i \, m^{\prime} \, \varphi} \, e^{i \, m \, \varphi} \dd{\varphi} = 2 \, \pi \, \delta_{m m^{\prime}}
\label{eq:ecuacion_10_121}
\end{align}
Por lo tanto, las funciones $e^{i \, m \, \varphi}$ son ortogonales en el intervalo $[2, \pi]$.
\par
Nótese que la ec. (\ref{eq:ecuacion_10_115}) se puede escribir como:
\begin{align}
\pdv{\theta} \left( \sin \theta \pdv{P_{\lambda}^{m} (\theta)}{\theta} \right) + \left(\lambda \, \sin \theta - \dfrac{m^{2}}{\sin \theta} \right) \, P_{\lambda}^{m} (\theta) = 0
\label{eq:ecuacion_10_122}
\end{align}
Como se puede observar, ésta ecuación es de tipo Sturm-Liouville. En este caso:
\begin{align*}
p(\theta) = \sin \theta \, \hspace{1.5cm} r(\theta) = - \dfrac{m^{2}}{\sin \theta}
\end{align*}
Considerando los resultados para las ecuaciones tipo Sturm-Liouville, como $p(0) = p(\pi) = 0$, se llega a:
\begin{align*}
(\lambda^{\prime} - \lambda) \, \scaleint{6ex}_{\bs 0}^{\pi} \sin \theta \, P_{\lambda^{\prime}}^{m} (\theta) \, P_{\lambda}^{m} (\theta) \dd{\theta} = 0
\end{align*}
En particular, si $\lambda^{\prime} \neq \lambda$, se tiene:
\begin{align}
\scaleint{6ex}_{\bs 0}^{\pi} \sin \theta \, P_{\lambda^{\prime}}^{m} (\theta) \, P_{\lambda}^{m} (\theta) \dd{\theta} = 0
\label{eq:ecuacion_10_123}
\end{align}
En general el resultado es:
\begin{align}
\scaleint{6ex}_{\bs 0}^{\pi} \sin \theta \, P_{\lambda^{\prime}}^{m} (\theta) \, P_{\lambda}^{m} (\theta) \dd{\theta} = \delta_{\lambda \lambda^{\prime}} \, \beta_{\lambda m} \hspace{1.5cm} \beta_{\lambda m} = \mbox{ constante } > 0
\label{eq:ecuacion_10_124}
\end{align}
Por lo tanto, las funciones $P_{\lambda^{\prime}}^{m} (\theta)$ son ortogonales.
\par
Empleando la ortogonalidad de las funciones $e^{i \, m \, \varphi}$ y $P_{\lambda}^{m}$, se puede escoger $\alpha_{l m}$ de tal forma que los armónicos esféricos sean ortonormales. En efecto, como los armónicos esféricos tienen la forma ec. (\ref{eq:ecuacion_10_113}), se encuentra que:
\begin{align}
\begin{aligned}[b]
&\int Y_{\lambda^\prime m^{\prime}}^{*} (\theta, \varphi) \, Y_{\lambda m} (\theta, \varphi) \dd{\Omega} = \\[0.5em]
&= \int_{0}^{2 \pi} \int_{0}^{\pi} \left( \sin \theta \, \alpha_{\lambda^{\prime} m^{\prime}}^{*} \, e^{-i  \, m \, \varphi} \, P_{\lambda^{\prime}}^{m^{\prime}} (\cos \theta) \, \alpha_{\lambda m} \, e^{i  \, m \, \varphi} \, P_{\lambda}^{m} (\cos \theta) \right) \dd{\theta} \dd{\varphi} = \\[0.5em]
&= \alpha_{\lambda^{\prime} m^{\prime}}^{*} \, \alpha_{\lambda m} \, \int_{0}^{2 \pi} e^{i (m - m^{\prime} \varphi)} \dd{\varphi} \, \int_{0}^{\pi} \sin \theta \, P_{\lambda^{\prime}}^{m^{\prime}} (\cos \theta) \, P_{\lambda}^{m} (\cos \theta) \dd{\theta} = \\[0.5em]
&= \alpha_{\lambda^{\prime} m^{\prime}}^{*} \, \alpha_{\lambda m} \, 2 \, \pi \, \delta_{m m^{\prime}} \, \int_{0}^{\pi} \sin \theta \, P_{\lambda^{\prime}}^{m^{\prime}} (\cos \theta) \, P_{\lambda}^{m} (\cos \theta) \dd{\theta} = \\[0.5em]
&= \alpha_{\lambda^{\prime} m}^{*} \, \alpha_{\lambda m} \, 2 \, \pi \, \delta_{m m^{\prime}} \, \int_{0}^{\pi} \sin \theta \, P_{\lambda^{\prime}}^{m} (\cos \theta) \, P_{\lambda}^{m} (\cos \theta) \dd{\theta} = \\[0.5em]
&= \alpha_{\lambda^{\prime} m}^{*} \, \alpha_{\lambda m} \, 2 \, \pi \, \delta_{m m^{\prime}} \, \beta_{\lambda m} \, \delta_{\lambda^{\prime} \lambda} = \\[0.5em]
&= \abs{\alpha_{\lambda^{\prime} m}}^{2} \, 2 \, \pi \, \beta_{\lambda m} \, \delta_{m m^{\prime}} \,\delta_{\lambda^{\prime} \lambda}
\end{aligned}
\label{eq:ecuacion_10_125}
\end{align}
Entonces si:
\begin{align}
\abs{\alpha_{\lambda^{\prime} m}}^{2} = \dfrac{1}{2 \, \pi \, \beta_{\lambda m}}
\label{eq:ecuacion_10_126}
\end{align}
se cumple que:
\begin{align}
\braket{Y_{\lambda^{\prime} m^{\prime}}(\theta, \varphi)}{Y_{\lambda m}(\theta, \varphi)} = \scaleint{6ex} Y_{\lambda^\prime m^{\prime}}^{*} (\theta, \varphi) \, Y_{\lambda m} (\theta, \varphi) \dd{\Omega} = \delta_{m m^{\prime}} \, \delta_{\lambda^{\prime} \lambda}
\label{eq:ecuacion_10_127}
\end{align}
Así los armónicos esféricos son ortonormales.

\subsection{Operadores escalera y espectro de \texorpdfstring{$L^{2}$}{L2}.}

Antes de resolver la ED (\ref{eq:ecuacion_10_111}) veamos algunas propiedades de las funciones propias $Y_{\lambda m}$ y sus valores propios $\lambda, m$. Primero notemos que para cualquier función $f$, se tiene que:
\begin{align*}
0 \leq \braket{L_{x} \, f}{L_{x} \, f} = \braket{f}{L_{x}^{\dagger} \, L_{x} \, f} = \braket{f}{L_{x}^{2} \, f} \\[0.5em]
0 \leq \braket{L_{y} \, f}{L_{y} \, f} = \braket{f}{L_{y}^{\dagger} \, L_{y} \, f} = \braket{f}{L_{y}^{2} \, f}
\end{align*}
por lo que:
\begin{align*}
0 \leq \braket{f}{(L_{2}^{2} + L_{y}^{2}) \, f} = \braket{f}{(L^{2} - L_{z}^{2}) \, f}
\end{align*}
En particular, si $f = Y_{\lambda m}$, se llega a:
\begin{align}
(L^{2} - L_{z}^{2}) \, Y_{\lambda m} = (\lambda - m^{2}) \, Y_{\lambda m}
\label{eq:ecuacion_10_128}
\end{align}
entonces:
\begin{align}
0 \leq (\lambda - m^{2}) \, \braket{Y_{\lambda m}}{Y_{\lambda m}}
\label{eq:ecuacion_10_129}
\end{align}
Nótese que la única forma de que se cumpla la desigualdad $(\lambda - m^{2}) < 0$ es que $\braket{Y_{\lambda m}}{Y_{\lambda m}} = 0$, es decir, $Y_{\lambda m} = 0$. Si $Y_{\lambda m} \geq 0$, entonces:
\begin{align}
- \sqrt{\lambda} \leq m \leq \sqrt{\lambda}
\label{eq:ecuacion_10_130}
\end{align}                                                   
Esta restricción nos permitirá encontrar los valores de $\lambda$.
\par
Veamos ahora los operadores de ascenso y descenso\footnote{También conocidos como operadores de creación y aniquilamiento, respectivamente.}:
\begin{align}
L_{\pm} = L_{x} \pm i \, L_{y}
\label{eq:ecuacion_10_131}
\end{align}
Notemos que:
\begin{align*}
L_{\mp} \, L_{\pm} &= (L_{x} \mp i \, L_{y})(L_{x} \pm i \, L_{y}) = \\[0.5em]
&= L_{x}^{2} + L_{y}^{2} \pm i (L_{x} \,L_{y} - L_{y} \,L_{x}) = \\[0.5em]
&= L_{x}^{2} + L_{y}^{2} + L_{z}^{2} - L_{z}^{2} \mp L_{z} = \\[0.5em]
&= L^{2} \mp L_{z} = \\[0.5em]
&= L^{2} - (L_{z}^{2} \pm L_{z})
\end{align*}
es decir:
\begin{align}
L_{\mp} \, L_{\pm} = L^{2} - (L_{z}^{2} \pm L_{z})
\label{eq:ecuacion_10_132}
\end{align}
Claramente $L_{\pm}$ conmuta con $L^{2}$:
\begin{align}
\comm{L^{2}}{L_{\pm}} = 0
\label{eq:ecuacion_10_133}
\end{align}
Esto indica que $L_{\pm}$ tiene funciones propias comunes con $L^{2}$. Además:
\begin{align}
\begin{aligned}[b]
\comm{L_{z}}{L_{x} \pm i \, L_{y}} &= \comm{L_{z}}{L_{x}} \pm \comm{L_{z}}{L_{y}} = \\[0.5em]
&= i \, L_{y} \pm L_{z} \\[0.5em]
&= \pm (L_{x} \pm i \, L_{y}) 
\end{aligned}
\label{eq:ecuacion_10_134}
\end{align}
es decir:
\begin{align}
\comm{L_{z}}{L_{\pm}} = \pm L_{\pm}
\label{eq:ecuacion_10_135}
\end{align}
Por lo que $L_{\pm}$ no tiene funciones propias comunes con $L_{z}$.
\par
Veamos qué efecto tiene el operador $L_{pm}$ sobre las funciones propias de $L_{z}$. Definimos la función:
\begin{align}
\tilde{Y} = L_{\pm} \, Y_{\lambda m}
\label{eq:ecuacion_10_136}
\end{align}
entonces tomando en cuenta que $L^{2}$ y $L_{\pm}$ conmutan, se tiene que:
\begin{align}
L^{2} \, \tilde{Y} = L^{2} \, L_{\pm} \, Y_{\lambda m} = L_{\pm} \, L^{2} \, Y_{\lambda m} = \lambda \, L_{\pm} \, Y_{\lambda m} = \lambda \tilde{Y} 
\label{eq:ecuacion_10_137}
\end{align}
Por lo tanto, $L_{\pm} \, Y_{\lambda m}$ también es una función propia de $L^{2}$ con el mismo valor propio $\lambda$ que $Y_{\lambda m}$. Además considerando el conmutador de la ec. (\ref{eq:ecuacion_10_135}), se tiene que:
\begin{align}
\begin{aligned}
L_{z} \, \tilde{Y} &= L_{z} \, L_{\pm} \, Y_{\lambda m} = \\[0.5em]
&= \big( L_{\pm} \, L_{z} \pm L_{\pm} \big) \, Y_{\lambda m} = \\[0.5em]
&= \big( L_{\pm} \, L_{z} \, Y_{\lambda m} \pm L_{\pm}  \, Y_{\lambda m} \big) = \\[0.5em]
&= \big( m \, L_{\pm} \, Y_{\lambda m} \pm L_{\pm} \, Y_{\lambda m} \big) = \\[0.5em]
&= (m \pm 1) L_{\pm} \, Y_{\lambda m} = \\[0.5em]
&= (m \pm  1) \, \tilde{Y}
\end{aligned}
\label{eq:ecuacion_10_138}
\end{align}
es decir:
\begin{align}
L_{z} \, L_{\pm} \, Y_{\lambda m} = (m \pm 1) L_{\pm} \, Y_{\lambda m}
\label{eq:ecuacion_10_139}
\end{align}
Por lo tanto, $\tilde{Y} = L_{\pm} \, Y_{\lambda m}$ es vector propio de $L_{z}$, pero no con el valor propio de $Y_{\lambda m}$, sino con el de $Y_{\lambda m \pm 1}$. De donde, como $L_{z}$ tiene un espectro no degenerado, se debe cumplir:
\begin{align}
L_{\pm} \, Y_{\lambda m} = \alpha (\lambda, m) \, Y_{\lambda m \pm 1} \hspace{1.5cm} \alpha(\lambda, m) = \mbox{ constante}
\label{eqe:ecuacion_10_140}
\end{align}
El valor de $\alpha (\lambda , m)$ se obtendrá más adelante.
\par
Como se puede ver, el operador $L_{\pm}$ es de mucha utilidad, ya que nos permite pasar de una función propia $Y_{\lambda m}$ a otra $Y_{\lambda m \pm 1}$. Si se aplica $n$ veces este operador, se obtiene:
\begin{align}
\big( L_{\pm} \big)^{n} \, Y_{\lambda m} = \tilde{\alpha} (\lambda, m) \, Y_{\lambda m \pm + n}
\label{eq:ecuacion_10_141}
\end{align}
con:
\begin{align}
L_{z} \, Y_{\lambda m \pm n} = (m \pm n) \, Y_{\lambda m \pm n}
\label{eq:ecuacion_10_142}
\end{align}
Así, $Y_{\lambda m \pm n}$ es vector propio de $L_{z}$  con valor propio $(m \pm +n)$, sin importar el valor de $n$. Es claro que dado cualesquiera dos números $\lambda$ y $m$ existen otros dos números: $n_{a}$ y $n_{b}$, tales que:
\begin{align*}
\sqrt{\lambda} < (m + n_{a}), \hspace{1.5cm} (m - n_{b}) < - \sqrt{\lambda}
\end{align*}
En estos casos no se cumple la restricción dada por la ec. (\ref{eq:ecuacion_10_130}), por lo que $Y_{\lambda m + n_{a}} = 0$ y $Y_{\lambda m -n_{b}} = 0$. Esto nos dice que:
\begin{align}
\big( L_{+} \big)^{n_{a}} \, Y_{\lambda m} = 0 \hspace{1.5cm} \big( L_{-} \big)^{n_{b}} \, Y_{\lambda m} = 0
\label{eq:ecuacion_10_143}
\end{align}

Ahora definamos el conjunto:
\begin{align}
A_{+} = \left\{ n | \big( L_{+} \big)^{n}  \, Y_{\lambda m } = 0\right\}
\label{eq:ecuacion_10_144}
\end{align}
y sea $N_{+}$ el mínimo de $A_{+}$. Notemos que:
\begin{align}
\big( L_{+} \big)^{N_{+}-1} \, Y_{\lambda m} \neq 0
\label{eq:ecuacion_10_145}
\end{align}
esto es cierto, pues de lo contrario $N_{+}$ no sería mínimo de $A_{+}$. También definimos $l = N_{+} - 1 + m$, es decir, $l - m = N_{+} - 1$, se puede revisar que:
\begin{align*}
\big( L_{+} \big)^{N_{+}-1} \, Y_{\lambda m} = \big( L_{+} \big)^{l-m} \, Y_{\lambda m} = \gamma \, Y_{\lambda m} \neq 0, \hspace{1.5cm} \gamma = \mbox{ constante}
\end{align*}
Por lo tanto, existe $l$ tal que $Y_{\lambda m} \neq 0$ y
\begin{align*}
L_{+} \, Y_{\lambda m} = 0
\end{align*}
notemos que $l$ es el valor máximo que puede tomar $m$, de forma que $Y_{\lambda m} \neq 0$.
\par
Ahora definimos el conjunto:
\begin{align}
A_{-} = \left\{ n | \big( L_{-} \big)^{n}  \, Y_{\lambda m } = 0 \right\}
\label{eq:ecuacion_10_146}
\end{align}
y sea $N_{-}$ el mínimo de $A_{-}$. Se puede observar que:
\begin{align}
\big( L_{-} \big)^{N_{-}-1}  \, Y_{\lambda m } \neq 0
\end{align}
pues de lo contrario $N_{-}$ no sería el mínimo de $A_{-}$. También se define:
\begin{align}
\pderivada{l} = m - (N_{-} - 1)
\end{align}
es decir:
\begin{align}
N_{-} - 1 = \pderivada{l} - m
\end{align}
notando que:
\begin{align*}
\big( L_{-} \big)^{N_{-}-1} \, Y_{\lambda m} = \big( L_{-} \big)^{\pderivada{l}-m} \, Y_{\lambda m} = \pderivada{\gamma} \, Y_{\lambda \pderivada{l}} \neq 0, \hspace{1.5cm} \pderivada{\gamma} = \mbox{ constante}
\end{align*}
Por lo tanto, existe $\pderivada{l}$ tal que $Y_{\lambda \pderivada{l}} \neq 0$ y
\begin{align*}
L_{-} \, Y_{\lambda \pderivada{l}} = 0
\end{align*}
notemos que $\pderivada{l}$ es el valor mínimo que puede tomar $m$, de forma que $Y_{\lambda m} \neq 0$. De estos resultados se tiene que si $Y_{\lambda m} \neq 0$, entonces se cumple $\pderivada{l} \leq m \leq l$.
\par
Ahora, como $l$ es el valor propio máximo que puede tener $L_{z}$, entonces considerando la ec. (\ref{eq:ecuacion_10_132}), se tiene:
\begin{align}
L_{-} \, L_{+} \, Y_{\lambda m} = \big( L^{2} - (L_{z}^{2} + L_{z}) \big) \, Y_{\lambda m} = \big[ \lambda - l (l + 1) \big] \, Y_{\lambda m} = 0
\end{align}
esto implica que:
\begin{align*}
\lambda = l (l + 1)
\end{align*}
Además, como $\pderivada{l}$ es el valor propio mínimo que puede tener $L_{z}$, entonces:
\begin{align*}
L_{+} \, \big( L_{-} \, Y_{\lambda \pderivada{l}} \big) &= \big[ L^{2} - (L_{z}^{2} + L_{z}) \big] \, Y_{\lambda \pderivada{l}} = \\[0.5em]
&= \big[ l (l + 1) - \pderivada{l} (\pderivada{l} - 1) \big] \, Y_{\lambda \pderivada{l}} = \\[0.5em]
&= 0
\end{align*}
de donde:
\begin{align}
\begin{aligned}[b]
l (l + 1) - \pderivada{l} (\pderivada{l} - 1) &= l^{2} - l^{\prime \, 2} + l + \pderivada{l} = \\[0.5em]
&= (l + \pderivada{l})(l - \pderivada{l}) + l + \pderivada{l} = \\[0.5em]
&= (l + \pderivada{l})(l + 1 - \pderivada{l}) = \\[0.5em]
&= 0
\end{aligned}
\label{eq:ecuacion_10_148}
\end{align}
Entonces:
\begin{align}
\pderivada{l}_{+} = (l + 1) \hspace{0.5cm} \mbox{o} \hspace{0.5cm} \pderivada{l}_{-} = - l
\label{eq:ecuacion_10_149}
\end{align}
claramente el único valor permitido es $\pderivada{l}_{-} = - l$, y el valor mínimo que puede tomar $m$ es $-l$. Así, los valores propios de $L_{z}$ deben de cumplir con:
\begin{align}
-l \leq m \leq l \hspace{1.5cm} l = 0, \pm 1, \pm 2, \ldots, \pm l
\label{eq:ecuacion_10_150}
\end{align}

Notemos que hay $(2 \, l + 1)$ funciones propias con el mismo valor propio $l (l + 1)$. Es decir, que para cada valor propio $l (l + 1)$ hay $2 \, l + 1$ funciones que satisfacen:
\begin{align*}
L^{2} \, Y_{l m} (\theta, \varphi) = l (l + 1)\, Y_{l m} (\theta, \varphi)
\label{eq:ecuacion_10_151}
\end{align*}
con valores propios de $L_{z}$ que satisfacen la ec. (\ref{eq:ecuacion_10_150}).
\par
Un resultado importante de todo este desarrollo es que , dado un valor $l$, basto obtener un armónico esférico $Y_{lm} (\theta, \varphi)$ para que mediante los operadores $L_{\pm}$, se obtengan los $2 \, l$ restantes. Por ejemplo, podemos obtener el armónico esférico $Y_{l l} (\theta, \varphi)$ y con el operador $L_{-}$ obtener los restantes. De la misma forma, si conocemos la función $Y_{l 0}$, cualquier otro armónico esférico está dado por:
\begin{align}
\big( L_{-} \big)^{m} \, Y_{l 0} = \alpha \, Y_{l, -m}
\end{align}
o por:
\begin{align}
\big( L_{+} \big)^{m} \, Y_{l 0} = \alpha \, Y_{l, m}
\end{align}

\subsection{Constante \texorpdfstring{$\alpha$}{a}.}

Haremos la definición de la $\alpha$ de la ec. (\ref{eqe:ecuacion_10_140}). Como sabemos que los armónicos esféricos son ortonormales y que se cumple $L_{\pm} \, Y_{l m} = \alpha \, Y_{l m \pm 1}$, entonces usando las propiedades del producto escalar se tiene lo siguiente:
\begin{align}
\begin{aligned}[b]
\ip{L_{\pm} \, Y_{l m}}{L_{\pm} \, Y_{l m}} &= \ip{\alpha_{\pm} \, Y_{l m}}{\alpha_{\pm} \, Y_{l m}} =  \\[0.5em]
&= \alpha_{\pm}^{*} \, \alpha_{\pm} \, \ip{Y_{l m \pm 1}}{Y_{l m \pm 1}} = \\[0.5em]
&= \abs{\alpha_{\pm}}^{2}
\end{aligned}
\label{eq:ecuacion_10_154}
\end{align}
además:
\begin{align}
\begin{aligned}[b]
\ip{L_{\pm} \, Y_{l m}}{L_{\pm} \, Y_{l m}} &= \ip{Y_{l m}}{L_{\pm}^{\dagger} \, L_{\pm} \, Y_{l m}} = \\[0.5em]
&= \ip{Y_{l m}}{\big[ L^{2} - (L_{z}^{2}) \big] \, Y_{ l m}} = \\[0.5em]
&= \ip{Y_{l m}}{ \big[ l (l + 1) - (m^{2} \pm m) \big] \, Y_{l m}} = \\[0.5em]
&= \big( l^{2} + l - m ^{2} \mp m \big) \, \ip{Y_{l m}}{Y_{l m}} = \\[0.5em]
&= \big( l^{2} - m ^{2} + l \mp m \big) = \\[0.5em]
&= ( l \pm m)(l \mp m) + (l \mp m) = \\[0.5em]
&= (l \mp m)(l \pm m + 1)
\end{aligned}
\label{eq:ecuacion_10_155}
\end{align}
Igualando la ec. (\ref{eq:ecuacion_10_154}) con la ec. (\ref{eq:ecuacion_10_155}), se encuentra que:
\begin{align}
\abs{\alpha_{\pm}}^{2} = (l \mp m)(l \pm m + 1)
\label{eq:ecuacion_10_156}
\end{align}
por lo tanto:
\begin{align}
\alpha_{\pm} = \sqrt{(l \mp m)(l \pm m + 1)}
\label{eq:ecuacion_10_157}
\end{align}
Entonces la ec. (\ref{eqe:ecuacion_10_140}) tiene la forma:
\begin{align}
L_{\pm} \, Y_{l m} (\theta, \varphi) = \sqrt{(l \mp m)(l \pm m + 1)} \, Y_{l m \pm 1} (\theta, \varphi)
\label{eq:ecuacion_10_158}
\end{align}

\subsection{Reglas de recurrencia.}

Como vamos a aplicar varias veces el operador $L_{\pm}$, se revisarán algunas reglas de recurrencia de estos operadores.
\par
Ocuparemos los operadores $L_{x}$ y $L_{y}$ en coordenadas esféricas, así:
\begin{align}
\begin{aligned}[b]
&L_{\pm} = L_{x} \pm i \, L_{y} = \\[0.5em]
&= i \bigg( \sin \varphi \pdv{\theta} + \cot \theta \cos \varphi \pdv{\varphi} \bigg) \pm i (-i) \times \\[0.5em]
&\times \bigg( \cos \varphi \pdv{\theta} - \cot \theta \sin \varphi \pdv{\varphi} \bigg) = \\[0.5em]
&= \bigg[ \big( \pm \cos \varphi + i \sin \varphi \big) \pdv{\theta} + \cot \theta \big( i \cos \varphi \mp \sin \varphi \big) \pdv{\varphi} \bigg] = \\[0.5em]
&= \bigg[ \big( \pm \cos \varphi \pm i \sin \varphi \big) \pdv{\theta} + i \cot \theta \big( i \cos \varphi \pm \sin \varphi \big) \pdv{\varphi} \bigg] = \\[0.5em]
&= \bigg[ \pm e^{\pm i \varphi} \, \pdv{\theta} + i \, e^{\pm i \varphi} \, \cot \theta \, \pdv{\varphi} \bigg] = \\[0.5em]
&= e^{\pm i \varphi} \bigg[ \pm \pdv{\theta} + i \, \cot \theta \, \pdv{\varphi} \bigg]
\end{aligned}
\label{eq:ecuacion_10_159}
\end{align}
Veamos ahora que:
\begin{align*}
\pdv{\theta} \bigg[ (\sin \theta)^{\mp k} \, f(\theta) \bigg] &= \mp k \, (\sin \theta)^{\mp k-1} \, (\cos \theta) f(\theta) + (\sin \theta)^{\mp k} \, \pdv{f(\theta)}{\theta} = \\[0.5em]
&= (\sin \theta)^{\mp k} \bigg[ \pdv{f(\theta)}{\theta} \mp k (\cot \theta) \, f(\theta) \bigg]
\end{align*}
así entonces:
\begin{align*}
(\sin \theta)^{\mp k} \, \pdv{\theta} \bigg[ (\sin \theta)^{\mp k} \, f(\theta) \bigg] = \bigg[ \pdv{f(\theta)}{\theta} \mp k (\cot \theta) \, f(\theta) \bigg]
\end{align*}
Entonces, usando que:
\begin{align*}
(\sin \theta)^{1 \pm k} = (\sin \theta)^{\pm (k \pm 1)}
\end{align*}
y con el cambio de variable $u = \cos \theta$, se llega a:
\begin{align*}
-(\sin \theta)^{\mp k \pm 1} \, \pdv{u} \bigg[ (\sin \theta)^{\mp k} \, f(\theta) \bigg] = \bigg[ \pdv{f(\theta)}{\theta} \mp k (\cot \theta) \, f(\theta) \bigg]
\end{align*}
ocupando este resultado, se obtiene lo siguiente:
\begin{align*}
L_{\pm} \, \big( f(\theta) \, e^{i k \varphi} \big) &= e^{\pm i \varphi} \bigg[ \pm \pdv{\theta} + i (\cot \theta) \, \pdv{\varphi} \bigg] \, \big( f(\theta) \, e^{i k \varphi} \big) = \\[0.5em]
&= e^{\pm i \varphi} \bigg[ \pm \pdv{f(\theta}{\theta} - k (\cot \theta) \, f(\theta) \big] = \\[0.5em]
&= \pm e^{i (k \pm 1) \varphi} \bigg[ \pdv{f(\theta}{\theta} \mp k (\cot \theta) \, f(\theta) \big] = \\[0.5em]
&= \mp e^{i (k \pm 1) \varphi} \, (\sin \theta)^{\pm (k \pm 1)} \, \pdv{u} \bigg[ (\sin \theta)^{\mp l} \, f(\theta) \bigg]
\end{align*}
Adicionalmente, utilizando esta igualdad y definiendo:
\begin{align*}
\pderivada{k} = k \pm 1, \hspace{1.5cm} g(\theta) = (\sin \theta)^{\pm (k \pm 1)} \, \pdv{u} \bigg[ (\sin \theta)^{\mp k} \, f(\theta) \bigg]
\end{align*}
se encuentra que:
\begin{align*}
L_{\pm}^{2} \big( f(\theta) \, e^{i k \varphi} \big) &= (\mp) L_{\pm} \bigg[ e^{i \pderivada{k} \varphi} \, g(\theta) \bigg] = \\[0.5em]
&= (\mp)^{2} \, e^{i (\pderivada{k} \pm 1) \varphi} \, (\sin \theta)^{\pm (\pderivada{k} \pm 1)} \, \pdv{u} \bigg[ (\sin \theta)^{\mp \pderivada{k}} \, g(\theta) \bigg] = \\[0.5em]
&= (\mp)^{2} \, e^{i (k \pm 2) \varphi} \, (\sin \theta)^{\pm (k \pm 2)} \times \\[0.5em]
&\times \pdv{u} \bigg[ (\sin \theta)^{\mp (k \pm 1)} \, (\sin \theta)^{\pm (k \pm 1)} \, \pdv{u} \bigg( (\sin \theta)^{\mp k} \, f(\theta) \bigg) \bigg] = \\[0.5em]
&= (\mp)^{2} \, e^{i (k \pm 2) \varphi} \, (\sin \theta)^{\pm (k \pm 2)} \, \pdv[2]{u} \bigg[ (\sin \theta)^{\mp k} \, f(\theta) \bigg]
\end{align*}
en general se tiene que:
\begin{align}
\big( L_{\pm} \big)^{2} \, \big( f(\theta) \, e^{i k \varphi} \big) = (\mp)^{n} \, e^{i (k \pm n) \varphi} \, (\sin \theta)^{\pm (k \pm n)} \, \pdv[n]{u} \bigg[ (\sin \theta)^{\mp k} \, f(\theta) \bigg]
\label{eq:ecuacion_10_160}
\end{align}
donde:
\begin{align*}
u = \cos \theta
\end{align*}

\subsection{El armónico esférico \texorpdfstring{$Y_{ll} (\theta, \varphi)$}{Yll (t, f)}.}

Se obtendrá el armónico esférico $Y_{ll} (\theta, \varphi)$, con el cual, se podrá construir el resto de los armónicos esféricos.
\par
De la ec. (\ref{eq:ecuacion_10_113}) se sabe que:
\begin{align}
Y_{l l} (\theta, \varphi) = \dfrac{\alpha_{l l} \, e^{i l \varphi} \, \Theta_{l}^{l} (\theta)}{\sqrt{2}}
\label{eq:ecuacion_10_161}
\end{align}
y de la ec. (\ref{eq:ecuacion_10_158}), se debe de cumplir que $L_{+} \, Y_{l l} (\theta, \varphi) = 0$, es decir:
\begin{align}
L_{+} \, Y_{l l} (\theta, \varphi) = e^{i \varphi} \bigg( \pdv{\theta} + i \cot \theta \, \pdv{\varphi} \bigg) \, \dfrac{\alpha_{l l} \, e^{i l \varphi} \, \Theta_{l}^{l} (\theta)}{\sqrt{2}} = 0
\label{eq:ecuacion_10_162}
\end{align}
Por lo que $\Theta_{l}^{l} (\theta)$ debe de satisfacer:
\begin{align}
\bigg( \pdv{\theta} - l \cot \theta \bigg) \, \Theta_{l}^{l} (\theta) = 0
\label{eq:ecuacion_10_163}
\end{align}
así se tiene:
\begin{align}
\Theta_{l}^{l} (\theta) = \alpha \, \sin^{l} \theta, \hspace{1.5cm} \alpha = \mbox{ constante}
\label{eq:ecuacion_10_164}
\end{align}
Esto nos dice que:
\begin{align}
Y_{l l} (\theta, \varphi) = \dfrac{\alpha_{l l} \, e^{i l \varphi} \, \sin^{l} (\theta)}{\sqrt{2}}
\label{eq:ecuacion_10_165}
\end{align}
La constante $\alpha_{l l}$ se determina pidiendo la condición $\ip{Y_{l l}}{Y_{l l}} = 1$, que es:
\begin{align*}
\scaleint{6ex} Y_{ll}^{*} (\theta, \varphi) \, Y_{ll} (\theta, \varphi) \dd{\Omega} &= \scaleint{6ex}_{\bs 0}^{2 \pi} \scaleint{6ex}_{\bs 0}^{\pi} \sin \theta \, \bigg( \dfrac{\alpha_{l l} \, e^{i l \varphi} \, \sin^{l} (\theta)}{\sqrt{2}} \bigg)^{*} \times \\[0.5em]
&\times \bigg( \dfrac{\alpha_{l l} \, e^{i l \varphi} \, \sin^{l} (\theta)}{\sqrt{2}} \bigg) \dd{\theta} \dd{\varphi} = \\[0.5em]
&= \dfrac{\abs{\alpha_{ll}}^{2}}{2 \, \pi} \scaleint{6ex}_{\bs 0}^{2 \pi} \scaleint{6ex}_{\bs 0}^{\pi} \sin^{2l+1} \theta \, e^{i(l-l) \varphi} \dd{\theta} \dd{\varphi} = \\[0.5em]
&= \abs{\alpha_{ll}}^{2} \scaleint{6ex}_{\bs 0}^{\pi} \sin^{2l+1} \theta \dd{\theta} = \\[0.5em]
&= 1
\end{align*}
entonces:
\begin{align}
\abs{\alpha_{ll}}^{2} = \dfrac{1}{\displaystyle \scaleint{6ex}_{\bs 0}^{\pi} \sin^{2l+1} \theta \dd{\theta}}
\label{eq:ecuacion_10_166}
\end{align}
Para obtener esta constante, notemos que:
\begin{align*}
\dv{\theta} \big( \sin^{n-1} \theta \, \cos \theta \big) &= (n - 1) \, \sin^{n-2} \theta \, \cos^{2} \theta - \sin^{n-1} \theta \, \sin \theta = \\[0.5em]
&= (n - 1) \, \sin^{n-2} \theta \, \big( 1 - \sin^{2} \theta \big) - \sin^{n} \theta
\end{align*}
de donde:
\begin{align}
\sin^{n} \theta = \dfrac{n - 1}{n} \, \sin^{n-2} \theta - \dfrac{1}{n} \, \dv{\theta} \big( \sin^{n-1} \theta \, \cos \theta \big)
\label{eq:ecuacion_10_167}
\end{align}
Así se tiene que:
\begin{align}
\scaleint{6ex}_{\bs 0}^{\pi} \sin^{n} \theta \dd{\theta} = \dfrac{n - 1}{n} \scaleint{6ex}_{\bs 0}^{\pi} \sin^{n-2} \theta \dd{\theta}
\label{eq:ecuacion_10_168}
\end{align}
En particular:
\begin{align*}
\scaleint{6ex}_{\bs 0}^{\pi} \sin^{2l+1} \theta \dd{\theta} &= \dfrac{2 \, l}{2 \, l + 1} \scaleint{6ex}_{\bs 0}^{\pi} \sin^{2l+1} \theta \dd{\theta} = \\[0.5em]
&= \dfrac{2 \, l}{2 \, l + 1} \scaleint{6ex}_{\bs 0}^{\pi} \sin^{2(l-1)+1} \theta \dd{\theta} = \\[0.5em]
&= \dfrac{(2 \, l)(2 (l - 1))}{(2 \, l + 1)(2 (l - 1) + 1)} \scaleint{6ex}_{\bs 0}^{\pi} \sin^{2(l-2)+1} \theta \dd{\theta} = \\[0.5em]
&= \dfrac{(2 \, l)(2 (l - 1))(2 (l - 2))}{(2 \, l + 1)(2 (l - 1) + 1)(2(l - 2) + 1)} \scaleint{6ex}_{\bs 0}^{\pi} \sin^{2(l-3)+1} \theta \dd{\theta} = \\[0.5em]
&= \vdots \\[0.5em]
&= \dfrac{(2 \, l)(2 (l - 1))(2 (l - 2)) \cdots (2 (l - i))}{(2 \, l + 1)(2 (l - 1) + 1)(2(l - 2) + 1)(2(l-i) + 1)} \times \\[0.5em]
&\times \bigg( \scaleint{6ex}_{\bs 0}^{\pi} \sin^{2(l-(i+1))+1} \theta \dd{\theta} \bigg)
\end{align*}
Este proceso se puede hacer hasta que $2 (l - (i + 1)) + 1 = 1 $, lo que implica $l = i + 1$, es decir, $i = l - 1$. En este caso:
\begin{align*}
\scaleint{6ex}_{\bs 0}^{\pi} \sin^{2l+1} \theta \dd{\theta} &= \dfrac{( 2 \, l) (2 \, l - 1)(2 (l - 2)) \cdots 2}{(2 \, l + 1)( 2 \, l - 1)(2 \, l - 3) \cdots 3} \, \scaleint{6ex}_{\bs 0}^{\pi} \sin \theta \dd{\theta} = \\[0.5em]
&= \dfrac{ 2^{l} \, (l) (l - 1)(l - 2)) \cdots 1}{(2 \, l + 1)( 2 \, l - 1)(2 \, l - 3) \cdots 3} \, 2 = \\[0.5em]
&= \dfrac{2^{l+1} \, l!}{(2 \, l + 1)( 2 \, l - 1)(2 \, l - 3) \cdots 3}
\end{align*}
considerando que:
\begin{align}
\begin{aligned}[b]
(2 l {+} 1)( &2 l {-} 1)(2 l {-} 3) \cdots 3 = \\[0.5em]
&= \dfrac{(2 l {+} 1)(2 l) (2 l {-} 1)(2 (l {-} 2))(2 l {-} 3) \cdots 3 \cdot 2}{(2 l)(2 (l {-} 1)) \cdots 2} = \\[0.5em]
&= \dfrac{(2 \, l + 1)!}{2^{l} \, l!}
\end{aligned}
\label{eq:ecuacion_10_169}
\end{align}
se llega a:
\begin{align}
\scaleint{6ex}_{\bs 0}^{\pi} \sin^{2l+1} \theta \dd{\theta} = \dfrac{2^{2l+1} \, (l!)^{2}}{(2 \, l + 1)!}
\label{eq:ecuacion_10_170}
\end{align}
Tomando en cuenta este resultado en la ec. (\ref{eq:ecuacion_10_166}), se tiene lo siguiente:
\begin{align}
\alpha_{ll} = \sqrt{\dfrac{(2 l + 1)!}{2}} \, \dfrac{1}{2^{l} \, l!}
\label{eq:ecuacion_10_171}
\end{align}
entonces:
\begin{align}
Y_{ll} (\theta, \varphi) = \sqrt{\dfrac{(2 l + 1)!}{4 \, \pi}} \, \dfrac{e^{i l \varphi}}{2^{l} \, l!} \, \sin^{l} \theta
\label{eq:ecuacion_10_172}
\end{align}

\subsection{Forma explícita de los armónicos esféricos.}

Una vez obtenido un armónico esférico, ocupando el operador escalera $L_{\pm}$ se pueden obtener los demás. Por ejemplo, supongamos que se tiene el armónico esférico $Y_{ll}$, entonces al utilizar de manera reiterada la ec. (\ref{eq:ecuacion_10_158}), se tiene:
\begin{align}
\begin{aligned}[b]
L_{-} \, Y_{ll} &= \sqrt{2 \, l} \, Y_{ll-1} \\[0.5em]
L_{-}^{2} \, Y_{ll} &= \sqrt{2 \, l} \, L_{-} \, Y_{ll-1} = \sqrt{(2 l)(2 l - 1) 2} \, Y_{ll-2} = \sqrt{\dfrac{(2 l)! \, 2!}{(2 l - 2)!}} \, Y_{ll-2} \\[0.5em]
L_{-}^{3} \, Y_{ll} &= \sqrt{\dfrac{(2 \, l)! 2!}{(2 l - 2)!}} \, L_{-} \, Y_{ll-2} = \sqrt{\dfrac{(2 \, l)! 2!}{(2 l - 2)!}} \,\sqrt{(2 l)(2 l - 1) 3} \, Y_{ll-3} = \\[0.5em]
&= \sqrt{\dfrac{(2 l)! \, 3!}{(2 l - 3)!}} \, Y_{ll-3} \\[0.5em]
&\vdots \\[0.5em]
L_{-}^{n} \, Y_{ll} &= \sqrt{\dfrac{(2 l)! \, n!}{(2 l - n)!}} \, Y_{ll-n}
\end{aligned}
\label{eq:ecuacion_10_173}
\end{align}
Por lo tanto, si $n = l - m$, se encuentra que:
\begin{align}
Y_{lm} (\theta, \varphi) = \sqrt{\dfrac{(l + m)!}{(2 l)! (l - m)!}} \, L_{-}^{l-m} \, Y_{ll}(\theta, \varphi)
\label{eq:ecuacion_10_174}
\end{align}
Se puede revisar que con los resultados previos de las ecs. (\ref{eq:ecuacion_10_172}) y (\ref{eq:ecuacion_10_160}) para $k = l$ y $n = l - m$, se tiene lo siguiente:
\begin{align*}
Y_{lm}(\theta, \varphi) &= \sqrt{\dfrac{(l + m)!}{(2 l)! (l - m)!}} \, L_{-}^{l-m} \, \bigg( \sqrt{\dfrac{(2 l + 1)!}{4 \, \pi}} \, \dfrac{e^{i l \varphi}}{2^{l} \, l!} \, \sin^{l} \theta \bigg) = \\[0.5em]
&= \sqrt{\dfrac{(l + m)!}{(2 l)! (l - m)!}} \, \dfrac{1}{2^{l} \, l!} (L_{-})^{l-m} \, \big( e^{i l \varphi} \, \sin^{l} \theta \big) = \\[0.5em]
&= \sqrt{\dfrac{(l + m)!}{(2 l)! (l - m)!}} \, \dfrac{1}{2^{l} \, l!} \, (+)^{l-m} \, e^{i(l-(l-m))\varphi} \times \\[0.5em]
&\times \sin^{-[l-(l-m)]} \theta \, \pdv[l-m]{u} \big( \sin^{l} \theta \, \sin^{l} \theta \big) = \\[0.5em]
&= \sqrt{\dfrac{(2 l + 1)(l + m)!}{4 \pi (l -m)!}} \, \dfrac{e^{i m \varphi}}{2^{l} \, l!} \sin^{-m} \theta \, \pdv[l-m]{(\sin \theta)^{2l}}{u} 
\end{align*}
Notemos que ninguna propiedad de estas funciones cambia si las multiplicamos por un factor de norma uno, es decir por una potencia de $\pm i$ o de $\pm 1$. En la literatura existen diferentes elecciones de este factor, por conveniencia tomaremos $(-1)^{l}$. Entonces, los armónicos esféricos son:
\begin{align}
\begin{aligned}
Y_{lm} (\theta, \varphi) &= (-1)^{l} \, \sqrt{\dfrac{(2 l + 1)(l + m)!}{4 \pi (l -m)!}} \, \dfrac{e^{i m \varphi}}{2^{l} \, l!} \sin^{-m} \theta \, \pdv[l-m]{(\sin^{2} \theta)^{l}}{u} \\[0.5em]
&u = \cos \theta \hspace{1.5cm} -l \leq m \leq l
\end{aligned}
\label{eq:ecuacion_10_175}
\end{align}

Esta expresión de los armónicos esféricos es común en la mecánica cuántica.

\section{Ejercicios a cuenta.}

%Ref. Arfken (2006) 12.7.5
\noindent
Verifica mediante cálculo explícito que:
\\
\noindent
\textbf{Ejercicio a cuenta (37).}  $L_{+} \, Y_{1}^{0} (\theta, \varphi) = - \sqrt{\dfrac{3}{4 \pi}} \, \sin \theta \, e^{i \varphi} = \sqrt{2} \, Y_{1}^{1} (\theta, \varphi)$ 
\\[1em]
\noindent
\textbf{Ejercicio a cuenta (38).}  $L_{-} \, Y_{1}^{0} (\theta, \varphi) = + \sqrt{\dfrac{3}{4 \pi}} \, \sin \theta \, e^{-i \varphi} = \sqrt{2} \, Y_{1}^{-1} (\theta, \varphi)$ 
\\[1em]
Estos signos (fase de Condon-Shortley) son una consecuencia de los operadores sucesivos $L_{+}$ y $L_{-}$.

\end{document}