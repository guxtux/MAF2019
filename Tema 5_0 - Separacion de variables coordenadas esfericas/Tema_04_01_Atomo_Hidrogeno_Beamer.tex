\documentclass[12pt]{beamer}
\usepackage{../Estilos/BeamerMAF}
\usepackage{../Estilos/ColoresLatex}
\input{../Preambulos/preambulo_Beamer_Frankfurt_beaver}

\setbeamercolor{section in foot}{bg=deepcarmine, fg=white}
\setbeamercolor{subsection in foot}{bg=flame, fg=white}
\setbeamercolor{date in foot}{bg=blue, fg=white}

\makeatletter
\setbeamertemplate{footline}
{
\leavevmode%
\hbox{%
\begin{beamercolorbox}[wd=.333333\paperwidth,ht=2.25ex,dp=1ex,center]{section in foot}%
  \usebeamerfont{section in foot} \insertsection
\end{beamercolorbox}%
\begin{beamercolorbox}[wd=.333333\paperwidth,ht=2.25ex,dp=1ex,center]{subsection in foot}%
  \usebeamerfont{subsection in foot}  \insertsubsection
\end{beamercolorbox}%
\begin{beamercolorbox}[wd=.333333\paperwidth,ht=2.25ex,dp=1ex,right]{date in head/foot}%
  \usebeamerfont{date in head/foot} \insertshortdate{} \hspace*{1.5em}
  \insertframenumber{} / \inserttotalframenumber \hspace*{2ex} 
\end{beamercolorbox}}%
\vskip0pt%
}
\makeatother
\usefonttheme{serif}
\setbeamercolor{frametitle}{bg=lavenderblue}
\resetcounteronoverlays{saveenumi}

\date{21 de abril de 2022}

\title{\large{Tema 4 - El átomo de hidrógeno}}
\subtitle{Funciones Especiales I}
\author{M. en C. Gustavo Contreras Mayén}

\begin{document}
\maketitle
\fontsize{14}{14}\selectfont
\spanishdecimal{.}

\section*{Contenido}
\frame[allowframebreaks]{\tableofcontents[currentsection, hideallsubsections]}


\section{Partícula en un potencial central}
\frame{\tableofcontents[currentsection, hideothersubsections]}
\subsection{Ecuación de Schrödinger}

%Ref. Schaum's (1998) Quantum mechanics.

\begin{frame}
\frametitle{Definiendo el problema}
El Hamiltoniano de una partícula de masa $M$ dentro de un potencial central $V (r)$ es:
\pause
\begin{align}
H = \dfrac{\vb{p}^{2}}{2 M} + V (r) = - \dfrac{\hbar^{2}}{2 M} \, \laplacian + V (r)
\label{eq:ecuacion_08_01}
\end{align}
\end{frame}
\begin{frame}
\frametitle{El Laplaciano}
Donde el Laplaciano $\laplacian$ en coordenadas esféricas es:
\pause
\begin{align}
\laplacian = \dfrac{1}{r} \pdv[2]{r} + \dfrac{1}{r^{2}} \bigg[ \pdv[2]{\theta} + \dfrac{1}{\sin \theta} \, \pdv{\theta} + \dfrac{1}{\sin^{2} \theta} \, \pdv[2]{\phi} \bigg]
\label{eq:ecuacion_08_02}
\end{align}
\end{frame}
\begin{frame}
\frametitle{El momento angular}
Considerando que el momento angular se puede expresar como:
\pause
\begin{align}
\begin{aligned}
\hat{L}_{x} &=& - i \, \hbar \, \left( y \, \pdv{z} - z \, \pdv{y} \right) \\[0.5em] 
\hat{L}_{y} &=& - i \, \hbar \, \left( z \, \pdv{x} - x \, \pdv{z} \right) \\[0.5em] 
\hat{L}_{z} &=& - i \, \hbar \, \left( x \, \pdv{y} - y \, \pdv{x} \right)
\end{aligned}
\label{eq:ecuacion_01_03a}
\end{align}
\end{frame}
\begin{frame}
\frametitle{Reescribiendo el Hamiltoniano}
El Hamiltoniano $H$ lo escribimos como:
\pause
\begin{align}
H = - \dfrac{\hbar^{2}}{2 M} \, \dfrac{1}{r} \, \pdv[2]{r} + \dfrac{1}{2 M r^{2}} \, \vb{L}^{2} + V (r)
\label{eq:ecuacion_08_03}
\end{align}
\end{frame}

%Las tres componentes del operador $\vb{L}$ conmutan con $\vb{L}^{2}$, y



% El potencial en el átomo de hidrógeno es el potencial de interacción de tipo Coulomb entre el núcleo y el electrón.
% \par
% Este es un potencial radial, es decir, depende solamente de la distancia al núcleo $(r)$:
% \begin{align}
% V = V(r) = - \dfrac{k \, Z \, e^{2}}{r}
% \label{eq:ecuacion_01}
% \end{align}
% donde $Z$ el número atómico (en este caso $Z=1$), $e$ es la carga del electrón y $k$ es la constante de Coulomb.
% \par
% Por lo tanto el Hamiltoniano cuántico (el operador correspondiente a la energía total de sistema) se escribe como:
% \begin{align}
% H = - \dfrac{\hbar^{2}}{2 \, m} \, \laplacian + V(r)
% \label{eq:ecuacion_02} 
% \end{align}

% El sistema de coordenadas esféricas es el más adecuado para el problema: la ecuación de Schrödinger será más fácil de resolver en este sistema.
% \par
% Como ya sabemos expresar el Laplaciano en este sistema, haremos uso de esa expresión:
% \begin{align*}
% \laplacian = \dfrac{1}{r^{2}} \pdv{r} \left( r^{2} \pdv{\phi}{r} \right) {+} \dfrac{1}{r^{2} \sin \theta} \pdv{\theta} \left( \sin \theta \pdv{\phi}{\theta} \right) {+} \dfrac{1}{r^{2} \sin^{2} \theta} \pdv[2]{\phi}{\phi} 
% \end{align*}
% La expresión para el Laplaciano es complicada así que buscaremos una expresión más adecuada para resolver la ecuación de Schrödinger más fácilmente.

% \subsection{Momento angular.}

% La teoría del momento angular en mecánica cuántica es de gran importancia tanto por el número como por la variedad de sus consecuencias.
% \par
% A partir de la espectroscopía rotacional, que depende del momento angular de las moléculas, se consigue información acerca de las dimensiones y formas de moléculas.
% \par
% Utilizando los espectros de resonancia magnética nuclear y de resonancia paramagnética electrónica, cuyo origen es el momento angular de espín de núcleos y electrones, se consigue información sobre la estructura y configuración de moléculas.
% \par
% El momento angular orbital de los electrones en los átomos define las forma de los orbitales atómicos los cuales, a su vez, determinan la orientación de los enlaces y la estereoquímica de las moléculas. El momento angular de un sistema es muy importante, cuando \emph{es una constante de movimiento}, es decir, cuando se conserva, porque en este caso sirve para clasificar los niveles de energía del sistema.
% \par
% En mecánica cuántica los operadores de momento angular orbital son:
% \begin{align}
% \begin{aligned}
% \hat{L}_{x} &=& - i \, \hbar \, \left( y \, \pdv{z} - z \, \pdv{y} \right) \\[0.5em] 
% \hat{L}_{y} &=& - i \, \hbar \, \left( z \, \pdv{x} - x \, \pdv{z} \right) \\[0.5em] 
% \hat{L}_{z} &=& - i \, \hbar \, \left( x \, \pdv{y} - y \, \pdv{x} \right)
% \end{aligned}
% \label{eq:ecuacion_01_03a}
% \end{align}

% El cuadrado del operador momento angular es tal que:
% \begin{align}
% \hat{L}^{2} = \hat{L} \cdot \hat{L} = \hat{L}_{x}^{2} + \hat{L}_{y}^{2} + \hat{L}_{z}^{2}
% \label{eq:ecuacion_01_03b}
% \end{align}

% Para aplicar estos operadores sobre funciones del tipo $\psi(r, \theta, \phi)$ es necesario expresarlos en coordenadas polares. Utilizando las relaciones:
% \begin{align*}
% r^{2} &= x^{2} + y^{2} +z^{2} \\
% \cos \theta &= \dfrac{z}{\sqrt{x^{2} + y^{2} +z^{2}}} \\
% \tan \phi &= \dfrac{y}{x}
% \end{align*}
 
% Para luego aplicar las derivadas parciales $\pdv*{x}$, $\pdv*{y}$ y $\pdv*{z}$, se tiene:
% \begin{align}
% \begin{aligned}
% \hat{L}_{x} &= + i\, \hbar \, \left( \sin \phi \,\pdv{\theta} + \cot \theta\, \cos \phi \, \pdv{\phi} \right) \\[0.5em] 
% \hat{L}_{y} &= - i\, \hbar \, \left( \cos \phi \,\pdv{\theta} - \cot \theta\, \sin \phi \, \pdv{\phi} \right) \\[0.5em] 
% \hat{L}_{z} &= - i\, \hbar \, \pdv{\phi}
% \end{aligned}
% \label{eq:ecuacion_01_04a}
% \end{align}

% El cuadrado del operador momento angular es:
% \begin{align}
% \hat{L}^{2} = - \hbar^{2} \left( \dfrac{1}{\sin \theta} \pdv{\theta} \, \sin \theta \, \pdv{\theta} + \dfrac{1}{\sin^{2} \theta} \, \pdv[2]{\phi} \right)
% \label{eq:ecuacion_01_04b}
% \end{align}

% Es importante notar que solo se utiliza el operador $\hat{L}^{2}$ o sus componentes, pero nunca el operador $\hat{L}$ directamente, ya que el momento angular es un vector $\va{L}$ y no un escalar.

% \subsection{Constante de movimiento.}

% La condición para que el operador $\hat{O}$ represente una \emph{constante de movimiento} de un sistema es que se cumpla la relación:
% \begin{align}
% \hat{O} \, \hat{H} = \hat{H} \, \hat{O}
% \label{eq:ecuacion_01_05}
% \end{align}
% donde $\hat{H}$ es el Hamiltoniano del sistema.
% \par
% La relación anterior implica que el conmutador:
% \begin{align}
% [\hat{O}, \hat{H}] = \hat{O} \hat{H} - \hat{H} \, \hat{O}
% \label{eq:ecuacion_01_06}
% \end{align}
% vale cero.
% \par
% En efecto, cuando dos operadores conmutan, existe un conjunto de funciones que son funciones propias de los dos operadores simultáneamente. Es decir, que la misma función $\psi$ que caracteriza el estado del sistema con energía $E$:
% \begin{align*}
% \hat{H} \, \psi = E \, \psi
% \end{align*}
% también caracteriza el estado del sistema con propiedad $\hat{O}$ igual a $0$:
% \begin{align*}
% \hat{O} \, \psi = 0 \, \psi
% \end{align*}

% Dicho de otra manera, cuando el sistema se encuentra en el estado caracterizado por $\psi$, su energía es $E$ y su propiedad $\hat{O}$ es $o$. Ambos valores $E$ y $o$ son constantes mientras el sistema permanezca en el mismo estado $\psi$.
% \par
% En los casos en los que $\psi$ sea degenerada, siempre será posible construir una combinación lineal de las funciones propias correspondientes a $E$ tal que sea también función propia de $\hat{O}$.

\subsection{Reglas de conmutación}

\begin{frame}
\frametitle{Conmutación entre operadores}
Las reglas de conmutación entre los operadores de momento angular y sus componentes pueden ser deducidas fácilmente utilizando las expresiones en coordenadas cartesianas y algunas identidades de los conmutadores como:
\pause
\begin{eqnarray*}
\begin{aligned}
\comm{\hat{A} + \hat{B}}{\hat{C}} &= \comm{\hat{A}}{\hat{C}} + \comm{\hat{B}}{\hat{C}} \\[0.5em]
\comm{\hat{A}^{2}}{\hat{B}} &= \comm{\hat{A}}{\hat{B}} \, \hat{A} + \hat{A} \, \comm{\hat{A}}{\hat{B}}
\end{aligned}
\end{eqnarray*}
\end{frame}
\begin{frame}
\frametitle{Conmutación entre componentes del momento}
Se cumple entonces que:
\pause
\begin{eqnarray}
\begin{aligned}
\comm{L_{x}}{L_{y}} &= i \, \hbar \, L_{z} \\[0.5em] \pause
\comm{L_{y}}{L_{z}} &= i \, \hbar \, L_{x} \\[0.5em] \pause
\comm{L_{z}}{L_{x}} &= i \, \hbar \, L_{y} \\[0.5em] \pause
\comm{\vb{L}^{2}}{L_{x}} = \comm{\vb{L}^{2}}{L_{y}} &= \comm{\vb{L}^{2}}{L_{z}} = 0
\end{aligned}
\label{eq:ecuacion_01_07}
\end{eqnarray}
\end{frame}
\begin{frame}
\frametitle{Resultado de la conmutación}
Entonces: \pause
\setbeamercolor{item projected}{bg=lava,fg=white}
\setbeamertemplate{enumerate items}{%
\usebeamercolor[bg]{item projected}%
\raisebox{1.5pt}{\colorbox{bg}{\color{fg}\footnotesize\insertenumlabel}}%
}
\begin{enumerate}[<+->]
\item $\vb{L}^{2}$ conmuta con cualquiera de sus componentes.
\item Pero las componentes no conmutan entre sí.
\end{enumerate}
\end{frame}
\begin{frame}
\frametitle{Conmutación entre momento y el Hamiltoniano}
Las propiedades de conmutación entre los operadores de momento angular orbital y el Hamiltoniano dependen del sistema y deben ser determinadas para cada problema.
\end{frame}
\begin{frame}
\frametitle{Caso particular}
Frecuentemente $\vb{L}^{2}$ y $L_{z}$ conmutan con $H$ \pause y en estos casos el módulo del momento angular y la componente sobre el eje $z$ del momento angular son constantes de movimiento.
\end{frame}
% \par
% Frecuentemente $\hat{L}^{2}$ y $\hat{L}_{z}$ conmutan con $\hat{H}$ y en estos casos el módulo del momento angular y la componente sobre el eje $z$ del momento angular son constantes de movimiento.
% \par
% Por ejemplo, en el caso de átomos hidrogenoides $\hat{H}$ y $\hat{L}_{z}$ conmutan, donde:
% \begin{align*}
% \hat{H} &= - \dfrac{\hbar^{2}}{2 \mu} \left[ \dfrac{1}{r^{2}} \pdv{r} \left( r^{2} \pdv{\phi}{r} \right) {+} \dfrac{1}{r^{2} \sin \theta} \pdv{\theta} \left( \sin \theta \pdv{\phi}{\theta} \right) {+} \right. \\[0.5em]
% &+ \left. \dfrac{1}{r^{2} \sin^{2} \theta} \pdv[2]{\phi}{\phi} \right] - \dfrac{Z \, e^{2}}{r} \\[1em]
% \hat{L}_{z} &= - i \, \hbar \, \pdv{\phi}
% \end{align*}

% Entonces:
% \begin{align*}
% \hat{H} \cdot \hat{L}_{z} &= + \dfrac{i \, \hbar^{3}}{2 \mu} \left\{ \left[ \dfrac{1}{r^{2}} \pdv{r} \left( r^{2} \pdv{\phi}{r} \right) {+} \dfrac{1}{r^{2} \sin \theta} \pdv{\theta} \left( \sin \theta \pdv{\phi}{\theta} \right) {+} \right. \right. \\[0.5em]
% &+ \left. \left. \dfrac{1}{r^{2} \sin^{2} \theta} \pdv[2]{\phi}{\phi} + \dfrac{Z \, e^{2}}{r} \right] \pdv{\phi} + \dfrac{1}{r^{2} \sin^{2} \theta} \pdv[3]{\phi}{\phi} \right\}
% \end{align*}

% Mientras que:
% \begin{align*}
% \hat{L}_{z} \cdot \hat{H}  &= + \dfrac{i \, \hbar^{3}}{2 \mu} \left\{ \pdv{\phi} \left[ \dfrac{1}{r^{2}} \pdv{r} \left( r^{2} \pdv{\phi}{r} \right) {+} \right. \right. \\[0.5em]
% &+ \dfrac{1}{r^{2} \sin \theta} \pdv{\theta} \left( \sin \theta \pdv{\phi}{\theta} \right) {+} \\[0.5em]
% &+ \left. \left. \dfrac{1}{r^{2} \sin^{2} \theta} \pdv[2]{\phi}{\phi} + \dfrac{Z \, e^{2}}{r} \right] + \dfrac{1}{r^{2} \sin^{2} \theta} \pdv[3]{\phi}{\phi} \right\}
% \end{align*}

% Sabiendo que:
% \begin{align*}
% \pdv{\phi} \, \pdv{r} = \pdv{r} \, \pdv{\phi} \\[0.5em]
% \pdv{\phi} \, \pdv{\theta} = \pdv{\theta} \, \pdv{\phi}
% \end{align*}

% Es decir, las dos expresiones son iguales, por lo que:
% \begin{align*}
% \big[ \hat{H}, \hat{L}_{z} \big] = \hat{H} \, \hat{L}_{z} - \hat{L}_{z} \, \hat{H} = 0
% \end{align*}

% En coordenadas cartesianas $\hat{L}^{2}$ depende de tres coordenadas $(x, y, z)$; en coordenadas esféricas, $\hat{L}^{2}$ depende solo de dos $(\theta, \phi)$.
% \par
% En coordenadas cartesianas una de las variables no es independiente; en coordenadas esféricas, $\hat{L}^{2}$ solo depende de los ángulos, y no de la distancia $r$.
% \par
% Los observables correspondientes a los operadores $\hat{L}_{x}$, $\hat{L}_{y}$ y $\hat{L}_{z}$, son totalmente equivalentes, lo único que cambia es su orientación con respecto al sistema de referencia. Por esta razón siempre se usa $\hat{L}_{z}$, ya que la expresión matemática de su operador es mucho más simple, depende de solo un ángulos.
% \par
% Nos apoyaremos en un resultado de la teoría de los operadores y conmutadores: : Si $\hat{A}$ y $\hat{B}$ conmutan, es decir, si  $[\hat{A}, \hat{B}] = 0$, entonces existe una solución común $\psi$  para el par de ecuaciones diferenciales correspondientes a las ecuaciones de valores propios de estos operadores, siendo $\psi$ la función propia mientras que $a$ y $b$ son los valores propios correspondientes:
% \begin{align*}
% \hat{A} \, \psi &= a \, \psi \\[0.5em]
% \hat{B} \, \psi &= b \, \psi
% \end{align*}

% Ahora bien, utilizando ese resultado y el hecho de que $[\hat{L}^{2}, \hat{L}_{z}] = 0$ podemos buscar una solución común, que escribimos como $Y(\theta, \phi)$, al par de las ecuaciones diferenciales:
% \begin{align*}
% \hat{L}_{z} \, Y(\theta, \phi) &= b \, Y(\theta, \phi) \\[0.5em]
% \hat{L}^{2} \, Y(\theta, \phi) &= c \, Y(\theta, \phi)
% \end{align*}



%Ref. Schaum's
\begin{frame}
\frametitle{Ecuación diferencial}
Tendremos la siguiente ecuación de eigenvalores:
\pause
\begin{align}
\begin{aligned}
\bigg[ - \dfrac{\hbar^{2}}{2 \mu} \, \dfrac{1}{r} \, \pdv[2]{r} (r) &+ \dfrac{\vb{L}^{2}}{2 \mu r^{2}} + V (r) \bigg] \, \psi (r, \theta, \phi) = \\[0.5em]
&= E \, \psi (r, \theta, \phi) 
\end{aligned}
\label{eq:ecuacion_08_01_02}
\end{align}
\end{frame}
\begin{frame}
\frametitle{Resultado de la conmutación}
Ya se revisó que las tres componentes de $\vb{L}$ conmutan con $\vb{L}^{2}$, por lo que también conmutan con $H$:
\pause
\begin{align}
\comm{H}{L_{x}} = \comm{H}{L_{y}} = \comm{H}{L_{z}} = 0
\label{eq:ecuacion_08_04}
\end{align}
\end{frame}
\begin{frame}
\frametitle{Observables en el átomo de hidrógeno}
Los tres observables $H$, $\vb{L}^{2}$ y $L_{z}$ conmutan, por lo que podemos buscar funciones $\psi (r, \theta, \phi)$, que también sean eigenfunciones de $\vb{L}^{2}$ y $L_{z}$.
\end{frame}
\begin{frame}
\frametitle{Componente en la dirección $z$ de $L$}
La componente $L_{z}$ del momento angular es:
\pause
\begin{align*}
L_{z} = - i \, \hbar \, \pdv{\phi}
\end{align*}
\end{frame}
\begin{frame}
\frametitle{Sistema de ecuaciones diferenciales de eigenvalores}
Ahora podremos resolver el siguiente sistema de ecuaciones diferenciales de eigenvalores:
\begin{eqnarray}
\begin{aligned}
H \, \psi (r, \theta, \phi) &= E \, \psi (r, \theta, \phi) \label{eq:ecuacion_08_05} \\[0.5em] \pause
\vb{L}^{2} \, \psi (r, \theta, \phi) &= \ell (\ell + 1) \, \hbar^{2} \, \psi (r, \theta, \phi) \label{eq:ecuacion_08_06} \\[0.5em] \pause
L_{z} \, \psi (r, \theta, \phi) &= m \, \hbar \, \psi (r, \theta, \phi) \label{eq:ecuacion_08_07}
\end{aligned}
\end{eqnarray}
\pause
y determinar aquellos estados que son eigenfunciones de $H$, $\vb{L}^{2}$ y $L_{z}$.
\end{frame}
\begin{frame}
\frametitle{Separación de variables}
Con la técnica de separación de variables, tenemos que una solución estaría dada por el producto de:
\pause
\begin{align}
\psi (r, \theta, \phi) = R (r) \, Y (\theta, \phi)
\label{eq:ecuacion_08_08}
\end{align}
\end{frame}


%Ref. Ghatak (2004) 9.3
\subsection{Problema de eigenvalores}

\begin{frame}
\frametitle{Problema de eigenvalores}
Sin pérdida de generalidad, podemos expresar nuestro problema de valores propios para $\vb{L}^{2}$ como:
\pause
\begin{align}
\vb{L}^{2} \, Y(\theta, \phi) = \lambda \, \hbar^{2} \, Y(\theta, \phi)
\label{eq:ecuacion_027}
\end{align}
\pause
donde $\lambda \, \hbar^{2}$ representan los eigenvalores de $\vb{L}^{2}$, \pause y $Y(\theta, \phi)$ corresponde a las eigenfunciones. 
\end{frame}
\begin{frame}
\frametitle{Los eigenvalores y eigenfunciones}
Veremos que:
\pause
\setbeamercolor{item projected}{bg=lava,fg=white}
\setbeamertemplate{enumerate items}{%
\usebeamercolor[bg]{item projected}%
\raisebox{1.5pt}{\colorbox{bg}{\color{fg}\footnotesize\insertenumlabel}}%
}
\begin{enumerate}[<+->]
\item $\lambda$ toma valores $\ell (\ell + 1)$ con $\ell = 0, 1, 2, \ldots$.
\item Las correspondientes eigenfunciones son los \textbf{\textcolor{cadetblue}{armónicos esféricos}}.
\end{enumerate}
\end{frame}
\begin{frame}
\frametitle{De los eigenvalores}
Para cada eigenvalor de $\ell$, habrá un orden $(2 \, \ell + 1)$ de degeneración, \pause es decir, habrá $(2 \, \ell + 1)$ eigenfunciones que corresponden al mismo eigenvalor $\ell (\ell + 1) \, \hbar^{2}$.
\end{frame}
\begin{frame}
\frametitle{Ocupando el operador momento angular}
El operador $\vb{L}^{2}$ lo sustituimos en la ec. (\ref{eq:ecuacion_027}), así que:
\pause
\begin{align}
\begin{aligned}
\dfrac{1}{\sin \theta} \pdv{\theta} \bigg[ \sin \theta \, \pdv{Y}{\theta} \bigg] &+ \dfrac{1}{\sin^{2} \theta} \, \pdv[2]{Y}{\phi} + \\[0.5em]
&+ \lambda \, Y(\theta, \phi) = 0
\end{aligned}
\end{align}
\end{frame}
\begin{frame}
\frametitle{Resolviendo la ecuación}
Para resolver esta ecuación, usamos la técnica de separación de variables.
\\
\bigskip
\pause
Proponemos una solución de la forma:
\pause
\begin{align}
Y(\theta, \phi) = \Theta(\theta) \, \Phi(\phi)
\label{eq:ecuacion_029}
\end{align}
\end{frame}
\begin{frame}
\frametitle{Avanzando en la técnica}
Que sustituimos en la expresión anterior, para luego multiplicar por:
\pause
\begin{align*}
\dfrac{\sin^{2} \theta}{Y(\theta, \phi)}
\end{align*}
\end{frame}
\begin{frame}
\frametitle{Resultado obtenido}
Entonces obtendremos:
\pause
\begin{align}
\begin{aligned}
\dfrac{\sin^{2} \theta}{\Theta} &\left[ \dfrac{1}{\sin \theta} \pdv{\theta} \, \sin \theta \, \pdv{\Theta}{\theta} + \lambda \, \Theta (\theta) \right] = \\[0.5em]
&= - \dfrac{1}{\Phi} \, \dv[2]{\Phi}{\phi} = m^{2}
\end{aligned}
\label{eq:ecuacion_030}
\end{align}
\end{frame}
\begin{frame}
\frametitle{Las variables separadas}
De hecho, las variables se han separado y hemos establecido cada lado igual a una constante positiva $m^{2}$, cuya razón quedará clara en breve.
\end{frame}
\begin{frame}
\frametitle{Ecuación resultante}
La ec. (\ref{eq:ecuacion_030}) nos da:
\pause
\begin{align*}
\dv[2]{\Phi}{\phi} + m^{2} \Phi (\phi) = 0
\end{align*}
\end{frame}
\begin{frame}
\frametitle{Solución a la ED}
Cuya solución está dada por:
\pause
\begin{align*}
\Phi(\phi) \sim e^{i m \phi}
\end{align*}
\end{frame}
\begin{frame}
\frametitle{Función univaluada}
Para que la función de onda sea univaluada, debe de ocurrir que:
\pause
\begin{align}
\Phi(\phi +  2 \, \pi) = \Phi(\phi)
\label{eq:ecuacion_031}
\end{align}
\end{frame}
\begin{frame}
\frametitle{De manera equivalente}
Que es equivalente a:
\pause
\begin{align*}
e^{2 \pi m i} = 1
\end{align*}
\end{frame}
\begin{frame}
\frametitle{Valor de la constante}
Obteniendo entonces que:
\pause
\begin{align*}
m = 0, \pm 1, \pm 2, \ldots
\end{align*}
\end{frame}
\begin{frame}
\frametitle{Elección de la constante}
En este paso se justifica que no podríamos haber establecido una constante positiva (o compleja) porque entonces la función de onda no habría sido de un solo valor.
\end{frame}
\begin{frame}
\frametitle{Las eigenfunciones}
Al identificar las funciones con un subíndice $m$, tenemos:
\pause
\begin{align}
\Phi_{m}(\phi) = \dfrac{1}{\sqrt{2 \, \pi}} \, e^{i m \phi} \hspace{1cm} m = \pm 1, \pm 2, \ldots
\label{eq:ecuacion_032}
\end{align}
\end{frame}
\begin{frame}
\frametitle{Relevancia del factor}
Donde el factor $\dfrac{1}{\sqrt{2 \, \pi}}$ asegura que:
\pause
\begin{align*}
\scaleint{6ex}_{\bs 0}^{2 \pi} \abs{\Phi_{m}(\phi)}^{2} \dd{\phi} = 1
\end{align*}
que es la condición de normalización.
\end{frame}
\begin{frame}
\frametitle{Condición de ortonormalización}
Entonces se tendrá que:
\pause
\begin{align}
\scaleint{6ex}_{\bs 0}^{2 \pi} \Phi_{\ptilde{m}}^{*}(\phi) \, \Phi_{m}(\phi) \dd{\phi} = \delta_{m \ptilde{m}}
\label{eq:ecuacion_033}
\end{align}
representa la condición de ortonormalización para $\Phi_{m}(\phi)$.
\end{frame}
\begin{frame}
\frametitle{Segunda variable}
Para la segunda ecuación $\Theta (\theta)$ (ec. \ref{eq:ecuacion_030}), tendremos que:
\pause
\begin{align}
\dfrac{1}{\sin \theta} \dv{\theta} \left( \sin \theta \, \dv{\Theta}{\theta} \right) + \left( \lambda - \dfrac{m^{2}}{\sin^{2} \theta} \right) \, \Theta (\theta) = 0
\label{eq:ecuacion_034}
\end{align}
\end{frame}
\begin{frame}
\frametitle{Cambiando la variable}
Hacemos el siguiente cambio de variable: $\cos \theta = \mu$ y $\Theta(\theta) = F(\mu)$, para obtener:
\pause
\begin{align}
\dv{\mu} \left[ (1 - \mu^{2}) \, \dv{F}{\mu} \right] + \left[ \lambda - \dfrac{m^{2}}{1 - \mu^{2}} \right] \, F(\mu) = 0
\label{eq:ecuacion_035}
\end{align}
\end{frame}
\begin{frame}
\frametitle{Resultado que depende de $m$}
Hay que considerar dos casos: $m = 0$ y $m \neq 0$.
\pause
\setbeamercolor{item projected}{bg=lava,fg=white}
\setbeamertemplate{enumerate items}{%
\usebeamercolor[bg]{item projected}%
\raisebox{1.5pt}{\colorbox{bg}{\color{fg}\footnotesize\insertenumlabel}}%
}
\begin{enumerate}[<+->]
\item Con $m = 0$, la ec. (\ref{eq:ecuacion_035}) se reduce a:
\pause
\begin{align}
(1 - \mu^{2}) \, \dv[2]{F}{\mu} - 2  \, \mu \, \dv{F}{\mu} + \lambda \, F(\mu) = 0
\label{eq:ecuacion_036}
\end{align}
\seti
\end{enumerate}
\end{frame}
\begin{frame}
\frametitle{Solución a la ED}
Las soluciones a la ec. (\ref{eq:ecuacion_036}) se conocerán como los \textbf{\textcolor{blue(munsell)}{polinomios ordinarios de Legendre}} $P_{n} (x)$.
\\
\bigskip
\pause
Las propiedades de esta función especial se revisarán en esta parte del Tema 4.
\end{frame}
\begin{frame}
\frametitle{El otro caso de $m$}
\setbeamercolor{item projected}{bg=lava,fg=white}
\setbeamertemplate{enumerate items}{%
\usebeamercolor[bg]{item projected}%
\raisebox{1.5pt}{\colorbox{bg}{\color{fg}\footnotesize\insertenumlabel}}%
}
\begin{enumerate}[<+->]
\conti
\item con $m \neq 0$: \pause El método de \enquote{fuerza bruta} para obtener $Y_{\ell m} (\theta, \phi)$ es resolver directamente la ec. (\ref{eq:ecuacion_035}).
\\
\bigskip
\pause
La solución corresponde a la función especial: \textbf{\textcolor{carmine}{polinomios asociados de Legendre}}: $P_{l}^{m} (x)$.
\end{enumerate}
\end{frame}
\begin{frame}
\frametitle{Método ocupando el momento angular}
Sin embargo, \pause la forma más sencilla y elegante de obtener las soluciones es mediante el uso de los \emph{\textcolor{darkorchid}{operadores de escalera}} del momento angular, que también revisaremos esa solución.
\end{frame}
\begin{frame}
\frametitle{Solución inicial}
Hemos llegado a plantear una ecuación diferencial para la parte angular, \pause nos falta considerar la parte radial.
\end{frame}
\begin{frame}
\frametitle{Otro par de funciones especiales}
De  la ecuación radial, se tendrán como solución:
\setbeamercolor{item projected}{bg=lava,fg=white}
\setbeamertemplate{enumerate items}{%
\usebeamercolor[bg]{item projected}%
\raisebox{1.5pt}{\colorbox{bg}{\color{fg}\footnotesize\insertenumlabel}}%
}
\begin{enumerate}[<+->]
\item \textbf{\textcolor{bole}{Polinomios asociados de Laguerre $L_{n}^{k} (r)$}}.
\item Como caso especial, los \textbf{\textcolor{darkred}{polinomios ordinarios de Laguerre $L_{n} (r)$}}.
\end{enumerate}
\end{frame}
\begin{frame}
\frametitle{Solución completa}
Para la solución completa de cualquier problema de una partícula con un potencial radial, se tiene que la función de onda es un producto de un factor radial y un armónico esférico:
\pause
\begin{align*}
\psi_{n \ell m} (r, \theta, \phi) = R_{n \ell} (r) \, Y_{\ell m} (\theta, \phi)
\end{align*}
\end{frame}
\end{document}