\documentclass[12pt]{beamer}
\usepackage{../Estilos/BeamerMAF}
%Sección para el tema de beamer, con el theme, usercolortheme y sección de footers
\usetheme{Frankfurt}
\usecolortheme{beaver}
%\useoutertheme{default}
\setbeamercovered{invisible}
% or whatever (possibly just delete it)
\setbeamertemplate{section in toc}[sections numbered]
\setbeamertemplate{subsection in toc}[subsections numbered]
\setbeamertemplate{subsection in toc}{\leavevmode\leftskip=3.2em\rlap{\hskip-2em\inserttocsectionnumber.\inserttocsubsectionnumber}\inserttocsubsection\par}
% \setbeamercolor{section in toc}{fg=blue}
% \setbeamercolor{subsection in toc}{fg=blue}
% \setbeamercolor{frametitle}{fg=blue}
\setbeamertemplate{caption}[numbered]

\setbeamertemplate{footline}
\beamertemplatenavigationsymbolsempty
\setbeamertemplate{headline}{}


\makeatletter
% \setbeamercolor{section in foot}{bg=gray!30, fg=black!90!orange}
% \setbeamercolor{subsection in foot}{bg=blue!30!yellow, fg=red}
% \setbeamercolor{date in foot}{bg=black, fg=white}
\setbeamertemplate{footline}
{
  \leavevmode%
  \hbox{%
  \begin{beamercolorbox}[wd=.333333\paperwidth,ht=2.25ex,dp=1ex,center]{section in foot}%
    \usebeamerfont{section in foot} \insertsection
  \end{beamercolorbox}%
  \begin{beamercolorbox}[wd=.333333\paperwidth,ht=2.25ex,dp=1ex,center]{subsection in foot}%
    \usebeamerfont{subsection in foot}  \insertsubsection
  \end{beamercolorbox}%
  \begin{beamercolorbox}[wd=.333333\paperwidth,ht=2.25ex,dp=1ex,right]{date in head/foot}%
    \usebeamerfont{date in head/foot} \insertshortdate{} \hspace*{2em}
    \insertframenumber{} / \inserttotalframenumber \hspace*{2ex} 
  \end{beamercolorbox}}%
  \vskip0pt%
}







\date{16 de noviembre de 2021}

\title{\large{Tema 4 - Sep. de variables en coord. esféricas}}
\subtitle{Funciones Especiales I}
\author{M. en C. Gustavo Contreras Mayén}

\begin{document}
\maketitle
\fontsize{14}{14}\selectfont
\spanishdecimal{.}

\section*{Contenido}
\frame[allowframebreaks]{\tableofcontents[currentsection, hideallsubsections]}

\section{Inicio con Funciones Especiales}
\frame{\tableofcontents[currentsection, hideothersubsections]}
\subsection{Estudio analítico}

\begin{frame}
\frametitle{Comenzando con las funciones especiales}
Las funciones especiales en física matemática, son funciones solución de ecuaciones diferenciales ordinarias, que modelan diversos problemas en matemáticas, y son aplicadas en física para abordar problemas en astronomía, mecánica clásica, mecánica cuántica y fluidos, óptica, entre otras.
\end{frame}
\begin{frame}
\frametitle{Posturas encontradas}
También encontraremos otras posturas sobre las funciones especiales: 
\begin{quote}
Las funciones (quizás mal llamadas) especiales de la Física Matemática no tienen nada de \enquote{especial}. En principio son tan \enquote{especiales} como las funciones trigonométricas o los logaritmos, aunque por supuesto son menos habituales.
\end{quote}
\end{frame}
\begin{frame}
\frametitle{Posturas encontradas}
Sigue la cita\footnote{Bravo, Y. S. (2006). \textit{Métodos Matemáticos Avanzados para Científicos e Ingenieros}, Colección Manuales UEX 48, Madrid, España}:
\begin{quote}
Un nombre más adecuado sería, sin duda, el de \emph{funciones útiles}. En todo caso, es lícito preguntarse por el motivo de estudiar estas funciones y sus propiedades.
\end{quote}
\end{frame}
\begin{frame}
\frametitle{Revisa el artículo}
Como una actividad de este tema, te pedimos que leas el siguiente artículo: 
\begin{thebibliography}{X}
\bibitem{Berry} Berry, M. (2001). \textit{Why are special functions special?}, Physics Today, \textbf{54} 4, 11, doi.org/10.1063/1.1372098.
\end{thebibliography}
Para que nos des tu opinión sobre lo que el autor expresa del tema de las funciones especiales.
\end{frame}

\section{¿Cómo se estudian las funciones especiales?}
\frame{\tableofcontents[currentsection, hideothersubsections]}
\subsection{Método analítico}

\begin{frame}
\frametitle{Primera forma}
La primera forma de estudiar el conjunto de funciones especiales, es mediante el planteamiento de la ecuación diferencial.
\\
\bigskip
\pause
Para posteriormente aplicar lo revisado en los tres primeros temas del curso de MAF.
\end{frame}
\begin{frame}
\frametitle{Ruta a seguir}
Una posible ruta de trabajo es la siguiente:
\setbeamercolor{item projected}{bg=blue!70!black,fg=yellow}
\setbeamertemplate{enumerate items}[circle]
\begin{enumerate}[<+->]
\item EDP2H.
\item Separación de variables.
\item Método de Frobenius.
\item Segunda solución linealmente independiente.
\item Base completa.
\end{enumerate}
\end{frame}
\begin{frame}
\frametitle{Ruta a seguir}
\setbeamercolor{item projected}{bg=blue!70!black,fg=yellow}
\setbeamertemplate{enumerate items}[circle]
\begin{enumerate}[<+->]
\setcounter{enumi}{5}
\item Función generatriz.
\item Relaciones de recurrencia.
\item Paridad.
\item Ortogonalidad.
\item Formula de Rodrigues.
\end{enumerate}
\end{frame}
\begin{frame}
\frametitle{Ruta a seguir}
\setbeamercolor{item projected}{bg=blue!70!black,fg=yellow}
\setbeamertemplate{enumerate items}[circle]
\begin{enumerate}[<+->]
\setcounter{enumi}{10}
\item Ejercicios.
\item Ejemplos físicos.
\item Siguiente función especial.
\end{enumerate}
\end{frame}

\subsection{Método con la física}

\begin{frame}
\frametitle{Método con aplicación}
Con esta manera de estudio, se plantea de inicio un problema físico que permitirá entonces deducir una serie de elementos necesarios que se \enquote{conectan} con el desarrollo: por qué un determinado valor para una constante de separación, por ejemplo.
\end{frame}
\begin{frame}
\frametitle{La física como primer elemento}
El estudiar un fenómeno físico nos llevará a plantear una EDP como modelo matemático, por lo que se continua el análisis de la función especial como en el listado anterior.
\end{frame}
\begin{frame}
\frametitle{¿Cuál es la ventaja de uno método del otro?}
Estando en el sexto semestre de la carrera, es necesario abordar el estudio de las funciones especiales siempre con un ejemplo que permita enlazar lo que vemos en otras asignaturas, con este estudio de la física matemática.
\end{frame}
\begin{frame}
\frametitle{Varios caminos para llegar a lo mismo}
Veremos que habrá varias funciones especiales que permitirán llegar a un mismo planteamiento, en lo que refiere a la EDP, pero ocupando distintas áreas de la física.
\\
\bigskip
\pause
Esto nos pone en evidencia la relevancia de las funciones especiales.
\end{frame}
\begin{frame}
\frametitle{Punto importante}
Cabe señalar que las funciones especiales que revisaremos en este curso de MAF, \emph{NO} son todas las funciones que se pueden encontrar en la física matemática, así como las transformadas integrales.
\end{frame}
\begin{frame}
\frametitle{Aportación para un estudio posterior}
Lo que buscamos también es que cuenten con las habilidades de trabajo para revisar una función especial muy particular con la que se encuentren, y puedan entonces resolver el problema al que se enfrentan.
\end{frame}

\section{Objetivos del Tema 4}
\frame{\tableofcontents[currentsection, hideothersubsections]}
\subsection{El átomo de hidrógeno}

\begin{frame}
\frametitle{Iniciando el Tema 4}
El Tema 4: Separación de variables en coordenadas esféricas, dará inicio con el estudio (\emph{breve}) del átomo de hidrógeno como ejemplo clásico de la mecánica cuántica.
\\
\bigskip
\pause
La naturaleza del problema nos sugiere ocupar un sistema coordenado esférico.
\end{frame}
\begin{frame}
\frametitle{Objetivos}
Al terminar el Tema 4 se espera que el alumno:
\setbeamercolor{item projected}{bg=blue!70!black,fg=yellow}
\setbeamertemplate{enumerate items}[circle]
\begin{enumerate}[<+->]
\item Utilice el operador momento angular como elemento necesario para construir la solución del problema del átomo de hidrógeno, en su parte angular.
\item Estudiará los armónicos esféricos como funciones propias del operador momento angular.
\seti
\end{enumerate}
\end{frame}
\begin{frame}
\frametitle{Objetivos}
\setbeamercolor{item projected}{bg=blue!70!black,fg=yellow}
\setbeamertemplate{enumerate items}[circle]
\begin{enumerate}[<+->]
\conti
\item Determinará las soluciones de la ecuación diferencial asociada de Legendre, así como de la ecuación diferencial ordinaria de Legendre.
\end{enumerate}
\end{frame}

\section{Evaluación del Tema}
\frame{\tableofcontents[currentsection, hideothersubsections]}
\subsection{Ejercicios a resolver}

\begin{frame}
\frametitle{Ejercicios a cuenta}
La distribución de ejercicios es la siguiente:
\pause
\begin{table}
\centering
\begin{tabular}{l c c}
Material & Semanales & Opcionales \\ \hline
Lectura Barry & 1 &  \\ \hline
Armónicos esfericos  & 2 & 1 \\ \hline
Ec. asociada Legendre & 2 & 1 \\ \hline    
Ec. ordinaria Legendre & 2 & 1 \\ \hline    
Teorema de adición & 1 & 1 \\ \hline    
\end{tabular}
\end{table}
\end{frame}

\subsection{Examen Tarea}

\begin{frame}
\frametitle{Entrega del examen tarea 3}
En la sesión del 3 de diciembre se entregarán los enunciados del examen tarea 4, \pause para enviar la solución el día 19 de diciembre.
\end{frame}



\section{Cronograma de trabajo}
\frame{\tableofcontents[currentsection, hideothersubsections]}
\subsection{Trabajo por semana}

\begin{frame}
\frametitle{Distribución de tiempos}
Para una revisión completa de los materiales de trabajo para el Tema 4, se presenta la siguiente distribución de tiempos:
\end{frame}
\begin{frame}
\frametitle{Distribución de tiempos}
\setbeamercolor{item projected}{bg=blue!70!black,fg=yellow}
\setbeamertemplate{enumerate items}[circle]
\begin{enumerate}
\item Lectura de Barry y átomo de hidrógeno durante la semana 9.
\item Armónicos esféricos y ec. asociada de Legendre, durante la semana 10.
\item Ec. ordinaria de Legendre y Teorema de adición, durante la semana 11
\end{enumerate}
\end{frame}

\begin{frame}
\frametitle{Sesiones síncronas de trabajo}
Se programan las siguientes sesiones de trabajo:
\begin{itemize}
\item Martes 16 y viernes 19 de noviembre.
\item Miércoles 24 y viernes 26 de noviembre.
\item Miércoles 1 y viernes 3 de dicembre.
\item Lunes 6 de diciembre.
\end{itemize}
\end{frame}
\end{document}