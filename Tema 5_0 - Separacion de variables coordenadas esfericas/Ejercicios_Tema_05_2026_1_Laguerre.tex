\documentclass[12pt]{article}
\usepackage[left=0.25cm,top=1cm,right=0.25cm,bottom=1cm]{geometry}
\textwidth = 20cm
\hoffset = -1cm
\usepackage[utf8]{inputenc}
\usepackage[spanish,es-tabla]{babel}
\usepackage[autostyle,spanish=mexican]{csquotes}
\usepackage[tbtags]{amsmath}
\usepackage{nccmath}
\usepackage{amsthm}
\usepackage{amssymb}
\usepackage{graphicx}
\usepackage{standalone}
\usepackage[outdir=./]{epstopdf}
\usepackage{siunitx}
\usepackage{physics}
\usepackage{color}
\usepackage{float}
\usepackage{multicol}
%\usepackage{milista}
\usepackage{enumitem}
\usepackage{anyfontsize}
\usepackage{anysize}
\usepackage{enumitem}
\usepackage{capt-of}
\usepackage{bm}
\usepackage{relsize}
\usepackage{placeins}
\usepackage{empheq}
\usepackage{cancel}
\usepackage{wrapfig}
\spanishdecimal{.}
\renewcommand{\baselinestretch}{1.5} 
\renewcommand\labelenumii{\theenumi.{\arabic{enumii}}}
\newcommand{\ptilde}[1]{\ensuremath{{#1}^{\prime}}}
\newcommand{\stilde}[1]{\ensuremath{{#1}^{\prime \prime}}}
\newcommand{\ttilde}[1]{\ensuremath{{#1}^{\prime \prime \prime}}}
\newcommand{\ntilde}[2]{\ensuremath{{#1}^{(#2)}}}


\usepackage{apacite}
\title{Ejercicios a cuenta del Tema 5 \\[0.3em]  \large{Matemáticas Avanzadas de la Física}\vspace{-3ex}}
\author{M. en C. Gustavo Contreras Mayén}
\date{ }
\begin{document}
\vspace{-4cm}
\maketitle
\fontsize{14}{14}\selectfont

%Ref. Arfken (2006) 13.2.9
De acuerdo con la ecuación:
\begin{align*}
&\psi_{n l m} (r , \theta, \varphi) = \left[ \left( \dfrac{2 \, Z}{n \, a_{0}} \right)^{3} \, \dfrac{(n - l -1)!}{2 \, n \, (n + l)!} \right]^{1/2} \times\\[1em]
&\times \exp \left( - \dfrac{\alpha \, r}{2} \right) (\alpha \, r)^{L} \, L_{n - l +1}^{2 l +1} \, (\alpha \, r) \, Y_{l}^{m} \, (\theta, \varphi)
\end{align*}
La parte normalizada de la función de onda para el átomo de hidrógeno es:
\begin{align*}
&R_{n l} (r) = \left[ \alpha^{3} \dfrac{(n -l -1)!}{2 \, n \, (n + l)!} \right]^{1/2} \times \\[1em]
&\times \exp \left( \dfrac{-\alpha \, r}{2} \right) \, (\alpha \, r)^{l} \, L_{n - l +1}^{2 l +1} \, (\alpha \, r) 
\end{align*}
en donde: 
\begin{align*}
\alpha = \dfrac{2 \, Z}{n \, a_{0}} = \dfrac{2 \, Z \, m \, e^{2}}{4 \, \pi \, \epsilon_{0} \, \hbar^{2}}
\end{align*}
La cantidad $\expval{r}$ es el desplazamiento promedio del electrón con respecto al núcleo, mientras que $\expval{r^{-1}}$ es el promedio del movimiento recíproco.
Demuestra que al evaluar las siguientes integrales se obtienen los valores indicados:
\begin{enumerate}
\item
\begin{align*}
\displaystyle \expval{r} &= \scaleint{6ex}_{\bs 0}^{\infty} r \, R_{n l} (\alpha \, r) \, R_{n l} (\alpha \, r) \, r^{2} \dd{r} = \\[1em]
&= \dfrac{a_{0}}{2} [3 \, n^{2} - l (l + 1)]
\end{align*}
\item 
\begin{align*}
\displaystyle \expval{r^{-1}} &= \scaleint{6ex}_{\bs 0}^{\infty} r^{-1} \, R_{n l} (\alpha \, r) \, R_{n l} (\alpha \, r) \, r^{2} \dd{r} = \\[1em]
&= \dfrac{1}{n^{2} \, a_{0}}
\end{align*}
\end{enumerate}
Tip: Ocupa las propiedades de ortonormalización de los polinomios asociados de Laguerre.

\end{document}