\documentclass[12pt]{article}
\usepackage[utf8]{inputenc}
\usepackage[spanish,es-lcroman, es-tabla]{babel}
\usepackage[autostyle,spanish=mexican]{csquotes}
\usepackage{amsmath}
\usepackage{amssymb}
\usepackage{nccmath}
\numberwithin{equation}{section}
\usepackage{amsthm}
\usepackage{graphicx}
\usepackage{epstopdf}
\DeclareGraphicsExtensions{.pdf,.png,.jpg,.eps}
\usepackage{color}
\usepackage{float}
\usepackage{multicol}
\usepackage{enumerate}
\usepackage[shortlabels]{enumitem}
\usepackage{anyfontsize}
\usepackage{anysize}
\usepackage{array}
\usepackage{multirow}
\usepackage{enumitem}
\usepackage{cancel}
\usepackage{tikz}
\usepackage{circuitikz}
\usepackage{tikz-3dplot}
\usetikzlibrary{babel}
\usetikzlibrary{shapes}
\usepackage{bm}
\usepackage{mathtools}
\usepackage{esvect}
\usepackage{hyperref}
\usepackage{relsize}
\usepackage{siunitx}
\usepackage{physics}
%\usepackage{biblatex}
\usepackage{standalone}
\usepackage{mathrsfs}
\usepackage{bigints}
\usepackage{bookmark}
\spanishdecimal{.}

\setlist[enumerate]{itemsep=0mm}

\renewcommand{\baselinestretch}{1.5}

\let\oldbibliography\thebibliography

\renewcommand{\thebibliography}[1]{\oldbibliography{#1}

\setlength{\itemsep}{0pt}}
%\marginsize{1.5cm}{1.5cm}{2cm}{2cm}


\newtheorem{defi}{{\it Definición}}[section]
\newtheorem{teo}{{\it Teorema}}[section]
\newtheorem{ejemplo}{{\it Ejemplo}}[section]
\newtheorem{propiedad}{{\it Propiedad}}[section]
\newtheorem{lema}{{\it Lema}}[section]

\author{}
\title{Ejercicios a cuenta \\ {\large Matemáticas Avanzadas de la Física}\vspace{-1.5\baselineskip}}
\date{ }
\begin{document}
\newgeometry{top=2cm, bottom=2cm, left=1.5cm, right=1.5cm,headsep=0pt}
\maketitle
\fontsize{14}{14}\selectfont
\begin{enumerate}
%Referencia: Arfken 3.2.15
\item Una descripción de las partículas de spin $1$ utiliza las matrices
\begin{align*}
M_{x} = \dfrac{1}{\sqrt{2}} \, \mqty(0 & 1 & 0 \\ 1 & 0 & 1 \\ 0 & 1 & 0) \hspace{1cm} M_{y} = \dfrac{1}{\sqrt{2}} \, \mqty(0 & -i & 0 \\ i & 0 & -i \\ 0 & i & 0) \hspace{1cm} M_{z} = \mqty(1 & 0 & 0 \\ 0 & 0 & 0 \\ 0 & 0 & -1)
\end{align*}
Demuestra que
\begin{enumerate}
\item El conmutador $\comm{M_{x}}{M_{y}} = i M_{z}$, así como los demás (con una permutación cíclica de los índices). Usando el símbolo de Levi-Civita, se puede ver que
\begin{align*}
\comm{M_{i}}{M_{j}} = i \, \epsilon_{ijk} \, M_{k}
\end{align*}
\item El cuadrado de $M^{2}= M_{x}^{2} + M_{y}^{2} + M_{z}^{2} =  2 \, I$, en donde $I_{3 \cp 3}$ es la matriz unitaria.
\item El conmutador $\comm{M^{2}}{M_{i}} = 0$
\item El conmutador $\comm{M_{z}}{L^{+}} = L^{+}$
\item El conmutador $\comm{L^{+}}{L^{-}} = 2 \, M_{z}$

donde 
\[ L^{+} \equiv M_{x} + i \, M_{y} \hspace{1cm} L^{-} \equiv M_{x} - i \, M_{y} \]
\end{enumerate}
%Referencia: Arfken 3.2.20
\item Las matrices $L^{+}$ y $L^{-}$ del ejercicio anterior son los \enquote{operadores de ascenso y descenso}: donde $L^{+}$ operando en un sistema de proyección de spin $m$, aumentará la proyección del spin hasta $m + 1$ cuando $m$ se encuentra abajo de su máximo. $L^{+}$ operando en $m_{\max}$ da como resultado cero. $L^{-}$ reduce la proyección del spin en etapas de uno de modo similar. Dividiendo entre $\sqrt{2}$ se tiene
\begin{align*}
L^{+} = \mqty(0 & 1 & 0 \\ 0 & 0 & 1 \\ 0 & 0 & 0) \hspace{1.5cm} L^{-} = \mqty(0 & 0 & 0 \\ 1 & 0 & 0 \\ 0 & 1 & 0)
\end{align*}
Demuestra que
\begin{align*}
L^{+} \ket{-1} &= \ket{0}, \hspace{5cm} L^{-} \ket{-1} = \mbox{vector columna nulo} \\
L^{+} \ket{0} &= \ket{1}, \hspace{5cm} L^{-} \ket{0} = \ket{-1} \\
L^{+} \ket{1} &= \mbox{vector columna nulo}, \hspace{1.5cm} L^{-} \ket{1} = \ket{0}
\end{align*}
en donde
\begin{align*}
\ket{-1} = \mqty(0 \\ 0 \\ 1) \hspace{1cm} \ket{0} = \mqty(0 \\ 1 \\ 0) \hspace{1cm} \ket{1} = \mqty(1 \\ 0 \\ 0)
\end{align*}
representan los estados de proyección de spin $-1$, $0$ y $1$, respectivamente.
%Referencia: Arfken 3.2.13
\item Las tres matrices de spin de Pauli son
\begin{align*}
\sigma_{1} = \mqty(0 & 1 \\ 1 & 0) \hspace{1cm} \sigma_{2} = \mqty(0 & -i \\ i & 0) \hspace{1cm} \sigma_{3} = \mqty(1 & 0 \\ 0 & -1)
\end{align*}
Demuestra que
\begin{enumerate}
\item $(\sigma_{i})^{2} = I_{2 \cp 2}$, donde $I_{2 \cp 2}$ es la matriz identidad de $2 \cp 2$
\item $\sigma_{i} \, \sigma_{j} =  i \, \sigma_{k}$ $(i, j, k) = (1, 2, 3), (2, 3, 1), (3, 1, 2)$ (permutación cíclica)
\item $\sigma_{i} \, \sigma_{j} + \sigma_{j} \, \sigma_{i} = 2 \, \delta_{ij} \, I_{2 \cp 2}$

Estas matrices se utilizan en la teoría no relativista del spin del electrón.
\end{enumerate}
\end{enumerate}
\end{document}