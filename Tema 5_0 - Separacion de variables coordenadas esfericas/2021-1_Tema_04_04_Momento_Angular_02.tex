\documentclass[hidelinks,12pt]{article}
\usepackage[left=0.25cm,top=1cm,right=0.25cm,bottom=1cm]{geometry}
%\usepackage[landscape]{geometry}
\textwidth = 20cm
\hoffset = -1cm
\usepackage[utf8]{inputenc}
\usepackage[spanish,es-tabla]{babel}
\usepackage[autostyle,spanish=mexican]{csquotes}
\usepackage[tbtags]{amsmath}
\usepackage{nccmath}
\usepackage{amsthm}
\usepackage{amssymb}
\usepackage{mathrsfs}
\usepackage{graphicx}
\usepackage{subfig}
\usepackage{standalone}
\usepackage[outdir=./Imagenes/]{epstopdf}
\usepackage{siunitx}
\usepackage{physics}
\usepackage{color}
\usepackage{float}
\usepackage{hyperref}
\usepackage{multicol}
%\usepackage{milista}
\usepackage{anyfontsize}
\usepackage{anysize}
%\usepackage{enumerate}
\usepackage[shortlabels]{enumitem}
\usepackage{capt-of}
\usepackage{bm}
\usepackage{relsize}
\usepackage{placeins}
\usepackage{empheq}
\usepackage{cancel}
\usepackage{wrapfig}
\usepackage[flushleft]{threeparttable}
\usepackage{makecell}
\usepackage{fancyhdr}
\usepackage{tikz}
\usepackage{bigints}
\usepackage{scalerel}
\usepackage{pgfplots}
\usepackage{pdflscape}
\pgfplotsset{compat=1.16}
\spanishdecimal{.}
\renewcommand{\baselinestretch}{1.5} 
\renewcommand\labelenumii{\theenumi.{\arabic{enumii}})}
\newcommand{\ptilde}[1]{\ensuremath{{#1}^{\prime}}}
\newcommand{\stilde}[1]{\ensuremath{{#1}^{\prime \prime}}}
\newcommand{\ttilde}[1]{\ensuremath{{#1}^{\prime \prime \prime}}}
\newcommand{\ntilde}[2]{\ensuremath{{#1}^{(#2)}}}

\newtheorem{defi}{{\it Definición}}[section]
\newtheorem{teo}{{\it Teorema}}[section]
\newtheorem{ejemplo}{{\it Ejemplo}}[section]
\newtheorem{propiedad}{{\it Propiedad}}[section]
\newtheorem{lema}{{\it Lema}}[section]
\newtheorem{cor}{Corolario}
\newtheorem{ejer}{Ejercicio}[section]

\newlist{milista}{enumerate}{2}
\setlist[milista,1]{label=\arabic*)}
\setlist[milista,2]{label=\arabic{milistai}.\arabic*)}
\newlength{\depthofsumsign}
\setlength{\depthofsumsign}{\depthof{$\sum$}}
\newcommand{\nsum}[1][1.4]{% only for \displaystyle
    \mathop{%
        \raisebox
            {-#1\depthofsumsign+1\depthofsumsign}
            {\scalebox
                {#1}
                {$\displaystyle\sum$}%
            }
    }
}
\def\scaleint#1{\vcenter{\hbox{\scaleto[3ex]{\displaystyle\int}{#1}}}}
\def\bs{\mkern-12mu}


%\usepackage{showframe}
\usepackage{apacite}
\title{Momento Angular - 2a. Parte \\ \large {Tema 4 - Sep. variables en coord. esféricas} \vspace{-3ex}}
\author{M. en C. Gustavo Contreras Mayén}
\date{ }
\begin{document}
\vspace{-4cm}
\maketitle
\fontsize{14}{14}\selectfont
\tableofcontents
\newpage
\section{Operadores de escalera.}

Definimos los operadores de escalera:
\begin{align}
L_{+} &= L_{x} + i \, L_{y} = \hbar \, e^{i \phi} \, \left[ \pdv{\theta} + i \, \cot \theta \, \dv{\phi} \right] \label{eq:ecuacion_54} \\[0.5em]
L_{-} &= L_{x} - i \, L_{y} = \hbar \, e^{-i \phi} \, \left[ - \pdv{\theta} + i \, \cot \theta \, \dv{\phi} \right] \label{eq:ecuacion_55}
\end{align}
Usando el álgebra de operadores se puede demostrar que:
\begin{align}
L_{+} \, Y_{\ell}^{m} &= [(\ell - m)(\ell + m + 1)]^{1/2} \, \hbar Y_{\ell}^{m+1} \label{eq:ecuacion_56} \\[0.5em]
L_{-} \, Y_{\ell}^{m} &= [(\ell + m)(\ell - m + 1)]^{1/2} \, \hbar Y_{\ell}^{m-1} \label{eq:ecuacion_57}
\end{align}
De consideraciones generales, vemos que para un valor dado de $\ell$, $m$ solo puede tomar valores entre $-\ell, -\ell + 1, \ldots, + \ell$. De las ecs. (\ref{eq:ecuacion_56}) y (\ref{eq:ecuacion_57}) se tiene que:
\begin{align}
L_{+} \, Y_{\ell}^{\ell} &= 0 \\[0.5em]
L_{-} \, Y_{\ell}^{-\ell} &= 0
\end{align}
Entonces las ecs. (\ref{eq:ecuacion_54}) y (\ref{eq:ecuacion_56})  se pueden escribir como:
\begin{align}
Y_{\ell}^{m+1} = [(\ell - m)(\ell + m + 1)]^{1/2} \, e^{i \phi} \, \left[ \pdv{\theta} + i \, \cot \theta \, \dv{\phi} \right] \, Y_{\ell}^{m}
\label{eq:ecuacion_60}
\end{align}
Por lo que si comenzamos en $m = 0$, es posible obtener directamente los
\begin{align*}
Y_{\ell}^{1}, Y_{\ell}^{2}, \ldots, Y_{\ell}^{\ell}
\end{align*}
por ejemplo, si hacemos que $\ell =  2$, de las ecs. (\ref{eq:ecuacion_56}) y del resultado obtenido:
\begin{align}
Y_{\ell}^{0} (\theta, \phi) = \left( \dfrac{2 \, \ell + 1}{4 \, \pi} \right) \, P_{\ell} (\cos \theta)
\label{eq:ecuacion_52}
\end{align} 
tendremos que:
\begin{align}
Y_{2}^{0} (\theta, \phi) &= \sqrt{\dfrac{5}{4 \, \pi}} \, P_{2} (\cos \theta) = \\[0.5em]
&= \sqrt{\dfrac{5}{16 \, \pi}} \, [3 \, \cos^{2} \theta - 1] \label{eq:ecuacion_61}
\end{align}
Por tanto:
\begin{align}
\begin{aligned}[b]
Y_{2}^{1} (\theta, \phi) &= \sqrt{\dfrac{5}{96 \, \pi}} \, e^{i \phi} \, \left[ \pdv{\theta} + i \, \cot \theta \, \dv{\phi} \right] \, (3 \, \cos^{2} \theta - 1) = \\[0.5em]
&= - \sqrt{\dfrac{15}{8 \, \pi}} \, \sin \theta \, \cos \theta \, e^{i \phi}
\end{aligned}
\label{eq:ecuacion_62}
\end{align}
de manera similar:
\begin{align}
\begin{aligned}[b]
Y_{2}^{2} (\theta, \phi) &= - \sqrt{\dfrac{15}{32 \, \pi}} \, e^{i \phi} \, \left[ \pdv{\theta} + i \, \cot \theta \, \dv{\phi} \right] \, \left( \dfrac{1}{2} \, \sin 2 \theta \, e^{i \phi} \right) = \\[0.5em]
&= - \sqrt{\dfrac{15}{32 \, \pi}} \, \sin^{2} \theta \,  e^{2 i \phi}
\end{aligned}
\label{eq:ecuacion_63}
\end{align}
De manera similar, usando la relación:
\begin{align}
Y_{\ell}^{m-1} = [(\ell + m)(\ell - m + 1)]^{-1/2} \, e^{-i \phi} \, \left[ - \pdv{\theta} + i \, \cot \theta \, \dv{\phi} \right] \, Y_{\ell}^{m}
\label{eq:ecuacion_64}
\end{align}
podemos obtener los $Y_{\ell}^{-1}$ y $Y_{\ell}^{-2}$.
\par
Las funciones de onda resultantes se normalizan automáticamente. Por tanto, si se conoce la expresión de $P_{\ell} (\cos \theta)$, mediante la aplicación repetida de las ecs. (\ref{eq:ecuacion_60}) y (\ref{eq:ecuacion_64}) se puede determinar explícitamente los $Y_{\ell}^{m}$ para todos los valores de $m$.
\subsection{Condición de ortonormalidad.}

Las expresiones para los armónicos esféricos satisfacen la siguiente condición de ortonormalidad:
\begin{align}
\int_{0}^{2 \pi} \int_{0}^{\pi} (Y_{\ell}^{m})^{*} (\theta, \phi) \, Y_{\ptilde{\ell}}^{\ptilde{m}} (\theta, \phi) \sin \theta \dd{\theta} \dd{\phi}
\label{eq:ecuacion_65}
\end{align}
\subsection{Ecuación de valores propios.}

Por lo tanto, la ecuación de valores propios para $\vb{L}^{2}$ que quedó pendiente en el primer material de trabajo, se expresa por:
\begin{align}
\vb{L}^{2} \, Y_{\ell}^{m} (\theta, \phi) = \ell (\ell + 1) \, \hbar^{2} \, Y_{\ell}^{m} (\theta, \phi)
\label{eq:ecuacion_66}
\end{align}
donde $\ell(\ell + 1) \, \hbar^{2}$ (con $\ell = 0, 1, 2, \ldots$) representan los \emph{valores propios} de $\vb{L}^{2}$, mientras que $Y_{\ell}^{m} (\theta, \phi)$ son las respectivas \emph{funciones propias}.
\par
El valor propio $\ell(\ell + 1) \, \hbar^{2}$ presenta una degeneración de orden $(2 \, \ell + 1)$, es decir, para un valor particular de $\ell$, habrá $(2 \, \ell + 1)$ valores de $m$:
\begin{align*}
m = -\ell, -\ell + 1, \ldots, \ell - 1, \ell
\end{align*}
Dada la dependencia de $\phi$ de los $Y_{\ell}^{m}(\theta, \phi)$, que es de la forma $e^{i m \phi}$, se obtiene que:
\begin{align*}
L_{z} \, Y_{\ell}^{m} (\theta, \phi) &= - i \hbar \, \pdv{\phi} \, \left[ \Theta_{\ell} (\theta) \, \dfrac{1}{\sqrt{2 \, \pi}} \, e^{i m \phi} \right]
\end{align*}
o de manera equivalente:
\begin{align}
L_{z} \, Y_{\ell}^{m} (\theta, \phi) = m \, \hbar \, Y_{\ell}^{m}
\label{eq:ecuacion_67}
\end{align}
Por tanto, los $Y_{\ell}^{m}(\theta, \phi)$ son de manera simultánea las funciones propias de $\vb{L}^{2}$ y de $L_{z}$. Es fácil demosrta que no hay funciones propias de $L_{x}$ y $L_{y}$, excepto para el caso $\ell = 0$ y $m = 0$; de hecho, es imposible construir funciones propias de cualquiera que no sea $\vb{L}^{2}$ y uno de los tres componentes de $\vb{L}$. Sin embargo, eligiendo combinaciones lineales apropiadas, es posible tener funciones propias simultáneas de $\vb{L}^{2}$ y $L_{x}$ y también de $\vb{L}^{2}$ y $L_{y}$.
\section{Partícula en un potencial central.}

A partir del estudio de una partícula en un potencial central, habíamos llegado a un expresión que determina el Hamiltoniano de ese sistema, y ahora ya podemos presentar la ecuación de valores propios:
\begin{align}
H \, \psi(r, \theta, \phi) &= E \, \psi(r, \theta, \phi) \label{eq:ecuacion_08_05} \\[0.5em]
\vb{L}^{2} \, \psi(r, \theta, \phi) &= \ell(\ell +  1) \, \psi(r, \theta, \phi) \label{eq:ecuacion_08_06} \\[0.5em]
L_{z} \, \psi(r, \theta, \phi) &= m \, \hbar \, \psi(r, \theta, \phi) \label{eq:ecuacion_08_07}
\end{align}
para determinar aquellos estados que son funciones propias de $H$, $\vb{L}^{2}$ y $L_{z}$, usando el método de separación de variables, tendremos:
\begin{align}
\psi(r, \theta, \phi) = R_{n \ell} \, Y_{\ell}^{m} (r, \theta, \phi)
\label{eq:ecuacion_08_08}
\end{align}
donde $Y_{\ell}^{m}$ son los armónicos esféricos de los que ya hemos comentado anteriormente, y $R_{n \ell}$ es la función radial, que no depende del número $m$.
\subsection{Parte radial.}
Hemos mencionado que la ecuación correspondiente a la variable radial $R_{n \ell}(r)$ es:
\begin{align}
\left[ - \dfrac{\hbar^{2}}{2 \, M} \, \dfrac{1}{r} \, \dv[2]{r} \, r + \dfrac{\ell (\ell + 1) \, \hbar^{2}}{2 \, M \, r^{2}} + V(r) \right] \, R_{n \ell} (r) = E \, R_{n \ell}
\label{eq:ecuacion_08_11}
\end{align}
Cabe señalar que el estudio y análisis de esta ecuación se verá en el \emph{Tema 5 - Funciones especiales}, ya que la correspondiente solución nos proporcionará los \emph{Polinomios de Laguerre}.
\par
La ec. (\ref{eq:ecuacion_08_11}) se puede simplificar si escribimos:
\begin{align}
R_{n \ell} (r) = \dfrac{1}{r} \, U_{n \ell} (r)
\label{eq:ecuacion_08_12}
\end{align}
por lo que tendremos:
\begin{align}
\left[ - \dfrac{\hbar^{2}}{2 \, M} \, \dv[2]{r} \, r + \dfrac{\ell (\ell + 1) \, \hbar^{2}}{2 \, M \, r^{2}} + V(r) \right] \, U_{n \ell} (r) = E \, U_{n \ell}
\label{eq:ecuacion_08_13}
\end{align}
La ec. (\ref{eq:ecuacion_08_13}) es análoga al problema unidimensional de una partícula de masa $M$ moviéndose en un potencial efectivo $V_{e}(r)$, donde:
\begin{align}
V_{e} (r) = V (r) + \dfrac{\ell (\ell + 1) \, \hbar^{2}}{2 \, M \, r^{2}}
\label{eq:ecuacion_08_14}
\end{align}
\subsection{Parte angular.}
Ya habíamos mostrado el resultado para la parte angular con las ecuaciones:
\begin{align}
&-i \,\pdv{\phi} Y_{m}^{\ell} (\theta, \phi) = m \, Y_{m}^{\ell} (\theta, \phi) \label{eq:ecuacion_08_15} \\[0.5em]
&- \left[ \dfrac{1}{\sin \theta} \, \pdv{\theta} \left( \sin \theta \, \pdv{\theta} \right) + \dfrac{1}{\sin^{2} \theta} \, \pdv[2]{\theta} \right] \, Y_{m}^{\ell}(\theta, \phi) = \ell(\ell + 1) \, Y_{m}^{\ell} (\theta, \phi) \label{eq:ecuacion_08_16} 
\end{align}
\subsection{Interacción entre dos partículas.}
Considera un sistema de dos partículas sin spin de masa $m_{1}$ y $m_{2}$ en las posiciones $\vb{r}_{1}$ y $\vb{r}_{2}$, respectivamente.
\par
Vamos a considerar que la energía potencial depende solo de la distancia entre las dos partículas: $V(\vb{r}_{1} - \vb{r}_{2})$. El estudio del movimiento de dos partículas se simplifica si usamos las coordenadas del centro de masa:
\begin{align}
\vb{r}_{c} = \dfrac{m_{1} \, \vb{r}_{1} + m_{2} \, \vb{r}_{2}}{m_{1} + m_{2}}
\label{eq:ecuacion_08_17}
\end{align}
y las coordenadas relativas:
\begin{align}
\vb{r} = \vb{r}_{1} - \vb{r}_{2}
\label{eq:ecuacion_08_18}
\end{align}
Veremos que es posible obtener las siguientes expresiones:
\begin{align}
- \dfrac{\hbar^{2}}{2(m_{1} + m_{2})} \laplacian{\phi (\vb{r}_{c})} = E_{c} \, \phi(\vb{r}_{c})
\label{eq:ecuacion_08_19}
\end{align}
y
\begin{align}
\left[ \dfrac{\hbar^{2}}{2 \, \mu} \, \laplacian + V(\vb{r}) \right] \, \chi (\vb{r}) = E_{c} \, \chi (\vb{r})
\label{eq:ecuacion_08_20}
\end{align}
donde $\mu$ es la \emph{masa reducida} de las dos partículas:
\begin{align}
\mu = \dfrac{m_{1} \, m_{2}}{m_{1} + m_{2}}
\label{eq:ecuacion_08_21}
\end{align}
De la ec. (\ref{eq:ecuacion_08_19}) podemos concluir que el centro de masa se comporta como una partícula libre de masa $m_{1} + m_{2}$ y con energía $E_{c}$.
\par
El movimiento relativo de las dos partículas, queda determinado por la ec. (\ref{eq:ecuacion_08_20}) y es semejante al movimiento de una partícula de masa $\mu$ colocada en un potencial $V (\vb{r})$.
\end{document}