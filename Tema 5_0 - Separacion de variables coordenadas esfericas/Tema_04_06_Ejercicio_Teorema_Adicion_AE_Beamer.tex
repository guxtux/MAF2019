\documentclass[12pt]{beamer}
\usepackage{../Estilos/BeamerMAF}
\usepackage{../Estilos/ColoresLatex}
\input{../Preambulos/preambulo_Beamer_Frankfurt_beaver}

\setbeamercolor{section in foot}{bg=almond, fg=black}
\setbeamercolor{subsection in foot}{bg=laserlemon, fg=black}
\setbeamercolor{date in foot}{bg=blue, fg=white}

\makeatletter
\setbeamertemplate{footline}
{
\leavevmode%
\hbox{%
\begin{beamercolorbox}[wd=.333333\paperwidth,ht=2.25ex,dp=1ex,center]{section in foot}%
  \usebeamerfont{section in foot} \insertsection
\end{beamercolorbox}%
\begin{beamercolorbox}[wd=.333333\paperwidth,ht=2.25ex,dp=1ex,center]{subsection in foot}%
  \usebeamerfont{subsection in foot}  \insertsubsection
\end{beamercolorbox}%
\begin{beamercolorbox}[wd=.333333\paperwidth,ht=2.25ex,dp=1ex,right]{date in head/foot}%
  \usebeamerfont{date in head/foot} \insertshortdate{} \hspace*{1.5em}
  \insertframenumber{} / \inserttotalframenumber \hspace*{2ex} 
\end{beamercolorbox}}%
\vskip0pt%
}
\makeatother
\usefonttheme{serif}
\setbeamercolor{frametitle}{bg=palecerulean}
\resetcounteronoverlays{saveenumi}

\date{12 de mayo de 2022}

\title{\large{Ejercicio - El teorema de adición AE}}
\subtitle{Funciones Especiales I}
\author{M. en C. Gustavo Contreras Mayén}

\begin{document}
\maketitle
\fontsize{14}{14}\selectfont
\spanishdecimal{.}

\section*{Contenido}
\frame[allowframebreaks]{\tableofcontents[currentsection, hideallsubsections]}

%Ref. Jackson 3.6 Problem Solution
\section{Ejercicio con la fórmula de adición}
\frame{\tableofcontents[currentsection, hideothersubsections]}
\subsection{Enunciado}

\begin{frame}
\frametitle{Enunciado para el ejercicio}
Dos cargas puntuales $q$ y $-q$ se localizan sobre el eje $z$ en los puntos $z = +a$ y $z = -a$, respectivamente.
\pause
\begin{figure}[H]
    \centering
    \begin{tikzpicture}
        \draw (0, -2) -- (0, 2) node [right, pos=1] {$z$};
        \draw [fill] (0, 1) circle (0.1);
        \draw [fill] (0, -1) circle (0.1);
        \draw (-0.2, 0) -- (0.2, 0);
        \node at (0.5, 1) {$q$};
        \node at (0.5, -1) {$-q$};
        \node at (-0.5, 0.5) {$a$};
        \node at (-0.5, -0.5) {$-a$};
    \end{tikzpicture}
\end{figure}
\end{frame}
\begin{frame}
\frametitle{A resolver}
Por resolver:
\pause
\setbeamercolor{item projected}{bg=olive,fg=white}
\setbeamertemplate{enumerate items}{%
\usebeamercolor[bg]{item projected}%
\raisebox{1.5pt}{\colorbox{bg}{\color{fg}\footnotesize\insertenumlabel}}%
}
\begin{enumerate}[<+->]
\item \label{item:inciso_1} Calcula el potencial electrostático como una expansión de los armónicos esféricos y de potencias de $r$, tanto para $r > a$ y $r < a$.
\seti
\end{enumerate}
\end{frame}
\begin{frame}
\frametitle{A resolver}
\setbeamercolor{item projected}{bg=olive,fg=white}
\setbeamertemplate{enumerate items}{%
\usebeamercolor[bg]{item projected}%
\raisebox{1.5pt}{\colorbox{bg}{\color{fg}\footnotesize\insertenumlabel}}%
}
\begin{enumerate}[<+->]    
\conti
\item \label{item:inciso_2} Manteniendo el producto $q \, a = p /2$ constante, toma el límite cuando $a \to 0$ y calcula el potencial para $r \neq 0$. \pause Esto corresponde a la definición de un dipolo sobre el eje $z$ y su potencial.
\seti
\end{enumerate}
\end{frame}
\begin{frame}
\frametitle{A resolver}
\setbeamercolor{item projected}{bg=olive,fg=white}
\setbeamertemplate{enumerate items}{%
\usebeamercolor[bg]{item projected}%
\raisebox{1.5pt}{\colorbox{bg}{\color{fg}\footnotesize\insertenumlabel}}%
}
\begin{enumerate}[<+->]    
\conti
\item \label{item:inciso_3} Supongamos ahora que el dipolo del inciso \ref{item:inciso_2} está rodeado por una capa esférica de radio $b$ concéntrica en el origen, puesta a tierra. Por superposición lineal, encuentra el potencial en todas partes dentro del cascarón.
\end{enumerate}
\end{frame}

\subsection{Solución Inciso 1}

\begin{frame}
\frametitle{Usando la teoría de electromagnetismo}
\textbf{Inciso \ref{item:inciso_1}:} Usado la ley de Coulomb, podemos escribir de manera directa el potencial de dos cargas puntuales:
\pause
\begin{align*}
\Phi = \dfrac{q}{4 \pi \varepsilon_{0}} ~ \bigg[ \dfrac{1}{\abs{\vb{r} - a \, \vu{k}}} - \dfrac{1}{\abs{\vb{r} + a \, \vu{k}}} \bigg]
\end{align*}
\end{frame}
\begin{frame}
\frametitle{Usando la teoría de MAF}
Usando el teorema de adición, expandimos en factores de $1/R$:
\pause
\begin{align*}
\dfrac{1}{\abs{\vb{r} {-} \vb{\pderivada{r}}}} = 4 \pi \nsum_{\ell=0}^{\infty} \nsum_{m=-\ell}^{\ell} \dfrac{1}{2 \ell {+} 1} \, \dfrac{r_{<}^{\ell}}{r_{>}^{\ell+1}} \, Y_{\ell m}^{*} (\pderivada{\theta}, \pderivada{\phi}) \, Y_{\ell m} (\theta, \phi)
\end{align*}
\pause
donde:
\begin{align*}
r_{<} = \min \left\{ r, \pderivada{r} \right\} \hspace{1.5cm} r_{>} = \max \left\{ r, \pderivada{r} \right\}
\end{align*}
\end{frame}
\begin{frame}
\frametitle{Reescribiendo el potencial}
Entonces el potencial se escribe como:
\pause
\begin{align*}
&\Phi {=} \dfrac{q}{4 \pi \varepsilon_{0}} \bigg[ 4 \pi \nsum_{\ell=0}^{\infty} \nsum_{m=-\ell}^{\ell} \dfrac{1}{2 \ell {+} 1} \dfrac{r_{<}^{\ell}}{r_{>}^{\ell+1}} Y_{\ell m}^{*} (0, 0) Y_{\ell m} (\theta, \phi) + \\[0.5em]
&- 4 \pi \nsum_{\ell=0}^{\infty} \nsum_{m=-\ell}^{\ell} \dfrac{1}{2 \ell {+} 1} \, \dfrac{r_{<}^{\ell}}{r_{>}^{\ell+1}} \, Y_{\ell m}^{*} (\pi, 0) \, Y_{\ell m} (\theta, \phi) \bigg]
\end{align*}
\pause
Ahora se tiene que:
\pause
\begin{align*}
r_{<} = \min \left\{ r, a \right\} \hspace{1.5cm} r_{>} = \max \left\{ r, a \right\}
\end{align*}
\end{frame}
\begin{frame}
\frametitle{Reorganizando la expresión}
Cancelando el término $4 \pi$ y factorizando las sumas con el armónico $Y_{\ell m} (\theta, \phi)$, el potencial queda:
\pause
\begin{align*}
\Phi &= \dfrac{q}{\varepsilon_{0}} ~ \nsum_{\ell=0}^{\infty} \nsum_{m=-\ell}^{\ell} \dfrac{1}{2 \ell + 1} \, \dfrac{r_{<}^{\ell}}{r_{>}^{\ell+1}} \, \times \\[0.5em]
&\times Y_{\ell m} (\theta, \phi) \bigg[ Y_{\ell m}^{*} (0, 0) - Y_{\ell m}^{*} (\pi, 0) \bigg]    
\end{align*} 
\end{frame}
\begin{frame}
\frametitle{Expresión para $r < a$}
Haciendo de manera explícita cada caso para $r$, se tiene que:
\pause
\begin{align*}
\Phi &= \dfrac{q}{\varepsilon_{0}} \,\nsum_{\ell=0}^{\infty} \nsum_{m=-\ell}^{\ell} \dfrac{1}{2 \ell {+} 1} \dfrac{r_{\ell}}{a^{\ell+1}} \, \times \\[0.5em]
&\times Y_{\ell m} (\theta, \phi) \bigg[ Y_{\ell m}^{*} (0, 0) {-} Y_{\ell m}^{*} (\pi, 0) \bigg]
\end{align*}
\end{frame}
\begin{frame}
\frametitle{Expresión para $r > a$}
\begin{align*}
\Phi &= \dfrac{q}{\varepsilon_{0}} \nsum_{\ell=0}^{\infty} \nsum_{m=-\ell}^{\ell} \dfrac{1}{2 \ell {+} 1} \dfrac{a^{\ell}}{r^{\ell+1}} \, \times \\[0.5em]
&\times Y_{\ell m} (\theta, \phi) \bigg[ Y_{\ell m}^{*} (0, 0) {-} Y_{\ell m}^{*} (\pi, 0) \bigg]
\end{align*}
\end{frame}
\begin{frame}
\frametitle{Misión cumplida}
La pregunta original pedía la respuesta en términos de los armónicos esféricos, así que eso es lo que hemos dado.
\\
\bigskip
\pause
Sin embargo, podemos simplificar la respuesta.
\end{frame}
\begin{frame}
\frametitle{Aprovechando la simetría}
Tengamos en cuenta que el problema presenta simetría azimutal, \pause por lo que la solución debe ser azimutalmente simétrica.
\\
\bigskip
\pause
Esto significa que todos los términos de la serie excepto el término $m = 0$ deben anularse:
\end{frame}
\begin{frame}
\frametitle{Consecuencia de la simetría}
\begin{eqnarray*}
\begin{aligned}
\Phi &= \pause \dfrac{q}{\varepsilon_{0}} \nsum_{\ell=0}^{\infty}  \dfrac{1}{2 \ell {+} 1} \dfrac{r_{<}^{\ell}}{r_{>}^{\ell+1}} Y_{\ell 0} (\theta, \phi) \bigg[ Y_{\ell 0}^{*} (0, 0) {-} Y_{\ell 0}^{*} (\pi, 0) \bigg] = \\[0.5em] \pause
&= \dfrac{q}{4 \pi \varepsilon_{0}} \nsum_{\ell=0}^{\infty} \dfrac{r_{<}^{\ell}}{r_{>}^{\ell+1}} \, P_{\ell} (\cos \theta) \bigg[ P_{\ell} (1) - P_{\ell} (-1) \bigg] = \\[0.5em] \pause
&= \dfrac{q}{4 \pi \varepsilon_{0}} \nsum_{\ell=0}^{\infty} \dfrac{r_{<}^{\ell}}{r_{>}^{\ell+1}} \, P_{\ell} (\cos \theta) \bigg[ 1 - (-1)^{\ell} \bigg]
\end{aligned}
\end{eqnarray*}
\end{frame}
\begin{frame}
\frametitle{Potencial electróstatico}
Por lo que el potencial electrostático está dado por:
\pause
\begin{align*}
\Phi = \dfrac{2 q}{4 \pi \varepsilon_{0}} \nsum_{\ell=0, \text{impar}}^{\infty} \dfrac{r_{<}^{\ell}}{r_{>}^{\ell+1}} \, P_{\ell} (\cos \theta)
\end{align*}
donde:
\pause
\begin{align*}
r_{<} = \min \left\{ r, a \right\} \hspace{1.5cm} r_{>} = \max \left\{ r, a \right\}
\end{align*}
\end{frame}

\subsection{Solución Inciso 2}

\begin{frame}
\frametitle{El inciso 2}
Manteniendo el producto $q \, a = p /2$ constante, toma el límite cuando $a \to 0$ y calcula el potencial para $r \neq 0$.
\pause
Con esta condición, siempre estaremos en la región $r > a$. 
\end{frame}
\begin{frame}
\frametitle{Usando el resultado del inciso 1}
Del resultado del inciso \ref{item:inciso_1}, se tiene que:
\pause
\begin{align*}
\Phi = \dfrac{2 q}{4 \pi \varepsilon_{0}} \nsum_{\ell=0, \text{impar}}^{\infty} \dfrac{a^{\ell}}{r^{\ell+1}} \, P_{\ell} (\cos \theta)
\end{align*}
\end{frame}
\begin{frame}
\frametitle{Condición del enunciado}
Como $p = 2 q a$, resulta:
\pause
\begin{eqnarray*}
\begin{aligned}
\Phi &= \dfrac{p}{4 \pi \varepsilon_{0}} \nsum_{\ell=0, \text{impar}}^{\infty} \dfrac{a^{\ell-1}}{r^{\ell+1}} \, P_{\ell} (\cos \theta) = \\[0.5em] \pause
&= \dfrac{p}{4 \pi \varepsilon_{0}} \bigg[ \dfrac{1}{r^{2}} P_{1} (\cos \theta) {+} \dfrac{a^{2}}{r^{4}} P_{3} (\cos \theta) + \\[0.5em]
&+ \dfrac{a^{4}}{r^{6}} P_{5} (\cos \theta) {+} \ldots \bigg]
\end{aligned}
\end{eqnarray*}
\end{frame}
\begin{frame}
\frametitle{Caso con el límite}
Cuando $a \to 0$, se obtiene:
\pause
\begin{align*}
\setlength{\fboxsep}{3\fboxsep}\boxed{
\Phi = \dfrac{p}{4 \pi \varepsilon_{0}} \, \dfrac{1}{r^{2}} \, \cos \theta }
\end{align*}
\pause
Este es el potencial de un dipolo perfecto en el origen que apunta a lo largo del eje $z$.
\end{frame}
\subsection{Solución Inciso 3}


\begin{frame}
\frametitle{Enunciado del Inciso 3}
\textbf{Inciso \ref{item:inciso_3}: } Si el dipolo del inciso \ref{item:inciso_2} está rodeado por una capa esférica conectada a tierra de radio $b$ concéntrica con el origen:
\setbeamercolor{item projected}{bg=olive,fg=white}
\setbeamertemplate{enumerate items}{%
\usebeamercolor[bg]{item projected}%
\raisebox{1.5pt}{\colorbox{bg}{\color{fg}\footnotesize\insertenumlabel}}%
}
\begin{enumerate}[<+->]
\item Habrá un potencial adicional debido a la esfera.
\item El potencial total en la superficie debe ser cero.
\end{enumerate}
\end{frame}
\begin{frame}
\frametitle{Ocupando el resultado del inciso 2}
Del resultado del inciso \ref{item:inciso_2}:
\pause
\begin{align*}
\Phi = \dfrac{p}{4 \pi \varepsilon_{0}} \, \dfrac{1}{r^{2}} \, \cos \theta
\end{align*}
\end{frame}
\begin{frame}
\frametitle{Sumando el otro potencial}
Agregamos el potencial adicional:
\pause
\begin{align*}
\Phi = \dfrac{p}{4 \pi \varepsilon_{0}} \, \dfrac{1}{r^{2}} \, \cos \theta + \nsum_{\ell=0}^{\infty} A_{\ell} \, r^{\ell} \, P_{\ell} (\cos \theta)
\end{align*}
\end{frame}
\begin{frame}
\frametitle{Usando la CDF}
Aplicando la condición de frontera:
\pause
\begin{eqnarray*}
\begin{aligned}
0 &= \dfrac{p}{4 \pi \varepsilon_{0}} \, \dfrac{1}{r^{2}} \, \cos \theta + \nsum_{\ell=0}^{\infty} A_{\ell} \, b^{\ell} \, P_{\ell} (\cos \theta) \\[0.5em] \pause
\Rightarrow \hspace{0.2cm} &- \dfrac{p}{4 \pi \varepsilon_{0}} \, \dfrac{1}{b^{2}} \, \cos \theta = \nsum_{\ell=0}^{\infty} A_{\ell} \, b^{\ell} \, P_{\ell} (\cos \theta)
\end{aligned}
\end{eqnarray*}
\end{frame}
\begin{frame}
\frametitle{Usando propiedades de los $P_{\ell} (\cos \theta)$}
Debido a la condición de ortogonalidad, solo el término $\ell = 1$ es no nulo:
\pause
\begin{align*}
A_{1} = - \dfrac{p}{4 \pi \varepsilon_{0}} \, \dfrac{1}{b^{3}}
\end{align*}
\end{frame}
\begin{frame}
\frametitle{Potencial en puntos internos}
Entonces el potencial en puntos dentro del cascarón está dado por:
\pause
\begin{align*}
\setlength{\fboxsep}{3\fboxsep}\boxed{
\Phi = \dfrac{p \, \cos \theta}{4 \pi \varepsilon_{0}} ~ \bigg[ \bigg( \dfrac{b}{r} \bigg)^{2} - \dfrac{r}{b} \bigg] \qed}
\end{align*}
\end{frame}
\end{document}