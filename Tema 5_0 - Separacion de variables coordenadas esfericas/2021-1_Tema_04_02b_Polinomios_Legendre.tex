\input{../Preambulos/preambulo_presentacion_Dresden_seahorse}
\usepackage{anyfontsize}
\title{\large{Ejercicios con Polinomios de Legendre}}
\subtitle{Potenciales y armónicos esféricos}
\author{M. en C. Gustavo Contreras Mayén}
\date{}
\institute{Facultad de Ciencias - UNAM}
\titlegraphic{\includegraphics[width=1.75cm]{../Imagenes/escudo-facultad-ciencias}\hspace*{4.75cm}~%
   \includegraphics[width=1.75cm]{../Imagenes/escudo-unam}
}
\setbeamertemplate{navigation symbols}{}
\begin{document}
\maketitle
\fontsize{14}{14}\selectfont
\spanishdecimal{.}
\section*{Contenido}
\frame[allowframebreaks]{\tableofcontents[currentsection, hideallsubsections]}
\section{Cálculo de potenciales eléctricos}
\frame{\tableofcontents[currentsection, hideothersubsections]}
\subsection{Ejercicio 1}
\begin{frame}
\frametitle{Enunciado del problema}
Considera un arreglo de cargas en el eje $x$:
\begin{itemize}
\item Una carga $+q$ está en la posición $x = -a$.
\item Una carga $-2 \, q$ está en la posición $x = 0$.
\item Una carga $+q$ está en la posición $x = a$.
\end{itemize}
\end{frame}
\begin{frame}
\frametitle{Geometría del problema}
Haciendo un esquema del problema tenemos que:
\begin{figure}
    \centering
    \includestandalone[scale=0.9]{Figuras/cuadrupolo_01}
\end{figure}
\pause
Calcula el potencial electrostático de este arreglo, conocido como el \emph{cuadruplo lineal eléctrico}.
\end{frame}
\subsection{Solución}
\begin{frame}
\frametitle{Estableciendo el potencial}
Como se revisó en el curso de Electromagnetismo, es posible describir el potencial debido a las tres cargas:
\pause
\begin{align}
V = k \, q \, \left( \dfrac{1}{r_{1}} + \dfrac{1}{r_{2}} \right) - \dfrac{2 \, k \, q}{r}
\label{eq:ecuacion_01}
\end{align}
\end{frame}
\begin{frame}
\frametitle{Estableciendo el potencial}
\begin{figure}
    \centering
    \includestandalone[scale=0.7]{Figuras/cuadrupolo_02}
\end{figure}
\pause
\fontsize{12}{12}\selectfont
donde:
\begin{itemize}
\item $r_{1}$ es la distancia de la carga ($0, -a)$ al punto $P$.
\item $r_{2}$ es la distancia de la carga ($0, +a)$ al punto $P$.
\item $r$ es la distancia de la carga en el origen al punto $P$.
\end{itemize}
\end{frame}
\begin{frame}
\frametitle{Solución}
A partir de la ley de los cosenos escribimos $r_{1}$ y $r_{2}$:
\pause
\begin{eqnarray*}
r_{1}^{2} &=& r^{2} + a^{2} - 2 \, a \, r \, \cos \theta =  \\[0.5em] \pause 
r_{1} &=& r \sqrt{1 + \left( \dfrac{a}{r} \right)^{2} - 2 \left( \dfrac{a}{r} \right) \, \cos \theta}
\end{eqnarray*}
\end{frame}
\begin{frame}
\frametitle{Solución}
Para $r_{2}$:
\pause
\begin{eqnarray*}
r_{2}^{2} &=& r^{2} + a^{2} + 2 \, a \, r \, \cos \theta =  \\[0.5em]
r_{2} &=& r \sqrt{1 + \left( \dfrac{a}{r} \right)^{2} + 2 \left( \dfrac{a}{r} \right) \, \cos \theta}
\end{eqnarray*}
\end{frame}
\begin{frame}
\frametitle{Uso de los polinomios de Legendre}
Sabemos que la expresión para el potencial es muy similar a la función generatriz de los polinomios de Legendre.
\\
\bigskip
\pause
Si nos enfocámos solamente en el potencial de las dos cargas en $(0,\pm a)$:
\end{frame}
\begin{frame}
\frametitle{Potencial de las dos cargas}
El potencial debido a esas dos cargas es:
\begin{align}
\begin{aligned}
V_{\pm a} &= \dfrac{k \, q}{r} \bigg[ \sum_{n=0}^{\infty} \left( \dfrac{a}{r} \right)^{\ell} P_{\ell}(x) + \\[0.5em]
&+ \sum_{n=0}^{\infty} (-1)^{\ell} \left( \dfrac{a}{r} \right)^{\ell} P_{\ell}(x) \bigg]
\end{aligned}
\label{eq:ecuacion_03}
\end{align}
\end{frame}
\begin{frame}
\frametitle{La expresión obtenida}
Vemos que la ec. (\ref{eq:ecuacion_03}) es muy similar a la expresión para el dipolo eléctrico lineal.
\\
\bigskip
\pause
La diferencia es que las dos sumas suman (en lugar de restar); esto ocurre porque las dos cargas en $(0, \pm a)$ son positivas, en lugar de tener signos opuestos como en el caso del dipolo.
\end{frame}
\begin{frame}
\frametitle{La expresión obtenida}
Observemos el término $(-1)^{\ell}$ en la segunda suma: esto surge de la expresión de la ley de cosenos para $r_{2}$, en la que el ángulo entre $r$ y $a$ es $(\ang{180} - \theta)$, de modo que $\cos (\ang{180} - \theta) = - \cos \theta$.
\end{frame}
\begin{frame}
\frametitle{Revisando la expresión}
Si revisamos la ec. (\ref{eq:ecuacion_03}), nos damos cuenta de que las sumas se anula si $n$ es impar, dejándonos solo los términos pares:
%\fontsize{12}{12}\selectfont
\begin{align}
\begin{aligned}
V_{\pm a} &= \dfrac{2 \, k \, q}{r} \bigg[ P_{0} \left( \dfrac{a}{r} \right)^{0} + P_{2} \left( \dfrac{a}{r} \right)^{2} \, P_{4} \left( \dfrac{a}{r} \right)^{4} + \ldots \bigg]
\end{aligned}
\label{eq:ecuacion_04}
\end{align}
\end{frame}
\begin{frame}
\frametitle{Completando la expresión}
Recordemos que la ec. (\ref{eq:ecuacion_04}) describe solo el potencial debido a las cargas $+q$ colocadas en $(0, \pm a)$.
\\
\bigskip
\pause
Usando el principio de superposición, podemos determinar el potencial total del sistema, agregando a la ec. (\ref{eq:ecuacion_04}) el potencial debido a la carga $-2q$ colocada en el origen.
\end{frame}
\begin{frame}
\frametitle{Completando la expresión}
El potencial debido solo a la carga $-2q$ es:
\begin{align*}
V_{O} = - \dfrac{2 \, k \, q}{r}
\end{align*}
\pause
Por lo que el potencial completo del cuadrupolo es:
\end{frame}
\begin{frame}
\frametitle{Potencial completo}
El potencial completo es:
\begin{align}
\begin{aligned}
V_{T} &= \dfrac{2 \, k \, q}{r} \bigg[ P_{0} \left( \dfrac{a}{r} \right)^{0} + \\[0.5em]
&+ P_{2} \left( \dfrac{a}{r} \right)^{2} \, P_{4} \left( \dfrac{a}{r} \right)^{4} + \ldots \bigg] - \dfrac{2 \, k \, q}{r}
\end{aligned}
\label{eq:ecuacion_05}
\end{align}
\end{frame}
\begin{frame}
\frametitle{Usando resultados previos}
Del primero término en los corchetes de la ec. (\ref{eq:ecuacion_05}), $P_{0}(x) = 1$, tendremos que:
\begin{eqnarray*}
V_{T} &=& \dfrac{2 \, k \, q}{r} + \dfrac{k \, q}{r} \bigg[ P_{2} \left( \dfrac{a}{r} \right)^{2} \, P_{4} \left( \dfrac{a}{r} \right)^{4} + \ldots \bigg] - \dfrac{2 \, k \, q}{r} = \\[0.5em] \pause
&=& \dfrac{2 \, k \, q}{r} \bigg[ P_{2} \left( \dfrac{a}{r} \right)^{2} \, P_{4} \left( \dfrac{a}{r} \right)^{4} + \ldots \bigg]
\end{eqnarray*}
\end{frame}
\begin{frame}
\frametitle{Solución}
Vemos que la expansión del cuadrupolo eléctrico comienza con el término $P_{2}(x)$:
\begin{align*}
V_{T} = \dfrac{2 \, k \, q}{r} \bigg[ P_{2} \left( \dfrac{a}{r} \right)^{2} \, P_{4} \left( \dfrac{a}{r} \right)^{4} + \ldots \bigg]
\end{align*}
\end{frame}
\section{Armónicos esféricos}
\frame{\tableofcontents[currentsection, hideothersubsections]}
\subsection{Ejercicio 2}
\begin{frame}
\frametitle{Tres incisos}
Para este ejercicio tendremos tres incisos:
\setbeamercolor{item projected}{bg=blue!70!black,fg=yellow}
\setbeamertemplate{enumerate items}[circle]
\begin{enumerate}[<+->]
\item Usando la relación:
\begin{align*}
Y_{\ell}^{0} \, (\theta, \phi) = \sqrt{\dfrac{2 \, \ell + 1}{4 \, \pi}} \, P_{\ell} (\cos \theta)
\end{align*}
\\
\bigskip
Determina $Y_{3}^{0} \, (\theta, \phi)$
\seti
\end{enumerate}
\end{frame}
\begin{frame}
\frametitle{Tres incisos}
\setbeamercolor{item projected}{bg=blue!70!black,fg=yellow}
\setbeamertemplate{enumerate items}[circle]
\begin{enumerate}[<+->]
\conti
\item Determina la expresión de $Y_{3}^{0}$ en coordenadas cartesianas.
\item A partir de $Y_{3}^{0} (\theta, \phi)$ infiere el valor de $Y_{3}^{\pm 1} (\theta, \phi)$
\end{enumerate}
\end{frame}
\begin{frame}
\frametitle{Inciso 1}
Usando la relación:
\begin{align*}
Y_{\ell}^{0} (\theta, \phi) = \sqrt{\dfrac{2 \, \ell + 1}{4 \, \pi}} \, P_{\ell} (\cos \theta)
\end{align*}
\\
\bigskip
\pause
Determina $Y_{3}^{0} (\theta, \phi)$
\end{frame}
\begin{frame}
\frametitle{Solución del inciso 1}
Sabemos que:
\begin{align*}
P_{3} (\cos \theta) = \dfrac{1}{2} (5 \, \cos^{3} \theta - 3 \, \cos \theta)
\end{align*}
\pause
entonces:
\begin{eqnarray*}
Y_{3}^{0} (\theta, \phi) &=& \sqrt{\dfrac{7}{4 \, \pi}} \, P_{3} (\cos \theta) = \\[0.5em] \pause
&=& \sqrt{\dfrac{7}{16 \, \pi}} \, (5 \, \cos^{3} \theta - 3 \, \cos \theta)
\end{eqnarray*}
\end{frame}
\begin{frame}
\frametitle{Inciso 2}
Determina la expresión de $Y_{3}^{0}$ en coordenadas cartesianas:
\\
\bigskip
\pause
Ya que $\cos \theta = z/r$, tenemos que:
\begin{eqnarray*}
5 \, \cos^{3} \theta - 3 \, \cos \theta &=& 5 \, \cos \theta(5 \, \cos^{2} \theta - 3) = \\[0.5em] \pause
&=& \dfrac{z \, (5 \, z^{2} - 3 \, r^{2})}{r^{3}}
\end{eqnarray*}
\end{frame}
\begin{frame}
\frametitle{Solución inciso 2}
Por lo tanto:
\pause
\begin{align*}
Y_{3}^{0} \, (x, y, z) = \sqrt{\dfrac{7}{16 \, \pi}} \, \dfrac{z}{r^{3}} \, (5 z^{2} - 3 \, r^{2})
\end{align*}
\end{frame}
\begin{frame}
\frametitle{Inciso 3}
A partir de $Y_{3}^{0} \, (\theta, \phi)$ infiere el valor de $Y_{3}^{\pm 1} \, (\theta, \phi)$
\\
\bigskip
\pause
Para calular $Y_{3}^{1}$ a partir de $Y_{3}^{0}$, necesitaremos utilizar el operador de ascenso $\hat{L}_{+}$ en $Y_{3}^{0}$.
\end{frame}
\begin{frame}
\frametitle{Primer paso}
De manera algebraica tenemos que:
\begin{align}
\hat{L}_{+} \, Y_{3}^{0} = \hbar \, \sqrt{3 (3 + 1) - 0} \, Y_{3}^{1} = 2 \, \hbar \, \sqrt{3} \, Y_{3}^{1}
\label{eq:ecuacion_05_194}
\end{align}
\pause
por lo que
\begin{align}
Y_{3}^{1} = \dfrac{1}{2 \, \hbar \, \sqrt{3}} \hat{L}_{+} \, Y_{3}^{0}
\label{eq:ecuacion_05_195}
\end{align}
\end{frame}
\begin{frame}
\frametitle{Segundo paso}
Ahora usamos la forma diferencial de $\hat{L}_{+}$:
\begin{eqnarray*}
&{}&\hat{L}_{+} \, Y_{3}^{0} \, (\theta, \phi) = \hbar \, e^{i \phi} \left[ \pdv{\theta} + i \, \dfrac{\cos \theta}{\sin \theta} \right] \, Y_{3}^{0} \, (\theta, \phi) = \\[0.5em] \pause
&=& \hbar \, \sqrt{\dfrac{7}{16 \, \pi}} \, e^{i \phi} \, \left[ \pdv{\theta} + i \, \dfrac{\cos \theta}{\sin \theta} \right] \, (5 \, \cos^{3} \theta {-} 3 \, \cos \theta) = \\[0.5em] \pause
&=& - 3 \, \hbar \, \sqrt{\dfrac{7}{16 \, \pi}} \, \sin \theta \, (5 \, \cos^{2} \theta - 1) \, e^{i \phi}
\end{eqnarray*}
\end{frame}
\begin{frame}
\frametitle{Solución}
Al agregar el resultado tenemos que:
\begin{align*}
Y_{3}^{1} &= \dfrac{1}{2 \, \hbar \, \sqrt{3}} \hat{L}_{+} \, Y_{3}^{0} = \\[0.5em]
&= - \sqrt{\dfrac{21}{64 \, \pi}} \sin \theta \, (5 \cos^{2} \theta - 1) \, e^{i \phi}
\end{align*}
\end{frame}
\begin{frame}
\frametitle{El otro resultado}
Para calcular $Y_{3}^{-1}$ a partir de $Y_{3}^{0}$ aplicamos ahora el operador de descenso $\hat{L}_{-}$:
\begin{align*}
\hat{L}_{-} \, Y_{3}^{0} = \hbar \, \sqrt{3 (3 + 1) - 0} \, Y_{3}^{-1} = 2 \, \hbar \, \sqrt{3} \, Y_{3}^{-1}
\end{align*}
Por lo que:
\begin{align}
Y_{3}^{-1} = \dfrac{1}{2 \, \hbar \, \sqrt{3}} \hat{L}_{-} \, Y_{3}^{0}
\label{eq:ecuacion_05_199}
\end{align}
\end{frame}
\begin{frame}
\frametitle{Segundo paso}
Ahora usamos la forma diferencial de $\hat{L}_{-}$:
\begin{eqnarray*}
&{}&\hat{L}_{-} \, Y_{3}^{0} \, (\theta, \phi) = - \hbar \, e^{-i \phi} \left[ \pdv{\theta}  - i \, \dfrac{\cos \theta}{\sin \theta} \right] \, Y_{3}^{0} \, (\theta, \phi) = \\[0.5em] \pause
&=& -\hbar \, \sqrt{\dfrac{7}{16 \, \pi}} \, e^{-i \phi} \, \left[ \pdv{\theta} - i \, \dfrac{\cos \theta}{\sin \theta} \right] \, (5 \, \cos^{3} \theta {-} 3 \, \cos \theta) = \\[0.5em] \pause
&=& 3 \, \hbar \, \sqrt{\dfrac{7}{16 \, \pi}} \, \sin \theta \, (5 \, \cos^{2} \theta {-} 1) \, e^{-i \phi}
\end{eqnarray*}
\end{frame}
\begin{frame}
\frametitle{Resultado}
Siguiendo el mismo procedimiento, tendremos que el armónico esférico es:
\begin{align*}
Y_{3}^{-1} &= \dfrac{1}{2 \, \hbar \, \sqrt{3}} \hat{L}_{-} \, Y_{3}^{0} = \\[0.5em]
&= - \sqrt{\dfrac{21}{64 \, \pi}} \sin \theta \, (5 \cos^{2} \theta - 1) \, e^{-i \phi}
\end{align*}
\end{frame}
\end{document}