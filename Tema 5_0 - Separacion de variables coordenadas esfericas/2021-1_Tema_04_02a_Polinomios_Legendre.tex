\input{../Preambulos/preambulo_presentacion_Dresden_seahorse}
\title{\large{Ejercicios con Polinomios de Legendre}}
\subtitle{Geometría esférica con simetría azimutal}
\author{M. en C. Gustavo Contreras Mayén}
\date{}
\institute{Facultad de Ciencias - UNAM}
\titlegraphic{\includegraphics[width=1.75cm]{../Imagenes/escudo-facultad-ciencias}\hspace*{4.75cm}~%
   \includegraphics[width=1.75cm]{../Imagenes/escudo-unam}
}
\setbeamertemplate{navigation symbols}{}
\begin{document}
\maketitle
\fontsize{14}{14}\selectfont
\spanishdecimal{.}
\section*{Contenido}
\frame[allowframebreaks]{\tableofcontents[currentsection, hideallsubsections]}
\section{Termodinámica y las esferas}
\frame{\tableofcontents[currentsection, hideothersubsections]}
\subsection{Ejercicio 1}
\begin{frame}
\frametitle{Enunciado}
Dos hemisferios sólidos conductores de calor de radio $a$, separados por un hueco muy pequeño aislante, forman una esfera. Las dos mitades de la esfera están en contacto - por fuera - con dos baños de calor (infinitos) a temperaturas $T_{0}$ y $-T_{0}$. 
\\
\bigskip
\pause
Queremos encontrar la distribución de temperatura $T (r, \theta, \varphi)$ en el interior de la esfera.
\end{frame}
\begin{frame}
\frametitle{Geometría del problema}
\begin{figure}[H]
\centering
\includestandalone{Figuras/esfera_2}
\caption{Dos semiesferas separadas infinitesimalmente a temperaturas contrarias.}
\label{fig:figura2}
\end{figure}
\end{frame}
\begin{frame}
\frametitle{Descripción del problema}
Elegimos un sistema de coordenadas esféricas en donde el origen coincide con el centro de la esfera y el eje polar es perpendicular al plano ecuatorial.
\\
\bigskip
El hemisferio con temperatura $T_{0}$ suponemos que es el hemisferio norte.
\end{frame}
\subsection{Solución completa Ec. Laplace}
\begin{frame}
\frametitle{Solución completa}
 Para encontrar la solución más general con simetría azimutal de la ecuación de Laplace en coordenadas esféricas, multiplicamos la solución radial y la solución angular (polinomio de Legendre) para cada $k$ y sumamos sobre todos los valores posibles de $k$:
 \pause
\begin{align}
\Phi (r, \theta) = \sum_{k=0}^{\infty} \left( A_{k} \; r^{k} + \dfrac{B_{k}}{r^{k+1}} \right) \; P_{k} (\cos \theta)
\label{eq:ecuacion_029a}
\end{align}
donde $A_{k}$ y $B_{k}$ son constantes y se ha sustituido $\cos \theta$ por $x$.
\end{frame}
\begin{frame}
\frametitle{Ventaja en el problema}
Dado que el problema tiene simetría azimutal, $T$ es independiente de $\varphi$, y podemos escribir inmediatamente la solución general de la ecuación (\ref{eq:ecuacion_029a}).
\end{frame}
\begin{frame}
\frametitle{Ventaja en el problema}
Sin embargo, dado que el origen se encuentra en la región de interés, es necesario excluir a todos los potencias negativas de $r$.
\begin{align*}
\Phi (r, \theta) = \sum_{k=0}^{\infty} \left( A_{k} \; r^{k} + \dfrac{B_{k}}{r^{k+1}} \right) \; P_{k} (\cos \theta)
\end{align*}
\pause    
Esto se logra dejando que todos los coeficientes de $B$ se anulen.
\end{frame}
\begin{frame}
\frametitle{Simplificación en la expresión}
Por lo tanto, tenemos
\begin{align}
T(r, \theta) = \sum_{n=0}^{\infty} A_{n} \; r^{n} \, P_{n} (\cos \theta)
\label{eq:ecuacion_050a}
\end{align}
\end{frame}
\begin{frame}
\frametitle{Coeficientes por calcular}
Quedando pendiente calcular las constantes $A_{n}$.
\\
\bigskip
\pause
Pero notemos que
\begin{align*}
T(a, \theta) = 
\begin{cases}
T_{0} & \mbox{ si } 0 \leq \theta < \dfrac{\pi}{2} \\[0.5em]
-T_{0} & \mbox{ si } \frac{\pi}{2} < \theta \leq \pi 
\end{cases}
\end{align*}
\end{frame}
\begin{frame}
\frametitle{Ajuste en las variables}
En términos de $u = \cos \theta$, podemos escribir
\begin{align*}
T(a, u) = 
\begin{cases}
T_{0} & \mbox{ si } -1 \leq u < 0 \\[0.5em]
-T_{0} & \mbox{ si } 0 < u \leq 1 
\end{cases}
\end{align*}
\end{frame}
\begin{frame}
\frametitle{Expresión de trabajo}
Sustituyendo en la ecuación (\ref{eq:ecuacion_050a}), obtenemos
\begin{align} 
\begin{aligned}
T(a, u) &= 
\begin{cases}
T_{0} & \mbox{ si } -1 \leq u < 0 \\[0.5em]
-T_{0} & \mbox{ si } 0 < u \leq 1 
\end{cases} = \\[0.5em]
&= \sum_{n=0}^{\infty} \underbrace{A_{n} a^{n}}_{\equiv c_{n}} \, P_{n}(u)
\end{aligned}
\label{eq:ecuacion_051a}
\end{align}
\end{frame}
\begin{frame}
\frametitle{Ocupando un resultado previo}
Podemos ocupar el resultado que se revisó en el material de trabajo: \pause La expansión en polinomios de Legendre de la función $f(x)$ definida por:
\begin{align*}
f(x) = \begin{cases}
V_{0} & \mbox{ si } 0 < x \leq 1 \\[0.5em]
- V_{0} & \mbox{ si } -1 \leq x < 0
\end{cases}
\end{align*}
\end{frame}
\begin{frame}
\frametitle{Ocupando un resultado previo}
La expansión de la función $f(x)$ en Polinomios de Legendre se escribe como:
\begin{align*}
f(x) &= \begin{cases}
V_{0} & \mbox{ si } 0 < x \leq 1 \\[0.5em]
-V_{0} & \mbox{ si } -1 \leq x < 0
\end{cases}
= \\[0.5em]
&= V_{0} \, \sum_{k=0}^{\infty} \dfrac{(-1)^{k}(4 \, k + 3)(2 \, k!)}{2^{2k+1} \; k! \; (k+1)!} \, P_{2k+1} (x)
\end{align*}
\end{frame}
\begin{frame}
\frametitle{Usando el resultado}
Usando $u$ en lugar de $x$ del resultado anterior, vemos que es equivalente a la expansión en series, tal que los coeficientes pares están ausentes, así
\begin{align*}
c_{2k+1} \equiv A_{2k+1} \, a^{2k+1} = \dfrac{(-1)^{k} \, (4 \, k + 3)(2 \, k)!}{2^{2k+1} \; k! \; (k + 1)!} \, T_{0}
\end{align*}
\end{frame}
\begin{frame}
\frametitle{Calculando los coeficientes}
Encontrando $A_{2k+1}$ de esta ecuación y agregándolo en la ecuación (\ref{eq:ecuacion_050a}) se obtiene
\begin{align*}
&T(r, \theta) = \\[0.5em]
&= T_{0} \sum_{k=0}^{\infty} \dfrac{(-1)^{k} (4 k {+} 3)(2 k)!}{2^{2k+1} \; k! \; (k {+} 1)!} \left( \dfrac{r}{a} \right)^{2k+1} \, P_{2k+1} (\cos \theta)
%\label{eq:ecuacion_052a}
\end{align*}
donde se ha sustituido $\cos \theta$ por $u$.
\end{frame}
\section{Esferas y electrostática}
\frame{\tableofcontents[currentsection, hideothersubsections]}
\subsection{Ejercicio 2}
\begin{frame}
\frametitle{Enunciado}
Considera dos semiesferas conductoras de radio $a$ que tienen una pequeña separación aislante en el ecuador.
\\
\bigskip
\pause
La semiesfera superior se encuentra a un potencial $V_{0}$, mientras que la semiesfera inferior a un potencial $-V_{0}$, como se muestra en la figura (\ref{fig:figura_03}):
\end{frame}
\begin{frame}
\frametitle{Geometría del problema}
\begin{figure}
    \centering
    \includegraphics[scale=0.9]{Figuras/esfera_3.tex}
    \caption{Dos semiesferas separadas infinitesimalmente con potenciales contrarios.}
\label{fig:figura_03}
\end{figure}
\end{frame}
\begin{frame}
\frametitle{Problema a resolver}
¿Cuál es el potencial en puntos por fuera de la esfera completa?
\\
\bigskip
\pause
Podemos ocupar nuevamente la expresión para la ecuación de Laplace con simetría azimutal.
\end{frame}
\begin{frame}
\frametitle{Elementos a considerar}
Ya que el potencial se debe de anular en el infinito, esperamos que el primer término de la ecuación se anule:
\begin{align*}
\Phi (r, \theta) = \sum_{k=0}^{\infty} \left( A_{k} \; r^{k} + \dfrac{B_{k}}{r^{k+1}} \right) \; P_{k} (\cos \theta)
\end{align*}
\pause
Es decir que $A_{k} = 0$.
\end{frame}
\begin{frame}
\frametitle{Calculando los coeficientes $B_{k}$}
Para encontrar los coeficientes $B_{k}$, \pause susituimos $a$ en $r$ en la ecuación anterior, y hacemos $\cos \theta = u$, entonces:
\begin{align*}
\Phi (a, u) = \sum_{k=0}^{\infty} \dfrac{B_{k}}{\underbrace{a^{k+1}}_{\equiv c_{k}}} \; P_{k} (u)
\end{align*}
\end{frame}
\begin{frame}
\frametitle{Calculando los coeficientes $B_{k}$}
Donde:
\begin{align*}
\Phi(a, u) &= \begin{cases}
-V_{0} & \mbox{ si } -1 < u < 0 \\[0.5em]
+V_{0} & \mbox{ si } 0 < u < 1
\end{cases}
\end{align*}
\pause
El cálculo de los coeficientes se realiza de la misma manera que en el ejemplo anterior.
\end{frame}
\begin{frame}
\frametitle{Calculando los coeficientes $B_{k}$}
Así, los $c_{k} = 0$ para los k pares y
\begin{align*}
c_{2m+1} = \dfrac{B_{2m+1}}{a^{2m+2}} = (-1)^{m} \, \dfrac{(4 \, m + 3) ( 2 \, m)!}{2^{2m+1} \, (m + 1)! \, m!} \, V_{0}
\end{align*}
\pause
Que es lo mismo:
\begin{align*}
B_{2m+1} = \dfrac{(-1)^{m} \, (4 \, m + 3) ( 2 \, m)!}{2^{2m+1} \, (m + 1)! \, m!} \, a^{2m+2} \, V_{0}
\end{align*}
\end{frame}
\begin{frame}
\frametitle{El potencial resultante}
Una vez calculados los coeficientes, podemos expresar el potencial como:
\begin{align*}
\Phi(r, \theta) &= V_{0} \, \sum_{m=0}^{\infty} (-1)^{m} \, \dfrac{(4 \, m + 3) ( 2 \, m)!}{2^{2m+1} \, (m + 1)! \, m!} \times \\[0.5em]
&\times \left( \dfrac{a}{r} \right)^{2m+2} \, P_{2m+1} (\cos \theta)
\end{align*}
\end{frame}
\begin{frame}
\frametitle{Conclusión}
El potencial encontrado $\Phi(r, \theta)$ es la expansión multipolar del potencial en las dos semiesferas.
\\
\bigskip
\pause
Es interesante observar que el término monopolar: el término de tipo $1/r$ no aparece.
\end{frame}
\begin{frame}
\frametitle{Conclusión}
Esto se debe a que la carga total en las dos semiesferas debe ser nula.
\\
\bigskip
\pause
Siendo consistente con la simentría del problema, en donde se espera una densidad superficial de carga igual pero de signos contrarios en las semiesferas.
\end{frame}
\section{Otros problemas}
\frame{\tableofcontents[currentsection, hideothersubsections]}
\subsection{Posibles problemas para el examen}
\begin{frame}
\frametitle{Posibles problemas}
Una vez revisados estos dos ejercicios, es claro que se pueden modificar las condiciones de los ejercicios, manteniendo una geometría esférica con simetría azimutal, como los siguientes casos:
\end{frame}
\begin{frame}
\frametitle{Posible ejercicio 1}
Dos esferas conductoras: una de radio $a$ a un potencial $V_{1}$ está dentro de otra esfera de radio $b$ a un potencial $V_{2}$. Calcular el potencial en los puntos dentro de la esfera con radio $a$, los puntos en $ a < r < b$ y los puntos por fuera de la esfera $b$.
\end{frame}
\begin{frame}
\frametitle{Posible ejercicio 1}
\begin{figure}
    \centering
    \includestandalone[scale=0.8]{Figuras/esfera_4}
\end{figure}
\end{frame}
\begin{frame}
\frametitle{Posible ejercicio 2}
\fontsize{12}{12}\selectfont
Calcular el potencial electrostático dentro de una esfera de radio $a$ con un pequeño espacio aislante en su ecuador, la semiesfera inferior se encuentra aterrizada, mientras que la semiesfera superior se mantiene a un potencial constante $V_{0}$.
\begin{figure}
    \centering
    \includestandalone[scale=0.7]{Figuras/esfera_5}
\end{figure}
\end{frame}
\begin{frame}
\frametitle{Posible ejercicio 3}
\fontsize{12}{12}\selectfont
Calcular el potencial electrostático en puntos dentro y fuera de una esfera de radio $a$ que se mantiene a un potencial constante $V_{0}$.
\begin{figure}
    \centering
    \includestandalone[scale=0.7]{Figuras/esfera_6}
\end{figure}
\end{frame}
\begin{frame}
\frametitle{Posible ejercicio 4}
Una esfera conductora de radio $a$ está dentro de otra esfera conductora de radio $b$, ésta se compone de dos semiesferas con una separación infinitesimal aislante.
\\
\bigskip
La esfera interior está a un potencial $V_{1}$. La semiesfera superior externa está a un potencial $+V_{2}$, mientras que la semiesfera inferior a un potencial $-V_{2}$.
\end{frame}
\begin{frame}
\frametitle{Posible ejercicio 4}
\begin{figure}
    \centering
    \includestandalone[scale=0.7]{Figuras/esfera_7}
\end{figure}
\fontsize{12}{12}\selectfont
Calcula el potencial dentro de la esfera inferior, entre las dos esferas, y por fuera de la esfera exterior.
\end{frame}
\begin{frame}
\frametitle{Posible ejercicio 5}
Una esfera conductora de radio $a$ se compone de dos semiesferas con una separación infinitesimal aislante, está dentro de otra esfera conductora de radio $b$, 
\\
\bigskip
La semiesfera superior interna está a un baño térmico $+T_{1}$, mientras que la semiesfera inferior a un baño térmico $-T_{1}$. La esfera exterior se mantiene en un baño térmico $T_{2}$.
\end{frame}
\begin{frame}
\frametitle{Posible ejercicio 5}
\begin{figure}
    \centering
    \includestandalone[scale=0.7]{Figuras/esfera_1}
\end{figure}
\fontsize{12}{12}\selectfont
Calcula la temperatura dentro de la esfera inferior, entre las dos esferas, y por fuera de la esfera exterior.
\end{frame}
    
\end{document}