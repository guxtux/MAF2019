\documentclass[hidelinks,12pt]{article}
\usepackage[left=0.25cm,top=1cm,right=0.25cm,bottom=1cm]{geometry}
%\usepackage[landscape]{geometry}
\textwidth = 20cm
\hoffset = -1cm
\usepackage[utf8]{inputenc}
\usepackage[spanish,es-tabla, es-lcroman]{babel}
\usepackage[autostyle,spanish=mexican]{csquotes}
\usepackage[tbtags]{amsmath}
\usepackage{nccmath}
\usepackage{amsthm, bm}
\usepackage{amssymb}
\usepackage{mathrsfs}
\usepackage{graphicx}
\usepackage{subfig}
\usepackage{caption}
%\usepackage{subcaption}
\usepackage{standalone}
\graphicspath{{Imagenes/}{../Imagenes/}}
\usepackage[outdir=./Imagenes/]{epstopdf}
\usepackage{siunitx}
\usepackage{physics}
\usepackage{color}
\usepackage{float}
\usepackage{hyperref}
\usepackage{multicol}
\usepackage{multirow}
%\usepackage{milista}
\usepackage{anyfontsize}
\usepackage{anysize}
%\usepackage{enumerate}
\usepackage[shortlabels]{enumitem}
\usepackage{capt-of}
\usepackage{bm}
\usepackage{mdframed}
\usepackage{relsize}
\usepackage{placeins}
\usepackage{empheq}
\usepackage{cancel}
\usepackage{pdfpages}
\usepackage{wrapfig}
\usepackage[flushleft]{threeparttable}
\usepackage{makecell}
\usepackage{fancyhdr}
\usepackage{tikz}
\usepackage{bigints}
\usepackage{tcolorbox}
\tcbuselibrary{breakable}
\usepackage{scalerel}
\usepackage{pgfplots}
\usepackage{pdflscape}
\usepackage{enumitem}
\pgfplotsset{compat=1.16}
\spanishdecimal{.}
\renewcommand{\baselinestretch}{1.5}
\renewcommand{\labelenumii}{\arabic{enumi}.\arabic{enumii}}
\renewcommand{\labelenumiii}{\arabic{enumi}.\arabic{enumii}.\arabic{enumiii}}

\newcommand{\ptilde}[1]{\ensuremath{{#1}^{\prime}}}
\newcommand{\stilde}[1]{\ensuremath{{#1}^{\prime \prime}}}
\newcommand{\ttilde}[1]{\ensuremath{{#1}^{\prime \prime \prime}}}
\newcommand{\ntilde}[2]{\ensuremath{{#1}^{(#2)}}}
\newcommand{\pderivada}[1]{\ensuremath{{#1}^{\prime}}}
\newcommand{\sderivada}[1]{\ensuremath{{#1}^{\prime \prime}}}
\newcommand{\tderivada}[1]{\ensuremath{{#1}^{\prime \prime \prime}}}
\newcommand{\nderivada}[2]{\ensuremath{{#1}^{(#2)}}}


\newtheorem{defi}{{\it Definición}}[section]
\newtheorem{teo}{{\it Teorema}}[section]
\newtheorem{ejemplo}{{\it Ejemplo}}[section]
\newtheorem{propiedad}{{\it Propiedad}}[section]
\newtheorem{lema}{{\it Lema}}[section]
\newtheorem{cor}{Corolario}
\newtheorem{ejer}{Ejercicio}[section]

\newlist{milista}{enumerate}{2}
\setlist[milista,1]{label=\arabic*)}
\setlist[milista,2]{label=\arabic{milistai}.\arabic*)}
\newlength{\depthofsumsign}
\setlength{\depthofsumsign}{\depthof{$\sum$}}
\newcommand{\nsum}[1][1.4]{% only for \displaystyle
    \mathop{%
        \raisebox
            {-#1\depthofsumsign+1\depthofsumsign}
            {\scalebox
                {#1}
                {$\displaystyle\sum$}%
            }
    }
}
\def\scaleint#1{\vcenter{\hbox{\scaleto[3ex]{\displaystyle\int}{#1}}}}
\def\scaleoint#1{\vcenter{\hbox{\scaleto[3ex]{\displaystyle\oint}{#1}}}}
\def\scaleiint#1{\vcenter{\hbox{\scaleto[3ex]{\displaystyle\iint}{#1}}}}
\def\scaleiiint#1{\vcenter{\hbox{\scaleto[3ex]{\displaystyle\iiint}{#1}}}}
\def\bs{\mkern-12mu}

\newcommand{\Cancel}[2][black]{{\color{#1}\cancel{\color{black}#2}}}

\AtBeginDocument{\RenewCommandCopy\qty\SI}
\ExplSyntaxOn
\msg_redirect_name:nnn { siunitx } { physics-pkg } { none }
\ExplSyntaxOff

\title{Desarrollo de una función \\ {\large Polinomios de Legendre}\vspace{-3ex}}
\author{M. en C. Gustavo Contreras Mayén}
\date{ }

\pagestyle{fancy}
\fancyhf{}
\rhead{Curso MAF}
\lhead{\leftmark}
\rfoot{\thepage}
\setlength{\headheight}{16pt}%

\def\changemargin#1#2{\list{}{\rightmargin#2\leftmargin#1}\item[]}
\let\endchangemargin=\endlist 

\begin{document}
\maketitle
\fontsize{14}{14}\selectfont

\section{Desarrollo de una función.}

Del Tema 3 - Bases completas y ortogonales, reconocemos que la ecuación de Legendre es de la forma Sturm-Liouville con $p = 1 - x^{2}$, $q = 0$, $\lambda = \ell (\ell + 1)$ y $\omega = 1$, y que su intervalo natural es $[-1, 1]$. Ya que los polinomios ordinarios de Legendre $P_{\ell} (x)$ son regulares en los puntos extremos $x = \pm 1$, deben ser mutuamente ortogonales en este intervalo, es decir:
\begin{align}
\scaleint{6ex}_{\bs -1}^{1} P_{\ell} (x) \, P_{k} (x) \dd{x} = 0 \hspace{1cm} \mbox{ si $\ell \neq k$}
\label{eq:ecuacion_18_12}
\end{align}
Como ya se comentó previamente, la ortogonalidad mutua (y completitud) de $P_{\ell} (x)$ significa que cualquier función razonable $f (x)$ (es decir, una que satisfaga las condiciones de Dirichlet) puede expresarse en el intervalo de $\abs{x} < 1$ como una suma infinita de polinomios ordinarios de Legendre:
\begin{align}
f (x) = \nsum_{\ell = 0}^{\infty} a_{\ell} \, P_{\ell} (x)
\label{eq:ecuacion_013}
\end{align}
donde los coeficientes $a_{\ell}$ están dados por:
\begin{align}
a_{\ell} = \dfrac{2 \ell + 1}{2} \scaleint{6ex}_{\bs -1}^{1} f (x) \, P_{\ell} (x) \dd{x}
\label{eq:ecuacion_18_14}
\end{align}

\section{Expansión de una función con \texorpdfstring{$P_{n} (x)$}{Pn (x)}}
%Ref. Hassani (2009) Example 26.6.1
Queremos encontrar la expansión de Legendre de una función $f (x)$ definida por:
\begin{align*}
f (x) = \begin{cases}
V_{0} & \mbox{ si } 0 < x \leq 1 \\[0.5em]
- V_{0} & \mbox{ si } -1 \leq x < 0
\end{cases}
\end{align*}
Utilizamos la ecuación (\ref{eq:ecuacion_18_14}) para determinar los coeficientes:
\begin{align*}
a_{\ell} &= \dfrac{2 \, \ell + 1}{2} \scaleint{6ex}_{\bs -1}^{1} f (x) \, P_{\ell} (x) \dd{x} \\[0.5em]
&= \dfrac{2 \, \ell + 1}{2} \scaleint{6ex}_{\bs -1}^{0} \underbrace{f(x)}_{=-V_0}  P_{\ell} (x) \dd{x} + \dfrac{2 \, \ell + 1}{2} \scaleint{6ex}_{\bs 0}^{1} \underbrace{f(x)}_{=+V_0} \, P_{\ell} (x) \dd{x} \\[0.5em]
&= \dfrac{2 \, \ell + 1}{2} \, V_{0} \left[ - \scaleint{6ex}_{\bs -1}^{0} P_{\ell} (x) \dd{x} + \scaleint{6ex}_{\bs 0}^{1} P_{\ell} (x) \dd{x} \right]
\end{align*}
En la primera integral de la última línea, hacemos el cambio de variable $x = -y$, por lo que:
\begin{align*}
\scaleint{6ex}_{\bs -1}^{0} P_{\ell} (x) \dd{x} &= \scaleint{6ex}_{\bs +1}^{0} P_{\ell} (-y) (-\dd{y}) = \scaleint{6ex}_{\bs 0}^{1} P_{\ell} (-y) \dd{y} = \\[0.5em]
&= (-1)^{\ell} \, \scaleint{6ex}_{\bs 0}^{1} P_{\ell} (x) \dd{x}
\end{align*}
donde ocupamos una la propiedad de paridad de los polinomios ordinarios de Legendre:
\begin{align*}
P_{\ell} (-u) = (-1)^{\ell} \, P_{\ell} (u)
\end{align*}
Además de cambiar nuevamente la variable de integración de $y$ a $x$. Sustituimos en la expresión que determina los coeficientes:
\begin{align*}
a_{\ell} &= \dfrac{2 \, \ell + 1}{2} \, V_{0} \,  \big[ 1 - (-1)^{\ell} \big] \scaleint{6ex}_{\bs 0}^{1} P_{\ell} (x) \dd{x} \\[0.5em]
&= \dfrac{2 \, \ell + 1}{2} \, V_{0} \begin{cases}
0 & \mbox{ si } \ell \mbox{ es par} \\[0.5em]
2 \, \displaystyle \scaleint{6ex}_{\bs 0}^{1} P_{2k+1} (x) \dd{x} & \mbox{ si } \ell = 2 k + 1 
\end{cases}
\end{align*}
donde para $\ell$ impar se definió como $\ell = 2 \, k + 1$ con $k = 0, 1, 2, \ldots$.
\par
Queda por evaluar la integral del polinomio de Legendre de orden impar en el intervalo $(0, 1)$. Para ello, utilizamos la fórmula de Rodrigues:
\begin{align*}
\scaleint{6ex}_{\bs 0}^{1} P_{2k+1} (x) \dd{x} &= \dfrac{1}{2^{2k+1} \; (2 \, k +1)!} \scaleint{6ex}_{\bs 0}^{1} \dv[2k+1]{x} \left[ (x^{2} {-} 1)^{2k+1} \right] \dd{x} = \\[0.5em]
&= \dfrac{1}{2^{2k+1} \; (2 \, k +1)!} \; \dv[2k]{x} \left[ (x^{2} {-} 1)^{2k+1} \right] \eval_{0}^{1} = \\[0.5em]
&= \dfrac{1}{2^{2k+1} \; (2 \, k +1)!} \; \bigg[ \dv[2k]{x} \left[ (x^{2} {-} 1)^{2k+1} \right] \eval_{x=1} + \\[0.5em]
&- \dv[2k]{x} \left[ (x^{2} {-} 1)^{2k+1} \right] \eval_{x=0} \bigg]
\end{align*}
El primer término resulta ser cero, porque no hay un número suficiente de diferenciaciones para deshacerse de todos los factores de $(x^{2} - 1)$. Para el segundo término, observamos que $(x^{2} - 1)^{2k + 1}$ es un polinomio en $x$ cuyas derivadas de varios órdenes, serán potencias de $x$. Estas potencias devolverán cero en $x = 0$, excepto para el término constante (de potencia cero). Por lo tanto, vamos a utilizar la expansión binomial para $(x^{2} - 1)^{2k + 1}$, que es igual a $-(1 {-} x^{2})^{2k + 1}$:
\begin{align*}
\dv[2k]{x} \left[ (x^{2} {-} 1)^{2k+1} \right] \eval_{x=0} &= - \dv[2k]{x} \left[ \sum_{j=0}^{2k+1} \dfrac{(2k+1)!}{j! \; (2k + 1 - j)!} \, (-x^{2})^{j} \right] \eval_{x=0} = \\[0.5em]
&= - \sum_{j=0}^{2k+1} \dfrac{(2k+1)!}{j! \; (2k + 1 - j)!} (-1)^{j} \, \dv[2k]{x} \left( x^{2j} \right) \eval_{x=0}
\end{align*}
de donde se obtiene un término constante cuando $k = j$, todos los demás términos de la suma se anulan ya sea por tener  demasiadas diferenciaciones (cuando $j < k$, terminamos derivando constantes), o por tener muy pocas diferenciaciones (cuando $j > k$, una potencia de $x$ permanece y se evalúa como cero en $x = 0$). Por tanto:
\begin{align*}
\dv[2k]{x} \left[ (x^{2} {-} 1)^{2k+1} \right] \eval_{x=0} &= - \dfrac{(2k+1)!}{k! \; (2k + 1)!} (-1)^{k} \, \dv[2k]{x} \left( x^{2k} \right) \eval_{x=0} \\[0.5em]
&= \dfrac{(2k+1)!}{k! \; (2k + 1)!} (-1)^{k+1} \; (2k)!
\end{align*}
así entonces:
\begin{align}
\begin{aligned}[b]
\scaleint{6ex}_{\bs 0}^{1} P_{2k+1} (x) \dd{x} &= - \dfrac{1}{2^{2k+1} \, (2k + 1)} \, \bigg[ \dfrac{(2k + 1)!}{k! \, (k + 1)!} \, (-1)^{k+1} \, (2k)! \bigg] = \\[0.5em]
&= \dfrac{(-1)^{k} \, (2k)!}{2^{2k+1} \, (k+1)!}
\end{aligned}
\label{eq:ecuacion_26_49}
\end{align}
Entonces el coeficiente $a_{2k+1}$ se escribe como:
\begin{align*}
a_{2k+1} &= 2 \, \dfrac{2 \, (2 \, k + 1) + 1}{2} \; V_{0} \scaleint{6ex}_{\bs 0}^{1} P_{2k+1} (x) \dd{x} = \\[0.5em]
&= \dfrac{(-1)^{k} \, (4 \, k + 3)(2 \, k)!}{2^{2k+1} \; k! \; (k+1)!} \, V_{0}
\end{align*}
con $a_{\ell} = 0$ para $\ell$ par. La expansión en series de la función $f (x)$ se escribe como:
\begin{align*}
f(x) = \begin{cases}
V_{0} & \mbox{ si } 0 < x \leq 1 \\[0.5em]
-V_{0} & \mbox{ si } -1 \leq x < 0
\end{cases}
= V_{0} \nsum_{k=0}^{\infty} \dfrac{(-1)^{k}(4 \, k + 3)(2 \, k)!}{2^{2k+1} \; k! \; (k+1)!} P_{2k+1} (x)
\end{align*}
Expresando los primeros términos :
\begin{align*}
f (x) = V_{0} \left[ \dfrac{3}{2} \, P_{1}(x) - \dfrac{7}{8} \, P_{3}(x) + \dfrac{11}{16} \, P_{5}(x) - \cdots \right]
\end{align*}

\end{document}