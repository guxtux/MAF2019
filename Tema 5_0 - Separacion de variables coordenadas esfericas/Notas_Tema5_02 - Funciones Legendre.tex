\documentclass[12pt]{article}
\usepackage[utf8]{inputenc}
\usepackage[spanish,es-lcroman, es-tabla]{babel}
\usepackage[autostyle,spanish=mexican]{csquotes}
\usepackage{amsmath}
\usepackage{amssymb}
\usepackage{nccmath}
\numberwithin{equation}{section}
\usepackage{amsthm}
\usepackage{graphicx}
\usepackage{epstopdf}
\DeclareGraphicsExtensions{.pdf,.png,.jpg,.eps}
\usepackage{color}
\usepackage{float}
\usepackage{multicol}
\usepackage{enumerate}
\usepackage[shortlabels]{enumitem}
\usepackage{anyfontsize}
\usepackage{anysize}
\usepackage{array}
\usepackage{multirow}
\usepackage{enumitem}
\usepackage{cancel}
\usepackage{tikz}
\usepackage{circuitikz}
\usepackage{tikz-3dplot}
\usetikzlibrary{babel}
\usetikzlibrary{shapes}
\usepackage{bm}
\usepackage{mathtools}
\usepackage{esvect}
\usepackage{hyperref}
\usepackage{relsize}
\usepackage{siunitx}
\usepackage{physics}
%\usepackage{biblatex}
\usepackage{standalone}
\usepackage{mathrsfs}
\usepackage{bigints}
\usepackage{bookmark}
\spanishdecimal{.}

\setlist[enumerate]{itemsep=0mm}

\renewcommand{\baselinestretch}{1.5}

\let\oldbibliography\thebibliography

\renewcommand{\thebibliography}[1]{\oldbibliography{#1}

\setlength{\itemsep}{0pt}}
%\marginsize{1.5cm}{1.5cm}{2cm}{2cm}


\newtheorem{defi}{{\it Definición}}[section]
\newtheorem{teo}{{\it Teorema}}[section]
\newtheorem{ejemplo}{{\it Ejemplo}}[section]
\newtheorem{propiedad}{{\it Propiedad}}[section]
\newtheorem{lema}{{\it Lema}}[section]

\author{}
\title{Funciones de Legendre \\ {\large Matemáticas Avanzadas de la Física}}
\date{ }
\begin{document}
\maketitle
\fontsize{14}{14}\selectfont
%referencia: Sepúlveda: Lecciones de física matemática, sec. 3.3.4
% \section{La función de Laplace en coordenadas esféricas.}
% En coordenadas esféricas la ecuación
% \[ \laplacian \phi (r, \theta, \phi) = 0 \]
% se escribe
% \[ \dfrac{1}{r^{2}} \, \pdv{r} \left( r^{2} \, \pdv{\phi}{r} \right) + \dfrac{1}{r^{2} \, \sin \theta} \, \pdv{\theta} \left( \sin \theta \, \pdv{\phi}{\theta} \right) + \dfrac{1}{r^{2} \, \sin^{2} \theta} \pdv[2]{\phi}{\phi} = 0 \]
% utilizando la identidad
% \[ \dfrac{1}{r^{2}} \, \pdv{r} \left( r^{2} \, \pdv{\phi}{r} \right) = \dfrac{1}{r} \, \pdv[2]{r} (r \, \phi) \]
% podemos escribir
% \[ \dfrac{1}{r} \, \pdv[2]{r} (r \, \phi) + \dfrac{1}{r^{2} \, \sin \theta} \, \pdv{\theta} \left( \sin \theta \, \pdv{\phi}{\theta} \right) + \dfrac{1}{r^{2} \, \sin^{2} \theta} \pdv[2]{\phi}{\phi} = 0 \]
% Introduzcamos ahora la separación de varibales en la forma
% \[ \phi (r, \theta, \phi) =  \dfrac{U(r)}{r} \, Y(\theta, \phi) \]
% La parte radial ha s    ido escrita U(r)/r con el fin de simplificar el primer término de la ecuación diferencial. Posteriormente separaremos $Y(\theta, \phi)$ en un producto de funciones en $\theta$ y $\phi$. Se sigue entonces:
% \[ \dfrac{r^{2}}{U} \, \dv[2]{U}{r} + \dfrac{1}{Y} \left[ \dfrac{1}{\sin \theta} \, \pdv{}{\theta} \left( \sin \theta \, \pdv{Y}{\theta} \right) + \dfrac{1}{\sin^{2} \theta} \, \pdv[2]{Y}{\phi} \right] = 0 \]
% La ecuación ha sido separada en una parte radial y otra angular.
% \par
% El cuadrado del operador
% \[ \vb{L} = \dfrac{1}{i} \vb{r} \times \nabla \]
% es el negativo del corchete en la última ecuación, entonces se puede escribir:
% \[ \dfrac{r^{2}}{U} \, \dv[2]{U}{r} - \dfrac{L^{2} \, Y}{Y} = 0 \]
% De acuerdo con la técnica de separación de variables, cada sumando ha de ser igual a una constante. Por razones que serán claras más adelante, la separación de variable se escribe en la forma $\ell (\ell + 1)$, por lo que
% \[ r^{2} \, \dv[2]{U}{r} - \ell (\ell + 1) \, U = 0, \hspace{2cm} L^{2} \, Y = \ell(\ell + 1) \, Y  \]
% La ecuación radial es homogénea en $r$, es decir, es una ecuación del tipo de Euler, cuya solución es
% \[ U(r) = A \, r^{\ell +1} + \dfrac{B}{r^{\ell}} \hspace{1.5cm} \ell \geq 0 \]
% La ecuación $L^{2} \, Y = \ell (\ell + 1) \, Y$ tiene la forma explícita
% \[ \dfrac{1}{\sen \theta} \, \pdv{}{\theta} \left( \sin \theta \, \pdv{Y}{\theta} \right) + \dfrac{1}{\sin^{2} \theta} \, \pdv[2]{Y}{\phi} + \ell (\ell +1) \, Y = 0 \]
% Separando ahora las variables $\theta$ y $\phi$ en la forma $Y(\theta, \phi) = P(\theta) \, G(\phi)$, luego de re-emplazar y multiplicar por $\sin^{2} \theta / P \, G$:
% \[ \dfrac{\sin \theta}{P} \, \dv{}{\theta} \left( \sin \theta \, \dv{P}{\theta} \right) + \ell (\ell + 1) \, \sin^{2} \theta = - \dfrac{1}{G} \, \dv[2]{G}{\phi}  \]
% En consecuencia, escogiendo la constante de separación de modo que permita soluciones armónicas en $\phi$, que a su vez garanticen la continuidad en $\phi$ de la solución, se tiene que
% \[ \dfrac{1}{G} \, \dv[2]{G}{\phi} = -m^{2} \]
% cuya solución es:
% \begin{align*}
% G &= C \, \exp(i \, m \, \phi) + D \, \exp(-i \, m \, \phi), \hspace{1cm} m = 1, 2, \ldots, y \\
% G &= a \, \phi + b \hspace{6cm} m = 0
% \end{align*}
% La ecuación en $\theta$ será entonces:
% \[ \sin \theta \, \dv{}{\theta} \left( \sin \theta \, \dv{P}{\theta} \right) + P \, \ell (\ell + 1) \, \sin^{2} \theta - m^{2} \, P = 0 \]
% con la sustitución $x = \cos \theta$ obtenemos la \emph{ecuación asociada de Legendre}:
% \[ \dv{}{x} \left[ (1 -x^{2}) \, \dv{P(x)}{x} \right] - \dfrac{m^{2} \, P(x)}{1- x^{2}} + \ell (\ell + 1) \, P(x) = 0 \]
% con $m=0$ se obtiene la \emph{ecuación ordinaria de Legendre}:
% \[ \dv{}{x} \left[ (1 -x^{2}) \, \dv{P(x)}{x} \right] + \ell (\ell + 1) \, P(x) = 0 \]
% Podemos presentar la ecuación anterior como
% \[ (1 - x^{2}) \, \dv[2]{y}{x} - 2 \, x \, \, dv{y}{x} + C \, x = 0 \]
% donde $C$ es el parámetro de la separación de variables.
% \par
% Propongamos la serie de Frobenius:
% \[ y = \sum_{\lambda=0}^{\infty} a_{\lambda} \, x^{\lambda + k} , \hspace{1cm} a_{0} \neq 0 \]
% por lo que al derivar la serie anterior, tenemos que
% \[ \dot{y} = \sum_{\lambda=0}^{\infty} (\lambda + k ) \, a_{\lambda} \, x^{\lambda + k - 1} \]
% nuevamente derivamos la serie, para obtener
% \[ \ddot{y} = \sum_{\lambda=0}^{\infty} (\lambda + k )(\lambda + k - 1) \, a_{\lambda} \, x^{\lambda + k - 2} \]
% Al sustituir en la ecuación diferencial inicial, resulta que
% \[ \sum_{\lambda=0}^{\infty} (\lambda + k )(\lambda + k - 1) \, a_{\lambda} \, x^{\lambda + k - 2} + \sum_{\lambda=0}^{\infty} a_{\lambda} \, [-(\lambda + k)(\lambda + k + 1) + C] \, x^{\lambda + k} = 0 \]
% al bajar el límite inferior de la primera suma y elevando el índice $\lambda$ dentro de la suma, resulta
% \[ \sum_{\lambda=-2}^{\infty} a_{\lambda + 2} \, (\lambda + k + 2)(\lambda + k + 1) \, x^{\lambda + k} + \sum_{\lambda=0}^{\infty} a_{\lambda} \, [-(\lambda + k)(\lambda + k + 1) + C] \, x^{\lambda + k} = 0 \]
% \section{Polinomios asociados de Legendre.}
% Se obtienen de la ED de Legendre con $m \neq 0$. Son útiles para describir situaciones en donde no hay simetría azimutal.
% \par
% En vez de resolver mediante el método de Frobenius, ahora se obtendrá la ecuación asociada a partir de la ecuación:
% \[ (1 - x^{2}) \, \ddot{y} - 2 \, x \, \dot{y} + n(n +1) \, y = 0 \]
% donde $y = P_{n} (x)$.
% \par
% Derivando $m$ veces y utilizando la fórmula de Leibinz para la derivación de un producto, tenemos que
% \[ \dv[m]{}{x} (f \, g) = \sum_{s=0}^{\infty} \dfrac{m!}{(m - s)! \, s!} \, \dv[m-s]{f}{x} \, \dv[s]{g}{x} \]
% con $f = \ddot{y}$, $g = 1 - x^{2}$ (repitiendo para $f = \dot{y}, g = x)$ se obtiene:
% \[ (1 - x^{2}) \, \dv[m+2]{y}{x} - 2 \, x \, (m + 1) \, \dv[m+1]{y}{x} + [n(n+1) - m(m + 1)] \, \dv[m]{y}{x} = 0 \]
% si hacemos
% \[ u = (1 - x^{2})^{m/2} \, \dv[m]{y}{x} \]
% se puede escribir ahora
% \[ (1 - x^{2}) \, \ddot{u} - 2 \, x \, \dot{u} + \left[ n(n + 1) - \dfrac{m^{2}}{1 - x^{2}} \right] \, u = 0 \]
% que es la ecuación asociada de Legendre, cuya solución $u$ es
% \[ u = P_{n}^{m} (x) \]
% Por tanto:
% \[ \boxed{P_{n}^{m}(x) = (1 - x^{2})^{m/2} \dv[m]{P_{n}(x)}{x} } \]
% donde las $P_{n}^{m}(x)$ son la \emph{funciones asociadas de Legendre}. Nótese que $P_{n}^{0}(x) = P_{n}(x)$.
% \par
% Utilizando la fórmula de Rodrigues, se obtiene el siguiente resultado
% \[ \boxed{P_{n}^{m} (x) = \dfrac{1}{2^{n} \, n!} \, (1 - x^{2})^{m/2} \, \dv[n+m]{}{x} (x^{2} - 1)^{n}}\]
% Escrito de esta forma, $P_{n}^{m}$ admite valores negativos de $m$, sólo que si tomamos en cuenta que el orden de la derivación debe ser menor o igual al orden del polinomio, debe ser cierto que $n + m \leq 2 \, n$, por tanto $m \leq n$.
% \par
% Además $n + m \geq 0$ pues el orden de la derivación debe de ser positivo. Como $m$ puede ser positivo o negativo (y ha de ser entero, pues la derivación de orden $n + m$ ha de ser entera), tendremos que
% \begin{itemize}
% \item Si $m > 0$, entonces $m \leq n$.
% \item Si $m < 0$, entonces $- \abs{m} \leq n$, de donde $\abs{m} \geq -n$.
% \end{itemize}
% Por tanto, los valores de $m$ son del tipo $-n \leq m \leq n$, con $m$ y $n$ enteros, y $n$ es positivo. Esta restricción a valores enteros de $m$ y $n$ aparece en mecánica cuántica como la cuantización del momento angular.
% \par
% En consecuencia se tiene que
% \[ P_{n}^{m} (x) = 0 \hspace{2cm} \mbox{para } m > n\]
% La ecuación asociada de Legendre es autoadjunta; el conjunto $ \left\{ P_{n}^{m} (x) \right\}$ es ortogonal en $n$ dentro del intervalo $(-1, 1)$, siendo $1$ el factor de peso:
% \[ \boxed{\int_{-1}^{1} P_{n}^{m}(x) \, P_{n^{\prime}}^{m} (x) \, \dd x = \dfrac{2}{2 \,n + 1 } \, \dfrac{(n + m)!}{(n - m)!}  \, \delta_{n \, n^{\prime}} }\]
% \subsection{Relaciones de recurrencia.}
% En términos de los \emph{operadores escalera}, que son los corchetes a la derecha en las ecuaciones que siguen, se tiene que
% \begin{align*}
% \sqrt{1 - x^{2}} \, P_{n}^{m+1} = \left[ m \, x + (1 - x^{2})\, \dv{}{x} \right] \, P_{n}^{m} \\[1em]
% (n + m)(n - m + 1) \, \sqrt{1 - x^{2}} \, P_{n}^{m-1} = \left[ m \, x - (1 - x^{2}) \dv{}{x} \right] \, P_{n}^{m} \\[1em]
% (n + m) \, P_{n-1}^{m} = \left[ n \, x + (1 - x^{2}) \dv{}{x} \right] \, P_{n}^{m} \\[1em]
% (n -m + 1) \, P_{n+1}^{m} = \left[ (n+1) \, x - (1-x^{2}) \, \dv{}{x} \right] \, P_{n}^{m}
% \end{align*}
% \subsection{Armónicos esféricos.}
% Las soluciones $\exp(\pm i \, m \, \phi)$ forman una base ortogonal respecto al índice $m$ en $(0, 2 \pi)$, mientras que $P_{n}^{m} (x)$ son ortogonales en $(-1, 1)$, respecto al índice $n$.
% \par
% Es posible definir una nueva base ortogonal en $theta$ y $\phi$ respecto a los índices $n$ y $m$, para ser más precisos, se tiene unas base bi-ortogonal. Las nuevas funciones, \emph{ortonormales} sobre una superficie esférica y conocidas como \emph{armónicos esféricos} se definen como:
% \[ \boxed{Y_{nm} (\theta, \phi) = \sqrt{\dfrac{2n+1}{4\pi} \, \dfrac{(n-m)!}{(n+m)!}} \, P_{n}^{m} (\cos \theta) \, \exp{i \, m \, \phi} } \]
% El radical ha sido escogido de tal manera que $\left\{ Y_{nm}(\theta, \phi) \right\}$ sea una base ortonormal:
% \[ \int_{\phi=0}^{2 \pi} \int_{\phi=0}^{\pi} Y_{nm}^{*} (\theta, \phi) \, Y_{n^{\prime}m^{\prime}} (\theta, \phi) \, \sin \theta \dd \theta \, \dd \phi = \delta_{n n^{\prime}} \, \delta_{m m^{\prime}}  \]
% y como el ángulo sólido es $\dd \Omega = \sin \theta \, \dd \theta \, \dd \phi$, se puede escribir
% \[ \boxed{\int_{4\pi} Y_{nm}^{*} (\theta, \phi) \, Y_{n^{\prime} m^{\prime}} (\theta, \phi) \dd \Omega = \delta_{n n^{\prime}} \, \delta_{m m^{\prime}} } \]
% Los armónicos esféricos son funciones propias del operador $L^{2}$, con valores propios $n(n+1)$, así:
% \[ L^{2} \, Y_{nm} (\theta, \phi) =  n(n+1) \, Y_{nm}(\theta, \phi) \]
% Puesto que $\left\{ Y_{nm}(\theta, \phi) \right\}$ es una base completa, una función $f( \theta, \phi)$ puede expandirse en armónicos esféricos:
% \[ f(\theta, \phi) = \sum_{n=0}^{\infty} \sum_{m=-n}^{n} A_{nm} \, Y_{nm} (\theta, \phi) \]
% Por lo que la base $\left\{ Y_{nm}(\theta, \phi) \right\}$ permite expandir funciones definidas sobre la superficie de una esfera.
% \subsection*{Función generadora.}
% La función generadora de los polinomios de Legendre es:
% \[ \dfrac{1}{\sqrt{1 - 2 \, z \, u + z^{2}}} = \sum_{n \geq 0} z^{n} \, P_{n} (u), \hspace{2cm} z < 1 \]
% \subsection{Teorema de adición de los armónicos esféricos.}
% Este teorema nos dice que si se tiene un vector en los ángulos $(\theta, \phi)$ y otro en los ángulos $(\theta^{\prime}, \phi^{\prime})$, y además, es $\alpha$ el ángulo entre estos dos vectores, entonces se cumple:
% \[ P_{n} (\cos \alpha) = \sum_{m=-n}^{m=n} \dfrac{4 \pi}{2 n + 1} \, Y_{nm}^{*} \, (\theta^{\prime}, \phi^{\prime}) \, Y_{nm} (\theta, \phi) \]
% El ángulo $\alpha$ puede ser bastante complicado, por ejemplo, supongamos que tenemos los vectores
% \begin{align*}
% \va{r}_{1} =  r_{1} (\cos \phi_{1} \sin \theta_{1}, \sin \phi_{1} \sin \theta_{1}, \cos \theta_{1}) \\
% \va{r}_{2} =  r_{2} (\cos \phi_{2} \sin \theta_{2}, \sin \phi_{2} \sin \theta_{2}, \cos \theta_{2})
% \end{align*}
% Entonces el ángulo que forman, está dado por
% \[ \vu{r}_{1} \vdot \vu{r}_{2} = \cos \alpha = \sin \theta_{1} \, \sin \theta_{2} \, \cos (\phi_{1} - \phi_{2}) + \cos \theta_{1} \, \cos \theta_{2} \]
% \subsection*{Implicaciones del teorema de adición.}
% Si $\theta = \theta_{1} = \theta_{2}$ y $\phi = \phi_{1} = \phi_{2}$, entonces tenemos que
% \[ \sum_{m=-n}^{m=n} \abs{Y_{nm}(\theta, \phi)}^{2} = \dfrac{2 \, n + 1}{4 \pi} \]
% que se le llama regla de suma de los armónicos esféricos.
% \par
% Introduciendo la expresión del teorema de adición en la relación de completes, se encuentra que
% \[ \delta (\phi - \phi^{\prime}) \, \delta (\cos \theta - \cos \theta^{\prime}) = \sum_{n\geq 0} \dfrac{4 \, \pi}{2 \, n + 1} P_{n} (\cos \alpha) \]
% Considerando que la delta de Dirac en coordenadas esféricas es:
% \[ \delta (\va{r} - \va{r}^{\prime}) = \dfrac{1}{r^{2}} \, \delta (\phi - \phi^{\prime}) \, \delta (\cos \theta - \cos \theta^{\prime}) \]
% se llega al resultado
% \[ \delta (\va{r} - \va{r}^{\prime}) = \dfrac{4 \, \pi}{2 \, n + 1} \sum_{n \geq 0} \dfrac{\delta (\va{r} - \va{r}^{\prime})}{r^{2}} \, P_{n} (\vu{r} \vdot \vu{r}^{\prime}) \]
% Otra aplicación se encuentra al considerar la función
% \[ \dfrac{1}{\abs{\va{r}_{1} - \va{r}_{2}}} = \dfrac{1}{\sqrt{r_{1}^{2} - 2 \, r_{1} \, r_{2} \, \cos \alpha + r_{2}^{2}}} \]
% con $\alpha$ el ángulo entre $\va{r}_{1}$ y $\va{r}_{2}$ que satisface $\vu{r}_{1} \vdot \vu{r}_{2}$. Ahora, si $r_{1} \neq r_{2}$, se define $r_{<} = \min \{r_{1}, r_{2} \}$ y $r_{>} = \max \{r_{1}, r_{2} \}$, es claro que
% \[ \left( \dfrac{r_{<}}{r_{>}} \right) < 1 \]
% Entonces, usando estas definiciones y la función generatriz con
% \[ z = \dfrac{r_{<}}{r_{>}}, \hspace{2cm} u = \cos \alpha \]
% se tiene que
% \[ \dfrac{1}{\abs{\va{r}_{1} - \va{r}_{2}}} =  \dfrac{1}{r_{>} \, \sqrt{1 - 2 \, \left( \dfrac{r_{<}}{r_{>}} \right) \, \cos \alpha + \left( \dfrac{r_{<}}{r_{>}} \right)^{2}}} = \dfrac{1}{r_{>}} \sum_{n \geq 0} \left( \dfrac{r_{<}}{r_{>}} \right)^{n} \, P_{n} (\cos \alpha) \]
% Al usar el teorema de adición de los armónicos esféricos, se tiene que
% \[ \dfrac{1}{\abs{\va{r}_{1} - \va{r}_{2}}} = \sum_{n \geq 0} \sum_{m=-n}^{m=n} \dfrac{4 \, \pi}{2 \, n + 1} \, \left( \dfrac{r_{<}^{m}}{r_{>}^{n+1}} \right) \, Y_{nm}^{*} (\theta_{2}, \phi_{2}) \, Y_{nm} (\theta_{1}, \phi_{1}) \]
% que es la función de Green de la ecuación de Poisson en términos de armónicos esféricos.
% \subsection*{Paridad.}
% La transformación de paridad está definida por
% \[ (x, y, z) \to (-x, -y, -z) \]
% Ahora, en coordenadas esféricas, se tiene que
% \begin{align*}
% -x &= - r \, \sin \theta \, \cos \phi =  r \, \sin (\pi - \theta) \, \cos (\pi + \phi) \\
% -y &= - r \, \sin \theta \, \sin \phi =  r \, \sin (\pi - \theta) \, \sin (\pi + \phi) \\
% -z &= - r \, \cos \theta = \cos (\pi - \theta)
% \end{align*}
% Por lo que la transformaci{on de paridad en coordenadas esféricas toma la forma
% \[ (r, \theta, \phi) \to (r, \pi - \theta, \pi + \phi) \]
% Veamos ahora cómo transforman los polinomios asociados de Legendre bajo paridad. Primero notemos que bajo paridad se tiene que
% \[ u = \cos \theta \to \cos (\pi - \theta) = -u \]
% También se tiene que
% \[ P_{n} (\cos (\pi - \theta))  = P_{n}(-\cos \theta) = (-)^{n} \, P_{n} (\cos \theta)  \]
% y también
% \begin{align*}
% P_{n}^{m} (\cos (\pi - \theta)) &= (-)^{n} (1 - (-u)^{2})^{m/2} \, \dv[m]{}{(-u)} P_{n}(-u) \\
% &= (-)^{n+m} \, (-)^{m} \, (1 - u^{2})^{m/2} \, \dv[m]{}{u} P_{n}(u) \\
% &= (-)^{n+m} \, P_{n}^{m} (\cos \theta)
% \end{align*}
% Otra identidad de utilidad es
% \begin{align*}
% \exp(i(\pi + \phi)m) &= \exp(i \, \pi \, m) \, \exp(i \, \phi \, m) = \\
% &=  (\cos \pi + i \, \sin \pi)^{m} \, \exp(i \, \phi  \, m) = \\
% &=  (-)^{m} \, \exp(i \, \phi)
% \end{align*}
% Por lo tanto, ocupando la definición de armónicos esféricos, se tiene que
% \[ Y_{nm}(r, \pi - \theta, \pi + \phi) = (-)^{n} \, Y_{nm} (r, \theta, \phi)  \]
\section{Ecuación de Schödinger en coordenadas esféricas.}
En el tema de los polinomios de Laguerre se introdujo la ec. de Schödinger en coordenadas esféricas para estudiar el átomo de hidrógeno, mediante la técnica de separación de variable, se propuso una solución del tipo:
\begin{align*}
\psi (r, \theta, \phi) = R(r) \, Y(\theta, \phi)
\end{align*}
y de ahí se estudió la correspondiente ecuación radial del átomo de hidrógeno.
\par
La parte angular de la ecuación viene dada por la siguiente expresión:
\begin{align}
    \begin{aligned}
\dfrac{1}{Y(\theta, \phi) \, \sin \theta} \, \pdv{\theta} &\left( \sin \theta \pdv{\theta} \right) \, Y(\theta, \phi) + \\
&+ \dfrac{1}{Y(\theta, \phi) \, \sin^{2} \theta} \, \pdv[2]{\theta} \, Y(\theta, \phi) = - \ell (\ell + 1)
\end{aligned}
\label{eq:ecuacion_10_07}
\end{align}
Para resolver esta ecuación de dos variables, nuevamente ocupamos la técnica de separación de variables, proponiendo una solución de la forma:
\begin{align*}
Y(\theta, \phi) = f(\theta) \, g(\phi)
\end{align*}
entonces tendremos
\begin{align*}
\dfrac{1}{f(\theta) \, g(\phi) \, \sin \theta} \, \pdv{\theta} &\left( \sin \theta \pdv{\theta} \right) \, f(\theta) \, g(\phi) + \\
+& \dfrac{1}{f(\theta) \, g(\phi) \, \sin^{2} \theta} \, \pdv[2]{\theta} f(\theta) \, g(\phi) = - \ell (\ell + 1) \\[1em]
\Rightarrow \dfrac{1}{f(\theta) \, \sin \theta} \, \pdv{\theta} &\left( \sin \theta \pdv{\theta} \right) \, f(\theta) + \\
+& \dfrac{1}{ g(\phi) \, \sin^{2} \theta} \, \pdv[2]{\theta} \, g(\phi) = - \ell (\ell + 1)
\end{align*}
multiplicando la ecuación por $\sin^{2}$ y ordenando los términos, llegamos a
\begin{align*}
\dfrac{\sin \theta}{f(\theta)} \, \pdv{\theta} \left( \sin \theta \, \pdv{\theta} \right) \, f(\theta) + \ell (\ell + 1) \, \sin^{2} \theta + \dfrac{1}{g(\phi)} \, \pdv[2]{\phi} \, g(\phi) = 0
\end{align*}
Los dos primeros términos dependen sólo de $\theta$ y el último depende sólo de $\phi$. Sabemos que la única solución no trivial en la que la suma es cero, es aquella en donde los términos que dependen de una sola variable sean una constante, sea entonces $m^{2}$ la constante de separación, así
\begin{equation}
\dfrac{\sin \theta}{f(\theta)} \, \pdv{\theta} \left( \sin \theta \, \pdv{\theta} \right) \, f(\theta) + \ell (\ell + 1) \, \sin^{2} \theta = m^{2}
\label{eq:ecuacion_10_08}
\end{equation}
y para $\phi$
\begin{equation}
\dfrac{1}{g(\phi)} \, \pdv[2]{\phi} \, g(\phi) = -m^{2}
\label{eq:ecuacion_10_09}
\end{equation}
\section{La ecuación del ángulo azimutal.}
La solución a la ecuación del ángulo azimutal (\ref{eq:ecuacion_10_09}) es
\begin{equation}
g(\phi) = \exp(i \, m \, \phi) \hspace{0.5cm} \Rightarrow \hspace{0.5cm} g_{m} (\phi) = \exp(i \, m \, \phi)
\label{eq:ecuacion_10_10}
\end{equation}
se agregó el subíndice $m$ a $g(\phi)$, ya que queda claro que habrá tantas soluciones como valores permitidos de $m$.
\section{La ecuación del ángulo polar.}
Esta ecuación presenta una mayor relevancia que la anterior. La ec. (\ref{eq:ecuacion_10_08}) se puede escribir como
\begin{align*}
\sin \theta \, \dv{\theta} \left( \sin \theta \, \dv{\theta} \right) \, f(\theta) + \ell (\ell + 1) \, \sin^{2} \theta \, f(\theta) - m^{2} \, f(\theta) = 0
\end{align*}
que al simplificar el primer término, nos deja
\begin{equation}
\sin^{2} \theta \, \dv[2]{f(\theta)}{\theta} + \sin \theta \, \cos \theta \, \dv{f(\theta)}{\theta} + \ell (\ell + 1) \, \sin^{2} \theta \, f(\theta) - m^{2} \, f(\theta) = 0
\label{eq:ecuacion_10_11}
\end{equation}
Al hacer el cambio de variable $x = \cos \theta$, se requiere entonces tomar las derivadas con respecto a $x$ en vez de $\theta$, así
\begin{align*}
\dv{f(\theta)}{\theta} = \dv{f(x)}{x} \, \dv{x}{\theta} = \dv{f(x)}{x} \, (- \sin \theta) = -\sin \theta \, \dv{f(x)}{x}
\end{align*}
y para la segunda derivada
\begin{align*}
\dv[2]{f(\theta)}{\theta} &= \dv{\theta} \left( -\sin \theta \, \dv{f(x)}{x} \right) = \\
&= - \cos \theta \, \dv{f(x)}{x} + \sin^{2} \, \dv[2]{f(x)}{x}
\end{align*}
que al sustituir en la ec. (\ref{eq:ecuacion_10_11})
\begin{align*}
\sin^{2} \theta \left( \sin^{2} \, \dv[2]{f(x)}{x} - \cos \theta \, \dv{f(x)}{x} \right) &+ \sin \theta \, \cos \theta \left( - \sin \theta \, \dv{f(x)}{x} \right) + \\
&+ \ell (\ell + 1) \, \sin^{2} \theta \, f(x) - m^{2} \, f(x) = 0
\end{align*}
con lo que obtenemos una ecuación tanto con $\theta$ como con $x$, lo cual no es formalmente apropiado. En aras de simplificar el proceso, dividimos por $\sin^{2} \theta$, para obtener
\begin{align*}
\sin^{2} \theta \, \dv[2]{f(x)}{x} - \cos \theta \, \dv{f(x)}{x} - \cos \theta \, \dv{f(x)}{x} + \ell (\ell + 1) \, f(x) - \dfrac{m^{2}}{\sin^{2} \theta} \, f(x) = 0
\end{align*}
el cambio de variable se completa al sumar las dos primeras derivadas, usando $\cos \theta = x$ y $\sin^{2} = 1 - \cos^{2} = 1 - x^{2}$, por tanto
\begin{align*}
(1 - x^{2}) \dv[2]{f(x)}{x} - 2 \, x \, \dv{f(x)}{x} + \ell (\ell +1 ) \, f(x)  - \dfrac{m^{2}}{1 - x^{2}} \, f(x) = 0
\end{align*}
que es una ecuación de una sola variable:
\begin{align*}
(1 - x^{2}) f^{\prime \prime}(x) - 2 \, x \, f^{\prime}(x) + \ell (\ell +1 ) \, f(x)  - \dfrac{m^{2}}{1 - x^{2}} \, f(x) = 0
\end{align*}
que se conoce como \emph{la ecuación asociada de Legrende}. Si $m=0$, entonces
\begin{align*}
(1 - x^{2}) f^{\prime \prime}(x) - 2 \, x \, f^{\prime}(x) + \ell (\ell +1 ) \, f(x) = 0
\end{align*}
se tiene \emph{la ecuación de Legendre.}
%referencia: Riley: Mathematical methods for physics and engineering
\section{Funciones de Legendre}
La ecuación diferencial de Legendre tiene la forma
\begin{equation}
(1 - x^{2}) \, y^{\prime \prime} - 2 \, x \, y^{\prime} + \ell (\ell + 1) y = 0
\label{eq:ecuacion_18_01}
 \end{equation}
y tiene tres puntos singulares en $x = -1, 1, \infty$. Se presenta en diversos problemas de la física, en particular en problemas con simetría axial que involucra el operador $\laplacian$, en donde se expresa en coordenadas esféricas.
\par
Normalmente la variable $x$ en la ecuación de Legendre es el coseno del ángulo en coordenadas polares, por lo que $-1 \leq x \leq 1$. El párametro $\ell$ es un número real, y la solución a la ecuación (\ref{eq:ecuacion_18_01}) se le denomina \emph{función de Legendre}.
\par
Es posible demostrar que $x = 0$ es un punto ordinario, por lo que podemos esperar dos soluciones lineamente independientes de la forma
\begin{align*}
y = \sum_{n=0}^{\infty} a_{n} x^{n}
\end{align*}
Sustituimos para encontrar
\begin{align*}
\sum_{n=0}^{\infty} \left[ n (n-1) , a_{n} \, x^{n-2} - n (n-1) \, a_{n} \, x^{n} - 2 n \, a_{n} \, x^{n} + \ell (\ell + 1) \, a_{n}  \, x^{n} \right] = 0 
\end{align*}
donde al juntar los términos, tenemos
\begin{align*}
 \sum_{n=0}^{\infty} \left[ (n+2)(n+1) \, a_{n+2} - [ n(n+1) - \ell (\ell + 1) ] \, a_{n} \right] \,x^{n} = 0
\end{align*}
La relación de recurrencia es por tanto
\begin{equation}
a_{n+2} = \dfrac{[n (n+1) - \ell ( \ell + 1)]}{(n+1)(n+2)} \, a_{n}
\label{eq:ecuacion_18_02}
\end{equation}
para $n = 0, 1, 2, \ldots$
\par
Si elegimos $a_{0} = 1$ y $a_{1} = 0$ entonces obtenemos la solución
\begin{equation}
y_{1}(x) = 1 - \ell (\ell + 1) \dfrac{x^{2}}{2!} + (\ell - 2)\; \ell \; (\ell + 1)\;(\ell + 3) \dfrac{x^{4}}{4!} - \ldots
\label{eq:ecuacion_18_03}
\end{equation}
Mientras que si escogemos $a_{0} = 0$ y $ a_{1} = 1 $, encontramos la segunda solución
\begin{equation}
y_{2}(x) = x - (\ell - 1)(\ell + 2) \dfrac{x^{3}}{3!} + (\ell - 3) (\ell - 1)(\ell + 2)(\ell + 4) \dfrac{x^{5}}{5!} - \ldots
\label{eq:ecuacion_18_04}
\end{equation}
Aplicando la prueba de convergencia de razón, se encuentra que ambas series convergen para $\abs{x} < 1$, y su radio de convergencia es unitario, que representa la distancia al punto singular más cercano de la ecuación. Dado que la ecuación (\ref{eq:ecuacion_18_03}) contiene sólo potencias pares de $x$ y la ecuación (\ref{eq:ecuacion_18_04}) contiene sólo potencias impares, esas dos soluciones no pueden ser proporcionales una de la otra, por lo tanto, son linealmente independientes. De aquí, la solución general para la ecuación (\ref{eq:ecuacion_18_01}) y con $\abs{x} < 1$ es
\begin{align*}
y(x) = c_{1} y_{1} + c_{2} y_{2}
\end{align*}
\subsection{Funciones de Legendre para enteros $\ell$.}
En varios problemas de la física, el parámetro $\ell$ en la ecuación de Legendre - ec. (\ref{eq:ecuacion_18_01})- es un entero, es decir $\ell = 0, 1, 2, \ldots$. En ese caso, la relación de recurrencia - ec. (\ref{eq:ecuacion_18_02})- queda dada por
\begin{align*}
a_{\ell + 2} = \dfrac{[ \ell (\ell + 1) - \ell (\ell + 1) ]}{(\ell + 1)(\ell + 2)}  \, a_{\ell} = 0
\end{align*}
Esto es, la serie termina y obtenemos una solución con un polinomio de orden $\ell$. En particular, si $\ell$ es par, entonces $y_{1}(x)$ en la ecuación (\ref{eq:ecuacion_18_03}) se reduce a un polinomio, mientras que si $\ell$ es impar, lo mismo le ocurre a $y_{2}$ en la ecuación (\ref{eq:ecuacion_18_04}).
\par
Esas soluciones (adecuadamente normalizadas) son llamadas \emph{Polinomios de Legendre de orden $\ell$}, se escriben $P_{\ell}(x)$ y son válidas para todo valor $x$ finito. De manera convencional, se normaliza $P_{\ell}(x)$ de tal manera que $P_{\ell}(1) =  1$, y como consecuencia $P_{\ell}(-1) = (-1)^{\ell}$. Los primeros polinomios se construyen fácilmente y están dados por:
\begingroup
\setlength{\tabcolsep}{20pt}
\renewcommand{\arraystretch}{2}
\begin{center}
\begin{tabular}{l l}
$P_{0}(x) = 1 $ & $P_{1}(x) = 1 $ \\
$P_{2}(x) = \dfrac{1}{2} (3 x^{2} - 1)$ & $P_{3}(x) = \dfrac{1}{2} (5 x^{2} - 3 x)$ \\ 
$P_{4}(x) = \dfrac{1}{8} (35 x^{4} - 30 x^{2} + 3)$ & $P_{5}(x) = \dfrac{1}{8} (63 x^{5} - 70 x^{3} + 15 x)$
\end{tabular}
\end{center}
\endgroup
A pesar de que si $\ell$ es un entero par o impar, respectivamente para $y_{1}(x)$ - ec. (\ref{eq:ecuacion_18_03}) - o $y_{2}(x)$ - ec. (\ref{eq:ecuacion_18_04}), se termina dando un múltiplo del correspondiente polinomio de Legendre $P_{\ell}(x)$, la otra serie en cada caso no termina y por tanto converge sólo para $\abs{x} < 1$.
\par
De acuerdo si $\ell$ es par o impar, se definen las \emph{funciones de Legendre de segunda clase} como $Q_{\ell}(x) =  \alpha_{\ell} \, y_{2}(x)$ o $Q_{\ell}(x) = \beta_{\ell} \, y_{1}(x)$, respectivamente, donde las constantes $\alpha_{\ell}$ y $\beta_{\ell}$ toman los valores
\begin{align}
\alpha_{\ell} &=& \dfrac{(-1)^{\ell/2} \; 2^{\ell} \; [(\ell / 2)!]^{2}}{\ell!} \hspace{3.5cm} \text{ para $\ell$ par} \label{eq:ecuacion_18_05}\\[1em]
\beta_{\ell} &=& \dfrac{(-1)^{(\ell + 1)/2} \; 2^{\ell - 1} \; \lbrace \left[ (\ell - 1) /2 \right] ! \rbrace^{2}}{\ell!} \hspace{1cm} \text{ para $\ell$ impar} \label{eq:ecuacion_18_06}
\end{align}
La normalización de los factores se elige de tal manera que $Q_{\ell}(x)$ obedece la misma relación de recurrencia de $P_{\ell}(x)$.
\par
La solución general para la ecuación de Legendre para enteros $\ell$ es por tanto
\begin{equation}
y(x) = c_{1} \, P_{\ell}(x) + c_{2} \, Q_{\ell} (x) 
\label{eq:ecuacion_18_07}
\end{equation}
Donde $P_{\ell}(x)$ es un polinomio de orden $\ell$, que converge para cualquier $x$, y $Q_{\ell}(x)$ es una serie infinita que converge sólo si $\abs{x} < 1$.
% \\
% Usando el método del Wronkisano, podemos obtener una forma cerrada para $Q_{\ell}(x)$:
% \\
% Una segunda solución para la ecuación de Legendre -ec. \ref{eq:ecuacion_001}), con $\ell = 0$ es
% \begin{eqnarray}
% y_{2}(x) &=& P_{0}(x) \int^{x} \dfrac{1}{[P_{0}(u)]^{2}} \exp \left( \int^{u} \dfrac{2v}{1-v^{2}} dv \right) du \nonumber \\
% &=& \int^{x} \exp [ - \ln (1 - u^{2}) ] du \nonumber \\
% &=& \int^{x} \dfrac{du}{(1-u^{2})} = \frac{1}{2} \ln \left( \dfrac{1+x}{1-x} \right) \label{eq:ecuacion_008}
% \end{eqnarray}
% En la segunda línea hemos utilizado el hecho de que $P_{0}(x)=1$.
% \\
% Lo que queda es ajustar la normalización de esta solución para que se corresponda con la ecuación (\ref{eq:ecuacion_005}). Expandiendo el logaritmo en la ec. (\ref{eq:ecuacion_008}) como una serie de Maclaurin, obtenemos
% \[ y_{2}(x) = x + \dfrac{x^{3}}{3} + \dfrac{x^{5}}{5} + \cdots \]
% Comparando esto con la expresión para $Q_{0}(x)$, usando la ec. (\ref{eq:ecuacion_004}) con $\ell = 0$ y normalizando -ec. (\ref{eq:ecuacion_005})-, encontramos que $y_{2}$ está correctamente normalizada, así
% \[ Q_{0} (x) = \dfrac{1}{2} \ln \left( \dfrac{1+x}{1-x} \right) \]
% Usando el mismo método para $\ell = 1$, tenemos que
% \[ Q_{1} (x) =  \frac{1}{2} x \ln \left( \dfrac{1+x}{1-x} \right) - 1 \]
% Se pueden encontrar formas cerradas para $Q_{\ell}(x)$ de mayor orden, usando la relacion de recurrencia.
\subsection{Propiedades de los Polinomios de Legendre.}
Como se mencionó anteriormente, cuando encontramos problemas físicos en donde la variable $x$ en la ecuación de Legendre es el coseno del ángulo polar $\theta$ en coordenadas esféricas, y entonces se requiere que la solución $y(x)$ que sea regular en $x = \pm 1$, que corresponde a $\theta = 0$ o $\theta = \pi$. Para que esto ocurra, requerimos que la ecuación tenga una solución polinomial, así el valor de $\ell$ debe ser un entero.
\par
Por otra parte, también requerimos que el coeficiente $c_{2}$ de la función $Q_{\ell}(x)$ en la ecuación (\ref{eq:ecuacion_18_07}) sea nulo, ya que $Q_{\ell}(x)$ es singular en $x = \pm 1$, como resultado de que la solución general es un múltiplo del polinomio de Legendre $P_{\ell}(x)$.
\subsubsection{Fórmula de Rodrigues.}
Como una ayuda para definir nuevas propiedades de los polinomios de Legendre, desarrollamos la representación de Rodrigues de éstas funciones. La \emph{fórmula de Rodrigues} para el $P_{\ell} (x)$ es
\begin{equation}
P_{\ell} (x) = \dfrac{1}{2^{\ell} \; \ell !} \, \dv[\ell]{x}  (x^{2} - 1)^{\ell}
\label{eq:ecuacion_18_09}
\end{equation}
\subsubsection{Ortogonalidad.}
De temas anteriores vemos que la ecuación de Legendre es de la forma Sturm-Liouville con $p = 1 - x^{2}$, $q = 0$, $\lambda = \ell (\ell + 1)$ y $\omega = 1$, y que su intervalo natural es $[-1, 1 ]$. Ya que los polinomios de Legendre $P_{\ell} (x)$ son regulares en los puntos extremos $x = \pm 1$, deben ser mutuamente ortogonales en este intervalo, es decir,
\begin{equation}
\int_{-1}^{1} P_{\ell}(x) \, P_{k}(x) dx = 0 \hspace{1cm} \text{ si } \ell \neq k
\label{eq:ecuacion_18_12}
\end{equation}
Como ya se comentó previamente, la ortogonalidad mutua (y completes) de  $P_{\ell} (x)$ significa que cualquier función razonable $f(x)$ (es decir, una que satisfaga las condiciones de Dirichlet) puede expresarse en el intervalo de $\abs{x} <1$ como una suma infinita de polinomios de Legendre,
\begin{equation}
f(x) = \sum_{\ell = 0}^{\infty} a_{\ell} \, P_{\ell} (x)
\label{eq:ecuacion_18_13}
\end{equation}
donde los coeficientes $a_{\ell}$ están dado por
\begin{equation}
a_{\ell} = \dfrac{2 \, \ell + 1}{2} \, \int_{-1}^{1} f(x) \, P_{\ell} (x) \dd{x}
\label{eq:ecuacion_18_14}
\end{equation}
\subsubsection{Función generatriz.}
Una manera útil para manipular y estudiar las secuencias de funciones o cantidades etiquetadas por una variable entera (en el caso de los polinomios de Legendre $P_{\ell} (x)$ están etiquetados por $\ell$), es mediante una función generatriz. 
\par
La función generatriz tiene quizás su mayor utilidad en el ámbito de la teoría de la probabilidad, sin embargo, también es de gran conveniencia en nuestro estudio.
\par
La función generatriz para decirlo, es una serie de funciones $f_{n} (x)$ para $n = 0, 1, 2,\ldots$, es una función $G (x, h)$ que contiene tanto a $x$, como una variable ficticia $h$, de tal manera que
\begin{align*}
G(x,h) = \sum_{n=0}^{\infty} f_{n} (x) \, h^{n}
\end{align*}
es decir, $f_{n}(x)$ es el coeficiente de $h^{n}$ en la expansión de $G$ en potencias de $h$. La utilidad de esta manera de trabajar la función, está en el hecho de que a veces es posible encontrar una forma cerrada para $G(x,h)$.
\par
En el caso de los polinomios de Legendre, usemos las funciones $P_{n}(x)$ definidas por
\begin{equation}
G(x,h) = (1 - 2 \, x \, h + h^{2})^{-1/2} =  \sum_{n=0}^{\infty} P_{n}(x) \, h^{n}
\label{eq:ecuacion_18_15}
\end{equation}
% Como veremos las funciones así definidas son idénticas a los polinomios de Legendre y la función $(1 - 2 \, x \, h + h^{2})^{-1/2}$ es de hecho la función generatriz para ellos. En el proceso también vamos a deducir varias relaciones útiles entre los diferentes polinomios y sus derivadas.
% \\
% Hacemos la anotación de que $d P_{n}(x) / dx$ es $P^{\prime} n$, derivamos la ecuación (\ref{eq:ecuacion_015}) con respecto a $x$ y obtenemos
% \begin{equation}
% h (1 - 2xh + h^{2})^{-3/2} = \sum P^{\prime}_{n} \; h^{n}
% \label{eq:ecuacion_016}
% \end{equation}
% También derivamos la ecuación (\ref{eq:ecuacion_015}) con respecto a $h$ por lo que
% \begin{equation}
% (x-h) (1- 2xh + h^{2})^{-3/2} = \sum n \; P_{n} \; h^{n-1}
% \label{eq:ecuacion_017}
% \end{equation}
% La ecuación (\ref{eq:ecuacion_016}) puede re-escribirse usando la ecuación (\ref{eq:ecuacion_015}) como
% \[ h \sum P_{n} \; h^{n} =  (1 - 2xh + h^{2}) \sum P^{\prime}_{n} h^{n} \]
% igualando los coeficientes de $h^{n+1}$, obtenemos la relación de recurrencia
% \begin{equation}
% P_{n} = P^{\prime}_{n+1} - 2x \; P^{\prime}_{n} + P^{\prime}_{n-1}
% \label{eq:ecuacion_018}
% \end{equation}
% Las ecuaciones (\ref{eq:ecuacion_016}) y (\ref{eq:ecuacion_017}) pueden combinarse como
% \[ (x-h) \sum P^{\prime}_{n} \; h^{n} = h \sum n \; P_{n} \; h^{n-1} \]
% donde el coeficiente de $h^{n}$ nos proporciona otra relación de recurrencia
% \begin{equation}
% x P^{\prime}_{n} - P^{\prime}_{n-1} =  n \; P_{n}
% \label{eq:ecuacion_019}
% \end{equation}
% eliminando $P^{\prime}_{n-1}$ entre las ecuaciones (\ref{eq:ecuacion_018}) y (\ref{eq:ecuacion_019}), el resulta que se obtiene es
% \begin{equation}
% (n+1)P_{n} = P^{\prime}_{n+1} - x \; P^{\prime}_{n}
% \label{eq:ecuacion_020}
% \end{equation}
% Si tomamos el resultado de la ecuación (\ref{eq:ecuacion_020}) re-emplazando $n$ por $n-1$ y sumamos $x$ veces, obtenemos
% \begin{equation}
% (1 - x^{2}) P^{\prime}_{n} = n \; (P_{n-1} - x P_{n})
% \label{eq:ecuacion_021}
% \end{equation}
% Finalmente, derivamos ambos lados con respecto a $x$ y usamos el resultado de la ecuación (\ref{eq:ecuacion_019}) para tener
% \[ \begin{split}
% (1-x^{2}) P^{\prime \prime}_{n} - 2x P^{\prime}_{n} &=  n [ (P^{\prime}_{n-1} - x P^{\prime}_{n}) - P_{n} ] \\
% &= n (-n P_{n} - P_{n}) \\
% &= -n (n+1) P_{n}
% \end{split} \]
% por lo que los $P_{n}$ definidos en la ecuación (\ref{eq:ecuacion_015}), satisfacen la ecuación de Legendre.
% \\
% El ejemplo anterior muestra 
Se tiene que las funciones $P_{n} (x)$ definidas por la ecuación (\ref{eq:ecuacion_18_15}) satisfacen la ecuación de Legendre con $\ell = n$ (un entero) y también de (\ref{eq:ecuacion_18_15}), estas funciones son regulares en $x = \pm 1$. Por lo tanto $P_{n}$ debe ser un múltiplo del n-ésimo polinomio de Legendre. Por lo tanto, sólo queda verificar la normalización. Esto se hace fácilmente en $x = 1$, cuando G se convierte en
\begin{align*}
G(1,h) = [(1 - h)^{2}]^{-1/2} =  1 + h + h^{2} + \cdots
\end{align*}
y podemos ver que todo $P_{n}$ así definido, se tiene $P_{n} (1) = 1$ como se requiere, por tanto son idénticos a los polinomios de Legendre.
\par
Un uso particular de la función generatriz (\ref{eq:ecuacion_18_15}) es la representación del inverso de la distancia entre dos puntos en el espacio tridimensional en términos de polinomios de Legendre. Si dos puntos $\vb{r}$ y $\vb{r}^{\prime}$ se encuentran a distancias $r$ y $r^{\prime}$, respectivamente, desde el origen, con $r^{\prime} < r$, se tiene
\begin{align}
\dfrac{1}{\abs{\vb{r} - \vb{r}^{\prime}}} &= \dfrac{1}{(r^{2} + r^{\prime \: 2} - 2 \, r \, r^{\prime} \, \cos \theta)^{1/2}} \nonumber \\[1em]
&= \dfrac{1}{r [ 1 - 2 (r^{\prime}/r) \cos \theta + (r^{\prime}/r)^{2}]^{1/2}} \nonumber \\[1em]
&= \dfrac{1}{r} \sum_{\ell = 0}^{\infty} \left( \dfrac{r^{\prime}}{r} \right)^{\ell} \, P_{\ell} (\cos \theta)
\label{eq:ecuacion_18_22}
\end{align}
donde $\theta$ es el ángulo entre los dos vectores de posición $\vb{r}$ y $\vb{r}^{\prime}$. Si $r^{\prime} > r$, entonces $r$ y $r^{\prime}$ deben de intercambiarse en la ecuación (\ref{eq:ecuacion_18_22}) o de lo contrario, la serie no converge.
\par
Este resultado puede ser utilizado por ejemplo, para escribir el potencial electrostático en un punto $\vb{r}$ debido a una carga $q$ en el punto $\vb{r}^{\prime}$. Entonces, en el caso $r^{\prime} < r$, se tiene que
\begin{align*}
V(\vb{r}) =  \dfrac{q}{4 \, \pi \, \epsilon_{0} \, r} \sum_{\ell = 0}^{\infty} \left( \dfrac{r^{\prime}}{r} \right)^{\ell} \, P_{\ell} (\cos \theta)
\end{align*}
Vemos el caso especial cuando la carga está en el origen, y $r^{\prime} =0$, entonces el término $\ell =0$ en la serie es no nulo, y al expresión se reduce a la forma ya conocida 
\begin{align*}
V(\vb{r}) = \frac{q}{4 \, \pi \, \epsilon_{0} \, r}
\end{align*}
\subsubsection*{Relaciones de recurrencia.}
%En nuestro análisis previo de la función generatriz, derivamos varias relaciones de recurrencia útiles que satisfacen los polinomios de Legendre $P_{n} (x)$. 
En particular, a partir de la función generatriz, podemos obtener las siguientes relaciones de recurrencia
\begin{align*}
P_{n} &= P_{n+1}^{\prime} - 2 \, x \; P_{n}^{\prime} + P_{n-1}^{\prime} \\[1em]
x \, P_{n}^{\prime} - P_{n-1}^{\prime} &=  n \, P_{n} \\[1em]
(n + 1) \, P_{n} &= P_{n+1}^{\prime} - x \, P_{n}^{\prime} \\[1em]
P_{n+1}^{\prime} + P_{n-1}^{\prime} &=  P_{n} + 2 \, x \, P_{n}^{\prime} \\[1em]
(1 - x^{2}) \, P_{n+1}^{\prime} &= n \, (P_{n-1} - x \, P_{n}) \\[1em]
(2 \, n + 1) \, P_{n} &= P_{n+1}^{\prime} - P_{n-1}^{\prime}
\end{align*}
% De las ecuaciones (\ref{eq:ecuacion_019}) a (\ref{eq:ecuacion_021}) tenemos las siguientes relaciones de recuerrencia con tres términos:
% \begin{eqnarray}
% P^{\prime}_{n+1} &=& (n+1) \; P_{n} + x \; P^{\prime}_{n} \\
% P^{\prime}_{n-1} &=& -n \; P_{n} + x \; P^{\prime}_{n} \\
% (1 - x^{2}) P^{\prime}_{n+1} &=& n \; (P_{n-1} - x \; P_{n}) \\
% (2n+1) P_{n} &=& P^{\prime}_{n+1} - P^{\prime}_{n-1}
% \end{eqnarray}
%Referncia Hassani. Mathematical methods for students of physics. Example 26.6.1
\subsubsection{Ejemplos.}
\textbf{Ejemplo 1.} Queremos encontrar la expansión de Legendre de una función $f(x)$ definida por
\begin{align*}
f(x) = \begin{cases}
V_{0} & \text{ si } 0 < x \leq 1 \\
- V_{0} & \text{ si } -1 \leq x < 0
\end{cases}
\end{align*}
Utilizamos las ecuaciones (\ref{eq:ecuacion_18_13}) y (\ref{eq:ecuacion_18_14}) para determinar los coeficientes:
\begin{align*}
a_{\ell} &= \dfrac{2 \ell + 1}{2} \int_{-1}^{1} f(x) P_{\ell} (x) \dd{x} \\[1em]
&= \dfrac{2 \ell + 1}{2} \int_{-1}^{0} \underbrace{f(x)}_{=-V_0}  P_{\ell} (x) \dd{x} + \dfrac{2 \ell + 1}{2} \int_{0}^{1} \underbrace{f(x)}_{=+V_0}  P_{\ell} (x) \dd{x} \\[1em]
&= \dfrac{2 \ell + 1}{2} V_{0} \left[ - \int_{-1}^{0} P_{\ell} (x) \dd{x} + \int_{0}^{1} P_{\ell} (x) \dd{x} \right]
\end{align*}
En la primera integral de la última línea, hacemos el cambio de variable $x = -y$, por lo que
\begin{align*}
\int_{-1}^{0} P_{\ell} (x) \dd{x} = \int_{+1}^{0} P_{\ell} (-y) (-\dd{y}) = \int_{0}^{1} P_{\ell} (-y) \dd{y} = (-1)^{\ell} P_{\ell} (x) \dd{x}
\end{align*}
donde ocupamos una de la propiedad de paridad de los polinomios de Legendre
\begin{align*}
P{\ell} (-u) = (-1)^{\ell} \, P_{\ell} (u)
\end{align*}
Sustituimos en la expresión para los coeficientes
\begin{align*}
a_{\ell} &= \dfrac{2 \ell + 1}{2} \, V_{0} \,  [1 - (-1)^{\ell} ] \int_{0}^{1} P_{\ell} (x) \dd{x} \\[1em]
&= \dfrac{2 \ell + 1}{2} \, V_{0} \begin{cases}
0 & \text{ si } \ell \text{ es par} \\
\displaystyle 2  \, \int_{0}^{1} P_{2 k + 1} (x) \dd{x} & \text{ si } \ell = 2 k + 1 
\end{cases}
\end{align*}
donde para $\ell$ impar se definió como $\ell = 2 k + 1$ con $k = 0, 1, 2, \ldots$.
\par
Queda por evaluar la integral del polinomio de Legendre de orden impar en el intervalo $(0, 1)$. Para ello, utilizamos la fórmula de Rodrigues
\begin{align*}
\int_{0}^{1} P_{2k+1} (x) \dd{x} &= \dfrac{1}{2^{2k+1} \; (2k +1)!} \int_{0}^{1} \dv[2k+1]{x} \left[ (x^{2} - 1)^{2k+1} \right] \dd{x} \\[1em]
&= \dfrac{1}{2^{2k+1} \; (2k +1)!} \; \dv[2k]{x} \left[ (x^{2} - 1)^{2k+1} \right] \eval_{0}^{1} \\[1em]
&= \dfrac{1}{2^{2k+1} \; (2k +1)!} \; \left\{ \dv[2k]{x} \left[ (x^{2} - 1)^{2k+1} \right] \eval_{x=1} + \right. \\[1em]
&- \left. \dv[2k]{x} \left[ (x^{2} - 1)^{2k+1} \right] \eval_{x=0} \right\}
\end{align*}
El primer término resulta ser cero, porque no hay un número suficiente de diferenciaciones para deshacerse de todos los factores de $(x^{2} - 1)$. Para el segundo término, observamos que $(x^{2} - 1)^{2k + 1}$ es un polinomio en $x$ cuyos derivadas de varios órdenes, serán potencias de $x$. Estas potencias devolverán cero en $x = 0$, excepto para el término constante (de potencia cero). Por lo tanto, vamos a utilizar la expansión binomial para $(x^{2} - 1)^{2k + 1}$, que es igual a $-(1 - x^{2})^{2k + 1}$:
\begin{align*}
\dv[2k]{x} \left[ (x^{2} - 1)^{2k+1} \right] \eval_{x=0} &= - \dv[2k]{x} \left[ \sum_{j=0}^{2k+1} \dfrac{(2k+1)!}{j! \, (2k + 1 - j)!} \, (-x^{2})^{j} \right] \eval_{x=0} \\[1em]
&= - \sum_{j=0}^{2k+1} \dfrac{(2k+1)!}{j! \; (2k + 1 - j)!} \, (-1)^{j} \dv[2k]{x} \left( x^{2j} \right) \eval_{x=0}
\end{align*}
de donde se obtiene un término constante cuando $k = j$, todos los demás términos de la suma se anulan ya sea por tener  demasiadas diferenciaciones (cuando $j <k$, terminamos derivando constantes), o por tener muy pocas diferenciaciones (cuando $j> k$, una potencia de $x$ permanece y se evalúa como cero en $x = 0$). Por tanto
\begin{align*}
\dv[2k]{x} \left[ (x^{2} - 1)^{2k+1} \right] \eval_{x=0} &= - \dfrac{(2k+1)!}{k! \, (k+1)!} \, (-1)^{k} \dv[2k]{x} \left( x^{2k} \right) \eval_{x=0} \\[1em]
&= \dfrac{(2k+1)!}{k! \; (k + 1)!} \, (-1)^{k+1} \, (2 \, k)!
\end{align*}
Entonces el coeficiente $a_{2k+1}$ se escribe como
\begin{align*}
a_{2k+1} = 2 \, \dfrac{2 \, (2k+1) + 1}{2} \, V_{0} \, \int_{0}^{1} P_{2k+1} (x) \dd{x} = \dfrac{(-1)^{k} \, (4k+3) \, (2k!)}{2^{2k+1} \, k! \, (k+1)!} \, V_{0}
\end{align*}
con $a_{\ell} = 0$ para $\ell$ par. La expansión final en series se escribe como
\begin{align*}
f(x) = \begin{cases}
V_{0} & \text{ si } 0 < x \leq 1 \\
-V_{0} & \text{ si } -1 \leq x < 0
\end{cases}
= V_{0} \, \sum_{k=0}^{\infty} \dfrac{(-1)^{k} \,(4k+3) \, (2k!)}{2^{2k+1} \, k! \, (k+1)!} \, P_{2k+1} (x)
\end{align*}
Expresando los primeros términos:
\begin{align*}
f(x) = V_{0} \, \left[ \dfrac{3}{2} P_{1}(x) - \dfrac{7}{8} P_{3}(x) + \dfrac{11}{16} P_{5}(x) - \cdots \right]
\end{align*}
\textbf{Ejercicio a cuenta: } Calcula la expansión mediante los polinomios de Legendre de la delta de Dirac.
\par
Solución:
\begin{align*}
\delta(x) = \sum_{k=0}^{\infty} (-1)^{k} \, \dfrac{(4k + 1) \, (2k!)}{2^{2k+1} \, (k!)^{2}} \, P_{2k} (x)
\end{align*}
\textbf{Ejemplo 2.}

Para encontrar la solución más general con simetría azimutal de la ecuación de Laplace en coordenadas esféricas, multiplicamos la solución radial y la solución angular (polinomios de Legendre) para cada $k$ y sumamos sobre todos los valores posibles de $k$:
\begin{equation}
\Phi (r, \theta) = \sum_{k=0}^{\infty} \left( A_{k} \; r^{k} + \dfrac{B_{k}}{r^{k+1}} \right) \; P_{k} (\cos \theta)
\label{eq:ecuacion_029a}
\end{equation}
donde $A_{k}$ y $B_{k}$ son constantes y se ha sustituido $\cos \theta$ por $x$.
\par
\textbf{Problema: } Dos hemisferios sólidos conductores de calor de radio $a$, separados por un hueco aislante muy pequeño, forman una esfera. Cada mitad de la esfera está en contacto - por fuera - con dos baños de calor (infinitos) a temperaturas $T_{0}$ y $-T_{0}$. Queremos encontrar la distribución de temperatura $T (r, \theta, \phi)$ en el interior de la esfera.
\begin{figure}[H]
\centering
\includestandalone{Figuras/esfera_2}
\caption{Dos semiesferas separadas infinitesimalmente a temperaturas opuestas.}
\label{fig:figura2}
\end{figure}
Elegimos un sistema de coordenadas esféricas en donde el origen coincide con el centro de la esfera y el eje polar es perpendicular al plano ecuatorial. El hemisferio con temperatura $T_{0}$ suponemos que es el hemisferio norte.
\par
Dado que el problema tiene simetría azimutal, $T$ es independiente de $\phi$, y podemos escribir inmediatamente la solución general de la ecuación (\ref{eq:ecuacion_029a}). Sin embargo, dado que el origen se encuentra en la región de interés, es necesario excluir a todos los potencias negativas de $r$. Esto se logra dejando que todos los coeficientes de $B$ se anulen. Por lo tanto, tenemos
\begin{equation}
T(r, \theta) = \sum_{n=0}^{\infty} A_{n} \, r^{n} \, P_{n} (\cos \theta)
\label{eq:ecuacion_050a}
\end{equation}
Quedando pendiente calcular las constantes $A_{n}$, pero notemos que
\begin{align*}
T(a, \theta) = 
\begin{cases}
T_{0} & \text{ si } 0 \leq \theta < \dfrac{\pi}{2} \\
-T_{0} & \text{ si } \dfrac{\pi}{2} < \theta \leq \pi 
\end{cases}
\end{align*}
en términos de $u= \cos \theta$, podemos escribir
\begin{align*}
T(a, u) = 
\begin{cases}
T_{0} & \text{ si } -1 \leq u < 0 \\
-T_{0} & \text{ si } 0 < u \leq 1 
\end{cases}
\end{align*}
sustituyendo en la ecuación (\ref{eq:ecuacion_050a}), obtenemos
\begin{equation} 
T(a, u) = 
\begin{cases}
T_{0} & \text{ si } -1 \leq u < 0 \\
-T_{0} & \text{ si } 0 < u \leq 1 
\end{cases} =
\sum_{n=0}^{\infty} \underbrace{A_{n} \, a^{n}}_{\equiv c_{n}} P_{n}(u)
\label{eq:ecuacion_051a}
\end{equation}
donde podemos ocupar el resultado (usando $u$ en lugar de $x$) del ejemplo anterior, y vemos que es equivalente a la expansión en series, tal que los coeficientes pares están ausentes, así
\begin{align*}
c_{2k+1} \equiv A_{2k+1} \, a^{2k+1} = \dfrac{(-1)^{k} \, (4k+3) \, (2k)!}{2^{2k+1} \, k! \, (k+1)!} \, T_{0}
\end{align*}
Por lo que encontramos $A_{2k+1}$ de esta ecuación y agregándolo en la ecuación (\ref{eq:ecuacion_050a}) se obtiene
\begin{equation}
T(r, \theta) = T_{0} \, \sum_{k=0}^{\infty} \dfrac{(-1)^{k} \, (4k+3) \, (2k)!}{2^{2k+1} \, k! \, (k+1)!} \left( \dfrac{r}{a} \right)^{2k+1} \, P_{2k+1} (\cos \theta)
\label{eq:ecuacion_052a}
\end{equation}
donde se ha sustituido $\cos \theta$ por $u$.
%Referencia Riley. 18.2 Funciones asociadas de Legendre
\section{Funciones asociadas de Legendre.}
La ecuación asociada de Legendre tiene la forma
\begin{equation}
(1 - x^{2}) y^{\prime \prime} - 2 \, x \, y^{\prime} + \left[ \ell (\ell + 1) - \dfrac{m^{2}}{1 - x^{2}} \right] \, y = 0
\label{eq:ecuacion_18_28}
\end{equation}
que tiene tres puntos singulares en $x = -1, 1, \infty$, se reduce a la ecuación de Legendre (\ref{eq:ecuacion_18_01}) cuando $m = 0$. Se presenta en problemas de la física que involucran el operador $\nabla^{2}$, cuando se expresa en coordenadas esféricas. En esos casos, $- \ell \leq m \leq \ell$ y $m$ está restringida a valores enteros. Como en el caso de la ecuación de Legendre, la variable $x$ es el coseno del ángulo polar en coordenadas esféricas, por tanto $-1 \leq x \leq 1$. Cualquier solución de la ecuación \ref{eq:ecuacion_18_28}) es llamada la \emph{función asociada de Legendre}.
\par
El punto $x = 0$ es un punto ordinario, y del cual se pueden obtener soluciones en series de la forma
\begin{align*}
y = \sum_{n=0} a_{n} \, x^{n}
\end{align*}
de la misma manera que se hizo para la ecuación de Legendre. En este caso, debemos de notar que si $u(x)$ es solución de la ecuación de Legendre, entonces
\begin{equation}
y(x) = (1 -x^{2})^{\abs{m}/ 2} \dv[\abs{m}]{u}{x}
\label{eq:ecuacion_18_29}
\end{equation}
es solución a la ecuación asociada.
\par
De las dos soluciones en series linealmente independientes de la ecuación de Legendre dada en \ref{eq:ecuacion_18_03}) y (\ref{eq:ecuacion_18_04}), que ahora denotamos por $u_{1} (x)$ y $u_{2}(x)$, podemos obtener dos soluciones en series linealmente independientes, $y_{1} (x)$ y $y_{2} (x)$, a la ecuación asociada mediante el uso de (\ref{eq:ecuacion_18_29}). A partir de la discusión general de la convergencia de la series de potencias, vemos que tanto $y_{1} (x)$ como $y_{2} (x)$ también convergen para $\abs{x} < 1$. Por lo tanto la solución general de la ecuación(\ref{eq:ecuacion_18_28}) en este rango está dada por
\begin{align*}
y(x) = c_{1} \, y_{1} (x) + c_{2} \, y_{2} (x)
\end{align*}
\subsection*{Funciones asociadas de Legendre para enteros $\ell$.}
Si $\ell$ y $m$ son ambos enteros, como suele encontrarse en varios problemas de la física, entonces la solución general de la ecuación (\ref{eq:ecuacion_18_28}) se expresa por
\begin{equation}
y(x) = c_{1} \, P_{\ell}^{m} (x) + c_{2} \, Q_{\ell}^{m} (x)
\label{eq:ecuacion_18_31}
\end{equation}
donde $P_{\ell}^{m} (x)$ y $Q_{\ell}^{m} (x)$ son las funciones asociadas de Legendre de primera y segunda clase, respectivamente. Para valores no negativos de $m$, esas funciones están relacionadas a las funciones de Legendre para enteros $\ell$ mediante
\begin{align}
\begin{aligned}
P_{\ell}^{m} (x) = (1 - x^{2})^{m/2} \, \dv[m]{P_{\ell}}{x} \\[1em]
Q_{\ell}^{m} (x) = (1 - x^{2})^{m/2} \, \dv[m]{Q_{\ell}}{x}
\end{aligned}
\label{eq:ecuacion_18_32}
\end{align}
Vemos inmediatamente que, en caso necesario, las funciones asociadas de Legendre se reducen a las funciones ordinarias de Legendre cuando $m = 0$. Dado que $m^{2}$ aparece en la ecuación asociada de Legendre (\ref{eq:ecuacion_18_28}), las funciones asociadas de Legendre para los valores negativos $m$ debe ser proporcional a la función correspondiente para valores no negativos $m$. La constante de proporcionalidad es una cuestión de convención. Para el $P_{\ell}^{m} (x) $, es habitual considerar la definición (\ref{eq:ecuacion_18_32}) como válida también para los valores negativos $m$. Aunque la diferenciación de un número negativo no está definida, cuando $P_{\ell}(x)$ se expresa en términos de la fórmula de Rodrigues (\ref{eq:ecuacion_18_09}), este problema no se presenta para $\ell \leq m \leq \ell$. En este caso,
\begin{equation}
P_{\ell}^{-m} (x) = (-1)^{m} \, \dfrac{(\ell - m)!}{(\ell + m)!} \, P_{\ell}^{m} (x)
\label{eq:ecuacion_18_33}
\end{equation}
Ya que $P_{\ell}(x)$ es un polinomio de orden $\ell$, tenemos que $P_{\ell}^{m}(x) = 0$ para $\abs{m} > \ell$. De esta definición, queda claro que $P_{\ell}^{m} (x)$ es también un polinomio de orden $\ell$ si $m$ es par, ya que contiene el factor $(1-x^{2})$ a una potencial fraccionaria si $m$ es impar. En cualquier caso $P_{\ell}^{m}(x)$ es regular en $x = \pm 1$.
\par
Las primeras funciones asociadas de Legendre de primera clase, se construyen fácilmente y están dadas por (se omiten los casos $m=0$):
\begin{align*}
P_{1}^{1} (x) &= (1-x^{2})^{1/2} \\
P_{2}^{1} (x) &= 3x (1-x^{2})^{1/2}  \\
P_{2}^{2} (x) &= 3(1-x^{2})  \\
P_{3}^{1} (x) &= \frac{3}{2}(5x^{2}-1)(1-x^{2})^{1/2} \\
P_{3}^{2} (x) &= 15x (1-x^{2}) \\
P_{3}^{3} (x) &= 15 (1-x^{2})^{3/2} 
\end{align*}
Debemos de mencionar que las funciones asociadas de Legendre de segunda clase $Q_{\ell}^{m} (x)$ como las $Q_{\ell}(x)$ son singulares en $x= \pm 1$.
\subsection*{Propiedades de las funciones asociadas de Legendre $P_{\ell}^{m}$.}
Cuando encontramos en problemas físicos, la variable $x$ de la ecuación asociada de Legendre (como en la ecuación ordinaria Legendre) es generalmente el coseno del ángulo polar $\theta$ en coordenadas polares esféricas, y entonces queremos que la solución $y (x)$ sea regular en $x = \pm 1$ (correspondiente a $\theta = 0$ o $\theta = \pi$). Para que esto ocurra, se requiere que $\ell$ sea un número entero y que el coeficiente $c_{2}$ de la función $Q_{\ell}^{m} (x)$ en la ecuación (\ref{eq:ecuacion_18_31}) sea cero, dado que $Q_{\ell}^{m}(x)$ es singular en $x = \pm 1$, con el resultado de que la solución general son múltiplos de las funciones asociadas de Legendre de primera clase $P_{\ell}^{m}(x)$.
\subsection*{Ortogonalidad mutua.}
Ya se mencionó anteriormente que la ecuación asociada de Legendre es del tipo Sturm-Liouville con
\begin{align*}
p &= 1 - x^{2} \\
q &= - \dfrac{m^{2}}{(1 - x^{2})} \\
\lambda &= \ell (\ell + 1) \\
w &= 1
\end{align*}
siendo su intervalo natural en $[-1,1]$.
\par
Dado que las funciones asociadas de Legendre $P_{\ell}^{m} (x)$ son regulares en los extremos $x = \pm 1$, entonces deben de ser mutuamente ortogonales en este intervalo para un valor fijo de $m$, es decir:
\begin{equation}
\int_{-1}^{1} P_{\ell}^{m} (x) \, P_{k}^{m} (x) \dd{x}  = 0, \hspace{1cm} \text{ si } \ell \neq	 k
\label{eq:ecuacion_18_36}
\end{equation}
Nótese que el valor de $m$ debe de ser el mismo en ambas funciones asociadas de Legendre para que la expresión sea válida. La condición de normalización cuando $\ell = k$ se obtiene de la fórmula de Rodrigues:
\begin{equation}
I_{\ell m} = \int_{-1}^{1} P_{\ell}^{m} \, (x) P_{\ell}^{m} (x) \dd{x} = \dfrac{2}{2 \ell + 1} \, \dfrac{(\ell +m)!}{(\ell - m)!}
\label{eq:ecuacion_18_37}
\end{equation}
Las condiciones de ortogonalidad y normalización, ecuaciones (\ref{eq:ecuacion_18_36}) y (\ref{eq:ecuacion_18_37}), respectivamente, significan que la funciones asociadas de Legendre $P_{\ell}^{m}(x)$, con $m$ fija, puede utilizarse de  manera similar a los polinomios de Legendre para expandir cualquier función $f(x)$ razonable en el intervalo $\abs{x} < 1$ en una serie de la forma
\begin{equation}
f(x) = \sum_{k=0}^{\infty} a_{m+k} \, P_{m+k}^{m} (x)
\label{eq:ecuacion_18_38}
\end{equation}
donde los coeficientes están dados por
\begin{align*}
a_{\ell} = \dfrac{2 \ell + 1}{2} \: \dfrac{(\ell - m)!}{(\ell + m)!} \int_{-1}^{1} f(x) \, P_{\ell}^{m} (x) \dd{x}
\end{align*}
\subsubsection{Función generatriz.}
La función generatriz para las funciones asociadas de Legendre, se obtienen de la combinación de su definición con la función generatriz de los polinomio de Legendre:
\begin{equation}
G(x,h) = \dfrac{(2m)! \, (1 - x^{2})^{m/2}}{2^{m} \, m! \, (1 - 2 \, h \, x + h^{2})^{m+1/2}} = \sum_{n=0}^{\infty} P_{n+m}^{m} (x) \, h^{n}
\label{eq:ecuacion_18_40}
\end{equation}
Como era de esperar, las funciones asociadas de Legendre satisfacen ciertas relaciones de recurrencia. De hecho, la presencia de los dos índices $n$ y $m$ significa que se puede derivar una gama mucho más amplia de relaciones de recurrencia. Presentaremos sólo cuatro de las relaciones más útiles:
\begin{align*}
P_{n}^{m+1} &= \dfrac{2 \, m \, x}{(1-x^{2})^{1/2}} P_{n}^{m} + [m(m - 1) - n (n + 1)] \, P_{n}^{m-1} \\
(2 \, n + 1) \, x \, P_{n}^{m} &= (n + m) \, P_{n-1}^{m} + (n - m + 1) \, P_{n+1}^{m} \\
(2 \, n + 1) \, (1 -  x^{2})^{1/2} \, P_{n}^{m} &= P_{n+1}^{m+1} - P_{n-1}^{m+1} \\
2 \, (1 - x^{2})^{1/2} \, (P_{n}^{m})^{\prime} &= P_{n}^{m+1} - (n + m) \, (n - m + 1) \, P_{n}^{m-1}
\end{align*}
Las relaciones de recurrencia son válidas tanto para valores negativos como positivos de $m$.
% \section{Armónicos esféricos.}
% Las funciones asociadas de Legendre discutidas anteriormente se presentan más comúnmente la solución de la ecuación de Laplace $\nabla^{2} =0$ en coordenadas polares esféricas. En particular, se encuentra que para las soluciones que son finitas en el eje polar, la parte angular de la solución viene dada por
% \[ \Theta (\theta) \Phi (\phi) = P_{\ell}^{m} (\cos \theta) (C \cos m \phi + D \sin m \phi) \]
% donde $\ell$ y $m$ son enteros con $- \ell \leq m \leq \ell$. Esta forma general es muy común para funciones particulares de $\theta$ y $\phi$, se les llama \emph{armónicos esféricos}, se definen por
% \begin{equation}
% Y_{\ell}^{m} (\theta, \phi) = (1-)^{m} \left[ \dfrac{2 \ell + 1}{4 \pi} \: \dfrac{(\ell + m)!}{(\ell - m)!} \right]^{1/2} P_{\ell}^{m} (\cos \theta) \exp(i m \phi)
% \label{eq:ecuacion_045}
% \end{equation}
% Usando la ecuación (\ref{eq:ecuacion_033}), encontramos que:
% \[ Y_{\ell}^{-m} (\theta, \phi) =  (-1)^{m} \left[ Y_{\ell}^{m} (\theta,\phi) \right]^{*} \]
% donde el asterisco indica el complejo conjugado. Los priemros armónicos esféricos $Y_{\ell}^{m}(\theta,\phi) = Y_{\ell}^{m}$ son:
% \begin{eqnarray}
% Y_{0}^{0} &=& \sqrt{\dfrac{1}{4 \pi}} \nonumber \\
% Y_{1}^{0} &=& \sqrt{\dfrac{3}{4 \pi}} \cos \theta \nonumber \\
% Y_{1}^{\pm 1} &=& \mp \sqrt{\dfrac{3}{8 \pi}} \sin \theta \exp(\pm i \phi) \nonumber \\
% Y_{2}^{0} &=& \sqrt{\dfrac{5}{16 \pi}} ( 3 \cos^{2} \theta - 1) \nonumber \\
% Y_{2}^{\pm 1} &=& \mp \sqrt{\dfrac{15}{8 \pi}} \sin \theta \cos \theta \exp(\pm i \phi) \nonumber \\
% Y_{2}^{\pm 2} &=& \sqrt{\dfrac{15}{32 \pi}} \sin^{2} \theta \exp(\pm 2 i \phi) \nonumber
% \end{eqnarray}
% Ya que contienen su $\theta$-dependiente de parte de la solución $P_{\ell}^{m}$ a la ecuación asociada de Legendre, el $Y_{\ell}^{m}$ son mutuamente ortogonales cuando se integra de $-1$ a $+1$ sobre $d(cos \theta)$. Su ortogonalidad mutua respecto de $\phi (0 \leq \phi \leq 2 \pi)$ es aún más evidente. El factor numérico en la ecuación (\ref{eq:ecuacion_045}) es elegido para hacer el $Y_{\ell}^{m}$ un conjunto ortonormal, es decir
% \begin{equation}
% \int_{-1}^{1} \int_{0}^{2 \pi} [ Y_{\ell}^{m} (\theta, \phi) ]^{*} Y_{\ell'}^{m'} (\theta, \phi) d \phi d(\cos \theta) = \delta_{\ell \ell'} \delta_{m m'}
% \label{eq:ecuacion_046}
% \end{equation}
% Adicionalmente, los armónicos esféricos forman un conjunto completo para cualquier función razonable de $\theta$ y $\phi$(como las que podemos encontrar en un problema físico), la función puede expandirse como una suma de tales funciones
% \begin{equation}
% f(\theta, \phi) = \sum_{\ell=0}^{\infty} \sum_{-\ell}^{\ell} a_{\ell m} Y_{\ell}^{m} (\theta, \phi)
% \label{eq:ecuacion_047}
% \end{equation}
% las constantes $a_{\ell m}$ están dadas por
% \begin{equation}
% a_{\ell m} = \int_{-1}^{1} \int_{0}^{2 \pi} [ Y_{\ell}^{m} (\theta, \phi) ]^{*} f (\theta, \phi) d \theta d (\cos \theta)
% \label{eq:ecuacion_048}
% \end{equation}
% Esto es una analogía exacta con una serie de Fourier y es un ejemplo particular de la propiedad general de soluciones de Sturm-Liouville. Aparte de la condición ortonormalidad (\ref{eq:ecuacion_046}), la relación más importante que cumplen los $Y_{\ell}^{m}$ es el teorema de adición de armónicos esféricos:
% \begin{equation}
% P_{\ell} (\cos \gamma) = \dfrac{4 \pi}{2 \ell + 1} \sum_{m = -\ell}^{\ell} Y_{\ell}^{m} (\theta, \phi) [ Y_{\ell}^{m} (\theta^{\prime}, \phi^{\prime}) ]^{*}
% \end{equation}
% donde $(\theta, \phi)$ y $(\theta^{\prime}, \phi^{\prime})$ denotan dos direcciones diferentes a nuestro sistema de coordenadas esféricas polares y que están separadas por un ángulo $\gamma$. En general, la trigonometría esférica (o vectorial) demuestra que estos ángulos obedecen la identidad
% \begin{equation}
% \cos \gamma = \cos \theta \cos \theta^{\prime} + \sin \theta \sin \theta^{\prime} \cos (\phi - \phi^{\prime})
% \label{eq:ecuacion_050}
% \end{equation}
\end{document}