\documentclass[12pt]{article}
\usepackage[utf8]{inputenc}
\usepackage[spanish,es-lcroman, es-tabla]{babel}
\usepackage[autostyle,spanish=mexican]{csquotes}
\usepackage{amsmath}
\usepackage{amssymb}
\usepackage{nccmath}
\numberwithin{equation}{section}
\usepackage{amsthm}
\usepackage{graphicx}
\usepackage{epstopdf}
\DeclareGraphicsExtensions{.pdf,.png,.jpg,.eps}
\usepackage{color}
\usepackage{float}
\usepackage{multicol}
\usepackage{enumerate}
\usepackage[shortlabels]{enumitem}
\usepackage{anyfontsize}
\usepackage{anysize}
\usepackage{array}
\usepackage{multirow}
\usepackage{enumitem}
\usepackage{cancel}
\usepackage{tikz}
\usepackage{circuitikz}
\usepackage{tikz-3dplot}
\usetikzlibrary{babel}
\usetikzlibrary{shapes}
\usepackage{bm}
\usepackage{mathtools}
\usepackage{esvect}
\usepackage{hyperref}
\usepackage{relsize}
\usepackage{siunitx}
\usepackage{physics}
%\usepackage{biblatex}
\usepackage{standalone}
\usepackage{mathrsfs}
\usepackage{bigints}
\usepackage{bookmark}
\spanishdecimal{.}

\setlist[enumerate]{itemsep=0mm}

\renewcommand{\baselinestretch}{1.5}

\let\oldbibliography\thebibliography

\renewcommand{\thebibliography}[1]{\oldbibliography{#1}

\setlength{\itemsep}{0pt}}
%\marginsize{1.5cm}{1.5cm}{2cm}{2cm}


\newtheorem{defi}{{\it Definición}}[section]
\newtheorem{teo}{{\it Teorema}}[section]
\newtheorem{ejemplo}{{\it Ejemplo}}[section]
\newtheorem{propiedad}{{\it Propiedad}}[section]
\newtheorem{lema}{{\it Lema}}[section]

\author{}
\title{El grupo de rotaciones y los armónicos esféricos \\ {\large Matemáticas Avanzadas de la Física}\vspace{-1.5\baselineskip}}
\date{ }
\begin{document}
\maketitle
\fontsize{14}{14}\selectfont
En este tema estudiaremos los polinomios de Legendre, los polinomios asociados de Legendre y los Armónicos esféricos, quienes son importantes para resolver la ecuación de Laplace y diversos problemas de electrodinámica y mecánica cuántica. Normalmente estas funciones se obtienen resolviendo ecuaciones diferenciales. Sin embargo, también es posible obtenerlas usando el grupo de rotaciones. Usaremos este último método debido a que nos introduce a las aplicación de la teoría de grupos en la física. Este tema se puede ver como una invitación al estudio de las aplicaciones de la teoría de grupos.
\section{Transformación de coordenadas lineales.}
Sea $f$ una función de la variable $x$, con el cambio de variable
\begin{align}
x^{\prime} = \alpha \, x \hspace{1.5cm} \mbox{con } \alpha = \mbox{ constante}
\label{eq:ecuacion_10_001}
\end{align}
al diferenciar con respecto a $x$ tenemos que
\begin{align}
\dv{f}{x^{\prime}} = \dv{x}{x^{\prime}} \, \dv{f}{x} = \dfrac{1}{\alpha} \, \dv{f}{x}
\label{eq:ecuacion_10_002}   
\end{align}
Entonces tenemos que si la variable $x$ transforma con $\alpha$, el operador derivada transforma con $\dfrac{1}{\alpha}$, es decir
\begin{align}
\begin{aligned}
x \rightarrow x^{\prime} &= \alpha \, x \\
\dv{x} \rightarrow \dv{x^{\prime}} &= \dfrac{1}{\alpha} \, \dv{f}{x}
\end{aligned}
\label{eq:ecuacion_10_003}
\end{align}
Veamos ahora que pasa en dos dimensiones. Consideremos la matriz de $2 \times 2$ con entradas constantes
\begin{align}
\Lambda = \mqty(a_{1} & a_{2} \\ a_{3} & a_{4})
\label{eq:ecuacion_10_004}   
\end{align}
cuya matriz inversa es
\begin{align}
\Lambda^{-1} = \dfrac{1}{\abs{\Lambda}} \, \mqty(a_{4} & - a_{2} \\ - a_{3} & a_{1}) \hspace{1.5cm} \abs{\Lambda} = a_{1} \, a_{4} - a_{2} \, a_{3}
\label{eq:ecuacion_10_005}   
\end{align}
También definamos los vectores columna
\begin{align}
X = \mqty(x^{1} \\ x^{2}) \hspace{1.5cm} \grad = \mqty(\displaystyle \pdv{x^{1}} \\[0.5em] \displaystyle \pdv{x^{2}})
\label{eq:ecuacion_10_006}   
\end{align}
Entonces podemos hacer una transformación lineal de coordenadas de la forma $X^{\prime} = \Lambda \, X$, es decir
\begin{align}
\mqty(x^{\prime \, 1} \\ x^{\prime \, 2}) =
\mqty(a_{1} & a_{2} \\ a_{3} & a_{4}) \, \mqty(x^{1} \\ x^{2})
\label{eq:ecuacion_10_007}   
\end{align}
La cual tiene la transformación inversa: $X = \Lambda^{-1} \, X^{\prime}$
\begin{align}
\mqty(x^{1} \\ x^{2}) = \dfrac{1}{\abs{\Lambda}} \, 
\mqty(a_{4} & - a_{2} \\ - a_{3} & a_{1}) \, \mqty(x^{\prime \, 1} \\ x^{\prime \, 2})
\label{eq:ecuacion_10_008}   
\end{align}
es decir
\begin{align}
x^{1} &= \dfrac{a_{4} \, x^{\prime \, 1} - a_{2} \, x^{\prime \, 2}}{\abs{\Lambda}} \label{eq:ecuacion_10_009} \\[0.5em]
x^{2} &= \dfrac{-a_{3} \, x^{\prime \, 1} + a_{1} \, x^{\prime \, 2}}{\abs{\Lambda}} \label{eq:ecuacion_10_010}
\end{align}
Además, al diferencias por la regla de la cadena se tiene
\begin{align}
\begin{aligned}
\pdv{x^{\prime \, 1}} &= \pdv{x^{1}}{x^{\prime \, 1}} \, \pdv{x^{1}} + \pdv{x^{2}}{x^{\prime \, 1}} \, \pdv{x^{2}} \\[0.5em]
\pdv{x^{\prime \, 2}} &= \pdv{x^{1}}{x^{\prime \, 2}} \, \pdv{x^{1}} + \pdv{x^{2}}{x^{\prime \, 2}} \, \pdv{x^{2}}
\end{aligned}
\label{eq:ecuacion_10_011}
\end{align}
de la regla de transformación ec. (\ref{eq:ecuacion_10_010}) se encuentra que
\begin{align}
\begin{aligned}
\pdv{x^{\prime \ 1}} &= \dfrac{1}{\abs{\Lambda}} \, \left( a_{4} \, \pdv{x^{1}} - a_{3} \, \pdv{x^{2}} \right) \\[1em]
\pdv{x^{\prime \ 2}} &= \dfrac{1}{\abs{\Lambda}} \, \left( - a_{2} \, \pdv{x^{1}} + a_{1} \, \pdv{x^{2}} \right)
\end{aligned}
\label{eq:ecuacion_10_012}
\end{align}
que se puede expresar como
\begin{align}
\mqty(\displaystyle \pdv{x^{\prime \, 1}} \\[0.75em] \displaystyle \pdv{x^{\prime \, 2}}) = \dfrac{1}{\abs{\Lambda}} \, \mqty(a_{4} & - a_{3} \\ - a_{2} & a_{1}) \, \mqty(\displaystyle \pdv{x^{1}} \\[0.75em] \displaystyle \pdv{x^{2}})
\label{eq:ecuacion_10_013}   
\end{align}
De esta ecuación podemos ver que la matriz involucrada en la transformación de las derivadas parciales es la transpuesta de la matriz inverza de $\Lambda$, es decir
$(\Lambda^{-1})^{T}$. Otra forma de escribir esta ecuación es
\begin{align}
\grad^{\prime} = (\Lambda^{-1})^{T} \, \grad
\label{eq:ecuacion_10_014}
\end{align}
por lo que
\begin{align}
\grad = (\Lambda)^{T} \, \grad^{\prime}
\label{eq:ecuacion_10_015}
\end{align}
Este resultado se puede generalizar para más dimensiones. En efecto, en general una transformación de coordenadas se escribe como
\begin{align}
x^{\prime \, i} = \Lambda_{ij} \, x^{j} \hspace{1.5cm} x^{i} = (\Lambda^{-1})_{ij} \, x^{\prime \, j}
\label{eq:ecuacion_10_016}
\end{align}
Por la regla de la cadena se tiene que
\begin{align}
\pdv{x^{\prime \, i}} &= \pdv{x^{j}}{x^{\prime \, i}} \, \pdv{x^{j}} = \nonumber \\[0.5em]
&= \pdv{(\Lambda^{-1})_{jk} \, x^{\prime \, k}}{x^{\prime \, i}} \, \pdv{x^{j}} = \nonumber \\[0.5em]
&= (\Lambda^{-1})_{jk} \, \pdv{x^{\prime , k}}{x^{\prime \, i}} \, \pdv{x^{j}} = \nonumber \\[0.5em]
&= (\Lambda^{-1})_{jk} \, \delta_{ki} \, \pdv{x^{j}} = \nonumber \\[0.5em]
&= (\Lambda^{-1})_{ji} \, \pdv{x^{j}} = \nonumber \\[0.5em]
&= \left( \left( \Lambda^{-1} \right)^{T} \right)_{ij} \, \pdv{x^{j}}
\label{eq:ecuacion_10_017}
\end{align}
Por lo tanto, para cualquier dimensión se cumple
\begin{align}
\grad^{\prime} = (\Lambda^{-1})^{T} \, \grad
\label{eq:ecuacion_10_018}
\end{align}
Esta ley de transformación será de gran utilidad para obtener las simetrías de la ecuación de Laplace.
\section{Laplaciano y elemento de línea.}
Supongamos que la matriz $\tilde{\eta}$ de $n \times n$, satisface
\begin{align}
\tilde{\eta} \, \tilde{\eta} = I
\label{eq:ecuacion_10_019}
\end{align}
donde la matriz $I$ es la matriz identidad de $n \times n$. Entonces podemos definir un elemento de línea como
\begin{align}
\dd{s^{2}} = \dd{X^{T}} \, \tilde{\eta} \dd{X}, \hspace{1.5cm} \dd{X} = \mqty(\dd{x^{1}} \\ \dd{x^{2}} \\ \vdots \\ \dd{x^{n}})
\label{eq:ecuacion_10_020}
\end{align}
A la matriz $\tilde{\eta}$ se le llama \emph{métrica}, un ejemplo en dos dimensiones de este tipo de matrices son
\begin{align}
\mqty(\pmat{0}) \hspace{1.5cm} \mqty(-1 & 0 \\ 0 & 1)
\label{eq:ecuacion_10_021}
\end{align}
Con la matriz $\tilde{\eta}$ el \enquote{Laplaciano} se define como
\begin{align}
\laplacian = \grad^{T} \, \tilde{\eta} \, \grad
\label{eq:ecuacion_10_022}
\end{align}
Notablemente, el elemento de línea está íntimamente relacionado con el Laplaciano, en particular tienen las mismas simetrías. Esta relación es importante
y se da también para espacios no euclidianos. Veamos como se da esta relación.
\par
Bajo una transformación lineal de coordenadas se tiene
\begin{align}
\dd{s^{\prime \, 2}} &= \dd{X^{\prime \, T}} \tilde{\eta} \dd{X^{\prime}} = \nonumber \\[0.5em]
&= \left(\Lambda \dd{X} \right)^{T} \, \tilde{\eta} \, \Lambda \, \dd{X} = \nonumber \\[0.5em]
&= \dd{X^{T}} \left( \Lambda^{T} \, \tilde{\eta} \, \Lambda \right) \dd{X}
\label{eq:ecuacion_10_023}
\end{align}
Las transformaciones $\Lambda$ que dejan invariante al elemento de línea, deben de cumplir $\dd{s^{2}} = \dd{s^{\prime \, 2}}$. Al igualar la ec. (\ref{eq:ecuacion_10_020}) con la ec. (\ref{eq:ecuacion_10_023}) se llega a la condición 
\begin{align}
\Lambda^{T} \, \tilde{\eta} \, \Lambda = \tilde{\eta}
\label{eq:ecuacion_10_024}
\end{align}
Además, considerando que bajo una transformación lineal de coordenadas el gradiente transforma como eq. (\ref{eq:ecuacion_10_018}), se obtiene
\begin{align}
\grad^{\prime \, 2} &= \grad^{\prime \, T} \, \tilde{\eta} \, \grad^{\prime} = \nonumber \\[0.5em]
&= \left( \left(\Lambda^{-1} \right)^{T} \, \grad \right)^{T} \, \tilde{\eta} \, \left( \Lambda^{-1} \right)^{T} \, \grad = \nonumber \\[0.5em]
&= \grad^{T} \, \left[ \left( \left( \Lambda^{-1} \right)^{T} \right)^{T} \, \tilde{\eta} \, \left( \Lambda^{-1} \right)^{T} \right] \, \grad = \nonumber \\[0.5em]
&= \grad^{T} \, \left( \Lambda^{-1} \, \tilde{\eta} \, \left( \Lambda^{-1} \right)^{T} \right) \, \grad
\label{eq:ecuacion_10_025}
\end{align}
Las transformaciones que dejan invariante al Laplaciano deben de cumplir
\begin{align*}
\laplacian = \grad^{\prime \, 2}
\end{align*}
entonces igualando la ec. (\ref{eq:ecuacion_10_022}) con la ec. (\ref{eq:ecuacion_10_025}), se tiene la condición
\begin{align}
\Lambda^{-1} \, \tilde{\eta} \, \left( \Lambda^{-1} \right)^{T} = \tilde{\eta}
\label{eq:ecuacion_10_026}
\end{align}
Como $(\tilde{\eta})^{-1} = \tilde{\eta}$, la condición dada por la ec. (\ref{eq:ecuacion_10_026}), tiene la forma
\begin{align}
\tilde{\eta} &= \left(\tilde{\eta} \right)^{-1} \nonumber \\[0.5em]
&= \left( \Lambda^{-1} \, \tilde{\eta} \, \left( \Lambda^{-1} \right)^{T} \right)^{-1} = \nonumber \\[0.5em]
&= \left( \left( \Lambda^{-1} \right)^{T} \right)^{-1} \, \tilde{\eta} \, \left( \Lambda^{-1} \right)^{-1} = \nonumber \\[0.5em]
&= \Lambda^{T} \, \tilde{\eta} \, \Lambda
\label{eq:ecuacion_10_027}
\end{align}
que coincide con la eq. (\ref{eq:ecuacion_10_024}). Por lo tanto, las transformaciones lineales, $\Lambda$, que
dejan invariante al elemento de línea eq. (\ref{eq:ecuacion_10_020}) también dejan invariante al
Laplaciano eq. (\ref{eq:ecuacion_10_022}), claramente la afirmació inversa también es correcta.
\section{Grupo de transformaciones.}
Antes de continuar recordemos lo que es un grupo. Sea $G$ un conjunto con una operación $\bullet \, : G \times G \rightarrow G$. El par $(G, \bullet)$ es un grupo si cumple:
\begin{enumerate}[label=\arabic*)]
\item Axioma de cerradura:
\begin{align}
g_{1} \in G, g_{2} \in G \hspace{1cm} \Rightarrow \hspace{1cm} g_{1} \bullet g_{2} \in G
\label{eq:ecuacion_10_028}
\end{align}
\item Axioma de asociatividad:
\begin{align}
g_{1} \in G, g_{2} \in G, g_{3} \in G \hspace{1cm} \Rightarrow \hspace{1cm} g_{1} \bullet (g_{2} \bullet g_{3}) = (g_{1} \bullet g_{2}) \bullet g_{3}
\label{eq:ecuacion_10_029}
\end{align}
\item Axioma del neutro:
\begin{align}
\exists \, e \in G, \hspace{1cm} g_{1} \in G \hspace{1cm} \Rightarrow \hspace{1cm} g_{1} \bullet e = e \bullet g_{1} = g_{1}
\label{eq:ecuacion_10_030}
\end{align}
\item Axioma del inverso:
\begin{align}
\forall \, g_{1} \in G, \exists \, g_{1}^{-1} \in G, \hspace{1cm} g_{1} \bullet g_{1}^{-1} = g_{-1} \bullet g_{1} = e
\label{eq:ecuacion_10_031}
\end{align}
\end{enumerate}
Definamos a $T$ como el conjunto de transformaciones $\Lambda$ que dejan invariante al Laplaciano. Estas transformaciones cumplen
\begin{align}
\Lambda^{T} \, \tilde{\eta} \, \Lambda = \tilde{\eta} \hspace{1.5cm} \tilde{\eta}^{2} = I
\label{eq:ecuacion_10_032}
\end{align}
Probaremos que $T$ es un grupo.
\par
Supongamos que $\Lambda_{1} \in T$ y $\Lambda_{2} \in T$, entonces cumplen
\begin{align}
\Lambda_{1}^{T} \, \tilde{\eta} \, \Lambda_{1} = \tilde{\eta}, \hspace{1.5cm} \Lambda_{2}^{T} \, \tilde{\eta} \, \Lambda_{2} = \tilde{\eta}
\label{eq:ecuacion_10_033}
\end{align}
de donde
\begin{align}
(\Lambda_{1} \, \Lambda_{2})^{T} \, \tilde{\eta} \, (\Lambda_{1} \, \Lambda_{2}) =\Lambda_{2}^{T} \, \Lambda_{1}^{T} \, \tilde{\eta} \, \Lambda_{1} \, \Lambda_{2} = \Lambda_{2}^{T} \, \tilde{\eta} \, \Lambda_{2} = \tilde{\eta}
\label{eq:ecuacion_10_034}
\end{align}
esto implica que $\Lambda_{1} \, \Lambda_{2} \in T$, es decir
\begin{align}
\Lambda_{1} \in T, \Lambda_{2} \in T \hspace{1cm} \Rightarrow \hspace{1cm} \Lambda_{1} \, \Lambda_{2} \in T
\label{eq:ecuacion_10_035}
\end{align}
Por lo tanto, se cumple el axioma de la cerradura.
\par
El producto de matrices es asociativo, en particular el producto de las matrices que satisfacen la eq. (\ref{eq:ecuacion_10_032}). Además, la identidad $I$ satisface la eq. (\ref{eq:ecuacion_10_032}), es decir, $I \in T$. Así se cumplen el axioma de la asociatividad y el del elemento
neutro.
\par
Se tiene que $\tilde{\eta}^{2} = I$, si $\Lambda$ está en $T$ entonces se cumple $\Lambda^{T} \, \tilde{\eta} \, \Lambda \, \tilde{\eta} = I$. De donde
\begin{align}
\tilde{\eta} \, \Lambda \, \tilde{\eta} = \left( \Lambda^{T} \right)^{-1} = \left( \Lambda^{-1} \right)^{T}
\label{eq:ecuacion_10_036}
\end{align}
por lo tanto
\begin{align}
\left( \Lambda^{-1} \right)^{T}\, \tilde{\eta} \, \Lambda^{-1} = ( \tilde{\eta} \, \Lambda \, \tilde{\eta} ) \, \tilde{\eta} \, \Lambda^{-1} = \tilde{\eta} \, \Lambda \, \Lambda^{-1} = \tilde{\eta}
\label{eq:ecuacion_10_037}
\end{align}
Así, cuando $\Lambda$ está en $T$, también $\Lambda^{-1}$ está en $T$. Esto nos indica que se cumple el axioma del inverso.
\par
En consecuencia el conjunto de matrices que satisface la eq. (\ref{eq:ecuacion_10_032}) es un grupo. Es decir, el conjunto de transformaciones que dejan invariante al elemento de línea eq. (\ref{eq:ecuacion_10_020}) forma un grupo, que es el mismo grupo que deja invariante al Laplaciano eq. (\ref{eq:ecuacion_10_022}).
\section{El grupo de rotaciones.}
Sea $\va{x} = (x_{1}, \ldots, x_{n})$ un vector en $\mathbb{R}^{n}$ y definamos la forma cuadrática $l^{2} = x_{1}^{2} + x_{2}^{2} + \ldots + x_{n}^{2}$, la cual representa la distancia de $\va{x}$ al origen. Note que si definimos la matriz columna
\begin{align}
X = \mqty(x_{1} \\ x_{2} \\ \vdots \\ x_{n})
\label{eq:ecuacion_10_038}
\end{align}
y la matriz renglón
\begin{align}
X^{T} = \mqty(x_{1} & x_{2} & \ldots & x_{n})
\label{eq:ecuacion_10_039}
\end{align}
la distancia se puede escribir como
\begin{align}
l^{2} = X^{T} \, X = X^{T} \, I \, X 
\label{eq:ecuacion_10_040}
\end{align}
Ahora, si $\Lambda$ es una matriz de $n \times n$ y se hace la transformación de coordenadas
\begin{align}
X^{\prime} = \Lambda \, X
\label{eq:ecuacion_10_041}
\end{align}
se tiene la distancia
\begin{align}
l^{2} = X^{\prime \, T} \, I , X^{\prime} = X^{T} \, \left( \Lambda^{T} \, I \, \Lambda \right) \, X 
\label{eq:ecuacion_10_042}
\end{align}
Por lo tanto, si $\Lambda$ es tal que deja la distancia invariante, es decir $l^{2} = l^{\prime \, 2}$, debe cumplir
\begin{align}
\Lambda^{T} \, I \, \Lambda = I
\label{eq:ecuacion_10_043}
\end{align}
Otra forma de expresar esta igualdad es $\Lambda^{T} = \Lambda^{-1}$. Claramente las matrices que cumplen la ec. (\ref{eq:ecuacion_10_043}) forman un grupo, a este grupo de matrices se le llama $O(n)$.
\par
Recordemos que para cualquier matriz $A$ se cumple que: $\det A = \det A^{T}$. Entonces, las matrices que satisfacen la eq. (\ref{eq:ecuacion_10_043}) deben cumplir $(\det \Lambda)^{2} = 1$, es decir $\det \Lambda = \pm 1$. El subconjunto de matrices $\Lambda$ que cumplen $\det \Lambda = - 1$ no forman un grupo, por ejemplo, la identidad no está en ese subconjunto. Sin embargo, las matrices $\Lambda$ que cumplen $\det \Lambda = 1$ sí forman un grupo, este es el grupo $SO(n)$.
\par
Note que la matriz de $n \times n$
\begin{align}
\Lambda = \mqty(\Lambda_{11} & \Lambda_{12} & \ldots & \Lambda_{1n} \\ \Lambda_{21} & \Lambda_{22} & \ldots & \Lambda_{2n} \\ \vdots & \vdots & \ddots & \vdots \\ \Lambda_{n1} & \Lambda_{n2} & \ldots & \Lambda_{nn})
\label{eq:ecuacion_10_44}
\end{align}
se puede formar con los vectores columna
\begin{align}
\va{C}_{1} = \mqty(\Lambda_{11} \\ \Lambda_{21} \\ \vdots \\ \Lambda_{n1}), \hspace{0.5cm} \va{C}_{2} = \mqty(\Lambda_{12} \\ \Lambda_{22} \\ \vdots \\ \Lambda_{n2}), \hspace{0.5cm} \ldots, \va{C}_{n} = \mqty(\Lambda_{1n} \\ \Lambda_{2n} \\ \vdots \\ \Lambda_{nn})
\label{eq:ecuacion_10_045}
\end{align}
Claramente, para la matriz transpuesta $\Lambda^{T}$, estos vectores representan los renglones. Por lo tanto, por la condición dada en la ec. (\ref{eq:ecuacion_10_043}), se puede escribir como
\begin{align}
\Lambda^{T} \, \Lambda &= \mqty(\Lambda_{11} & \Lambda_{21} & \ldots & \Lambda_{n1} \\ \Lambda_{12} & \Lambda_{22} & \ldots & \Lambda_{nn} \\ \vdots & \vdots & \ddots & \vdots \\ \Lambda_{1n} & \Lambda_{2n} & \ldots & \Lambda_{nn}) \, \mqty(\Lambda_{11} & \Lambda_{12} & \ldots & \Lambda_{1n} \\ \Lambda_{21} & \Lambda_{22} & \ldots & \Lambda_{2n} \\ \vdots & \vdots & \ddots & \vdots \\ \Lambda_{n1} & \Lambda_{n2} & \ldots & \Lambda_{nn}) \nonumber \\[1em]
&= \mqty(\va{C}_{1} \cdot \va{C}_{1} & \va{C}_{1} \cdot \va{C}_{2} & \ldots & \va{C}_{1} \cdot \va{C}_{n} \\ \va{C}_{2} \cdot \va{C}_{1} & \va{C}_{2} \cdot \va{C}_{2} & \ldots & \va{C}_{n} \cdot \va{C}_{2} \\ \vdots & \vdots & \ddots & \vdots \\ \va{C}_{n} \cdot \va{C}_{1} & \va{C}_{n1} \cdot \va{C}_{2} & \ldots & \va{C}_{n} \cdot \va{C}_{n})
\label{eq:ecuacion_10_046}
\end{align}
Otra forma de expresar esta igualdad es
\begin{align}
\va{C}_{i} \cdot \va{C}_{j} = \delta_{ij}
\label{eq:ecuacion_10_047}
\end{align}
es decir si una matriz satisface la eq. (\ref{eq:ecuacion_10_043}), tiene sus columnas ortonormales.
Ahora, si $\Lambda$ satisface la condición (\ref{eq:ecuacion_10_043}), entonces $\Lambda^{-1}$ también la satisface. Por lo tanto, $\Lambda^{-1}$ tiene sus columnas ortonormales entre sí. Pero se debe cumplir $\Lambda^{-1} = \Lambda^{T}$, entonces las columnas de $\Lambda^{T}$ son ortonormales entre sí.
\par
Considerando que las columnas de $\Lambda^{T}$ son los renglones de $\Lambda$, podemos ver que los renglones de $\Lambda$ son ortonormales entre sí. En conclusión, si $\Lambda$ satisface la eq. (\ref{eq:ecuacion_10_043}), sus columnas y renglones son ortonormales entre sí.
\par
Una matriz de $n \times n$ tiene $n^{2}$ parámetros libres, pero si satisface la ec. (\ref{eq:ecuacion_10_046}) no todos sus parámetros son libres. De la ec. (\ref{eq:ecuacion_10_046}) se puede ver que $\Lambda^{T} \, \Lambda$ es una matriz simétrica, por lo que la ec. (\ref{eq:ecuacion_10_046}) sólo tiene $\dfrac{n (n+1)}{2}$ ecuaciones independientes. Así, los parámetros libres de una matriz que satisface la ec. (\ref{eq:ecuacion_10_046}) son
\begin{align*}
n^{2} - \dfrac{n (n+1)}{2} = \dfrac{n (n-1)}{2}
\end{align*}
Para el caso $n = 3$ hay tres parámetros libres, para ésta dimensión cualquier matriz se puede escribir como
\begin{align}
\Lambda = \mqty(a_{1} & b_{1} & c_{1} \\ a_{2} & b_{2} & c_{2} \\ a_{3} & b_{3} & c_{3})
\label{eq:ecuacion_10_048}
\end{align}
Si ésta matriz satisface la ec. (\ref{eq:ecuacion_10_046}), debe cumplir
\begin{align}
\begin{aligned}
\va{a} \vdot \va{a} = \va{b} \vdot \va{b} = \va{c} \vdot \va{c} =  1 \\[0.5em]
\va{a} \vdot \va{b} = \va{a} \vdot \va{c} = \va{b} \vdot \va{c} =  0
\end{aligned}
\label{eq:ecuacion_10_049}
\end{align}
esto nos dice que la punta de los vectores $\va{a}, \va{b},\va{c}$ están en una esfera y que son ortonormales entre sí. Un ejemplo de este tipo de matrices son
\begin{align}
\Lambda_{x} (\theta) = \mqty(1 & 0 & 0 \\ 0 & \cos \theta & \sin \theta \\ 0 & -\sin \theta & \cos \theta) \label{eq:ecuacion_10_050} \\[1em]
\Lambda_{y} (\psi) = \mqty(\cos \psi & 0 & \sin \psi \\ 0 & 1 & 0 \\ - \sin \psi & 0 & \cos \psi) \label{eq:ecuacion_10_051} \\[1em]
\Lambda_{z} (\phi) = \mqty(\cos \phi & \sin \phi & 0 \\ - \sin \phi & \cos \phi & 0 \\ 0 & 0 & 1) \label{eq:ecuacion_10_052}
\end{align}
Cada una de estas matrices representa una rotación
\begin{itemize}
\item $\Lambda_{x} (\theta)$ representa una rotación sobre el eje $x$.
\item $\Lambda_{y} (\psi)$ representa una rotación sobre el eje $y$.
\item $\Lambda_{z} (\phi)$ representa una rotación sobre el eje $z$
\end{itemize}
Por lo tanto, las rotaciones dejan invariante la distancia y al Laplaciano. Nótese que las tres matrices son linealmente independientes.
\par
Existen otras formas de reparametrizar una matriz de rotación, por ejemplo la base unitaria en coordenadas esféricas
\begin{align*}
\vu{e}_{r} &= \cos \varphi \sin \theta \, \vu{i} + \sin \varphi \sin \theta \, \vu{j} + \cos \theta , \vu{k} \\
\vu{e}_{\theta} &= \cos \varphi \cos \theta \, \vu{i} + \sin \varphi \cos \theta \, \vu{j} - \sin \theta \, \vu{k}
\end{align*}
satisfacen la condición dada por la eq. (\ref{eq:ecuacion_10_049}), pero no son la solución más general, pues sólo dependen de dos parámetros mientras que la solución general de la eq. (\ref{eq:ecuacion_10_049}) depende de tres. Pero estos vectores nos sirven para obtener la solución general. Propondremos a $\va{c}$ como
\begin{align}
\va{c} = (\sin \psi \sin \theta, \cos \psi \sin \theta, \cos \theta)
\label{eq:ecuacion_10_053}
\end{align}
Los vectores $\va{b}$ y $\va{c}$ deben ser tales que si $\theta = 0$ se cumple
\begin{align}
\eval{\va{b}}_{\phi=0} &= \vu{e}_{\theta} = (\sin \psi \cos \theta, \cos \psi \cos \theta, - \sin \theta) \label{eq:ecuacion_10_054} \\[1em]
\eval{\va{c}}_{\phi=0} &= \vu{e}_{\psi} = (\cos \psi, -\sin \psi, 0) \label{eq:ecuacion_10_055}
\end{align}
Un par de vectores que satisfacen estas condiciones son:
\begin{changemargin}{-0.25cm}{-0.25cm} 
\begin{align*}
\va{a} &= (\sin \phi \sin \psi \cos \theta + \cos \phi \cos \psi, - \sin \phi \cos \psi \cos \theta - \cos \phi \sin \psi, \sin \phi \sin \theta) \\[1em]
\va{b} &= (\sin \phi \cos \psi + \cos \theta \sin \psi \cos \theta, -\sin \phi \sin \psi + \cos \phi \cos \psi \cos \theta, - \cos \phi \sin \theta)
\end{align*}
\end{changemargin}
Se puede demostrar que los vectores $\va{a}, \va{b}$ y $\va{c}$ cumplen con la ec. (\ref{eq:ecuacion_10_049}). De esta manera $\Lambda$ se puede escribir como
\begin{align}
\begin{aligned}
&\Lambda (\phi, \theta, \psi) = \\[0.5em]
&= \mqty(
c \, \phi \, c \, \psi - s \, \phi \, s \, \psi \, c \, \theta & s \, \theta \, c \, \phi + c \, \phi \, s \, \psi \, c \, \theta & s \, \psi \, s \, \theta \\
- c \, \phi \, s \, \theta \, - s \, \phi \, c \, \psi c \, \theta & - s \, \phi \, s \, \psi + c \, \phi \, c \, \psi \, c \, \theta & c \, \psi \, s \, \theta \\
s \, \phi \, s \, \theta & - c \, \phi \, s \, \theta & c \, \theta)
\end{aligned}
\label{eq:ecuacion_10_056}
\end{align}
Claramente aún hay cierta arbitrariedad, pues podemos cambiar el lugar de los vectores $\va{a}, \va{b}, \va{c}$, también podemos cambiar renglones por columnas y se seguirá cumpliendo la eq. (\ref{eq:ecuacion_10_043}). La ventaja de escribir $\Lambda$ de la forma (\ref{eq:ecuacion_10_056}) es que se puede expresar como el producto de tres matrices
\begin{align}
\Lambda (\phi, \theta, \psi) = \Lambda_{1} (\phi) \, \Lambda_{2} (\theta) \, \Lambda_{3} (\psi) 
\label{eq:ecuacion_10_057}
\end{align}
con
\begin{align}
\Lambda_{1} (\phi) = \mqty(\cos \phi & \sin \phi & 0 \\ - \sin \phi & \cos \phi & 0 \\ 0 & 0 & 1) \label{eq:ecuacion_10_058} \\[1em]
\Lambda_{2} (\theta) = \mqty(1 & 0 & 0 \\ 0 & \cos \theta & \sin \theta \\ 0 & - \sin \theta & \cos \theta) \label{eq:ecuacion_10_059} \\[1em]
\Lambda_{3} (\psi) = \mqty(\cos \psi & \sin \psi & 0 \\ - \sin \psi & \cos \psi & 0 \\ 0 & 0 & 1) \label{eq:ecuacion_10_060}
\end{align}
Esta descomposición es muy útil para el estudio del movimiento del cuerpo rígido. A los parámetros $\theta, \psi, \phi$ se les llama \emph{ángulos de Euler}.
\par
Las rotaciones no son las únicas transformaciones de $O(3)$. Por ejemplo, las matrices
\begin{align}
\mqty(-1 & 0 & 0 \\ 0 & 1 & 0 \\ 0 & 0 & 1) \hspace{1cm} \mqty(1 & 0 & 0 \\ 0 & -1 & 0 \\ 0 & 0 & -1) \hspace{1cm} \mqty(-1 & 0 & 0 \\ 0 & -1 & 0 \\ 0 & 0 & -1)
\label{eq:ecuacion_10_061}
\end{align}
cumplen la ec. (\ref{eq:ecuacion_10_049}) pero no son de rotación, tampoco son de $SO(3)$. Las rotaciones representan las transformaciones de $O(3)$ que se pueden conectar con la matriz unidad $I$. Pues al hacer $\phi = 0, \theta = 0, \psi = 0$, se obtiene la unidad $I$.
\section{Transformaciones infinitesimales.}
Las transformaciones de $SO(n)$ que están infinitesimalmente cercanas a la unidad, es decir, aquellas que cumplen
\begin{align}
\Lambda \approx I + \varepsilon \, M \hspace{1.5cm} \varepsilon \ll 1
\label{eq:ecuacion_10_062}
\end{align}
son particularmente importantes. Para este tipo de transformaciones la condición dada por la eq. (\ref{eq:ecuacion_10_043}) implica
\begin{align}
I = \Lambda^{T} \, \Lambda &\approx (I + \varepsilon \, M)^{T} \, (I + \varepsilon \, M) \label{eq:ecuacion_10_063} \\[0.5em]
&= (I + \varepsilon \, M^{T}) \, (I + \varepsilon \, M)  \approx I + \varepsilon (M^{T} + M) \label{eq:ecuacion_10_064}
\end{align}
Por lo tanto
\begin{align}
M = - M^{T}
\label{eq:ecuacion_10_065}
\end{align}
es decir, $M$ debe ser antisimétrica. Nótese que una matriz antisimétrica de $n \times n$ sólo puede tener
\begin{align*}
\dfrac{n(n-1)}{2}
\end{align*}
parámetros libres, este número de grados de libertad coincide con los parámetros libres del grupo $SO(n)$. Para el caso particular $n = 3$, cualquier matriz antisimétrica se puede escribir como
\begin{align}
M = \mqty(0 & -\alpha_{3} & \alpha_{2} \\ \alpha_{3} & 0 & -\alpha_{1} \\ -\alpha_{2} & \alpha_{1} & 0)
\label{eq:ecuacion_10_066}
\end{align}
Definamos los vectores $\va{r} = (x, y ,z)$, $\delta \, \va{\alpha} = \varepsilon (\alpha_{1}, \alpha_{2}, \alpha_{3})$, entonces
\begin{align}
\delta \, X &= \varepsilon \, M \, X = \varepsilon \, \mqty(0 & -\alpha_{3} & \alpha_{2} \\ \alpha_{3} & 0 & -\alpha_{1} \\ -\alpha_{2} & \alpha_{1} & 0) \mqty(x \\ y \\ z) = \nonumber \\[0.5em]
&= \delta \, \va{\alpha} \cp \va{r} \label{eq:ecuacion_10_067}
\end{align}
Por lo tanto, una rotación infinitesimal está dada por
\begin{align}
X^{\prime} = \Lambda \, X \approx ( I + \varepsilon \, M) \, X = (I \, M + \varepsilon \, M \, X)
\label{eq:ecuacion_10_068}
\end{align}
es decir
\begin{align}
\va{r}^{\, \prime} = \va{r} + \delta \, \va{\alpha} \cp \va{r}
\label{eq:ecuacion_10_069}
\end{align}
Estas rotaciones son importantes porque definen al resto de las transformaciones de $SO(3)$, para ver esto primero notemos que
\begin{align}
M &= \mqty(0 & -\alpha_{3} & \alpha_{2} \\ \alpha_{3} & 0 & -\alpha_{1} \\ -\alpha_{2} & \alpha_{1} & 0) = \alpha_{1} \, m_{1} + \alpha_{2} \, m_{2} + \alpha_{3} \, m_{3} = \nonumber \\[0.5em]
&= \va{\alpha} \cdot \va{m} \label{eq:ecuacion_10_070}
\end{align}
donde
\begin{align*}
m_{1} = \mqty(0 & 0 & 0 \\ 0 & 0 & -1 \\ 0 & 1 & 0), \hspace{1cm} m_{2} = \mqty(0 & 0 & 1 \\ 0 & 0 & 0 \\ -1 & 0 & 0), \hspace{1cm} m_{3} = \mqty(0 & -1 & 0 \\ 1 & 0 & 0 \\ 0 & 0 & 0)
\end{align*}
Se puede observar que la matriz (\ref{eq:ecuacion_10_070}) no se modifica si hacemos el cambio $\va{\alpha} \rightarrow - i \, \va{\alpha}$ y $\va{m} \rightarrow - i \, \va{m} = \va{M}$, es decir
\begin{align*}
M_{1} = i \, \mqty(0 & 0 & 0 \\ 0 & 0 & -1 \\ 0 & 1 & 0), \hspace{1cm} M_{2} = i \, \mqty(0 & 0 & 1 \\ 0 & 0 & 0 \\ -1 & 0 & 0), \hspace{1cm} M_{3} = i \, \mqty(0 & -1 & 0 \\ 1 & 0 & 0 \\ 0 & 0 & 0)
\end{align*}
La ventaja de ocupar la matrices $M_{i}$ es que son Hermíticas, es decir, $M_{i}^{\dagger} = M_{i}$. Esto implica que sus valores propios son reales, por lo tanto las matrices
$\va{M} = (M_{1} , M_{2} , M_{3})$ pueden representar cantidades físicas.
\par
Una transformación finita se debe hacer como producto infinito de transformaciones infinitesimales. Por ejemplo, ocupando
\begin{align}
X^{\prime} &\approx \left( \Lambda \left( \dfrac{\va{\alpha}}{N} \right) \right)^{N} \, X = \nonumber \\[0.5em]
&= (I - i \, \delta \, \va{\alpha} \cdot \va{M})^{N} \, X = \nonumber \\[0.5em]
&= \left( I - i \, \dfrac{1}{N} \, \va{\alpha} \cdot \va{M} \right)^{N} \, X \label{eq:ecuacion_10_071}
\end{align}
y considerando el resultado
\begin{align}
\lim_{N \to \infty} \left(  I + \dfrac{\beta \, x}{N} \right)^{N} = e^{\beta \, x}
\label{eq:ecuacion_10_072}
\end{align}
se tiene que
\begin{align}
\lim_{N \to \infty} \left(  I - i \, \dfrac{1}{N} \, \va{\alpha} \cdot \va{M} \right)^{N} = e^{- i \, \va{\alpha} \cdot \va{M}}
\label{eq:ecuacion_10_073}
\end{align}
Por lo tanto, una transformación finita, está dada por
\begin{align}
X^{\prime} = e^{- i \, \va{\alpha} \cdot \va{M}} \, X
\label{eq:ecuacion_10_074}
\end{align}
Así, cualquier transformación infinitesimal conectada con la identidad tiene la forma
\begin{align}
\Lambda (\va{\alpha}) = e^{- i \, \va{\alpha} \cdot \va{M}}
\label{eq:ecuacion_10_075}
\end{align}
En este sentido se dice que $M_{1}, M_{2}, M_{3}$ son los generadores del grupo $SO(3)$.
\par
Ahora veamos de forma explícita la expresión dada por la ec.(\ref{eq:ecuacion_10_075}) para algunos casos particulares. Primero notemos que si $n \geq 1$, se tiene
\begin{align}
M_{1}^{2n} = \mqty(0 & 0 & 0 \\ 0 & 1 & 0 \\ 0 & 0 & 1) = T_{1}, \hspace{1cm} M_{1}^{2n + 1} = M_{1}  \label{eq:ecuacion_10_076} \\[1em]
M_{2}^{2n} = \mqty(1 & 0 & 0 \\ 0 & 0 & 0 \\ 0 & 0 & 1) = T_{2}, \hspace{1cm} M_{2}^{2n + 1} = M_{2}  \label{eq:ecuacion_10_077} \\[1em]
M_{3}^{2n} = \mqty(1 & 0 & 0 \\ 0 & 1 & 0 \\ 0 & 0 & 0) = T_{3}, \hspace{1cm} M_{3}^{2n + 1} = M_{3}  \label{eq:ecuacion_10_078}
\end{align}
Entonces, considerando estos resultados, juntos con las series de $\cos \beta$ y $\sin \beta$, se tiene
\begin{align}
e^{i \, \beta \, M_{i}} &= \sum_{n \geq 0} \dfrac{1}{n!} \, (- i \, \beta \, M_{i})^{n} \nonumber \\[0.5em]
&= \sum_{n \geq 0} \dfrac{1}{(2 n)!} \, (- i \, \beta)^{2 n} \, (M_{i})^{2 n} + \sum_{n \geq 0} \dfrac{1}{(2 n +1)!} \, (- i \, \beta)^{2 n + 1} \, (M_{i})^{2 n + 1} = \nonumber \\[0.5em]
&= I + T_{i} \, \sum_{n \geq 1} \dfrac{1}{(2 n)!} \, (- i \, \beta)^{2 n} + M_{i} \, \sum_{n \geq 0} \dfrac{1}{(2 n +1)!} \, (- i \, \beta)^{2 n + 1} = \nonumber \\[0.5em]
&= I - T{i} + T_{i} + \sum_{n \geq 1} \dfrac{(-1)^{n}}{(2 n)!} \, \beta^{2 n} - i \, M \, \sum_{n \geq 0} \dfrac{(-1)^{n}}{(2 n +1)!} \, \beta^{2 n + 1} = \nonumber \\[0.5em]
&= I - T{i} + T_{i} \, \sum_{n \geq 0} \dfrac{(-1)^{n} \beta^{2 n}}{(2 n)!} - i \, M_{i} \, \sin \beta \nonumber \\[0.5em]
&= I - T{i} + T_{i} \, \cos \beta - i \, M_{i} \, \sin \beta \label{eq:ecuacion_10_079}
\end{align}
De donde
\begin{align}
\begin{aligned}
e^{-i \, \alpha_{1} \, M_{1}} = \mqty(1 & 0 & 0 \\ 0 & \cos \alpha_{1} & - \sin \alpha_{1} \\ 0 & \sin \alpha_{1} & \cos \alpha_{1}) \\[0.5em]
e^{-i \, \alpha_{2} \, M_{2}} = \mqty(\cos \alpha_{2} & 0 & \sin \alpha_{2} \\ 0 & 1 & 0 \\ - \sin \alpha_{2} & 0 & \cos \alpha_{2}) \\[0.5em]
e^{-i \, \alpha_{3} \, M_{3}} = \mqty(\cos \alpha_{3} & - \sin \alpha_{3} & 0 \\ \sin \alpha_{3} & \cos \alpha_{3} & 0 \\ 0 & 0 & 1)
\end{aligned}
\label{eq:ecuacion_10_080}
\end{align}
Así, $e^{-i \, \alpha_{1} \, M_{1}}$ representa una rotación sobre el eje $x$, $e^{-i \, \alpha_{2} \, M_{2}}$ representa una rotación sobre el eje $y$, mientras que $e^{-i \, \alpha_{3} \, M_{3}}$ representa una rotación sobre el eje $z$.
\par
Veamos que reglas de conmutación cumplen los generadores de las rotaciones $M_{1}, M_{2}, M_{3}$. Primero notemos que
\begin{align}
\begin{aligned}
M_{1} \, M_{2} = - \mqty(0 & 0 & 0 \\ 1 & 0 & 0 \\ 0 & 0 & 0) \\[0.5em] 
M_{1} \, M_{3} = - \mqty(0 & 0 & 0 \\ 0 & 0 & 0 \\ 1 & 0 & 0) \\[0.5em]
M_{2} \, M_{3} = - \mqty(0 & 0 & 0 \\ 0 & 0 & 0 \\ 0 & 1 & 0)
\end{aligned}
\qquad
\begin{aligned}
M_{2} \, M_{1} = - \mqty(0 & 1 & 0 \\ 0 & 0 & 0 \\ 0 & 0 & 0) \\[0.5em]
M_{3} \, M_{1} = - \mqty(0 & 0 & 1 \\ 0 & 0 & 0 \\ 0 & 0 & 0) \\[0.5em]
M_{3} \, M_{2} = - \mqty(0 & 0 & 0 \\ 0 & 0 & 1 \\ 0 & 0 & 0)
\end{aligned}
\label{eq:ecuacion_10_081}
\end{align}
por lo que
\begin{align}
\comm{M_{1}}{M_{2}} = i \, M_{3}, \hspace{1cm} \comm{M_{3}}{M_{1}} = i \, M_{2}, \hspace{1cm} \comm{M_{2}}{M_{3}} = i \, M_{1} \label{eq:ecuacion_10_082}
\end{align}
Estas reglas de conmutación se pueden escribir como
\begin{align}
\comm{M_{i}}{M_{j}} = i \, \varepsilon_{ijk} \, M_{k}
\label{eq:ecuacion_10_083}
\end{align}
Por el hecho de que el conmutador de dos generadores de rotación nos de otro generador de rotación, se dice que \emph{estos generadores forman un álgebra de Lie}, el álgebra de Lie de $SO(3)$. Nótese que el conmutador entre dos generadores no es cero, dos generadores no se pueden diagonalizar simultáneamente. Ahora definamos
\begin{align}
M^{2} = M_{1}^{2} + M_{2}^{2} + M_{3}^{2}
\label{eq:ecuacion_10_084}
\end{align}
de donde
\begin{align}
M^{2} = 2 \, I
\label{eq:ecuacion_10_085}
\end{align}
por lo tanto
\begin{align}
\comm{M^{2}}{M_{i}} = 0
\label{eq:ecuacion_10_086}
\end{align}
Así, los valores propios de $M^{2}$ se pueden obtener al mismo tiempo que los generadores $M_{i}$.
\par
Los valores propios de los generadores $M_{i}$ se pueden calcular directamente y están dados por
\begin{align}
M_{1} \, V = \lambda \, V \hspace{0.5cm} \Rightarrow \hspace{0.5cm} \lambda = \pm 1 \hspace{1cm} V = a \dfrac{1}{\sqrt{2}} \mqty(0 \\ 1 \\ \pm i) \label{eq:ecuacion_10_087} \\[0.5em]
M_{2} \, V = \lambda \, V \hspace{0.5cm} \Rightarrow \hspace{0.5cm} \lambda = \pm 1 \hspace{1cm} V = a \dfrac{1}{\sqrt{2}} \mqty(1 \\ 0 \\ \pm i) \label{eq:ecuacion_10_088} \\[0.5em]
M_{3} \, V = \lambda \, V \hspace{0.5cm} \Rightarrow \hspace{0.5cm} \lambda = \pm 1 \hspace{1cm} V = a \dfrac{1}{\sqrt{2}} \mqty(1 \\ \pm i \\ 0) \label{eq:ecuacion_10_089}
\end{align}
Si $a = 1$, en cada caso se tienen vectores propios ortonormales. Estos resultados no son difíciles de obtener, posteriormente veremos que los generadores de las rotaciones en el espacio de las funciones de $\va{x}$ están relacionados con operadores diferenciales.
\par
Con el símbolo $\epsilon_{ijk}$ las matrices $\va{M} = (M_{1}, M_{2}, M_{3})$ se pueden escribir de forma más económica. En efecto, en componentes tenemos
\begin{align}
(M_{1})_{ij} = i \, \epsilon_{i 1 j} \hspace{1cm} (M_{2})_{ij} = i \, \epsilon_{i 2 j} \hspace{1cm} (M_{3})_{ij} = i \, \epsilon_{i 3 j} \label{eq:ecuacion_10_090}
\end{align}
Por ejemplo
\begin{align}
(M_{1})_{ij} = i \, \epsilon_{i 1 j} = i \, \mqty(\epsilon_{111} & \epsilon_{112} & \epsilon_{113} \\ \epsilon_{211} & \epsilon_{212} & \epsilon_{213} \\ \epsilon_{311} & \epsilon_{312} & \epsilon_{313}) = i \, \mqty(0 & 0 & 0 \\ 0 & 0 & -1 \\ 0 & 1 & 0)
\label{eq:ecuacion_10_091}
\end{align}
se puede demostrar que las demás igualdades se cumplen.
\par
Así, las componentes de cualquier matriz antisimétrica se pueden escribir como
\begin{align}
M_{ik} = - i \, \alpha_{j} (M_{j})_{ik} = \alpha_{j} \, \epsilon_{ijk}
\label{eq:ecuacion_10_092}
\end{align}
Ahora, si $\alpha_{i}$ es infinitesimal, por ejemplo $\delta \, \alpha_{i} = \alpha_{i} / N$ con $N$ grande, una transformación infinitesimal en el espacio de $\va{x}$ está dada por
\begin{align}
X^{\prime} = \Lambda \, X = (I - i \, \delta \, \va{\alpha} \cdot \va{M}) \, X
\label{eq:ecuacion_10_093}
\end{align}
En componentes se tiene
\begin{align}
x_{i}^{\prime} &\approx \Lambda_{ik} \, x_{k} = (I - i \, \delta \, \alpha_{j} \, M_{j})_{ik} \, x_{k} = \nonumber \\[0.5em]
&= (\delta_{ik} - i \, \delta \, \alpha_{j} \, (M_{j})_{ik}) \, x_{k} = \nonumber \\[0.5em]
&= (\delta_{ik} + \delta \, \alpha_{j} \, \epsilon_{ijk}) \, x_{k} = \nonumber \\[0.5em]
&= x_{i} + \epsilon_{ijk} \, \delta \, \alpha_{j} \, x_{k} = \nonumber \\[0.5em]
&= x_{i} + (\delta \, \va{\alpha} \cp \va{x})_{i}
\label{eq:ecuacion_10_094}
\end{align}
es decir
\begin{align}
x_{i}^{\prime} &\approx x_{i} + (\delta \, \va{\alpha} \cp \va{x})_{i} \label{eq:ecuacion_10_095}
\end{align}
\section{Armónicos esféricos.}
Hasta el momento nos hemos enfocado en las trasformaciones que pasan del espacio $\va{r}$ al $\va{r}^{\, \prime}$. Ahora veamos que pasa con las funciones que actúan en estos espacios. Supongamos que tenemos una función, $F$, de $\mathbb{R}^{3} \rightarrow \mathbb{R}$. Entonces al evaluar esta función en un punto $\va{r}^{\, \prime}$ y ocupando la transformación infinitesimal dada por la eq. (\ref{eq:ecuacion_10_069}), se encuentra
\begin{align}
F(\va{r}^{\, \prime}) \approx F (\va{r} + \delta \, \va{\alpha} \cp \va{r}) = F(\va{r}) + (\delta \, \va{\alpha} \cp \va{r}) \cdot \va{\grad} F(\va{r})
\label{eq:ecuacion_10_096}
\end{align}
Usando la propiedad cíclica del triple producto escalar, se tiene que
\begin{align*}
(\delta \, \va{\alpha} \cp \va{r}) \cdot \va{\grad} F(\va{r}) &= (\va{\grad} F(\va{r}) \cp \delta \, \va{\alpha}) \cdot \va{r} \\
&= (\va{r} \cp \va{\grad} F(\va{r})) \cdot \delta \, \va{\alpha} \\
&= i \, (\delta \, \va{\alpha} \cdot \va{L}) \, F(\va{r}) 
\end{align*}
con
\begin{align}
\va{L} = - i \, \va{r} \cp \va{\grad}
\label{eq:ecuacion_10_097}
\end{align}
Entonces tenemos
\begin{align}
F(\va{r}^{\, \prime}) \approx (1 + i \, \delta \va{\alpha} \cdot \va{L}) \, F (\va{r})
\label{eq:ecuacion_10_098}
\end{align}
Por lo tanto, el operador $\va{L}$ es el generador de las rotaciones en el espacio de las funciones. Anteriormente se vio que este operador es Hermítico, por lo que sus valores propios son reales.
\par
Veamos qué forma tiene una rotación finita. Al igual que en caso del espacio $\va{r}$ tomaremos $\delta \, \va{\alpha} = \va{\alpha} / N$ y consideraremos que para tener una transformación finita debemos hacer el producto de un número infinito de transformaciones infinitesimales. Entonces,
\end{document}