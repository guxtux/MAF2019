\documentclass[12pt]{article}
\usepackage[left=0.25cm,top=1cm,right=0.25cm,bottom=1cm]{geometry}
\textwidth = 20cm
\hoffset = -1cm
\usepackage[utf8]{inputenc}
\usepackage[spanish,es-tabla]{babel}
\usepackage[autostyle,spanish=mexican]{csquotes}
\usepackage[tbtags]{amsmath}
\usepackage{nccmath}
\usepackage{amsthm}
\usepackage{amssymb}
\usepackage{graphicx}
\usepackage{standalone}
\usepackage[outdir=./]{epstopdf}
\usepackage{siunitx}
\usepackage{physics}
\usepackage{color}
\usepackage{float}
\usepackage{multicol}
%\usepackage{milista}
\usepackage{enumitem}
\usepackage{anyfontsize}
\usepackage{anysize}
\usepackage{enumitem}
\usepackage{capt-of}
\usepackage{bm}
\usepackage{relsize}
\usepackage{placeins}
\usepackage{empheq}
\usepackage{cancel}
\usepackage{wrapfig}
\spanishdecimal{.}
\renewcommand{\baselinestretch}{1.5} 
\renewcommand\labelenumii{\theenumi.{\arabic{enumii}}}
\newcommand{\ptilde}[1]{\ensuremath{{#1}^{\prime}}}
\newcommand{\stilde}[1]{\ensuremath{{#1}^{\prime \prime}}}
\newcommand{\ttilde}[1]{\ensuremath{{#1}^{\prime \prime \prime}}}
\newcommand{\ntilde}[2]{\ensuremath{{#1}^{(#2)}}}


\title{Ejercicios opcionales \\[0.3em]  \large{Teorema de adición de los armónicos esféricos} \vspace{-3ex}}
\author{M. en C. Gustavo Contreras Mayén}
\date{ }

\begin{document}
\vspace{-4cm}
\maketitle
\fontsize{14}{14}\selectfont


\noindent
%Ref. Arfken (2006) 12.8.3
\textbf{Ejercicio opcional (18).} El potencial de un electrón en el punto $\vb{r}_{e}$ en el campo de $Z$ protones en puntos $\vb{r}_{p}$ es:
\begin{align*}
\Phi = - \dfrac{e^{2}}{4 \pi \varepsilon_{0}} \, \nsum_{p=1}^{Z} \dfrac{1}{\abs{\vb{r}_{e} - \vb{r}_{p}}}
\end{align*}
\begin{enumerate}[label=\roman*)]
\item Demuestra que esto se puede escribir de la forma:
\begin{align*}
\Phi = - \dfrac{e^{2}}{4 \pi \varepsilon_{0} \, r_{e}} \, \nsum_{p=1}^{Z} \nsum_{\ell, m} \bigg( \dfrac{r_{p}}{r_{e}} \bigg)^{\ell} \dfrac{4 \pi}{2 \ell + 1} \big[ Y_{\ell}^{m} (\theta_{p}, \phi_{p}) \big]^{*} \, Y_{\ell}^{m} (\theta_{e}, \phi_{e})
\end{align*}
donde $r_{e} > r_{p}$.
\item ¿Cómo debe de escribirse $\Phi$ para $r_{e} < r_{p}$?
\end{enumerate}
\end{document}