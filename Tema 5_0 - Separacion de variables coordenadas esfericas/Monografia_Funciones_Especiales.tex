\documentclass[12pt]{article}
\usepackage[left=0.25cm,top=1cm,right=0.25cm,bottom=1cm]{geometry}
\textwidth = 20cm
\hoffset = -1cm
\usepackage[utf8]{inputenc}
\usepackage[spanish,es-tabla]{babel}
\usepackage[autostyle,spanish=mexican]{csquotes}
\usepackage[tbtags]{amsmath}
\usepackage{nccmath}
\usepackage{amsthm}
\usepackage{amssymb}
\usepackage{graphicx}
\usepackage{standalone}
\usepackage[outdir=./]{epstopdf}
\usepackage{siunitx}
\usepackage{physics}
\usepackage{color}
\usepackage{float}
\usepackage{multicol}
%\usepackage{milista}
\usepackage{enumitem}
\usepackage{anyfontsize}
\usepackage{anysize}
\usepackage{enumitem}
\usepackage{capt-of}
\usepackage{bm}
\usepackage{relsize}
\usepackage{placeins}
\usepackage{empheq}
\usepackage{cancel}
\usepackage{wrapfig}
\spanishdecimal{.}
\renewcommand{\baselinestretch}{1.5} 
\renewcommand\labelenumii{\theenumi.{\arabic{enumii}}}
\newcommand{\ptilde}[1]{\ensuremath{{#1}^{\prime}}}
\newcommand{\stilde}[1]{\ensuremath{{#1}^{\prime \prime}}}
\newcommand{\ttilde}[1]{\ensuremath{{#1}^{\prime \prime \prime}}}
\newcommand{\ntilde}[2]{\ensuremath{{#1}^{(#2)}}}


%\usepackage{showframe}
\usepackage{apacite}
\title{Monografía de las funciones especiales \\ \large {Matemáticas Avanzadas de la Física} \vspace{-3ex}}
\author{M. en C. Gustavo Contreras Mayén}
\date{ }
\begin{document}
\vspace{-4cm}
\maketitle
\fontsize{14}{14}\selectfont
En la segunda parte del curso estaremos revisando un conjunto de funciones especiales de la Física Matemática, aparte de la consulta de los materiales de trabajo, así como de los textos clásicos, consideramos una buena oportunidad que elabores un resumen sobre cada una de las funciones especiales:
\begin{enumerate}
\item Indicar el o los ejemplos de la física a partir de los cuales se obtienen las expresiones que nos conducen a las funciones especiales.
\item La geometría en particular donde se estudia la función especial.
\item La ecuación diferencial de segundo orden.
\item Las soluciones correspondientes.
\item La solución general de la ecuación diferencial de segundo orden.
\item La función generatriz.
\item Algunas de las relaciones de recurrencia.
\item La expresión integral (en los casos que aplique)
\item La propiedad de paridad.
\item Las condiciones de ortogonalidad y normalización.
\item La fórmula de Rodrigues.
\item La relación de completez.
\end{enumerate}
De esta manera tendrás un material importante como referencia posterior en la carrera de Física.
\par
Recuerda que lo que revisamos en el curso no es un estudio completo y exhaustivo, sino una revisión general que podrás complementar consultando otros textos de Física Matemática.
\par
A modo de ejemplo puedes construir un esquema de monografía para cada función especial a partir del siguiente modelo:

\renewcommand\arraystretch{2}
\begin{table}[H]
    \centering
\begin{tabular}{| p{5cm} | p{10cm} |} \hline
\multicolumn{2}{|c|}{\textbf{Polinomios ordinarios de Legendre}} \\ \hline
Problema(s) de la Física & \\ \hline
Geometría & \\ \hline
Ecuación diferencial & \\ \hline
Soluciones & \\ \hline
Solución general & \\ \hline
Función generatriz & \\ \hline
Relaciones de recurrencia & \\ \hline
Expresión integral & \\ \hline
Paridad & \\ \hline
$\cdots$ & \\ \hline
\end{tabular}
\end{table}

\end{document}