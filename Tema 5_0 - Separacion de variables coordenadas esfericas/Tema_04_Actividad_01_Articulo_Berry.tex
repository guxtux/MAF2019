\documentclass[hidelinks,12pt]{article}
\usepackage[left=0.25cm,top=1cm,right=0.25cm,bottom=1cm]{geometry}
%\usepackage[landscape]{geometry}
\textwidth = 20cm
\hoffset = -1cm
\usepackage[utf8]{inputenc}
\usepackage[spanish,es-tabla]{babel}
\usepackage[autostyle,spanish=mexican]{csquotes}
\usepackage[tbtags]{amsmath}
\usepackage{nccmath}
\usepackage{amsthm}
\usepackage{amssymb}
\usepackage{mathrsfs}
\usepackage{graphicx}
\usepackage{subfig}
\usepackage{standalone}
\usepackage[outdir=./Imagenes/]{epstopdf}
\usepackage{siunitx}
\usepackage{physics}
\usepackage{color}
\usepackage{float}
\usepackage{hyperref}
\usepackage{multicol}
%\usepackage{milista}
\usepackage{anyfontsize}
\usepackage{anysize}
%\usepackage{enumerate}
\usepackage[shortlabels]{enumitem}
\usepackage{capt-of}
\usepackage{bm}
\usepackage{relsize}
\usepackage{placeins}
\usepackage{empheq}
\usepackage{cancel}
\usepackage{wrapfig}
\usepackage[flushleft]{threeparttable}
\usepackage{makecell}
\usepackage{fancyhdr}
\usepackage{tikz}
\usepackage{bigints}
\usepackage{scalerel}
\usepackage{pgfplots}
\usepackage{pdflscape}
\pgfplotsset{compat=1.16}
\spanishdecimal{.}
\renewcommand{\baselinestretch}{1.5} 
\renewcommand\labelenumii{\theenumi.{\arabic{enumii}})}
\newcommand{\ptilde}[1]{\ensuremath{{#1}^{\prime}}}
\newcommand{\stilde}[1]{\ensuremath{{#1}^{\prime \prime}}}
\newcommand{\ttilde}[1]{\ensuremath{{#1}^{\prime \prime \prime}}}
\newcommand{\ntilde}[2]{\ensuremath{{#1}^{(#2)}}}

\newtheorem{defi}{{\it Definición}}[section]
\newtheorem{teo}{{\it Teorema}}[section]
\newtheorem{ejemplo}{{\it Ejemplo}}[section]
\newtheorem{propiedad}{{\it Propiedad}}[section]
\newtheorem{lema}{{\it Lema}}[section]
\newtheorem{cor}{Corolario}
\newtheorem{ejer}{Ejercicio}[section]

\newlist{milista}{enumerate}{2}
\setlist[milista,1]{label=\arabic*)}
\setlist[milista,2]{label=\arabic{milistai}.\arabic*)}
\newlength{\depthofsumsign}
\setlength{\depthofsumsign}{\depthof{$\sum$}}
\newcommand{\nsum}[1][1.4]{% only for \displaystyle
    \mathop{%
        \raisebox
            {-#1\depthofsumsign+1\depthofsumsign}
            {\scalebox
                {#1}
                {$\displaystyle\sum$}%
            }
    }
}
\def\scaleint#1{\vcenter{\hbox{\scaleto[3ex]{\displaystyle\int}{#1}}}}
\def\bs{\mkern-12mu}


\title{Control de Lectura \\ \large {Artículo de Michael Berry}\vspace{-3ex}}

\author{M. en C. Gustavo Contreras Mayén}
\date{ }

\begin{document}
\maketitle
\fontsize{14}{14}\selectfont

A partir de la lectura del artículo de Michael Berry \cite{Berry} responde las siguientes preguntas:

\begin{enumerate}[label=\alph*)]
\item Identifica cuántas funciones especiales menciona el autor en su artículo. En caso de que tengas duda sobre si es o no función especial, te puedes apoyar con una consulta en internet, es decir, busca la referencia, en el resultado se mencionará si es considerada una función especial.
\item ¿Consideras que el desarrollo de la computación ha sido importante para entender y aplicar las funciones especiales? Menciona un ejemplo que el autor cite en su trabajo.
\item Revisamos en la presentación del Tema 4, la siguiente cita\footnote{Bravo, Y. S. (2006). \textit{Métodos Matemáticos Avanzados para Científicos e Ingenieros}, Colección Manuales UEX 48, Madrid, España}:
\begin{quote}
Las funciones (quizás mal llamadas) especiales de la Física Matemática no tienen nada de \enquote{especial}. En principio son tan \enquote{especiales} como las funciones trigonométricas o los logaritmos, aunque por supuesto son menos habituales.
\par
Un nombre más adecuado sería, sin duda, el de \emph{funciones útiles}. En todo caso, es lícito preguntarse por el motivo de estudiar estas funciones y sus propiedades.
\end{quote}

¿Cuál es tu opinión al respecto de la postura de Bravo y la de Berry?, ¿hay coincidencias? ¿hay diferencias?
\end{enumerate}


\begin{thebibliography}{X}
\bibitem{Berry} Berry, M. (2001). \textit{Why are special functions special?}, Physics Today, \textbf{54} 4, 11, doi.org/10.1063/1.1372098.
\end{thebibliography}
\end{document}


