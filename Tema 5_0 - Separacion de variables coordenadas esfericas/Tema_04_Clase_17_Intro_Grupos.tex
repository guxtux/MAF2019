\documentclass[12pt]{beamer}
\usepackage{../Estilos/BeamerMAF}
\usepackage{arydshln}
%Sección para el tema de beamer, con el theme, usercolortheme y sección de footers
\usetheme{Frankfurt}
\usecolortheme{beaver}
%\useoutertheme{default}
\setbeamercovered{invisible}
% or whatever (possibly just delete it)
\setbeamertemplate{section in toc}[sections numbered]
\setbeamertemplate{subsection in toc}[subsections numbered]
\setbeamertemplate{subsection in toc}{\leavevmode\leftskip=3.2em\rlap{\hskip-2em\inserttocsectionnumber.\inserttocsubsectionnumber}\inserttocsubsection\par}
% \setbeamercolor{section in toc}{fg=blue}
% \setbeamercolor{subsection in toc}{fg=blue}
% \setbeamercolor{frametitle}{fg=blue}
\setbeamertemplate{caption}[numbered]

\setbeamertemplate{footline}
\beamertemplatenavigationsymbolsempty
\setbeamertemplate{headline}{}


\makeatletter
% \setbeamercolor{section in foot}{bg=gray!30, fg=black!90!orange}
% \setbeamercolor{subsection in foot}{bg=blue!30!yellow, fg=red}
% \setbeamercolor{date in foot}{bg=black, fg=white}
\setbeamertemplate{footline}
{
  \leavevmode%
  \hbox{%
  \begin{beamercolorbox}[wd=.333333\paperwidth,ht=2.25ex,dp=1ex,center]{section in foot}%
    \usebeamerfont{section in foot} \insertsection
  \end{beamercolorbox}%
  \begin{beamercolorbox}[wd=.333333\paperwidth,ht=2.25ex,dp=1ex,center]{subsection in foot}%
    \usebeamerfont{subsection in foot}  \insertsubsection
  \end{beamercolorbox}%
  \begin{beamercolorbox}[wd=.333333\paperwidth,ht=2.25ex,dp=1ex,right]{date in head/foot}%
    \usebeamerfont{date in head/foot} \insertshortdate{} \hspace*{2em}
    \insertframenumber{} / \inserttotalframenumber \hspace*{2ex} 
  \end{beamercolorbox}}%
  \vskip0pt%
}







\title{\large{Teoría de grupos}}
\subtitle{Tema 4}

\author{M. en C. Gustavo Contreras Mayén}
\date{19 de noviembre de 2021}

\begin{document}
\maketitle
\fontsize{14}{14}\selectfont
\spanishdecimal{.}

\section*{Contenido}
\frame[allowframebreaks]{\tableofcontents[currentsection, hideallsubsections]}

%Ref. Arfken (1981) 4.7 Introducción a la teoría de grupos
\section{Marco de referencia}
\frame{\tableofcontents[currentsection, hideothersubsections]}
\subsection{Motivación}

\begin{frame}
\frametitle{Usando la teoría de grupos}
La teoría de los grupos finitos, desarrollada originalmente como una rama de las matemáticas puras, puede convertirse en algo divertido, bonito y fascinante.
\end{frame}
\begin{frame}
\frametitle{Usando la teoría de grupos}    
Para el físico, la teoría de grupos, sin ninguna pérdida de lo divertido, bonito y fascinante, también constituye una herramienta de extraordinaria utilidad en la formalización de los conceptos semiintuitivos y para la explotación de las simetrías.
\end{frame}
\begin{frame}
\frametitle{Usando la teoría de grupos}    
La teoría de grupos se transforma en una herramienta útil en el desarrollo de la física de la \textcolor{blue}{cristalografía} y del \textcolor{red}{estado sólido} cuando se introducen las representaciones específicas (\emph{matrices}) y se inicia el cálculo de los caracteres de grupo (\emph{trazas}).
\end{frame}
\begin{frame}
\frametitle{Relevancia de la teoría de grupos}
Probablemente de mayor importancia en la física es la extensión de la teoría de grupos a los grupos continuos y la aplicación de estos grupos continuos a la \emph{teoría cuántica} y \emph{las partículas de la física de altas energías}.
\end{frame}
\begin{frame}
\frametitle{Desarrollando teorías}
A medida que el conocimiento de nuestro mundo físico se detonó en la primera tercera parte del siglo pasado, Wigner y otros comprendieron que la \emph{invariancia} era un concepto clave en la comprensión de los nuevos fenómenos y en el desarrollo de las teorías apropiadas.
\end{frame}
\begin{frame}
\frametitle{La herramienta matemática necesaria}
La herramienta matemática para el tratamiento de la invariancia y las simetrías es la \textbf{teoría de grupos}.
\\
\bigskip
\pause
Representa la unificación y formalización de los principios tales como: la \emph{paridad} y el \textcolor{OliveGreen}{momento angular}, de amplia utilización en la mecánica cuántica.
\end{frame}
\begin{frame}
\frametitle{Principios relacionados}
La paridad está relacionada con la invariancia bajo la inversión.
\\
\bigskip
\pause
La conservación del momento angular es una consecuencia directa de la simetría rotacional, lo cual significa invariancia bajo las rotaciones espaciales.
\end{frame}
\begin{frame}
\frametitle{La herramienta matemática necesaria}
Aun cuando las técnicas formales de la teoría de grupos pudieran no ser necesarias, estas técnicas matemáticas poderosas pueden ahorrar mucho trabajo.
\\
\bigskip
\pause
La teoría de grupos puede producir la unificación que (una vez comprendida) permite lograr una mayor simplificación.
\end{frame}
\begin{frame}
\frametitle{¿Quién fue Wigner?}
Eugene Paul Wigner (Budapest, 1902 - Princeton, 1995). \pause Fue un físico norteamericano de origen húngaro.
\\
\bigskip
\pause
Los trabajos de Wigner versaron sobre la física de sólidos, los núcleos atómicos y los reactores nucleares.
\end{frame}
\begin{frame}
\frametitle{Los trabajos de Wigner}
En física nuclear, formuló el \emph{principio de la simetría} de las partículas elementales o de conservación de la paridad.
\pause
\\
\bigskip
Es muy conocida su hipótesis de que las energías potenciales de interacción entre nucleones son iguales si tienen el mismo momento angular y el mismo spin.
\end{frame}

\section{Teoría de grupos}
\frame{\tableofcontents[currentsection, hideothersubsections]}
\subsection{Definiciones}

\begin{frame}
\frametitle{Definición de grupo}
Un grupo $G$ puede definirse como un conjunto de objetos u operaciones (denominados \emph{elementos}), que pueden combinarse o \enquote{multiplicarse} para formar un producto bien definido y que satisface las siguientes cuatro condiciones.
\end{frame}
\begin{frame}
\frametitle{Definición de grupo}
Establecemos el conjunto de elementos $a, b, c, \ldots$ :
\setbeamercolor{item projected}{bg=blue!70!black,fg=yellow}
\setbeamertemplate{enumerate items}[circle]
\begin{enumerate}[<+->]
\item Si $a$ y $b$ son dos elementos cualesquiera, entonces el producto $a \, b$ también es un miembro del conjunto.
\item La multiplicación definida es asociativa: $(a \, b) \, c =  a \, (b \, c)$.
\seti
\end{enumerate}
\end{frame}
\begin{frame}
\frametitle{Definición de grupo}
\setbeamercolor{item projected}{bg=blue!70!black,fg=yellow}
\setbeamertemplate{enumerate items}[circle]
\begin{enumerate}[<+->]    
\conti
\item Existe un elemento unitario $I$ tal que $I \, a = a \, I = a$ para cada elemento en el conjunto.
\item Debe existir un inverso o recíproco de cada elemento. El conjunto debe contener un elemento $b = a^{-1}$ tal que $a \, a^{-1} = a^{-1} \, a = I$ para cada elemento del conjunto.
\end{enumerate}
\end{frame}
\begin{frame}
\frametitle{Los grupos en física}
En la física, estas condiciones abstractas frecuentemente adquieren un significado físico directo en términos de las transformaciones de los vectores, spins y tensores.
\end{frame}
\begin{frame}
\frametitle{Ejemplo de grupo}
Como ejemplo simple, pero no trivial, considera el conjunto $1, a, b, c$ que se combina de acuerdo con la tabla de multiplicación de grupos:
\end{frame}
\begin{frame}
\frametitle{Ejemplo de grupo}
\begin{table}[H]
\large
\centering
\begin{tabular}{c | c c c c}
& $1$ & \multicolumn{1}{c:}{$a$} & \multicolumn{1}{c:}{$b$} & $c$ \\ \hline \pause
$1$ & $1$ & \multicolumn{1}{c:}{$a$} & \multicolumn{1}{c:}{$b$} & $c$ \\  \cdashline{1-3} \pause
$a$ & $a$ & $b$ & \multicolumn{1}{c:}{$c$} & $1$ \\ \cdashline{1-4} \pause
$b$ & $b$ & $c$ & $1$ & $a$ \\ 
$c$ & $c$ & $1$ & $a$ & $b$ \\ 
\end{tabular}
\end{table}
\end{frame}
\begin{frame}
\frametitle{Representación de los elementos}
Para representar estos elementos de grupo, sea:
\pause
\begin{align}
1 \to 1, \hspace{0.5cm} a \to i, \hspace{0.5cm} b \to -1, \hspace{0.5cm} c \to -i
\label{eq:ecuacion_04_183}
\end{align}
combinando mediante la multiplicación ordinaria.
\end{frame}
\begin{frame}
\frametitle{Comprobando que se tiene un grupo}
Evidentemente, las cuatro condiciones de grupo se satisfacen, y estos cuatro elementos forman un grupo.
\\
\bigskip
\pause
Ya que la multiplicación de los elementos de grupo es conmutativa, el grupo se denomina \textcolor{deepblue}{conmutativo o abeliano}.
\end{frame}
\begin{frame}
\frametitle{Comprobando que se tiene un grupo}
Nuestro grupo también es un \textcolor{ao}{grupo cíclico} en cuanto a que los elementos pueden indicar­se como potencias sucesivas de un elemento, que en este caso es:
\pause
\begin{align*}
i^{n}, n = 0, 1, 2, 3
\end{align*}
\\
\bigskip
\pause
Observa que al establecer la ec. \ref{eq:ecuacion_04_183}, hemos seleccionado una \emph{representación específica} para este grupo de cuatro objetos.
\end{frame}
\begin{frame}
\frametitle{Elementos de grupo como rotaciones}
Podemos reconocer que los elementos del grupo $1, i, -1, -i$ pueden interpretarse como rotaciones de $\SI{90}{\degree}$ sucesivas en el plano complejo.
\\
\bigskip
\pause
Luego, a partir de:
\pause
\begin{align}
\vb{A} = \mqty(
\cos \varphi & \sin \varphi \\
-\sin \varphi & \cos \varphi )
\end{align}
\end{frame}

\subsection{Elementos de grupo como matrices}

\begin{frame}
\frametitle{Elementos de grupo como matrices}
Se establece el conjunto de cuatro matrices de $2 \times 2$:
\pause
\begin{align}
\begin{aligned}
\vb{1} &= \mqty(
1 & 0 \\
0 & 1 ) \hspace{2cm} \vb{A} = \mqty(
0 & -1 \\
1 & 0 ) \\[1em]
\vb{B} &= \mqty(
-1 & 0 \\
0 & -1 ) \hspace{1.5cm} \vb{C} = \mqty(
0 & 1 \\
-1 & 0 )
\end{aligned}
\label{eq_ecuacion_04_184}
\end{align}
\end{frame}
\begin{frame}
\frametitle{Representación en términos de matrices}
Este conjunto de cuatro matrices forman un grupo en que la ley de combinación es la multiplicación de matrices.
\\
\bigskip
\pause
Esto establece una segunda representación, ahora en términos de matrices. \pause Con un poco de trabajo y con la multiplicación de matrices se verifica que esta representación también es abeliana y cíclica. 
\end{frame}
\begin{frame}
\frametitle{Relación entre las representaciones}
Evidentemente existe una correspondencia entre las dos representaciones:
\pause
\begin{eqnarray*}
&1 \leftrightarrow 1 \leftrightarrow \vb{1} \hspace{1cm}  \pause a \leftrightarrow i \leftrightarrow \vb{A} \\[0.5em] \pause
&b \leftrightarrow -1 \leftrightarrow \vb{B} \hspace{1cm} \pause c \leftrightarrow -i \leftrightarrow \vb{C}
\end{eqnarray*} 
\end{frame}

\subsection{Isomorfismo y Homomorfismo}

\begin{frame}
\frametitle{Relación entre los elementos de dos grupos}
Si la correspondencia entre los elementos de dos grupos (o entre sus representaciones) es:
\pause
\setbeamercolor{item projected}{bg=blue!70!black,fg=yellow}
\setbeamertemplate{enumerate items}[circle]
\begin{enumerate}[<+->]
\item De \textcolor{blue}{uno a uno} con cada conjunto de elementos satisfaciendo la misma tabla de multiplicación de grupo, \pause se dice que los grupos son \textcolor{alizarin}{isomórficos}. \pause
\item De \textcolor{brown}{dos a uno (o muchos a uno)}, pero aún se conservan las relaciones de multiplicación, \pause entonces los grupos son \textcolor{cadetblue}{homomórficos}.
\end{enumerate}
\end{frame}
\begin{frame}
\frametitle{Identificando el ejemplo}
En el ejemplo que hemos considerado, las dos representaciones $(1, i, -1, -i)$ y $(\vb{1}, \vb{A}, \vb{B}, \vb{C})$ son isomórficas.
\\
\bigskip
\pause
La representación siempre posible pero trivial $(1, 1, 1, 1)$ sería homomórfica.
\end{frame}
\begin{frame}
\frametitle{Un grupo alemán}
En contraste con esto, no existe correspondencia entre cualquiera de estas representaciones y otro grupo de cuatro objetos, el vierergruppe\footnote{En alemán significa \emph{grupo de cuatro}.}.
\\
\bigskip
\pause
Como confirmación de esto, observa que aun cuando el vierergruppe es abeliano, no es cíclico.
\end{frame}

\subsection{Representaciones de matriz}

\begin{frame}
\frametitle{Representaciones de matriz}
La representación de los elementos de grupo por medio de matrices es una técnica bastante poderosa y ha sido adoptada casi universalmente en la física.
\\
\bigskip
\pause
El uso de las matrices no impone ninguna limitación considerable.
\end{frame}
\begin{frame}
\frametitle{Matrices unitarias}
Se puede demostrar que los elementos de cualquier \emph{grupo finito} y de los \emph{grupos continuos} pueden representarse por medio de matrices y, en particular, mediante matrices unitarias.
\end{frame}
\begin{frame}
\frametitle{Matrices unitarias}
En la mecánica cuántica, estas representaciones unitarias adquieren una importancia particular ya que las matrices unitarias se pueden diagonalizar, y los valores propios pueden servir para la clasificación de los estados cuánticos.
\end{frame}
\begin{frame}
\frametitle{Matrices en diagonal}
En caso de existir una transformación unitaria que transforme nuestras matrices de representación a la forma diagonal o de bloque-diagonal.
\end{frame}
\begin{frame}
\frametitle{Matrices en diagonal}
Por ejemplo:
\begin{eqnarray}
\mqty(
r_{11} & r_{12} & r_{13} & r_{14} \\
r_{21} & r_{22} & r_{23} & r_{24} \\
r_{31} & r_{32} & r_{33} & r_{34} \\
r_{41} & r_{42} & r_{43} & r_{44} ) \pause \hspace{0.2cm} \to \hspace{0.2cm} \pause \mqty(
\textcolor{red}{p_{11}} & \textcolor{red}{p_{12}} & 0 & 0 \\
\textcolor{red}{p_{21}} & \textcolor{red}{p_{22}} & 0 & 0 \\
0 & 0 & \textcolor{blue}{q_{11}} & \textcolor{blue}{q_{12}} \\
0 & 0 & \textcolor{blue}{q_{21}} & \textcolor{blue}{q_{22}} )
\label{eq:ecucion_04_186}
\end{eqnarray}
\end{frame}
\begin{frame}
\frametitle{Matriz reducible}
De modo que las porciones menores o submatrices ya no se encuentren
acopladas entre sí, \pause entonces la representación original es \textcolor{deepblue}{reducible}.
\end{frame}
\begin{frame}
\frametitle{Matriz reducible}
De manera equivalente, se tiene:
\begin{align}
\vb{S} \, \textcolor{ao}{\vb{R}} \, \vb{S}^{-1} = \mqty( 
\textcolor{blue}{\vb{P}} & \vb{0} \\
\vb{0} & \textcolor{red}{\vb{Q}} )
\label{eq:ecuacion_04_187}
\end{align}
\end{frame}
\begin{frame}
\frametitle{Tamaño de las matrices}
Si $\textcolor{ao}{\vb{R}}$ es una matriz de $n \times n$, \pause $\textcolor{blue}{\vb{P}}$ podría ser una matriz de $m \times m$, \pause $\textcolor{red}{\vb{Q}}$ una matriz de $(n-m) \times (n - m)$.
\\
\bigskip
\pause
Los términos $\textcolor{red}{\vb{Q}}$ son entonces matrices rectangulares $m \times (n - m)$ y $(n - m) \times m$ en que todos los elementos son cero.
\end{frame}
\begin{frame}
\frametitle{Matriz descompuesta en representaciones}
Este resultado se puede indicar en la forma:
\pause
\begin{align}
\textcolor{ao}{\vb{R}} = \textcolor{blue}{\vb{P}} \oplus \textcolor{red}{\vb{Q}}
\label{eq:ecuacion_04_188}
\end{align}
\pause
y decir que $\textcolor{ao}{\vb{R}}$ ha sido descompuesta en las representaciones $\textcolor{blue}{\vb{P}}$ y $\textcolor{red}{\vb{Q}}$.
\\
\bigskip
\pause
Por ejemplo, todas las representaciones de dimensión mayor que uno de los grupos abelianos son reducibles. 
\end{frame}
\begin{frame}
\frametitle{Representaciones irreducibles}
Las \textcolor{atomictangerine}{representaciones irreducibles} tienen una intervención en la teoría de grupos que, en términos generales, es análoga a los vectores unitarios del análisis vectorial.
\\
\bigskip
\pause
Constituyen las representaciones más simples; todas las demás representaciones pueden elaborarse a partir de las mismas.
\end{frame}
\begin{frame}
\frametitle{Matrices de rotación}
Se sabe que una matriz se transforma bajo la rotación de las coordenadas por medio de una transformación de similaridad ortogonal.
\\
\bigskip
\pause
Dependiendo de la selección de la estructura de referencia, prácticamente la misma matriz puede adquirir una infinidad de formas distintas.
\end{frame}
\begin{frame}
\frametitle{Transformaciones unitarias}
De modo similar, nuestras re­presentaciones de grupo pueden establecerse en una infinidad de formas distintas utilizando las transformaciones unitarias.
\\
\bigskip
\pause
Sin embargo, cada una de las representaciones transformadas es isomórfica, con respecto a la original.
\end{frame}
\begin{frame}
\frametitle{Transformaciones unitarias}
La traza de cada elemento (cada matriz de nuestra representación) es invariante bajo las transformaciones unitarias.
\\
\bigskip
\pause
Simplemente a causa de que es invariante, la traza (que ahora se ha denominado \textcolor{candypink}{carácter}) tiene una intervención de cierta importancia en la teoría de grupos, particularmente en las aplicaciones a la física del estado sólido.
\end{frame}
\begin{frame}
\frametitle{Subconjuntos de elementos}
Frecuentemente sucede que un subconjunto de los elementos de grupo (incluyendo el elemento unitario $\vb{1}$) satisface por si mismo los requisitos de grupo y consecuentemente constituye un grupo.
\\
\bigskip
\pause
Tal subconjunto se denomina \textcolor{cadmiumred}{subgrupo}. \pause Los elementos $1$ y $b$ del grupo de cuatro elementos considerado antes forman un subgrupo.
\end{frame}

\subsection{Subgrupos}

\begin{frame}
\frametitle{Subconjuntos de elementos}
En algunas ocasiones, la transformación de similaridad de cada miembro $x$ del subgrupo por todos los miembros $g$ del grupo entero es un miembro de un subgrupo:
\pause
\begin{align}
y = g \, x \, g^{-1}
\label{eq:ecuacion_04_189}
\end{align}
\end{frame}
\begin{frame}
\frametitle{Subgrupo invariante}
Tal subgrupo se denomina \textcolor{goldenrod}{subgrupo invariante} y está relacionado con los múltipletes de los espectros atómico y nuclear. 
\\
\bigskip
\pause
Todos los subgrupos de un grupo abeliano son automáticamente invariantes.
\end{frame}




\end{document}