\input{../Preambulos/preambulo_presentacion_Dresden_seahorse
}
\title{\large{Tema 4 - Separación de variables en coord. esféricas}}
\subtitle{Objetivos}
\author{M. en C. Gustavo Contreras Mayén}
\date{}
\institute{Facultad de Ciencias - UNAM}
\titlegraphic{\includegraphics[width=1.75cm]{../Imagenes/escudo-facultad-ciencias}\hspace*{4.75cm}~%
   \includegraphics[width=1.75cm]{../Imagenes/escudo-unam}
}
\setbeamertemplate{navigation symbols}{}
\begin{document}
\maketitle
\fontsize{14}{14}\selectfont
\spanishdecimal{.}
\section*{Contenido}
\frame[allowframebreaks]{\tableofcontents[currentsection, hideallsubsections]}
\section{El átomo de hidrógeno}
\frame{\tableofcontents[currentsection, hideothersubsections]}
\subsection{Ecuación de Schrödinger }
\begin{frame}
\frametitle{Potencial en el átomo de hidrógeno}
El potencial en el átomo de hidrógeno es el potencial de interacción de tipo Coulomb entre el núcleo y el electrón.
\end{frame}
\begin{frame}
\frametitle{Potencial en el átomo de hidrógeno}
Este es un potencial radial, es decir, depende solamente de la distancia al núcleo $(r)$:
\begin{align}
V = V(r) = - \dfrac{k \, Z \, e^{2}}{r}
\label{eq:ecuacion_01}
\end{align}
donde $Z$ el número atómico (en este caso $Z=1$), $e$ es la carga del electrón y $k$ es la constante de Couloumb.
\end{frame} 
\begin{frame}
\frametitle{El Hamiltoniano cuántico}
Por lo tanto el Hamiltoniano cuántico (el operador correspondiente a la energía total de sistema) se escribe como:
\begin{align}
H = - \dfrac{\hbar^{2}}{2 \, m} \, \laplacian + V(r)
\label{eq:ecuacion_02} 
\end{align}
\end{frame}
\begin{frame}
\frametitle{Sistema de coordenadas conveniente}
El sistema de coordenadas esféricas es el más adecuado para el problema: la ecuación de Schrödinger va a ser más fácil de resolver en este sistema.
\\
\bigskip
\pause
Como ya sabemos expresar el Laplaciano en este sistema, haremos uso de esa expresión:
\fontsize{12}{12}\selectfont
\begin{align*}
\laplacian = \dfrac{1}{r^{2}} \pdv{r} \left( r^{2} \pdv{\phi}{r} \right) {+} \dfrac{1}{r^{2} \sin \theta} \pdv{\theta} \left( \sin \theta \pdv{\phi}{\theta} \right) {+} \dfrac{1}{r^{2} \sin^{2} \theta} \pdv[2]{\phi}{\phi} 
\end{align*}
\end{frame}
\begin{frame}
\frametitle{Adecuando la expresión}
La expresión para el Laplaciano es complicada así que buscaremos una expresión más adecuada para resolver la ecuación de Schrödinger más fácilmente.
\end{frame}
\subsection{Momento angular}
\begin{frame}
\frametitle{El momento angular}
La teoría del momento angular en mecánica cuántica es de gran importancia tanto por el número como por la variedad de sus consecuencias.
\\
\bigskip
\pause
A partir de la espectroscopía rotacional, que depende del momento angular de las moléculas, se consigue información acerca de las dimensiones y formas de moléculas.
\end{frame}
\begin{frame}
\frametitle{El momento angular}
Utilizando los espectros de resonancia magnética nuclear y de resonancia paramagnética electrónica, cuyo origen es el momento angular de espín de núcleos y electrones, se consigue información sobre la estructura y configuración de moléculas.
\end{frame}
\begin{frame}
\frametitle{El momento angular}
El momento angular orbital de los electrones en los átomos define las forma de los orbitales atómicos los cuales, a su vez, determinan la orientación de los enlaces y la estereoquímica de las moléculas.
\end{frame}
\begin{frame}
\frametitle{Relevancia del momento angular}
El momento angular de un sistema es muy importante, cuando \emph{es una constante de movimiento}, es decir, cuando se conserva, porque en este caso sirve para clasificar los niveles de energía del sistema.
\end{frame}
\begin{frame}
\frametitle{Operadores de momento angular}
En mecánica cuántica los operadores de momento angular orbital son:
\begin{eqnarray}
\begin{aligned}
\hat{L}_{x} &=& - i \, \hbar \, \left( y \, \pdv{z} - z \, \pdv{y} \right) \\[0.5em] \pause
\hat{L}_{y} &=& - i \, \hbar \, \left( z \, \pdv{x} - x \, \pdv{z} \right) \\[0.5em] \pause
\hat{L}_{z} &=& - i \, \hbar \, \left( x \, \pdv{y} - y \, \pdv{x} \right)
\end{aligned}
\label{eq:ecuacion_01_03a}
\end{eqnarray}
\end{frame}
\begin{frame}
\frametitle{Cuadrado del operador momento angular}
El cuadrado del operador momento angular es tal que:
\begin{align}
\hat{L}^{2} = \hat{L} \cdot \hat{L} = \hat{L}_{x}^{2} + \hat{L}_{y}^{2} + \hat{L}_{z}^{2}
\label{eq:ecuacion_01_03b}
\end{align}
\end{frame}
\begin{frame}
\frametitle{Cambio de coordenadas}
Para poder aplicar estos operadores sobre funciones del tipo $\psi(r, \theta, \phi)$ es necesario expresarlos en coordenadas polares.
\\
\bigskip
\pause
Utilizando las relaciones:
\begin{align*}
r^{2} &= x^{2} + y^{2} +z^{2} \\
\cos \theta &= \dfrac{z}{\sqrt{x^{2} + y^{2} +z^{2}}} \\
\tan \phi &= \dfrac{y}{x}
\end{align*}
\end{frame}
\begin{frame}
\frametitle{Tomando las derivadas parciales}
Aplicando las derivadas parciales $\pdv*{x}$, $\pdv*{y}$ y $\pdv*{z}$, se tiene:
\begin{eqnarray}
\begin{aligned}
\hat{L}_{x} &= + i\, \hbar \, \left( \sin \phi \,\pdv{\theta} + \cot \theta\, \cos \phi \, \pdv{\phi} \right) \\[0.5em] \pause
\hat{L}_{y} &= - i\, \hbar \, \left( \cos \phi \,\pdv{\theta} - \cot \theta\, \sin \phi \, \pdv{\phi} \right) \\[0.5em] \pause
\hat{L}_{z} &= - i\, \hbar \, \pdv{\phi}
\end{aligned}
\label{eq:ecuacion_01_04a}
\end{eqnarray}
\end{frame}
\begin{frame}
\frametitle{Cuadrado del operador momento angular}
El cuadrado del operador momento angular es:
\begin{align}
\hat{L}^{2} = - \hbar^{2} \left( \dfrac{1}{\sin \theta} \pdv{\theta} \, \sin \theta \, \pdv{\theta} + \dfrac{1}{\sin^{2} \theta} \, \pdv[2]{\phi} \right)
\label{eq:ecuacion_01_04b}
\end{align}
\pause
Es importante notar que solo se utiliza el operador $\hat{L}^{2}$ o sus componentes, pero nunca el operador $\hat{L}$ directamente, ya que el momento angular es un vector $\va{L}$ y no un escalar.
\end{frame}
\subsection{Constante de movimiento}
\begin{frame}
\frametitle{Constante de movimiento}
La condición para que el operador $\hat{O}$ represente una \emph{constante de movimiento} de un sistema es que se cumpla la relación:
\begin{align}
\hat{O} \, \hat{H} = \hat{H} \, \hat{O}
\label{eq:ecuacion_01_05}
\end{align}
\pause
donde $\hat{H}$ es el Hamiltoniano del sistema.
\end{frame}
\begin{frame}
\frametitle{Constante de movimiento}
La relación anterior implica que el conmutador:
\begin{align}
[\hat{O}, \hat{H}] = \hat{O} \hat{H} - \hat{H} \, \hat{O}
\label{eq:ecuacion_01_06}
\end{align}
vale cero.
\\
\bigskip
\pause
En efecto, cuando dos operadores conmutan, existe un conjunto de funciones que son funciones propias de los dos operadores simultáneamente,
\end{frame}
\begin{frame}
\frametitle{Funciones propias}
Es decir, que la misma función $\psi$ que caracteriza el estado del sistema con energía $E$:
\begin{align*}
\hat{H} \, \psi = E \, \psi
\end{align*}
\pause
también caracteriza el estado del sistema con propiedad $\hat{O}$ igual a $o$:
\begin{align*}
\hat{0} \, \psi = o \, \psi
\end{align*}
\end{frame}
\begin{frame}
\frametitle{Constante de movimiento}
Dicho de otra manera, cuando el sistema se encuentra en el estado caracterizado por $\psi$, su energía es $E$ y su propiedad $\hat{O}$ es $o$.
\\
\bigskip
\pause
Ambos valores $E$ y $o$ son constantes mientras el sistema permanezca en el mismo estado $\psi$.
\end{frame}
\begin{frame}
\frametitle{Caso degenerado}
En los casos en los que $\psi$ sea degenerada, siempre será posible construir una combinación lineal de autofunciones correspondientes a $E$ tal que sea también autofunción de $\hat{O}$.
\end{frame}
\subsection{Reglas de conmutación}
\begin{frame}
\frametitle{Reglas de conmutación}
Las reglas de conmutación entre los operadores de momento angular y sus componentes pueden ser deducidas fácilmente utilizando las expresiones en coordenadas cartesianas y algunas identidades de los conmutadores como:
\begin{align*}
[ \hat{A} + \hat{B}, \hat{C}] &= [\hat{A}, \hat{C}] + [\hat{B} + \hat{C}] \\[0.5em]
[ \hat{A}^{2} , \hat{B}] &= [\hat{A}, \hat{B}] \, \hat{A} +  \hat{A} \, [\hat{A} , \hat{B}]
\end{align*}
\end{frame}
\begin{frame}
\frametitle{Conmutación momento angular}
Se cumple entonces que:
\begin{align}
\begin{aligned}
[ \hat{L}_{x}, \hat{L}_{y} ] &= i \, \hbar \, \hat{L}_{z} \\[0.5em]
[ \hat{L}_{y}, \hat{L}_{z} ] &= i \, \hbar \, \hat{L}_{x} \\[0.5em]
[ \hat{L}_{z}, \hat{L}_{x} ] &= i \, \hbar \, \hat{L}_{y} \\[0.5em]
[\hat{L}^{2}, \hat{L}_{x}] = [\hat{L}^{2}&, \hat{L}_{y}] = [\hat{L}^{2}, \hat{L}_{z}] = 0
\end{aligned}
\label{eq:ecuacion_01_07}
\end{align}
\fontsize{12}{12}\selectfont
Entonces: $\hat{L}^{2}$ conmuta con cualquiera de sus componentes, pero las componentes no conmutan entre sí.
\end{frame}
\begin{frame}
\frametitle{Conmutación momento angular y Hamiltoniano}
Las propiedades de conmutación entre los operadores de momento angular orbital y el Hamiltoniano dependen del sistema y deben ser determinadas para cada problema.
\end{frame}
\begin{frame}
\frametitle{Conmutación momento angular y Hamiltoniano}
Frecuentemente $\hat{L}^{2}$ y $\hat{L}_{z}$ conmutan con $H$ y en estos casos el móulo del momento angular y la componente sobre el eje $z$ del momento angular son constantes de movimiento. 
\end{frame}
\begin{frame}
\frametitle{Conmutación momento angular y Hamiltoniano}
Frecuentemente $\hat{L}^{2}$ y $\hat{L}_{z}$ conmutan con $\hat{H}$ y en estos casos el módulo del momento angular y la componente sobre el eje $z$ del momento angular son constantes de movimiento.
\end{frame}
\begin{frame}
\frametitle{Conmutación de $\hat{H}$ y $\hat{L}_{z}$}
Por ejemplo, en el caso de átomos hidrogenoides $\hat{H}$ y $\hat{L}_{z}$ conmutan, donde
\fontsize{12}{12}\selectfont
\begin{align*}
\hat{H} &= - \dfrac{\hbar^{2}}{2 \mu} \left[ \dfrac{1}{r^{2}} \pdv{r} \left( r^{2} \pdv{\phi}{r} \right) {+} \dfrac{1}{r^{2} \sin \theta} \pdv{\theta} \left( \sin \theta \pdv{\phi}{\theta} \right) {+} \right. \\[0.5em]
&+ \left. \dfrac{1}{r^{2} \sin^{2} \theta} \pdv[2]{\phi}{\phi} \right] - \dfrac{Z \, e^{2}}{r} \\[1em]
\hat{L}_{z} &= - i \, \hbar \, \pdv{\phi}
\end{align*}
\end{frame}
\begin{frame}
\frametitle{Conmutación de $\hat{H}$ y $\hat{L}_{z}$}
Entonces:
\fontsize{12}{12}\selectfont
\begin{align*}
\hat{H} \cdot \hat{L}_{z} &= + \dfrac{i \, \hbar^{3}}{2 \mu} \left\{ \left[ \dfrac{1}{r^{2}} \pdv{r} \left( r^{2} \pdv{\phi}{r} \right) {+} \dfrac{1}{r^{2} \sin \theta} \pdv{\theta} \left( \sin \theta \pdv{\phi}{\theta} \right) {+} \right. \right. \\[0.5em]
&+ \left. \left. \dfrac{1}{r^{2} \sin^{2} \theta} \pdv[2]{\phi}{\phi} + \dfrac{Z \, e^{2}}{r} \right] \pdv{\phi} + \dfrac{1}{r^{2} \sin^{2} \theta} \pdv[3]{\phi}{\phi} \right\}
\end{align*}
\end{frame}
\begin{frame}
\frametitle{Conmutación de $\hat{H}$ y $\hat{L}_{z}$}
Mientras que:
\fontsize{12}{12}\selectfont
\begin{align*}
\hat{L}_{z} \cdot \hat{H}  &= + \dfrac{i \, \hbar^{3}}{2 \mu} \left\{ \pdv{\phi} \left[ \dfrac{1}{r^{2}} \pdv{r} \left( r^{2} \pdv{\phi}{r} \right) {+} \right. \right. \\[0.5em]
&+ \dfrac{1}{r^{2} \sin \theta} \pdv{\theta} \left( \sin \theta \pdv{\phi}{\theta} \right) {+} \\[0.5em]
&+ \left. \left. \dfrac{1}{r^{2} \sin^{2} \theta} \pdv[2]{\phi}{\phi} + \dfrac{Z \, e^{2}}{r} \right] + \dfrac{1}{r^{2} \sin^{2} \theta} \pdv[3]{\phi}{\phi} \right\}
\end{align*}
\end{frame}
\begin{frame}
\frametitle{Conmutación de $\hat{H}$ y $\hat{L}_{z}$}
Sabiendo que:
\begin{align*}
\pdv{\phi} \, \pdv{r} = \pdv{r} \, \pdv{\phi} \\[0.5em]
\pdv{\phi} \, \pdv{\theta} = \pdv{\theta} \, \pdv{\phi}
\end{align*}
Es decir, las dos expresiones son iguales, por lo que:
\end{frame}
\begin{frame}
\frametitle{Conmutación de $\hat{H}$ y $\hat{L}_{z}$}
Por lo que:
\begin{align*}
[\hat{H}, \hat{L}_{z}] = \hat{H} \, \hat{L}_{z} - \hat{L}_{z} \, \hat{H} = 0
\end{align*}
\end{frame}
\begin{frame}
\frametitle{El uso del operador momento angular}
En coordenadas cartesianas $\hat{L}^{2}$ depende de tres coordenadas $(x, y, z)$; en coordenadas esféricas, $\hat{L}^{2}$ depende solo de dos $(\theta, \phi)$.
\\
\bigskip
\pause
En coordenadas cartesianas una de las variables no es independiente; en coordenadas esféricas, $\hat{L}^{2}$ solo depende de los ángulos, y no de la distancia $r$.
\end{frame}
\begin{frame}
\frametitle{El uso del operador momento angular}
Los observables correspondientes a los operadores $\hat{L}_{x}$, $\hat{L}_{y}$ y $\hat{L}_{z}$, son totalmente equivalentes, lo único que cambia es su orientación con respecto al sistema de referencia.
\\
\bigskip
\pause
Por esta razón siempre se usa $\hat{L}_{z}$, ya que la expresión matemática de su operador es mucho más simple, depende de solo un ángulos.
\end{frame}
\end{document}