\input{../Preambulos/preambulo_presentacion_Dresden_seahorse
}
\title{\large{Tema 4 - Separación de variables en coord. esféricas}}
\subtitle{Objetivos}
\author{M. en C. Gustavo Contreras Mayén}
\date{}
\institute{Facultad de Ciencias - UNAM}
\titlegraphic{\includegraphics[width=1.75cm]{../Imagenes/escudo-facultad-ciencias}\hspace*{4.75cm}~%
   \includegraphics[width=1.75cm]{../Imagenes/escudo-unam}
}
\setbeamertemplate{navigation symbols}{}
\begin{document}
\maketitle
\fontsize{14}{14}\selectfont
\spanishdecimal{.}
\section*{Contenido}
\frame[allowframebreaks]{\tableofcontents[currentsection, hideallsubsections]}
\section{El átomo de hidrógeno}
\frame{\tableofcontents[currentsection, hideothersubsections]}
\subsection{Ecuación de Schrödinger }
\begin{frame}
\frametitle{Potencial en el átomo de hidrógeno}
El potencial en el átomo de hidrógeno es el potencial de interacción de tipo Coulomb entre el núcleo y el electrón.
\end{frame}
\begin{frame}
\frametitle{Potencial en el átomo de hidrógeno}
Este es un potencial radial, es decir, depende solamente de la distancia al núcleo $(r)$:
\begin{align}
V = V(r) = - \dfrac{k \, Z \, e^{2}}{r}
\label{eq:ecuacion_01}
\end{align}
donde $Z$ el número atómico (en este caso $Z=1$), $e$ es la carga del electrón y $k$ es la constante de Couloumb.
\end{frame} 
\begin{frame}
\frametitle{El Hamiltoniano cuántico}
Por lo tanto el Hamiltoniano cuántico (el operador correspondiente a la energía total de sistema) se escribe como:
\begin{align}
H = - \dfrac{\hbar^{2}}{2 \, m} \, \laplacian + V(r)
\label{eq:ecuacion_02} 
\end{align}
\end{frame}
\begin{frame}
\frametitle{Sistema de coordenadas conveniente}
El sistema de coordenadas esféricas es el más adecuado para el problema: la ecuación de Schrödinger va a ser más fácil de resolver en este sistema.
\\
\bigskip
\pause
Como ya sabemos expresar el Laplaciano en este sistema, haremos uso de esa expresión:
\fontsize{12}{12}\selectfont
\begin{align*}
\laplacian = \dfrac{1}{r^{2}} \pdv{r} \left( r^{2} \pdv{\phi}{r} \right) {+} \dfrac{1}{r^{2} \sin \theta} \pdv{\theta} \left( \sin \theta \pdv{\phi}{\theta} \right) {+} \dfrac{1}{r^{2} \sin^{2} \theta} \pdv[2]{\phi}{\phi} 
\end{align*}
\end{frame}
\begin{frame}
\frametitle{Adecuando la expresión}
La expresión para el Laplaciano es complicada así que buscaremos una expresión más adecuada para resolver la ecuación de Schrödinger más fácilmente.
\end{frame}
\subsection{Momento angular}
\begin{frame}
\frametitle{El momento angular}
La teoría del momento angular en mecánica cuántica es de gran importancia tanto por el número como por la variedad de sus consecuencias.
\\
\bigskip
\pause
A partir de la espectroscopía rotacional, que depende del momento angular de las moléculas, se consigue información acerca de las dimensiones y formas de moléculas.
\end{frame}
\begin{frame}
\frametitle{El momento angular}
Utilizando los espectros de resonancia magnética nuclear y de resonancia paramagnética electrónica, cuyo origen es el momento angular de espín de núcleos y electrones, se consigue información sobre la estructura y configuración de moléculas.
\end{frame}
\begin{frame}
\frametitle{El momento angular}
El momento angular orbital de los electrones en los átomos define las forma de los orbitales atómicos los cuales, a su vez, determinan la orientación de los enlaces y la estereoquímica de las moléculas.
\end{frame}
\begin{frame}
\frametitle{Relevancia del momento angular}
El momento angular de un sistema es muy importante, cuando \emph{es una constante de movimiento}, es decir, cuando se conserva, porque en este caso sirve para clasificar los niveles de energía del sistema.
\end{frame}
\begin{frame}
\frametitle{Operadores de momento angular}
En mecánica cuántica los operadores de momento angular orbital son:
\begin{eqnarray}
\begin{aligned}
\hat{L}_{x} &=& - i \, \hbar \, \left( y \, \pdv{z} - z \, \pdv{y} \right) \\[0.5em] \pause
\hat{L}_{y} &=& - i \, \hbar \, \left( z \, \pdv{x} - x \, \pdv{z} \right) \\[0.5em] \pause
\hat{L}_{z} &=& - i \, \hbar \, \left( x \, \pdv{y} - y \, \pdv{x} \right)
\end{aligned}
\label{eq:ecuacion_01_03a}
\end{eqnarray}
\end{frame}
\begin{frame}
\frametitle{Cuadrado del operador momento angular}
El cuadrado del operador momento angular es tal que:
\begin{align}
\hat{L}^{2} = \hat{L} \cdot \hat{L} = \hat{L}_{x}^{2} + \hat{L}_{y}^{2} + \hat{L}_{z}^{2}
\label{eq:ecuacion_01_03b}
\end{align}
\end{frame}
\begin{frame}
\frametitle{Cambio de coordenadas}
Para poder aplicar estos operadores sobre funciones del tipo $\psi(r, \theta, \phi)$ es necesario expresarlos en coordenadas polares.
\\
\bigskip
\pause
Utilizando las relaciones:
\begin{align*}
r^{2} &= x^{2} + y^{2} +z^{2} \\
\cos \theta &= \dfrac{z}{\sqrt{x^{2} + y^{2} +z^{2}}} \\
\tan \phi &= \dfrac{y}{x}
\end{align*}
\end{frame}
\begin{frame}
\frametitle{Tomando las derivadas parciales}
Aplicando las derivadas parciales $\pdv*{x}$, $\pdv*{y}$ y $\pdv*{z}$, se tiene:
\begin{eqnarray}
\begin{aligned}
\hat{L}_{x} &= + i\, \hbar \, \left( \sin \phi \,\pdv{\theta} + \cot \theta\, \cos \phi \, \pdv{\phi} \right) \\[0.5em] \pause
\hat{L}_{y} &= - i\, \hbar \, \left( \cos \phi \,\pdv{\theta} - \cot \theta\, \sin \phi \, \pdv{\phi} \right) \\[0.5em] \pause
\hat{L}_{z} &= - i\, \hbar \, \pdv{\phi}
\end{aligned}
\label{eq:ecuacion_01_04a}
\end{eqnarray}
\end{frame}
\begin{frame}
\frametitle{Cuadrado del operador momento angular}
El cuadrado del operador momento angular es:
\begin{align}
\hat{L}^{2} = - \hbar^{2} \left( \dfrac{1}{\sin \theta} \pdv{\theta} \, \sin \theta \, \pdv{\theta} + \dfrac{1}{\sin^{2} \theta} \, \pdv[2]{\phi} \right)
\label{eq:ecuacion_01_04b}
\end{align}
\pause
Es importante notar que solo se utiliza el operador $\hat{L}^{2}$ o sus componentes, pero nunca el operador $\hat{L}$ directamente, ya que el momento angular es un vector $\va{L}$ y no un escalar.
\end{frame}
\subsection{Constante de movimiento}
\begin{frame}
\frametitle{Constante de movimiento}
La condición para que el operador $\hat{O}$ represente una \emph{constante de movimiento} de un sistema es que se cumpla la relación:
\begin{align}
\hat{O} \, \hat{H} = \hat{H} \, \hat{O}
\label{eq:ecuacion_01_05}
\end{align}
\pause
donde $\hat{H}$ es el Hamiltoniano del sistema.
\end{frame}
\begin{frame}
\frametitle{Constante de movimiento}
La relación anterior implica que el conmutador:
\begin{align}
[\hat{O}, \hat{H}] = \hat{O} \hat{H} - \hat{H} \, \hat{O}
\label{eq:ecuacion_01_06}
\end{align}
vale cero.
\\
\bigskip
\pause
En efecto, cuando dos operadores conmutan, existe un conjunto de funciones que son funciones propias de los dos operadores simultáneamente,
\end{frame}
\begin{frame}
\frametitle{Funciones propias}
Es decir, que la misma función $\psi$ que caracteriza el estado del sistema con energía $E$:
\begin{align*}
\hat{H} \, \psi = E \, \psi
\end{align*}
\pause
también caracteriza el estado del sistema con propiedad $\hat{O}$ igual a $o$:
\begin{align*}
\hat{0} \, \psi = o \, \psi
\end{align*}
\end{frame}
\begin{frame}
\frametitle{Constante de movimiento}
Dicho de otra manera, cuando el sistema se encuentra en el estado caracterizado por $\psi$, su energía es $E$ y su propiedad $\hat{O}$ es $o$.
\\
\bigskip
\pause
Ambos valores $E$ y $o$ son constantes mientras el sistema permanezca en el mismo estado $\psi$.
\end{frame}
\begin{frame}
\frametitle{Caso degenerado}
En los casos en los que $\psi$ sea degenerada, siempre será posible construir una combinación lineal de autofunciones correspondientes a $E$ tal que sea también autofunción de $\hat{O}$.
\end{frame}
\subsection{Reglas de conmutación}
\begin{frame}
\frametitle{Reglas de conmutación}
Las reglas de conmutación entre los operadores de momento angular y sus componentes pueden ser deducidas fácilmente utilizando las expresiones en coordenadas cartesianas y algunas identidades de los conmutadores como:
\begin{align*}
[ \hat{A} + \hat{B}, \hat{C}] &= [\hat{A}, \hat{C}] + [\hat{B} + \hat{C}] \\[0.5em]
[ \hat{A}^{2} , \hat{B}] &= [\hat{A}, \hat{B}] \, \hat{A} +  \hat{A} \, [\hat{A} , \hat{B}]
\end{align*}
\end{frame}
\begin{frame}
\frametitle{Conmutación momento angular}
Se cumple entonces que:
\begin{align}
\begin{aligned}
[ \hat{L}_{x}, \hat{L}_{y} ] &= i \, \hbar \, \hat{L}_{z} \\[0.5em]
[ \hat{L}_{y}, \hat{L}_{z} ] &= i \, \hbar \, \hat{L}_{x} \\[0.5em]
[ \hat{L}_{z}, \hat{L}_{x} ] &= i \, \hbar \, \hat{L}_{y} \\[0.5em]
[\hat{L}^{2}, \hat{L}_{x}] = [\hat{L}^{2}&, \hat{L}_{y}] = [\hat{L}^{2}, \hat{L}_{z}] = 0
\end{aligned}
\label{eq:ecuacion_01_07}
\end{align}
\fontsize{12}{12}\selectfont
Entonces: $\hat{L}^{2}$ conmuta con cualquiera de sus componentes, pero las componentes no conmutan entre sí.
\end{frame}
\begin{frame}
\frametitle{Conmutación momento angular y Hamiltoniano}
Las propiedades de conmutación entre los operadores de momento angular orbital y el Hamiltoniano dependen del sistema y deben ser determinadas para cada problema.
\end{frame}
\begin{frame}
\frametitle{Conmutación momento angular y Hamiltoniano}
Frecuentemente $\hat{L}^{2}$ y $\hat{L}_{z}$ conmutan con $\hat{H}$ y en estos casos el módulo del momento angular y la componente sobre el eje $z$ del momento angular son constantes de movimiento. 
\end{frame}
\begin{frame}
\frametitle{Conmutación momento angular y Hamiltoniano}
Frecuentemente $\hat{L}^{2}$ y $\hat{L}_{z}$ conmutan con $\hat{H}$ y en estos casos el módulo del momento angular y la componente sobre el eje $z$ del momento angular son constantes de movimiento.
\end{frame}
\begin{frame}
\frametitle{Conmutación de $\hat{H}$ y $\hat{L}_{z}$}
Por ejemplo, en el caso de átomos hidrogenoides $\hat{H}$ y $\hat{L}_{z}$ conmutan, donde
\fontsize{12}{12}\selectfont
\begin{align*}
\hat{H} &= - \dfrac{\hbar^{2}}{2 \mu} \left[ \dfrac{1}{r^{2}} \pdv{r} \left( r^{2} \pdv{\phi}{r} \right) {+} \dfrac{1}{r^{2} \sin \theta} \pdv{\theta} \left( \sin \theta \pdv{\phi}{\theta} \right) {+} \right. \\[0.5em]
&+ \left. \dfrac{1}{r^{2} \sin^{2} \theta} \pdv[2]{\phi}{\phi} \right] - \dfrac{Z \, e^{2}}{r} \\[1em]
\hat{L}_{z} &= - i \, \hbar \, \pdv{\phi}
\end{align*}
\end{frame}
\begin{frame}
\frametitle{Conmutación de $\hat{H}$ y $\hat{L}_{z}$}
Entonces:
\fontsize{12}{12}\selectfont
\begin{align*}
\hat{H} \cdot \hat{L}_{z} &= + \dfrac{i \, \hbar^{3}}{2 \mu} \left\{ \left[ \dfrac{1}{r^{2}} \pdv{r} \left( r^{2} \pdv{\phi}{r} \right) {+} \dfrac{1}{r^{2} \sin \theta} \pdv{\theta} \left( \sin \theta \pdv{\phi}{\theta} \right) {+} \right. \right. \\[0.5em]
&+ \left. \left. \dfrac{1}{r^{2} \sin^{2} \theta} \pdv[2]{\phi}{\phi} + \dfrac{Z \, e^{2}}{r} \right] \pdv{\phi} + \dfrac{1}{r^{2} \sin^{2} \theta} \pdv[3]{\phi}{\phi} \right\}
\end{align*}
\end{frame}
\begin{frame}
\frametitle{Conmutación de $\hat{H}$ y $\hat{L}_{z}$}
Mientras que:
\fontsize{12}{12}\selectfont
\begin{align*}
\hat{L}_{z} \cdot \hat{H}  &= + \dfrac{i \, \hbar^{3}}{2 \mu} \left\{ \pdv{\phi} \left[ \dfrac{1}{r^{2}} \pdv{r} \left( r^{2} \pdv{\phi}{r} \right) {+} \right. \right. \\[0.5em]
&+ \dfrac{1}{r^{2} \sin \theta} \pdv{\theta} \left( \sin \theta \pdv{\phi}{\theta} \right) {+} \\[0.5em]
&+ \left. \left. \dfrac{1}{r^{2} \sin^{2} \theta} \pdv[2]{\phi}{\phi} + \dfrac{Z \, e^{2}}{r} \right] + \dfrac{1}{r^{2} \sin^{2} \theta} \pdv[3]{\phi}{\phi} \right\}
\end{align*}
\end{frame}
\begin{frame}
\frametitle{Conmutación de $\hat{H}$ y $\hat{L}_{z}$}
Sabiendo que:
\begin{align*}
\pdv{\phi} \, \pdv{r} = \pdv{r} \, \pdv{\phi} \\[0.5em]
\pdv{\phi} \, \pdv{\theta} = \pdv{\theta} \, \pdv{\phi}
\end{align*}
Es decir, las dos expresiones son iguales, por lo que:
\end{frame}
\begin{frame}
\frametitle{Conmutación de $\hat{H}$ y $\hat{L}_{z}$}
Por lo que:
\begin{align*}
[\hat{H}, \hat{L}_{z}] = \hat{H} \, \hat{L}_{z} - \hat{L}_{z} \, \hat{H} = 0
\end{align*}
\end{frame}
\begin{frame}
\frametitle{El uso del operador momento angular}
En coordenadas cartesianas $\hat{L}^{2}$ depende de tres coordenadas $(x, y, z)$; en coordenadas esféricas, $\hat{L}^{2}$ depende solo de dos $(\theta, \phi)$.
\\
\bigskip
\pause
En coordenadas cartesianas una de las variables no es independiente; en coordenadas esféricas, $\hat{L}^{2}$ solo depende de los ángulos, y no de la distancia $r$.
\end{frame}
\begin{frame}
\frametitle{El uso del operador momento angular}
Los observables correspondientes a los operadores $\hat{L}_{x}$, $\hat{L}_{y}$ y $\hat{L}_{z}$, son totalmente equivalentes, lo único que cambia es su orientación con respecto al sistema de referencia.
\\
\bigskip
\pause
Por esta razón siempre se usa $\hat{L}_{z}$, ya que la expresión matemática de su operador es mucho más simple, depende de solo un ángulos.
\end{frame}
\begin{frame}
\frametitle{Resultado de la teoría de operadores}
Nos apoyaremos en un resultado de la teoría de los operadores y conmutadores: : Si $\hat{A}$ y $\hat{B}$ conmutan, es decir, si  $[\hat{A}, \hat{B}] = 0$, entonces existe una solución común $\psi$  para el par de ecuaciones diferenciales correspondientes a las ecuaciones de valores propios de estos operadores, siendo $\psi$ la función propia mientras que $a$ y $b$ son los valores propios correspondientes:
\end{frame}
\begin{frame}
\frametitle{Resultado de la teoría de operadores}
Tal que:
\begin{align*}
\hat{A} \, \psi &= a \, \psi \\[0.5em]
\hat{B} \, \psi &= b \, \psi
\end{align*}
\end{frame}
\begin{frame}
\frametitle{Ocupando este resultado}
Ahora bien, utilizando ese resultado y el hecho de que $[\hat{L}^{2}, \hat{L}_{z}] = 0$ podemos buscar una solución común, que escribimos como $Y(\theta, \phi)$, al par de las ecuaciones diferenciales:
\begin{align*}
\hat{L}_{z} \, Y(\theta, \phi) &= b \, Y(\theta, \phi) \\[0.5em]
\hat{L}^{2} \, Y(\theta, \phi) &= c \, Y(\theta, \phi)
\end{align*}
\end{frame}
%Ref. Ghatak (2004) 9.3
\subsection{Problema de valores propios}
\begin{frame}
\frametitle{Planteando el problema}
Sin pérdida de generalidad, podemos expresar nuestro problema de valores propios para $\hat{L}^{2}$ como:
\begin{align}
\hat{L}^{2} \, Y(\theta, \phi) = \lambda \, \hbar^{2} \, Y(\theta, \phi)
\label{eq:ecuacion_027}
\end{align}
donde $\lambda \, \hbar^{2}$ representan los valores propios de $\hat{L}^{2}$, y $Y(\theta, \phi)$ corresponde a las funciones propias. 
\end{frame}
\begin{frame}
\frametitle{Hacia donde nos dirigimos}
Veremos que $\lambda$ toma valores $\ell (\ell + 1)$ con $\ell = 0, 1, 2, \ldots$ y las correspondientes funciones propias son los \emph{armónicos esféricos}.
\end{frame}
\begin{frame}
\frametitle{Casos degenerados}
Para cada valor de $\ell$, habrá un orden $(2 \, \ell + 1)$ de degeneración, es decir, habrá $(2 \, \ell + 1)$ funciones propias que corresponden al mismo valor propio $\ell (\ell + 1) \, \hbar^{2}$.
\end{frame}
\begin{frame}
\frametitle{Resolviendo la ED}
El operador $\hat{L}^{2}$ de la ec. (\ref{eq:ecuacion_01_04b}) lo sustituimos en la ec. (\ref{eq:ecuacion_027}), así que:
{\fontsize{12}{12}\selectfont
\begin{align}
\dfrac{1}{\sin \theta} \pdv{\theta} \, \sin \theta \, \pdv{Y}{\theta} + \dfrac{1}{\sin^{2} \theta} \, \pdv[2]{Y}{\phi} + \lambda \, Y(\theta, \phi) = 0
\end{align}}
\\
\bigskip
\pause
Para resolver esta ecuación, usamos la técnica de separación de variables.
\end{frame}
\begin{frame}
\frametitle{Usando separación de variables}
Proponemos una solución de la forma:
\begin{align}
Y(\theta, \phi) = \Theta(\theta) \, \Phi(\phi)
\label{eq:ecuacion_029}
\end{align}
\\
\bigskip
\pause
Que susituimos en la expresión anterior, para luego multiplicar por
\begin{align*}
\dfrac{\sin^{2} \theta}{Y(\theta, \phi)}
\end{align*}
\end{frame}
\begin{frame}
\frametitle{Usando separación de variables}
Entonces obtendremos:
{\fontsize{12}{12}\selectfont
\begin{align}
\dfrac{\sin^{2} \theta}{\Theta} \left[ \dfrac{1}{\sin \theta} \pdv{\theta} \, \sin \theta \, \pdv{\Theta}{\theta} + \lambda \, \Theta (\theta) \right] = -  \dfrac{1}{\Phi} \, \dv[2]{\Phi}{\phi} = m^{2}
\label{eq:ecuacion_030}
\end{align}}
\end{frame}
\begin{frame}
\frametitle{Usando separación de variables}
De hecho, las variables se han separado y hemos establecido cada lado igual a una constante positiva $m^{2}$, cuya razón quedará clara en breve.
\end{frame}
\begin{frame}
\frametitle{Primera separación}
La ec. (\ref{eq:ecuacion_030}) nos da:
\begin{align*}
\dv[2]{\Phi}{\phi} + m^{2} \Phi (\phi) = 0
\end{align*}
\\
\bigskip
\pause
cuya solución está dada por:
\begin{align*}
\Phi(\phi) \sim e^{i m \phi}
\end{align*}
\end{frame}
\begin{frame}
\frametitle{Función univaluada}
Par que la función de onda sea univaluada, debe de ocurrir que:
\begin{align}
 \Phi(\phi +  2 \, \pi) = \Phi(\phi)
 \label{eq:ecuacion_031}
\end{align}
\\
\bigskip
\pause
o equivalentemente:
\begin{align*}
e^{2 \pi m i} = 1
\end{align*}
\end{frame}
\begin{frame}
\frametitle{Valor de la constante de separación}
Obteniendo entonces que:
\begin{align*}
m = 0, \pm 1, \pm 2, \ldots
\end{align*}
\pause
En este paso se justifica que no podríamos haber establecido una constante positiva (o compleja) porque entonces la función de onda no habría sido de un solo valor.
\end{frame}
\begin{frame}
\frametitle{Primera función obtenida}
Al identificar las funciones con un subíndice $m$, tenemos:
\begin{align}
\Phi_{m}(\phi) = \dfrac{1}{\sqrt{2 \, \pi}} \, e^{i m \phi} \hspace{1cm} m = \pm 1, \pm 2, \ldots
\label{eq:ecuacion_032}
\end{align}
\\
\bigskip
\pause
Donde el factor $\dfrac{1}{\sqrt{2 \, \pi}}$ asegura que:
\begin{align*}
\int_{0}^{2 \pi} \abs{\Phi_{m}(\phi)}^{2} \dd{\phi} = 1
\end{align*}
que es la condición de normalización.
\end{frame}
\begin{frame}
\frametitle{Condición de normalización}
Entonces se tendrá que:
\begin{align}
\int_{0}^{2 \pi} \Phi_{\ptilde{m}}^{*}(\phi) \, \Phi_{m}(\phi) \dd{\phi} = \delta_{m \ptilde{m}}
\label{eq:ecuacion_033}
\end{align}
representa la condición de ortonormalización para $\Phi_{m}(\phi)$.
\end{frame}
\begin{frame}
\frametitle{La segunda ecuación $\Theta(\theta)$}
Para la segunda ecuación $\Theta (\theta)$ (ec. \ref{eq:ecuacion_030}), tendremos que:
\fontsize{12}{12}\selectfont
\begin{align}
\dfrac{1}{\sin \theta} \dv{\theta} \left( \sin \theta \, \dv{\Theta}{\theta} \right) + \left( \lambda - \dfrac{m^{2}}{\sin^{2} \theta} \right) \, \Theta (\theta) = 0
\label{eq:ecuacion_034}
\end{align}
\end{frame}
\begin{frame}
\frametitle{Cambio de variable}
Hacemos el siguiente cambio de variable: $\cos \theta = \mu$ y $\Theta(\theta) = F(\mu)$, para obtener:
\pause
\fontsize{12}{12}\selectfont
\begin{align}
\dv{\mu} \left[ (1 - \mu^{2}) \, \dv{F}{\mu} \right] + \left[ \lambda - \dfrac{m^{2}}{1 - \mu^{2}} \right] \, F(\mu) = 0
\label{eq:ecuacion_035}
\end{align}
\\
\bigskip
\pause
Hay que considerar dos casos: $m = 0$ y $m \neq 0$
\end{frame}
\begin{frame}
\frametitle{Caso con $m = 0$}
Con $m = 0$, la ec. (\ref{eq:ecuacion_035}) se reduce a:
\begin{align}
(1 - \mu^{2}) \, \dv[2]{F}{\mu} - 2  \, \mu \, \dv{F}{\mu} + \lambda \, F(\mu) = 0
\label{eq:ecuacion_036}
\end{align}
\end{frame}
\begin{frame}
\frametitle{Caso con $m \neq 0$}
El método de \enquote{fuerza bruta} para obtener es resolver directamente la ec. (\ref{eq:ecuacion_035}); esto lo veremos posteriormente.
\\
\bigskip
\pause
Sin embargo, la forma más sencilla y elegante de obtener las soluciones es mediante el uso de operadores de escalera del momento angular, que también revisaremos esa solución.
\end{frame}
\begin{frame}
\frametitle{Ecuación diferencial obtenida}
Hemos llegado a plantear una ecuación diferencial para la parte angular, nos falta considerar la parte radial.
\\
\bigskip
Para la solución completa de cualquier problema de una partícula con un potencial radial, se tiene que la función de onda es un producto de un factor radial y un armónico esférico.
\end{frame}
\end{document}