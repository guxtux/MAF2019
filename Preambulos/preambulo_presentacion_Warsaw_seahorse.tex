\RequirePackage{currfile}
\documentclass[12pt]{beamer}
\usepackage[utf8]{inputenc}
\usepackage[spanish]{babel}
\usepackage{standalone}
\usepackage{color}
\usepackage{siunitx}
\usepackage{hyperref}
%\hypersetup{colorlinks,linkcolor=,urlcolor=blue}
%\hypersetup{colorlinks,urlcolor=blue}
\usepackage{xcolor,soul}
\usepackage{etoolbox}
\usepackage{amsmath}
\usepackage{amsthm}
\usepackage{physics}
\usepackage{multicol}
\usepackage{bookmark}
\usepackage{longtable}
\usepackage{listings}
\usepackage{graphicx}
\usepackage{tikz}
\usetikzlibrary{patterns, matrix, backgrounds, decorations,shapes, arrows.meta}
\usepackage[autostyle,spanish=mexican]{csquotes}
\usepackage[os=win]{menukeys}
\usepackage{pifont}
\usepackage{pbox}
\usepackage{caption}
\captionsetup{font=scriptsize,labelfont=scriptsize}
%\usepackage[sfdefault]{roboto}  %% Option 'sfdefault' only if the base font of the document is to be sans serif

%Sección de definición de colores
\definecolor{ao}{rgb}{0.0, 0.5, 0.0}
\definecolor{bisque}{rgb}{1.0, 0.89, 0.77}
\definecolor{amber}{rgb}{1.0, 0.75, 0.0}
\definecolor{armygreen}{rgb}{0.29, 0.33, 0.13}
\definecolor{alizarin}{rgb}{0.82, 0.1, 0.26}
\definecolor{cadetblue}{rgb}{0.37, 0.62, 0.63}
\definecolor{deepblue}{rgb}{0,0,0.5}
\definecolor{brown}{rgb}{0.59, 0.29, 0.0}
\definecolor{OliveGreen}{rgb}{0,0.25,0}


\usefonttheme[onlymath]{serif}
%Sección de definición de nuevos comandos

\newcommand*{\TitleParbox}[1]{\parbox[c]{1.75cm}{\raggedright #1}}%
\newcommand{\python}{\texttt{python}}
\newcommand{\textoazul}[1]{\textcolor{blue}{#1}}
\newcommand{\azulfuerte}[1]{\textcolor{blue}{\textbf{#1}}}
\newcommand{\funcionazul}[1]{\textcolor{blue}{\textbf{\texttt{#1}}}}
\newcommand{\ptilde}[1]{\ensuremath{{#1}^{\prime}}}
\newcommand{\stilde}[1]{\ensuremath{{#1}^{\prime \prime}}}
\newcommand{\ttilde}[1]{\ensuremath{{#1}^{\prime \prime \prime}}}
\newcommand{\ntilde}[2]{\ensuremath{{#1}^{(#2)}}}
\renewcommand{\arraystretch}{1.5}

\newcounter{saveenumi}
\newcommand{\seti}{\setcounter{saveenumi}{\value{enumi}}}
\newcommand{\conti}{\setcounter{enumi}{\value{saveenumi}}}
\renewcommand{\rmdefault}{cmr}% cmr = Computer Modern Roman

\linespread{1.5}

\usefonttheme{professionalfonts}
%\usefonttheme{serif}
\DeclareGraphicsExtensions{.pdf,.png,.jpg}


%Sección para el tema de beamer, con el theme, usercolortheme y sección de footers
\mode<presentation>
{
  \usetheme{Warsaw}
  
  %\useoutertheme{infolines}
  \useoutertheme{default}
  \usecolortheme{spruce}
  \setbeamercovered{invisible}
  % or whatever (possibly just delete it)
  \setbeamertemplate{section in toc}[sections numbered]
  \setbeamertemplate{subsection in toc}[subsections numbered]
  \setbeamertemplate{subsection in toc}{\leavevmode\leftskip=3.2em\rlap{\hskip-2em\inserttocsectionnumber.\inserttocsubsectionnumber}\inserttocsubsection\par}
  \setbeamercolor{section in toc}{fg=blue}
  \setbeamercolor{subsection in toc}{fg=blue}
  \setbeamercolor{frametitle}{fg=blue}

  \setbeamertemplate{footline}
  \beamertemplatenavigationsymbolsempty
  \setbeamertemplate{headline}{}
}

\makeatletter
\setbeamercolor{section in foot}{bg=gray!30, fg=black!90!orange}
\setbeamercolor{subsection in foot}{bg=blue!30!yellow, fg=red}
\setbeamertemplate{footline}
{
  \leavevmode%
  \hbox{%
  \begin{beamercolorbox}[wd=.333333\paperwidth,ht=2.25ex,dp=1ex,center]{section in foot}%
    \usebeamerfont{section in foot} \insertsection
  \end{beamercolorbox}}%
  \begin{beamercolorbox}[wd=.333333\paperwidth,ht=2.25ex,dp=1ex,center]{subsection in foot}%
    \usebeamerfont{subsection in foot}  \insertsubsection
  \end{beamercolorbox}%
  \begin{beamercolorbox}[wd=.333333\paperwidth,ht=2.25ex,dp=1ex,right]{date in head/foot}%
    \usebeamerfont{date in head/foot} \insertshortdate{} \hspace*{2em}
    \insertframenumber{} / \inserttotalframenumber \hspace*{2ex} 
  \end{beamercolorbox}}%
  \vskip0pt%
\makeatother  

\makeatletter
\patchcmd{\beamer@sectionintoc}
  {\vfill}
  {\vskip\itemsep}
  {}
  {}
\makeatother

% \makeatletter
% \patchcmd{\hyper@link@}
%   {{\Hy@tempb}{#4}}
%   {{\Hy@tempb}{\ul{#4}}}
%   {}{}
% \makeatother


% Sección para el código

\definecolor{Code}{rgb}{0,0,0}
\definecolor{Keywords}{rgb}{255,0,0}
\definecolor{Strings}{rgb}{255,0,255}
\definecolor{Comments}{rgb}{0,0,255}
\definecolor{Numbers}{rgb}{255,128,0}

\DeclareCaptionFont{white}{\color{white}}
\DeclareCaptionFormat{listing}{\colorbox{gray}{\parbox{0.99\textwidth}{#1#2#3}}}
\captionsetup[lstlisting]{format=listing,labelfont=white,textfont=white}
\renewcommand{\lstlistingname}{Código}

\lstset{
basicstyle=\ttfamily,
columns=fullflexible,
breaklines=true
}

\lstdefinestyle{codigopython}{%
  language=Python,                % choose the language of the code
  %basicstyle=\footnotesize\small,       % the size of the fonts that are used for the code
  numbers=left,                   % where to put the line-numbers
  numberstyle=\scriptsize,      % the size of the fonts that are used for the line-numbers
  stepnumber=1,                   % the step between two line-numbers. If it is 1 each line will be numbered
  numbersep=5pt,                  % how far the line-numbers are from the code
  backgroundcolor=\color{white},  % choose the background color. You must add \usepackage{color}
  showspaces=false,               % show spaces adding particular underscores
  showstringspaces=false,         % underline spaces within strings
  showtabs=false,                 % show tabs within strings adding particular underscores
  frame=single,   		% adds a frame around the code
  tabsize=2,  		% sets default tabsize to 2 spaces
  captionpos=t,   		% sets the caption-position to bottom
  breaklines=true,    	% sets automatic line breaking
  breakatwhitespace=false,    % sets if automatic breaks should only happen at whitespace
  escapeinside={\#},  % if you want to add a comment within your code
  stringstyle =\color{OliveGreen},
  texcl = true,
  %otherkeywords={{as}},             % Add keywords here
  keywordstyle = \color{blue},
  commentstyle = \color{black},
  identifierstyle = \color{black},
  % literate=%
  %         {á}{{\'a}}1
  %         {é}{{\'e}}1
  %         {í}{{\'i}}1
  %         {ó}{{\'o}}1
  %         {ú}{{\'u}}1
  %
  %keywordstyle=\ttb\color{deepblue}
  %fancyvrb = true,
literate={0}{{\textcolor{red}{0}}}{1}%
            {1}{{\textcolor{red}{1}}}{1}%
            {2}{{\textcolor{red}{2}}}{1}%
            {3}{{\textcolor{red}{3}}}{1}%
            {4}{{\textcolor{red}{4}}}{1}%
            {5}{{\textcolor{red}{5}}}{1}%
            {6}{{\textcolor{red}{6}}}{1}%
            {7}{{\textcolor{red}{7}}}{1}%
            {8}{{\textcolor{red}{8}}}{1}%
            {9}{{\textcolor{red}{9}}}{1}%
            {.0}{{\textcolor{red}{.0}}}{2}% Following is to ensure that only periods
            {.1}{{\textcolor{red}{.1}}}{2}% followed by a digit are changed.
            {.2}{{\textcolor{red}{.2}}}{2}%
            {.3}{{\textcolor{red}{.3}}}{2}%
            {.4}{{\textcolor{red}{.4}}}{2}%
            {.5}{{\textcolor{red}{.5}}}{2}%
            {.6}{{\textcolor{red}{.6}}}{2}%
            {.7}{{\textcolor{red}{.7}}}{2}%
            {.8}{{\textcolor{red}{.8}}}{2}%
            {.9}{{\textcolor{red}{.9}}}{2}%
            {\ }{{ }}{1}% handle the space
        ,%
        %mathescape=true
        %escapeinside={*@}
        escapeinside={A_}{_B}
}
