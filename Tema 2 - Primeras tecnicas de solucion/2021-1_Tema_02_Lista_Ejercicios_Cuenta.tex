\documentclass[hidelinks,12pt]{article}
\usepackage[left=0.25cm,top=1cm,right=0.25cm,bottom=1cm]{geometry}
%\usepackage[landscape]{geometry}
\textwidth = 20cm
\hoffset = -1cm
\usepackage[utf8]{inputenc}
\usepackage[spanish,es-tabla]{babel}
\usepackage[autostyle,spanish=mexican]{csquotes}
\usepackage[tbtags]{amsmath}
\usepackage{nccmath}
\usepackage{amsthm}
\usepackage{amssymb}
\usepackage{mathrsfs}
\usepackage{graphicx}
\usepackage{subfig}
\usepackage{standalone}
\usepackage[outdir=./Imagenes/]{epstopdf}
\usepackage{siunitx}
\usepackage{physics}
\usepackage{color}
\usepackage{float}
\usepackage{hyperref}
\usepackage{multicol}
%\usepackage{milista}
\usepackage{anyfontsize}
\usepackage{anysize}
%\usepackage{enumerate}
\usepackage[shortlabels]{enumitem}
\usepackage{capt-of}
\usepackage{bm}
\usepackage{relsize}
\usepackage{placeins}
\usepackage{empheq}
\usepackage{cancel}
\usepackage{wrapfig}
\usepackage[flushleft]{threeparttable}
\usepackage{makecell}
\usepackage{fancyhdr}
\usepackage{tikz}
\usepackage{bigints}
\usepackage{scalerel}
\usepackage{pgfplots}
\usepackage{pdflscape}
\pgfplotsset{compat=1.16}
\spanishdecimal{.}
\renewcommand{\baselinestretch}{1.5} 
\renewcommand\labelenumii{\theenumi.{\arabic{enumii}})}
\newcommand{\ptilde}[1]{\ensuremath{{#1}^{\prime}}}
\newcommand{\stilde}[1]{\ensuremath{{#1}^{\prime \prime}}}
\newcommand{\ttilde}[1]{\ensuremath{{#1}^{\prime \prime \prime}}}
\newcommand{\ntilde}[2]{\ensuremath{{#1}^{(#2)}}}

\newtheorem{defi}{{\it Definición}}[section]
\newtheorem{teo}{{\it Teorema}}[section]
\newtheorem{ejemplo}{{\it Ejemplo}}[section]
\newtheorem{propiedad}{{\it Propiedad}}[section]
\newtheorem{lema}{{\it Lema}}[section]
\newtheorem{cor}{Corolario}
\newtheorem{ejer}{Ejercicio}[section]

\newlist{milista}{enumerate}{2}
\setlist[milista,1]{label=\arabic*)}
\setlist[milista,2]{label=\arabic{milistai}.\arabic*)}
\newlength{\depthofsumsign}
\setlength{\depthofsumsign}{\depthof{$\sum$}}
\newcommand{\nsum}[1][1.4]{% only for \displaystyle
    \mathop{%
        \raisebox
            {-#1\depthofsumsign+1\depthofsumsign}
            {\scalebox
                {#1}
                {$\displaystyle\sum$}%
            }
    }
}
\def\scaleint#1{\vcenter{\hbox{\scaleto[3ex]{\displaystyle\int}{#1}}}}
\def\bs{\mkern-12mu}


\title{Tema 2 - Lista de ejercicios a cuenta\\ \large{Matemáticas Avanzadas de la Física}\vspace{-3ex}}
\author{M. en C. Gustavo Contreras Mayén}
\date{ }
\begin{document}
\vspace{-4cm}
\maketitle
\fontsize{14}{14}\selectfont
\section{Presentación 2}
\begin{enumerate}
\item  Del ejercicio (\ref{item2}) al (\ref{item7}) clasifica la EDP que se muestra (orden, linealidad, tipo de EDP), tendrás que mostrar tu argumento para la respuesta, es decir, justificar por qué clasificaste de esa forma.
\begin{enumerate}
\item $2 \, u_{xx} + 6 \, u_{xy} + 5 \, u_{yy} + u_{x} = 0$ \label{item2}
\item $u_{xx} - 2 \, u_{xy} + u_{yy} + 3 \, u_{x} - u_{y} = 0$ \label{item3}
\item $u_{xx} + 6 \, u_{xy} + 9 \, u_{yy} + 3 \, y \, u_{y} = 0$ \label{item5}
\item $u_{xx} - 2 \, \cos x \, u_{xy} +  (2 - \sin^{2} x) \, u_{yy} + u = 0$ \label{item7}
\end{enumerate}
\item Demuestra que la ecuación de Helmholtz
\begin{align*}
\laplacian \psi + k^{2} \: \psi = 0
\end{align*}
es separable en coordenadas cilíndricas circulares si $k^{2}$ se generaliza como
\begin{align*}
k^{2} + f(\rho) + \left( \dfrac{1}{\rho^{2}} \right) \: g(\varphi) + h(z)
\end{align*}
es decir, la ecuación de Helmholtz es:
\begin{align*}
\laplacian \psi + \left( k^{2} + f(\rho) + \left( \dfrac{1}{\rho^{2}} \right) \: g(\varphi) + h(z) \right) \, \psi = 0
\end{align*}
\end{enumerate}
\section{Presentación 3.}
\begin{enumerate}
\item Determina los puntos singulares de las siguientes ED, clasifica cada punto singular en regular o irregular.
\begin{enumerate}
\item $x^{3} \, \stilde{y} + 4 \, x^{2} \, \ptilde{y} + 3 \, y = 0$
\item $x \, \stilde{y} - (x + 3)^{-2} \, y = 0$
\item $(x^{2} - 9)^{2} \, \stilde{y} + (x + 3) \, \ptilde{y} + 2 \, y = 0$
\item $\stilde{y} - \dfrac{1}{x} \, \ptilde{y} + \dfrac{1}{(x - 1)^{3}} \, y = 0$
\end{enumerate}
\item Resuelve las siguientes ED con el método de Frobenius:
\begin{enumerate}
\item $2 \, x \, \stilde{y} - \ptilde{y} + 2 \, y = 0$
\item $2 \, x \, \stilde{y} + 5 \, \ptilde{y} + x \, y = 0$
\item $x (x - 1) \, \stilde{y} + 3 \, \ptilde{y} - 2 \, y = 0$
\item $\stilde{y} - \dfrac{3}{x} \, \ptilde{y} - 2 \, y = 0$
\end{enumerate}
\end{enumerate}
\section{Presentación 4.}
\begin{enumerate}
\item Si el Wronskiano de dos funciones $y_{1}$ e $y_{2}$ es igual a cero, demuestra mediante integración directa que:
\begin{align*}
y_{1} = c \, y_{2}
\end{align*}
Considera la suposición de que las funciones tienen derivadas continuas y que al menos una de las funciones no desaparece en e intervalo bajo consideración.
\item Considera las siguientes funciones:
\begin{align*}
\varphi_{1} =  x \hspace{2cm} \varphi_{2} = \abs{x} = x \, \, \mbox{sgn} \, x
\end{align*}
Donde la función \textit{sgn} es precisamente el signo de $x$. Como 
\begin{align*}
\ptilde{\varphi}_{1} &=  1 \hspace{2cm} \ptilde{\varphi}_{2} = \mbox{sgn} \, x \\
&\Rightarrow W(\varphi_{1}, \varphi_{2}) = 0
\end{align*}
para cualquier intervalo, incluyendo $[-1, +1]$.
\par
Comprueba si la cancelación del Wronskiano en $[-1, +1]$ demustra que $\varphi_{1}$ y $\varphi_{2}$ son linealmente independientes. Evidentemente que no lo son. ¿En dónde está el error? (Tip: gráfica las dos funciones en el intervalo para visualizar las funciones)
\item Considerando que una solución de
\begin{align*}
\stilde{R} + \dfrac{1}{r} \, \ptilde{R} - \dfrac{m^{2}}{r^{2}} \, R = 0
\end{align*}
es $R = r^{m}$. Demuestra que la ecuación (\ref{eq:ecuacion_09_127}) predice una segunda solución: $R = r^{-m}$
\begin{align}
\setlength{\fboxsep}{2\fboxsep}\boxed{y_{2}(x) =  y_{1} \: (x) \int^{x} \dfrac{\exp \left[ \displaystyle - \int^{x_{2}} P(x_{1}) \: \dd{x_{1}} \right]}{[y_{1}(x_{2})]^{2}} \dd{x_{2}}}
\label{eq:ecuacion_09_127}
\end{align}
\item Una solución para la ecuación diferencial de Hermite
\begin{align*}
\stilde{y} - 2 \, x \, y + 2 \, \alpha \, y = 0
\end{align*}
para $\alpha = 0$ es $y_{1} = 1$. Encuentra una segunda solución $y_{2}(x)$, usando la ecuación (\ref{eq:ecuacion_09_127})
\end{enumerate}
\section{Ejercicios Opcionales 1}
\begin{enumerate}
\item $u_{xx} +  2 \, \cosh x \, u_{xy} + (\sinh^{2} x - 8) \, u_{yy} + u = 0$
\item $x^{2} \, u_{xx} + 4 \, x \, y \, u_{xy} + 4 \, y^{2} \, u_{yy} + 2 \, u_{y} = 0$
\item $y^{2} \, u_{xx} + x^{2} \, u_{yy} = 0$
\item $y^{2} \, u_{xx} + 2 \, x \, y \, u_{xy} + 2 \, x^{2} \, u_{yy} + x \, u_{x} = 0$
\item $y^{2} u_{xx} + u_{yy} = 0$
\item $u_{xx} + 2 \, u_{xy} + \cos^{2} x \, u_{yy} = 0$
\end{enumerate}
\emph{Sugerencia: } Dentro de la respuesta en la plataforma, anota el valor del discriminante $D = B^{2} - 4 \, A \, C$, y si hay que describir algún caso en particular, te ayudará para esa aclaración.
\section{Ejercicios Opcionales 2.}
\begin{enumerate}
\item Determina los puntos singulares de las siguientes ecuaciones diferenciales. Clasifica el(los) punto(s) como singular o irregular.
\begin{enumerate}
\item $x^{2} \, (x - 5) \stilde{y} - 2 \, x \, \ptilde{y} + 6 \, y = 0$
\item $(x^{2} + x - 6) \, \stilde{y} + (x + 3)  \, \ptilde{y} + (x - 2) \, y = 0$
\item $x^{3} \, (x^{2} - 25) \, (x - 2)^{2} \, \stilde{y} + 3 \, x \, (x - 2) \, \ptilde{y} + 7 \, (x + 5) \, y = 0$
\end{enumerate}
\item Ocupa el método de Frobenius y presenta: la ecuación de índices, las raíces así como la(s) solución(es) de la EDO2 de los problemas:
\begin{enumerate}
\item $3 \, x \, \stilde{y} + (2 - x) \, \ptilde{y} - y = 0$
\item $x^{2} \, \stilde{y} + x \, \ptilde{y} + \left( x^{2} - \dfrac{1}{4} \right) \, y = 0$
\end{enumerate}
\end{enumerate}
\section{Ejercicios Opcionales 3.}
En los siguientes ejercicios (1-4) determina si el par de funciones dadas son linealmente independientes o dependientes.
\begin{enumerate}
\item $f(x) = x^{2} + 5 \, x \hspace{2cm} g(x) = x^{2} - 5 \, x$ 
\item $f(x) = \cos (3 \, x) \hspace{2cm} g(x) = 4 \, \cos x - 3 \, \cos x$
\item $f(x) = e^{\lambda x} \, \cos (\mu \, x) \hspace{1.2cm} g(x) = e^{\lambda x} \, \sin \mu x \hspace{0.4cm} \mu \neq 0 $
\item $f(x) = e^{3 x} \hspace{3.1cm} g(x) = e^{3(x-1)}$
% \item $f(x) = 3 \, x - 5 \hspace{2.3cm} g(x) = 9 \, x - 15$
% \item $f(x) = x \hspace{3.5cm} g(x) = x^{-1}$
\item El Wronskiano de dos funciones es
\begin{align*}
W(x) = x \, \sin^{2} x
\end{align*}
¿Las funciones son linealmente independientes o dependientes?¿Por qué?
\item Calcula al menos los primeros tres términos no nulos en el desarrollo en serie de potencias alrededor de $x=0$ para obtener una solución general de:
\begin{align*}
9 \, x^{2} \, \stilde{y} + 9 \, x^{2} \, \ptilde{y} + 2 \, y = 0
\end{align*}
con $x > 0$. Deberás de obtener la ecuación de índices, así como la primera solución mediante el método de Frobenius, posteriormente, expresar la segunda solución linealmente independiente.
\end{enumerate}
\end{document}