% \RequirePackage{currfile}
\documentclass[12pt]{beamer}
\usepackage[utf8]{inputenc}
\usepackage[spanish]{babel}
\usepackage{standalone}
\usepackage{color}
\usepackage{siunitx}
\usepackage{hyperref}
\usepackage[outdir=./]{epstopdf}
%\hypersetup{colorlinks,linkcolor=,urlcolor=blue}
%\hypersetup{colorlinks,urlcolor=blue}
\usepackage{xcolor,soul}
\usepackage{etoolbox}
\usepackage{amsmath}
\usepackage{amsthm}
\usepackage{mathtools}
\usepackage{tcolorbox}
\usepackage{physics}
\usepackage{multicol}
\usepackage{bookmark}
\usepackage{longtable}
\usepackage{listings}
\usepackage{cancel}
\usepackage{wrapfig}
\usepackage{empheq}
\usepackage{graphicx}
\usepackage{tikz}
\usetikzlibrary{calc, patterns, matrix, backgrounds, decorations,shapes, arrows.meta}
\usepackage[autostyle,spanish=mexican]{csquotes}
\usepackage[os=win]{menukeys}
\usepackage{pifont}
\usepackage{pbox}
\usepackage{bm}
\usepackage{caption}
\captionsetup{font=scriptsize,labelfont=scriptsize}
%\usepackage[sfdefault]{roboto}  %% Option 'sfdefault' only if the base font of the document is to be sans serif

%Sección de definición de colores
\definecolor{ao}{rgb}{0.0, 0.5, 0.0}
\definecolor{bisque}{rgb}{1.0, 0.89, 0.77}
\definecolor{amber}{rgb}{1.0, 0.75, 0.0}
\definecolor{armygreen}{rgb}{0.29, 0.33, 0.13}
\definecolor{alizarin}{rgb}{0.82, 0.1, 0.26}
\definecolor{cadetblue}{rgb}{0.37, 0.62, 0.63}
\definecolor{deepblue}{rgb}{0,0,0.5}
\definecolor{brown}{rgb}{0.59, 0.29, 0.0}
\definecolor{OliveGreen}{rgb}{0,0.25,0}
\definecolor{mycolor}{rgb}{0.122, 0.435, 0.698}

\newcommand*{\boxcolor}{orange}
\makeatletter
\newcommand{\boxedcolor}[1]{\textcolor{\boxcolor}{%
\tikz[baseline={([yshift=-1ex]current bounding box.center)}] \node [rectangle, minimum width=1ex, thick, rounded corners,draw] {\normalcolor\m@th$\displaystyle#1$};}}
 \makeatother

\newtcbox{\mybox}{on line,
  colframe=mycolor,colback=mycolor!10!white,
  boxrule=0.5pt,arc=4pt,boxsep=0pt,left=6pt,right=6pt,top=6pt,bottom=6pt}

\usefonttheme[onlymath]{serif}
%Sección de definición de nuevos comandos

\newcommand*{\TitleParbox}[1]{\parbox[c]{1.75cm}{\raggedright #1}}%
\newcommand{\python}{\texttt{python}}
\newcommand{\textoazul}[1]{\textcolor{blue}{#1}}
\newcommand{\azulfuerte}[1]{\textcolor{blue}{\textbf{#1}}}
\newcommand{\funcionazul}[1]{\textcolor{blue}{\textbf{\texttt{#1}}}}
\newcommand{\ptilde}[1]{\ensuremath{{#1}^{\prime}}}
\newcommand{\stilde}[1]{\ensuremath{{#1}^{\prime \prime}}}
\newcommand{\ttilde}[1]{\ensuremath{{#1}^{\prime \prime \prime}}}
\newcommand{\ntilde}[2]{\ensuremath{{#1}^{(#2)}}}
\renewcommand{\arraystretch}{1.5}

\newcounter{saveenumi}
\newcommand{\seti}{\setcounter{saveenumi}{\value{enumi}}}
\newcommand{\conti}{\setcounter{enumi}{\value{saveenumi}}}
\renewcommand{\rmdefault}{cmr}% cmr = Computer Modern Roman

\linespread{1.5}

\usefonttheme{professionalfonts}
%\usefonttheme{serif}
\DeclareGraphicsExtensions{.pdf,.png,.jpg}


%Sección para el tema de beamer, con el theme, usercolortheme y sección de footers
\mode<presentation>
{
  \usetheme{CambridgeUS}
  
  %\useoutertheme{infolines}
  \useoutertheme{default}
  \usecolortheme{beaver}
  \setbeamercovered{invisible}
  % or whatever (possibly just delete it)
  \setbeamertemplate{section in toc}[sections numbered]
  \setbeamertemplate{subsection in toc}[subsections numbered]
  \setbeamertemplate{subsection in toc}{\leavevmode\leftskip=3.2em\rlap{\hskip-2em\inserttocsectionnumber.\inserttocsubsectionnumber}\inserttocsubsection\par}
  \setbeamercolor{section in toc}{fg=blue}
  \setbeamercolor{subsection in toc}{fg=blue}
  \setbeamercolor{frametitle}{fg=blue}
  \setbeamertemplate{caption}[numbered]

  \setbeamertemplate{footline}
  \beamertemplatenavigationsymbolsempty
  \setbeamertemplate{headline}{}
}

\makeatletter
\setbeamercolor{section in foot}{bg=gray!30, fg=black!90!orange}
\setbeamercolor{subsection in foot}{bg=blue!30!yellow, fg=red}
\setbeamertemplate{footline}
{
  \leavevmode%
  \hbox{%
  \begin{beamercolorbox}[wd=.333333\paperwidth,ht=2.25ex,dp=1ex,center]{section in foot}%
    \usebeamerfont{section in foot} \insertsection
  \end{beamercolorbox}}%
  \begin{beamercolorbox}[wd=.333333\paperwidth,ht=2.25ex,dp=1ex,center]{subsection in foot}%
    \usebeamerfont{subsection in foot}  \insertsubsection
  \end{beamercolorbox}%
  \begin{beamercolorbox}[wd=.333333\paperwidth,ht=2.25ex,dp=1ex,right]{date in head/foot}%
    \usebeamerfont{date in head/foot} \insertshortdate{} \hspace*{2em}
    \insertframenumber{} / \inserttotalframenumber \hspace*{2ex} 
  \end{beamercolorbox}}%
  \vskip0pt%
\makeatother  

\makeatletter
\patchcmd{\beamer@sectionintoc}
  {\vfill}
  {\vskip\itemsep}
  {}
  {}
\makeatother



\documentclass[12pt]{beamer}
\usepackage{../Estilos/BeamerMAF}
\usepackage{../Estilos/ColoresLatex}

%Sección para el tema de beamer, con el theme, usercolortheme y sección de footers
\usetheme{CambridgeUS}
\usecolortheme{beaver}
%\useoutertheme{default}
\setbeamercovered{invisible}
% or whatever (possibly just delete it)
\setbeamertemplate{section in toc}[sections numbered]
\setbeamertemplate{subsection in toc}[subsections numbered]
\setbeamertemplate{subsection in toc}{\leavevmode\leftskip=3.2em\rlap{\hskip-2em\inserttocsectionnumber.\inserttocsubsectionnumber}\inserttocsubsection\par}
\setbeamercolor{section in toc}{fg=blue}
\setbeamercolor{subsection in toc}{fg=blue}
\setbeamercolor{frametitle}{fg=blue}
\setbeamertemplate{caption}[numbered]

\setbeamertemplate{footline}
\beamertemplatenavigationsymbolsempty
\setbeamertemplate{headline}{}


\makeatletter
\setbeamercolor{section in foot}{bg=gray!30, fg=black!90!orange}
\setbeamercolor{subsection in foot}{bg=blue!30!yellow, fg=red}
\setbeamercolor{date in foot}{bg=black, fg=white}
\setbeamertemplate{footline}
{
  \leavevmode%
  \hbox{%
  \begin{beamercolorbox}[wd=.333333\paperwidth,ht=2.25ex,dp=1ex,center]{section in foot}%
    \usebeamerfont{section in foot} \insertsection
  \end{beamercolorbox}%
  \begin{beamercolorbox}[wd=.333333\paperwidth,ht=2.25ex,dp=1ex,center]{subsection in foot}%
    \usebeamerfont{subsection in foot}  \insertsubsection
  \end{beamercolorbox}%
  \begin{beamercolorbox}[wd=.333333\paperwidth,ht=2.25ex,dp=1ex,right]{date in head/foot}%
    \usebeamerfont{date in head/foot} \insertshortdate{} \hspace*{2em}
    \insertframenumber{} / \inserttotalframenumber \hspace*{2ex} 
  \end{beamercolorbox}}%
  \vskip0pt%
}
\makeatother\newlength{\depthofsumsign}
\setlength{\depthofsumsign}{\depthof{$\sum$}}
\newcommand{\nsum}[1][1.4]{% only for \displaystyle
    \mathop{%
        \raisebox
            {-#1\depthofsumsign+1\depthofsumsign}
            {\scalebox
                {#1}
                {$\displaystyle\sum$}%
            }
    }
}
\def\scaleint#1{\vcenter{\hbox{\scaleto[3ex]{\displaystyle\int}{#1}}}}
\def\bs{\mkern-12mu}






\AtBeginDocument{\RenewCommandCopy\qty\SI}
\ExplSyntaxOn
\msg_redirect_name:nnn { siunitx } { physics-pkg } { none }
\ExplSyntaxOff

\title{\large{Objetivos del Tema 2 - Primeras técnicas de solución}}
% \subtitle{Curso MAF}
\author{M. en C. Gustavo Contreras Mayén}

\setbeamertemplate{navigation symbols}{}

\setbeamertemplate{footline}
{
  \leavevmode%
  \hbox{%
  \begin{beamercolorbox}[wd=.333333\paperwidth,ht=2.25ex,dp=1ex,center]{section in foot}%
    \usebeamerfont{section in foot} \insertsection
  \end{beamercolorbox}%
  \begin{beamercolorbox}[wd=.333333\paperwidth,ht=2.25ex,dp=1ex,center]{subsection in foot}%
    \usebeamerfont{subsection in foot}  \insertsubsection
  \end{beamercolorbox}%
  \begin{beamercolorbox}[wd=.333333\paperwidth,ht=2.25ex,dp=1ex,right]{date in head/foot}%
    \usebeamerfont{date in head/foot}  \hspace*{2em}
    \insertframenumber{} / \inserttotalframenumber \hspace*{2ex} 
  \end{beamercolorbox}}%
  \vskip0pt%
}
\makeatother
\date{}

\begin{document}
\maketitle
\fontsize{14}{14}\selectfont
\spanishdecimal{.}

\section*{Contenido}
\frame[allowframebreaks]{\frametitle{Contenido} \tableofcontents[currentsection, hideallsubsections]}

\section{Introducción}
\frame{\frametitle{Temas a revisar} \tableofcontents[currentsection, hideothersubsections]}
\subsection{Las EDP}

\begin{frame}
\frametitle{Las Ecs. Diferenciales Parciales}
De la ecuación de movimiento de una partícula para conocer la fuerza por ejemplo, en general nos conduce a una ecuación diferencial.
\\
\bigskip
\pause
Por lo tanto, en casi toda la física  básica y en una mayor parte de la física teórica avanzada se expresa en términos de ecuaciones diferenciales.
\end{frame}
\begin{frame}
\frametitle{Las Ecs. Diferenciales Parciales}
A veces, esas son ecuaciones diferenciales ordinarias en una variable (EDO).
\\
\bigskip
\pause
Más a menudo las ecuaciones son expresiones de ecuaciones diferenciales parciales (EDP) en dos o más variables.
\end{frame}
\begin{frame}
\frametitle{Apoyo con el Tema 1}
Al haber revisado la construcción de sistemas curvilíneos ortogonales en el Tema 1, así como la caracterización de esos sistemas con el uso de operadores diferenciales, ahora contamos con una herramienta importante.
\end{frame}
\begin{frame}
\frametitle{Apoyo con el Tema 1}
El modelar un fenómeno en una geometría particular mediante una EDP, nos devolverá un EDP y como primera técnica de solución, ocuparemos \textocolor{coquelicot}{la técnica de separación de variables}.
\end{frame}
\begin{frame}
\frametitle{Punto importante}
Cabe señalar que la construcción de una ED a partir de un fenómeno de la física, no se realizará en el curso, es decir, ocuparemos una EDP ya definida, considerando que la formalización de dicha expresión la revisaron en los cursos de los semestres previos a MAF.
\end{frame}
\begin{frame}
\frametitle{Punto importante}
Por lo que es recomendable que le den un repaso a la construcción de dichas expresiones, consultando las referencias o bibliografía en particular.
\end{frame}
\begin{frame}
\frametitle{Soluciones obtenidas con la técnica}
Como veremos en la técnica, reducieremos el orden de la ED, además de que pasamos de tener una EDP a un sistema de EDO.
\\
\bigskip
Veremos un hecho curioso de que la gran mayoría de las EDP de la física matemática, son EDP de segundo orden.
\end{frame}
\begin{frame}
\frametitle{Soluciones obtenidas con la técnica}
El sistema de EDO obtenido se resuelve de una manera mucho más sencilla, tal y como lo vieron en su curso de \emph{Ecuaciones diferenciales I} en la carrera.
\end{frame}
\begin{frame}
\frametitle{Etapas de trabajo}
En todo momento esperamos que ocupen la siguiente estrategia de trabajo:
\pause
\setbeamercolor{item projected}{bg=bananayellow,fg=blue}
\setbeamertemplate{enumerate items}{%
\usebeamercolor[bg]{item projected}%
\raisebox{1.5pt}{\colorbox{bg}{\color{fg}\footnotesize\insertenumlabel}}%
}
\begin{enumerate}[<+->]
\item Formulación.
\item Solución.
\item Intepretación.
\end{enumerate}
\end{frame}

\section{Objetivos}
\frame{\frametitle{Temas a revisar} \tableofcontents[currentsection, hideothersubsections]}
\subsection{Tema 2}

\begin{frame}
\frametitle{Objetivos del Tema 2}
Al concluir el Tema 2, el alumno:
\setbeamercolor{item projected}{bg=corn,fg=black}
\setbeamertemplate{enumerate items}{%
\usebeamercolor[bg]{item projected}%
\raisebox{1.5pt}{\colorbox{bg}{\color{fg}\footnotesize\insertenumlabel}}%
}
\begin{enumerate}
\item Clasificará una ecuación diferencial parcial a partir de las propiedades que presente la expresión.
\seti
\end{enumerate}
\end{frame}
\begin{frame}
\frametitle{Objetivos del Tema 2}
\setbeamercolor{item projected}{bg=corn,fg=black}
\setbeamertemplate{enumerate items}{%
\usebeamercolor[bg]{item projected}%
\raisebox{1.5pt}{\colorbox{bg}{\color{fg}\footnotesize\insertenumlabel}}%
}
\begin{enumerate}
\conti
\item Obtendrá las soluciones de una ecuación diferencial parcial lineal y homogénea mediante la técnica de separación de variables.
\seti
\end{enumerate}
\end{frame}
\begin{frame}
\frametitle{Objetivos del Tema 2}
\setbeamercolor{item projected}{bg=corn,fg=black}
\setbeamertemplate{enumerate items}{%
\usebeamercolor[bg]{item projected}%
\raisebox{1.5pt}{\colorbox{bg}{\color{fg}\footnotesize\insertenumlabel}}%
}
\begin{enumerate}[<+->]
\conti
\item Obtendrá las soluciones en series de potencias de una ecuación diferencial parcial a partir del método de Frobenius.
\item Identificará las singularidades presentes en la EDP, para realizar la remoción de las mismas y procederá a resolver la ecuación con el método de Frobenius.
\seti
\end{enumerate}
\end{frame}
\begin{frame}
\frametitle{Objetivos del Tema 2}
\setbeamercolor{item projected}{bg=corn,fg=black}
\setbeamertemplate{enumerate items}{%
\usebeamercolor[bg]{item projected}%
\raisebox{1.5pt}{\colorbox{bg}{\color{fg}\footnotesize\insertenumlabel}}%
}
\begin{enumerate}
\conti
\item Identificará el caso en donde sea posible y resolverá para obtener una segunda solución linealmente independiente de una EDP de segundo orden.
\seti
\end{enumerate}
\end{frame}
\begin{frame}
\frametitle{Objetivos del Tema 2}
\setbeamercolor{item projected}{bg=corn,fg=black}
\setbeamertemplate{enumerate items}{%
\usebeamercolor[bg]{item projected}%
\raisebox{1.5pt}{\colorbox{bg}{\color{fg}\footnotesize\insertenumlabel}}%
}
\begin{enumerate}
\conti
\item Expondrá la interpretación de (los) resultado(s) obtenido(s), para vincularlos con el fenómeno de estudio.
\seti
\end{enumerate}
\end{frame}
\begin{frame}
\frametitle{Apoyo adicional}
En este tema como se habrán dado cuenta, será necesario apoyarse \enquote{fuertemente} en las técnicas de solución de las EDO.
\\
\bigskip
\pause
Así como del manejo de series de potencias que se revisa en el curso de \emph{Álgebra lineal.}
\end{frame}
\begin{frame}
\frametitle{Apoyo con software}
También podrán apoyarse con software para corroborar sus resultados y en su caso, para explorar algunas características de las soluciones de las EDO, para la interpretación de lo que obtienen.
\end{frame}
\begin{frame}
\frametitle{Materiales de revisión}
Los materiales de trabajo que tendrán en la plataforma son los siguientes:
\setbeamercolor{item projected}{bg=red,fg=white}
\setbeamertemplate{enumerate items}{%
\usebeamercolor[bg]{item projected}%
\raisebox{1.5pt}{\colorbox{bg}{\color{fg}\footnotesize\insertenumlabel}}%
}
\begin{enumerate}[<+->]
\item Introducción Ecs. Diferenciales Parciales.
\item Separación de variables.
\item Método de Frobenius y remoción de singularidades.
\item Segunda solución linealmente independiente.
\end{enumerate}
\end{frame}

\end{document}