\documentclass[hidelinks,12pt]{article}
\usepackage[left=0.25cm,top=1cm,right=0.25cm,bottom=1cm]{geometry}
%\usepackage[landscape]{geometry}
\textwidth = 20cm
\hoffset = -1cm
\usepackage[utf8]{inputenc}
\usepackage[spanish,es-tabla]{babel}
\usepackage[autostyle,spanish=mexican]{csquotes}
\usepackage[tbtags]{amsmath}
\usepackage{nccmath}
\usepackage{amsthm}
\usepackage{amssymb}
\usepackage{mathrsfs}
\usepackage{graphicx}
\usepackage{subfig}
\usepackage{standalone}
\usepackage[outdir=./Imagenes/]{epstopdf}
\usepackage{siunitx}
\usepackage{physics}
\usepackage{color}
\usepackage{float}
\usepackage{hyperref}
\usepackage{multicol}
%\usepackage{milista}
\usepackage{anyfontsize}
\usepackage{anysize}
%\usepackage{enumerate}
\usepackage[shortlabels]{enumitem}
\usepackage{capt-of}
\usepackage{bm}
\usepackage{relsize}
\usepackage{placeins}
\usepackage{empheq}
\usepackage{cancel}
\usepackage{wrapfig}
\usepackage[flushleft]{threeparttable}
\usepackage{makecell}
\usepackage{fancyhdr}
\usepackage{tikz}
\usepackage{bigints}
\usepackage{scalerel}
\usepackage{pgfplots}
\usepackage{pdflscape}
\pgfplotsset{compat=1.16}
\spanishdecimal{.}
\renewcommand{\baselinestretch}{1.5} 
\renewcommand\labelenumii{\theenumi.{\arabic{enumii}})}
\newcommand{\ptilde}[1]{\ensuremath{{#1}^{\prime}}}
\newcommand{\stilde}[1]{\ensuremath{{#1}^{\prime \prime}}}
\newcommand{\ttilde}[1]{\ensuremath{{#1}^{\prime \prime \prime}}}
\newcommand{\ntilde}[2]{\ensuremath{{#1}^{(#2)}}}

\newtheorem{defi}{{\it Definición}}[section]
\newtheorem{teo}{{\it Teorema}}[section]
\newtheorem{ejemplo}{{\it Ejemplo}}[section]
\newtheorem{propiedad}{{\it Propiedad}}[section]
\newtheorem{lema}{{\it Lema}}[section]
\newtheorem{cor}{Corolario}
\newtheorem{ejer}{Ejercicio}[section]

\newlist{milista}{enumerate}{2}
\setlist[milista,1]{label=\arabic*)}
\setlist[milista,2]{label=\arabic{milistai}.\arabic*)}
\newlength{\depthofsumsign}
\setlength{\depthofsumsign}{\depthof{$\sum$}}
\newcommand{\nsum}[1][1.4]{% only for \displaystyle
    \mathop{%
        \raisebox
            {-#1\depthofsumsign+1\depthofsumsign}
            {\scalebox
                {#1}
                {$\displaystyle\sum$}%
            }
    }
}
\def\scaleint#1{\vcenter{\hbox{\scaleto[3ex]{\displaystyle\int}{#1}}}}
\def\bs{\mkern-12mu}


\title{Ejercicios opcionales \\[0.3em]  \large{Clasificación de EDP} \vspace{-3ex}}
\author{M. en C. Gustavo Contreras Mayén}
\date{ }

\begin{document}
\vspace{-4cm}
\maketitle
\fontsize{14}{14}\selectfont


En los siguientes ejercicios clasifica cada una de las EDP: orden, linealidad, tipo de EDP mostrando el argumento para la respuesta, es decir, justificar por qué clasificaste la ecuación de esa forma.

%Ref. DuChateu (1986) - Schaum's Theory and Problems PDE 2.19
\textbf{Ejercicio opcional (1).}

\begin{enumerate}[label=\roman*)]
\item \Large{$2 \, u_{xx} + 6 \, u_{xy} + 5 \, u_{yy} + u_{x} = 0$} \label{item2}
\item $u_{xx} - 2 \, u_{xy} + u_{yy} + 3 \, u_{x} - u_{y} = 0$ \label{item3}
\item $u_{xx} + 6 \, u_{xy} + 9 \, u_{yy} + 3 \, y \, u_{y} = 0$ \label{item5}
\item $u_{xx} - 2 \, \cos x \, u_{xy} +  (2 - \sin^{2} x) \, u_{yy} + u = 0$ \label{item7}
\end{enumerate}
%Ref. DuChateu (1986) - Schaum's Theory and Problems PDE 2.1
\textbf{Ejercicio opcional (2).}
\begin{enumerate}[label=\roman*), resume]
\item \Large{$2 \, u_{xx} - 4\, u_{xy} - 6 \, u_{uu} + u_{x} = 0$}
\item $4 \, u_{xx} + 12 \, u_{xy} + 9 \, u_{yy} - 2 \, u_{x} + u = 0$
\item $u_{xx} - x^{2}\, y \, u_{yy} = 0 \hspace{0.7cm} y > 0$
\item $e^{2x} \, u_{xx} + 2 \, e^{x+y} \, u_{xy} + e^{2y} \, u_{yy} = 0$
\end{enumerate}
\end{document}