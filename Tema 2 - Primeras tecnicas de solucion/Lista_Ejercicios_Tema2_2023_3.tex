\documentclass[12pt]{beamer}
\usepackage{../Estilos/BeamerMAF}
\usepackage{../Estilos/ColoresLatex}
\input{../Preambulos/preambulo_Beamer_Warsaw_seahorse}

\title{\large{Evaluación Semanal Tema 2}}
\subtitle{Tema 2 - Primeras técnicas de solución}

\author{M. en C. Gustavo Contreras Mayén}

\date{}
\resetcounteronoverlays{saveenumi}

\begin{document}
\maketitle
\fontsize{14}{14}\selectfont
\spanishdecimal{.}

\section*{Contenido}
\frame[allowframebreaks]{\tableofcontents[currentsection, hideallsubsections]}

\section{Enunciados}
\frame{\tableofcontents[currentsection, hideothersubsections]}
\subsection{Ejercicio 1}

\begin{frame}
\frametitle{Enunciado Ejercicio 1}
Un cubo hecho de material cuya conductividad es $k$, tiene como seis caras los planos $x = \pm a$, $y = \pm a$ y $z = \pm a$, y no contiene fuentes de calor internas. Verifica que la distribución de temperatura:
\begin{align*}
u (x, y, z, t) = A \, \cos \dfrac{\pi \, x}{a} \, \sin \dfrac{\pi \, z}{a} \, \exp \left( - \dfrac{2 \, k \, \pi^{2} \, t}{a^{2}} \right)
\end{align*}
\end{frame}

\begin{frame}
\frametitle{Verifica que:}
\setbeamercolor{item projected}{bg=bananayellow,fg=ao}
\setbeamertemplate{enumerate items}{%
\usebeamercolor[bg]{item projected}%
\raisebox{1.5pt}{\colorbox{bg}{\color{fg}\footnotesize\insertenumlabel}}%
}
\begin{enumerate}[<+->]
\item Satisface la correspondiente ecuación de difusión.
\item ¿A través de qué caras hay flujo de calor?
\item ¿Cuál es la dirección y la velocidad del flujo de calor en el punto \hfill \break $\left(\dfrac{3 a}{4}, \dfrac{a}{4}, a \right)$ en el tiempo $t = \dfrac{a^{2}}{(k \, \pi^{2})}$?
\end{enumerate}
\end{frame}

\subsection{Ejercicio 2}

\begin{frame}
\frametitle{Enunciado del Ejercicio 2}
Determina la naturaleza de los puntos singulares y resuelve la ecuación de Legendre:
\begin{align*}
(1 -x^{2}) \, \stilde{y} -  2 \, x \, \ptilde{y} + n(n + 1) \, y = 0
\end{align*}
\end{frame}

\begin{frame}
\frametitle{Por resolver}
\setbeamercolor{item projected}{bg=carmine,fg=white}
\setbeamertemplate{enumerate items}{%
\usebeamercolor[bg]{item projected}%
\raisebox{1.5pt}{\colorbox{bg}{\color{fg}\footnotesize\insertenumlabel}}%
}
\begin{enumerate}[<+->]
\item Comprueba que la ecuación de índices es: $k \, (k - 1) = 0$
\item Con $k = 0$, 
\begin{itemize}
\item Expresa la relación de recurrencia.
\item Demuestra que la solución se expresa como una serie de potencias pares de $x$ (con $a_{1} = 0$):
\begin{align*}
    y_{\mbox{par}} = a_{0} \bigg[ 1 - \dfrac{n (n - 1)}{2!} \, x^{2} + \dfrac{n (n - 2)(n + 1)(n + 3)}{4!} \, x^{4} + \ldots \bigg]
\end{align*}
\end{itemize}
\seti
\end{enumerate}
\end{frame}

\begin{frame}
\frametitle{Por resolver}
\setbeamercolor{item projected}{bg=carmine,fg=white}
\setbeamertemplate{enumerate items}{%
\usebeamercolor[bg]{item projected}%
\raisebox{1.5pt}{\colorbox{bg}{\color{fg}\footnotesize\insertenumlabel}}%
}
\begin{enumerate}[<+->]
\conti
\item Con $k = 1$, 
\begin{itemize}
\item Expresa la relación de recurrencia.
\item Demuestra que la solución se expresa como una serie de potencias pares de $x$ (con $a_{1} = 1$):
\begin{align*}
y_{\mbox{impar}} &= a_{1} \bigg[ x - \dfrac{(n - 1)(n + 2)}{3!} \, x^{3} + \\[0.5em] 
&+ \dfrac{(n - 1)(n - 3)(n + 2)(n + 4)}{5!} \, x^{5} + \ldots \bigg]
\end{align*}
\end{itemize}
\end{enumerate}
\end{frame}

% \subsection{Enunciado 3}

% \begin{frame}
% \frametitle{Enunciado del Ejercicio 3}
% Demuestra que en coordenadas esféricas $(r, \cos \theta, \varphi)$ la función delta de Dirac $\delta (\vb{r}_{1} - \vb{v_{2}})$ es:
% \begin{align*}
% \dfrac{1}{r_{1}^{2}} \delta (r_{1} - r_{2}) \delta (\cos \theta_{1} - \theta_{2}) \delta (\varphi_{1} - \varphi_{2})
% \end{align*}
% \end{frame}

% \begin{frame}
% \frametitle{Punto extra}
% Generaliza el resultado para coordenadas curvilíneas ortogonales $(q_{1}, q_{2}, q_{3})$ con los factores de escala $h_{1}, h_{2}$ y $h_{3}$.
% \end{frame}
\end{document}
