\documentclass[hidelinks,12pt]{article}
\usepackage[left=0.25cm,top=1cm,right=0.25cm,bottom=1cm]{geometry}
%\usepackage[landscape]{geometry}
\textwidth = 20cm
\hoffset = -1cm
\usepackage[utf8]{inputenc}
\usepackage[spanish,es-tabla, es-lcroman]{babel}
\usepackage[autostyle,spanish=mexican]{csquotes}
\usepackage[tbtags]{amsmath}
\usepackage{nccmath}
\usepackage{amsthm}
\usepackage{amssymb}
\usepackage{mathrsfs}
\usepackage{graphicx}
\usepackage{subfig}
\usepackage{caption}
%\usepackage{subcaption}
\usepackage{standalone}
\graphicspath{{Imagenes/}{../Imagenes/}}
\usepackage[outdir=./Imagenes/]{epstopdf}
\usepackage{siunitx}
\usepackage{physics}
\AtBeginDocument{\RenewCommandCopy\qty\SI}
\ExplSyntaxOn
\msg_redirect_name:nnn { siunitx } { physics-pkg } { none }
\ExplSyntaxOff
\usepackage{color}
\usepackage{float}
\usepackage{hyperref}
\usepackage{multicol}
\usepackage{multirow}
%\usepackage{milista}
\usepackage{anyfontsize}
\usepackage{anysize}
%\usepackage{enumerate}
\usepackage[shortlabels]{enumitem}
\usepackage{capt-of}
\usepackage{bm}
\usepackage{mdframed}
\usepackage{relsize}
\usepackage{placeins}
\usepackage{empheq}
\usepackage{cancel}
\usepackage{pdfpages}
\usepackage{wrapfig}
\usepackage[flushleft]{threeparttable}
\usepackage{makecell}
\usepackage{fancyhdr}
\usepackage{tikz}
\usepackage{bigints}
\usepackage{tcolorbox}
\tcbuselibrary{breakable}
\usepackage{scalerel}
\usepackage{pgfplots}
\usepackage{pdflscape}
\usepackage{enumitem}
\pgfplotsset{compat=1.16}
\spanishdecimal{.}
\renewcommand{\baselinestretch}{1.5}
\def\scaleint#1{\vcenter{\hbox{\scaleto[3ex]{\displaystyle\int}{#1}}}}
\def\scaleoint#1{\vcenter{\hbox{\scaleto[3ex]{\displaystyle\oint}{#1}}}}
\def\scaleiint#1{\vcenter{\hbox{\scaleto[3ex]{\displaystyle\iint}{#1}}}}
\def\scaleiiint#1{\vcenter{\hbox{\scaleto[3ex]{\displaystyle\iiint}{#1}}}}
\def\bs{\mkern-12mu}

\newcommand{\Cancel}[2][black]{{\color{#1}\cancel{\color{black}#2}}}

% \newcommand{\qed}{\tag*{$\blacksquare$}}
\renewcommand{\qed}{\hfill\blacksquare}

\newcommand{\pderivada}[1]{\ensuremath{{#1}^{\prime}}}
\newcommand{\sderivada}[1]{\ensuremath{{#1}^{\prime \prime}}}
\newcommand{\tderivada}[1]{\ensuremath{{#1}^{\prime \prime \prime}}}
\newcommand{\nderivada}[2]{\ensuremath{{#1}^{(#2)}}}

\title{Lista de ejercicios del Tema 2 \\[0.3em]  \large{Matemáticas Avanzadas de la Física}\vspace{-3ex}}
\author{M. en C. Gustavo Contreras Mayén}
\date{ }

\begin{document}
\vspace{-4cm}
\maketitle

\fontsize{14}{14}\selectfont

\textbf{Indicaciones: } Se te pide gentilmente que resuelvas de manera detallada, clara y ordenada los siguientes ejercicios, el puntaje que otorga cada enunciado es de \textbf{1 punto}. En caso de que requieras apoyarte en alguna propiedad, si fue vista en clase, solo indícalo, pero si esa propiedad aunque esté relacionada al ejercicio y no se haya mencionado en clase, habrá que demostrarla debidamente.

\begin{enumerate}
\item Demuestra que la ecuación de Helmholtz:
\begin{align*}
\laplacian \psi + k^{2} \: \psi = 0
\end{align*}
es separable en coordenadas cilíndricas circulares si $k^{2}$ se generaliza como:
\begin{align*}
k^{2} + f(\rho) + \left( \dfrac{1}{\rho^{2}} \right) \: g(\varphi) + h(z)
\end{align*}
es decir, la ecuación de Helmholtz es:
\begin{align*}
\laplacian \psi + \left( k^{2} + f(\rho) + \left( \dfrac{1}{\rho^{2}} \right) \: g(\varphi) + h(z) \right) \, \psi = 0
\end{align*}
\item Con la ecuación de onda:
\begin{align*}
\laplacian{\psi} - \dfrac{1}{c^{2}} \, \pdv[2]{\psi}{t} = 0
\end{align*}
Demuestra que:
\begin{enumerate}[a)]
\item Es separable en el caso unidimensional.
\item La solución temporal es de la forma:
\begin{align*}
T (t) = A \, \cos \omega t + B \sin \omega t
\end{align*}
\item Que la solución espacial satisface la EDP:
\begin{align*}
\laplacian{R} + k^{2} \, R = 0
\end{align*}
\end{enumerate}
\item Determina los puntos singulares de las siguientes ED, clasifica cada punto singular en regular o irregular.
\begin{enumerate}[label=\roman*)]
\item $x^{3} \, \sderivada{y} + 4 \, x^{2} \, \pderivada{y} + 3 \, y = 0$
\item $x \, \sderivada{y} - (x + 3)^{-2} \, y = 0$
\item $(x^{2} - 9)^{2} \, \sderivada{y} + (x + 3) \, \pderivada{y} + 2 \, y = 0$
\item $\sderivada{y} - \dfrac{1}{x} \, \pderivada{y} + \dfrac{1}{(x - 1)^{3}} \, y = 0$
\end{enumerate}
\item Resuelve las siguientes ED con el método de Frobenius:
\begin{enumerate}[label=\Roman*)]
\item $2 \, x \, \sderivada{y} - \pderivada{y} + 2 \, y = 0$
\item $2 \, x \, \sderivada{y} + 5 \, \pderivada{y} + x \, y = 0$
\item $x (x - 1) \, \sderivada{y} + 3 \, \pderivada{y} - 2 \, y = 0$
\item $\sderivada{y} - \dfrac{3}{x} \, \pderivada{y} - 2 \, y = 0$
\end{enumerate}
\item Si el Wronskiano de dos funciones $y_{1}$ e $y_{2}$ es igual a cero, demuestra mediante integración directa que:
\begin{align*}
y_{1} = c \, y_{2}
\end{align*}
Considera la suposición de que las funciones tienen derivadas continuas y que al menos una de las funciones no desaparece en e intervalo bajo consideración.
\item Considera las siguientes funciones:
\begin{align*}
\varphi_{1} =  x \hspace{2cm} \varphi_{2} = \abs{x} = x \, \, \mbox{sgn} \, x
\end{align*}
Donde la función \textit{sgn} es precisamente el signo de $x$. Como:
\begin{align*}
\pderivada{\varphi}_{1} &=  1 \hspace{2cm} \pderivada{\varphi}_{2} = \mbox{sgn} \, x \\
&\Rightarrow W(\varphi_{1}, \varphi_{2}) = 0
\end{align*}
para cualquier intervalo, incluyendo $[-1, +1]$. Comprueba si la cancelación del Wronskiano en $[-1, +1]$ demustra que $\varphi_{1}$ y $\varphi_{2}$ son linealmente independientes.
\par
Evidentemente que no lo son. ¿En dónde está el error? (Tip: gráfica las dos funciones en el intervalo para visualizar las funciones)
\item Considerando que una solución de:
\begin{align*}
\sderivada{R} + \dfrac{1}{r} \, \pderivada{R} - \dfrac{m^{2}}{r^{2}} \, R = 0
\end{align*}
es $R = r^{m}$. Demuestra que la ecuación:
\begin{align}
\setlength{\fboxsep}{2\fboxsep}\boxed{y_{2} (x) =  y_{1} \: (x) \scaleint{6ex}^{x} \dfrac{\exp \left[ \displaystyle - \scaleint{6ex}^{x_{2}} P (x_{1}) \: \dd{x_{1}} \right]}{[y_{1} (x_{2})]^{2}} \dd{x_{2}}}
\label{eq:ecuacion_09_127}
\end{align}
predice una segunda solución: $R = r^{-m}$
\item Una solución para la ecuación diferencial de Hermite:
\begin{align*}
\sderivada{y} - 2 \, x \, \pderivada{y} + 2 \, \alpha \, y = 0
\end{align*}
para $\alpha = 0$ es $y_{1} = 1$. Encuentra una segunda solución $y_{2} (x)$, usando la ecuación (\ref{eq:ecuacion_09_127})
\end{enumerate}


\end{document}