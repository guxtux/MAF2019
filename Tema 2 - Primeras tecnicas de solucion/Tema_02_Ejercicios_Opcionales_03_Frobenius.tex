\documentclass[12pt]{article}
\usepackage[left=0.25cm,top=1cm,right=0.25cm,bottom=1cm]{geometry}
\textwidth = 20cm
\hoffset = -1cm
\usepackage[utf8]{inputenc}
\usepackage[spanish,es-tabla]{babel}
\usepackage[autostyle,spanish=mexican]{csquotes}
\usepackage[tbtags]{amsmath}
\usepackage{nccmath}
\usepackage{amsthm}
\usepackage{amssymb}
\usepackage{graphicx}
\usepackage{standalone}
\usepackage[outdir=./]{epstopdf}
\usepackage{siunitx}
\usepackage{physics}
\usepackage{color}
\usepackage{float}
\usepackage{multicol}
%\usepackage{milista}
\usepackage{enumitem}
\usepackage{anyfontsize}
\usepackage{anysize}
\usepackage{enumitem}
\usepackage{capt-of}
\usepackage{bm}
\usepackage{relsize}
\usepackage{placeins}
\usepackage{empheq}
\usepackage{cancel}
\usepackage{wrapfig}
\spanishdecimal{.}
\renewcommand{\baselinestretch}{1.5} 
\renewcommand\labelenumii{\theenumi.{\arabic{enumii}}}
\newcommand{\ptilde}[1]{\ensuremath{{#1}^{\prime}}}
\newcommand{\stilde}[1]{\ensuremath{{#1}^{\prime \prime}}}
\newcommand{\ttilde}[1]{\ensuremath{{#1}^{\prime \prime \prime}}}
\newcommand{\ntilde}[2]{\ensuremath{{#1}^{(#2)}}}


\title{Ejercicios opcionales \\[0.3em]  \large{Material 3 - Método de Frobenius} \vspace{-3ex}}
\author{M. en C. Gustavo Contreras Mayén}
\date{ }

\begin{document}
\vspace{-4cm}
\maketitle
\fontsize{14}{14}\selectfont

%Ref. Arfken (6th Ed.) 9.5.5

\noindent
\textbf{Ejercicio opcional (5).} Determina la naturaleza de los puntos singulares y resuelve la ecuación\footnote{Conviene que se le llame por un nombre a cada ecuación diferencial, de esa manera estaremos identificando la expresión que posteriormente abordaremos las soluciones y conoceremos que nos conducirán a lo que denominaremos como funciones especiales.} de Legendre:
\begin{align*}
(1 -x^{2}) \, \stilde{y} -  2 \, x \, \ptilde{y} + n(n + 1) \, y = 0
\end{align*}
\begin{enumerate}
\item Comprueba que la ecuación de índices es: $k \, (k - 1) = 0$
\item Con $k = 0$, 
\begin{enumerate}
\item Expresa la relación de recurrencia.
\item Demuestra que la solución se expresa como una serie de potencias pares de $x$ (con $a_{1} = 0$):
\begin{align*}
    y_{\mbox{par}} = a_{0} \bigg[ 1 - \dfrac{n (n - 1)}{2!} \, x^{2} + \dfrac{n (n - 2)(n + 1)(n + 3)}{4!} \, x^{4} + \ldots \bigg]
\end{align*}
\end{enumerate}
\item Con $k = 1$, 
\begin{enumerate}
\item Expresa la relación de recurrencia.
\item Demuestra que la solución se expresa como una serie de potencias pares de $x$ (con $a_{1} = 1$):
\begin{align*}
y_{\mbox{impar}} &= a_{1} \bigg[ x - \dfrac{(n - 1)(n + 2)}{3!} \, x^{3} + \\[0.5em] 
&+ \dfrac{(n - 1)(n - 3)(n + 2)(n + 4)}{5!} \, x^{5} + \ldots \bigg]
\end{align*}
\end{enumerate}
\end{enumerate}

\textbf{Ejercicio opcional (5).} Continuamos con la ecuación de Legendre y los resultados obtenidos en el Ejercicio anterior.

\begin{enumerate}[resume]
\item Demuestra que ambas soluciones: $y_{\mbox{par}}$ e $y_{\mbox{impar}}$, divergen para $x = \pm 1$, \textbf{si la serie continua hacia el infinito}.
\item Demuestra que mediante una elección apropiada de $n$, una serie puede convertirse en un polinomio, evitando así la catástrofe de la divergencia. En mecánica cuántica, esta restricción de $n$ a valores enteros corresponde a la \textbf{cuantización del momento angular}.
\end{enumerate}

\end{document}