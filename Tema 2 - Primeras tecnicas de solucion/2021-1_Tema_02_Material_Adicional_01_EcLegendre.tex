
\documentclass[12pt]{article}
\usepackage[left=0.25cm,top=1cm,right=0.25cm,bottom=1cm]{geometry}
\textwidth = 20cm
\hoffset = -1cm
\usepackage[utf8]{inputenc}
\usepackage[spanish,es-tabla]{babel}
\usepackage[autostyle,spanish=mexican]{csquotes}
\usepackage[tbtags]{amsmath}
\usepackage{nccmath}
\usepackage{amsthm}
\usepackage{amssymb}
\usepackage{graphicx}
\usepackage{standalone}
\usepackage[outdir=./]{epstopdf}
\usepackage{siunitx}
\usepackage{physics}
\usepackage{color}
\usepackage{float}
\usepackage{multicol}
%\usepackage{milista}
\usepackage{enumitem}
\usepackage{anyfontsize}
\usepackage{anysize}
\usepackage{enumitem}
\usepackage{capt-of}
\usepackage{bm}
\usepackage{relsize}
\usepackage{placeins}
\usepackage{empheq}
\usepackage{cancel}
\usepackage{wrapfig}
\spanishdecimal{.}
\renewcommand{\baselinestretch}{1.5} 
\renewcommand\labelenumii{\theenumi.{\arabic{enumii}}}
\newcommand{\ptilde}[1]{\ensuremath{{#1}^{\prime}}}
\newcommand{\stilde}[1]{\ensuremath{{#1}^{\prime \prime}}}
\newcommand{\ttilde}[1]{\ensuremath{{#1}^{\prime \prime \prime}}}
\newcommand{\ntilde}[2]{\ensuremath{{#1}^{(#2)}}}


\title{Ecuación de Legendre\\ \large{Matemáticas Avanzadas de la Física}\vspace{-3ex}}
\author{M. en C. Abraham Lima Buendía}
\date{ }
\begin{document}
\vspace{-4cm}
\maketitle
\fontsize{14}{14}\selectfont
\section{Ecuación de Legendre.}

En el ejercicio de separación de variables, en un campo central de mecánica cuántica, encontramos que la parte angular del Hamiltoniano está determinada por la ecuación:
\begin{align*}
\left[ \dfrac{1}{\sin \theta} \, \pdv{\theta} \sin \theta \, \pdv{\theta} + \dfrac{1}{\sin^{2} \theta} \, \pdv[2]{\varphi}  \right] \, Y (\theta, \varphi) = - C_{1} \, Y(\theta, \varphi)
\end{align*}
posteriormente se verá que el valor de $C_{1} = \ell (\ell +1)$ con $\ell$ un entero.
\par
Aplicando el método de separación de variables, tomamos
\begin{align*}
Y(\theta, \varphi) = f(\theta) \, g(\varphi)
\end{align*}
con lo cual tenemos
\begin{align*}
\left[ \dfrac{g(\varphi)}{\sin \theta} \, \pdv{\theta} \sin \theta \, \pdv{f(\theta)}{\theta} + \dfrac{f(\theta)}{\sin^{2} \theta} \, \pdv[2]{g(\varphi}{\varphi}  \right] \, Y (\theta, \varphi) = -  \ell (\ell + 1)\, f(\theta) \, g(\varphi)
\end{align*}
por lo que
\begin{align*}
\dfrac{1}{\sin \theta} \, \dv{\theta} \sin \theta \, \dv{f(\theta)}{\theta} + \dfrac{f(\theta)}{\sin^{2} \theta} \left[ \dfrac{1}{g(\varphi)} \, \dv[2]{\varphi} \, g(\varphi) \right] + \ell (\ell + 1) \, f(\theta) = 0
\end{align*}
Al ordenar los términos que corresponden a una sola variable, tendremos que:
\begin{align*}
\dfrac{\sin \theta}{f(\theta)} \, \dv{\theta} \sin \theta \, \dv{f(\theta)}{\theta} + \ell (\ell + 1) \, \sin^{2}(\theta) = - \dfrac{1}{g(\varphi)} \, \dv[2]{\varphi} \, g(\varphi) = C_{2} 
\end{align*}
donde $C_{2}$ es la constante de separación.
\par
Entonces obtenemos dos ecuaciones diferenciales ordinarias:
\begin{align*}
\addtolength{\fboxsep}{5pt}\boxed{
\dfrac{1}{\sin^{2} \theta} \, \dv{\theta} \sin \theta \, \dv{f(\theta)}{\theta} - \dfrac{C_{2}}{\sin^{2} \theta} \, f(\theta) + \ell (\ell + 1) \, f(\theta) = 0}
\end{align*}
y la otra ecuación es
\begin{align*}
\dfrac{1}{g(\varphi)} \, \dv[2]{\varphi} \, g(\varphi) = C_{2} 
\end{align*}
Entonces
\begin{align*}
\addtolength{\fboxsep}{5pt}\boxed{
\dv[2]{\varphi} \, g(\varphi) = - C_{2} \, g(\varphi)}
\end{align*}
Resolviendo para la ecuación que involucra la variable $\varphi$, se tiene que:
\begin{align*}
g(\varphi) = e^{\pm \sqrt{c_{2}} \, \varphi}
\end{align*}
Para esta variable, se necesita de una condición periódica sobre el ángulo azimutal, es decir:
\begin{align*}
g(\varphi +  2 \, \pi) = g (\varphi)
\end{align*}
entonces
\begin{align*}
e^{\pm \sqrt{c_{2}} \, (\varphi + 2 \pi)} = e^{\pm \sqrt{c_{2}} \, \varphi} \hspace{0.3cm} \Longrightarrow \hspace{0.3cm} e^{\pm 2 \pi \, \sqrt{c_{2}}} = 1
\end{align*}
que será válido si y sólo si $\sqrt{c_{2}}$ es un número entero.

Así la ecuación para el ángulo polar es:
\begin{align*}
\dfrac{1}{\sin \theta} \, \dv{\theta} \sin \theta \, \dv{f(\theta)}{\theta} - \dfrac{m^{2}}{\sin^{2} \theta} \, f(\theta) + \ell (\ell + 1) \, f(\theta) = 0
\end{align*}
Esta ecuación es conocida como la \emph{ecuación diferencial de Legendre}.

Cuando $m = 0$, se le conoce como \emph{ecuación ordinaria de Legendre}.
\begin{align*}
\dfrac{1}{\sin \theta} \, \dv{\theta} \sin \theta \, \dv{f(\theta)}{\theta} + \ell (\ell + 1) \, f(\theta) = 0
\end{align*}
Resolveremos esta ecuación mediante el siguiente cambio de variable con $x = \cos \theta$:
\begin{align*}
\dv{\theta} = \dv{x} \, \dv{x}{\theta} = - \sin \theta \, \dv{x} = - \sqrt{1 - x^{2}} \, \dv{x}
\end{align*}
así tenemos que:
\begin{align*}
\dfrac{1}{\sin \theta} \left[ + \sqrt{1 -x^{2}} \, \dv{x} (1 - x^{2})^{1/2} \, (1 - x^{2})^{1/2} \, \dv{x} f(x) \right] + \ell (\ell + 1) \, f(x) = 0
\end{align*}
entonces
\begin{align*}
\dv{x} (1 - x^{2}) \, \dv{x} f(x) + \ell (\ell + 1) \, f(x) = 0
\end{align*}
que de manera equivalente es:
\begin{align*}
\addtolength{\fboxsep}{5pt}\boxed{
(1 - x^{2}) \, \dv[2]{x} f(x) - 2 \, x \, \dv{x} f(x) + \ell (\ell + 1) \, f(x) = 0}
\end{align*}
Reescribimos la ecuación de la siguiente forma:
\begin{align*}
\dv[2]{x} f(x) - \dfrac{2 \, x}{(1 - x)(1 + x)} \, \dv{x} f(x) + \dfrac{\ell (\ell + 1)}{(1 - x^{2})} \, f(x) = 0
\end{align*}
Revisando esta ecuación, notamos que tiene dos puntos singulares regulares en $x = \pm 1$, desarrollamos una solución en series de potencias alrededor del origen, por esta razón, tomamos una solución de la forma:
\begin{align*}
f(x) = \sum_{m=0}^{\infty} a_{m} \, x^{m}
\end{align*}
que al sustituir en la ecuación diferencial, nos devuelve:
\begin{align*}
(1 - x^{2}) \sum_{m=2}^{\infty} a_{m} \, m( m - 1)  x^{m-2} -  2 x \, \sum_{m=1}^{\infty} a_{m} \, m \, x^{m-1} + \ell (\ell + 1) \, \sum_{m=0}^{\infty} a_{m} \, x^{m} = 0
\end{align*}
desarrollando tenemos
\begin{align*}
\sum_{m=2}^{\infty} &a_{m} \, m (m - 1) \, x^{m-2} + \sum_{m=2}^{\infty} a_{m} \, m (m - 1) \, x^{m} + \\[0.5em]
&- 2 \sum_{m=1}^{\infty} a_{m} \, m \, x^{m} + \ell (\ell + 1) \sum_{m=0}^{\infty} a_{m} \, x^{m} = 0
\end{align*}
Ahora ordenamos todos los sumando para iniciar en $m = 0$
\begin{align*}
\sum_{m=0}^{\infty} &a_{m+2} (m + 2) (m + 1) \, x^{m} + \sum_{m=0}^{\infty} a_{m} \, m (m - 1) \, x^{m} + \\[0.5em]
&- 2 \sum_{m=0}^{\infty} a_{m} \, m \, x^{m} + \ell (\ell + 1) \sum_{m=0}^{\infty} a_{m} \, x^{m} = 0
\end{align*}
Entonces pasamos a:
\begin{align*}
\sum_{m=0}^{\infty} a_{m+2} (m + 2) (m + 1) \, x^{m} + \sum_{m=0}^{\infty} a_{m} \underbrace{\big[ \ell (\ell + 1) - m (m + 1) \big]}_{\substack{\text{esta serie agrupa los últimos tres} \\ \text{sumandos de la ecuación anterior}}} \, x^{m} = 0
\end{align*}çpor tanto, es posible obtener la relación de recurrencia:
\begin{align*}
a_{m+2} = \dfrac{a_{m} \big[ - \ell (\ell + 1) + m (m + 1) \big]}{(m + 2)(m + 1)}
\end{align*}
\textbf{Observaciones:}
\begin{enumerate}
\item La relación de recurrencia define una paridad en la solución.
\item Existe un valor $m$ a partir del cual la serie se rompe, llevando con ello a una solución polinomial.
\end{enumerate}
Hay que analizar los siguientes dos casos:
\begin{align*}
\mbox{Caso 1:} \hspace{0.3cm} a_{0} \neq 0 \hspace{0.3cm} \mbox{y} \hspace{0.3cm} a_{1} = 0 \\[0.5em]
\mbox{Caso 2:} \hspace{0.3cm} a_{0} = 0 \hspace{0.3cm} \mbox{y} \hspace{0.3cm} a_{1} \neq 0
\end{align*}
\textbf{Caso 1:} Cuando $a_{0} \neq 0 \hspace{0.3cm} \mbox{y} \hspace{0.3cm} a_{1} = 0$
De la regla de recurrencia:
\begin{align*}
a_{2} &= \dfrac{a_{0} \big[ - \ell (\ell +  1) \big]}{2 \cdot 1} \\[0.5em]
a_{4} &= \dfrac{a_{2} \big[ - \ell (\ell +  1) + 2 \cdot 3 \big]}{4 \cdot 3} = - \dfrac{a_{0} \, \ell (\ell + 1) \big[- \ell (\ell + 1) + 6 \big]}{4!} = \\[0.5em]
&= \dfrac{a_{0} \, ( - \ell (\ell + 1)) ( 6 - \ell (\ell + 1))}{4!} \\[0.5em]
a_{6} &= \dfrac{ a_{4} \big[ - \ell (\ell +  1) + 4 \cdot 5 \big] }{6 \cdot 5} = \\[0.5em]
&= - \dfrac{a_{0} \, \big[ \ell (\ell + 1) \big] \big[ - \ell (\ell + 1) + 6 \big] \big[ -\ell (\ell + 1) + 20 \big]}{4!} \\[0.5em]
\vdots
\end{align*}
\textbf{Caso 2:} Cuando $a_{0} = 0 \hspace{0.3cm} \mbox{y} \hspace{0.3cm} a_{1} \neq 0$:
\begin{align*}
a_{3} &= \dfrac{a_{1} \big[ 1 (2) - \ell (\ell + 1) \big]}{2 \cdot 3} = \dfrac{a_{1} \big[ 2 - \ell (\ell + 1 ) \big]}{3!} \\[0.5em]
a_{5} &= \dfrac{a_{3} \big[ 3 (4) - \ell (\ell + 1) \big]}{5 \cdot 4} = \dfrac{a_{1} \big[ 2 - \ell (\ell + 1 ) \big] \big[ 12 - \ell (\ell + 1) \big]}{5!} \\[0.5em]
a_{7} &= \dfrac{a_{5} \big[ 5 (6) - \ell (\ell + 1) \big]}{7 \cdot 6} = \dfrac{a_{1} \big[ 2 - \ell (\ell + 1 ) \big] \big[ 12 - \ell (\ell + 1) \big] \big[ 30 - \ell (\ell + 1) \big]}{7!} \\[0.5em]
\vdots
\end{align*}
De manera análoga a los polinomios de Hermite, tenemos que calcular las constantes $a_{0}$ y $a_{1}$, mediante la integral de normalización (este paso se justifica en el Tema 3 del curso, \emph{bases completas}):
\begin{align*}
\dfrac{2 \, n + 1}{2} \int_{-1}^{1} P_{m} (x) P_{n} (x) = \delta_{m, n}
\end{align*}
Con este resultado obtendremos lo siguiente:
\begin{align*}
P_{0} &= 1 \\[0.5em]
P_{1} &= x \\[0.5em]
P_{2} &= \dfrac{1}{2} \left( 3 \, x^{2} - 1 \right) \\[0.5em]
P_{3} &= \dfrac{1}{2} \left( 5 \, x^{3} - 3 \, x \right) \\[0.5em]
P_{4} &= \dfrac{1}{8} \left( 35 \, x^{4} - 30 \, x^{2} + 3 \right) \\[0.5em]
\vdots
\end{align*}
\end{document}