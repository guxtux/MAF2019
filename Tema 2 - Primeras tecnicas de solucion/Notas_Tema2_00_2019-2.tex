\documentclass[12pt]{article}
\usepackage[utf8]{inputenc}
\usepackage[spanish,es-lcroman, es-tabla]{babel}
\usepackage[autostyle,spanish=mexican]{csquotes}
\usepackage{amsmath}
\usepackage{amssymb}
\usepackage{nccmath}
\numberwithin{equation}{section}
\usepackage{amsthm}
\usepackage{graphicx}
\usepackage{epstopdf}
\DeclareGraphicsExtensions{.pdf,.png,.jpg,.eps}
\usepackage{color}
\usepackage{float}
\usepackage{multicol}
\usepackage{enumerate}
\usepackage[shortlabels]{enumitem}
\usepackage{anyfontsize}
\usepackage{anysize}
\usepackage{array}
\usepackage{multirow}
\usepackage{enumitem}
\usepackage{cancel}
\usepackage{tikz}
\usepackage{circuitikz}
\usepackage{tikz-3dplot}
\usetikzlibrary{babel}
\usetikzlibrary{shapes}
\usepackage{bm}
\usepackage{mathtools}
\usepackage{esvect}
\usepackage{hyperref}
\usepackage{relsize}
\usepackage{siunitx}
\usepackage{physics}
%\usepackage{biblatex}
\usepackage{standalone}
\usepackage{mathrsfs}
\usepackage{bigints}
\usepackage{bookmark}
\spanishdecimal{.}

\setlist[enumerate]{itemsep=0mm}

\renewcommand{\baselinestretch}{1.5}

\let\oldbibliography\thebibliography

\renewcommand{\thebibliography}[1]{\oldbibliography{#1}

\setlength{\itemsep}{0pt}}
%\marginsize{1.5cm}{1.5cm}{2cm}{2cm}


\newtheorem{defi}{{\it Definición}}[section]
\newtheorem{teo}{{\it Teorema}}[section]
\newtheorem{ejemplo}{{\it Ejemplo}}[section]
\newtheorem{propiedad}{{\it Propiedad}}[section]
\newtheorem{lema}{{\it Lema}}[section]

\title{Tema 2 - Ecuaciones Diferenciales Parciales \\ {\large Matemáticas Avanzadas de la Física}}
\date{ }
\begin{document}
\maketitle
\fontsize{14}{14}\selectfont
%Referencia Sepúlveda - Lecciones de Física Matemática 3.2
\section{Introducción}
Una gran cantidad de situaciones físicas puede ser descrita utilizando ecuaciones diferenciales que incluyen funciones de dos o más variables. Las conocemos como ecuaciones diferenciales parciales pues incluyen derivadas respecto a cada una de las variables.
\par
Se denomina \emph{ecuación diferencial parcial de segundo orden} en las variables independientes $x$ y $y$, a una relación entre la función incógnita $u(x, y)$ y sus derivadas parciales hasta el segundo orden:
\[ F (x, y, u_{x}, u_{y}, u_{xx}, u_{xy}, u_{yy}) = 0 \]
y análogamente para un número mayor de variables independientes.
\par
Una ecuación diferencial parcial es lineal respecto a la derivada de segundo orden si tiene la forma:
\[ A \: u_{xx} + B \: u_{xy} + C \: u_{yy} + F_{1} (x, y, u, u_{x}, u_{y} ) = 0 \]
donde $A$, $B$, $C$ son en general, funciones de $x$ y $y$.
\par
La ecuación diferencialserá lineal, si lo es respecto a la función $u$ y a sus primeras y segundas derivadas:
\[ A \: u_{xx} + B \: u_{xy} + C \: u_{yy} + D \: u_{x} + E \: u_{y} + F \: u = G \]
donde $A$, $B$, $C$, $D$, $E$, $F$, $G$ son en general, funciones de $x$ y $y$. La ecuación es homogénea si $G = 0$.
\section{Clasificación de las EDP}
Las ecuaciones diferenciales parciales (EDP) pueden ser reducidas a formas más simples con un doble propósito:
\begin{enumerate}
\item El primero es clasificar a las EDP con respecto con las condiciones de frontera (CDF).
\item  El segundo es utilizar la técnica de separación de variables, para resolver la ecuación.
\end{enumerate}
Antes de entrar al problema de la reducción de las EDP consideremos el correspondiente problema algebraico: \emph{reducción de las cónicas a su forma canónica}.
\subsection{Las cónicas.}
La expresión:
\[ A \: x^{2} + B \: x \: y + C \: y^{2} + D \: x + E \: y + F = 0 \]
describe una cónica, como puede comprobarse si se realiza una rotación de coordenadas que elimine el término cruzado $x \: y$. Tal procedimiento alinea los ejes coordenados con los ejes de simetría de las cónicas.
\par
Bajo la rotación:
\begin{align*}
x &= x^{\prime} \: \cos \theta - y^{\prime} \: \sin \theta \\
y &= y^{\prime} \: \cos \theta + x^{\prime} \: \sin \theta
\end{align*}
la ecuación de las cónicas toma la forma
\[ A^{\prime} \: x^{\prime \: 2} + B^{\prime} \: x^{\prime} \:  y^{\prime} + C^{\prime} \: y^{\prime \: 2} + D^{\prime} \: x^{\prime} + E^{\prime} \: y^{\prime} + F = 0 \]
donde:
\begin{align*}
A^{\prime} &= A \: \cos^{2} \theta + C \: \sin^{2} \theta + B \: \sin \theta  \: \cos \theta \\
B^{\prime} &= (C - A) \: \sin^{2} \theta + B \: \cos^{2} \theta \\
C^{\prime} &= A \: \sin^{2} \theta + C \: \cos^{2} \theta - B \: \sin \theta \: \cos \theta \\
D^{\prime} &= D \: \cos \theta + E \: \sin \theta \\
E^{\prime} &= -D \: \sin \theta + E \: \cos \theta
\end{align*}
El término $x^{\prime} \: y^{\prime}$ puede eliminarse si se hace $B^{\prime} = 0$, esto ocurre si:
\[ \tan 2 \ \theta = \dfrac{B}{(A - C)} \]
De este modo la ecuación diagonalizada es:
\begin{equation}
A^{\prime} \: x^{\prime \: 2} + C \: y^{\prime \: 2} + D^{\prime} \: x^{\prime} + E^{\prime} \: y^{\prime} + F^{\prime} = 0
\label{eq:ecuacion_03_019}
\end{equation}
Si $A^{\prime} \neq 0$ y $C^{\prime} \neq 0$ esta ecuación es, equivalentemente:
\[ A^{\prime} \left( x^{\prime} + \dfrac{D^{\prime}}{2 \: A^{\prime}} \right)^{2} + C^{\prime} \left( y^{\prime} + \dfrac{E^{\prime}}{2 \: C^{\prime}} \right)^{2} + F^{\prime} = 0 \]
donde:
\[ F^{\prime} = F - \dfrac{D^{\prime \: 2}}{4 \: A^{\prime}} - \dfrac{E^{\prime \: 2}}{4 \: C^{\prime}} \]
Obtenemos así elipses e hipérbolas, cuyas formas canónicas son:
\begin{align*}
A^{\prime} \left( x^{\prime} - x_{0}^{\prime} \right)^{2} + C^{\prime} \left( y^{\prime} - y_{0}^{\prime} \right)^{2} + F^{\prime} = 0
\end{align*}
Las elipses corresponden cuando $A^{\prime} \: C^{\prime} > 0$, y las hipérbolas cuando $A^{\prime} \: C^{\prime} < 0$. Con el caso de las parábolas, correspondiente cuando $A^{\prime} \: C^{\prime} = 0$, se estudia partiendo de la ecuación diagonalizada y sin factorizar (\ref{eq:ecuacion_03_019}).
\subsection{Ecuaciones diferenciales.}
Consideremos las ecuaciones diferenciales en dos variables, con coeficientes constantes, del tipo:
\begin{equation}
A \: u_{xx} + B \: u_{xy} + C \: u_{yy} + D \: u_{x} + E \: u_{y} + F \: u = G(x, y)
\label{eq:ecuacion_03_020}
\end{equation}
donde: $u = u(x, y)$.
\par
Bajo la rotación de coordenadas:
\begin{align*}
x^{\prime} &= x \: \cos \theta + y \: \sin \theta \\
y^{\prime} &= -x \: \sin \theta + y \: \cos \theta
\end{align*}
Se obtiene:
\begin{align*}
\pdv{u}{x} &= \pdv{u}{x^{\prime}} \: \pdv{x^{\prime}}{x} + \pdv{u}{y^{\prime}} \: \pdv{y^{\prime}}{x} = u_{x}^{\prime} \: \cos \theta - u_{y}^{\prime} \: \sin \theta = u_{x} \\[1em]
\pdv[2]{u}{x} &= u_{xx}^{\prime} \: \cos^{2} \theta - 2 \: u_{xy}^{\prime} \: \sin \theta \: \cos \theta + u_{yy}^{\prime} \: \sin^{2} \theta = u_{xx} \\[1em]
\pdv{u}{y} &= u_{x}^{\prime} \: \sin \theta + u_{y}^{\prime} \: \cos \theta = u_{y} \\[1em]
\pdv[2]{u}{y} &= u_{xx}^{\prime} \: \sin^{2} \theta + 2\: u_{xy}^{\prime} \: \sin \theta \: \cos \theta + u_{yy}^{\prime} \: cos^{2} \theta = u_{yy} \\[1em]
\pdv{u}{x}{y} &= u_{xx}^{\prime} \: \sin \theta \: \cos \theta + u_{xy}^{\prime} \: (\cos^{2} \theta - \sin^{2} \theta) - u_{yy}^{\prime} \: \sin \theta \: \cos \theta = u_{xy}
\end{align*}
re-emplazando y factorizando términos en la ecuación (\ref{eq:ecuacion_03_020}), se obtiene:
\begin{align*}
[ A \: \cos^{2} \theta &+ B \: \sin \theta \: \cos \theta + C \: \sin^{2} \theta ] \: u_{xx}^{\prime} + \\
&+ [(C - A) \: \sin^{2} \theta + B \: \cos^{2} \theta ] \: u_{xy}^{\prime} + \\
&+ [A \: \sin^{2} \theta - B \: \sin \theta \: \cos \theta + C \: \cos^{2} \theta ] \: u_{yy}^{\prime} + \\
&+ [ D \: \cos \theta + E \: \sin \theta ] \: u_{x}^{\prime} + \\
&+ [ -D \: \sin \theta + E \: \cos \theta ] \: u_{y}^{\prime} + \\
&+ F \: u^{\prime} =  \\
&= G^{\prime}
\end{align*}
o de manera equivalente
\begin{align*}
A^{\prime} \: u_{xx}^{\prime} + B^{\prime} \: u_{xy}^{\prime} + C^{\prime} \: u_{yy} + D^{\prime} \: u_{x}^{\prime} + E^{\prime} \: u_{y}^{\prime} + F \: u^{\prime} = G^{\prime}
\end{align*}
En el nuevo sistema de coordenadas el término mixto $u_{xy}^{\prime}$ desaparece si $B^{\prime} = 0$, es decir si
\begin{equation}
\tan 2 \: \theta = \dfrac{B}{(A - C)}
\label{eq:ecuacion_03_021}
\end{equation}
así la ecuación diferencial resulta ser:
\begin{equation}
A^{\prime} \: u_{xx}^{\prime} + C^{\prime} \: u_{yy}^{\prime} + D^{\prime} \: u_{x}^{\prime} + E^{\prime} \: u_{y}^{\prime} + F \: u^{\prime} = G^{\prime}
\label{eq:ecuacion_03_022}
\end{equation}
En los coeficientes $A^{\prime}$, $C^{\prime}$, $D^{\prime}$, $E^{\prime}$ el ángulo $\theta$ está dado por la ecuación (\ref{eq:ecuacion_03_021}).
\par
La ecuación diferencial (\ref{eq:ecuacion_03_022}), puede factorizarse de la siguiente manera:
\begin{align*}
A^{\prime} \left[ u_{xx}^{\prime} + \dfrac{D^{\prime}}{A^{\prime}} \: u_{x}^{\prime} \right] + C^{\prime} \left[ u_{yy}^{\prime} + \dfrac{E^{\prime}}{C^{\prime}} \: u_{y}^{\prime}  \right] + F \: u^{\prime} = G^{\prime}
\end{align*}
Con $A^{\prime} \neq 0$ y $C^{\prime} \neq 0$.
\par
La ecuación se puede escribir también de la siguiente forma:
\begin{equation}
A^{\prime} \left[ D_{x^\prime} + \dfrac{D^{\prime}}{2 \: A^{\prime}} \right]^{2} \: u^{\prime} + C^{\prime} \left[ D_{y^\prime} + \dfrac{E^{\prime}}{2 \: C^{\prime}} \right] \: u^{\prime} + F^{\prime} \: u^{\prime} - G^{\prime} = 0
\label{eq:ecuacion_03_023}
\end{equation}
donde
\begin{align*}
D_{x^{\prime}} &\equiv \pdv{x^{\prime}} \\
D_{y^{\prime}} &\equiv \pdv{y^{\prime}} \\
F^{\prime} &= F - \dfrac{D^{\prime \: 2}}{4 \: A^{\prime}} - \dfrac{E^{\prime \: 2}}{4 \: C^{\prime}} 
\end{align*}
La ecuación (\ref{eq:ecuacion_03_023}) se asemeja a las cónicas, ya que su forma genérica es
\begin{align*}
A^{\prime} (x - x_{0})^{2} + C^{\prime} (y - y_{0})^{2} + F = 0 
\end{align*}
Así pues, diremos que la ecuación (\ref{eq:ecuacion_03_022}) es de tipo:
\begin{enumerate}
\item Elíptico si $A^{\prime} \: C^{\prime} > 0$,
\item Hiperbólico si $A^{\prime} \: C^{\prime} < 0$, 
\item Parabólico si $A^{\prime} \: C^{\prime} = 0$ (en el último caso el análisis se hace partiendo de la ecuación (\ref{eq:ecuacion_03_022}) para evitar los infinitos que aparecen en la ecuación (\ref{eq:ecuacion_03_023})). 
\end{enumerate}
Si $C^{\prime} = 0$ el tipo parabólico. la ecuación toma la forma:
\begin{align*}
A^{\prime} \left[ D_{x^\prime} + \dfrac{D^{\prime}}{2 \: A^{\prime}} \right]^{2} \: u^{\prime} + E^{\prime} \: u_{y} + F^{\: \prime \prime} \: u^{\prime} - G^{\prime} = 0
\end{align*}
Ahora bien, conviene expresar las condiciones $A^{\prime} \: C^{\prime} > 0$,$A^{\prime} \: C^{\prime} < 0$ y $A^{\prime} \: C^{\prime} = 0$ de tal forma que aparezcan directamente los coeficientes $A, B, C, \ldots . . .$ de la ecuación original (\ref{eq:ecuacion_03_020}). Teniendo en cuenta que:
\begin{align*}
A^{\prime} &= A \: \cos^{2} \theta + B \: \sin \theta \: \cos \theta + C \: \sin^{2} \theta \\
&= \dfrac{1}{2} [ A (1 + \cos 2 \: \theta) + B \sin 2 \: \theta +  C (1 - \cos 2 \: \theta)] \\[1em]
C^{\prime} &= A \: \sin^{2} \theta - B \: \sin \theta \: \cos \theta + C \: \cos^{2} \theta \\
&= \dfrac{1}{2} [ A (1 - \cos 2 \: \theta) - B \sin 2 \: \theta +  C (1 - \cos 2 \: \theta)]
\end{align*}
y que de
\[ \tan 2 \: \theta = \dfrac{B}{(A - C)} \]
se sigue entonces que:
\begin{align}
\begin{aligned}
\sin 2 \: \theta = \dfrac{B}{\sqrt{(A - C)^{2} + B^{2}}} \\
\cos 2 \: \theta = \dfrac{A - C}{\sqrt{(A - C)^{2} + B^{2}}}
\end{aligned}
\label{eq:ecuacion_03_024}
\end{align}
se puede escribir:
\begin{align*}
A^{\prime} &= \dfrac{1}{2} {\left[ (A+ C) +  \sqrt{(A - C)^{2} + B^{2}} \right]} \\
C^{\prime} &= \dfrac{1}{2} {\left[ (A+ C) -  \sqrt{(A - C)^{2} + B^{2}} \right]}
\end{align*}
de donde
\[ A^{\prime} \: C^{\prime} = \dfrac{4 \: A \: C - B^{2}}{4} \]
se sigue entonces:
\begin{enumerate}[label=\alph*)]
\item $A^{\prime} \: C^{\prime} > 0$ si $-B^{2} + 4 \: A \: C > 0$
\item $A^{\prime} \: C^{\prime} < 0$ si $-B^{2} + 4 \: A \: C < 0$
\item $A^{\prime} \: C^{\prime} = 0$ si $-B^{2} + 4 \: A \: C = 0$
\end{enumerate}
En consecuencia: una EDP en dos variables es de tipo:
\begin{enumerate}[label=\roman*)]
\item Elíptico si $-B^{2} + 4 \: A \: C < 0$
\item Hiperbólico si $-B^{2} + 4 \: A \: C > 0$
\item Parabólico si $-B^{2} + 4 \: A \: C = 0$
\end{enumerate}
La ecuación (\ref{eq:ecuacion_03_022}) puede hacerse de manera particular a expresiones bidimensionales de ecuaciones de la física muy conocidas:
\begin{center}
\begin{tabular}{| p{7.3cm} | p{4cm} | }
\hline
\makecell{Condiciones} & \makecell{Ecuación} \\ \hline
\makecell{$A^{\prime} = C^{\prime}$ = 1 \\ $D^{\prime} = E^{\prime} = F = 0$} & \makecell{Poisson} \\ \hline
\makecell{Las anteriores y \\ $G^{\prime} = 0$} & \makecell{Laplace} \\ \hline
\makecell{$A^{\prime} = 1, E = -K$ \\ $C^{\prime} = D^{\prime} =  F = G^{\prime} = 0$ \\ y si además: \\ $y$ representa el tiempo} & \makecell{Difusión} \\ \hline
\makecell{$A^{\prime} = 1, C^{\prime} = -\dfrac{1}{v^{2}}$ \\ $D^{\prime} =  E^{\prime} = F = 0$ \\ y si además \\
$y$ representa el tiempo} & \makecell{Ec. de onda \\ con fuentes} \\ \hline
\end{tabular}
\end{center}
Las ecuaciones de Laplace y Poisson en 2D son de tipo elípticas, la ecuación de ondas es de tipo hiperbólica y la de difusión es de tipo parabólica.
\par
%Arfken
\section{Condiciones de frontera.}
Por lo general, cuando sabemos que un sistema físico en algún momento está sometido a una ley que rige tal sistema, entonces seremos capaces de predecir la evolución de ese sistema. Tales valores iniciales son las condiciones de contorno (de frontera) más comunes que se asocian a las EDO y las EDP. La búsqueda de soluciones que responden a determinados puntos, curvas o superficies corresponde a problemas con condiciones de frontera. Las soluciones normalmente deben de satisfacer determinados condiciones de frontera impuestas (por ejemplo, asintóticas). Estas condiciones de frontera pueden clasificar en tres formas:
\begin{enumerate}
\item \textbf{Condiciones de frontera de Direchlet}. El valor de una función se especifica en la frontera.
\item \textbf{Condiciones de frontera de Neumann}. La derivada normal (gradiente) de una función se especifica en la frontera. En el caso de la electrostática, serían $E_{n}$ y $\sigma$, la densidad de carga superficial.
\item \textbf{Condiciones de frontera de Cauchy}. El valor de una función y la derivada se especifican en la frontera. En la electrostática esto significaría, el potencial $\varphi$, y la componente normal del campo eléctrico $E_{n}$.
\end{enumerate}
%Sepulveda - EDP Cap. 3
Se puede hacer una segunda clasificación de las CDF, en función de su alcance: \emph{generales} y \emph{específicas}.
\subsection*{CDF Generales.}
Una condición de frontera general corresponde a situaciones donde un campo se extiende de un medio a otro. En estos casos no es posible especificar los valores de los campos y/o sus derivadas sobre las superficies de separación (interfases), sino alguna relación entre sus valores a ambos lados.
\par
Para un campo electrostático, por ejemplo, el potencial es continuo a través de la interfase $(\phi_{1} \eval_{S} = \phi_{2} \eval_{S})$ y, si no hay carga superficial en la interfase, la componente normal del vector de desplazamiento es continua a través de la interfase $(\vb{D}_{1} \vdot \vu{n} \eval_{S} = \vb{D}_{2} \vdot \vu{n} \eval_{S})$.
\par
En el caso del campo de temperatura, ésta es continua en la frontera de separación de dos medios ($T_{1} \eval_{S} = T_{2} \eval_{S})$.
\subsection*{CDF Específica.}
Una CDF específica establece los valores de los campos y/o sus derivadas espaciales y/o temporales en las fronteras espaciales (sean ellas superficies o líneas o puntos) y en puntos iniciales en el tiempo. A este tipo pertenecen las condiciones de Dirichlet, Neumann y Cauchy.
\par
Las condiciones de frontera específicas pueden subdividirse en homogéneas e inhomogéneas.
\begin{enumerate}
\item  Una CDF homogénea es del tipo:
\[ \phi \eval_{S} = 0 \hspace{1cm} \mbox{o} \hspace{1cm} \pdv{\phi}{n} \eval_{S} \hspace{1.5cm} \hspace{1cm} \mbox{o} \hspace{1cm} \alpha \: \phi \eval_{S} + \beta \: \pdv{\phi}{n} \eval_{S} = 0 \]
La solución de problemas de tipo Sturm-Liouville, exige este tipo de condiciones.
\item Las CDF inhomogéneas son del tipo:
\[ \phi \eval{S} = f(\vb{r}) \hspace{1cm} \mbox{o} \hspace{1cm}\pdv{\phi}{n} \eval{S} = g(\vb{r})\hspace{1cm} \mbox{o} \hspace{1cm} \psi(\vb{r}, t) \eval_{t=0} = h(\vb{r}) \]
entre otras.
\end{enumerate}
\subsection*{Recapitulación.}
La clasificación canónica anterior apunta hacia la conexión entre EDP y condiciones de frontera (CDF). Generalizando, podemos decir:
\begin{enumerate}[label=\alph*)]
\item Las ecuaciones elípticas satisfacen las CDF de tipo Dirichlet o Newmann o mixtas.
\item Las ecuaciones hiperbólicas satisfacen las CDF de tipo Cauchy.
\item Las ecuaciones parabólicas satisfacen las CDF de tipo de la ecuación de calor.
\end{enumerate}
La clasificación de las EDP es válida aún si los coeficientes $A, B, C,\ldots$ son funciones de $x$ y $y$. Esto implica que la clasificación es válida localmente; así, \emph{la misma ecuación puede ser de diferentes tipos en diferentes puntos}; la clasificación es además invariante en cada punto bajo transformación de coordenadas.
\par
Ejemplos:
\begin{itemize}
\item $u_{xx} + u_{yy} + 3 \: u_{xy} = 0$ es hiperbólica.
\item $u_{xx} + u_{yy} + u_{xy} = 0$ es elíptica.
\item $u_{xx} + u_{yy} + 2 \: u_{xy} = 0$ es parabólica.
\item $u_{xx} - u_{yy} = 0$ es hiperbólica, con $B^{2} - 4 \: A \: C = 4$
\item $u_{xx} + u_{yy} + u = x \: y$ es elíptica, con $B^{2} - 4 \: A \: C = -4$
\item  $u_{xx} + u{x} - u_{y} + u = 0$ es parabólica, con $B^{2}- 4 \: A \: C = 0$
\item $u_{xx} + x \: u_{yy} = 0$ es elíptica para $x > 0$ e hiperbólica para $x < 0$
\end{itemize}
\newpage
\begin{landscape}
\begin{center}
\captionof{table}{Tabla que relaciona el tipo de EDP y las CDF.}
\begin{tabular}{ | c | c | c | c | c |} \hline
\multirow{3}{3.5cm}[-7pt]{\makecell{Condiciones \\ de frontera}} & \multirow{3}{3.5cm}[-7pt]{\makecell{Tipo de \\ superficie}} & \multicolumn{3}{c |}{Tipo de EDP} \\ \cline{3-5}
 & & \makecell{Elíptica} & \makecell{Hiperbólica} & \makecell{Parabólica} \\ \cline{3-5}
 & & \makecell{Laplace, Poisson \\ en $(x,y)$} & \makecell{Ecuación de onda \\ en $(x,t)$} & \makecell{Ecuación de difusión \\ en $(x,t)$} \\ \hline
\multirow{2}*{\textbf{Cauchy}} & \makecell{Superficie \\ abierta} & \makecell{Resultados sin \\ interpretación física} & \makecell{\underline{\textbf{Solución única}} \\ \underline{\textbf{estable}}} & \makecell{Demasiado \\ restrictiva} \\ \cline{2-5}
 & \makecell{Superficie \\ cerrada} & \makecell{Demasiado \\ restrictiva} & \makecell{Demasiado \\ restrictiva} & \makecell{Demasiado \\ restrictiva} \\ \hline
 \multirow{2}*{\textbf{Dirichlet}} & \makecell{Superficie \\ abierta} & \makecell{Insuficiente} & \makecell{Insuficiente}  &\makecell{\underline{\textbf{Solución única}} \\ \underline{\textbf{estable} en una dirección}} \\ \cline{2-5} 
& \makecell{Superficie \\ cerrada} & \makecell{\underline{Solución única} \\ \underline{estable}} & \makecell{Más de \\ una solución} & \makecell{Demasiado \\ restrictiva} \\ \hline
 \multirow{2}*{\textbf{Neumann}} & \makecell{Superficie \\ abierta} & \makecell{Insuficiente} & \makecell{Insuficiente}  &\makecell{\underline{\textbf{Solución única}} \\ \underline{\textbf{estable} en una dirección}} \\ \cline{2-5}
& \makecell{Superficie \\ cerrada} & \makecell{\underline{\textbf{Solución única}} \\ \underline{\textbf{estable}}} & \makecell{Más de \\ una solución} & \makecell{Demasiado \\ restrictiva} \\ \hline
\end{tabular}
\end{center}
\end{landscape}
\section{Separación de variables.}
La propuesta de separar variables en una EDP comienza por asumir que existe una solución, que es el producto de funciones de cada una de las variables:
\[ u(x, y) = A(x) \: B(y) \]
Este método no es de validez general y sólo cabe ensayarlo en ecuaciones lineales y homogéneas. Es aplicable a ecuaciones de diversos tipos, inicialmente veremos su desarrollo en un sistema de coordenadas cartesianas:
\par
Sea la EDP2H:
\[ u_{xx} - u_{y} = 0 \]
Proponemos la solución $u$ dada por:
\[ u = A(x) \: B(y) \]
por lo que entonces:
\[ \dfrac{1}{A} \: \dv[2]{A}{x} = \dfrac{1}{B} \: \dv[2]{B}{y} \]
Si hacemos variar $x$ mientras $y$ permanece fijo, el término de la derecha es una constante, por tanto también el de la izquierda; así, una primera solución para valores reales y positivos de $\alpha$ es:
\begin{align}
\begin{aligned}
\dfrac{1}{A} \: \dv[2]{A}{x} = \alpha^{2} \\[1em]
\dfrac{1}{B} \: \dv{B}{y} = \alpha^{2}
\end{aligned}
\label{eq:ecuacion_03_025}
\end{align}
de modo que
\[ A \propto  \exp(\pm \alpha \: x), \hspace{1cm} B \propto \exp(\alpha^{2} \: y) \]
Por tanto la solución es:
\begin{equation}
u = \left( c_{1} \: \exp(\alpha \: x) + c_{2} \: \exp(-\alpha \: x) \right) \: \exp(\alpha^{2} \: y)
\label{eq:ecuacion_03_026}
 \end{equation}
La constante $\alpha$ se conoce como constante de separación. Como $\alpha$ es real (positivo o negativo), la parte en $x$ de la solución es real y el exponencial en $y$ es creciente.
\par
En vez de de la ec. (\ref{eq:ecuacion_03_025}) podemos escribir (con $\beta$ real)
\[ \dfrac{1}{A} \: \dv[2]{A}{x} = - \beta^{2} \\[1em]
\dfrac{1}{B} \: \dv{B}{y} = -\beta^{2}\]
por lo cual es posible una segunda solución donde la parte en $x$ es compleja mientras aquella en $y$ es exponencial decreciente. Con $\beta > 0$ la solución es:
\begin{equation}
u = \left[ d_{1} \: \exp(i \: \beta \: x) + d_{2} \: \exp(-i \: \beta \: x) \right] \: \exp(-\beta^{2} \: y)
\label{eq:ecuacion_03_027}
\end{equation}
\textbf{Nota: } Las soluciones dadas por las ecs. (\ref{eq:ecuacion_03_026}) y (\ref{eq:ecuacion_03_027}) provienen, de un modo equivalente, de tomar en la ec. (\ref{eq:ecuacion_03_025}): $\alpha = a + i \: b$ con $a$ y $b$ reales. Si $b = 0$ se obtiene la ec. (\ref{eq:ecuacion_03_027}) y si $a = 0$ se obtiene la ec. (\ref{eq:ecuacion_03_026}).
\par
Una tercera solución corresponde cuando $\alpha = 0$ en la ec. (\ref{eq:ecuacion_03_025}), de donde $u = a \: x + b.$
De acuerdo a la teoría general de ecuaciones diferenciales, la solución general ha de expresarse como combinación lineal de las anteriores soluciones. Así pues, con $\alpha > 0$ y $\beta > 0$:
\begin{align*}
u = &\left( c_{1} \: \exp(\alpha \: x) + c_{2} \: \exp(-\alpha \: x) \right) \: \exp(\alpha^{2} \: y) + \\
&+ \left[ d_{1} \: \exp(i \: \beta \: x) + d_{2} \: \exp(-i \: \beta \: x) \right] \: \exp(-\beta^{2} \: y) + \\
&+ a \: x + b
\end{align*}
En principio la solución en la ec. (\ref{eq:ecuacion_03_026}) es válida para todos los valores reales de $\alpha$. Por
tanto la solución general es una combinación lineal (en este caso una integral, pues $\alpha$ es real) de la forma:
\begin{equation}
u(x) = \int_{0}^{\infty} \left[ c(\alpha) \: \exp(\alpha \: x) + d(\alpha) \: \exp(- \alpha \: x) \right] \exp(\alpha^{2} \: y) \dd{\alpha}
\label{eq:ecuacion_03_028}
\end{equation}
o de manera equivalente (nótese los límites de la integral):
\begin{align*} 
u(x) = \int_{-\infty}^{\infty} c(\alpha) \: \exp(\alpha \: x + \alpha^{2} \: y) \dd{x}
\end{align*}
De igual manera, la ecuación (\ref{eq:ecuacion_03_027}) es válida para todos los valores reales de $\beta$, por lo cual:
\begin{equation}
u(x) = \int_{0}^{\infty} \left[ a(\beta) \, \exp(i \, \beta \, x) + b(\beta) \, \exp(- i \, \beta \, x) \right] \, \exp(\beta^{2} \, y) \dd{\beta}
\label{eq:ecuacion_03_029}
\end{equation}
La solución general a la EDP $u_{xx} - u_{y} = 0$, tiene la forma:
\begin{align}
\begin{aligned}
u(x, y) &=  \int_{0}^{\infty} \left[ c(\alpha) \: \exp(\alpha \: x) + d(\alpha) \: \exp(- \alpha \: x) \right] \: \exp(\alpha^{2} \: y) \dd{\alpha} + \\
&+ \int_{0}^{\infty} \left[ a(\beta) \, \exp(i \, \beta \, x) + b(\beta) \: \exp(- i \: \beta \: x) \right] \: \exp(\beta^{2} \: y) \dd{\beta} + \\
&+ a \, x + b
\end{aligned}
\label{eq:ecuacion_03_030}
\end{align}
Es frecuente que $\alpha$ (o $\beta$) tomen valores proporcionales a un entero, en tal caso las integrales en la ec. (\ref{eq:ecuacion_03_030}) se convierten en una suma. Las ecuaciones (\ref{eq:ecuacion_03_026}) y (\ref{eq:ecuacion_03_027}) son utilizables sumando sobre el índice entero $n$.
\subsection{Teoría general cartesiana.}
La forma más general de una EDP2LH en coordenadas cartesianas, tiene la forma:
\[ A(x,y) \: u_{xx} + B(x,y) \: u_{xy} + C(x,y) \: u_{yy} + D(x,y) \: u_{x} + E(x,y) \: u_{y} + F(x,y) \:u = 0 \]
La propuesta de separación de variables dice que: $u(x, y) = X(x) \:Y (y)$. Reemplazando y dividiendo por $X \:Y$ se sigue, después de agrupar:
\begin{align*}
\left( \dfrac{A \, \ddot{X}}{X} + \dfrac{D \, \dot{X}}{X} \right) + \left( \dfrac{B \, \dot{X} \, \dot{Y}}{X \, Y} \right) + \left( \dfrac{C \, \ddot{Y}}{Y} + \dfrac{E \, \dot{Y}}{Y} \right) + F = 0
\end{align*}
Esta ecuación es separable, es decir cada uno de los paréntesis es función sólo de $x$ o de $y$ si:
\begin{enumerate}
\item  El primer paréntesis es función sólo de $x$, lo que se logra si $A = A(x)$ y $D = (x)$.
\item  Si $B = 0$, es decir, si no hay términos cruzados $u_{xy}$. Esto puede lograrse diagonalizando la ecuación diferencial.
\item Si el tercer paréntesis es función sólo de $y$, es decir si $C = C(y)$ y $E = E(y)$.
\item Si $F =F(x)$ o $F =F(y)$ o $F =$ constante.
\end{enumerate}
Así pues, en el caso en que $F = F (x)$, (incluyendo valor constante) tendremos:
\begin{align*}
\dfrac{A(x) \, \ddot{X}}{X} + \dfrac{D(x) \, \dot{X}}{X} + F(x) &= \pm \alpha^{2} \\[1em]
\dfrac{C(y) \, \ddot{Y}}{Y} + \dfrac{E(y) \, \dot{Y}}{Y} &= \mp \alpha^{2}, \hspace{1cm} \mbox{y} \hspace{1cm} B = 0
\end{align*}
o si $F = F(y)$ o es una constante
\begin{align*}
\dfrac{A(x) \, \ddot{X}}{X} + \dfrac{D(x) \, \dot{X}}{X} &= \pm \alpha^{2} \\[1em]
\dfrac{C(y) \, \ddot{Y}}{Y} + \dfrac{E(y) \, \dot{Y}}{Y}  + F(y) &= \mp \alpha^{2}, \hspace{1cm} \mbox{y} \hspace{1cm} B = 0
\end{align*}
Asumimos que $\alpha$ es real y positivo y con dos signos posibles $\pm \alpha$. Los signos alternados $\pm$ significan que si en la ecuación en $x$ se toma el $+$, en $y$ se tomará el $-$, y recíprocamente. Anticipamos que, mediante una diagonalización, siempre se logra que $B$ desaparezca.
\par
La solución que hemos propuesto es en principio válida para todo valor de $\alpha^{2}$ es real (por ello hemos introducido los signos $\pm$, aunque pudimos haber considerado un $\alpha$ real o imaginario puro, y entonces hubiera bastado con $\alpha^{2}$). La solución con $\alpha = 0$ debe ser tomada en cuenta como solución independiente.
\par
Es claro que la separación de variables \emph{sólo puede realizarse si la ecuación diferencial es homogénea}; aunque puede aplicarse a ecuaciones inhomogéneas que sean reducibles a homogéneas.
\par
Así pues, toda ecuación diferencial en dos variables, homogenizable, puede ser escrita, mediante una previa diagonalización, en una forma que no contenga al término $u_{xy}$, y es separable si tiene la forma:
\begin{align*}
A(x) \, u_{xx} + C(y) \, u_{yy} + D(x) \, u_{x} + E(y) \, u_{y} + F \, u = 0
\end{align*}
donde $F$ es una función sólo de $x$ o sólo de $y$.
\par
Si $F = F (x)$ las ecuaciones diferenciales ordinarias originadas por la separación de variables son
\begin{align*}
A(x) \, \ddot{X} +D(x) \, \dot{X} + F(x) \, X \mp \alpha^{2} \, X &= 0 \\
C(y) \, \ddot{Y} + E(y) \, \dot{Y} \pm \alpha^{2} \, Y &= 0
\end{align*}
Dado que $X$ y $Y$ dependen también del parámetro $\alpha$, la solución para $u$ deberá de escribirse en la forma:
\begin{align*}
u(x, y) = X(x, \alpha) \: Y (y, \alpha)
\end{align*}
La solución que hemos propuesto es válida para todo $\alpha$ real: de acuerdo a la teoría de ecuaciones diferenciales, la solución general será la combinación lineal de las soluciones para cada $\alpha$, y deberá involucrar una integral sobre $\alpha$; así pues, la solución general es:
\begin{equation}
u(x, y) = \int_{0}^{\infty} X(x, \alpha) \: Y(y, \alpha) \dd{\alpha}
\label{eq:ecuacion_03_031}
\end{equation}
Los límites de integración que hemos escogido aquí son convencionales y pueden extenderse de ($-\infty)$ a $+\infty)$. En los casos en los que las condiciones de frontera exijan que $\alpha$ sea proporcional a un entero, la ecuación (\ref{eq:ecuacion_03_031}) ha de escribirse:
\begin{equation}
u(x, y) = \sum_{n} X(x, \alpha) \: Y(y, \alpha)
\label{eq:ecuacion_03_032}
\end{equation}
Utilizando la técnica de separación de variables en problemas de la física, resulta equivalente utilizar alguna de las  soluciones anteriores, con una restricción simple: cuando el problema físico no ponga exigencias sobre $\alpha$, la solución deberá en el primer caso considerar una integral sobre $\alpha$, o si el problema exige que $\alpha$ sea proporcional a un entero $n$, deberá hacerse una suma sobre $n$. Esto ya está dicho en la forma integral de la solución, la que, además se transforma en una suma si $\alpha$ es proporcional a un entero.
\par
\textbf{Nota al margen: } Sólo en 11 sistemas de coordenados es posible separar la ecuación de ondas
y la ecuación de Schrödinger (en esta última la separabilidad sólo ocurre para ciertas formas del potencial): cartesianas, cilíndricas, esféricas, cilíndricas elípticas, parabólicas, parabólicas cilíndricas, paraboloidales, esferoidales oblatas, esferoidales prolatas, cónicas, elipsoidales.
\par
La ecuación de Laplace en dos dimensiones es separable en cualquier sistema de coordenadas que sea una transformación conforme del cartesiano, y en tres dimensiones es separable en los 11 sistemas anteriores.
\subsection{Separación de la ecuación de Laplace.}
\subsubsection{Coordenadas cartesianas en 2D.}
La ecuación 
\begin{align*}
u_{xx} + u_{yy} = 0
\end{align*} 
es de tipo elíptico pues $B^{2} - 4 \, A \, C = -4$. En consecuencia debe proveerse el valor de la función $u$ (o de su derivada normal) sobre la superficie.
\par
De acuerdo con la técnica de separación de variables, proponemos una solución del tipo
\begin{align*}
u(x, y) = A(x) \, B(y)
\end{align*}
tal que
\begin{align*}
\dv[2]{A}{x} \, B + A \, \dv[2]{B}{y} = 0  \\[1em]
\dfrac{1}{A} \: \dv[2]{A}{x} \, B + \dfrac{1}{B} \dv[2]{B}{y} = 0
\end{align*}
por lo que
\begin{align*}
\dfrac{1}{A} \, \dv[2]{A}{x} = - \alpha^{2} \\[1em]
\dfrac{1}{B} \, \dv[2]{B}{y} = + \alpha^{2}
\end{align*}
entonces la solución es
\begin{equation}
u(x, y) = (c_{1} \, \exp(i \, \alpha \, x) + c_{2} \, \exp(-i \, \alpha \, x))(c_{3} \, \exp(\alpha \, y)(c_{4} \, \exp(-\alpha \, y))
\label{eq:ecuacion_03_033}
\end{equation}
En vez de exponenciales podemos, equivalentemente, utilizar funciones trigonométricas e hiperbólicas, con lo cual:
\begin{equation}
u(x, y) = (d_{1} \, \cos \alpha \, x + d_{2} \sin \alpha \, x)(d_{3} \, \cosh \alpha \, y + d_{4} \, \sinh \alpha \, y)
\label{eq:ecuacion_03_034}
\end{equation}
En principio, la ecuación de Laplace es satisfecha por esta solución para todos los valores de $\alpha$ positivos (escrita en la forma dada arriba la solución produce redundancias si $\alpha$ toma valores positivos y negativos). Así pues, y de acuerdo a la teoría de las ecuaciones diferenciales, la combinación lineal de las soluciones (hay una para cada $\alpha$) es la solución general. Es obvio que puesto que $\alpha$ es continuo la combinación lineal es una integral sobre $\alpha$. Así pues, podemos escribir:
\begin{align}
\begin{aligned}
u(x,y) = \int_{0}^{\infty} & \left[ A(\alpha) \, \exp(i \, \alpha \, x) + B(\alpha) \, \exp(-i \, \alpha \, x) \right] \cp \\
& \cp \left[  C(\alpha) \, \exp(\alpha \, y) + D(\alpha) \, \exp(-\alpha \, y) \right] \dd{\alpha}
\end{aligned}
\label{eq:ecuacion_03_035}
\end{align}
Puede probarse fácilmente que dos formas equivalentes donde $\alpha$ toma valores positivos y negativos son:
\begin{align*}
u(x, y) &= \int_{-\infty}^{\infty} \left[ A(\alpha) \, \exp(\alpha \, y) + B(\alpha) \, \exp(-\alpha \, y) \right] \, \exp(i \, \alpha \, x) \dd{\alpha} \\[1em]
&= \int_{-\infty}^{\infty} \left[ A(\alpha) \, \exp(\alpha \, x) + B(\alpha) \, \exp(-\alpha \, x) \right] \, \exp(\alpha \, y) \dd{\alpha}
\end{align*}
\par
\textbf{Ejercicio:} Considera la solución a la ecuación de Laplace en el dominio
\[ 0 \leq x \leq L, \hspace{1cm} 0 \leq y \leq \infty \]
con las siguientes condiciones de Dirichlet:
\[ u(0, y) = 0, \hspace{0.7cm} u(L, y) = 0 \hspace{0.7cm} u(x, 0) = f(x) \hspace{0.7cm} u(x, \infty) \to 0 \]
Veamos la solución por partes:
\begin{enumerate}
\item Si $u \to 0$ para $y to \infty$, se tiene entonces que en la ec. (\ref{eq:ecuacion_03_035}): $C(\alpha) = 0$

\end{enumerate}
\end{document}