\documentclass[12pt]{article}
\usepackage[utf8]{inputenc}
\usepackage[spanish,es-lcroman, es-tabla]{babel}
\usepackage[autostyle,spanish=mexican]{csquotes}
\usepackage{amsmath}
\usepackage{amssymb}
\usepackage{nccmath}
\numberwithin{equation}{section}
\usepackage{amsthm}
\usepackage{graphicx}
\usepackage{epstopdf}
\DeclareGraphicsExtensions{.pdf,.png,.jpg,.eps}
\usepackage{color}
\usepackage{float}
\usepackage{multicol}
\usepackage{enumerate}
\usepackage[shortlabels]{enumitem}
\usepackage{anyfontsize}
\usepackage{anysize}
\usepackage{array}
\usepackage{multirow}
\usepackage{enumitem}
\usepackage{cancel}
\usepackage{tikz}
\usepackage{circuitikz}
\usepackage{tikz-3dplot}
\usetikzlibrary{babel}
\usetikzlibrary{shapes}
\usepackage{bm}
\usepackage{mathtools}
\usepackage{esvect}
\usepackage{hyperref}
\usepackage{relsize}
\usepackage{siunitx}
\usepackage{physics}
%\usepackage{biblatex}
\usepackage{standalone}
\usepackage{mathrsfs}
\usepackage{bigints}
\usepackage{bookmark}
\spanishdecimal{.}

\setlist[enumerate]{itemsep=0mm}

\renewcommand{\baselinestretch}{1.5}

\let\oldbibliography\thebibliography

\renewcommand{\thebibliography}[1]{\oldbibliography{#1}

\setlength{\itemsep}{0pt}}
%\marginsize{1.5cm}{1.5cm}{2cm}{2cm}


\newtheorem{defi}{{\it Definición}}[section]
\newtheorem{teo}{{\it Teorema}}[section]
\newtheorem{ejemplo}{{\it Ejemplo}}[section]
\newtheorem{propiedad}{{\it Propiedad}}[section]
\newtheorem{lema}{{\it Lema}}[section]

%\author{M. en C. Gustavo Contreras Mayén. \texttt{curso.fisica.comp@gmail.com}}
\title{Matemáticas Avanzadas de la Física \\ {\large Solución no homogénea}}
\date{ }
\begin{document}
%\renewcommand\theenumii{\arabic{theenumii.enumii}}
\renewcommand\labelenumii{\theenumi.{\arabic{enumii}}}
\maketitle
\fontsize{14}{14}\selectfont
\section{Teorema de Green.}
Si $u$ y $v$ son dos funciones escalares, consideremos las siguientes identidades:
\begin{eqnarray}
\nabla \cdot ( u \nabla v) &=& u \nabla \cdot \nabla v + (\nabla u) \cdot (\nabla v) \\
\nabla \cdot ( v \nabla u) &=& v \nabla \cdot \nabla u + (\nabla v) \cdot (\nabla u)
\end{eqnarray}
Restando la segunda expresión de la primera, luego integramos sobre un volumen (suponemos que $u$, $v$ y sus derivadas son continuas), y usando el teorema de Gauss
\begin{equation}
\int_{S} \mathbf{V} \cdot d \bm{\sigma} = \int_{V} \mathbf{\nabla} \cdot \mathbf{V} d \tau
\end{equation}
Que dice que la integral de superficie de un vector sobre una superficie cerrada, es igual a la integral de volumen de la divergencia de ese vector, integrada sobre el volumen que encierra la superficie.
\\
Se obtiene entonces
\begin{equation}
\int_{V} (u \nabla \cdot \nabla v - v \nabla \cdot \nabla u) d \tau = \int_{S} (u \nabla v - v \nabla u) \cdot d \bm{\sigma}
\end{equation}
Que es el teorema de Green.
\section{Operadores lineales.}
Un operador lineal $\mathcal{L}$ se define como un operador con las siguientes dos propiedades:
\begin{enumerate}
\item $\mathcal{L}(a \psi) = a \mathcal{L} \psi$ con $a$ constante.
\item $\mathcal{L} (\psi_{1} + \psi_{2}) =  \mathcal{L} \psi_{1} + \mathcal{L} \psi_{2}$
\end{enumerate}
\section{Función delta de Dirac.}
La función delta de Dirac se define normalmente por sus propiedades:
\begin{eqnarray}
\delta(x) &=& 0, \hspace{1cm} x \neq 0 \\
\int_{-\infty}^{\infty} \delta (x) dx  &=& 1 \\
\int_{-\infty}^{\infty} f(x) \delta (x) dx &=& f(0) \label{eq:derivada1}
\end{eqnarray}
Donde hemos supuesto que $f(x)$ es continua en $x=0$.
\\
NO es una función en el sentido usual, por que ella tiene que ser la función nula en todos los puntos excepto en cero donde ella vale el infinito!!
\\
Se puede hacer una aproximación de la función delta:
\begin{eqnarray}
\delta_{n} (x) &=& \begin{cases}
0, \hspace{0.5cm} x < - \dfrac{1}{2n} \\
n, \hspace{0.5cm} - \dfrac{1}{2n} < x < \dfrac{1}{2n} \\
0, \hspace{0.5cm} x > \dfrac{1}{2n}
\end{cases} \label{eq:propiedad_integral} \\
\delta_{n} (x) &=& \dfrac{n}{\sqrt{\pi}} \exp(-n^{2} x^{2}) \label{eq:derivada2} \\
\delta_{n} (x) &=& \dfrac{n}{\pi} \dfrac{1}{1 + n^{2} x^{2}} \\
\delta_{n} (x) &=& \dfrac{\sin nx}{\pi x} = \dfrac{1}{2 \pi} \int_{-n}^{n} e^{ixt} dt \label{eq:seriesF}
\end{eqnarray}
Las variaciones nos dan diferentes grados de utilidad:
\begin{enumerate}
\item La ecuación (\ref{eq:propiedad_integral}) es muy útil para encontrar de manera simple, propiedades de integración.
\item Las ecuaciones (\ref{eq:derivada1}) y (\ref{eq:derivada2}) son convenientes para diferenciar, lo que nos llevaría a los polinomios de Hermite.
\item La ecuación (\ref{eq:seriesF}) es bastante útil en el análisis de Fourier y sus aplicaciones en la mecánica cuántica.
\end{enumerate}
Considerando estas aproximaciones, podemos suponer que $f(x)$ es bien portada para valores grandes de $x$.
\\
Para la mayoría de fines en la física, las aproximaciones son bastante buenas, pero desde el punto de vista matemático, el punto es crítico, ya que el límite
\[ \lim_{n \to \infty} \delta_{n} (x) \]
\textbf{no existe}.
\\
Para cubrir esta dificultad, nos apoyamos con la teoría de distribuciones, reconocemos de la ecuación (\ref{eq:derivada1}) la propiedad fundamental, nos enfocamos en ésta más no en $\delta(x)$ en sí.
\\
El conjunto de las ecuaciones anteriores con $n=1,2,3,\ldots$
\end{document}