\documentclass[12pt]{article}
\usepackage[utf8]{inputenc}
\usepackage[spanish,es-lcroman, es-tabla]{babel}
\usepackage[autostyle,spanish=mexican]{csquotes}
\usepackage{amsmath}
\usepackage{amssymb}
\usepackage{nccmath}
\numberwithin{equation}{section}
\usepackage{amsthm}
\usepackage{graphicx}
\usepackage{epstopdf}
\DeclareGraphicsExtensions{.pdf,.png,.jpg,.eps}
\usepackage{color}
\usepackage{float}
\usepackage{multicol}
\usepackage{enumerate}
\usepackage[shortlabels]{enumitem}
\usepackage{anyfontsize}
\usepackage{anysize}
\usepackage{array}
\usepackage{multirow}
\usepackage{enumitem}
\usepackage{cancel}
\usepackage{tikz}
\usepackage{circuitikz}
\usepackage{tikz-3dplot}
\usetikzlibrary{babel}
\usetikzlibrary{shapes}
\usepackage{bm}
\usepackage{mathtools}
\usepackage{esvect}
\usepackage{hyperref}
\usepackage{relsize}
\usepackage{siunitx}
\usepackage{physics}
%\usepackage{biblatex}
\usepackage{standalone}
\usepackage{mathrsfs}
\usepackage{bigints}
\usepackage{bookmark}
\spanishdecimal{.}

\setlist[enumerate]{itemsep=0mm}

\renewcommand{\baselinestretch}{1.5}

\let\oldbibliography\thebibliography

\renewcommand{\thebibliography}[1]{\oldbibliography{#1}

\setlength{\itemsep}{0pt}}
%\marginsize{1.5cm}{1.5cm}{2cm}{2cm}


\newtheorem{defi}{{\it Definición}}[section]
\newtheorem{teo}{{\it Teorema}}[section]
\newtheorem{ejemplo}{{\it Ejemplo}}[section]
\newtheorem{propiedad}{{\it Propiedad}}[section]
\newtheorem{lema}{{\it Lema}}[section]

\usepackage{mathrsfs}
%\usepackage{enumerate}
%\author{M. en C. Gustavo Contreras Mayén. \texttt{curso.fisica.comp@gmail.com}}
\title{Completez de las funciones propias \\ {\large Matemáticas Avanzadas de la Física}}
\date{ }
\begin{document}
%\renewcommand\theenumii{\arabic{theenumii.enumii}}
\renewcommand\labelenumii{\theenumi.{\arabic{enumii}}}
\maketitle
\fontsize{14}{14}\selectfont
\section{Completes de las funciones propias.}
La tercera propiedad importante de un operador Hermitiano es que las funciones propias forman un conjunto completo. Esta completez significa que cualquier función bien portada (en partes pero continua) $F(x)$ se puede aproximar por una serie
\begin{equation}
F(x) = \sum_{n=0}^{\infty} a_{n} \varphi_{n}(x) \label{eq:ecuacion_61}
\end{equation}
con cualquier grado de precisión. El conjunto $\varphi_{n}$ se dice completo, si el límite del error medio cuadrado se anula:
\begin{equation}
lim_{m \to \infty} \int_{a}^{b} \left[ F(x) - \sum_{n=0}^{m} a_{n} \varphi_{n} \right]^{2} w(x) dx = 0
\label{eq:ecuacion_62}
\end{equation}
Técnicamente, esta es una integra de Lebesgue. No necesariamente el error es nulo en $[a,b]$, pero sólo la integral del error al cuadrado debe ser cero.
\\
La convergencia a la media debe compararse con la convergencia uniforme. La convergencia uniforme implica la convergencia a la media, pero de manera inversa no se garantiza, la convergencia a la media es menos restrictiva.
\\
En la ecuación (\ref{eq:ecuacion_62}) no es válida para funciones continuas en piezas, ya que hay un número finito de discontinuidades.
\\
En términos del álgebra lineal, tenemos un espacio lineal, un espacio de funciones. Las funciones linelamente independientes, ortonormales $\varphi_{n}(x)$ forman una base de ese espacio (infinito-dimensional). La ecuación (\ref{eq:ecuacion_61}) es un punto que nos dice que las funciones $\varphi_{n}(x)$ cubre ese espacio lineal. Con un producto punto, el espacio lineal que tenemos, se convierte en un espacio de Hilbert.
\\
En la ecuación (\ref{eq:ecuacion_61}) la expasión de los coeficientes se determina por
\begin{equation}
a_{m} = \int_{a}^{b} F(x) \varphi_{m}(x) w(x) dx \label{eq:ecuacion_64}
\end{equation}
Que se obtiene al multiplicar la ecuación (\ref{eq:ecuacion_61}) por $\varphi_{m} w(x)$ y luego se integra. De la ortogonalidad de las funciones propias $\varphi_{n}(x)$, solo el $m$-término sobrevive, por lo que la ortogonalidad es importante. la ecuación (\ref{eq:ecuacion_64}) puede compararse con el producto interno de vectores.
\\
Para una función conocida $F(x)$,  la ecuación (\ref{eq:ecuacion_64}) devuelve $a_{m}$ como una integral definida que siempre se puede evaluar, ya sea numéricamente si es que no es de manera analítica.
\\
La relación entre funciones propias, conjunto de funciones ortogonales, conjunto de funciones linealmente independientes, unicidad de la representación, se muestra en la siguiente figura.
\subsection{Desigualdad de Bessel}
Si el conjunto de funciones $\varphi_{n} (x)$ no forma un conjunto completo, posiblemente sea por que no se han incluido el número infinito de elementos del conjunto completo,  esto nos guía a la \emph{desigualdad de Bessel}. Consideremos primero un caso finito. Sea $A$ 
\begin{equation}
\mathbf{A} = \mathbf{e}_{1} a_{1} + \mathbf{e}_{2} a_{2} + \ldots + \mathbf{e}_{n} a_{n} 
\label{eq:ecuacion_65}
\end{equation}
en donde $\mathbf{e}_{i}$ es un vector unitario y $a_{i}$ es la correspondiente componente (proyección) de $\mathbf{A}$, esto es
\begin{equation}
a_{i} = \mathbf{A} \cdot \mathbf{e}_{i} \label{eq:ecuacion_66}
\end{equation}
Entonces
\begin{equation}
\left( \mathbf{A} - \sum_{i} \mathbf{e}_{i} a_{i} \right)^{2} \geq 0 \label{eq:ecuacion_67}
\end{equation}
Si sumamos todos los $n$ componentes, la suma se iguala a $\mathbf{A}$ por lo que la ecuación (\ref{eq:ecuacion_65}) se mantiene, pero si la suma, no incluye a todos los $n$ componentes, la desigualdad se mantiene. Expandiendo la ecuación (\ref{eq:ecuacion_67}) y recordando que los vectores unitarios satisfacen la relación de ortogonalidad
\begin{equation}
\mathbf{e}_{i} \cdot \mathbf{e}_{j} =  \delta_{ij} \label{eq:ecuacion_68}
\end{equation}
tenemos
\begin{equation}
A^{2} \geq \sum_{i} a_{i}^{2} \label{eq:ecuacion_69}
\end{equation}
Que es la desigualdad de Bessel.
\\
Para funciones debemos de considerar la integral
\begin{equation}
\int_{a}^{b} \left[ f(x) - \sum_{i} a_{i} \varphi_{i}(x) \right]^{2} w(x) dx \geq 0 \label{eq:ecuacion_70}
\end{equation}
que es el análogo de la ecuación (\ref{eq:ecuacion_67}), haciendo $n \to \infty$ y re-emplazando la suma por la integración. Nuevamente, con el factor de peso $w(x) >0 $, el integrando es no negativo. La integral se anula por la ecaución (\ref{eq:ecuacion_61}) si tenemos un conjunto completo. De otra forma, es positiva. Si expandemos el término al cuadrado obtenemos
\begin{equation}
\int_{a}^{b} [ f(x) ]^{2} w(x) dx - 2 \sum_{i} a_{i} \int_{a}^{b} f(x) \varphi w(x) dx  + \sum_{i} a_{i}^{2} \geq 0
\label{eq:ecuacion_71}
\end{equation}
Usando la ecuación (\ref{eq:ecuacion_64}), tenemos
\begin{equation}
int_{a}^{b} [f(x)]^{2} w(x) dx \geq \sum_{i} a_{i}^{2} \label{eq:ecuacion_72}
\end{equation}
De aquóo que la suma de los cuadrados de la expansión de los coeficientes $a_{i}$ es menor o igual que la integral de peso de $[f(x)]^{2}$, la igualdad se mantiene si y sólo si, la expansión es exacta, esto ocurre si el conjunto de soluciones $\varphi_{n}(x)$ es un conjunto completo.
\subsection{Desigualdad de Schwarz}
La desigualdad de Schwarz se usa comúnmente y es similar a la desigualdad de Bessel. Consideremos la ecuación cuadrática
\begin{equation}
\sum_{i=1}^{n} (a_{i} x + b_{i})^{2} = \sum_{i=1}^{n} a_{i}^{2} \left( x + \frac{b_{i}}{a_{i}} \right)^{2} = 0
\label{eq:ecuacion_73}
\end{equation}
Si $b_{i}/a_{i}$ es constante, $c$, la solución es $x= - c$. 
\\
Si $b_{i}/a_{i}$ no es constante, todos los términos no se anulan simultáneamente para un $x$ real, por lo que la solución debe de ser compleja.
Expandiendo, tenemos que
\begin{equation}
x^{2} \sum_{i}^{n} a_{i}^{2} + 2x \sum_{i}^{n} a_{i} b_{i} + \sum_{i}^{n} b_{i}^{2} = 0
\label{eq:ecuacion_74}
\end{equation}
y como $x$ es complejo (o = $-b_{i}/a_{i}$), la fórmula cuadrática para $x$ conduce a 
\begin{equation}
\left( \sum_{i=1}^{n} a_{i} b_{i} \right)^{2} \leq \left( \sum_{i=1}^{n} a_{i}^{2} \right) \left( \sum_{i=1}^{n} b_{i}^{2} \right)
\label{eq:ecuacion_75}
\end{equation}
la igualdad se mantiene cuando $b_{i}/a_{i}$ es una constante.
\\
Nuevamente, en términos de vectores, tenemos
\begin{equation}
( \mathbf{a} \cdot \mathbf{b} )^{2} =  a^{2} b^{2} \cos^{2} \theta \leq a^{2} b^{2}
\label{eq:ecuacion_76}
\end{equation}
La desigualdad de Schwarz para funciones tiene la expresión
\begin{equation}
\Bigg\vert \int_{a}^{b} f^{*} (x) g(x) dx \Bigg\vert^{2} \leq \int_{a}^{b} f^{*}(x) f(x) dx \int_{a}^{b} g^{*}(x) g(x) dx
\label{eq:ecuacion_77}
\end{equation}
La desigualdad se mantiene si y sólo si $g(x) = \alpha f(x)$, con $\alpha$ una constante. Para probar esta forma de la función de la desigualdad de Schwarz, consideremos la función compleja $\psi(x) = f(x) + \lambda g(x)$ con $\lambda$ una constante compleja.
\\
Las funciones $f(x)$ y $g(x)$ son cualesquiera dos funciones (para las cuales, la integral existe). Multiplicadno por el conjugado complejo y luego integrando, tenemos
\begin{equation}
\int_{a}^{b} \psi^{*} \psi dx \equiv \int_{a}^{b} f^{*} f dx +  \lambda \int{a}^{b} f^{*} g dx + \lambda^{*} \int_{a}^{b} g^{*} g dx \geq 0
\label{eq:ecuacion_78}
\end{equation}
El $\geq =$ aparece ya que $\psi^{*} \psi$ es no negativo, el signo igual $(=)$ se mantiene solo si $\psi(x)$ es idéntico a cero. Nótese que $\lambda$ y $\lambda^{*}$ son linealmente independientes, diferenciamos con respecto a uno de ellos, e igualamos la derivada a cero para minimizar $\int_{a}^{b} \psi^{*} \psi dx$
\begin{equation*}
\dfrac{\partial}{\partial \lambda^{*}} \int_{a}^{b} \psi^{*} \psi = \int_{a}^{b} g^{*} f dx  + \lambda \int_{a}^{b} g^{*} g dx = 0
\end{equation*}
que nos da
\begin{equation}
\lambda = - \dfrac{\int_{a}^{b} g^{*} f dx}{\int_{a}^{b} g^{*} g dx}
\label{eq:ecuacion_79a}
\end{equation}
tomando el conjugado complejo 
\begin{equation}
\lambda^{*} = - \dfrac{\int_{a}^{b} f^{*} g dx}{\int_{a}^{b} g^{*} g dx}
\label{eq:ecuacion_79b}
\end{equation}
sustituyendo esos valores de $\lambda$ y $\lambda^{*}$ en la ecuación (\ref{eq:ecuacion_78}), obtenemos la ecuación (\ref{eq:ecuacion_77}), la desigualdad de Schwarz.
\\
En mecánica cuántica las funciones $f(x)$ y $g(x)$ podrían representar un estado o una configuración de un sistema físico. Entonces la desigualdad e Shwarz garantiza que el producto punto $\int_{a}^{b} f^{*} g(x9 dx$ existe. En algunos textos, la desigualdad de Schwarz es un paso para llegar al principio de incertidumbre de Heinsenberg.
\\
La notación de las funciones de las ecuaciones (\ref{eq:ecuacion_77}) y (\ref{eq:ecuacion_78}) es a veces incómoda, por lo que es común usar una notación distinta:
\[ < f \vert g > \equiv \int_{a}^{b} f^{*}(x) g(x) dx \]
Usando esta notación, entendemos que el rango de integración $[a,b]$ y cualquier función de peso. La desigualdad de Schwarz ahora se representa
\begin{equation}
\vert <f \vert g > \vert^{2} \leq < f \vert f > < g \vert g >
\label{eq:ecuacion_77a}
\end{equation}
Si $g(x)$ es una función propia normalizada, $\varphi_{i}(x)$, la ecuación (\ref{eq:ecuacion_77}) lleva a (con $w(x)=1$)
\begin{equation}
a_{i}^{*} a_{i} \leq \int_{a}^{b} f^{*}(x) f(x) dx 
\label{eq:ecuacion_80}
\end{equation}
Un resultado que se sigue de la ecuación (\ref{eq:ecuacion_72}).
\section{Función Gamma}
Existen al menos tres formas convenientes de definir la función Gamma.
\subsection{Límite infinito (Euler)}
La primera definición, llamada de Euler es
\begin{equation}
\Gamma(z) \equiv \lim_{n \to \infty} \dfrac{1 \cdot 2 \cdot 3 \cdots n}{z (z+1) (z+2) \cdots (z+n)} n^{z}, \hspace{1.5cm} z \neq	0, -1,-2,-3, \ldots
\label{eq:ecuacion_10_1}
\end{equation}
Esta definición de $\Gamma(z)$ es útil para el desarrollo del producto infinito de Weierstrass de $\Gamma (z)$. Sin pérdida de generalidad, $z$ puede ser real o complejo. Re-emplazando $z$ con $z+1$, tenemos
\begin{eqnarray}
\begin{aligned}
\Gamma (z+1) &= \lim_{n \to \infty} \dfrac{1 \cdot 2 \cdot 3 \cdots n}{(z+1) (z+2)(z+3) \cdots (z+n+1)} n^{z+1} \\
&= \lim_{n \to \infty} \dfrac{nz}{z+n+1} \dfrac{1 \cdot 2 \cdot 3 \cdots n}{z (z+1) (z+2)(z+3) \cdots (z+n)} n^{z} \\
&= z \Gamma (z)
\label{eq:ecuacion_10_2}
\end{aligned}
\end{eqnarray}
De la definición
\begin{eqnarray}
\begin{aligned}
\Gamma (1) &= \lim_{n \to \infty} \dfrac{1 \cdot 2 \cdot 3 \cdots n}{1 \cdot 2 \cdot 3 \cdots n(n+1)} n \\
&= 1
\label{eq:ecuacion_10_3}
\end{aligned}
\end{eqnarray}
Usando de la ecuación (\ref{eq:ecuacion_10_2}), tenemos
\begin{eqnarray}
\begin{aligned}
\Gamma (2) &= 1 \\
\Gamma (3) &=  2 \Gamma(2) =  2 \\
\Gamma (n) &= 1 \cdot 2 \cdot 3 \cdots (n-1) =  (n-1)!
\label{eq:ecuacion_10_4}
\end{aligned}
\end{eqnarray}
\subsection{Integral definida (Euler)}
Una segunda definición es la llamada forma de Euler:
\begin{equation}
\Gamma (z) \equiv \int_{0}^{\infty} e^{-t} t^{z-1} dt, \hspace{1.5cm} \mathbb{R}(z) > 0
\label{eq:ecuacion_10_5}
\end{equation}
La restricción en $z$ es necesaria para prevenir la divergencia de la integral. Cuando la función Gamma aparece en problemas de la física, a menudo tiene una variante
\begin{eqnarray}
\Gamma (z) &= 2 \int_{0}^{\infty} e^{-t^{2}} t^{2z-1} dt, \hspace{1.5cm} \mathbb{R}(z) > 0  \label{eq:ecuacion_10_6} \\
\Gamma (z) &=  \int_{0}^{1} \left[ ln\left(\dfrac{1}{t} \right) \right]^{z-1} dt, \hspace{1.5cm} \mathbb{R}(z) > 0 \label{eq:ecuacion_10_7}
\end{eqnarray}
Cuando $z=1/2$, la ecuación (\ref{eq:ecuacion_10_6}) es la función de error gaussiana, que nos devuelve el siguiente resultado interesante:
\begin{equation}
\Gamma (1/2) = \sqrt{\pi}
\label{eq:ecuacion_10_8}
\end{equation}
Para mostrar la equivalencia de esas dos definiciones, las ecuaciones (\ref{eq:ecuacion_10_1}) y (\ref{eq:ecuacion_10_5}) consideremos la función de dos variables
\begin{equation}
F(z,n) = \int_{0}^{n} \left( 1 - \dfrac{t}{n} \right)^{n} t^{z-1} dt, \hspace{1.5cm} \mathbb{R}(z) > 0
\label{eq:ecuacion_10_9}
\end{equation}
con $n$ entero positivo. Ya que
\begin{equation}
\lim_{n \to \infty} \left( 1 - \dfrac{t}{n} \right)^{n} \equiv e^{-t}
\label{eq:ecuacion_10_10}
\end{equation}
de la definición de función exponencial.
\begin{equation}
\lim_{n \to \infty} F(z,n) = F(z,\infty) = \int_{0}^{\infty} e^{-t} t^{z-1} dt \equiv \Gamma (z)
\label{eq:ecuacione_10_11}
\end{equation}
por la ecuación (\ref{eq:ecuacion_10_5}).
\\
Regresado a $F(z,n)$, evaluamos sucesiamente la integral por partes, hacemos de manera conveniente $u = t/n$. Entonces
\begin{equation}
F(z,n) = n^{2} \int_{0}^{1} (1-u)^{n} u^{z-1} du
\label{eq:ecuacione_10_12}
\end{equation}
Integrando por partes
\begin{equation}
\dfrac{F(z,n)}{n^{2}} =  (1-u)^{n} \dfrac{u^{z}}{z} \Bigg\vert_{0}^{1} + \dfrac{n}{z} \int_{0}^{1} (1-u)^{n-1} u^{z} du
\label{eq:ecuacion_10_13}
\end{equation}
Repitiendo esto cada vez con el integrando se anula en ambos extremos, por lo que
\begin{eqnarray}
\begin{aligned}
F(z,n) &= n^{z} \dfrac{n(n-1) \cdots 1}{z(z+1) \cdots (z+n+-1)} \int_{0}^{1} u^{z+n-1} du \\
&= \dfrac{1 \cdot 2 \cdot 3 \cdots n}{z(z+1)(z+2) \cdots (z+n)} n^{z}
\label{eq:ecuacion_10_14}
\end{aligned}
\end{eqnarray}
Que es idéntico con la expresión del lado derecho de la ecuación (\ref{eq:ecuacion_10_1}). De aquí que
\begin{equation}
\lim_{n \to \infty} F(z,n) = F(z,\infty) \equiv \Gamma (z)
\label{eq:ecuacion_10_15}
\end{equation}
\subsection{Producto infinito (Weierstrass)}
La tercera forma es
\begin{equation}
\dfrac{1}{\Gamma (z)} \equiv z e^{\gamma z} \prod_{n=1}^{\infty} \left( 1 + \dfrac{z}{n} \right) e^{-z/n}
\label{eq:ecuacion_10_16}
\end{equation}
donde $\gamma$ es la constante de Euler-Mascheroni
\begin{equation}
\gamma = 0.577216
\label{eq:ecuacion_10_17}
\end{equation}
\subsection{Notación factorial}
El valor de $-1$ en el exponente $z-1$ de la segunda definición (ecuación \ref{eq:ecuacion_10_5}) es una molestia continua, podemos re-escribir
\begin{equation}
\int_{0}^{\infty} e^{-t} t^{z} dt \equiv z!,  \hspace{1.5cm} \mathbb{R}(z) > 0
\label{eq:ecuacion_10_25}
\end{equation}
que define la \emph{función factorial} $z!$. A veces se puede encontrar escrita como
\begin{equation}
\prod(z) = z!
\end{equation}
donde la notación $\Gamma$ se debe a Legendre. Ka función factorial de la ecuación (\ref{eq:ecuacion_10_25}) está relacionada con la función Gamma
\[ \Gamma (z) = (z-1)! \]
o
\begin{equation}
\Gamma (z+1) = z!
\label{eq:ecuacion_10_27}
\end{equation}
Si $z=n$, un entero positivo (ecuación \ref{eq:ecuacion_10_4}), muestra que
\begin{equation}
z! = n! = 1 \cdot 2 \cdot 3 \cdots n
\label{eq:ecuacion_10_28}
\end{equation}
\subsection{Notación doble factorial}
En varios problemas de la física, en particular con los polinomios de Legendra, encontraremos productos de enteros impares positivos y de pares positivos, por conveniencia, definimos el doble factorial:
\begin{eqnarray}
\begin{aligned}
1 \cdot 3 \cdot 5 \cdots (2n+1) &=& (2n+1) !! \\
2 \cdot 4 \cdot 6 \cdots (2n) &=& (2n) !!
\end{aligned}
\end{eqnarray}
Que están relacionado con la función factorial
\begin{equation}
(2n)!! =  2^{n} n! \text{ y } (2n+1)!! = \dfrac{(2n+1)!}{2^{n}n!}
\end{equation}

\end{document}