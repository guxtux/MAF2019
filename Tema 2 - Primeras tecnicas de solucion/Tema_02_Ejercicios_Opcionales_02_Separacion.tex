\documentclass[hidelinks,12pt]{article}
\usepackage[left=0.25cm,top=1cm,right=0.25cm,bottom=1cm]{geometry}
%\usepackage[landscape]{geometry}
\textwidth = 20cm
\hoffset = -1cm
\usepackage[utf8]{inputenc}
\usepackage[spanish,es-tabla]{babel}
\usepackage[autostyle,spanish=mexican]{csquotes}
\usepackage[tbtags]{amsmath}
\usepackage{nccmath}
\usepackage{amsthm}
\usepackage{amssymb}
\usepackage{mathrsfs}
\usepackage{graphicx}
\usepackage{subfig}
\usepackage{standalone}
\usepackage[outdir=./Imagenes/]{epstopdf}
\usepackage{siunitx}
\usepackage{physics}
\usepackage{color}
\usepackage{float}
\usepackage{hyperref}
\usepackage{multicol}
%\usepackage{milista}
\usepackage{anyfontsize}
\usepackage{anysize}
%\usepackage{enumerate}
\usepackage[shortlabels]{enumitem}
\usepackage{capt-of}
\usepackage{bm}
\usepackage{relsize}
\usepackage{placeins}
\usepackage{empheq}
\usepackage{cancel}
\usepackage{wrapfig}
\usepackage[flushleft]{threeparttable}
\usepackage{makecell}
\usepackage{fancyhdr}
\usepackage{tikz}
\usepackage{bigints}
\usepackage{scalerel}
\usepackage{pgfplots}
\usepackage{pdflscape}
\pgfplotsset{compat=1.16}
\spanishdecimal{.}
\renewcommand{\baselinestretch}{1.5} 
\renewcommand\labelenumii{\theenumi.{\arabic{enumii}})}
\newcommand{\ptilde}[1]{\ensuremath{{#1}^{\prime}}}
\newcommand{\stilde}[1]{\ensuremath{{#1}^{\prime \prime}}}
\newcommand{\ttilde}[1]{\ensuremath{{#1}^{\prime \prime \prime}}}
\newcommand{\ntilde}[2]{\ensuremath{{#1}^{(#2)}}}

\newtheorem{defi}{{\it Definición}}[section]
\newtheorem{teo}{{\it Teorema}}[section]
\newtheorem{ejemplo}{{\it Ejemplo}}[section]
\newtheorem{propiedad}{{\it Propiedad}}[section]
\newtheorem{lema}{{\it Lema}}[section]
\newtheorem{cor}{Corolario}
\newtheorem{ejer}{Ejercicio}[section]

\newlist{milista}{enumerate}{2}
\setlist[milista,1]{label=\arabic*)}
\setlist[milista,2]{label=\arabic{milistai}.\arabic*)}
\newlength{\depthofsumsign}
\setlength{\depthofsumsign}{\depthof{$\sum$}}
\newcommand{\nsum}[1][1.4]{% only for \displaystyle
    \mathop{%
        \raisebox
            {-#1\depthofsumsign+1\depthofsumsign}
            {\scalebox
                {#1}
                {$\displaystyle\sum$}%
            }
    }
}
\def\scaleint#1{\vcenter{\hbox{\scaleto[3ex]{\displaystyle\int}{#1}}}}
\def\bs{\mkern-12mu}


\title{Ejercicios opcionales \\[0.3em]  \large{Material 2 - Separación de variables} \vspace{-3ex}}
\author{M. en C. Gustavo Contreras Mayén}
\date{ }

\begin{document}
\vspace{-4cm}
\maketitle
\fontsize{14}{14}\selectfont

%Ref. Riley (2006) - Mathematical methods. 21.1

\textbf{Ejercicio opcional (3).} Un cubo hecho de material cuya conductividad es $k$, tiene como seis caras los planos $x = \pm a$, $y = \pm a$ y $z = \pm a$, y no contiene fuentes de calor internas. Verifica que la distribución de temperatura:
\begin{align*}
u (x, y, z, t) = A \, \cos \dfrac{\pi \, x}{a} \, \sin \dfrac{\pi \, z}{a} \, \exp \left( - \dfrac{2 \, k \, \pi^{2} \, t}{a^{2}} \right)
\end{align*}
\begin{enumerate}[label=\alph*)]
\item Satisface la correspondiente ecuación de difusión.
\item ¿A través de qué caras hay flujo de calor?
\item ¿Cuál es la dirección y la velocidad del flujo de calor en el punto \hfill \break $\left(\dfrac{3 a}{4}, \dfrac{a}{4}, a \right)$ en el tiempo $t = \dfrac{a^{2}}{(k \, \pi^{2})}$?
\end{enumerate}
%Ref. 21.5
\textbf{Ejercicio opcional (4).} Denotando los tres términos de $\laplacian$ en coordenadas esféricas por $\laplacian_{r}$, $\laplacian_{\theta}$, $\laplacian_{\phi}$, de manera obvia, evalúa $\laplacian{u}_{r}$, etc., para las dos funciones dadas a continuación y verifica que, en cada caso, aunque los términos individuales son no necesariamente cero, su suma $\laplacian{u}$ es cero. Determina los valores correspondientes de $l$ y $m$, las constantes de separación.
\begin{enumerate}[label=\alph*)]
\item $u(r, \theta, \phi) = \left( A \, r^{2} + \dfrac{B}{r^{3}} \right) \, \dfrac{3 \cos^{2} \theta - 1}{2}$
\item $u(r, \theta, \phi) = \left( A \, r^{2} + \dfrac{B}{r^{2}} \right) \, \sin \theta \, \exp(i \, \phi)$
\end{enumerate}
\end{document}