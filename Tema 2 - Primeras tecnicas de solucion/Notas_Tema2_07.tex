\documentclass[12pt]{article}
\usepackage[utf8]{inputenc}
\usepackage[spanish,es-lcroman, es-tabla]{babel}
\usepackage[autostyle,spanish=mexican]{csquotes}
\usepackage{amsmath}
\usepackage{amssymb}
\usepackage{nccmath}
\numberwithin{equation}{section}
\usepackage{amsthm}
\usepackage{graphicx}
\usepackage{epstopdf}
\DeclareGraphicsExtensions{.pdf,.png,.jpg,.eps}
\usepackage{color}
\usepackage{float}
\usepackage{multicol}
\usepackage{enumerate}
\usepackage[shortlabels]{enumitem}
\usepackage{anyfontsize}
\usepackage{anysize}
\usepackage{array}
\usepackage{multirow}
\usepackage{enumitem}
\usepackage{cancel}
\usepackage{tikz}
\usepackage{circuitikz}
\usepackage{tikz-3dplot}
\usetikzlibrary{babel}
\usetikzlibrary{shapes}
\usepackage{bm}
\usepackage{mathtools}
\usepackage{esvect}
\usepackage{hyperref}
\usepackage{relsize}
\usepackage{siunitx}
\usepackage{physics}
%\usepackage{biblatex}
\usepackage{standalone}
\usepackage{mathrsfs}
\usepackage{bigints}
\usepackage{bookmark}
\spanishdecimal{.}

\setlist[enumerate]{itemsep=0mm}

\renewcommand{\baselinestretch}{1.5}

\let\oldbibliography\thebibliography

\renewcommand{\thebibliography}[1]{\oldbibliography{#1}

\setlength{\itemsep}{0pt}}
%\marginsize{1.5cm}{1.5cm}{2cm}{2cm}


\newtheorem{defi}{{\it Definición}}[section]
\newtheorem{teo}{{\it Teorema}}[section]
\newtheorem{ejemplo}{{\it Ejemplo}}[section]
\newtheorem{propiedad}{{\it Propiedad}}[section]
\newtheorem{lema}{{\it Lema}}[section]

\usepackage{mathrsfs}
\usepackage{bigints}
\spanishdecimal{.}
%\usepackage{enumerate}
%\author{M. en C. Gustavo Contreras Mayén. \texttt{curso.fisica.comp@gmail.com}}
\title{Ejercicios \\ {\large Matemáticas Avanzadas de la Física}}
\date{ }
\begin{document}
%\renewcommand\theenumii{\arabic{theenumii.enumii}}
\renewcommand\labelenumii{\theenumi.{\arabic{enumii}}}
\maketitle
\fontsize{14}{14}\selectfont
\begin{enumerate}
\item Una partícula de masa $m$ en el cuadrante $x$ positivo es atraída hacia el origen por una fuerza variable tal que el producto de la magnitud de la fuerza por la distancia al origen, es una constante $k$. La partícula parte del reposo en $x=L$. Calcular el tiempo necesario para que alcance el origen.
\\
\textbf{Solución: } Partimos de Newton-2, $F=ma$, donde la fuerza $F$ y la aceleración $a$ son vectores cuyas componentes son nulas, excepto en el eje $x$. Al igualar las componentes a lo largo del eje $x$, tenemos
\[  - \dfrac{k}{x} = m \dfrac{d^{2} x}{d t^{2}} \]
Intercambiamos los lados de la ecuación, multiplicamos ambos lados por $dx/dt$, para luego integrar
\begin{eqnarray*}
\dfrac{1}{2} m \left( \dfrac{d x}{d t} \right)^{2} &= - k \bigints_{L}^{x} \dfrac{\left(\dfrac{dx}{dt}\right)}{x} dt \\
&= - k ln x + k ln L
\end{eqnarray*}
Checa que la derivada de $(dx/dt)^{2}$ es $2(dx/dt)(d^{2}x/dt^{2})r$

\end{enumerate}

\end{document}
