\documentclass[hidelinks,12pt]{article}
\usepackage[left=0.25cm,top=1cm,right=0.25cm,bottom=1cm]{geometry}
%\usepackage[landscape]{geometry}
\textwidth = 20cm
\hoffset = -1cm
\usepackage[utf8]{inputenc}
\usepackage[spanish,es-tabla]{babel}
\usepackage[autostyle,spanish=mexican]{csquotes}
\usepackage[tbtags]{amsmath}
\usepackage{nccmath}
\usepackage{amsthm}
\usepackage{amssymb}
\usepackage{mathrsfs}
\usepackage{graphicx}
\usepackage{subfig}
\usepackage{standalone}
\usepackage[outdir=./Imagenes/]{epstopdf}
\usepackage{siunitx}
\usepackage{physics}
\usepackage{color}
\usepackage{float}
\usepackage{hyperref}
\usepackage{multicol}
%\usepackage{milista}
\usepackage{anyfontsize}
\usepackage{anysize}
%\usepackage{enumerate}
\usepackage[shortlabels]{enumitem}
\usepackage{capt-of}
\usepackage{bm}
\usepackage{relsize}
\usepackage{placeins}
\usepackage{empheq}
\usepackage{cancel}
\usepackage{wrapfig}
\usepackage[flushleft]{threeparttable}
\usepackage{makecell}
\usepackage{fancyhdr}
\usepackage{tikz}
\usepackage{bigints}
\usepackage{scalerel}
\usepackage{pgfplots}
\usepackage{pdflscape}
\pgfplotsset{compat=1.16}
\spanishdecimal{.}
\renewcommand{\baselinestretch}{1.5} 
\renewcommand\labelenumii{\theenumi.{\arabic{enumii}})}
\newcommand{\ptilde}[1]{\ensuremath{{#1}^{\prime}}}
\newcommand{\stilde}[1]{\ensuremath{{#1}^{\prime \prime}}}
\newcommand{\ttilde}[1]{\ensuremath{{#1}^{\prime \prime \prime}}}
\newcommand{\ntilde}[2]{\ensuremath{{#1}^{(#2)}}}

\newtheorem{defi}{{\it Definición}}[section]
\newtheorem{teo}{{\it Teorema}}[section]
\newtheorem{ejemplo}{{\it Ejemplo}}[section]
\newtheorem{propiedad}{{\it Propiedad}}[section]
\newtheorem{lema}{{\it Lema}}[section]
\newtheorem{cor}{Corolario}
\newtheorem{ejer}{Ejercicio}[section]

\newlist{milista}{enumerate}{2}
\setlist[milista,1]{label=\arabic*)}
\setlist[milista,2]{label=\arabic{milistai}.\arabic*)}
\newlength{\depthofsumsign}
\setlength{\depthofsumsign}{\depthof{$\sum$}}
\newcommand{\nsum}[1][1.4]{% only for \displaystyle
    \mathop{%
        \raisebox
            {-#1\depthofsumsign+1\depthofsumsign}
            {\scalebox
                {#1}
                {$\displaystyle\sum$}%
            }
    }
}
\def\scaleint#1{\vcenter{\hbox{\scaleto[3ex]{\displaystyle\int}{#1}}}}
\def\bs{\mkern-12mu}


\title{Lo verde de la Función de Green \\ \large {Tema 2 - Matemáticas Avanzadas de la Física}\vspace{-3ex}}

\author{M. en C. Gustavo Contreras Mayén}
\date{ }

\pagestyle{fancy}
\fancyhf{}
\rhead{Curso MAF}
\lhead{\leftmark}
\rfoot{\thepage}
\setlength{\headheight}{16pt}%


\begin{document}
\maketitle
\fontsize{14}{14}\selectfont

En 1828, un molinero inglés de Nottingham publicó un ensayo matemático que generó poca respuesta. Sin embargo, el análisis de George Green ha encontrado aplicaciones en áreas que van desde la electrostática clásica hasta la moderna teoría cuántica de campos.
\par
Nottingham, una ciudad atractiva y próspera en las Midlands inglesas, es famosa por su asociación con Robin Hood, cuya estatua se encuentra a la sombra de la muralla del castillo. El Sheriff de Nottingham todavía tiene un papel especial en el gobierno de la ciudad, aunque felizmente ya no siembra el terror en los corazones de los buenos ciudadanos.
\par
Recientemente ha aparecido una nueva atracción, un molino de viento, en el horizonte de Nottingham (ver figura 1). Las velas giran en los días ventosos y la tienda del molino contiguo vende paquetes de harina molida en piedra pero también, lo que es más sorprendente, tratados sobre física matemática. La conexión entre la harina y la física es parte del carácter único del molino y se explica por una placa que alguna vez se adjuntó al costado de la torre del molino que decía:
\par
\begin{center}
\textsc{Aquí Vivió y Trabajó \\
George Green \\
Matemático \\
N.1793 - M.1841.}
\end{center}

Es el teorema de Green familiar para los estudiantes universitarios de física de todo el mundo, y de las funciones de Green que se utilizan en muchas ramas de la física clásica y cuántica.

\section{Edad temprana y educación.}
\par
El padre de George Green tenía una panadería cerca del centro de Nottingham, entonces una ciudad con una población de unos 30 000 habitantes. Su único hijo, George, fue bautizado el 14 de julio de 1793. En marzo de 1801, George se inscribió como alumno número 255 en Academia de Robert Goodacre, donde permaneció solo 18 meses. Cuando tenía nueve años, lo pusieron a trabajar en la panadería de su padre. El período que pasó en la Academia de Goodacre fue la única educación formal de Green antes de que, a la edad de 40 años, fuera a la Universidad de Cambridge. Tuvo la suerte de haber tenido incluso esa cantidad de educación. La mayoría de los niños no tenían ninguno, aunque a algunos se les enseñaba a leer y escribir en las escuelas dominicales. Green también tuvo la suerte de que su padre pudiera permitirse el lujo de educarlo en forma privada y eligió la academia de Robert Goodacre, un hombre conocido por su entusiasmo por la astronomía y las ciencias naturales.
\par
Green pasó cinco años en la panadería de su padre y luego lo enviaron a aprender a ser molinero en el molino de torre que su padre había construido en una colina en el pueblo de Sneinton, a una milla de Nottingham. El molino tenía cinco pisos de altura y se encontraba en un patio con graneros, establos para ocho caballos y una cabaña para el molinero John Smith, su esposa y su hija Jane. Las calles de Nottingham no eran lugares seguros para caminar después del anochecer, por lo que, durante muchos años antes de que su familia construyera una casa junto al molino, Green pasó la mayor parte de sus días y muchas de sus noches trabajando y viviendo en el molino. Cuando tenía 31 años, Jane Smith le dio una hija. Tuvieron siete hijos en total, pero nunca se casaron. Se dijo que el padre de Green sintió que Jane no era una esposa adecuada para el hijo de un próspero comerciante y terrateniente y amenazó con desheredarlo.
\par
Poco se sabe sobre la vida de Green desde 1802 hasta 1823. En particular, no se sabe si recibió alguna ayuda en su desarrollo matemático o si fue completamente autodidacta. Es posible que haya recibido ayuda de John Toplis, miembro del Queens' College de la Universidad de Cambridge y director de la Nottingham Grammar School. La traducción de Toplis del libro Mécanique Céleste de Pierre-Simon Laplace, publicado en Nottingham en 1814, parece una fuente probable del interés de Green por la teoría del potencial. El trabajo era inusual en Gran Bretaña en ese momento, ya que Toplis usó la notación más conveniente de Gottfried Leibniz para diferenciales en lugar de la de Isaac Newton. Debido a que Green adaptó la notación de Leibniz, parece plausible que Green fuera influenciado por Toplis, pero no hay evidencia de que Toplis actuara de alguna manera como su tutor.
\par
En 1823, Green se unió a la Biblioteca de suscripción de Nottingham, el centro de actividad intelectual de la ciudad. La biblioteca estaba situada en Bromley House (ver figura 2). La membresía de la biblioteca proporcionó a Green aliento, apoyo y acceso a Philosophical Transactions of the Royal Society y otras revistas científicas. Estos no incluían revistas extranjeras, pero las Transacciones enumeraban el contenido de esas revistas, y eso le habría permitido a Green obtener reimpresiones directamente de los autores.

\section{Ensayo de Green de 1828.}

El primer trabajo publicado de Green, en 1828, fue un ensayo sobre la aplicación del análisis matemático a las teorías de la electricidad y el magnetismo. Este importante trabajo, de unas 70 páginas, contiene la derivación del teorema de Green y aplica el teorema, junto con las funciones de Green, a problemas electrostáticos. Su portada se reproduce en la figura 3.
\par
La ruta habitual para la publicación de artículos científicos en Inglaterra era entonces a través de una de las revistas de las dos sociedades científicas, la Royal Society y la Cambridge Philosophical Society. Pero Green, sin calificaciones ni contactos con el establecimiento científico, sintió que sería presuntuoso enviar su artículo a una revista. Así que pagó para que su artículo se publicara de forma privada en Nottingham. La forma tentativa en la que se acercó a la publicación es evidente en su prólogo. Allí expresó su esperanza de que \enquote{la dificultad del tema incline a los matemáticos a leer esta obra con indulgencia, más particularmente cuando se les informe que fue escrita por un joven, que se ha visto obligado a obtener la pequeña el conocimiento que posee, a intervalos y por medios tales como otras vocaciones indispensables que ofrecen pero pocas oportunidades de mejora mental, otorgadas}.
\par
Cada copia del ensayo de Green costaba 7 chelines y medio, aproximadamente el salario semanal de un fabricante de medias de Nottingham. Los compradores incluían médicos locales, maestros de escuela, clérigos y fabricantes de encajes y calcetines, la principal industria de Nottingham en ese momento. Casi la mitad de los que compraron la obra eran miembros de la Biblioteca de suscripción de Nottingham. Pocos, si es que alguno, podrían haber entendido el ensayo, por lo que evidentemente deben haber tenido una confianza considerable en las habilidades de su autor.
\par
El propósito principal de Green al publicar habría sido llamar la atención de otros matemáticos en el Reino Unido y en el extranjero sobre su trabajo. Parece, sin embargo, que con una excepción, hubo poca o ninguna respuesta. Eso debe haber sido muy descorazonador.
\par
La excepción fue Edward Bromhead, quien evidentemente quedó impresionado por el trabajo de Green. Respondió de inmediato y se ofreció a ayudar a Green a publicar cualquier artículo futuro en una revista. Bromhead era una persona rica e influyente, una figura pública y un benefactor de la ciudad de Lincoln, 35 millas al noreste de Nottingham. Había estudiado matemáticas en la Universidad de Glasgow y luego
en Cambridge, y aunque su posición en la sociedad lo alejó de una carrera académica, estuvo en estrecho contacto con varios matemáticos y científicos británicos destacados, incluidos Charles Babbage, John Herschel y William Whewell.
\par
Así que Bromhead estaba en una buena posición para poner a Green en contacto con esos y otros matemáticos, y debe haberse sorprendido y decepcionado de no recibir respuesta de Green. Finalmente llegó uno, unos 20 meses después. La larga carta de Green explicaba lo complacido y agradecido que había estado de saber de Bromhead y cómo originalmente tenía la intención de escribir y aceptar su oferta. Pero le habían dicho que la carta de Bromhead para él fue escrita por cortesía y que, dadas las diferencias en sus posiciones sociales, no habría sido apropiado responder. Solo más tarde descubrió lo pobre que había sido ese consejo.
\par
Bromhead respondió rápidamente y el intercambio fue seguido por muchos otros y por reuniones en la casa de Bromhead cerca de Lincoln. Con la ayuda de Bromhead, en 1833, Green publicó su primer artículo en una revista. Ese artículo trataba nuevamente sobre la electricidad, pero siguiendo el consejo de Bromhead, Green abandonó el tema, que era de poco interés para los matemáticos del Reino Unido en ese momento. Pasó a los temas más de moda de la hidrodinámica, el movimiento de las olas y la óptica, los temas de sus ocho artículos publicados entre 1835 y 1839.
\par
Poco después de conocerse, Green le contó a Bromhead su sueño de ir a Cambridge. Volvió al tema en una carta de abril de 1833, en la que escribió: \enquote{Sabes que tengo una inclinación por Cambridge si hubiera una perspectiva justa de éxito. Desafortunadamente, sé poco latín, menos griego, he visto demasiados inviernos y, por lo tanto, estoy en un estado de suspenso por motivos opuestos}. Green también debe haber sido consciente de que la existencia de Jane y sus hijos (cuatro en ese momento) estaba algo en desacuerdo con el requisito universitario de que los miembros fueran célibes. Pero parecía que la falta de celibato podía tolerarse siempre que el miembro no estuviera realmente casado. Con la influencia de Bromhead, Green ingresó a Cambridge en octubre de 1833 como estudiante universitario, convirtiéndose en miembro de Gonville $\&$ Caius College, la misma universidad a la que asistió Bromhead. Green se graduó en 1837 y en noviembre de 1839 se convirtió en compañero universitario. Desafortunadamente, pronto se enfermó, regresó a Nottingham en la primavera de 1840 y murió de influenza el 31 de mayo de 1841. Él y Jane están enterrados uno al lado del otro en el patio de la iglesia de St. Stephen, a unos cientos de metros del molino. No parece haber ningún retrato de él, y murió poco después de que se inventara la fotografía.

\section{Redescubrimiento del ensayo de Green.}

A principios de la década de 1840, gran parte del trabajo de Green estaba disponible en la literatura abierta, pero su contribución más importante, su ensayo, aún no se había publicado en una revista. Es posible que no se hubiera descubierto durante muchos años si no hubiera sido por William Thomson, que se muestra en la figura 4. Thomson, más tarde Lord Kelvin, fue a Cambridge poco después de la muerte de Green. Su padre era profesor de matemáticas en Glasgow y Thomson ya se había licenciado en Glasgow antes de ir a Cambridge. Estaba interesado en la electricidad y había visto una referencia al ensayo de Green en una nota a pie de página de un artículo de Robert Murphy. En una carta sobre Green escrita poco antes de su muerte en 1907, Thomson le escribió a Joseph Larmor:
\begin{quote}
Cuando fui a Cambridge como estudiante de primer año, pregunté en todas las librerías de Cambridge por el Ensayo sobre electricidad y magnetismo de Green, y no pude saber nada de él.
\par
El día antes de irme de Cambridge a París después de obtener mi título, en enero de 1845, me encontré con [William] Hopkins en lo que creo que entonces se llamaba el Paseo de los Wranglers Senior, y le dije que había preguntado en vano por el Ensayo de Green \ldots Él dijo: \enquote{Tengo algunas copias}. Me acompañó y me llevó a su casa, y allí, en su principal sala de entrenamiento en la que había estado día tras día durante dos años, encontró tres ejemplares del Ensayo de Green en su estantería y me los dio.
\par
Solo tuve tiempo esa noche para mirar algunas páginas, lo que me asombró. Al día siguiente, si mal no recuerdo, en el techo de una diligencia [carruaje] de camino a París, logré leer algo más.
\end{quote}

Thomson se fue a París durante cuatro meses para adquirir experiencia en física experimental. Pero también quería conocer a teóricos como Michel Chasles, Joseph Liouville y Charles Sturm. Su carta a Larmor continúa diciendo que Liouville "prestó gran atención" al ensayo de Green cuando Thomson se lo mostró en su casa. Más tarde, mientras Thomson estaba con un colega de Cambridge, llegó Sturm, jadeando por el esfuerzo. Ansiosamente pidió ver el ensayo de Green. “Así que se lo entregué. Se sentó y pasó las páginas con avidez. Se detuvo en un lugar gritando: ‘Ah voila mon affaire [Ahí está mi trabajo]’”. Inevitablemente, muchos de los hallazgos de Green se habían redescubierto durante los 17 años transcurridos desde que apareció el ensayo. Thomson señala que, además de Sturm, Chasles encontró sus propios resultados y demostraciones en el ensayo.

Cuando regresó a Inglaterra, Thomson dispuso que el ensayo de Green se volviera a publicar en Crelle's Journal. Edmund Whittaker, en su A History of the Theories of Aether and Electricity (Dover, 1989), dice:
\begin{quote}
Es imposible dejar de notar a lo largo de toda la obra de Kelvin evidencias de la profunda impresión que le causaron los escritos de Green. Lo mismo puede decirse del amigo y contemporáneo de Kelvin, [George] Stokes, y de hecho no es exagerado describir a Green como el fundador de esa “Escuela de Cambridge” de filósofos de la naturaleza de la que Kelvin, Stokes, [Lord] Rayleigh, [James ] Clerk Maxwell, [Horace] Lamb, JJ Thomson, Larmor y [Augustus] Love fueron los miembros más ilustres de la segunda mitad del siglo XIX.
\end{quote}

\section{La contribución de Green a la ciencia.}

La inspiración para el ensayo de Green vino de Francia, de Laplace y Siméon Poisson. La ley del cuadrado inverso para las fuerzas entre dos cargas se había establecido experimentalmente recientemente y Poisson había demostrado cómo determinaba la distribución de carga sobre las superficies de los conductores. Las técnicas que usó solo eran aplicables a superficies con geometría simple, por lo que Green ideó técnicas poderosas para obtener las distribuciones de cualquier superficie. Hizo un gran uso del potencial eléctrico y le dio ese nombre. Uno de los teoremas que desarrolló, ahora llamado teorema de Green, se simplifica fácilmente a lo que a menudo se llama teorema de la divergencia o teorema de Gauss. Sin embargo, muchos de los primeros libros de texto también llaman a esta simplificación el teorema de Green, presumiblemente para enfatizar sus afirmaciones de precedencia. La otra poderosa herramienta desarrollada en el ensayo ahora se llama función de Green o de Green.
\par
El trabajo posterior de Green sobre la elasticidad se recuerda por el tensor de Green. Green se interesó en la elasticidad a través de consideraciones sobre el \enquote{éter}, que, por supuesto, tenía que ser un sólido porque las ondas de luz son transversales.
\par
Demostró que generalmente se requieren 21 módulos para tener en cuenta las propiedades elásticas de un medio anisotrópico, y explicó cómo la simetría puede reducir ese número.
\par
En otro artículo, Green realizó los primeros cálculos correctos de las proporciones de energía reflejada y refractada en una interfaz y explicó el fenómeno de la reflexión interna total; incluida en esa explicación había una descripción de la onda evanescente que existe en el medio de mayor índice de refracción. En ese trabajo, se convirtió en uno de los primeros en escribir el principio de la conservación de la energía. El trabajo posterior de Green contenía una serie de otras primicias matemáticas. Estos incluyen la derivación de un método aproximado para resolver ecuaciones diferenciales a partir de su artículo sobre el movimiento de las ondas de agua en un canal de ancho y profundidad variables. Su enfoque reapareció más de un siglo después como el método Wentzel-Kramers-Brillouin (WKB). También fue el primero en formular el principio de minimización funcional de Dirichlet, aunque Bernhard Riemann le dio el nombre con el que se le suele conocer.

\section{Teorema de Green y funciones de Green.}

El concepto de función de Green se ilustra más fácilmente considerando la dinámica de una partícula inicialmente en reposo bajo la influencia de una fuerza dependiente del tiempo $F(t)$. Primero se considera una fuerza que actúa durante un tiempo muy breve: un golpe o impulso brusco. El impulso se elige para inducir un cambio unitario en la cantidad de movimiento en un tiempo $\pderivada{t}$. En un momento posterior $t$, el desplazamiento $s(t)$ de la partícula se define como la función de Green $G (t,\pderivada{t})$. Pero una fuerza $F (\pderivada{t})$ que actúa durante un intervalo de tiempo infinitesimal $\Delta \pderivada{t}$ es un impulso de magnitud $F (\pderivada{t}) \, \Delta \pderivada{t}$, y se puede considerar que una fuerza aplicada continuamente en el tiempo genera una secuencia de tales impulsos. Uno puede encontrar el movimiento de la partícula sumando los efectos de todos los impulsos aplicados desde el tiempo inicial $t_{0}$ hasta el tiempo $t$. Por lo tanto:
\begin{align}
s (t) = \scaleint{6ex}_{\bs t_{0}}^{t} G (t, \pderivada{t}) \, F (\pderivada{t}) \dd{\pderivada{t}}
\label{eq:ecuacion_01}
\end{align}
que cumple las condiciones iniciales: $s = 0$, $\dv*{s}{t} = 0$. La respuesta del sistema a fuerzas arbitrarias se puede calcular fácilmente una vez que se haya encontrado la función de Green. Tenga en cuenta que la función de Green depende del sistema dinámico pero no de la forma de la fuerza aplicada. El Cuadro 1 proporciona un cálculo de muestra explícito.
\par
La superposición que se muestra en la ec. (\ref{eq:ecuacion_01}) de los efectos de impulsos sucesivos solo es válida para un sistema lineal, en el que la respuesta es proporcional a la fuerza aplicada. No obstante, la técnica de Green claramente tiene una amplia aplicación en una variedad de sistemas, tanto mecánicos como eléctricos, para los cuales la respuesta lineal a fuerzas o voltajes es importante en aplicaciones prácticas.

\begin{tcolorbox}[title={\centering Cuadro 1. Ejemplo simple de una función de Green.}]

Supongamos que se desea encontrar la velocidad $v(t)$ de una partícula que parte del reposo y sobre la que actúa una fuerza viscosa $\alpha \, v$ y una fuerza arbitraria $F (t)$. La ecuación de movimiento es:
\begin{align*}
m \, \dv{v}{t} + \alpha \, v = F (t)
\end{align*}

Para una unidad de fuerza impulsiva aplicada en el momento $t = \pderivada{t}$, la solución es la función de Green $G (t, \pderivada{t})$. La velocidad inmediatamente después del impulso es $1/m$ y luego decae exponencialmente, de modo que, para $t > \pderivada{t}$:
\begin{align*}
G (t, \pderivada{t}) = \dfrac{1}{m} \exp\bigg( - \dfrac{\alpha (t - \pderivada{t})}{m} \bigg) \hspace{1cm} t \geq \pderivada{t}
\end{align*}

Para $t < \pderivada{t}$, la función de Green se anula. Por tanto, la solución general del problema dinámico para una fuerza arbitraria es:
\begin{align*}
v (t) =  \scaleint{6ex}_{\bs 0}^{t} \dfrac{1}{m} \exp\bigg( - \dfrac{\alpha (t - \pderivada{t})}{m} \bigg) \, F (\pderivada{t}) \dd{\pderivada{t}}
\end{align*}

Para este ejemplo simple, la solución general también se puede obtener mediante la integración directa de la ecuación de movimiento utilizando un factor de integración.
\end{tcolorbox}

El trabajo original de Green estaba dirigido a la solución de problemas electrostáticos en regiones limitadas. En ese caso, la función de Green $G (\vb{r},\pderivada{\vb{r}})$ es el potencial en el punto $\vb{r}$ producido por una unidad de carga puntual en $\vb{\pderivada{r}}$. La carga puntual es el análogo espacial de la fuerza impulsiva que solo actúa en un solo instante en el tiempo. La función de Green no es lo mismo que el potencial electrostático de Coulomb generado por la carga puntual, porque una fuente puntual también induce cargas en los límites. Con lo que ahora se conoce como el teorema de Green, el potencial electrostático $\phi (\vb{r})$ se puede expresar en notación moderna (ver el Cuadro 2 para más detalles) como:
\begin{align}
\begin{aligned}
\phi (\vb{r}) &= \scaleint{6ex}_{\bs \tau} G (\vb{r}, \vb{\pderivada{r}}) \, \rho (\pderivada{r}) \dd{\pderivada{\tau}} + \\[0.5em]
&+ \scaleint{6ex}_{\bs S} \bigg[ \phi (\pderivada{\vb{r}}) \, \pderivada{\nabla} G (\vb{r}, \vb{\pderivada{r}}) - G (\vb{r}, \vb{\pderivada{r}}) \, \pderivada{\nabla} \phi (\pderivada{\vb{r}}) \bigg] \cdot \dd{\vb{\pderivada{S}}}
\end{aligned}
\label{eq:ecuacion_02}
\end{align}

El potencial $\phi (r)$ es una superposición de los efectos debido a la densidad de carga espacial $\rho (\vb{\pderivada{r}})$ en el volumen $\tau$ y las cargas inducidas en la superficie límite del volumen $\vb{S}$, cuyas influencias son transmitidas por la función de Green función $G (\vb{r}, \vb{\pderivada{r}})$. La importancia del trabajo de Green radica en su generalidad. No hay restricción en la geometría de la superficie. Es sencillo mostrar, por ejemplo, que si cualquier superficie se mantiene a potencial cero (puesta a tierra) y no contiene carga $(\rho = 0)$, entonces la solución interior es $(\phi = 0)$ en todas partes. Es decir, el volumen interno está completamente protegido de las influencias electrostáticas externas.

\section{Funciones de Green en la teoría de la dispersión.}

Las técnicas de función de Green se utilizaron cada vez más en la última parte del siglo XIX para resolver las ecuaciones diferenciales parciales, todas muy similares, que describen fenómenos eléctricos, magnéticos, mecánicos y térmicos. Las funciones de Green también se pueden utilizar para formular la teoría de la dispersión de ondas clásica. De hecho, debido a que la ecuación de Schrödinger tiene una forma similar a la ecuación de onda, las funciones de Green también pueden usarse para describir la dispersión no relativista de una sola partícula por una energía potencial externa $V (\vb{r})$. Para una partícula de energía total $E$ y masa $m$, la parte independiente del tiempo de la función de onda $\psi (\vb{r})$ obedece:
\begin{align}
- \dfrac{\hbar^{2}}{2 \, m} \laplacian{\psi} + V (\vb{r}) \, \psi (\vb{r}) = E \, \psi (\vb{r})
\label{eq:ecuacion_03}
\end{align}

Al tratar el término $V \psi$ de manera similar a la densidad de carga impuesta en un problema electrostático, se puede escribir una solución formal de la ecuación 3:
\begin{align}
\psi (\vb{r}) = \psi_{0} (\vb{r}) + \scaleint{6ex} G (\vb{r}, \vb{\pderivada{r}}) \, \dfrac{2 m}{\hbar^{2}} \, V (\vb{\pderivada{r}}) \, \psi (\vb{\pderivada{r}}) \dd{\pderivada{\tau}}
\label{eq:ecuacion_04}
\end{align}
donde $\psi_{0} (\vb{\pderivada{r}})$ es una onda incidente para una partícula de energía $E$ y la función de Green $G (\vb{r}, \vb{\pderivada{r}})$ es la amplitud de onda en $\vb{r}$ inducida por una fuente puntual dada en $\vb{\pderivada{r}}$. Pero a diferencia de la ec. (\ref{eq:ecuacion_01}), la ec. (\ref{eq:ecuacion_04}) no es una solución explícita, porque la función de onda desconocida aparece dentro de la integral. La razón física es que la fuente de la onda dispersa, el término $V \, \psi$ en la ec. (\ref{eq:ecuacion_03}), solo existe cuando está presente una onda incidente, a diferencia de la fuerza en la ec. ()\ref{eq:ecuacion_01}) que está determinada por influencias externas.
\par
La forma habitual de resolver la ec. (\ref{eq:ecuacion_04}) es por iteración. El primer término iterativo, con $\psi (\vb{\pderivada{r}})$ dentro de la integral reemplazada por $\psi_{0} (\vb{\pderivada{r}})$, es la conocida aproximación de Born de la teoría de la dispersión. Los términos sucesivos en la iteración involucran múltiples integrales de la forma:
\begin{align}
\scaleint{6ex} G (\vb{r}, \vb{\pderivada{r}}) \, V (\vb{\pderivada{r}}) \, \psi_{0} (\vb{\pderivada{r}}) \, G (\vb{\pderivada{r}}, \vb{\sderivada{r}}) \, \psi_{0} (\vb{\sderivada{r}}) \ldots \dd{\sderivada{\tau}} \dd{\pderivada{\tau}}
\label{eq:ecuacion_05}
\end{align}
Dichos términos corresponden a múltiples eventos de dispersión en los que la onda incidente se dispersa en los puntos $\pderivada{r}, \sderivada{r}$, etc., antes de llegar a $\vb{r}$. Debido a que hay un cambio de cantidad de movimiento en cada punto de dispersión, $G (\vb{r}, \vb{\pderivada{r}})$ es la respuesta a un impulso tal como lo es la función de Green $G (t, \pderivada{t})$ dependiente del tiempo.

\section{Desarrollos posteriores.}

En la física de partículas elementales, la dispersión tiene una importancia crucial porque la única forma que tenemos de investigar las propiedades de las partículas elementales es a través de sus interacciones entre sí. Pero esas interacciones deben describirse en términos de campos cuantificados que transmiten fuerzas entre partículas a través del intercambio de cuantos virtuales de energía y momento. Para las partículas cargadas que interactúan electromagnéticamente, esos cuantos son los fotones de la electrodinámica cuántica (QED). La teoría de la interacción de los electrones (y positrones) con el campo electromagnético cuantificado se desarrolló a fines de la década de 1940 para explicar el cambio de Lamb del nivel de energía $1s$ del átomo de hidrógeno y la desviación del momento magnético del electrón del valor de Dirac. Los dos fenómenos se atribuyeron a fluctuaciones en los campos, pero la teoría inicial se vio perturbada por correcciones infinitas hasta que se demostró que los infinitos podían eliminarse mediante una renormalización de la masa y la carga de los electrones.
\par
Julian Schwinger y Richard Feynman lograron una formulación consistente de QED de forma independiente. Schwinger, que anteriormente había utilizado funciones de Green para describir la propagación de microondas cuando trabajaba en un radar, dio un tratamiento teórico de campo formal en el que las funciones de Green parecían propagar campos entre puntos del espacio-tiempo. Algunos aspectos de su teoría eran muy similares al trabajo de Sin-Itiro Tomonaga en Japón, quien estaba desarrollando un enfoque de teoría cuántica de campos para la renormalización. Feynman usó su enfoque intuitivo del espacio-tiempo para la mecánica cuántica. En esa formulación, la amplitud de probabilidad para un proceso dado es la suma de las amplitudes para cada ruta de espacio-tiempo disponible. Para cualquier ruta en particular, la amplitud es el producto de los factores correspondientes a la propagación libre o a las interacciones de dispersión como en la ecuación 5. Cada secuencia de propagadores e interacciones se puede representar como un diagrama de Feynman que brinda una imagen física útil del proceso. . Posteriormente, Freeman Dyson demostró la equivalencia de las teorías de Feynman y Schwinger y sistematizó el cálculo de los efectos de orden superior en QED.
\par
En el tratamiento de Feynman, queda claro por qué las funciones de Green juegan un papel tan natural y fundamental en la teoría de la física de partículas. Las interacciones entre partículas son eventos complicados de dispersión múltiple en los que las fuerzas se transmiten mediante campos cuánticos. Pero la propagación de campos entre puntos es precisamente para lo que se diseñaron originalmente las funciones de Green. Funciones de Green, Figura 5. Monumento a Green en la Abadía de Westminster, Londres. En 1993 se llevó a cabo la ceremonia de inauguración del memorial, 200 años después del nacimiento de Green. En el libro de Mary Cannell, George Green: Mathematician $\&$ Physicist, 1793 - 1841, se ofrece un relato completo de las celebraciones del bicentenario de Green, incluidas las conferencias impartidas por Julian Schwinger y Freeman Dyson (ver lecturas adicionales a continuación). a menudo llamados propagadores de Feynman en física de partículas, se encuentran entre las herramientas de trabajo estándar del análisis teórico en la física cuántica moderna.
\par
En las décadas de 1950 y 1960, los físicos comenzaron a utilizar métodos de función de Green para describir las interacciones de muchos cuerpos de la física de la materia condensada. La conductividad eléctrica de un sólido es esencialmente la respuesta lineal a un campo externo. Por lo tanto, se puede expresar en términos de una función de Green, como lo demostraron Ryogo Kubo y otros. Las fuentes de resistencia eléctrica en un sólido son procesos de dispersión, que naturalmente se describen mediante funciones de Green. Pero a diferencia de los procesos generalmente descritos en física de partículas, los procesos de materia condensada ocurren en forma finita. Una generalización de la teoría conduce a funciones de Green térmicas o de temperatura finita. Pero sorprendentemente, después de introducir una coordenada de tiempo compleja cuya parte imaginaria es proporcional a la temperatura inversa, las funciones térmicas de Green se pueden calcular con técnicas muy parecidas a las funciones ordinarias de Green de temperatura cero. Entonces, hoy en día, los físicos aplican funciones de Green en áreas que ni siquiera se concibieron en la época de Green, y es probable que continúen haciéndolo independientemente de lo que se desarrolle en el futuro.

\newpage

\begin{tcolorbox}[breakable, title={\centering Cuadro 2. Funciones de Green, Funciones Delta y Condiciones de Frontera.}]

Las funciones de Green generalmente se usan para resolver un campo clásico $\phi (\vb{r})$ en una ecuación diferencial parcial de la forma general:
\begin{align*}
\laplacian{\phi (\vb{r})} + k^{2} \, \phi (\vb{r}) = \rho (\vb{r})
\end{align*}
donde $k = 0$ para campos electrostáticos pero es distinto de cero para campos de ondas, y $\rho (r)$ es una función fuente. La función de Green $G (\vb{r}, \vb{\pderivada{r}})$ es la solución para una fuente puntual en $\vb{\pderivada{r}}$ representada por una función delta de Dirac $\delta (\vb{r} - \vb{\pderivada{r}})$. Es decir, la función de Green satisface la ecuación:
\begin{align*}
\laplacian{G (\vb{r}, \vb{\pderivada{r}})} + k^{2} \, G (\vb{r}, \vb{\pderivada{r}}) = \delta (\vb{r} - \vb{\pderivada{r}})
\end{align*}

Aplicando el teorema de Green a las funciones $\phi$ y $G$:
\begin{align*}
\scaleint{6ex}_{\bs \tau} \bigg( \phi \, \laplacian{G} - G \, \laplacian{\phi} \bigg) \dd{\tau} = \scaleint{6ex}_{\bs S} \bigg( \phi \, \grad{G} -  G \, \grad{\phi} \bigg) \cdot \dd{\vb{S}}
\end{align*}
conduce a la ec. (\ref{eq:ecuacion_02}) después de intercambiar $\vb{r}$ y $\vb{\pderivada{r}}$ y explotar la simetría de $G (\vb{r}, \vb{\pderivada{r}})$.
\par
Para determinar una solución única para $\phi (\vb{r})$, se deben especificar las condiciones de contorno. También se deben elegir condiciones de contorno adecuadas para $G (\vb{r}, \vb{\pderivada{r}})$. Así, la función de Green es el campo debido a una fuente puntual pero modificado por los efectos de las fronteras. En electrostática, los efectos de frontera surgen de las cargas superficiales. En los problemas de ondas, los límites dan lugar a ondas reflejadas que pueden calcularse resolviendo la ecuación sin fuente apropiada.
\par
En el tratamiento de función de Green del movimiento de partículas, un impulso unitario está representado por una fuerza $F (t) = \delta (\vb{r} - \vb{\pderivada{r}})$ y es el análogo de la fuente puntual unitaria en problemas espaciales. Las condiciones iniciales juegan el papel de condiciones de contorno.
\par
En los problemas de dispersión, la función de Green se define como la solución de la ec. (\ref{eq:ecuacion_03}) con el término $V \, \psi$ llevado al lado derecho y reemplazado por una función delta fuente $\delta (\vb{r} - \vb{\pderivada{r}})$. Para tales problemas, las condiciones de contorno requieren que la solución sea una onda saliente a grandes distancias de la fuente porque las ondas entrantes implicarían reflejos no físicos en el infinito.
\end{tcolorbox}

\section{Honor en su propio país.}

La máquina de vapor puso fin a la era de los molinos de viento y Green's Mill se deterioró gradualmente. En 1920, el molino fue comprado por Oliver Hind, un hombre de negocios y filántropo local, quien reparó su tapa de madera. En el mismo año, Holbrook Bequest pagó la placa conmemorativa del molino. Durante un tiempo, el molino se convirtió en una fábrica de cera para muebles, pero en 1947 se incendió y fue destruido por completo. Hind selló el caparazón tapando con tablas las puertas y ventanas y reemplazando la tapa con una losa de concreto. El interés local en Green se había mantenido a lo largo de los años en un nivel modesto, en particular como resultado de la actividad de los miembros de la Universidad de Nottingham. En 1945, H. Gwynedd Green (sin pariente), miembro del departamento de matemáticas de la universidad, escribió una biografía de Green. Mary Cannell publicó un trabajo sustancialmente más largo en 1993; una segunda edición ampliada apareció en 2001.
\par
El plan para restaurar el molino fue impulsado inicialmente por un rumor de que sería derribado para un nuevo desarrollo. Eso condujo a la formación del Fondo Conmemorativo de George Green por parte de físicos de la Universidad de Nottingham y otros, y finalmente, como un memorial apropiado para Green, la ciudad de Nottingham restauró el molino. Ahora es uno de los principales museos y atracciones turísticas de Nottingham. El segundo monumento importante de Green está en la Abadía de Westminster, Londres. Ese es el lugar de descanso de los reyes y reinas británicos y, en tiempos más recientes, el lugar donde se recuerda a muchas figuras literarias y científicas famosas. En 1993, en el bicentenario del nacimiento de Green, Michael Atiyah, presidente de la Royal Society, inauguró un monumento. La placa, que se muestra en la figura 5, se encuentra en el piso de la Abadía junto a los monumentos a Newton, Michael Faraday, Maxwell y Kelvin. La congregación incluía a varios descendientes de Green y muchos otros científicos y matemáticos, incluidos Schwinger y Dyson. El memorial de Kelvin tuvo que moverse hacia los lados para acomodar el de Green. Pero seguramente no se habría opuesto.
\end{document}