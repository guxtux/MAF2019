\documentclass[12pt,landscape]{article}
\usepackage[utf8]{inputenc}
\usepackage[spanish,es-lcroman, es-tabla]{babel}
\usepackage[autostyle,spanish=mexican]{csquotes}
\usepackage{amsmath}
\usepackage{amssymb}
\usepackage{nccmath}
\numberwithin{equation}{section}
\usepackage{amsthm}
\usepackage{graphicx}
\usepackage[outdir=./]{epstopdf}
\DeclareGraphicsExtensions{.pdf,.png,.jpg,.eps}
\usepackage{color}
\usepackage{float}
\usepackage{fancyhdr}
\usepackage{multicol}
\usepackage{enumerate}
\usepackage[shortlabels]{enumitem}
\usepackage{anyfontsize}
\usepackage{anysize}
\usepackage{array}
\usepackage{multirow}
\usepackage{enumitem}
\usepackage{cancel}
\usepackage{nameref}
\usepackage{pdflscape}
\usepackage{makecell}
\usepackage{longtable}
\usepackage{pgfplots}
\pgfplotsset{compat=1.12}
\usepackage{tikz}
\usepackage{circuitikz}
\usepackage{tikz-3dplot}
\usepackage{caption}
\usepackage{bm}
\usepackage{mathtools}
\usepackage{esvect}
\usepackage{hyperref}
\usepackage{relsize}
\usepackage{siunitx}
\usepackage{physics}
%\usepackage[backend=biber]{biblatex}
\usepackage{standalone}
\usepackage{mathrsfs}
\usepackage{bigints}
\usepackage{bookmark}
%Quita el número de la página
\pagenumbering{gobble}
\spanishdecimal{.}
\setlength{\voffset}{-0.75in}
\title{Tabla que relaciona el tipo de EDP y las CDF \\ {\large Tema 2 - Primeras técnicas de solución - Curso MAF}}
\date{ }
\begin{document}
\renewcommand\labelenumii{\theenumi.{\arabic{enumii}}}
\maketitle
\fontsize{14}{14}\selectfont
\vspace{-2cm}
\begin{center}
{\renewcommand{\arraystretch}{2}%
{\setlength\extrarowheight{3pt}
\begin{tabular}{ | c | c | c | c | c |} \hline
\multirow{3}{3.5cm}[-7pt]{\makecell{Condiciones \\ de frontera}} & \multirow{3}{3.5cm}[-7pt]{\makecell{Tipo de \\ superficie}} & \multicolumn{3}{c |}{Tipo de EDP} \\ \cline{3-5}
 & & \makecell{Elíptica} & \makecell{Hiperbólica} & \makecell{Parabólica} \\ \cline{3-5}
 & & \makecell{Laplace, Poisson \\ en $(x,y)$} & \makecell{Ecuación de onda \\ en $(x,t)$} & \makecell{Ecuación de difusión \\ en $(x,t)$} \\ \hline
\multirow{2}*{\textbf{Cauchy}} & \makecell{Superficie \\ abierta} & \makecell{Resultados sin \\ interpretación física} & \makecell{\underline{\textbf{Solución única}} \\ \underline{\textbf{estable}}} & \makecell{Demasiado \\ restrictiva} \\ \cline{2-5}
 & \makecell{Superficie \\ cerrada} & \makecell{Demasiado \\ restrictiva} & \makecell{Demasiado \\ restrictiva} & \makecell{Demasiado \\ restrictiva} \\ \hline
 \multirow{2}*{\textbf{Dirichlet}} & \makecell{Superficie \\ abierta} & \makecell{Insuficiente} & \makecell{Insuficiente}  &\makecell{\underline{\textbf{Solución única}} \\ \underline{\textbf{estable} en una dirección}} \\ \cline{2-5} 
& \makecell{Superficie \\ cerrada} & \makecell{\underline{Solución única} \\ \underline{estable}} & \makecell{Más de \\ una solución} & \makecell{Demasiado \\ restrictiva} \\ \hline
 \multirow{2}*{\textbf{Neumann}} & \makecell{Superficie \\ abierta} & \makecell{Insuficiente} & \makecell{Insuficiente}  &\makecell{\underline{\textbf{Solución única}} \\ \underline{\textbf{estable} en una dirección}} \\ \cline{2-5}
& \makecell{Superficie \\ cerrada} & \makecell{\underline{\textbf{Solución única}} \\ \underline{\textbf{estable}}} & \makecell{Más de \\ una solución} & \makecell{Demasiado \\ restrictiva} \\ \hline
\end{tabular}}}
\end{center}
\end{document}