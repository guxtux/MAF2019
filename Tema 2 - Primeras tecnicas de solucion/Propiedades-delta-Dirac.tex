\documentclass[12pt]{article}
\usepackage[utf8]{inputenc}
\usepackage[spanish,es-lcroman, es-tabla]{babel}
\usepackage[autostyle,spanish=mexican]{csquotes}
\usepackage{amsmath}
\usepackage{amssymb}
\usepackage{nccmath}
\numberwithin{equation}{section}
\usepackage{amsthm}
\usepackage{graphicx}
\usepackage{epstopdf}
\DeclareGraphicsExtensions{.pdf,.png,.jpg,.eps}
\usepackage{color}
\usepackage{float}
\usepackage{multicol}
\usepackage{enumerate}
\usepackage[shortlabels]{enumitem}
\usepackage{anyfontsize}
\usepackage{anysize}
\usepackage{array}
\usepackage{multirow}
\usepackage{enumitem}
\usepackage{cancel}
\usepackage{tikz}
\usepackage{circuitikz}
\usepackage{tikz-3dplot}
\usetikzlibrary{babel}
\usepackage{bm}
\usepackage{mathtools}
\usepackage{esvect}
\usepackage{hyperref}
\usepackage{relsize}
\usepackage{siunitx}
\usepackage{physics}
%\usepackage{biblatex}
\usepackage{standalone}
\usepackage{mathrsfs}
\usepackage{bigints}
\usepackage{bookmark}
\spanishdecimal{.}

\setlist[enumerate]{itemsep=0mm}

\renewcommand{\baselinestretch}{1.5}

\let\oldbibliography\thebibliography

\renewcommand{\thebibliography}[1]{\oldbibliography{#1}

\setlength{\itemsep}{0pt}}
%\marginsize{1.5cm}{1.5cm}{2cm}{2cm}


\newtheorem{defi}{{\it Definición}}[section]
\newtheorem{teo}{{\it Teorema}}[section]
\newtheorem{ejemplo}{{\it Ejemplo}}[section]
\newtheorem{propiedad}{{\it Propiedad}}[section]
\newtheorem{lema}{{\it Lema}}[section]

\author{}
\title{Propiedades de la delta de Dirac \\ \large {Tema 2 - Matemáticas Avanzadas de la Física}  \vspace{-1.5\baselineskip}}
\date{}
\begin{document}
%\renewcommand\theenumii{\arabic{theenumii.enumii}}
\renewcommand\labelenumii{\theenumi.{\arabic{enumii}}}
\maketitle
\fontsize{14}{14}\selectfont
A continuación se presentan una lista con propiedades importantes y útiles de la delta de Dirac en la física matemática, es posible demostrar cada una de ellas a partir de la definición y propiedades de la teoría de las distribuciones.
\subsection*{Propiedades.}
\begin{multicols}{2}
\begin{enumerate}[itemsep=10pt,parsep=2pt, label=\roman*)]
\item $\displaystyle \int_{-\infty}^{+\infty} \delta (x) \: f(x) \dd{x} = f(0)$
\item $\delta (x) \: f(x) = \delta(x) \: f(0)$
\item $\delta (-x) = \delta(x)$
\item $\delta^{\prime} (x) = - \delta^{\prime}(x)$
\item $\delta (a \, x) =  \dfrac{\delta (x)}{\abs{a}}$
\item $\delta (x^{2} - a^{2}) = \dfrac{1}{2 \, a} [\delta (x + a) +  \delta (x-a)] \hspace{0.5cm} a > 0$
\item $\displaystyle \int_{-\infty}^{+\infty} \delta^{\prime} (x) f(x) \dd{x} = - f^{\prime} (0)$
\item $\delta (x - x^{\prime}) = 0, \hspace{0.5cm} x \neq x^{\prime}$
\item $\displaystyle \int_{a}^{b} \delta (x - x^{\prime}) \dd{x} = 1 \hspace{0.5cm} a < x < b$
\item $\delta (x - x^{\prime}) =  \delta (x^{\prime} - x)$
\item $\displaystyle \int_{-\infty}^{+\infty} \delta (x - x^{\prime}) f(x^{\prime}) \dd{x} = f(x)$
\item $\displaystyle \int_{-\infty}^{+\infty} \delta (x^{\prime \prime} - x^{\prime}) \delta (x^{\prime \prime} - x) \dd{x^{\prime \prime}} = \delta (x^{\prime} - x)$
\item $\displaystyle \delta (f(x)) = \sum_{i=1}^{N} \dfrac{\delta (x - x_{i})}{\abs{f^{\prime}(x)} \eval_{x=x_{i}}}$
\\
\bigskip
donde los $x_{i}$ son las raíces simples de $f(x)$.
\end{enumerate}
\end{multicols}
\newpage
\subsection*{Delta de Dirac en 3D}
Se define
\begin{align*}
\delta(\va{x} - \va{x^{\prime}}) = \delta(x - x^{\prime}) \,\delta(y - y^{\prime}) \, \delta(z - z^{\prime})
\end{align*}
la cual cumple con las siguientes propiedades:
\begin{enumerate}[itemsep=10pt,parsep=2pt, label=\roman*)]
\item $\displaystyle \int_{V} \delta(\va{x} - \va{x^{\prime}}) \, \dd[3]{x^{\prime}} = \begin{cases}
1 & \mbox{si } \va{x} \in V \\
0 & \mbox{si } \va{x} \not\in V \\
\end{cases}$
\item $\displaystyle \int_{R^{3}} \delta(\va{x} - \va{x^{\prime}}) \, \dd[3]{x^{\prime}} = 1$
\item $\displaystyle \int_{R^{3}} \delta(\va{x} - \va{x^{\prime}}) \, f(\va{x}^{\, \prime}) \, \dd[3]{x^{\prime}} = f(\va{x})$
\end{enumerate}
\subsection*{Delta de Dirac para un sistema curvilíneo.}
Consideremos un sistema coordenado curvilíneo con coordenadas $(\xi_{1}, \xi_{2}, \xi_{3})$ y con factores de escala
\begin{align*}
h_{i} = \left[ \left( \pdv{x}{\xi_{i}} \right)^{2} + \left( \pdv{y}{\xi_{i}} \right)^{2} + \left( \pdv{z}{\xi_{i}} \right)^{2}\right]^{1/2}
\end{align*}
Se expresa la delta de Dirac:
\begin{align*}
\delta(\va{r} - \va{r}_{0}) = \dfrac{\delta(\xi_{1} - \xi_{10})}{h_{1}} \: \dfrac{\delta(\xi_{2} - \xi_{20})}{h_{2}} \: \dfrac{\delta(\xi_{3} - \xi_{30})}{h_{3}}
\end{align*}
\end{document}