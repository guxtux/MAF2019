\documentclass[12pt]{article}
\usepackage[utf8]{inputenc}
\usepackage[spanish,es-lcroman, es-tabla]{babel}
\usepackage[autostyle,spanish=mexican]{csquotes}
\usepackage{amsmath}
\usepackage{amssymb}
\usepackage{nccmath}
\numberwithin{equation}{section}
\usepackage{amsthm}
\usepackage{graphicx}
\usepackage{epstopdf}
\DeclareGraphicsExtensions{.pdf,.png,.jpg,.eps}
\usepackage{color}
\usepackage{float}
\usepackage{multicol}
\usepackage{enumerate}
\usepackage[shortlabels]{enumitem}
\usepackage{anyfontsize}
\usepackage{anysize}
\usepackage{array}
\usepackage{multirow}
\usepackage{enumitem}
\usepackage{cancel}
\usepackage{tikz}
\usepackage{circuitikz}
\usepackage{tikz-3dplot}
\usetikzlibrary{babel}
\usetikzlibrary{shapes}
\usepackage{bm}
\usepackage{mathtools}
\usepackage{esvect}
\usepackage{hyperref}
\usepackage{relsize}
\usepackage{siunitx}
\usepackage{physics}
%\usepackage{biblatex}
\usepackage{standalone}
\usepackage{mathrsfs}
\usepackage{bigints}
\usepackage{bookmark}
\spanishdecimal{.}

\setlist[enumerate]{itemsep=0mm}

\renewcommand{\baselinestretch}{1.5}

\let\oldbibliography\thebibliography

\renewcommand{\thebibliography}[1]{\oldbibliography{#1}

\setlength{\itemsep}{0pt}}
%\marginsize{1.5cm}{1.5cm}{2cm}{2cm}


\newtheorem{defi}{{\it Definición}}[section]
\newtheorem{teo}{{\it Teorema}}[section]
\newtheorem{ejemplo}{{\it Ejemplo}}[section]
\newtheorem{propiedad}{{\it Propiedad}}[section]
\newtheorem{lema}{{\it Lema}}[section]

%\author{M. en C. Gustavo Contreras Mayén. \texttt{curso.fisica.comp@gmail.com}}
\title{ Series de potencias \\ \large {Matemáticas Avanzadas de la Física}}
\date{ }
\begin{document}
%\renewcommand\theenumii{\arabic{theenumii.enumii}}
\renewcommand\labelenumii{\theenumi.{\arabic{enumii}}}
\maketitle
\fontsize{14}{14}\selectfont
\section{Repaso de serie de potencias.}
\begin{enumerate}
\item Se dice que una serie de potencias
\[ \sum_{n=0}^{\infty} a_{n} (x - x_{0})^{n} \]
converge a un punto $x$ si
\[ \lim_{m \to \infty} \sum_{n=0}^{m} a_{n} (x - x_{0})^{n} \]
existe. La serie converge para $x = x_{0}$, pero puede converger para toda $x$ o puede hacerlo para algunos valores de $x$ y no para otros.
\item Se dice que la serie
\[ \sum_{n=0}^{\infty} a_{n} (x - x_{0})^{n} \]
converge absolutamente en un punto $x$ si la serie
\[ \sum_{n=0}^{\infty} \vert a_{n} (x - x_{0})^{n} \vert \]
converge. Se puede demostrar que si la serie converge absolutamnete, entonces la serie también converge, sin embargo, a la inversa no es necesariamente cierta.
\item La prueba de la razón se utiliza para averiguar si una serie de potencias converge o no absolutamente. Si $a_{n} \neq 0$ y si para un valor fijo de $x$
\[ \lim_{n \to \infty} \bigg| \dfrac{a_{n+1}(x - x_{0})^{n+1}}{a_{n} (x - x_{0})^{n}} \bigg| =  \vert x - x_{0} \vert \lim_{n \to \infty} \bigg| \dfrac{a_{n+1}}{a_{n}} \bigg| = l \]
entonces la serie de potencias converge absolutamente en ese valor de $x$ si $ l < 1$ y diverge si $l > 1$. Si $l = 1$, la prueba no es concluyente.
\item Si la serie de potencias
\[ \sum_{n=0}^{\infty} a_{n} (x - x_{0})^{n} \]
converge en $x =  x_{1}$, converge absolutamente para $ \vert x - x_{0} \vert < \vert x_{1} - x_{0} \vert $ y si diverge en $x = x_{1}$, diverge para $\vert x - x_{0} \vert > \vert x_{1} - x_{0} \vert $
\item Existe un número no negativo $\rho$ denominado \textbf{radio de convergencia}, tal que
\[ \sum_{n=0}^{\infty} a_{n} (x - x_{0})^{n} \]
converge absolutamente para $\vert x - x_{0} \vert < \rho$ y diverge para $\vert x - x_{0} \vert > \rho$.
\\
Para una serie que no converge en punto alguno excepto en $x_{0}$, $\rho$ se define como cero; para una serie que converge para toda $x$, se dice que $\rho$ es infinito. Si $\rho > 0$, entonces el intervalo $\vert x - x_{0} \vert < \rho$ se llama \textbf{intervalo de convergencia}. La serie puede converger o diverger cuando $\vert x - x_{0} \vert =  \rho$.
\begin{figure}[H]
\centering
\begin{tikzpicture}
	\draw [->] (0,0) -- (8,0) node [above, pos=0.99] {$x$};
	\draw (2,-0.2) -- (2,0.2);
	\draw [fill] (4,0) circle (0.1);
	\draw (2,-0.5) node {$x_{0} - \rho$};
	\draw (4,-0.5) node {$x_{0}$};
	\draw (6,-0.2) -- (6,0.2);
	\draw (6,-0.5) node {$x_{0} + \rho$};
	\draw (2,0.3) -- (2,1.2);
	\draw [<->] (0,1) -- (2,1) node [above, pos=0.1] {la serie diverge};
	\draw (6,0.3) -- (6,1.2);
	\draw [<->] (6,1) -- (8,1) node [above, pos=0.9] {la serie diverge};
	\draw [<->, align=left] (2,1) -- (6,1) node [above, pos=0.5] {la serie converge \\ absolutamente};
	
	
\end{tikzpicture}

\end{figure}






\end{enumerate}
\end{document}