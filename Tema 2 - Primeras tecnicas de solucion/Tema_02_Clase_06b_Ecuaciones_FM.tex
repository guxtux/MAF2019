\documentclass[12pt]{beamer}
\usepackage{../Estilos/BeamerMAF}
\input{../Preambulos/preambulo_Beamer_Cambridge_beaver}

\setbeamercolor{section in foot}{bg=carmine!70!white, fg=white}
\setbeamercolor{subsection in foot}{bg=bisque, fg=black}
\setbeamercolor{date in foot}{bg=goldenrod, fg=white}

\makeatletter
\setbeamertemplate{footline}
{
  \leavevmode%
  \hbox{%
  \begin{beamercolorbox}[wd=.333333\paperwidth,ht=2.25ex,dp=1ex,center]{section in foot}%
    \usebeamerfont{section in foot} \insertsection
  \end{beamercolorbox}%
  \begin{beamercolorbox}[wd=.333333\paperwidth,ht=2.25ex,dp=1ex,center]{subsection in foot}%
    \usebeamerfont{subsection in foot}  \insertsubsection
  \end{beamercolorbox}%
  \begin{beamercolorbox}[wd=.333333\paperwidth,ht=2.25ex,dp=1ex,right]{date in head/foot}%
    \usebeamerfont{date in head/foot} \insertshortdate{} \hspace*{2em}
    \insertframenumber{} / \inserttotalframenumber \hspace*{2ex} 
  \end{beamercolorbox}}%
  \vskip0pt%
}
\makeatother

\newenvironment{innerItemize}{%
  \begin{itemize}[<1->]%
}{%
  \end{itemize}%
}

\usefonttheme{serif}
\resetcounteronoverlays{saveenumi}

\date{3 de marzo de 2022}

\title{\large{Ecuaciones de la Física Matemática}}
\subtitle{Tema 2 - Primeras técnicas de solución}
\author{M. en C. Gustavo Contreras Mayén}


\begin{document}
\maketitle
\fontsize{14}{14}\selectfont
\spanishdecimal{.}

% %Ref. http://www.scholarpedia.org/article/Partial_differential_equation

\section{Ecuaciones en la Física Matemática}
\frame[allowframebreaks]{\tableofcontents[currentsection, hideothersubsections]}
\subsection{Ecuaciones lineales}

\begin{frame}
\frametitle{Ecuaciones de la Física Matemática}
Enumeremos algunas EDP lineales conocidas:
\pause
\setbeamercolor{item projected}{bg=blue!70!black,fg=yellow}
\setbeamertemplate{enumerate items}[circle]
\begin{enumerate}[<+->]
\item La ecuación de calor (ecuación parabólica):
\pause
\begin{align}
\text{\Large{$u_{t} - u_{xx} = 0$}}
\label{eq:ecuacion_S11}
\end{align}
donde las variables $t$ y $x$ juegan el papel de tiempo y una coordenada espacial, respectivamente. \pause Revisemos que en cuenta que la ecuación (\ref{eq:ecuacion_S11}) contiene solo un término con una derivada más alta.
\seti
\end{enumerate}
\end{frame}
\begin{frame}
\frametitle{La ecuación de calor}
La ecuación (\ref{eq:ecuacion_S11}) se encuentra a menudo en la teoría de la transferencia de calor y de masa.
\\
\bigskip
\pause
Describe procesos térmicos inestables unidimensionales en medios inactivos o sólidos con difusividad térmica constante. \pause Se utiliza una ecuación similar para estudiar los correspondientes procesos de intercambio de masa inestables unidimensionales con difusividad constante.
\end{frame}
\begin{frame}
\frametitle{Ecuaciones de la Física Matemática}
\setbeamercolor{item projected}{bg=blue!70!black,fg=yellow}
\setbeamertemplate{enumerate items}[circle]
\begin{enumerate}[<+->]
\conti
\item La ecuación de onda (ecuación hiperbólica):
\pause
\begin{align}
\text{\Large{$u_{tt} - u_{xx} = 0$}}
\label{eq:ecuacion_S12}
\end{align}
donde las variables $t$ y $x$ juegan el papel del tiempo y la coordenada espacial, respectivamente. \pause Revisemos que los términos de la derivada más alta en la ecuación (\ref{eq:ecuacion_S12}) difieren en un signo.
\seti
\end{enumerate}
\end{frame}
\begin{frame}
\frametitle{La ecuación de onda}
Esta ecuación también se conoce como \emph{ecuación de vibración de una cuerda}.
\\
\bigskip
\pause
A menudo se encuentra en elasticidad, aerodinámica, acústica y electrodinámica.
\end{frame}
\begin{frame}
\frametitle{Ecuaciones de la Física Matemática}
\setbeamercolor{item projected}{bg=blue!70!black,fg=yellow}
\setbeamertemplate{enumerate items}[circle]
\begin{enumerate}[<+->]
\conti
\item Ecuación de Laplace (ecuación elíptica):
\pause
\begin{align}
\text{\Large{$u_{xx} + u_{yy} = 0$}}
\label{eq:ecuacion_S14}
\end{align}
donde $x$ e $y$ juegan el papel de las coordenadas espaciales. \pause Tomemos en cuenta que los términos de la derivada más alta en la ecuación (\ref{eq:ecuacion_S14}) tienen signos similares.
\seti
\end{enumerate}
\end{frame}
\begin{frame}
\frametitle{La ecuación de Laplace}
La ecuación de Laplace a menudo se escribe brevemente como $\laplacian{u} = 0$, donde $\laplacian$ es el operador de Laplace o Laplaciano.
\\
\bigskip
\pause
La ecuación de Laplace se encuentra a menudo en la teoría de transferencia de calor y masa, mecánica de fluidos, elasticidad, electrostática y otras áreas de la mecánica y la física.
\end{frame}
\begin{frame}
\frametitle{La ecuación de Laplace}
Por ejemplo, en la teoría de transferencia de calor y masa, esta ecuación describe la distribución de temperatura en estado estable en ausencia de fuentes de calor y sumideros en el dominio en estudio.
\end{frame}

\subsection{Ecuaciones no lineales}

\begin{frame}
\frametitle{Ecuaciones no lineales}
\setbeamercolor{item projected}{bg=blue!70!black,fg=yellow}
\setbeamertemplate{enumerate items}[circle]
\begin{enumerate}[<+->]
\item Ecuación no lineal de calor:
\pause
\begin{align}
\pdv{u}{t} = \pdv{x} \left[ f(u) \, \pdv{u}{x} \right]
\label{eq:ecuacion_S27}  
\end{align}
Esta ecuación describe procesos térmicos inestables unidimensionales en medios o sólidos en reposo en el caso en que la difusividad térmica depende de la temperatura, $f (u)> 0$. 
\\
\bigskip
\pause
En el caso especial $f (w) \equiv 1$, la ecuación no lineal (\ref{eq:ecuacion_S27}) se convierte en la ecuación de calor lineal (\ref{eq:ecuacion_S11}).
\seti
\end{enumerate}
\end{frame}
\begin{frame}
\frametitle{Ecuaciones no lineales}
\setbeamercolor{item projected}{bg=blue!70!black,fg=yellow}
\setbeamertemplate{enumerate items}[circle]
\begin{enumerate}[<+->]
\conti
\item Ecuación Kolmogorov-Petrovskii-Piskunov:
\pause
\begin{align}
\pdv{u}{t} = a \, \pdv[2]{w}{x} + f(u), \hspace{1cm} a > 0
\label{eq:ecuacion_S28}
\end{align}
Las ecuaciones de esta forma se encuentran a menudo en varios problemas de transferencia de masa y calor (siendo $f$ la reacción de cambio del volumen en una reacción química), teoría de la combustión, biología y ecología.
\seti
\end{enumerate}
\end{frame}
\begin{frame}
\frametitle{Caso especial de la ec. KPP}
En el caso especial de $f (u) \equiv 0$ y $a = 1$, la ecuación no lineal (\ref{eq:ecuacion_S28}) se convierte en la ecuación de calor lineal (\ref{eq:ecuacion_S11}).
\\
\bigskip
\pause
Observación: La ecuación (\ref{eq:ecuacion_S28}) también se le conoce como \emph{ecuación de calor con una fuente no lineal}.
\end{frame}
\begin{frame}
\frametitle{Ecuaciones no lineales}
\setbeamercolor{item projected}{bg=blue!70!black,fg=yellow}
\setbeamertemplate{enumerate items}[circle]
\begin{enumerate}[<+->]
\conti
\item Ecuación de Burgers:
\pause
\begin{align}
\pdv{w}{t} + u \, \pdv{u}{x} = \pdv[2]{u}{x}
\label{eq:ecuacion_S29}
\end{align}
Se ocupa para describir procesos ondulatorios en la dinámica de gases, hidrodinámica, en acústica y para el flujo del tráfico.
\seti
\end{enumerate}
\end{frame}
\begin{frame}
\frametitle{Ecuaciones no lineales}
\setbeamercolor{item projected}{bg=blue!70!black,fg=yellow}
\setbeamertemplate{enumerate items}[circle]
\begin{enumerate}[<+->]
\conti
\item Ecuación de onda no lineal:
\pause
\begin{align}
\pdv[2]{u}{t} = \pdv{x} \left[ f(u) \, \pdv{u}{x} \right]
\label{eq:ecuacion_S30}
\end{align}
Esta ecuación se encuentra en dinámica de ondas y gases, con $f(u) > 0$. En el caso especial $f(u) \equiv 1$, la ecuación no lineal (\ref{eq:ecuacion_S30}) pasa a ser la ecuación de onda lineal (\ref{eq:ecuacion_S12}).
\seti
\end{enumerate}
\end{frame}
\begin{frame}
\frametitle{Ecuaciones no lineales}
\setbeamercolor{item projected}{bg=blue!70!black,fg=yellow}
\setbeamertemplate{enumerate items}[circle]
\begin{enumerate}[<+->]
\conti
\item Ecuación Klein - Gordon no lineal:
\pause
\begin{align}
\pdv[2]{u}{t} = a \, \pdv[2]{u}{x} + f(u),\hspace{1cm} a > 0
\label{eq:ecuacion_S31}
\end{align}
Las ecuaciones de esta forma surgen en geometría diferencial y diversas áreas de la física: superconductividad, dislocaciones en cristales, ondas en materiales ferromagnéticos, pulsos de láser en medios bifásicos, entre otros.
\\
\bigskip
\pause
Para $f (u) \equiv 0$ y $a = 1$, la ecuación coincide con la ecuación de onda lineal.
\seti
\end{enumerate}
\end{frame}
\begin{frame}
\frametitle{Ecuaciones no lineales}
\setbeamercolor{item projected}{bg=blue!70!black,fg=yellow}
\setbeamertemplate{enumerate items}[circle]
\begin{enumerate}[<+->]
\conti
\item Ecuación de Laplace no lineal:
\pause
\begin{align}
\pdv[2]{u}{x} + \pdv[2]{u}{y} = f(u)
\label{eq:ecuacion_S32}
\end{align}
A esta ecuación se le conoce también como \emph{ecuación estacionaria de calor con una fuente no lineal.}
\seti
\end{enumerate}
\end{frame}
\begin{frame}
\frametitle{Ecuaciones no lineales}
\setbeamercolor{item projected}{bg=blue!70!black,fg=yellow}
\setbeamertemplate{enumerate items}[circle]
\begin{enumerate}[<+->]
\conti
\item Ecuación Monge - Ampere:
\pause
\begin{align}
\left( \pdv[2]{u}{x}{y} \right)^{2} - \pdv[2]{u}{x} \, \pdv[2]{u}{y} = (x, y)
\label{eq:ecuacion_S33}
\end{align}
Esta ecuación se encuentra en geometría diferencial, dinámica de gases y meteorología.
\end{enumerate}
\end{frame}

\subsection{E L de derivadas de orden mayor}

\begin{frame}
\frametitle{Ecuaciones lineales de orden mayor}
\setbeamercolor{item projected}{bg=blue!70!black,fg=yellow}
\setbeamertemplate{enumerate items}[circle]
\begin{enumerate}[<+->]    
\item Ecuación de vibración de un varilla elástica:
\pause
\begin{align*}
\pdv[2]{w}{t} + a^{2} \, \pdv[4]{u}{x} = 0
\end{align*}
\seti
\end{enumerate}
\end{frame}
\begin{frame}
\frametitle{Ecuaciones lineales de orden mayor}
\setbeamercolor{item projected}{bg=blue!70!black,fg=yellow}
\setbeamertemplate{enumerate items}[circle]
\begin{enumerate}[<+->]    
\conti
\item La ecuación biarmónica:
\pause
\begin{align*}
\nabla^{4} = 0
\end{align*}
donde $\nabla^{4}$ es el operador biarmónico:
\begin{align*}
\nabla^{4} = \pdv[4]{x} + 2 \, \dfrac{\partial^{4}}{\partial x^{2} \, \partial y^{2}} + \pdv[4]{u}{y}
\end{align*}
\end{enumerate}
\end{frame}
\begin{frame}
\frametitle{La ecuación biarmónica}
La ecuación biarmónica se encuentra en problemas bidimensionales de elasticidad (como la función de tensión de Airy).
\\
\bigskip
\pause
También se utiliza para describir flujos lentos de fluidos viscosos incompresibles ($u$ es la función de corriente).
\end{frame}
\begin{frame}
\frametitle{Observación importante}
Las ecuaciones que se han mencionado no son todas en la física matemática, habrá otras ecuaciones que extenderán este listado.
\\
\bigskip
\pause
Dependiendo del área de interés que decidas abordar, lo más seguro es que te encuentres con nuevas ecuaciones, que esperamos logres identificar las características principales, así como un abordaje para su solución.
\end{frame}
\begin{frame}
\frametitle{Comenzando con la separación de variables}
En el siguiente material de trabajo se revisará la técnica de solución: \emph{separación de variables.}
\end{frame}
\end{document}