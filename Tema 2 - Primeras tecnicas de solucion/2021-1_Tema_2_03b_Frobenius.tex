\input{../Preambulos/preambulo_presentacion_CambridgeUS_beaver}
\title{\large{Ejercicio 2 con el método de Frobenius}}
\subtitle{Tema 2 - Primeras técnicas de solución}
\author{M. en C. Gustavo Contreras Mayén}
\date{}
\institute{Facultad de Ciencias - UNAM}
\titlegraphic{\includegraphics[width=1.75cm]{../Imagenes/escudo-facultad-ciencias}\hspace*{4.75cm}~%
   \includegraphics[width=1.75cm]{../Imagenes/escudo-unam}
}
\setbeamertemplate{navigation symbols}{}
\begin{document}
\maketitle
\fontsize{14}{14}\selectfont
\spanishdecimal{.}
\section*{Contenido}
\frame{\tableofcontents[currentsection, hideallsubsections]}
\section{Ejercicio de mecánica cuántica}
\frame{\tableofcontents[currentsection, hideothersubsections]}
\subsection{Planteamiento}
\begin{frame}
\frametitle{El problema de estudio}
A partir del estudio en mecánica cuántica del efecto Stark (en coordenadas parabólicas), nos conduce a la ecuación diferencial
\begin{align*}
\dv{\xi} \left( \xi\, \dv{u}{\xi} \right) + \left( \dfrac{1}{2} \, E \, \xi + \alpha - \dfrac{m^{2}}{4 \, \xi} - \dfrac{1}{4} \, F \, \xi^{2} \right) \, u = 0
\end{align*}
donde $\alpha$ es la constante de separación, $E$ es la energía total del sistema, $F$ es una constante, $Fz$ es la energía potencial que se agrega al introducir un campo eléctrico.
\end{frame}
\begin{frame}
\frametitle{El problema de estudio}
Usando la raíz más grande de la ecuación de índices, desarrolla una solución en series de potencias, alrededor de $\xi=0$.
\\
\bigskip
Evalúa los primeros tres coeficientes en términos de $a_{0}$.
\end{frame}
\begin{frame}
\frametitle{El resultado esperado}
Debemos de tomar en cuenta que lo que se nos pide es obtener lo siguiente:
\begin{align*}
u(\xi) = a_{0} \, \xi^{r} + a_{1} \, \xi ^{1+k} + a_{2} \, \xi ^{2+k} + \ldots 
\end{align*}
\end{frame}
\section{Solución}
\frame{\tableofcontents[currentsection, hideothersubsections]}
\subsection{Desarrollo}
\begin{frame}
\frametitle{Simplificando la expresión}
Inicialmente simplificamos la expresión que contiene la segunda derivada:
\pause
\begin{align*}
\xi \, \dv[2]{u}{\xi} + \dv{u}{\xi} + \left( \dfrac{1}{2} \, E \, \xi + \alpha - \dfrac{m^{2}}{4 \, \xi} - \dfrac{1}{4} \, F \, \xi^{2} \right) \, u = 0
\end{align*}
\end{frame}
\begin{frame}
\frametitle{Puntos singulares}
Al estudiar los puntos singulares de la ED, tenemos que llevar la expresión a la forma conocida:
\begin{align*}
\dfrac{1}{\xi} \, \xi^{2} \, \dv[2]{u}{\xi} + \dfrac{1}{\xi} \, \xi \, \dv{u}{\xi} + \left( \dfrac{1}{2} \, E \, \xi + \alpha - \dfrac{m^{2}}{4 \, \xi} - \dfrac{1}{4} \, F \, \xi^{2} \right) \, u = 0
\end{align*}
\pause
Simplicando la expresión, tenemos que:
\end{frame}
\begin{frame}
\frametitle{Puntos singulares}
\vspace*{-1cm}
\begin{align*}
\xi^{2} \, \dv[2]{u}{\xi} + \xi \, \dv{u}{\xi} + \left( \dfrac{1}{2} \, E +  \dfrac{\alpha}{\xi} - \dfrac{m^{2}}{4 \, \xi^{2}} - \dfrac{1}{4} \, F \, \xi \right) \, u = 0
\end{align*}
\pause
Por lo que las funciones $p(\xi)$ y $q(\xi)$ son:
\end{frame}
\begin{frame}
\frametitle{Puntos singulares}
\begin{align*}
p(\xi) &= 1 \\[0.5em]
q(\xi) &= \left( \dfrac{1}{2} \, E +  \dfrac{\alpha}{\xi} - \dfrac{m^{2}}{4 \, \xi^{2}} - \dfrac{1}{4} \, F \, \xi \right)
\end{align*}
\pause
Ahora tomamos el límite cuando $\xi \to 0$ en ambas funciones.
\end{frame}
\begin{frame}
\frametitle{Puntos singulares}
\begin{align*}
\lim_{\xi \to 0} p(\xi) &= 1 \\[0.5em]
\lim_{\xi \to 0} q(\xi) &= \dfrac{E}{2} +  \dfrac{\alpha}{\xi} - \dfrac{m^{2}}{4 \, \xi^{2}} - \dfrac{1}{4} \, F \, \xi
\end{align*}
\pause
El límite para $p(\xi)$ existe, pero el límite de $q(\xi)$:
\begin{align*}
\lim_{\xi \to 0} q(\xi) = \dfrac{E}{2} - \lim_{\xi \to 0} \dfrac{m^{2}}{4 \, \xi^{2}} 
\end{align*}
\end{frame}
\begin{frame}
\frametitle{Singularidad irregular}
Aunque ocupemos la regla de L'Hôpital, tendremos entonces que:
\begin{align*}
\lim_{\xi \to 0} \dfrac{m^{2}}{4 \, \xi^{2}} = \lim_{\xi \to 0} \dfrac{0}{8 \, \xi} \to \infty
\end{align*}
\pause
Por lo que tendremos que $\xi = 0$ es un punto singular irregular. 
\end{frame}
\begin{frame}
\frametitle{Resolviendo el problema}
Veremos la bondad del método de Frobenius, ya que será posible remover esa singularidad y presentar una solución a la ecuación diferencial.
\end{frame}
\begin{frame}
\frametitle{Solución propuesta}
Proponemos una solución $u(\xi)$ como una serie de potencias:
\begin{align*}
u(\xi) = \sum_{n=0}^{\infty} a_{n} \, \xi^{n+k}
\end{align*}
\pause
Procedemos a calcular la primera y segunda derivada de $u(\xi)$.
\end{frame}
\begin{frame}
\frametitle{Derivadas de $u(\xi)$}
Las derivadas son:
\begin{align*}
\ptilde{u} &= \sum_{n=0}^{\infty} a_{n} \, (n + k) \, \xi^{n+k-1} \\[1em]
\stilde{u} &= \sum_{n=0}^{\infty} a_{n} \, (n + k) \, (n + k + 1) \, \xi^{n+k-2}
\end{align*}
\end{frame}
\begin{frame}
\frametitle{Sustitución en la ED}
Sustituimos las derivadas en la ED inicial
\begin{align*}
&\xi \, \left[ \sum_{n=0}^{\infty} a_{n} \, (n {+} k) \, (n {+} k {+} 1) \, \xi^{n{+}k{-}2} \right] + \\[1em]
&+ \sum_{n=0}^{\infty} a_{n} \, (n {+} k) \, \xi^{n{+}k{-}1} + \\[1em]
&+ \left( \dfrac{1}{2} \, E \, \xi {+} \alpha {-} \dfrac{m^{2}}{4 \, \xi} {-} \dfrac{1}{4} \, F \, \xi^{2} \right) \, \sum_{n=0}^{\infty} a_{n} \, \xi^{n{+}k} = 0
\end{align*}
\end{frame}
\begin{frame}
\frametitle{Simplificando la expresión}
Con la finalidad de reducir términos y simplificar la expresión, multiplicamos donde aparezca $\xi$ con la respectiva potencia.
\\
\bigskip
Además de que separamos cada término.
\end{frame}
\begin{frame}
\frametitle{Ecuación simplificada}
\vspace{-1cm}
\begin{eqnarray*}
&{}& \sum_{n=0}^{\infty} a_{n} (n {+} k) (n {+} k {+} 1) \xi^{n{+}k{-}1} + \\[0.5em] \pause
&+& \sum_{n=0}^{\infty} a_{n} (n {+} k) \xi^{n{+}k{-}1} + \pause \sum_{n=0}^{\infty} \dfrac{E}{2} \, a_{n} \, \xi^{n{+}k{+}1} + \\[0.5em] \pause
&+& \sum_{n=0}^{\infty} \alpha \, a_{n} \, \xi^{n{+}k} - \pause \sum_{n=0}^{\infty}  \dfrac{m^{2}}{4} \, a_{n} \, \xi^{n{+}k{-}1} + \\[0.5em]\pause
&-& \sum_{n=0}^{\infty} \dfrac{F}{4} \, a_{n} \, \xi^{n{+}k{+}2} = 0
\end{eqnarray*}
\end{frame}
\begin{frame}
\frametitle{Factorizando}
Vemos que hay tres términos con la potencia $\xi^{n+k-1}$, por lo que los factorizamos en una sola suma.
\end{frame}
\begin{frame}
\frametitle{Ecuación simplificada}
\vspace{-1cm}
\begin{eqnarray*}
&{}& \sum_{n=0}^{\infty} a_{n} \, (n {+} k) \, (n {+} k {+} 1) \, \xi^{n{+}k{-}1} + \\[0.5em]
&+& \sum_{n=0}^{\infty} a_{n} \, (n {+} k) \, \xi^{n{+}k{-}1} + \sum_{n=0}^{\infty} \dfrac{E}{2} \, a_{n} \, \xi^{n{+}k{+}1} + \\[0.5em]
&+& \sum_{n=0}^{\infty} \alpha \, a_{n} \, \xi^{n{+}k} - \sum_{n=0}^{\infty}  \dfrac{m^{2}}{4} \, a_{n} \, \xi^{n{+}k{-}1} + \\[0.5em]
&-& \sum_{n=0}^{\infty} \dfrac{F}{4} \, a_{n} \, \xi^{n{+}k{+}2} = 0
\end{eqnarray*}
\begin{tikzpicture}[overlay]
\draw[fill=red, opacity=0.3] (1.8, 5.9) rectangle (9, 7.4);
\draw[fill=red, opacity=0.3] (1.8, 4) rectangle (6.65, 5.6);
\draw[fill=red, opacity=0.3] (5.4, 2.2) rectangle (9.25, 3.8);
\end{tikzpicture}
\end{frame}
\begin{frame}
\frametitle{Factorizando}
Luego de factorizar los términos comunes a una potencia,  es momento de ordenar los términos de la potencia menor a la mayor.
\end{frame}
\begin{frame}
\frametitle{Nueva simplificación}
\vspace{-1cm}
\begin{eqnarray*}
&{}& \sum_{n=0}^{\infty} \bigg[ \big( a_{n} \, (n {+} k) \, (n {+} k {+} 1) \big) + \\[0.5em]
&+& \big( a_{n} \, (n {+} k) \big) {+} a_{n} \, \left( \dfrac{m^{2}}{4} \right) \bigg] \, \xi^{n+k-1} + \\[0.5em] \pause
&+& \sum_{n=0}^{\infty} \alpha \, a_{n} \, \xi^{n+k} + \pause \sum_{n=0}^{\infty} \dfrac{E}{2} \, a_{n} \, \xi^{n+k+1} + \\[0.5em] \pause
&-& \sum_{n=0}^{\infty} \dfrac{F}{4} \, a_{n} \, \xi^{n+k+2} = 0
\end{eqnarray*}
\end{frame}
\begin{frame}
\frametitle{Simplificando nuevamente}
Vamos a simplificar los términos de la potencia $\xi^{n+k-1}$, para reducir al máximo la expresión.
\end{frame}
\begin{frame}
\frametitle{Nueva simplificación}
\vspace{-1cm}
\begin{eqnarray*}
&{}& \sum_{n=0}^{\infty} \bigg[ \big( a_{n} \, (n {+} k) \, (n {+} k {+} 1) \big) + \\[0.5em]
&+& \big( a_{n} \, (n {+} k) \big) {+} a_{n} \, \left( \dfrac{m^{2}}{4} \right) \bigg] \, \xi^{n+k-1} + \\[0.5em] 
&+& \sum_{n=0}^{\infty} \alpha \, a_{n} \, \xi^{n+k} + \sum_{n=0}^{\infty} \dfrac{E}{2} \, a_{n} \, \xi^{n+k+1} + \\[0.5em] 
&-& \sum_{n=0}^{\infty} \dfrac{F}{4} \, a_{n} \, \xi^{n+k+2} = 0
\end{eqnarray*}
\begin{tikzpicture}[overlay]
   \draw[fill=yellow, opacity=0.3] (1.8, 4) rectangle (10.4, 7.5);
\end{tikzpicture}
\end{frame}
\begin{frame}
\frametitle{Simplificando otra vez}
\vspace{-1cm}
\begin{eqnarray*}
&{}& \bigg[ \big( a_{n} (n {+} k) (n {+} k {+} 1) \big) {+} \big( a_{n} (n {+} k) \big) {+} a_{n} \left( \dfrac{m^{2}}{4} \right) \bigg] = \\[0.5em] \pause
&=& a_{n} \, \bigg[  (n {+} k) \, (n {+} k {-} 1) {+} (n {+} k) {-} \dfrac{m^{2}}{4} \bigg] = \\[0.5em] \pause
&=& a_{n} \, \bigg[  (n {+} k) \, (n {+} k {-} 1 {+} {1}) {-} \dfrac{m^{2}}{4} \bigg] = \\[0.5em] \pause
&=& a_{n} \, \bigg[  (n {+} k)^{2} {-} \dfrac{m^{2}}{4} \bigg]
\end{eqnarray*}
\end{frame}
\begin{frame}
\frametitle{Regresemos a la expresión}
Una vez simplificado el término, lo regresamos a la suma y continuamos en el desarrollo.
\end{frame}
\begin{frame}
\frametitle{Expresión simplificada}
\vspace{-1cm}
\begin{align*}
&{} \sum_{n=0}^{\infty} \left[ a_{n} \, \left( (n {+} k)^{2} {-} \dfrac{m^{2}}{4} \right) \right] \, \xi^{n+k-1} + \\[0.5em] 
&+ \sum_{n=0}^{\infty} \alpha \, a_{n} \, \xi^{n+k} + \sum_{n=0}^{\infty} \dfrac{E}{2} \, a_{n} \, \xi^{n+k+1} + \\[0.5em] 
&- \sum_{n=0}^{\infty} \dfrac{F}{4} \, a_{n} \, \xi^{n+k+2} = 0
\end{align*}
\end{frame}
\begin{frame}
\frametitle{Simplificación máxima}
Hemos realizado la máxima simplificación para los términos de la potencia común, además de tener ordenada la expresión de la potencia de menor valor a el mayor valor.
% \\
% \bigskip
% Procederemos a calcular la \emph{ecuación de índices.}
\end{frame}
\begin{frame}
\frametitle{Paso modificado}
En otro caso con un punto singular regular, podríamos revisar el paso para obtener los coeficientes de la potencia más baja de la expresión, pero en este caso haremos una modificación del procedimiento.
\end{frame}
\begin{frame}
\frametitle{Paso modificado}
Obtendremos la ecuación de índices de la potencia más baja, pero ahoremos un desarrollo con toda la expresión ocupando la raíz más alta, para que de esta manera obtengamos los términos necesarios para resolver el ejercicio.
\end{frame}
\subsection{Ecuación de índices}
\begin{frame}
\frametitle{Potencia menor}
Sabemos que para obtener la ecuación de índices, debemos de considerar el término con la potencia menor, en este caso $\xi^{n+k-1}$
\pause
\begin{align*}
a_{n} \left[ (n {+} k)^{2} {-} \dfrac{m^{2}}{4} \right] 
\end{align*}
\end{frame}
\begin{frame}
\frametitle{Ecuación de índices}
Sabemos que los todos los coeficientes de la serie se anulan y además $a_{0}$, por tanto:
\begin{eqnarray*}
&a_{0} \left( k^{2} {-} \dfrac{m^{2}}{4} \right) = 0 \\[1em] \pause
\Longrightarrow \hspace{0.2cm} & \addtolength{\fboxsep}{5pt}\boxed{ \left( k^{2} {-} \dfrac{m^{2}}{4} \right) = 0}
\end{eqnarray*}
\end{frame}
% \begin{frame}
% \frametitle{Ecuación de índices}
% \vspace{-1cm}
% \begin{eqnarray*}
% &a_{0}& \!\! \left( k^{2} {-} \dfrac{m^{2}}{4} \right) \xi^{k} +  \pause \sum_{n=1}^{\infty} \left[ a_{n} \left( (n {+} k)^{2} {-} \dfrac{m^{2}}{4} \right) \right] \xi^{n+k-1} + \\[0.5em] \pause
% &+& \sum_{n=0}^{\infty} \alpha \, a_{n} \, \xi^{n+k} + \sum_{n=0}^{\infty} \dfrac{E}{2} \, a_{n} \, \xi^{n+k+1} + \\[0.5em] 
% &-& \sum_{n=0}^{\infty} \dfrac{F}{4} \, a_{n} \, \xi^{n+k+2} = 0
% \end{eqnarray*}
% \end{frame}
% \begin{frame}
% \frametitle{Ecuación de índices}
% Sabemos que todos los coeficientes de las potencias de $\xi$ se anulan y que $a_{0}$, por lo que:
% \begin{eqnarray*}
% &a_{0} \left( k^{2} {-} \dfrac{m^{2}}{4} \right) = 0 \\[1em] \pause
% \Longrightarrow \hspace{0.2cm} & \addtolength{\fboxsep}{5pt}\boxed{ \left( k^{2} {-} \dfrac{m^{2}}{4} \right) = 0}
% \end{eqnarray*}
% Es la ecuación de índices buscada.
% \end{frame}
\begin{frame}
\frametitle{Raíces de la ecuación}
Ahora determinamos el valor de las raíces de la ecuación:
\begin{align*}
k^{2} {-} \dfrac{m^{2}}{4} = 0 \hspace{0.5cm} \Rightarrow \hspace{0.5cm} k_{1} = \dfrac{m}{2} \hspace{0.5cm} k_{2} = -\dfrac{m}{2}
\end{align*}
\pause
El enunciado nos pide que ocupemos la raíz más grande, en este caso: $k_{1} = m/2$
\end{frame}
\subsection{Tres primeros coeficientes}
\begin{frame}
\frametitle{Observación importante}
El enunciado nos pide que calculemos los tres primeros términos de la solución en serie de potencias, por lo que no será necesario determinar la \emph{regla de recurrencia} en este ejercicio.
\\
\bigskip
Ocuparemos el valor de $k_{1}$ en la serie que tenemos:
\end{frame}
\begin{frame}
\frametitle{La expresión con la raíz}
\vspace{-1cm}
\begin{eqnarray*}
&{}& \sum_{n=0}^{\infty} \left[ a_{n} \left( \left(n {+} \dfrac{m}{2} \right)^{2} {-} \dfrac{m^{2}}{4} \right) \right] \xi^{n+\frac{m}{2}-1} + \\[0.5em] \pause
&+& \sum_{n=0}^{\infty} \alpha \, a_{n} \, \xi^{n+\frac{m}{2}} + \sum_{n=0}^{\infty} \dfrac{E}{2} \, a_{n} \, \xi^{n+\frac{m}{2}+1} + \\[0.5em] 
&-& \sum_{n=0}^{\infty} \dfrac{F}{4} \, a_{n} \, \xi^{n+\frac{m}{2}+2} = 0
\end{eqnarray*}
\end{frame}
\begin{frame}
\frametitle{Simplificando de nuevo}
Para tener una mayor legibilidad en la ecuación, podemos factorizar una potencia $\xi^{m/2}$, y continuar el desarrollo para determinar los dos coeficientes restantes.
\end{frame}
\begin{frame}
\frametitle{La expresión con la raíz}
\vspace{-1cm}
\begin{eqnarray*}
&{}& \xi^{m/2} \left\{  \sum_{n=0}^{\infty} \left[ a_{n} \left( \left(n {+} \dfrac{m}{2} \right)^{2} {-} \dfrac{m^{2}}{4} \right) \right] \xi^{n-1} +  \right. \\[0.5em] \pause
&+& \sum_{n=0}^{\infty} \alpha \, a_{n} \, \xi^{n} + \sum_{n=0}^{\infty} \dfrac{E}{2} \, a_{n} \, \xi^{n+1} + \\[0.5em] 
&-& \left. \sum_{n=0}^{\infty} \dfrac{F}{4} \, a_{n} \, \xi^{n+2} \right\} = 0
\end{eqnarray*}
\end{frame}
\begin{frame}
\frametitle{Simplificando la expresión}
Simplificamos el coeficiente de la potencia más baja, para facilitar el desarrollo.
% \vspace{-1cm}
\begin{eqnarray*}
&{}& \xi^{m/2} \left\{  \sum_{n=0}^{\infty} \left[ a_{n} \left( n^{2} {+} n m {+} \dfrac{m^{2}}{4} {-} \dfrac{m^{2}}{4} \right) \right] \xi^{n-1} +  \right. \\[0.5em] \pause
&+& \sum_{n=0}^{\infty} \alpha \, a_{n} \, \xi^{n} + \sum_{n=0}^{\infty} \dfrac{E}{2} \, a_{n} \, \xi^{n+1} + \\[0.5em] 
&-& \left. \sum_{n=0}^{\infty} \dfrac{F}{4} \, a_{n} \, \xi^{n+2} \right\} = 0
\end{eqnarray*}
\end{frame}
\begin{frame}
\frametitle{Ecuación simplificada}
\vspace{-1cm}
\begin{eqnarray*}
&{}& \xi^{m/2} \left\{  \sum_{n=0}^{\infty} \left[ a_{n} \left( n^{2} {+} n \, m \right) \right] \xi^{n-1} +  \right. \\[0.5em] \pause
&+& \sum_{n=0}^{\infty} \alpha \, a_{n} \, \xi^{n} + \sum_{n=0}^{\infty} \dfrac{E}{2} \, a_{n} \, \xi^{n+1} + \\[0.5em] 
&-& \left. \sum_{n=0}^{\infty} \dfrac{F}{4} \, a_{n} \, \xi^{n+2} \right\} = 0
\end{eqnarray*}
\end{frame}
% \begin{frame}
% \frametitle{Recorriendo índices}
% Recordemos que si queremos factorizar nuevamente términos de potencias en común, el índice de las sumas debe de iniciar en el mismo valor, tenemos que para la primera suma, el índice es $n=1$, por lo que debemos de recorrer dicho índice.
% \end{frame}
\begin{frame}
\frametitle{Expansión de la serie}
Ahora expandimos cada término de la serie hasta $\xi^{2}$, para revisar entonces lo que ocurre con los coeficientes.
\end{frame}
\begin{frame}
\frametitle{Hacemos el desarrollo}
\begin{eqnarray*}
&a_{0}& \big[ 0^{2} + (0)(m) \big] \xi^{-1} + \alpha \, a_{0} \, \xi^{0} + \dfrac{E}{2} a_{0} \xi^{1} - \dfrac{F}{4} a_{0} \, \xi^{2} + \\[0.5em] \pause
&a_{1}& \big[ 1^{2} + (1)(m) \big] \xi^{0} + \alpha \, a_{1} \, \xi^{1} + \dfrac{E}{2} a_{1} \xi^{2} - \dfrac{F}{4} a_{1} \, \xi^{3} + \\[0.5em] \pause
&a_{2}& \big[ 2^{2} + (2)(m) \big] \xi^{1} + \alpha \, a_{2} \, \xi^{2} + \dfrac{E}{2} a_{2} \xi^{3} - \dfrac{F}{4} a_{2} \, \xi^{4}
\end{eqnarray*}
\end{frame}
\begin{frame}
\frametitle{Removiendo la singularidad}
Vemos que el coeficiente de la potencia $\xi^{-1}$ se cancela, que es donde teníamos el problema de que es divergente la serie, pero al cancelarse el término, los restantes son válidos.
\end{frame}
\begin{frame}
\frametitle{Agrupando términos}
Ahora agrupamos los términos de acuerdo a la potencia que tienen, recordemos que los términos de potencia mayor a $\xi^{2}$ no son de interés para nuestra solución.
\end{frame}
\begin{frame}
\frametitle{Agrupando términos}
\vspace*{-1cm}
\begin{align*}
&\bigg[ \alpha \, a_{0} + a_{1} \, m (m+1)\bigg] \xi^{0} + \\[0.5em]
&+ \bigg[ 2 (m + 2) \, a_{2} + \dfrac{E}{2} \, a_{0} + \alpha \, a_{1} \bigg] \xi + \\[0.5em]
&+ \bigg[\dfrac{E}{2} \, a_{1} + \alpha \, a_{2} - \dfrac{F}{4} a_{0} \bigg] \xi^{2} = 0
\end{align*}
\end{frame}
\begin{frame}
\frametitle{Obteniendo los coeficientes}
De la ecuación anterior es posible obtener los coeficientes $a_{1}$ y $a_{2}$ en términos de $a_{0}$:
\begin{align*}
a_{1} &= - \dfrac{\alpha \, a_{0}}{(1{+}m)} \\[1em]
a_{2} &=  \dfrac{{-}\alpha \, a_{1} {+} \dfrac{E}{2} \, a_{0}}{2(2{+}m)}
\end{align*}
\end{frame}
\begin{frame}
\frametitle{Obteniendo los coeficientes}
Sustituyendo el valor de $a_{1}$
\begin{align*}
a_{2} =  \left[ \dfrac{\alpha^{2}}{2(1+m)(2+m)} - \dfrac{E}{4(1+m)(2+m)} \right] \, a_{0}
\end{align*}
\end{frame}
% \begin{frame}
% \frametitle{El índice recorrido}
% \vspace{-1cm}
% \begin{eqnarray*}
% &{}& \xi^{m/2} \left\{  \sum_{n=0}^{\infty} \left[ a_{n+1} \left( \left(n+1 {+} \dfrac{m}{2} \right)^{2} {-} \dfrac{m^{2}}{4} \right) \right] \xi^{n} +  \right. \\[0.5em] \pause
% &+& \sum_{n=0}^{\infty} \alpha \, a_{n} \, \xi^{n} + \sum_{n=0}^{\infty} \dfrac{E}{2} \, a_{n} \, \xi^{n+1} + \\[0.5em] 
% &-& \left. \sum_{n=0}^{\infty} \dfrac{F}{4} \, a_{n} \, \xi^{n+2} \right\} = 0
% \end{eqnarray*}
% \end{frame}
% \begin{frame}
% \frametitle{Factorizando}
% Ahora tenemos dos términos de la expresión que podemos factorizar y reducir.
% \end{frame}
% \begin{frame}
% \frametitle{El índice recorrido}
% \vspace{-1cm}
% \begin{eqnarray*}
% &{}& \xi^{m/2} \left\{  \sum_{n=0}^{\infty} \left[ a_{n+1} \left( \left(n+1 {+} \dfrac{m}{2} \right)^{2} {-} \dfrac{m^{2}}{4} \right) \right] \xi^{n} +  \right. \\[0.5em]
% &+& \sum_{n=0}^{\infty} \alpha \, a_{n} \, \xi^{n} + \sum_{n=0}^{\infty} \dfrac{E}{2} \, a_{n} \, \xi^{n+1} + \\[0.5em] 
% &-& \left. \sum_{n=0}^{\infty} \dfrac{F}{4} \, a_{n} \, \xi^{n+2} \right\} = 0
% \end{eqnarray*}
% \begin{tikzpicture}[overlay]
% \draw[fill=blue, opacity=0.2] (3.1, 4.2) rectangle (10.8, 5.7);
% \draw[fill=blue, opacity=0.2] (1.1, 2.3) rectangle (4, 3.9);
% \end{tikzpicture}
% \end{frame}
% \begin{frame}
% \frametitle{Términos agrupados}
% \vspace{-1cm}
% \begin{eqnarray*}
% &{}& \xi^{m/2} \bigg\{  \\[0.5em]
% &{}& \sum_{n=0}^{\infty} \left[ a_{n{+}1} \left[ \left(n{+}1 {+} \dfrac{m}{2} \right)^{2} {-} \dfrac{m^{2}}{4} \right] {+} \alpha \, a_{n}   \right] \xi^{n} +  \\[0.5em] \pause
% &+& \left. \sum_{n=0}^{\infty} \dfrac{E}{2} \, a_{n} \, \xi^{n+1} - \sum_{n=0}^{\infty} \dfrac{F}{4} \, a_{n} \, \xi^{n+2} \right\} = 0
% \end{eqnarray*}
% \end{frame}
% \begin{frame}
% \frametitle{Obteniendo el segundo coeficiente}
% El segundo coeficiente se obtiene de la suma con menor potencia haciendo que $n=0$.
% \end{frame}
% \begin{frame}
% \frametitle{Segundo coeficiente}
% \vspace{-1cm}
% \begin{eqnarray*}
% &{}& \xi^{m/2} \left\{ \left( a_{1} \left[ \left( 1 {+} \dfrac{m}{2} \right)^{2} {-} \dfrac{m^{2}}{4} \right] {+} \alpha \, a_{0} \right) \right. + \\[0.5em] \pause
% &{+}& \sum_{n=1}^{\infty} \left[ a_{n{+}1} \left[ \left(n{+}1 {+} \dfrac{m}{2} \right)^{2} {-} \dfrac{m^{2}}{4} \right] {+} \alpha \, a_{n}   \right] \xi^{n} +  \\[0.5em] \pause
% &+& \left. \sum_{n=0}^{\infty} \dfrac{E}{2} \, a_{n} \, \xi^{n+1} - \sum_{n=0}^{\infty} \dfrac{F}{4} \, a_{n} \, \xi^{n+2} \right\} = 0
% \end{eqnarray*}
% \end{frame}
% \begin{frame}
% \frametitle{Anulando coeficientes}
% Volvemos a utilizar la premisa de que todos los coeficientes de la serie de potencias se anulan, por lo que:
% \pause
% \begin{align*}
% a_{1} \left[ \left( 1 {+} \dfrac{m}{2} \right)^{2} {-} \dfrac{m^{2}}{4} \right] {+} \alpha \, a_{0} = 0
% \end{align*}
% entonces simplificamos la expresión y así conocer el valor de $a_{1}$
% \end{frame}
% \begin{frame}
% \frametitle{Simplificando coeficientes}
% \begin{eqnarray*}
% &{}& a_{1} \left[ \left( 1 {+} \dfrac{m}{2} \right)^{2} {-} \dfrac{m^{2}}{4} \right] {+} \alpha \, a_{0} = \\[0.5em] \pause
% &=& a_{1} \left[ 1 {+} m {+} \dfrac{m^{2}}{4} {-} \dfrac{m^{2}}{4} \right] {+} \alpha \, a_{0} = \\[0.5em] \pause
% &=& a_{1} \left( 1 {+} m \right) {+} \alpha \, a_{0} = 0 \\[0.5em] \pause
% &\Rightarrow& a_{1} = - \dfrac{\alpha \, a_{0}}{(1{+}m)}
% \end{eqnarray*}
% \end{frame}
% \begin{frame}
% \frametitle{Para el tercer coeficiente}
% Regresamos a la expresión para determinar el tercer coeficiente, pero notamos que el índice de la suma inicia en $n=1$, por lo que recorremos el índice y vemos si es posible factorizar términos.
% \end{frame}
% \begin{frame}
% \frametitle{Para el tercer coeficiente}
% \vspace{-1cm}
% \begin{eqnarray*}
% &{}& \xi^{m/2} \left\{ \sum_{n=1}^{\infty} \left[ a_{n{+}1} \left[ \left(n{+}1 {+} \dfrac{m}{2} \right)^{2} {-} \dfrac{m^{2}}{4} \right] {+} \alpha \, a_{n} \right] \xi^{n} + \right. \\[0.5em] \pause
% &+& \left. \sum_{n=0}^{\infty} \dfrac{E}{2} \, a_{n} \, \xi^{n+1} - \sum_{n=0}^{\infty} \dfrac{F}{4} \, a_{n} \, \xi^{n+2} \right\} = 0
% \end{eqnarray*}
% Recorremos el índice de $n=1$ a $n=0$
% \end{frame}
% \begin{frame}
% \frametitle{Recorriendo el índice}
% \vspace{-1cm}
% \begin{eqnarray*}
% &{}& \xi^{m/2} \left\{ \sum_{n=0}^{\infty} \left[ a_{n{+}2} \! \left[ \left(n{+}2 {+} \dfrac{m}{2} \right)^{2} \! {-} \dfrac{m^{2}}{4} \right] {+} \alpha \, a_{n+1} \right] \xi^{n+1} + \right. \\[0.5em] \pause
% &+& \left. \sum_{n=0}^{\infty} \dfrac{E}{2} \, a_{n} \, \xi^{n+1} - \sum_{n=0}^{\infty} \dfrac{F}{4} \, a_{n} \, \xi^{n+2} \right\} = 0
% \end{eqnarray*}
% \end{frame}
% \begin{frame}
% \frametitle{Factorizando términos}
% Encontramos una potencia en común $\xi^{n+1}$ por lo que podemos factorizar nuevamente.
% \end{frame}
% \begin{frame}
% \frametitle{Factorizando términos}
% \vspace{-1cm}
% \begin{align*}
% &\xi^{m/2} \left\{ \sum_{n=0}^{\infty} \left[ a_{n{+}2} \! \left[ \left(n{+}2 {+} \dfrac{m}{2} \right)^{2} \! {-} \dfrac{m^{2}}{4} \right] {+} \alpha \, a_{n+1} \right] \xi^{n+1} + \right. \\[0.5em] \pause
% &+ \left. \sum_{n=0}^{\infty} \dfrac{E}{2} \, a_{n} \, \xi^{n+1} - \sum_{n=0}^{\infty} \dfrac{F}{4} \, a_{n} \, \xi^{n+2} \right\} = 0
% \end{align*}
% \begin{tikzpicture}[overlay]
% \draw[fill=cadetblue, opacity=0.2] (1.7, 2.2) rectangle (11.3, 4);
% \draw[fill=cadetblue, opacity=0.2] (0.8, 0.5) rectangle (4, 2.1);
% \end{tikzpicture}
% \end{frame}
% \begin{frame}
% \frametitle{Factorizando términos}
% \vspace{-1cm}
% \begin{eqnarray*}
% &\xi^{m/2}& \left\{ \sum_{n=0}^{\infty} \bigg[ a_{n{+}2} \! \left[ \left(n{+}2 {+} \dfrac{m}{2} \right)^{2} \! {-} \dfrac{m^{2}}{4} \right] + \right. \\[0.5em]
% &{+}& \! \alpha \, a_{n+1} {+} \dfrac{E}{2} \, a_{n} \bigg] \xi^{n+1} + \\[0.5em] \pause
% &-& \! \left. \sum_{n=0}^{\infty} \dfrac{F}{4} \, a_{n} \, \xi^{n+2} \right\} = 0
% \end{eqnarray*}
% \end{frame}
% \begin{frame}
% \frametitle{Anulando coeficientes}
% Nuevamente tomamos el hecho de que todos los coeficientes de la suma se deben de anular, por lo que en la potencia más baja hacemos que $n = 0$
% \end{frame}
% \begin{frame}
% \frametitle{Anulando coeficientes}
% \vspace{-1cm}
% \begin{eqnarray*}
% &\xi^{m/2}& \left\{ a_{2} \! \left[ \left(2 {+} \dfrac{m}{2} \right)^{2} \! {-} \dfrac{m^{2}}{4} \right] + \alpha \, a_{1} {+} \dfrac{E}{2} \, a_{0} + \right. \\[0.5em] \pause
% &+& \!\sum_{n=0}^{\infty} \bigg[ a_{n{+}2} \! \left[ \left(n{+}2 {+} \dfrac{m}{2} \right)^{2} \! {-} \dfrac{m^{2}}{4} \right] + \\[0.5em]
% &{+}& \! \alpha \, a_{n+1} {+} \dfrac{E}{2} \, a_{n} \bigg] \xi^{n+1} + \\[0.5em] \pause
% &-& \! \left. \sum_{n=0}^{\infty} \dfrac{F}{4} \, a_{n} \, \xi^{n+2} \right\} = 0
% \end{eqnarray*}
% \end{frame}
% \begin{frame}
% \frametitle{Tercer coeficiente}
% Tenemos entonces que:
% \begin{eqnarray*}
% &a_{2}& \! \left[ \left(2 {+} \dfrac{m}{2} \right)^{2} \! {-} \dfrac{m^{2}}{4} \right] + \alpha \, a_{1} {+} \dfrac{E}{2} \, a_{0} = \\[0.5em] \pause
% &a_{2}& \! \left[ 4 + 2 m {+} \dfrac{m^{2}}{4} {-} \dfrac{m^{2}}{4} \right] + \alpha \, a_{1} {+} \dfrac{E}{2} \, a_{0} = \\[0.35em] \pause
% &a_{2}& \! \left[ 2 (2 + m) \right] + \alpha \, a_{1} {+} \dfrac{E}{2} \, a_{0} = 0 \\[0.35em] \pause
% &a_{2}& \! =  \dfrac{{-}\alpha \, a_{1} {+} \dfrac{E}{2} \, a_{0}}{2(2{+}m)}
% \end{eqnarray*}
% \end{frame}
% \begin{frame}
% \frametitle{Ocupando un valor conocido}
% Ahora ocupamos el valor que ya conocemos de $a_{1}$ para sustituirlo en la expresión anterior:
% \begin{eqnarray*}
% &a_{2}& \! =  \dfrac{{-}\alpha \, a_{1} {+} \dfrac{E}{2} \, a_{0}}{2(2{+}m)} = \dfrac{{-}\alpha \, \left[ - \dfrac{\alpha a_{0}}{(1+m)} \right] {+} \dfrac{E}{2} \, a_{0}}{2(2{+}m)} = \\[0.5em] \pause
% &=& \dfrac{\left[ \dfrac{\alpha^{2} \, a_{0}}{(1+m)} \right] {-} \dfrac{E \, a_{0}}{2}}{2(2{+}m)} = \\[0.5em] \pause
% \end{eqnarray*}
% \end{frame}
% \begin{frame}
% \frametitle{Simplificando el coeficiente}
% \vspace*{-1cm}
% \begin{eqnarray*}
% &a_{2}& \! =  \dfrac{\left[ \dfrac{\alpha^{2}}{(1+m)} {-} \dfrac{E}{2} \right] a_{0}}{2(2{+}m)} = \\[0.5em] \pause
% &=& \dfrac{\left[ \dfrac{2 \, \alpha^{2} - E (1- m)}{2(1{+}m)} \right] a_{0}}{2(2+m)} = \\[0.5em] \pause
% &=& \dfrac{\left[ 2 \, \alpha^{2} - E (1- m) \right] a_{0}}{4(1+m)(2+m)}= 
% \end{eqnarray*}
% \end{frame}
% \begin{frame}
% \frametitle{El tercer coeficiente}
% Entonces el tercer coeficiente que requiere el enunciado es:
% \begin{align*}
% a_{2} =  \left[ \dfrac{\alpha^{2}}{2(1+m)(2+m)} - \dfrac{E}{4(1+m)(2+m)} \right] \, a_{0}
% \end{align*}
% \end{frame}
% \begin{frame}
% \frametitle{Coeficientes calculados}
% Ya hemos obtenido los coeficientes $a_{0}$, $a_{1}$ y $a_{2}$ con los que podemos responder al enunciado del ejercicio.
% \\
% \bigskip
% \pause
% Entonces tendremos:
% \end{frame}
\begin{frame}
\frametitle{Solución al ejercicio}
\vspace*{-1cm}
\begin{align*}
u(\xi) = a_{0} \, \xi^{r} + a_{1} \, \xi ^{1+k} + a_{2} \, \xi ^{2+k} + \ldots 
\end{align*}
\pause
Así que:   
\begin{align*}
u(\xi) &= a_{0} \, \xi^{m/2} - \dfrac{\alpha \, a_{0}}{1+m} \, \xi^{1+m/2} + \left[ \dfrac{\alpha^{2}}{2(1+m)(2+m)} + \right. \\[0.5em]
&- \left. \dfrac{E}{4(1+m)(2+m)} \right] \, a_{0} \, \xi^{2+m/2} + \ldots
\end{align*}
\end{frame}
\begin{frame}
\frametitle{Simplificando la solución}
Factorizando la expresión, tendremos la solución esperada:
\begin{align*}
%\addtolength{\fboxsep}{5pt}\boxed{
u(\xi) &= a_{0} \, \xi^{m/2}  \bigg[ 1 - \dfrac{\alpha}{1+m} \, \xi + \bigg[ \dfrac{\alpha^{2}}{2(1+m)(2+m)} + \\[0.5em]
&- \dfrac{E}{4(1+m)(2+m)} \bigg] \, \xi^{2} + \ldots \bigg]
\end{align*}
\end{frame}
\begin{frame}
\frametitle{Conclusión}
Luego de haber resuelto el ejercicio, nos damos cuenta de que la perturbación $F$ se presenta hasta que aparece el término $a_{3}$.
\end{frame}
\end{document}