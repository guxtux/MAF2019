\documentclass[12pt]{beamer}
\usepackage{../Estilos/BeamerMAF}
%Sección para el tema de beamer, con el theme, usercolortheme y sección de footers
\usetheme{CambridgeUS}
\usecolortheme{beaver}
%\useoutertheme{default}
\setbeamercovered{invisible}
% or whatever (possibly just delete it)
\setbeamertemplate{section in toc}[sections numbered]
\setbeamertemplate{subsection in toc}[subsections numbered]
\setbeamertemplate{subsection in toc}{\leavevmode\leftskip=3.2em\rlap{\hskip-2em\inserttocsectionnumber.\inserttocsubsectionnumber}\inserttocsubsection\par}
\setbeamercolor{section in toc}{fg=blue}
\setbeamercolor{subsection in toc}{fg=blue}
\setbeamercolor{frametitle}{fg=blue}
\setbeamertemplate{caption}[numbered]

\setbeamertemplate{footline}
\beamertemplatenavigationsymbolsempty
\setbeamertemplate{headline}{}


\makeatletter
\setbeamercolor{section in foot}{bg=gray!30, fg=black!90!orange}
\setbeamercolor{subsection in foot}{bg=blue!30!yellow, fg=red}
\setbeamercolor{date in foot}{bg=black, fg=white}
\setbeamertemplate{footline}
{
  \leavevmode%
  \hbox{%
  \begin{beamercolorbox}[wd=.333333\paperwidth,ht=2.25ex,dp=1ex,center]{section in foot}%
    \usebeamerfont{section in foot} \insertsection
  \end{beamercolorbox}%
  \begin{beamercolorbox}[wd=.333333\paperwidth,ht=2.25ex,dp=1ex,center]{subsection in foot}%
    \usebeamerfont{subsection in foot}  \insertsubsection
  \end{beamercolorbox}%
  \begin{beamercolorbox}[wd=.333333\paperwidth,ht=2.25ex,dp=1ex,right]{date in head/foot}%
    \usebeamerfont{date in head/foot} \insertshortdate{} \hspace*{2em}
    \insertframenumber{} / \inserttotalframenumber \hspace*{2ex} 
  \end{beamercolorbox}}%
  \vskip0pt%
}
\makeatother\newlength{\depthofsumsign}
\setlength{\depthofsumsign}{\depthof{$\sum$}}
\newcommand{\nsum}[1][1.4]{% only for \displaystyle
    \mathop{%
        \raisebox
            {-#1\depthofsumsign+1\depthofsumsign}
            {\scalebox
                {#1}
                {$\displaystyle\sum$}%
            }
    }
}
\def\scaleint#1{\vcenter{\hbox{\scaleto[3ex]{\displaystyle\int}{#1}}}}
\def\bs{\mkern-12mu}






\makeatletter
\setbeamertemplate{footline}
{
  \leavevmode%
  \hbox{%
  \begin{beamercolorbox}[wd=.333333\paperwidth,ht=2.25ex,dp=1ex,center]{section in foot}%
    \usebeamerfont{section in foot} \insertsection
  \end{beamercolorbox}%
  \begin{beamercolorbox}[wd=.333333\paperwidth,ht=2.25ex,dp=1ex,center]{subsection in foot}%
    \usebeamerfont{subsection in foot}  \insertsubsection
  \end{beamercolorbox}%
  \begin{beamercolorbox}[wd=.333333\paperwidth,ht=2.25ex,dp=1ex,right]{date in head/foot}%
    \usebeamerfont{date in head/foot} \insertshortdate{} \hspace*{2em}
    \insertframenumber{} / \inserttotalframenumber \hspace*{2ex} 
  \end{beamercolorbox}}%
  \vskip0pt%
}
\makeatother
\date{13 de octubre de 2021}

\title{\large{Solución EDO con términos trascendentes}}
\subtitle{Tema 2 - Primeras técnicas de solución}
\author{M. en C. Gustavo Contreras Mayén}

\begin{document}
\maketitle
\fontsize{14}{14}\selectfont
\spanishdecimal{.}

\section*{Contenido}
\frame{\tableofcontents[currentsection, hideallsubsections]}

\section{Planteamiento}
\frame{\tableofcontents[currentsection, hideothersubsections]}
\subsection{Problemas Tipo}

\begin{frame}
\frametitle{En dónde tendremos estos problemas}
Al resolver problemas de dispersión en mecánica cuántica, la ecuación de Schrödinger incluye términos trascendentes.
\\
\bigskip
\pause
Resolveremos el caso más simple de una ecuación diferencial, que incluye un término exponencial.
\end{frame}

\section{El problema a resolver}
\frame{\tableofcontents[currentsection, hideothersubsections]}
\subsection{Sustituyendo la solución en series}

\begin{frame}
\frametitle{Con el método de Frobenius}
Ocupando el método de Frobenius, encuentra los primeros seis términos de la solución de la siguiente EDO2H:
\pause
\begin{align}
\stilde{y} - y \, e^{x} = 0
\label{ec:ecuacion_01}
\end{align}
\end{frame}
\begin{frame}
\frametitle{Tipo de ecuación}
Tenemos que la EDO no presenta singularidades, por lo que proponemos una solución de la forma:
\pause
\begin{align*}
y = \sum_{n=0}^{\infty} a_{n} \, x^{n}
\end{align*}
\pause
Que debemos de sustituir en la EDO inicial.
\end{frame}
\begin{frame}
\frametitle{Las derivadas de $y$}
Se tiene entonces que:
\begin{eqnarray*}
\ptilde{y} &=& \sum_{n=0}^{\infty} n \, a_{n} \, x^{n-1} \\[0.5em] \pause
\stilde{y} &=& \sum_{n=0}^{\infty} n \, (n - 1) \, a_{n} \, x^{n-2}
\end{eqnarray*}
\end{frame}
\begin{frame}
\frametitle{EDO como serie de potencias}
La ec. (\ref{ec:ecuacion_01}) en términos de una serie de potencias es:
\pause
\begin{align}
\nsum_{n=0}^{\infty} n( n - 1) \, a_{n} \, x^{n-2} -  \nsum_{n=0}^{\infty} a_{n} \, x^{n} \, e^{x} = 0
\end{align}
\end{frame}
\begin{frame}
\frametitle{Ajuste en el índice de la primera suma}
Sin pérdida de generalidad, ajustamos el índice de la primera suma, ya que los dos primeros términos (con $n= 0, 1)$ no aportan elementos:
\pause
\begin{align}
\nsum_{n=2}^{\infty} n( n - 1) \, a_{n} \, x^{n-2} - \nsum_{n=0}^{\infty} a_{n} \, x^{n} \, e^{x} = 0
\end{align}
\end{frame}

\subsection{Regla de recurrencia}

\begin{frame}
\frametitle{El siguiente paso}
El método de Frobenius requiere que se establezca una relación de recurrencia para obtener los coeficientes de la serie de potencias, \pause pero tenemos la función trascendente $e^{x}$.
\end{frame}
\begin{frame}
\frametitle{Función trascendente}
Recordemos que una función trascendente no satisface una ecuación polinomial cuyos coeficientes sean a su vez polinomios.
\\
\bigskip
\pause
A diferencia con las funciones algebraicas, las cuales satisfacen dicha ecuación.
\end{frame}
\begin{frame}
\frametitle{Función trascendente}
Una función trascendente es una función que \emph{trasciende} al álgebra en el sentido que no puede ser expresada en términos de una secuencia infinita de operaciones algebraicas de suma, resta y extracción de raíces.
\end{frame}
\begin{frame}
\frametitle{Complicación para la relación de recurrencia}
Tenemos entonces un problema que debemos de resolver antes de lograr la relación de recurrencia.
\\
\bigskip
\pause
Como el enunciado nos pide los primeros seis términos de las soluciones, \pause ocupemos un desarrollo en una serie de Taylor de $e^{x}$.
\end{frame}
\begin{frame}
\frametitle{Desarrollando la serie de Taylor}
Considerando el caso $0 < x \ll 1$ para desarrollar la serie de Taylor de $e^{x}$:
\pause
\begin{align*}
\nsum_{n=2}^{\infty} n( n - 1) \, a_{n} \, x^{n-2} - \nsum_{n=0}^{\infty} a_{n} \, x^{n} \, \bigg[ \textcolor{blue}{1 + x + \dfrac{x^{2}}{2} + \ldots} \bigg] = 0
\end{align*}
\pause
Procedemos a separar los términos de la segunda suma.  
\end{frame}
\begin{frame}
\frametitle{Desarrollando las sumas}
Se tiene que:
\pause
\begin{align*}
&\textcolor{blue}{\underbrace{\nsum_{n=2}^{\infty} n( n - 1) \, a_{n} \, x^{n-2}}_{1}} \textcolor{red}{\underbrace{- \nsum_{n=0}^{\infty} a_{n} \, x^{n}}_{2}} \textcolor{brown}{\underbrace{- \nsum_{n=0}^{\infty} a_{n} \, x^{n+1}}_{3}}+ \\[0.5em] 
&\textcolor{OliveGreen}{\underbrace{- \dfrac{1}{2} \, \nsum_{n=0}^{\infty} a_{n} \, x^{n+2}}_{4}} - \ldots = 0
\end{align*}
\end{frame}
\begin{frame}
\frametitle{Ajuste en los índices}
Para operar las sumas se requiere que todas deben de iniciar siempre en el mismo índice, la primera suma comienza en $n=2$.
\\
\bigskip
\pause
Para ajustar los índices en esa primera suma, \enquote{bajamos} el índice a $n = 0$, modificando los respectivos términos de $n$.
\end{frame}
\begin{frame}
\frametitle{Sumas con el mismo índice}
Las sumas son:
\pause
\begin{align*}
&\textcolor{blue}{\underbrace{\nsum_{n=0}^{\infty} (n + 2)(n +1) \, a_{n+2} \, x^{n}}_{1}} \, \textcolor{red}{\underbrace{- \nsum_{n=0}^{\infty} a_{n} \, x^{n}}_{2}} + \\[0.5em] 
&\textcolor{brown}{\underbrace{- \nsum_{n=0}^{\infty} a_{n} \, x^{n+1}}_{3}} \, \textcolor{OliveGreen}{\underbrace{- \dfrac{1}{2} \, \nsum_{n=0}^{\infty} a_{n} \, x^{n+2}}_{4}} - \ldots = 0
\end{align*} 
\pause
Ahora ya podemos agrupar los coeficientes de $x^{n}$ en las primeras dos sumas.
\end{frame}
\begin{frame}
\frametitle{Expresión simplificada}
Entonces se tiene que:
\pause
\begin{align*}
&\textcolor{blue}{\underbrace{\nsum_{n=0}^{\infty} \bigg[ (n + 2)(n +1) \, a_{n+2}  - a_{n} \bigg] \, x^{n}}_{1}} + \\[0.5em] 
&\textcolor{brown}{\underbrace{- \nsum_{n=0}^{\infty} a_{n} \, x^{n+1}}_{2}} \, \textcolor{OliveGreen}{\underbrace{- \dfrac{1}{2} \, \nsum_{n=0}^{\infty} a_{n} \, x^{n+2}}_{3}} - \ldots = 0
\end{align*} 
\pause
Desarrollamos los primeros términos de las sumas anteriores.
\end{frame}
\begin{frame}
\frametitle{Los primeros términos}
Encontramos que:
\pause
\begin{align*}
&\textcolor{blue}{[2 a_{2} {-} a_{0}] x^{0} + [6 a_{3} {-} a_{1}] x^{1} + [12 a_{4} {-} a_{2}] x^{2} +} \\[0.5em]
&\textcolor{blue}{+ [20 a_{5} - a_{3}] x^{3} + [30 a_{6} - a_{4}] x^{4} + \ldots} + \\[0.5em]
&\textcolor{brown}{- a_{0} x^{2} -a_{1} x^{2} -a_{2} x^{3} -a_{3} x^{4} - \ldots} + \\[0.5em]
&\textcolor{OliveGreen}{-\dfrac{a_{0} x^{2}}{2} - \dfrac{a_{1} x^{3}}{2} - \dfrac{a_{2} x^{4}}{2} \ldots} - = 0 
\end{align*}
\end{frame}
\begin{frame}
\frametitle{Agrupando los términos}
Agrupamos los coeficientes comunes:
\pause
\begin{align*}
&\big[2 a_{2} {-} a_{0}\big] x^{0} + \big[6 a_{3} {-} a_{1} {-} a_{0}\big] x^{1} + \big[12 a_{4} {-} a_{2} {-} a_{1} {-} \dfrac{a_{0}}{2} \big] x^{2} + \\[0.5em]
&+ \big[ 20 a_{5} {-}a_{3} {-} a_{2} {-} \dfrac{a_{1}}{2} \big] x^{3} + \big[ 30 a_{6} - a_{4} - a_{3} - \dfrac{a_{2}}{2} \big] x^{4} + \\[0.5em]
&+ \ldots + = 0
\end{align*}
\end{frame}
\begin{frame}
\frametitle{Propiedad de las sumas}
Se revisó en las notas de trabajo que para una suma infinita se tiene que los coeficientes de la variable $x$ deben de anularse, por ello $a_{0} \neq 0$.
\\
\bigskip
\pause
Siguiendo esta definición, se tiene que:
\end{frame}
\begin{frame}
\frametitle{Anulando los coeficientes}
\begin{eqnarray*}
\begin{aligned}
2 a_{2} {-} a_{0} &= 0 \pause \hspace{0.3cm} \Rightarrow a_{2} = \dfrac{a_{0}}{2} \\[0.5em]
6 a_{3} {-} a_{1} {-} a_{0} &= 0 \pause \hspace{0.3cm} \Rightarrow a_{3} = \dfrac{a_{1} + a_{0}}{6} \\[0.5em]
12 a_{4} {-} a_{2} {-} a_{1} {-} \dfrac{a_{0}}{2} &= 0 \pause \hspace{0.3cm} \Rightarrow a_{4} = \dfrac{a_{1} + a_{0}}{12} \\[0.5em]
20 a_{5} {-}a_{3} {-} a_{2} {-} \dfrac{a_{1}}{2} &= 0 \pause \hspace{0.3cm} \Rightarrow a_{5} = \dfrac{a_{1} + a_{0}}{30} \\[0.5em]
30 a_{6} - a_{4} - a_{3} - \dfrac{a_{2}}{2} &= 0 \pause \hspace{0.3cm} \Rightarrow a_{6} = \dfrac{a_{1} + 2 \, a_{0}}{120}
\end{aligned}
\end{eqnarray*}
\end{frame}
\begin{frame}
\frametitle{Punto importante}
Vemos que conseguir una relación de recurrencia no es posible debido al término de la función $e^{x}$, pero con el cálculo de los primeros seis coeficientes, tendremos la solución al problema.
\\
\bigskip
\pause
Notemos que se requieren los coeficientes: $a_{0}$ y $a_{1}$.
\end{frame}
\begin{frame}
\frametitle{Aproximación a la solución}
Ocupando los resultados obtenidos llegamos a:
\begin{align*}
y(x) &= a_{0} + a_{1} x + \left( \dfrac{a_{0}}{2} \right) x^{2} + \left( \dfrac{a_{1} {+} a_{0}}{6} \right) x^{3} + \\[0.5em]
&+ \left( \dfrac{a_{1} {+} a_{0}}{12} \right) x^{4} + \left( \dfrac{a_{1} {+} a_{0}}{30} \right) x^{5} + \left( \dfrac{a_{1} {+} 2 a_{0}}{120} \right) x^{6} + \ldots +
\end{align*}
\end{frame}  
\begin{frame}
\frametitle{Dos soluciones independientes}
Para obtener dos soluciones linealmente independientes $y_{1}(x)$ y $y_{2}(x)$, en este tipo de problemas, \pause se recurre al siguiente paso:
\begin{eqnarray*}
y_{1}(x) \quad \mbox{se obtiene con } a_{0} = 1, a_{1} = 0 \\[0.5em] \pause
y_{2}(x) \quad \mbox{se obtiene con } a_{0} = 0, a_{1} = 1
\end{eqnarray*}
\pause
Si el enunciado indica los coeficientes $a_{0}$ y $a_{1}$, entonces se deberán de ocupar para la solución.
\end{frame}
\begin{frame}
\frametitle{Las soluciones $y_{1}(x)$ $y_{2}(x)$}
Las soluciones son:
\pause
\begin{eqnarray*}
\begin{aligned}
&y_{1}(x) = a_{0} \bigg[ 1 + \dfrac{x^{2}}{2} + \dfrac{x^{3}}{6} + \dfrac{x^{4}}{12} + \dfrac{x^{5}}{30} + \dfrac{x^{6}}{60} + \ldots + \bigg] \\[0.5em] \pause
&y_{2}(x) = a_{1} \, x \, \bigg[ 1 + \dfrac{x^{2}}{6} + \dfrac{x^{3}}{12} + \dfrac{x^{4}}{302} + \dfrac{x^{5}}{120} + \ldots + \bigg]
\end{aligned}
\end{eqnarray*}
\pause
Las soluciones $y_{1}(x)$ e $y_{2}(x)$ son linealmente independientes. $\qed$
\end{frame}
\begin{frame}
\frametitle{Más términos}
En caso de que se solicite incorporar más términos debidos a la función exponencial, el desarrollo se continua como se ha expuesto.
\\
\bigskip
En las soluciones se tendrán términos de orden mayor.
\end{frame}
\end{document}