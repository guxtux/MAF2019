\documentclass[hidelinks,12pt]{article}
\usepackage[left=0.25cm,top=1cm,right=0.25cm,bottom=1cm]{geometry}
%\usepackage[landscape]{geometry}
\textwidth = 20cm
\hoffset = -1cm
\usepackage[utf8]{inputenc}
\usepackage[spanish,es-tabla]{babel}
\usepackage[autostyle,spanish=mexican]{csquotes}
\usepackage[tbtags]{amsmath}
\usepackage{nccmath}
\usepackage{amsthm}
\usepackage{amssymb}
\usepackage{mathrsfs}
\usepackage{graphicx}
\usepackage{subfig}
\usepackage{standalone}
\usepackage[outdir=./Imagenes/]{epstopdf}
\usepackage{siunitx}
\usepackage{physics}
\usepackage{color}
\usepackage{float}
\usepackage{hyperref}
\usepackage{multicol}
%\usepackage{milista}
\usepackage{anyfontsize}
\usepackage{anysize}
%\usepackage{enumerate}
\usepackage[shortlabels]{enumitem}
\usepackage{capt-of}
\usepackage{bm}
\usepackage{relsize}
\usepackage{placeins}
\usepackage{empheq}
\usepackage{cancel}
\usepackage{wrapfig}
\usepackage[flushleft]{threeparttable}
\usepackage{makecell}
\usepackage{fancyhdr}
\usepackage{tikz}
\usepackage{bigints}
\usepackage{scalerel}
\usepackage{pgfplots}
\usepackage{pdflscape}
\pgfplotsset{compat=1.16}
\spanishdecimal{.}
\renewcommand{\baselinestretch}{1.5} 
\renewcommand\labelenumii{\theenumi.{\arabic{enumii}})}
\newcommand{\ptilde}[1]{\ensuremath{{#1}^{\prime}}}
\newcommand{\stilde}[1]{\ensuremath{{#1}^{\prime \prime}}}
\newcommand{\ttilde}[1]{\ensuremath{{#1}^{\prime \prime \prime}}}
\newcommand{\ntilde}[2]{\ensuremath{{#1}^{(#2)}}}

\newtheorem{defi}{{\it Definición}}[section]
\newtheorem{teo}{{\it Teorema}}[section]
\newtheorem{ejemplo}{{\it Ejemplo}}[section]
\newtheorem{propiedad}{{\it Propiedad}}[section]
\newtheorem{lema}{{\it Lema}}[section]
\newtheorem{cor}{Corolario}
\newtheorem{ejer}{Ejercicio}[section]

\newlist{milista}{enumerate}{2}
\setlist[milista,1]{label=\arabic*)}
\setlist[milista,2]{label=\arabic{milistai}.\arabic*)}
\newlength{\depthofsumsign}
\setlength{\depthofsumsign}{\depthof{$\sum$}}
\newcommand{\nsum}[1][1.4]{% only for \displaystyle
    \mathop{%
        \raisebox
            {-#1\depthofsumsign+1\depthofsumsign}
            {\scalebox
                {#1}
                {$\displaystyle\sum$}%
            }
    }
}
\def\scaleint#1{\vcenter{\hbox{\scaleto[3ex]{\displaystyle\int}{#1}}}}
\def\bs{\mkern-12mu}


\title{Enunciados para el Examen Intermedio - Tema 2 \\[0.3em]  \large{Matemáticas Avanzadas de la Física}\vspace{-3ex}}
\author{M. en C. Gustavo Contreras Mayén}
\date{ }
\begin{document}
\vspace{-4cm}
\maketitle
\fontsize{14}{14}\selectfont

\textbf{Indicaciones: } Deberás de resolver cada ejercicio de la manera más completa, ordenada y clara posible, anotando cada paso así como las operaciones involucradas. El puntaje de cada ejercicio es de \textbf{1 punto}.
\par
Este examen requiere que presentes todo el desarrollo de acuerdo como se trabajó con separación de variables, con el método de Frobenius: identificar puntos singulares en una EDO, se requiere que presentes la ecuación de índices, así como la relación de recurrencia y las dos soluciones linealmente independientes.

\begin{enumerate}
\item Demuestra que la ecuación de Poisson:
\begin{align*}
\laplacian{\varphi} (\eta, \psi, z) = 0
\end{align*}
es separable en el sistema coordenado cilíndrico elíptico.
%Ref. Pinchover 5.6
\item Con el método de separación de variables (MSP): 
\begin{enumerate}[label=\roman*)]
\item Calcula una solución al siguiente problema periódico con la ecuación de calor:
\begin{table}[H]
\centering
\large
\begin{tabular}{l l}
$u_{t} - k \, u_{xx} = 0$ & $0 < x < 2 \, \pi,  t > 0$, \\
$u(0, t) = u(2 \, \pi, t) = 0, \quad u_{x} (0, t) = u_{x}(2 \, \pi, t)$ & $t \geq 0$, \\
$u(x, 0) = f(x)$ & $0 \leq x \leq 2 \, \pi$
\end{tabular}
\end{table}
donde $f$ es una función periódica \emph{suave} (bien portada, continua, derivable). Este sistema describe la evolución del calor en un alambre circular aislado de longitud $2 \, \pi$.
\item Encuentra el $\displaystyle{\lim_{t \to \infty}} \, u(x, t)$ para todo $0 < x < 2 \, \pi$, ¿qué interpretación física se tiene de este resultado?
\end{enumerate}
%Ref. Arfken (2006) 9.3.5
\item Una partícula atómica (mecánica cuántica) está confinada dentro de una caja rectangular de lados $a, b$ y $c$. La partícula se describe mediante una función de onda $\psi$ que satisface la ecuación de onda de Schrödinger:
\begin{align*}
- \dfrac{\hbar^{2}}{2 \, m} \, \laplacian{\psi} = E \, \psi
\end{align*}
Se requiere que la función de onda desaparezca en cada superficie de la caja (pero no que sea idénticamente cero). Esta condición impone restricciones a las constantes de separación y, por lo tanto, a la energía $E$. Demuestra que:
\begin{align*}
E = \dfrac{\pi^{2} \, \hbar^{2}}{2 \, m} \bigg( \dfrac{1}{a^{2}} + \dfrac{1}{b^{2}} + \dfrac{1}{c^{2}} \bigg)
\end{align*}
es el valor más pequeño de $E$ para el que se puede obtener esa solución.
%Ref. Arfken 9.5.7
\item Resuelve la ecuación diferencial de Laguerre mediante una solución en series:
\begin{align*}
x \, \sderivada{L}_{n} (x) + (1 - x) \, \pderivada{L}_{n} (x) + n \, L_{n} (x) = 0
\end{align*}
Elige el parámetro $n$ tal que se trunque la serie y la solución se exprese como un polinomio.
\item Identifica el(los) punto(s) singular(es) y en caso de que sea conducente, con el método de Frobenius resuelve las siguientes ecuaciones diferenciales.
\begin{enumerate}[label=\alph*)]
%Ref. Duffy (1986) A.2.1
\item $(2 x^{2} + x^{3}) \stilde{y} + (x + 3 x^{2}) \ptilde{y} - (1 - 4 x) y = 0$
%Ref. Duffy Pag. 477 8)
%\item $2 x^{2} \stilde{y} - 3 (x + x^{2}) \ptilde{y} + (2 + 3 x) y = 0$
%Ref. Duffy Pag. 480 4
%\item $x^{2} \stilde{y} - x (1 + x) \ptilde{y} + y = 0$
%REf. Duffy Pag. 480 8)
\item $x^{2} \stilde{y} + x (2 x - 1) \ptilde{y} + x(x - 1) y = 0$
\end{enumerate}
%Ref. Arfken 9.6.3
\item Usando el determinante Wronskiano, demuestra que el conjunto de funciones:
\begin{align*}
\left\{ 1, \dfrac{x^{n}}{n!} \hspace{0.2cm} (n = 1, 2, \ldots, N) \right\}
\end{align*}
es linealmente independiente.
%Ref. Arfken (2006) 9.6.15
\item Considerando que una solución de:
\begin{align*}
\sderivada{R} + \dfrac{1}{r} \, \pderivada{R} - \dfrac{m^{2}}{r^{2}} \, R = 0
\end{align*}
es $R = r^{m}$. Demuestra que la ecuación (\ref{eq:ecuacion_09_127}) predice una segunda solución: $R = r^{-m}$
\begin{align}
\setlength{\fboxsep}{2\fboxsep}\boxed{y_{2}(x) =  y_{1} \: (x) \int^{x} \dfrac{\exp \left[ \displaystyle - \int^{x_{2}} P(x_{1}) \: \dd{x_{1}} \right]}{[y_{1}(x_{2})]^{2}} \dd{x_{2}}}
\label{eq:ecuacion_09_127}
\end{align}
%Ref. Arfken (2006) 9.6.18
\item Una solución para la ecuación diferencial de Hermite:
\begin{align*}
\stilde{y} - 2 \, x \, y + 2 \, \alpha \, y = 0
\end{align*}
\begin{enumerate}[label=\alph*)]
\item Para $\alpha = 0$ es $y_{1} (x) = 1$.
\item Para $\alpha = 1$ es $y_{1} (x) = 1$.
\end{enumerate}
Usando la ecuación (\ref{eq:ecuacion_09_127}). Encuentra una segunda solución $y_{2}(x)$ para cada inciso.
%Ref. Herman (2015)
\item Sea $S$ una superficie cerrada y $V$ un volumen cerrado. Demuestra la primera identidad de Green:
\begin{align*}
\scaleint{6ex}_{\bs S} \phi \, \grad{\psi} \cdot \vb{n} \dd{S} = \scaleint{6ex}_{\bs V} \big( \phi \, \laplacian{\psi} + \grad{\phi} \cdot \grad{\psi} \big) \dd{V}
\end{align*}
\end{enumerate}

\end{document}