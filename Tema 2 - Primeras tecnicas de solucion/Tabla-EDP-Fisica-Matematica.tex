\documentclass[12pt,landscape]{article}
\usepackage[utf8]{inputenc}
\usepackage[letterpaper, margin=0.5cm, footskip=-0.5cm]{geometry}
%\usepackage{anysize}
%\marginsize{1cm}{1cm}{1cm}{1cm}
\usepackage[spanish,es-lcroman, es-tabla]{babel}
\usepackage[autostyle,spanish=mexican]{csquotes}
\usepackage{amsmath}
\usepackage{amssymb}
\usepackage{nccmath}
\numberwithin{equation}{section}
\usepackage{amsthm}
\usepackage{graphicx}
\usepackage[outdir=./]{epstopdf}
\DeclareGraphicsExtensions{.pdf,.png,.jpg,.eps}
\usepackage{color}
\usepackage{float}
\usepackage{fancyhdr}
\usepackage{multicol}
\usepackage{enumerate}
\usepackage[shortlabels]{enumitem}
\usepackage{anyfontsize}
\usepackage{anysize}
\usepackage{array}
\usepackage{multirow}
\usepackage{enumitem}
\usepackage{cancel}
\usepackage{nameref}
\usepackage{pdflscape}
\usepackage{makecell}
\usepackage{longtable}
\usepackage{pgfplots}
\pgfplotsset{compat=1.12}
\usepackage{tikz}
\usepackage{circuitikz}
\usepackage{tikz-3dplot}
\usepackage{caption}
\usepackage{bm}
\usepackage{mathtools}
\usepackage{esvect}
\usepackage{hyperref}
\usepackage{relsize}
\usepackage{siunitx}
\usepackage{physics}
%\usepackage[backend=biber]{biblatex}
\usepackage{standalone}
\usepackage{mathrsfs}
\usepackage{bigints}
\usepackage{bookmark}
%Quita el número de la página
\pagenumbering{gobble}
\spanishdecimal{.}
%\setlength{\voffset}{-0.75in}
\author{}
\date{ }
\title{Ecuaciones Diferenciales de la Física Matemática \\ {\large Tema 2 - Primeras técnicas de solución - Curso MAF}}
\begin{document}
\renewcommand\labelenumii{\theenumi.{\arabic{enumii}}}
\maketitle
\fontsize{14}{14}\selectfont
\addtolength{\voffset}{-2cm}
\vspace{-2cm}
\begin{center}
{\renewcommand{\arraystretch}{2}%
{\setlength\extrarowheight{1.5pt}
\begin{tabular}{ | c | p{6cm} | p{9cm} | c | c | } \hline
 & \makecell{Ecuación} & \makecell{Expresión} & \makecell{Punto regular \\ singular \\ $x = $} & \makecell{Punto irregular \\ singular \\ $x = $} \\  \hline
1 & Hipergeométrica & $x(x-1) y^{\prime \prime} + [1 + a + b - c]y^{\prime} + aby = 0$ & $0, 1, \infty$ & $--$\\ \hline
2 & Legendre & $(1-x^{2}) y^{\prime\prime} - 2 x y^{\prime} + \ell(\ell + 1)y = 0$ & $-1, 1, \infty$ & $--$ \\ \hline
3 & Chebychev & $(1-x^{2}) y^{\prime \prime} - x y^{\prime} + n^{2}y =0$ & $-1, 1, \infty$ & $--$ \\ \hline
4 & Hipergeométrica confluente & $ x y^{\prime \prime} + (c - x) y^{\prime} - a y = 0$ & $0$ & $\infty$ \\ \hline
5 & Bessel & $x^{2} y^{\prime \prime} + x y^{\prime} + (x^{2}- n^{2}) y = 0$ & $0$ & $\infty$ \\ \hline
6 & Laguerre & $x y^{\prime \prime} + (1 - x) y^{\prime} + a y = 0$ & $0$ & $\infty$ \\ \hline
7 & Hermite & $y^{\prime \prime} - 2 x y^{\prime} + 2 \alpha y = 0$ & $--$ & $\infty$ \\ \hline
8 & Oscilador armónico simple & $y^{\prime \prime} + \omega^{2} y = 0$ & $--$ & $\infty$ \\ \hline
\end{tabular}}}
\end{center}

En la tabla anterior se enlista una serie de EDP2H conocidas de la Física Matemática, pero no implica que sean las únicas ecuaciones, a modo de tarea moral, podrán incluir en la tabla tanto la expresión, así como el(los) punto(s) regular(es) que presente o en su caso, los irregulares: Ecuación de Helmholtz, la ec. biarmónica, la ec. de Schrödinger, la ec de Klein-Gordon, la ec. del telégrafo, la ec. de la óptica geométrica, la ec. de Fokker-Planck, entre otras más.
\end{document}