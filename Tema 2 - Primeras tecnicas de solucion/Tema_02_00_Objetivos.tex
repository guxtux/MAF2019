\documentclass[12pt]{beamer}
\usepackage{../Estilos/BeamerMAF}
%Sección para el tema de beamer, con el theme, usercolortheme y sección de footers
\usetheme{CambridgeUS}
\usecolortheme{beaver}
%\useoutertheme{default}
\setbeamercovered{invisible}
% or whatever (possibly just delete it)
\setbeamertemplate{section in toc}[sections numbered]
\setbeamertemplate{subsection in toc}[subsections numbered]
\setbeamertemplate{subsection in toc}{\leavevmode\leftskip=3.2em\rlap{\hskip-2em\inserttocsectionnumber.\inserttocsubsectionnumber}\inserttocsubsection\par}
\setbeamercolor{section in toc}{fg=blue}
\setbeamercolor{subsection in toc}{fg=blue}
\setbeamercolor{frametitle}{fg=blue}
\setbeamertemplate{caption}[numbered]

\setbeamertemplate{footline}
\beamertemplatenavigationsymbolsempty
\setbeamertemplate{headline}{}


\makeatletter
\setbeamercolor{section in foot}{bg=gray!30, fg=black!90!orange}
\setbeamercolor{subsection in foot}{bg=blue!30!yellow, fg=red}
\setbeamercolor{date in foot}{bg=black, fg=white}
\setbeamertemplate{footline}
{
  \leavevmode%
  \hbox{%
  \begin{beamercolorbox}[wd=.333333\paperwidth,ht=2.25ex,dp=1ex,center]{section in foot}%
    \usebeamerfont{section in foot} \insertsection
  \end{beamercolorbox}%
  \begin{beamercolorbox}[wd=.333333\paperwidth,ht=2.25ex,dp=1ex,center]{subsection in foot}%
    \usebeamerfont{subsection in foot}  \insertsubsection
  \end{beamercolorbox}%
  \begin{beamercolorbox}[wd=.333333\paperwidth,ht=2.25ex,dp=1ex,right]{date in head/foot}%
    \usebeamerfont{date in head/foot} \insertshortdate{} \hspace*{2em}
    \insertframenumber{} / \inserttotalframenumber \hspace*{2ex} 
  \end{beamercolorbox}}%
  \vskip0pt%
}
\makeatother\newlength{\depthofsumsign}
\setlength{\depthofsumsign}{\depthof{$\sum$}}
\newcommand{\nsum}[1][1.4]{% only for \displaystyle
    \mathop{%
        \raisebox
            {-#1\depthofsumsign+1\depthofsumsign}
            {\scalebox
                {#1}
                {$\displaystyle\sum$}%
            }
    }
}
\def\scaleint#1{\vcenter{\hbox{\scaleto[3ex]{\displaystyle\int}{#1}}}}
\def\bs{\mkern-12mu}





\date{}
\title{\large{Objetivos del Tema 2 - Primeras técnicas de solución}}
\subtitle{Curso MAF}
\author{M. en C. Gustavo Contreras Mayén}


\begin{document}
\maketitle
\fontsize{14}{14}\selectfont
\spanishdecimal{.}
\section*{Contenido}
\frame[allowframebreaks]{\tableofcontents[currentsection, hideallsubsections]}
\section{Introducción}
\subsection{Las EDP}
\begin{frame}
\frametitle{Las Ecuaciones Diferenciales Parciales (EDP)}
De la ecuación de movimiento de una partícula para conocer la fuerza por ejemplo, en general nos conduce a una ecuación diferencial.
\\
\bigskip
Por lo tanto, en casi toda la física  básica y en una mayor parte de la física teórica avanzada se expresa en términos de ecuaciones diferenciales.
\end{frame}
\begin{frame}
\frametitle{Las Ecuaciones Diferenciales Parciales (EDP)}
A veces, esas son ecuaciones diferenciales ordinarias en una variable (EDO).
\\
\bigskip
Más a menudo las ecuaciones son expresiones de ecuaciones diferenciales parciales (EDP) en dos o más variables.
\end{frame}
\begin{frame}
\frametitle{Apoyo con el Tema 1}
Al haber revisado la construcción de sistemas curvilíneos ortogonales en el Tema 1, así como la caracterización de esos sistemas con el uso de operadores diferenciales, ahora contamos con una herramienta importante.
\end{frame}
\begin{frame}
\frametitle{Apoyo con el Tema 1}
El modelar un fenómeno en una geometría particular mediante una EDP\footnote{Para abreviar espacio, anotaremos EDP para indicar una \emph{Ecuación Diferencial Parcial.}}, se nos devolverá un EDP que tendremos que resolver mediante algún método, como primera técnica de solución, ocuparemos \emph{la técnica de separación de variables}.
\end{frame}
\begin{frame}
\frametitle{Punto importante}
Cabe señalar que la construcción de una ED\footnote{Se abreviará ED para indicar una \emph{Ecuación Diferencial}, cuando se conozca el grado y tipo de ecuación diferencial, se anotará por ejemplo EDO2H, para señalar una ecuación diferencial ordinaria de segundo grado homogénea.} a partir de un fenómeno de la física, no se realizará en el curso, es decir, ocuparemos la EDP ya definida, considerando que la formalización de dicha expresión la revisaron en los cursos de los semestres previos a MAF.
\end{frame}
\begin{frame}
\frametitle{Punto importante}
Por lo que es recomendable que le den un repaso a la construcción de dichas expresiones, consultando las referencias o bibliografía en particular.
\end{frame}
\begin{frame}
\frametitle{Soluciones obtenidas con la técnica}
Como veremos en la técnica de separación de variables, reducieremos el orden de la EDP, además de que pasaremos de tener una EDP a un sistema de EDO.
\\
\bigskip
Veremos un hecho curioso de que la gran mayoría de las EDP de la física matemática, son EDP de segundo orden.
\end{frame}
\begin{frame}
\frametitle{Soluciones obtenidas con la técnica}
El sistema de EDO obtenido se resuelve de una manera mucho más sencilla, tal y como lo vieron en su curso de \emph{Ecuaciones diferenciales I} en la carrera.
\end{frame}
\begin{frame}
\frametitle{Etapas de trabajo}
En todo momento esperamos que ocupen la siguiente estrategia de trabajo:
\setbeamercolor{item projected}{bg=blue!70!black,fg=yellow}
\setbeamertemplate{enumerate items}[circle]
\begin{enumerate}[<+->]
\item Formulación.
\item Solución.
\item Intepretación.
\end{enumerate}
\end{frame}
\section{Objetivos}
\frame{\tableofcontents[currentsection, hideothersubsections]}
\subsection{Tema 2}
\begin{frame}
\frametitle{Objetivos del Tema 2}
Al concluir el Tema 2, el alumno:
\setbeamercolor{item projected}{bg=blue!70!black,fg=yellow}
\setbeamertemplate{enumerate items}[circle]
\begin{enumerate}
\item Clasificará una ecuación diferencial parcial a partir de las propiedades que presente la expresión.
\seti
\end{enumerate}
\end{frame}
\begin{frame}
\frametitle{Objetivos del Tema 2}
\setbeamercolor{item projected}{bg=blue!70!black,fg=yellow}
\setbeamertemplate{enumerate items}[circle]
\begin{enumerate}
\conti
\item Obtendrá las soluciones de una ecuación diferencial parcial lineal y homogénea mediante la técnica de separación de variables.
\seti
\end{enumerate}
\end{frame}
\begin{frame}
\frametitle{Objetivos del Tema 2}
\setbeamercolor{item projected}{bg=blue!70!black,fg=yellow}
\setbeamertemplate{enumerate items}[circle]
\begin{enumerate}
\conti
\item Obtendrá las soluciones en series de potencias de una ecuación diferencial parcial a partir del método de Frobenius.
\item Identificará las singularidades presentes en la EDP, para realizar la remoción de las mismas y procederá a resolver la ecuación con el método de Frobenius.
\seti
\end{enumerate}
\end{frame}
\begin{frame}
\frametitle{Objetivos del Tema 2}
\setbeamercolor{item projected}{bg=blue!70!black,fg=yellow}
\setbeamertemplate{enumerate items}[circle]
\begin{enumerate}
\conti
\item Identificará el caso en donde sea posible y resolverá para obtener una segunda solución linealmente independiente de una EDP de segundo orden.
\seti
\end{enumerate}
\end{frame}
\begin{frame}
\frametitle{Objetivos del Tema 2}
\setbeamercolor{item projected}{bg=blue!70!black,fg=yellow}
\setbeamertemplate{enumerate items}[circle]
\begin{enumerate}
\conti
\item Expondrá la interpretación de (los) resultado(s) obtenido(s), para vincularlos con el fenómeno de estudio.
\seti
\end{enumerate}
\end{frame}
\begin{frame}
\frametitle{Apoyo adicional}
En este tema como se habrán dado cuenta, será necesario apoyarse \enquote{fuertemente} en las técnicas de solución de las EDO.
\\
\bigskip
\pause
Así como del manejo de series de potencias que se revisa en el curso de \emph{Álgebra lineal.}
\end{frame}
\begin{frame}
\frametitle{Otra nota importante}
Se hace el comentario de que en las presentaciones y sesiones de trabajo síncrono, se presentará la solución directa de las EDO, pero tanto para los ejercicios a cuenta como del examen-tarea, tendrán que incluir todo el procedimiento de solución.
\end{frame}
\begin{frame}
\frametitle{Apoyo con software}
También podrán apoyarse con software para corroborar sus resultados y en su caso, para explorar algunas características de las soluciones de las EDO, para la interpretación de lo que obtienen.
\end{frame}
\section{Cronograma de trabajo}
\frame{\tableofcontents[currentsection, hideothersubsections]}
\subsection{Trabajo semanal}
\begin{frame}
\frametitle{Distribución de actividades}
Para alcanzar los objetivos del Tema 2, se han preparado una serie de materiales de lectura y otros adicionales, que deberán de leer, revisar y apoyarse para resolver los ejercicios a cuenta.
\end{frame}
\begin{frame}
\frametitle{Materiales de revisión}
Los materiales de trabajo que tendrán en la plataforma son los siguientes:
\setbeamercolor{item projected}{bg=blue!70!black,fg=yellow}
\setbeamertemplate{enumerate items}[circle]
\begin{enumerate}[<+->]
\item Introducción Ecs. Diferenciales Parciales.
\item Separación de variables.
\item Método de Frobenius y remoción de singularidades.
\item Segunda solución linealmente independiente.
\end{enumerate}
\end{frame}
\begin{frame}
\frametitle{Trabajo por semana}
Se espera que revisen el material de trabajo de acuerdo al siguiente esquema:
\setbeamercolor{item projected}{bg=blue!70!black,fg=yellow}
\setbeamertemplate{enumerate items}[circle]
\begin{enumerate}[<+->]
\item %Semana 3 y 4 (8 y 9 Oct.) Presentaciones 1 y 2.
\item %Semana 5 - Presentación 3.
\item %Semana 6 - Presentación 4.
\end{enumerate}
\end{frame}
\begin{frame}
\frametitle{Material adicional}
También tendrán disponible una serie de materiales adicionales en donde se trabajarán EDP que nos encontraremos más adelante en el curso, pero serán de utilidad para el Tema 2.
\end{frame}
\section{Evaluación}
\frame{\tableofcontents[currentsection, hideothersubsections]}
\subsection{Ejercicios a cuenta}
\begin{frame}
\frametitle{Ejercicios semanales}
Se tendrán ejercicios a cuenta en las presentaciones semanales, por lo que se espera que los resuelvan y envíen como se ha acordado previamente.
\end{frame}
\begin{frame}
\frametitle{Actividades en Moodle}
Con el fin de estimular y motivar el trabajo, se presentarán una serie de ejercicios en la plataforma Moodle, que a diferencia de los ejercicios a cuenta, la respuesta a los ejercicios será directa.
\\
\bigskip
Es decir, al leer el enunciado deberán de realizar la solución por su cuenta, y anotar la respuesta en el espacio correspondiente.
\end{frame}
\begin{frame}
\frametitle{Actividades en Moodle}
Por ejemplo, se presentará una lista de EDP y se les pide identificar algunas propiedades de ellas: si son lineales o no, su clasificación a partir del discriminante, etc.
\\
\bigskip
Por lo que su respuesta se deberá de anotar en el espacio de Moodle.
\end{frame}
\begin{frame}
\frametitle{Puntuación de los ejercicios}
La solución a los ejercicios es \textbf{opcional}, no penaliza la calificación en caso de que no se resuelvan, pero \textbf{SI} aportará puntaje para los ejercicios de la semana.
\end{frame}
\begin{frame}
\frametitle{Puntaje de los ejercicios}
Se le dará un puntaje de los ejercicios opcionales en Moodle, siendo el máximo puntaje de $0.5$ que se aplicará a los ejercicios de la semana.
\\
\bigskip
\pause
Se agregará la parte propocional de los ejercicios con respuesta correcta.
\end{frame}
\begin{frame}
\frametitle{Puntaje de los ejercicios}
Es decir, si en la semana $4$ hay $10$ ejercicios a cuenta, y se resuelven los ejercicios opcionales, y todos tienen la respuesta correcta se sumará $0.5$ al puntaje obtenido luego de la evaluación de los $10$ ejercicios semanales.
\end{frame}
\begin{frame}
\frametitle{Puntaje de los ejercicios}
Si tuve 6 ejercicios opcionales correctos, entonces el puntaje que se sumará a la semana, es decir: 
\begin{align*}
\dfrac{6}{10} \times 0.5 = 0.3 \mbox{ puntos}
\end{align*}
\end{frame}
\begin{frame}
\frametitle{Puntaje de los ejercicios}
Como ejemplo: si en la semana $4$, obtuve un puntaje de $8.5$ luego de la revisión de los $10$ ejercicios, entonces:
\setbeamercolor{item projected}{bg=blue!70!black,fg=yellow}
\setbeamertemplate{enumerate items}[circle]
\begin{enumerate}[<+->]
\item Si tengo correctos todas las respuestas de los ejercicios adicionales, mi puntaje de ejercicios  será de $9.0$
\item Si tengo una parte de los ejercicios con respuesta correcta $(6/10)$, mi puntaje de ejercicios será de $8.8$
\end{enumerate}
\end{frame}
\begin{frame}
\frametitle{Plazo para resolver}
Los ejercicios opcionales tendrán un plazo y vencimiento: se deberán de responder durante la semana de trabajo, es decir, para la semana $4$, se podrán resolver y responder en la plataforma.
\end{frame}
\begin{frame}
\frametitle{Plazo para resolver}
Quedará programada la apertura para el día lunes y el cierre para el día viernes a las 6 pm.
\\
\bigskip
Un vez alcanzado el horario, la plataforma ya no permitirá enviar respuestas.
\end{frame}
\begin{frame}
\frametitle{Puntaje de los Ejercicios}
Recuerden que al final del semestre, se sumará el total de puntos obtenidos en los ejercicios a cuenta, por lo que es atractivo resolver algunos ejercicios más para obtener puntos adicionales.
\end{frame}
\section{Sesiones síncronas}
\frame{\tableofcontents[currentsection, hideothersubsections]}
\subsection{Día de trabajo en línea}
\begin{frame}
\frametitle{Día de trabajo adicional}
De acuerdo al calendario que se ha presentado, se considera oportuno que los días miércoles de las semanas 4, 5 y 6, es decir, los días 14, 21 y 28 de octubre, se realice una sesión de trabajo en Zoom.
\end{frame}
\begin{frame}
\frametitle{Día de trabajo adicional}
De tal manera que se revisen otros ejercicios con ciertas características, que buscarían complementar el trabajo que están realizando.
\\
\bigskip
Se espera también que revisen los materiales de trabajo de manera previa.
\end{frame}
\end{document}