\documentclass[hidelinks,12pt]{article}
\usepackage[left=0.25cm,top=1cm,right=0.25cm,bottom=1cm]{geometry}
%\usepackage[landscape]{geometry}
\textwidth = 20cm
\hoffset = -1cm
\usepackage[utf8]{inputenc}
\usepackage[spanish,es-tabla]{babel}
\usepackage[autostyle,spanish=mexican]{csquotes}
\usepackage[tbtags]{amsmath}
\usepackage{nccmath}
\usepackage{amsthm}
\usepackage{amssymb}
\usepackage{mathrsfs}
\usepackage{graphicx}
\usepackage{subfig}
\usepackage{standalone}
\usepackage[outdir=./Imagenes/]{epstopdf}
\usepackage{siunitx}
\usepackage{physics}
\usepackage{color}
\usepackage{float}
\usepackage{hyperref}
\usepackage{multicol}
%\usepackage{milista}
\usepackage{anyfontsize}
\usepackage{anysize}
%\usepackage{enumerate}
\usepackage[shortlabels]{enumitem}
\usepackage{capt-of}
\usepackage{bm}
\usepackage{relsize}
\usepackage{placeins}
\usepackage{empheq}
\usepackage{cancel}
\usepackage{wrapfig}
\usepackage[flushleft]{threeparttable}
\usepackage{makecell}
\usepackage{fancyhdr}
\usepackage{tikz}
\usepackage{bigints}
\usepackage{scalerel}
\usepackage{pgfplots}
\usepackage{pdflscape}
\pgfplotsset{compat=1.16}
\spanishdecimal{.}
\renewcommand{\baselinestretch}{1.5} 
\renewcommand\labelenumii{\theenumi.{\arabic{enumii}})}
\newcommand{\ptilde}[1]{\ensuremath{{#1}^{\prime}}}
\newcommand{\stilde}[1]{\ensuremath{{#1}^{\prime \prime}}}
\newcommand{\ttilde}[1]{\ensuremath{{#1}^{\prime \prime \prime}}}
\newcommand{\ntilde}[2]{\ensuremath{{#1}^{(#2)}}}

\newtheorem{defi}{{\it Definición}}[section]
\newtheorem{teo}{{\it Teorema}}[section]
\newtheorem{ejemplo}{{\it Ejemplo}}[section]
\newtheorem{propiedad}{{\it Propiedad}}[section]
\newtheorem{lema}{{\it Lema}}[section]
\newtheorem{cor}{Corolario}
\newtheorem{ejer}{Ejercicio}[section]

\newlist{milista}{enumerate}{2}
\setlist[milista,1]{label=\arabic*)}
\setlist[milista,2]{label=\arabic{milistai}.\arabic*)}
\newlength{\depthofsumsign}
\setlength{\depthofsumsign}{\depthof{$\sum$}}
\newcommand{\nsum}[1][1.4]{% only for \displaystyle
    \mathop{%
        \raisebox
            {-#1\depthofsumsign+1\depthofsumsign}
            {\scalebox
                {#1}
                {$\displaystyle\sum$}%
            }
    }
}
\def\scaleint#1{\vcenter{\hbox{\scaleto[3ex]{\displaystyle\int}{#1}}}}
\def\bs{\mkern-12mu}


\title{Oscilador armónico cuántico\\ \large{Matemáticas Avanzadas de la Física}\vspace{-3ex}}
\author{M. en C. Abraham Lima Buendía}
\date{ }
\begin{document}
\vspace{-4cm}
\maketitle
\fontsize{14}{14}\selectfont
\section{Oscilador armónico cuántico.}

El oscilador armónico cuántico está descrito mediante el Hamiltoniano:
\begin{align*}
- \dfrac{k^{2}}{2 m} \, \dv[2]{x} \psi + \dfrac{m \, \omega^{2}}{2} \, x^{2} \, \psi
 = E \, \psi
\end{align*}
Encontraremos el espectro de energía y las soluciones de esta ecuación diferencial.
\par
Para facilitar el trabajo, construimos una variable adimensional, en la cual resolveremos la ecuación diferencial.
\par
Dividimos el Hamiltoniano entre $\- 2 \, m/ \hbar^{2}$, así:
\begin{align*}
\dv[2]{x} \psi - \left( \dfrac{m \, \omega}{\hbar} \right)^{2} \, x^{2} \, \psi =  \dfrac{2 \, m \, E}{\hbar^{2}} = \left( - \dfrac{2 \, E}{\hbar \, \omega} \right) \bigg( \dfrac{m \, \omega}{\hbar} \bigg) \, \psi
\end{align*}
Sea la variable $\xi$ tal que:
\begin{align*}
\xi = \sqrt{\dfrac{m \, \omega}{\hbar}} \, x
\end{align*}
Entonces hacemos
\begin{align*}
\dv{x} = \dv{\xi} \, \dv{\xi}{x} = \sqrt{\dfrac{m \, \omega}{\hbar}} \, \dv{\xi} \hspace{0.2cm} \Longrightarrow \hspace{0.2cm} \dv[2]{x} = \left( \dfrac{m \, \omega}{\hbar} \right) \, \dv[2]{\xi}
\end{align*}
Así llegamos a:
\begin{align*}
\left( \dfrac{m \, \omega}{\hbar} \right) \, \dv[2]{\xi} \psi - \left( \dfrac{m \, \omega}{\hbar} \right) \left( \sqrt{\dfrac{m \, \omega}{\hbar}} \, x \right)^{2} \, \psi = \left( - \dfrac{2 \, E}{\hbar \, \omega} \right) \bigg( \dfrac{m \, \omega}{\hbar} \bigg) \, \psi
\end{align*}
Entonces
\begin{align*}
\dv[2]{\xi} \psi -  \left( \sqrt{\dfrac{m \, \omega}{\hbar}} \, x \right)^{2} \, \psi = \left( - \dfrac{2 \, E}{\hbar \, \omega} \right) \, \psi = \dv[2]{\xi} \psi - \xi^{2} \, \psi = - k \, \psi
\end{align*}
con $k = 2 \, E / \hbar \, \omega$.
\par
Por lo anterior, la ecuación diferencial a resolver es:
\begin{align*}
\dv[2]{\xi} \psi + (k - \xi^{2}) \, \psi = 0
\end{align*}
Hemos revisado la forma de remover las singularidades del infinito (la ecuación es regular en el origen), sin embargo, veremos una forma alternativa de estudiar las divergencias en el infinito.
\par
Tomemos el límite cuando $\xi \to \infty$:
\begin{align*}
\dv[2]{\xi} \psi \simeq \xi^{2} \, \psi
\end{align*}
La forma de remover las singularidades en el infinito es con una función exponencial, con la potencia más alta, con la cual diverge la ecuación diferencial, es decir, con $\xi^{2}$.
\par
La solución propuesta con la que trabajaremos es
\begin{align*}
\psi = f(\xi) \, e^{-a \xi^{2}}
\end{align*}
La justificación de esta función de prueba es construir una función que converga, para ello removeremos el término $\xi^{2}$ mediante la función exponencial, posteriormente encontraremos la expansión en una serie de potencias de la función $f(\xi)$.
\par
Para simplificar, podemos ocupar la regla de Leibinz para obtener la derivada de orden $n$ de un producto de funciones:
\begin{align*}
\dv[n]{x} f(x) \cdot g(x) = \sum_{m=0}^{n} \begin{pmatrix}
n \\ m \end{pmatrix} \, \dv[n-m]{f(x)}{x}  \, \dv[m]{g(x)}{x}
\end{align*}
Entonces tendremos que:
\begin{align*}
\psi &= f(\xi) \, e^{-a \xi^{2}} \\[0.5em]
\ptilde{\psi} &= \ptilde{f}(\xi) \, e^{-a \xi^{2}} - 2 \, a \, \xi  \, e^{-a \xi^{2}} \, f(\xi) = \\[0.5em]
&= e^{-a \xi^{2}} \big[ \ptilde{f}(\xi) -  2 \, a \, \xi \, f(\xi) \big] \\[0.5em]
\stilde{\psi} &= \stilde{f}(\xi) \, e^{-a \xi^{2}} - 4 \, a \, e^{-a \xi^{2}} \, \ptilde{f}(\xi) \, \xi + f(\xi) \big[ 4 \, a^{2} \, \xi^{2} - 2 \, a \big] = \\[0.5em]
&= e^{-a \xi^{2}} \big[ \stilde{f}(\xi) - 4 \, a \, \ptilde{f}(\xi) \, \xi + f(\xi) (4 \, a^{2} \xi^{2} - 2 \, a) \big]
\end{align*}
Sustituyendo las expresiones anteriores en la ecuación de Schrödinger, tenemos que:
\begin{align*}
e^{-a \xi^{2}} \bigg[ &\stilde{f}(\xi) - 4 \, a \, \ptilde{f}(\xi) \, \xi + f(\xi) (4 \, a^{2} \xi^{2} - 2 \, a) \bigg] + \\[0.5em]
&+ k \, e^{-a \xi^{2}} \, f(\xi) - \xi^{2} \, e^{-a \xi^{2}} \, f(\xi) = 0
\end{align*}
Por lo que:
\begin{align*}
\stilde{f}(\xi) - 4 \, a \, \ptilde{f}(\xi) \, \xi + f(\xi) \big[ 4 \, a^{2} \xi^{2} - 2 \, a) \big] + k \, f(\xi) - \xi^{2} \, f(\xi) = 0
\end{align*}
Removemos los términos con potencias $\xi^{2}$
\begin{align*}
\stilde{f}(\xi) - 4 \, a \, \ptilde{f}(\xi) \, \xi + f(\xi) \big[ 4 \, a^{2} \xi^{2} - 2 \, a - \xi^{2} +  k) \big] = 0
\end{align*}
Por tanto:
\begin{align*}
4 \, a^{2} \, \xi^{2} =  \xi^{2} \hspace{0.2cm} \Longrightarrow \hspace{0.2cm} a = \dfrac{1}{2}
\end{align*}
Así llegamos a:
\begin{align*}
\stilde{f}(\xi) - 4 \, a \, \xi \, \ptilde{f}(\xi) + (k - 2 \, a) \, f(\xi) = 0
\end{align*}
Entonces
\begin{align*}
\addtolength{\fboxsep}{5pt}\boxed{
\stilde{f} (\xi) - 2 \, \xi \, \ptilde{f}(\xi) + (k - 1) \, f(\xi) = 0}
\end{align*}
que es la \emph{ecuación diferencial de Hermite}.
\par
El siguiente paso es resolver la ecuación diferencial de Hermite mediante una serie de potencias.
\par
Sea
\begin{align*}
f(\xi) = \sum_{\ell=0}^{\infty} a_{\ell} \, \xi^{\ell}
\end{align*}
Se propones esta solución ya que la ecuación de Hermite no posee singularidades en un valor infinito.
\par
Entonces diferenciamos la solución en dos ocasiones:
\begin{align*}
\ptilde{f}(\xi) &= \sum_{\ell=1}^{\infty} a_{\ell} \, \ell \, \xi^{\ell-1} \\[0.5em]
\stilde{f}(\xi) &= \sum_{\ell=2}^{\infty} a_{\ell} \, \ell \, (\ell - 1) \, \xi^{\ell-2} 
\end{align*}
Entonces al sustituir en la ecuación diferencial:
\begin{align*}
\underbrace{\sum_{\ell=2}^{\infty} a_{\ell} \, \ell \, (\ell - 1) \, \xi^{\ell-2} - 2 \, \ell \, \sum_{\ell=1}^{\infty} a_{\ell} \, \ell \, \xi^{\ell-1}}_{\text{esta suma se reorganiza para comenzar en } 0} + (k - 1) \sum_{\ell=0}^{\infty} a_{\ell} \, \xi^{\ell} = 0
\end{align*}
por lo que
\begin{align*}
\underbrace{\sum_{\ell=2}^{\infty} a_{\ell} \, \ell \, (\ell - 1) \, \xi^{\ell-2}}_{\text{se recorren los índices}} - 2 \underbrace{\sum_{\ell=0}^{\infty} a_{\ell} \, \ell \, \xi^{\ell}}_{\text{(*)}} + (k - 1) \sum_{\ell=0}^{\infty} a_{\ell} \, \xi^{\ell} = 0
\end{align*}
(*) ponemos el primer contador en cero, ya que el primer término es nulo.
\par

Agrupando las sumas, tenemos que:
\begin{align*}
\sum_{\ell=0}^{\infty} \big[ a_{\ell+2} \, (\ell + 2)(\ell + 1) + (- 2 \, \ell + k - 1) \, a_{\ell} \big] \, \xi^{\ell} = 0
\end{align*}
de donde obtenemos la relación de recurrencia:
\begin{align*}
a_{\ell + 2} = \dfrac{2\, \ell - (k - 1)}{(\ell + 2)(\ell + 1)} \, a_{\ell}
\end{align*}
Nótese que la relación es entre ordenes de dos grados consecutivos, es decir, la solución construye una función de paridad definida, lo que corresponde a la situación física que modelamos.
\par
Consideremos entonces las siguientes condiciones iniciales:
\begin{align*}
a_{0} \neq 0, \hspace{1cm} a_{1} = 0
\end{align*}
Entonces:
\begin{align*}
a_{2} &= \dfrac{- (k - 1)}{(2)(1)} \, a_{0} = \dfrac{-(k - 1)}{2!} \, a_{0} \\[0.5em]
a_{4} &= \dfrac{4 - (k - 1)}{(4)(3)} \, a_{2} = \dfrac{4 -(k - 1)(0 - (k - 1))}{4!} \, a_{0} \\[0.5em]
a_{6} &= \dfrac{6 - (k - 1)}{(6)(5)} \, a_{4} = \dfrac{(6 - k - 1)) (4 -(k - 1))(0 - (k - 1))}{6!} \, a_{0} \\[0.5em]
\vdots \\[0.5em]
a_{m} &= \dfrac{[m - (k - 1)]}{m (m -1)} \, a_{m-2}
\end{align*}
Tomamos ahora el otro caso: $a_{0} = 0, \hspace{0.2cm} a_{1} \neq 0$, entonces la relación de recurrencia es:
\begin{align*}
a_{3} &= \dfrac{[2 - (k - 1)]}{(2)(3)} \, a_{1} \\[0.5em]
a_{5} &= \dfrac{[6 - (k - 1)]}{(5)(4)} \, a_{3} = \dfrac{[6 - (k - 1)][2 - (k -  1)]}{5!} \, a_{1} \\[0.5em]
a_{7} &= \dfrac{[10 - (k - 1)][6 - (k - 1)]}{7!} \, a_{1} \\[0.5em]
\vdots \\[0.5em]
a_{4n+3} &= \dfrac{[4 \, n + 2 - (k - 1)]}{(4 \, n + 2)} \, a_{4n+1}
\end{align*}
Notemos que las series tienen la forma:
\begin{align*}
f(\xi) = a_{0} + a_{2} \, \xi^{2} + a_{4} \, \xi^{4} + \ldots + a_{2n} \, \xi^{2n} \simeq e^{\xi^{2}}
\end{align*}
y
\begin{align*}
f(\xi) &= a_{1} \, \xi + a_{3} \, \xi^{3} + \ldots + a_{2n+1} \, \xi^{2n+1} = \\[0.5em]
&= \xi \left( a_{1} + a_{3} \, \xi^{2} + \ldots + a_{2n+1} \, \xi^{2} \right) \simeq \xi \, e^{\xi^{2}}
\end{align*}
Es decir, las series formadas divergen como una exponencial cuadrática, recordemos que nuestra función de onda ha sido propuesta como
\begin{align*}
\psi = e^{\xi^{2}/k} \, f(\xi)
\end{align*}
Entonces la función de onda será divergente a menos que la serie $f(\xi)$ se vea truncada. Para conseguirlo, tomamos la condición de los coeficientes, es decir, la estructura recurrente de
\begin{align*}
\text{Caso par} \hspace{0.3cm} &a_{m} = \dfrac{[m - (k - 1)]}{m(m - 1)} \, a_{m-2} \\[0.5em]
\text{Caso impar} \hspace{0.3cm} &a_{4m+3} = \dfrac{[4 \, m + 2 - (k - 1)]}{4 \, m + 2)} \, a_{4m+1}
\end{align*}
Para ambos casos tenemos:
\begin{align*}
a_{\ell + 2} = \dfrac{a_\ell [2 \, \ell - k + 1]}{(\ell + 2)(\ell + 1)}
\end{align*}
Entonces para que con cierto valor $m$ la serie se anule, tomamos:
\begin{align*}
2 \, m - k + 1 = 0 \hspace{0.3cm} \Longrightarrow \hspace{0.3cm} 2 \, m + 1 = k = \dfrac{2 \, E}{\hbar \, \omega}
\end{align*}
Así tenemos que:
\begin{align*}
E = \hbar \, \omega \left( m + \dfrac{1}{2} \right)
\end{align*}
Donde vemos que la energía está cuantizada y tiene valores igualmente espaciados.
\par
Entonces hemos obtenido una solución polinomial para la función de onda:
\begin{align*}
\psi = e^{\xi^{2}/2} \, H_{n} (\xi)
\end{align*}
donde los $H_{n} (\xi)$ son polinomios.
\par
Cabe señalar que aún no hemos resuelto los valores de $a_{0}$ y de $a_{1}$, éstos se calculan mediante la condición de normalización:
\begin{align*}
\int_{-\infty}^{\infty} \psi^{*} (\xi) \, \psi(\xi) \dd{\xi} = \int_{-\infty}^{\infty} e^{\xi^{2}} \, H_{n} (\xi) \, H_{m} (\xi) =  1
\end{align*}
Esas integrales se analizarán en el \emph{Tema 3 - Bases completas}; por ahora listemos los resultados que se justificarán más adelante:
\begin{align*}
H_{0} &= 1 \\[0.5em]
H_{1} &= 2 \, x \\[0.5em]
H_{2} &= 4 \, x^{2} - 2 \\[0.5em]
H_{3} &= 8 \, x^{3} - 12 \, x \\[0.5em]
H_{4} &= 16 \, x^{4} - 48 \, x^{2} + 12 \\[0.5em]
H_{5} &= 32 \, x^{5} - 160 \, x^{3} + 120 \, x \\[0.5em]
\vdots
\end{align*}

\end{document}