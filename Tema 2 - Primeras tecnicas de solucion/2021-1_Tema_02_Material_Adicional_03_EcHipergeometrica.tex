\documentclass[hidelinks,12pt]{article}
\usepackage[left=0.25cm,top=1cm,right=0.25cm,bottom=1cm]{geometry}
%\usepackage[landscape]{geometry}
\textwidth = 20cm
\hoffset = -1cm
\usepackage[utf8]{inputenc}
\usepackage[spanish,es-tabla]{babel}
\usepackage[autostyle,spanish=mexican]{csquotes}
\usepackage[tbtags]{amsmath}
\usepackage{nccmath}
\usepackage{amsthm}
\usepackage{amssymb}
\usepackage{mathrsfs}
\usepackage{graphicx}
\usepackage{subfig}
\usepackage{standalone}
\usepackage[outdir=./Imagenes/]{epstopdf}
\usepackage{siunitx}
\usepackage{physics}
\usepackage{color}
\usepackage{float}
\usepackage{hyperref}
\usepackage{multicol}
%\usepackage{milista}
\usepackage{anyfontsize}
\usepackage{anysize}
%\usepackage{enumerate}
\usepackage[shortlabels]{enumitem}
\usepackage{capt-of}
\usepackage{bm}
\usepackage{relsize}
\usepackage{placeins}
\usepackage{empheq}
\usepackage{cancel}
\usepackage{wrapfig}
\usepackage[flushleft]{threeparttable}
\usepackage{makecell}
\usepackage{fancyhdr}
\usepackage{tikz}
\usepackage{bigints}
\usepackage{scalerel}
\usepackage{pgfplots}
\usepackage{pdflscape}
\pgfplotsset{compat=1.16}
\spanishdecimal{.}
\renewcommand{\baselinestretch}{1.5} 
\renewcommand\labelenumii{\theenumi.{\arabic{enumii}})}
\newcommand{\ptilde}[1]{\ensuremath{{#1}^{\prime}}}
\newcommand{\stilde}[1]{\ensuremath{{#1}^{\prime \prime}}}
\newcommand{\ttilde}[1]{\ensuremath{{#1}^{\prime \prime \prime}}}
\newcommand{\ntilde}[2]{\ensuremath{{#1}^{(#2)}}}

\newtheorem{defi}{{\it Definición}}[section]
\newtheorem{teo}{{\it Teorema}}[section]
\newtheorem{ejemplo}{{\it Ejemplo}}[section]
\newtheorem{propiedad}{{\it Propiedad}}[section]
\newtheorem{lema}{{\it Lema}}[section]
\newtheorem{cor}{Corolario}
\newtheorem{ejer}{Ejercicio}[section]

\newlist{milista}{enumerate}{2}
\setlist[milista,1]{label=\arabic*)}
\setlist[milista,2]{label=\arabic{milistai}.\arabic*)}
\newlength{\depthofsumsign}
\setlength{\depthofsumsign}{\depthof{$\sum$}}
\newcommand{\nsum}[1][1.4]{% only for \displaystyle
    \mathop{%
        \raisebox
            {-#1\depthofsumsign+1\depthofsumsign}
            {\scalebox
                {#1}
                {$\displaystyle\sum$}%
            }
    }
}
\def\scaleint#1{\vcenter{\hbox{\scaleto[3ex]{\displaystyle\int}{#1}}}}
\def\bs{\mkern-12mu}


\title{Ecuación hipergeométrica \\ \large{Matemáticas Avanzadas de la Física}\vspace{-3ex}}
\author{M. en C. Abraham Lima Buendía}
\date{ }
\begin{document}
\vspace{-4cm}
\maketitle
\fontsize{14}{14}\selectfont
\section{La ecuación hipergeométrica.}
Consideremos la siguiente ecuación diferencial
\begin{align*}
x (x - 1) \, \stilde{y} + [(1 + a + b) \, x - c] \, \ptilde{y} + a \, b \, y = 0
\end{align*}
que es conocida como la \emph{ecuación diferencial hipergeométrica}. Esta ecuación posee $3$ singularidades, para analizar las singularidades reescribimos de la siguiente manera:
\begin{align*}
\stilde{y} + \dfrac{[(1 + a + b) \, x - c]}{x (x -1)} \, \ptilde{y} + \dfrac{a \, b}{x (x - 1)} \, y = 0
\end{align*}
Este proceso nos permite evaluar las singularidades en $x = 0$ y $x = 1$, la tercera singularidad está en $x = \infty$, la cual puede resolverse al hacer el cambio de variable $z = 1 /x$ en la ecuación diferencial.
\par
Desarrollaremos la solución en torno al punto $x = 0$, tomamos entonces:
\begin{align*}
y &= \sum_{\ell=0}^{\infty} q_{\ell} \, x^{\ell + \lambda} \\[0.5em]
\ptilde{y} &= \sum_{\ell=0}^{\infty} q_{\ell} \, (\ell + \lambda) \, x^{\ell+\lambda-1} \\[0.5em]
\stilde{y} &= \sum_{\ell=0}^{\infty} q_{\ell} \, (\ell + \lambda) \, (\ell + \lambda - 1) \, x^{\ell+\lambda-2}
\end{align*}
Sustituimos en la ecuación diferencial para obtener:
\begin{align*}
&x^{2} \, \sum_{\ell=0}^{\infty} q_{\ell} \, (\ell + \lambda) \, (\ell + \lambda - 1) \, x^{\ell+\lambda-2} - x \, \sum_{\ell=0}^{\infty} q_{\ell} \, (\ell + \lambda) \, (\ell + \lambda - 1) \, x^{\ell+\lambda-2} + \\[0.5em]
+& (1 + a + b) \, \sum_{\ell=0}^{\infty} q_{\ell} \, (\ell + \lambda) \, x^{\ell+\lambda} - c \, \sum_{\ell=0}^{\infty} q_{\ell} \, (\ell + \lambda) \, x^{\ell+\lambda-1} + \\[0.5em]
&+ a \, b \, \sum_{\ell=0}^{\infty} q_{\ell} \, x^{\ell + \lambda} = 0
\end{align*}
Desarrollamos los productos:
\begin{align*}
&\sum_{\ell=0}^{\infty} q_{\ell} \, (\ell + \lambda) \, (\ell + \lambda - 1) \, x^{\ell+\lambda} - \sum_{\ell=0}^{\infty} q_{\ell} \, (\ell + \lambda) \, (\ell + \lambda - 1) \, x^{\ell+\lambda-1} + \\[0.5em]
&+ (1 + a + b) \, \sum_{\ell=0}^{\infty} q_{\ell} \, (\ell + \lambda) \, x^{\ell+\lambda} - c \, \sum_{\ell=0}^{\infty} q_{\ell} \, (\ell + \lambda) \, x^{\ell+\lambda-1} + \\[0.5em]
&+ a \, b \, \sum_{\ell=0}^{\infty} q_{\ell} \, x^{\ell + \lambda} = 0
\end{align*}
Reorganizamos las series:
\begin{align*}
&\sum_{\ell}^{\infty} q_{\ell} \big[ (\ell + \lambda) (\ell + \lambda + a + b) + a \, b \big] \, x^{\ell + \lambda} + \\[0.5cm]
&- \sum_{\ell=0}^{\infty} q_{\ell} \, (\ell + \lambda) (\ell + \lambda - 1 + c) \, x^{\ell+\lambda-1} = 0
\end{align*}
Tomamos la potencia más baja para remover la singularidad, esto es, en $\ell = 0$, por lo que tomamos:
\begin{align*}
q_{0} \big[ (\lambda) (\lambda + a + b) + a \, b \big] \, x^{\lambda} - q_{0} \, \lambda (\lambda - 1 + c) \, x^{\lambda} \, \dfrac{1}{x} = 0
\end{align*}
Entonces tenemos que la potencia más baja es $x^{-1}$ y deseamos removerla, con ello tenemos que la ecuación de índices es:
\begin{align*}
\lambda (\lambda - 1 + c) = 0
\end{align*}
cuyas raíces son:
\begin{align*}
\lambda = 0 \hspace{1cm} \lambda = 1 - c
\end{align*}
que serán las raíces con las que trabajaremos.
\par
Tomaremos el caso cuando $\lambda = 0$:
\begin{align*}
\sum_{\ell=0}^{\infty} q_{\ell} \big[ (\ell) (\ell + a + b) + a \, b \big] \, x^{\ell} - \underbrace{\sum_{\ell=0}^{\infty} q_{\ell} \, \ell (\ell + c - 1) \, x^{\ell-1}}_{\text{(*)}} = 0
\end{align*}
(*) Esta suma comienza en $\ell = 1$, por el factor $\ell$ que multiplica a $a_{\ell}$.

Entonces:
\begin{align*}
\sum_{\ell=0}^{\infty} q_{\ell} \big[ (\ell) (\ell + a + b) + a \, b \big] \, x^{\ell} - \sum_{\ell=1}^{\infty} \underbrace{q_{\ell}}_{\text{*}} \, \ell (\ell + c - 1) \, x^{\ell-1} = 0
\end{align*}
(*) Reescribimos este término, reorganizando los índices.

Así pues:
\begin{align*}
\sum_{\ell=0}^{\infty} q_{\ell} \big[ (\ell^{2} + \ell \, a + \ell \, b + a \, b \big] \, x^{\ell} - \sum_{\ell=0}^{\infty} q_{\ell+1} \, (\ell + 1) (\ell + c) \, x^{\ell} = 0
\end{align*}
Que de manera equivalente es:
\begin{align*}
\sum_{\ell=0}^{\infty} q_{\ell} \big[ (\ell (\ell + a) + (\ell + a) \, b \big] \, x^{\ell} - \sum_{\ell=0}^{\infty} q_{\ell+1} \, (\ell + 1) (\ell + c) \, x^{\ell} = 0
\end{align*}
Entonces:
\begin{align*}
\sum_{\ell=0}^{\infty} q_{\ell} \big[ (\ell (\ell +  b) + (\ell + a) \big] \, x^{\ell} - \sum_{\ell=0}^{\infty} q_{\ell+1} \, (\ell + 1) (\ell + c) \, x^{\ell} = 0
\end{align*}
por lo que:
\begin{align*}
\sum_{\ell=0}^{\infty} \bigg[ q_{\ell} \big[ (\ell (\ell +  b) + (\ell + a) \, b \big] \, x^{\ell} - \sum_{\ell=0}^{\infty} q_{\ell+1} \, (\ell + 1) (\ell + c) \, x^{\ell} \bigg] \, x^{\ell} = 0
\end{align*}
De donde
\begin{align*}
q_{\ell+1} = q_{\ell} \dfrac{(\ell +  b) (\ell + a)}{(\ell + 1) (\ell + c)}
\end{align*}
Con esto construimos los coeficientes de la serie:
\begin{align*}
q_{0} &\neq 0 \\[0.5em]
q_{1} &= q_{0} \, \dfrac{(0+b)(0+1)}{(0+c)(1)} \\[0.5em]
q_{2} &= q_{1} \, \dfrac{(1+b)(1+a)}{(1+c)(2)} = q_{0} \, \dfrac{[(b)(1+b)][(a)(a+1)]}{[(c)(1+c)] \, 2!} \\[0.5em]
q_{3} &= q_{2} \, \dfrac{(2+b)(2+a)}{(2+c)(3)} = q_{0} \, \dfrac{[(b)(1+b)(2+b)][(a)(a+1)]}{[(c)(1+c)(2+c)] \, 3!} \\[0.5em]
&\vdots
\end{align*}
Para reescribir los coeficientes de la siguiente manera:
\begin{align*}
\mathlarger{\Gamma}(\ell + n) &= (\ell + n - 1)! = 1 \cdot 2 \cdot 3 \ldots (\ell + n - 1) \\[0.5em]
\mathlarger{\Gamma} (\ell) &= (\ell - 1)! = 1 \cdot 2 \ldots (\ell - 1) \\[0.5em]
&\Rightarrow \dfrac{\mathlarger{\Gamma(\ell + n)}}{\mathlarger{\Gamma}} = \ell (\ell + 1) (\ell + 2) \ldots (\ell + n -1)  = (\ell)_{n}
\end{align*}
El término $(\ell)_{n}$ se conoce como \emph{símbolo de Pochhammer}, con esto, el $n$-ésimo coeficiente está determinado por:
\begin{align*}
q_{n} =  q_{0} \, \dfrac{(b)_{n} \, (a)_{n}}{(c_{n}) \, n!}
\end{align*}
De esta forma tendremos
\begin{align*}
y(x) = \sum_{n=0}^{\infty} \dfrac{(b)_{n} \, (a)_{n}}{(c_{n}) \, n!} \, x^{n} = {}_{2}F_{1} (a, b, c, x)
\end{align*}
Donde ${}_{2}F_{1} (a, b, c, x)$ representa la función hipergeométrica:
\begin{align*}
\addtolength{\fboxsep}{5pt}\boxed{
y(x) = {}_{2}F_{1} (a, b, c, x) }
\end{align*}
Consideremos el segundo caso para la raíz, es decir $\lambda = 1 - c$, por lo que tenemos:
\begin{align*}
&\sum_{\ell=0}^{\infty} q_{\ell} (\ell + 1 - c)\big[ (\ell + 1 - c + a + b ) + a \, b \big] \, x^{\ell} + \\[0.5em]
&- \sum_{\ell=0}^{\infty} q_{\ell} (\ell + 1 - c) (\ell + 1 - c - 1 + c) \, x^{\ell-1} = 0
\end{align*}
Volvemos a reagrupar términos:
\begin{align*}
&\sum_{\ell=0}^{\infty} q_{\ell} (\ell + 1 - c) \big[ (\ell + 1 - c + [ a + b] + a \, b \big] \, x^{\ell} + \\[0.5em]
&- \sum_{\ell=0}^{\infty} q_{\ell} (\ell + 1 - c) (\ell) \, x^{\ell-1} = 0
\end{align*}
Entonces:
\begin{align*}
&\sum_{\ell=0}^{\infty} q_{\ell} \bigg[ (\ell + 1 - c)^{2} + (\ell + 1 - c) \, a (\ell + 1 - c) \, b] + a \, b \bigg] \, x^{\ell} + \\[0.5em]
&- \sum_{\ell=0}^{\infty} q_{\ell} (\ell + 1 - c) (\ell) \, x^{\ell-1} = 0
\end{align*}
Por lo que:
\begin{align*}
&\sum_{\ell=0}^{\infty} q_{\ell} \bigg[ (\ell + 1 - c)^{2} + (\ell + 1 - c) \, a (\ell + 1 - c) \, b] + a \, b \bigg] \, x^{\ell} + \\[0.5em]
&- \sum_{\ell=0}^{\infty} q_{\ell+1} (\ell + 2 - c) (\ell + 1) \, x^{\ell} = 0
\end{align*}
así:
\begin{align*}
&\sum_{\ell=0}^{\infty} q_{\ell} \bigg[ (\ell + 1 - c) (\ell + 1 - c + a) (\ell + 1 - c +  a) \, b] + b \bigg] \, x^{\ell} + \\[0.5em]
&- \sum_{\ell=0}^{\infty} q_{\ell+1} (\ell + 2 - c) (\ell + 1) \, x^{\ell} = 0
\end{align*}
que al agrupar los términos:
\begin{align*}
\sum_{\ell=0}^{\infty} \bigg[ q_{\ell} \, (\ell + 1 - c + a) (\ell + 1 - c + b) + q_{\ell+1} (\ell + 2 - c)(\ell + 1) \bigg] \, x^{\ell} = 0
\end{align*}
De donde obtenemos la siguiente regla de recurrencia:
\begin{align*}
\addtolength{\fboxsep}{5pt}\boxed{
q_{\ell + 1} = q_{\ell} \, \dfrac{(\ell + 1 - c + a) (\ell + 1 - c + b)}{(\ell + 2 - c)(\ell + 1)}}
\end{align*}
La solución es análoga al caso $\lambda = 0$, pero en esta ocasión los coeficientes tienen la forma:
\begin{align*}
q_{\ell+1} = q_{\ell} \dfrac{(a + 1 -c)_{\ell} \, (b + 1 - c)_\ell}{(2 - c)_{\ell} \, \ell!}
\end{align*}
Por lo que:
\begin{align*}
y(x) = x^{1-c} \, \sum_{\ell=0}^{\infty} \dfrac{(a + 1 -c)_{\ell} \, (b + 1 - c)_\ell}{(2 - c)_{\ell} \, \ell!} \, x^{\ell}
\end{align*}
Que en términos de la función hipergeométrica:
\begin{align*}
\addtolength{\fboxsep}{5pt}\boxed{
y(x) = x^{1-c} \, {}_{2}F_{1} (a+1-c, b+1-c, 2-x; x)}
\end{align*}
\end{document}