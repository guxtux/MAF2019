\chapter{Capítulo 1 - Método de Frobenius}
\section{Puntos singulares}
Se presenta el concepto de un punto singular o singularidad (como se aplica a una ecuación diferencial). El interés en este concepto proviene de su utilidad en
\begin{enumerate}
\item Clasificar las ODE.
\item Revisar la viabilidad de una solución en series, esta viabilidad es parte del teorema de Fuchs.
\end{enumerate}
\par
Las ED que hemos mencionado previamente, pueden resolverse para $\displaystyle \dv[2]{y}{x}$, usando la notación $\displaystyle \dv[2]{y}{x} = y^{\prime \prime}$, tenemos:
\begin{align}
y^{\prime \prime} = f(x, y, y^{\prime})
\label{eq:ecuacion_09_74}
\end{align}
Si escribimos este EDO2 como
\begin{equation}
y^{\prime \prime} + P(x) \: y^{\prime} + Q(x) \: y = 0
\label{eq:ecuacion_09_75}
\end{equation}
De esta manera ya podemos definir los \emph{puntos ordinarios} y los \emph{puntos singulares}. Si las funciones $P(x)$ y $Q(x)$ son finitas en $x = x_{0}$, el punto $x = x_{0}$ es un \emph{punto ordinario}. Al 
contrario, si $P(x)$ y/o $Q(x)$ divergen mientras $x \to x_{0}$, el punto $x_{0}$ es un \emph{punto singular}.
\par
Usando la ecuación (\ref{eq:ecuacion_09_75}) podemos distinguir entre dos tipos de puntos singulares:
\begin{enumerate}
\item Si $P(x)$ y/o $Q(x)$ divergen mientras $x \to x_{0}$, pero $(x - x_{0}) \: P(x)$ y $(x - x_{0})^{2} \: Q(x)$ se mantiene finito mientras $x \to x_{0}$, entonces el punto $x = x_{0}$ se llama \textbf{punto singular regular o punto singular no esencial}.
\item Si $P(x)$ diverge más rápidamente que $1/(x - x_{0})$, de tal manera que $(x - x_{0}) \: P(x)$ tiene a infinito, mientras $x \to x_{0}$, o si $Q(x)$ diverge más rápidamente que $1/(x - x_{0})^{2}$, de tal manera que $(x - x_{0})^{2} \: Q(x)$ tiene a infinito, mientras $x \to x_{0}$, entonces el punto $x = x_{0}$ se llama \textbf{singularidad esencial o singularidad irregular}.
\end{enumerate}
\par
Estas definiciones son válidas para cualquier valor finito $x_{0}$. El análisis de los puntos al infinito $(x \to \infty)$ es similar al tratamiento que se hace para las funciones en variable compleja. Hacemos el cambio de variable $x = 1/z$, sustituyendo en la ED y entonces hacemos que $z \to 0$. Haciendo el cambio de variable en las derivadas:
\begin{equation}
\dv{y(x)}{x} = \dv{y(z^{-1})}{z} \: \dv{z}{x} = - \dfrac{1}{x^{2}} \dv{y(z^{-1})}{z} = -z^{2} \: \dv{y(z^{-1})}{z}
\label{eq:ecuacion_09_76}
\end{equation}
\begin{align}
\begin{aligned}
\dv[2]{y(x)}{x} = \dv{z} \left[ \dv{y(x)}{x} \right] \dv{z}{x} &= (-z^{2}) \left[ -2 \: z \dv{y(z^{-1})}{z} - z^{2} \: \dv[2]{y(z^{-1})}{z} \right] = \\
&= 2 \: z^{3} \: \dv{y(z^{-1})}{z} + z^{4} \: \dv[2]{y(z^{-1})}{z}
\end{aligned}
\end{align}
Usando estos resultados, podemos transformar la ecuación (\ref{eq:ecuacion_09_75}) en
\begin{equation}
z^{4} \: \dv[2]{y}{z} + [ 2 \: z^{3} - z^{2} \: P(z^{-1})] \: \dv{y}{z} + Q(z^{-1}) \: y = 0
\label{eq:ecuacion_09_78}
\end{equation}
El comportamiento en $x = \infty, (z = 0)$ entonces dependerá del comportamiento de los nuevos coeficientes
\[ \dfrac{2 \: z - P(z^{-1})}{z^{2}} \hspace{1cm} \text{ y } \hspace{1cm} \dfrac{Q(z^{-1})}{z^{4}}\]
mientras $z \to 0$.
\par
Si estas dos expresiones se mantienen finitas, el punto $x = \infty$ es un punto ordinario. Si las expresiones divergen no tan rápido como $1/z$ y $1/z^{2}$, respectivamente, el punto $x = \infty$ es un punto regular singular, de otra manera, el punto es irregular singular (una singularidad esencial).
\subsection*{Ejemplo: La ecuación de Bessel}
La ecuación de Bessel es
\begin{equation}
x^{2} \: y^{\prime \prime} + x \: y^{\prime} + (x^{2} - n^{2}) \: y = 0
\label{eq:ecuacion_09_79}
\end{equation}
Comparando contra la ecuación (\ref{eq:ecuacion_09_75}), tenemos que
\[ P(x) =  \dfrac{1}{x} \hspace{2cm} Q(x) = 1 - \dfrac{n^{2}}{x^{2}}\]
lo que muestra que le punto $x = 0$ es una singularidad regular.
\par
Por inspección vemos que no hay otros puntos singulares en el rango finito. Mientras $x \to \infty (z \to 0)$ de la ecuación (\ref{eq:ecuacion_09_78}) tenemos los coeficientes
\[ \dfrac{2 \:z - z}{z^{2}} \hspace{2cm} \dfrac{1 - n^{2} \: z^{2}}{z^{4}}\]
ya que la última expresión diverge como $z^{4}$, el punto $x = \infty$ es una singularidad irregular o esencial.
\section{Método de Frobenius}
En esta parte desarrollaremos un método para obtener una solución de la EDO2 lineal y homogénea. El método, que es un desarrollo en series, funcionará siempre y cuando el punto de expansión no es tan malo que un punto singular regular. En física, esta condición casi siempre satisface.
\par
Una EDO2H puede expresarse de la forma:
\begin{equation}
\boxed{\dv[2]{y}{x} + P(x) \: \dv{y}{x} + Q(x) \: y = 0}
\label{eq:ecuacion_09_80}
\end{equation}
Esta ecuación es homogénea, lineal y sin productos entre la función $y$ y sus derivadas. Con este método de Frobenius, se obtendrá al menos una solución de la ecuación.
\par
Más adelante veremos que se puede obtener una segunda solución independiente, y se demostrará que no existe una tercera solución independiente.
\par
La solución más general para la ecuación (\ref{eq:ecuacion_09_80}) se expresa por
\begin{equation}
\boxed{y(x) = c_{1} \: y_{1}(x) + c_{2} \: y_{2}}
\label{eq:ecuacion_09_81}
\end{equation}
En la realidad de la física, el problema nos puede conducir a una EDO2 no homogénea:
\begin{equation}
\boxed{\dfrac{d^{2} y}{d x^{2}} + P(x) \: \dfrac{dy}{dx} + Q(x) \: y = F(x)}
\label{eq:ecuacion_09_82}
\end{equation}
La función de la derecha, $F(x)$, representa una fuente (tal como una carga electrostática) o una fuerza de desplazamiento (como en el oscilador mecánico). Las soluciones específicas de esta ecuación no homogénea se pueden obtener usando las técnicas de la función de Green, y con la técnica de transformada de Laplace que se verá más adelante en el curso.
\par
Al llamar a este solución $y_{p}$, podemos agregarla en cualquier solución de la ecuación homogénea correspondiente (Ec. \ref{eq:ecuacion_09_82}). Por lo tanto \textbf{la solución más general} de la ecuación (\ref{eq:ecuacion_09_82}) es
\begin{equation}
\boxed{y(x) = c_{1} \: y_{1}(x) + c_{2} \: y_{2} + y_{p} (x) }
\label{eq:ecuacion_09_83}
\end{equation}
Las constantes $c_{1}$ y $c_{2}$ normalmente se establecen por las condiciones de frontera.
\par
Para nuestro estudio, suponemos que $F(x) = 0$ por lo que nuestra ecuación diferencial es homogénea.
\par
Intentaremos desarrollar una solución de nuestra EDO2H, la Ec. (\ref{eq:ecuacion_09_80}), mediante la sustitución en una serie de potencias con coeficientes indeterminados. Se manejará como parámetro si la potencia del menor del término de la serie es no nulo. Para ilustrar esto, veamos el método de dos ecuaciones diferenciales importantes, la primera es la ecuación del oscilador lineal
\begin{equation}
\dv[2]{y}{x} + \omega^{2} \: y = 0
\label{eq:ecuacion_09_84}
\end{equation}
de la que conocemos sus soluciones: $y= \sin \omega \: x, \cos \omega \: x$.
\par
Intentamos con
\begin{align}
\begin{aligned}
y(x) &= x^{k} (a_{0} + a_{1} \: x + a_{2} \: x^{2} + a_{3} \: x^{3} + \ldots ) \\
&= \sum_{\lambda = 0}^{\infty} a_{\lambda} \: x^{k + \lambda}, \hspace{1cm} a_{0} \neq 0
\end{aligned}
\label{eq:ecuacion_09_85}
\end{align}
donde el exponente $k$ y todos los coeficientes $a_{\lambda}$ son indeterminados. Nótese que no necesariamente $k$ es un entero. Diferenciando dos veces, tenemos
\begin{align*}
\dv{y}{x} &= \sum_{\lambda=0}^{\infty} a_{\lambda} \: (k + \lambda) x^{k + \lambda - 1} \\
\dv[2]{y}{x} &= \sum_{\lambda=0}^{\infty} a_{\lambda} \: (k + \lambda) (k + \lambda - 1) \: x^{k + \lambda - 2} \nonumber
\end{align*}
Al sustituir en la ecuación (\ref{eq:ecuacion_09_84}), obtenemos
\begin{equation}
\sum_{\lambda=0}^{\infty} a_{\lambda} \: (k + \lambda) \: (k + \lambda - 1) \: x^{k + \lambda - 2} + \omega^{2} \: \sum_{\lambda = 0}^{\infty} a_{\lambda} \: x^{k + \lambda} = 0
\label{eq:ecuacion_09_86}
\end{equation}
Del análisis de la unicidad de las series de potencias, los coeficientes de cada potencia de $x$ en la parte izquierda de la ecuación (\ref{eq:ecuacion_09_86}) deben de anularse.
\par
La potencia menor de $x$ que aparece en la ecuación (\ref{eq:ecuacion_09_86}) es $x^{k - 2}$ para $\lambda = 0$ en la primera suma. Para que el coeficiente se anule, se necesita que
\[ a_{0} \: k \: (k - 1) = 0 \]
Se escoge $a_{0}$ como el coeficiente no nulo del término menor de la serie \ref{eq:ecuacion_09_85}, por lo que de la definición, $a_{0} \neq 0$, por lo que tenemos
\begin{equation}
 k \: (k - 1) = 0
\label{eq:ecuacion_09_87}
\end{equation}
Esta ecuación, que proviene del coeficiente de la menor potencia de $x$, se llama la \emph{ecuación indicial} o \emph{ecuación de índices}. La ecuación indicial y sus raíces son muy importantes en este análisis.
\par
Si $k = 1$, el coeficiente $a_{1} \: (k + 1)k$ de $x^{k - 1}$ se anula, por lo que $a_{1} = 0$, en este ejemplo se nota de inmediato que $k = 0$ o $k = 1$.
\par
Antes de considerar estas dos posibilidades para $k$, regresemos a la ecuación (\ref{eq:ecuacion_09_86}) y la condición de que los coeficientes netos restantes, digamos, los coeficientes de $x^{k + j} \; (j \geq 0)$, se anulen.
\par
Hemos establecido $ \lambda = j + 2$ en la primera suma y $\lambda = j$ en el segunda. (Son sumas independientes y $\lambda$ es un índice mudo.) Esto da lugar a
\[ a_{j + 2} \: (k + j + 2) \: (k + j + 1) + \omega^{2} \: a_{j} = 0 \]
o
\begin{equation}
a_{j + 2} = - a_{j} \dfrac{\omega^{2}}{(k + j + 2) \: (k + j + 1)}
\label{eq:ecuacion_09_88}
\end{equation}
Esta es una relación de recurrencia de dos términos: dado $a_{j}$ podemos calcular $a_{j + 2}$ y luego $a_{j + 4}$, $a_{j + 6}$, y así sucesivamente hasta donde lo queramos. Tomemos en cuenta que para este ejemplo, si partimos con $a_{0}$ Ec. (\ref{eq:ecuacion_09_88}) conduce a los coeficientes pares $a_{2}$, $a_{4}$ y así sucesivamente, y no considera a $a_{1}$, $a_{3}$, $a_{5}, \ldots$. Dado que una $a_{1}$ es arbitrario, si $k = 0$ y necesariamente cero si $k = 1$, hacemos que sea igual a cero  y luego por la Ec. (\ref{eq:ecuacion_09_88})
\[ a_{3} = a_{5} = a_{7} = \ldots = 0 \]
todos los coeficientes impares se anulan. Las potencias pares de $x$ se presentan cuando se utiliza la segunda raíz de la ecuación indicial.
\par
Regresando a la ecuación (\ref{eq:ecuacion_09_87}) de la ecuación indicial, intentamos con la solución $k = 0$, la relación de recurrencia (Ec. \ref{eq:ecuacion_09_88})ahora es
\begin{equation}
a_{j + 2} = - a_{j} \: \dfrac{\omega^{2}}{(j+2) \:(j+1)}
\label{eq:ecuacion_09_89}
\end{equation}
que nos conduce a
\begin{align*}
a_{2} &= - a_{0} \: \dfrac{\omega^{2}}{1 \cdot 2} = - \dfrac{\omega^{2}}{2!} \: a_{0} \nonumber \\
a_{4} &= - a_{2} \dfrac{\omega^{2}}{3 \cdot 4} = + \dfrac{\omega^{4}}{4!} \: a_{0} \nonumber \\
a_{6} &= - a_{4} \dfrac{\omega^{2}}{5 \cdot 6} = - \dfrac{\omega^{6}}{6!} \: a_{0} \nonumber \hspace{1cm} \mbox{ y así sucesivamente}
\end{align*}
Aplicando inducción matemática, tenemos
\begin{equation}
a_{2n} = (-1)^{n} \: \dfrac{\omega^{2n}}{(2n)!} \: a_{0}
\label{eq:ecuacion_09_90}
\end{equation}
y la solución es
\begin{equation}
y(x)_{k = 0} = a_{0} \: \left[ 1 - \dfrac{(\omega x)^{2}}{2!} + \dfrac{(\omega x)^{4}}{4!} - \dfrac{(\omega x)^{6}}{6!} + \ldots \right] = a_{0} \: \cos \omega x
\label{eq:ecuacion_09_91}  
\end{equation}
Si elegimos la raíz $k = 1$ de la ecuación indicial (Ec. \ref{eq:ecuacion_09_88}), la relación de recurrencia es
\begin{equation}
a_{j + 2} = - a_{j} \: \dfrac{\omega^{2}}{(j+3) \: (j+2)}
\label{eq:ecuacion_09_92}
\end{equation}
sustituyendo en $j = 0, 2, 4$ sucesivamente, resulta
\begin{align*}
a_{2} &= - a_{0} \: \dfrac{\omega^{2}}{2 \cdot 3} = - \dfrac{\omega^{2}}{3!} \: a_{0} \nonumber \\
a_{4} &= - a_{2} \dfrac{\omega^{2}}{4 \cdot 5} = + \dfrac{\omega^{4}}{5!}  \: a_{0} \nonumber \\
a_{6} &= - a_{4} \dfrac{\omega^{2}}{6 \cdot 7} = - \dfrac{\omega^{6}}{7!} \: a_{0} \nonumber \hspace{1cm} \mbox{y así sucesivamente}
\end{align*}
Por inducción matemática
\begin{equation}
a_{2n} = (-1)^{n} \: \dfrac{\omega^{2n}}{(2n + 1)!} \: a_{0}
\label{eq:ecuacion_09_93}
\end{equation}
Para este valor $k = 1$, se obtiene
\begin{align}
y(x)_{k = 1} &= a_{0} \: x \: \left[ 1 - \dfrac{(\omega x)^{2}}{3!} + \dfrac{(\omega x)^{4}}{5!} - \dfrac{(\omega x)^{6}}{7!} + \ldots \right] \nonumber \\
&= \dfrac{a_{0}}{\omega} \: \left[ (\omega x) - \dfrac{(\omega x)^{3}}{3!} + \dfrac{(\omega x)^{5}}{5!} - \dfrac{(\omega x)^{7}}{7!} + \ldots \right] \nonumber \\
&= \dfrac{a_{0}}{\omega} \: \sin \omega x
\label{eq:ecuacion_09_94}
\end{align}
Esta sustitución en series de potencias, es conocida como el \emph{método de Frobenius}, y nos ha dado dos soluciones en series de la ecuación del oscilador lineal. Sin embargo, hay que considerar dos puntos sobre dichas soluciones en series en los que se debe de hacer énfasis:
\begin{enumerate}
\item La solución en series siempre debe de sustituirse en la ecuación diferencial, para ver si funciona, como medida de precaución contra los errores algebraicos y de lógica. Si funciona, es una solución.
\item Aceptar una solución en series depende de su convergencia (incluida la convergencia asintótica). Es muy posible que el método de Frobenius devuelva una solución en series que satisface la ecuación diferencial original cuando se sustituye en la ecuación, pero no converga en el intervalo de interés.
\end{enumerate}
%Referencia Echeverria - Tema 4
\subsection{Teorema de Frobenius.}
El Teorema de Frobenius permite hallar al menos una solución en forma de serie de potencias para la ecuación 
\begin{equation}
y^{\prime \prime} + p(x) \: y^{\prime} + q(x) \: y = 0
\label{eq:ecuacion_04_02_04}
\end{equation}
alrededor de $x_{0}$ cuando el punto es un punto singular regular.
\par
Entonces la ecuación (\ref{eq:ecuacion_04_02_04}) posee al menos una solución de la forma
\begin{equation}
y(x) = (x - x_{0})^{m} \: \sum_{k=0}^{\infty} c_{k} \: (x - x_{0})^{k}
\label{eq:ecuacion_04_02_05}
\end{equation}
donde $m$ es un número por determinar. Tal serie converge en el intervalo común de convergencia de
\begin{equation}
P(x) \equiv (x - x_{0}) \: p(x) \hspace{1cm} Q(x) \equiv (x - x_{0})^{2} \: q(x)
\label{eq:ecuacion_04_02_06}
\end{equation}
excepto quizás en el punto $x = x_{0}$.
\par
Sin pérdida de generalidad, puede tomarse $x_{0} = 0$ pues siempre puede realizarse un cambio de variable o traslación para centrar el problema alrededor del origen. En tal caso, para resolver la ec. (\ref{eq:ecuacion_04_02_04}) primero se escribe en función de la ec. (\ref{eq:ecuacion_04_02_06}) como
\begin{equation}
y^{\prime \prime} + \dfrac{P(x)}{x} \: y^{\prime} + \dfrac{Q(x)}{x^{2}} \: y = 0
\label{eq:ecuacion_04_02_07} 
\end{equation}
luego puede multiplicarse por $x^{2}$ a ambos lados para obtener la ecuación
\begin{equation}
x^{2} \: y^{\prime \prime} + x \: P(x) \: y^{\prime} + Q(x) \: y = 0
\label{eq:ecuacion_04_02_08} 
\end{equation}
Luego, como $P(x)$ y $Q(x)$ son analíticas alrededor de cero puede escribirse
\begin{equation}
P(x) \equiv \sum_{k=0}^{\infty} a_{k} \: x^{k} \hspace{1cm} Q(x) \equiv \sum_{k=0}^{\infty} b_{k} \: x^{k}
\label{eq:ecuacion_04_02_09}
\end{equation}
y sustituyendo en la ec. (\ref{eq:ecuacion_04_02_08}), se escribe
\begin{equation}
x^{2} \: y^{\prime \prime} + \sum_{k=0}^{\infty} a_{k} \: x^{k+1} \: y^{\prime} + \sum_{k=0}^{\infty} b_{k} \: x^{k} \: y = 0
\label{eq:ecuacion_04_02_10}
\end{equation}
Por el Teorema de Frobenius se busca una solución de la forma
\begin{equation}
y(x) = x^{m} \: \sum_{k=0}^{\infty} c_{k} \: x^{k} =  (c_{0} \: x^{m} + c_{1} \: x^{m+1} + c_{2} \: x^{m+2} + \ldots )
\label{eq:ecuacion_04_02_11}
\end{equation}
sin pérdida de generalidad puede tomarse $c_{0} \neq 0$ pues de lo contrario puede factorizarse un $x$ en la serie y escribir $x^{m+1}$ en vez de $x^{m}$. Sustituyendo la ec. (\ref{eq:ecuacion_04_02_11}) en la ec. (\ref{eq:ecuacion_04_02_10}) se obtiene
\begin{align}
\begin{aligned}
 &{} x^{2} \, (m (m-1) \, c_{0} \, x^{m-2} + (m+1) m \, c_{1} \, x^{m-1} + (m+2)(m+1) \, c_{2} \, x^{m} + \ldots ) \\
 &+ (a_{0} \, x + a_{1} \, x^{2} + a_{2} \, x^{3} + \ldots + )(m \, c_{0} \, x^{m-1} + (m+1) \, c_{1} \, x^{m} + (m+2) \, c_{2} \, x^{m+1} + \ldots ) \\
 &+ (b_{0} + b_{1} \, x + b_{2} \, + \ldots )(c_{0} \, x^{m} + c_{1} \, x^{m+1} + c_{2} \, x^{m+2} + \ldots ) = 0
\end{aligned}
\label{eq:ecuacion_04_02_12}
\end{align}
se puede observar que la menor potencia de $x$ que aparece al realizar los productos respectivos es $x^{m-2}$, de hecho si se agrupan los términos según la potencia de $x$, el coeficiente que multiplica a $x^{m-2}$ es
\begin{equation}
m(m-1) \: c_{0} + m \: a_{0} \: c_{0} + b_{0} \: c_{0}
\label{eq:ecuacion_04_02_13}
\end{equation}
dado que $c_{0} \neq 0$ y la serie de potencias está igualada a cero, cada coeficiente que multiplica a cada potencia de $x$ debe ser cero, en particular, el coeficiente que multiplica a $x^{m-2}$ debe ser cero, lo cual significa que $m$ debe cumplir la ecuación
\begin{equation}
m (m-1) + m \: a_{0} + b_{0} = 0
\label{eq:ecuacion_04_02_14}
\end{equation}
La ecuación (\ref{eq:ecuacion_04_02_14}) se llama la \emph{ecuación indicial}. Dado que es una ecuación cuadrática, en general hay dos raíces $m_{1}$ y $m_{2}$. Dependiendo de tales raíces, el método de Frobenius garantiza una segunda solución.
\subsection{Casos especiales.}
\subsubsection{Raíces con diferencia no entera.}
Si $m_{1}$, $m_{2}$ son las raíces de la ec.(\ref{eq:ecuacion_04_02_14}) y $m_{1} - m_{2} \not \in \mathbb{Z}$ entonces el método de Frobenius
\begin{equation}
y(x) = x^{m} \: \sum_{k=0}^{\infty} c_{k} \: x^{k}
\label{eq:ecuacion_04_02_15}
\end{equation}
genera dos soluciones linealmente independientes para la ec. (\ref{eq:ecuacion_04_02_04}):
\begin{equation}
y_{1}(x) = x^{m_{1}} \sum_{k=0}^{\infty} a_{k} \: x^{k} \hspace{1cm} y_{2}(x) = x^{m_{2}} \sum_{k=0}^{\infty} b_{k} \: x^{k}
\label{eq:ecuacion_04_02_16}
\end{equation}
\subsubsection{Raíces distintas con diferencia entera.}
Si $m_{1}$, $m_{2}$ son las raíces de la ec.(\ref{eq:ecuacion_04_02_14}) y si $m_{1} - m_{2} \in \mathbb{Z}$, suponiendo que $m_{1} > m_{2}$. Definimos
\begin{equation}
N \equiv m_{1} - m_{2}
\label{eq:ecuacion_04_02_37}
\end{equation}
\begin{enumerate}
\item Si después de expandir la ec. (\ref{eq:ecuacion_04_02_08}) en series de potencias se llega a que el coeficiente que multiplica a $x^{m_{2}+N}$ es automáticamente cero, entonces usando la raíz más pequeña se pueden hallar dos soluciones en series de Frobenius.
\item Si después de expandir (\ref{eq:ecuacion_04_02_08}) en series de potencias se llega a que el coeficiente que multiplica a $x^{m_{2}+N}$ no es automáticamente cero, entonces usando la raíz más grande hay una solución en serie de la forma
\begin{equation}
y_{1}(x) = x^{m_{1}} \sum_{k=0}^{\infty} a_{k} \: x^{k}
\label{eq:ecuacion_04_02_38}
\end{equation}
y la segunda solución es de la forma
\begin{equation}
y_{2}(x) = - b_{N} \: y_{1}(x) \: \ln x + x^{m_{2}} \sum_{k=0}^{\infty} b_{k} \: x^{k}
\label{eq:ecuacion_04_02_39}
\end{equation}
\end{enumerate}
\subsubsection{Raíces repetidas.}
Si $m_{1} = m_{2}$ en la ec. (\ref{eq:ecuacion_04_02_14}), las soluciones son de la forma:
\begin{align}
\begin{aligned}
y_{1}(x) &= x^{m_{1}} \sum_{k=0}^{\infty} a_{k} \: x^{k} \\
y_{2}(x) &= y_{1}(x) \: \ln x + x^{m_{1}} \sum_{k=0}^{\infty} b_{k} \: x^{k}
\end{aligned}
\label{eq:ecuacion_04_02_76}
\end{align}
\subsubsection{Ejercicios.}
Encuentra la solución general de las siguientes EDO2H mediante el método de Frobenius:
\begin{enumerate}
\item $2 \: x \: y^{\prime \prime} + (1 + x) \: y^{\prime} + y = 0 $
\item $x^{2} \: y^{\prime \prime} + x \: y^{\prime} + (x^{2} - \frac{1}{4}) \: y = 0 $
\item $x^{2} \: y^{\prime \prime} + x \: y^{\prime} + x^{2} \: y = 0$
\end{enumerate}
