\input{../Preambulos/preambulo_presentacion_CambridgeUS_beaver}
\title{\large{Ejercicio con Separación de variables}}
\subtitle{Tema 2 - Primeras técnicas de solución}
\author{M. en C. Gustavo Contreras Mayén}
\date{}
\institute{Facultad de Ciencias - UNAM}
\titlegraphic{\includegraphics[width=1.75cm]{../Imagenes/escudo-facultad-ciencias}\hspace*{4.75cm}~%
   \includegraphics[width=1.75cm]{../Imagenes/escudo-unam}
}
\setbeamertemplate{navigation symbols}{}
\begin{document}
\maketitle
\fontsize{14}{14}\selectfont
\spanishdecimal{.}
\section*{Contenido}
\frame{\tableofcontents[currentsection, hideallsubsections]}
\section{Clasificación de EDP}
\frame{\tableofcontents[currentsection, hideothersubsections]}
\subsection{Valor del discriminante}
\begin{frame}
\frametitle{EDP general}
Ya hemos comentado que una EDP en dos variables se puede escribir de la forma:
\begin{align*}
A \, u_{xx} + B \, u_{xy} + C \, u_{yy} + D \, u_{x} + E \, u_{y} + F \, u = 0
\end{align*}
\end{frame}
\begin{frame}
\frametitle{Clasificación de las EDP}
Una EDP se clasifica de acuerdo al discriminante:
\setbeamercolor{item projected}{bg=blue!70!black,fg=yellow}
\setbeamertemplate{enumerate items}[circle]
\begin{enumerate}[<+->]
\item Elíptico si $B^{2} - 4 \, A \, C < 0$
\item Hiperbólico si $B^{2} - 4 \, A \, C > 0$
\item Parabólico si $B^{2} = 4 \, A \, C$
\end{enumerate}
\end{frame}
\begin{frame}
\frametitle{Ejemplos}
A partir del discriminante identifica la siguientes EDP de dos variables.
\end{frame}
\begin{frame}
\frametitle{Ejemplo 1}
La EDP es:
\begin{align*}
u_{xx} + u_{yy} + 3 \, u_{xy} = 0
\end{align*}
\pause
Entonces tenemos que $A = 1$, $B = 3$ y $C = 1$, entonces:
\begin{align*}
B^{2} - 4 \, A \, C = (3)^{2} - 4 (1)(1) = 9 - 4 = 5 > 0
\end{align*}
\pause
entonces la EDP es \textcolor{blue}{hiperbólica}.
\end{frame}
\begin{frame}
\frametitle{Ejemplo 2}
La EDP es:
\begin{align*}
u_{xx} + u_{yy} + u_{xy} = 0
\end{align*}
\pause
Entonces tenemos que $A = 1$, $B = 1$ y $C = 1$, entonces:
\begin{align*}
B^{2} - 4 \, A \, C = (1)^{2} - 4 (1)(1) = 1 - 4 = -3 < 0
\end{align*}
\pause
entonces la EDP es \textcolor{blue}{elíptica}.
\end{frame}
\begin{frame}
\frametitle{Ejemplo 3}
La EDP es:
\begin{align*}
u_{xx} + u_{yy} + 2 \, u_{xy} = 0
\end{align*}
\pause
Entonces tenemos que $A = 1$, $B = 2$ y $C = 1$, entonces:
\begin{align*}
B^{2} - 4 \, A \, C = (2)^{2} - 4 (1)(1) = 4 - 4 = 0
\end{align*}
entonces la EDP es \textcolor{blue}{parabólica}.
\end{frame}
\begin{frame}
\frametitle{Ejemplo 4}
La EDP es:
\begin{align*}
u_{xx} + x \, u_{yy} = 0
\end{align*}
\pause
Entonces tenemos que $A = 1$, $B = 0$ y $C = x$, entonces:
\begin{align*}
B^{2} - 4 \, A \, C = (0)^{2} - 4 (1)(x) = - 4 \, x
\end{align*}
\pause
Tenemos que fijarnos en los valores de $x$:
\end{frame}
\begin{frame}
\frametitle{Ejemplo 4}
Si $x > 0$ entonces tenemos que el discriminante es negativo, por lo tanto la EDP es \textcolor{blue}{elíptica}.
\\
\bigskip
\pause
Si $x < 0$ entonces tenemos que el discriminante es positivo, por lo tanto la EDP es \textcolor{red}{hiperbólica}.
\end{frame}
\section{Ecuación de Laplace}
\frame{\tableofcontents[currentsection, hideothersubsections]}
\subsection{Planteamiento}
\begin{frame}
\frametitle{Problema de estudio}
Ahora consideramos la ecuación de Laplace en un cubo.
\\
\bigskip
Resolveremos el problema en un cubo cuyos lados tienen una longitud de $\pi$, en lugar de un paralelepípedo arbitrario, esto con la finalidad de hacer los cálculos algo menos engorrosos.
\end{frame}
\begin{frame}
\frametitle{Problema de estudio}
Debemos de considerar que el problema plantea el estudio en seis caras en lugar de cuatro bordes y las condiciones de frontera son funciones de dos variables.
\end{frame}
\begin{frame}
\frametitle{Características del problema}
El problema a resolver cuenta con las siguientes características
\fontsize{12}{12}\selectfont
\begin{align*}
\laplacian{u(x, y, z)} = \pdv[2]{u}{x} + \pdv[2]{u}{y} + \pdv[2]{u}{z} = 0 \hspace{1cm} \begin{cases}
0 < x < \pi \\
0 < y < \pi \\
0 < z < \pi 
\end{cases}
\end{align*}
\end{frame}
\begin{frame}
\frametitle{Condiciones iniciales}
Además tenemos que:
\begin{align*}
u(x, y , z) = 0 \mbox{  si  } x = 0, x = \pi, y = 0, y = \pi , z = \pi 
\end{align*}
\pause
\begin{align*}
u(x, y , 0) = f (x, y)
\end{align*}
\end{frame}
\subsection{Separación de variables}
\begin{frame}
\frametitle{Usando el método}
Para resolver esta ecuación mediante el método de separación de variables, haremos la suposición de que tiene una solución del tipo:
\begin{align*}
u(x, y, z) = X(x) \, Y(y) \, Z(z)
\end{align*}
\end{frame}
\begin{frame}
\frametitle{Usando el método}
Entonces tendremos que:
\begin{eqnarray*}
\pdv[2]{u}{x} &=& \stilde{X} (x) \, Y(y) \, Z (z) \\[0.5em] \pause
\pdv[2]{u}{y} &=& X (x) \, \stilde{Y} (y) \, Z (z) \\[0.5em] \pause
\pdv[2]{u}{z} &=& X (x) \, Y(y) \, \stilde{Z} (z)
\end{eqnarray*}
\end{frame}
\begin{frame}
\frametitle{Sustituyendo}
Al sustituir en la ec. de Laplace, tendremos:
\begin{align*}
\pdv[2]{u}{x} + \pdv[2]{u}{y} + \pdv[2]{u}{z} &= \stilde{X} (x) \, Y(y) \, Z (z) + \\[0.5em]
&+ X (x) \, \stilde{Y} (y) \, Z (z) + \\[0.5em]
&+ X (x) \, Y(y) \, \stilde{Z} (z) = 0
\end{align*}
\end{frame}
\begin{frame}
\frametitle{Dividiendo términos}
Al dividir entre $X(x) \, Y(y) \, Z(z)$ obtenemos:
\begin{align*}
\dfrac{\stilde{X} (x)}{X(x)} + \dfrac{\stilde{Y} (y)}{Y(y)} + \dfrac{\stilde{Z} (z)}{Z(z)} =
\end{align*}
\pause
de manera equivalente:
\begin{align}
\dfrac{\stilde{X} (x)}{X(x)} + \dfrac{\stilde{Y} (y)}{Y(y)} = -  \dfrac{\stilde{Z} (z)}{Z(z)}
\label{eq:ecuacion_Kejer_01}
\end{align}
\end{frame}
\begin{frame}
\frametitle{Primera constante de separación}
Vemos que el lado izquierdo de la ec. (\ref{eq:ecuacion_Kejer_01}) es \emph{función solo de x e y}, mientras que el lado derecho es \emph{función solo de z}.
\\
\bigskip
\pause
Por lo que debe de ser igual a una constante: $\alpha$, la primera constante de separación.
\end{frame}
\begin{frame}
\frametitle{Ecuaciones resultantes}
Entonces ahora tenemos:
\begin{align*}
\dfrac{\stilde{X} (x)}{X(x)} + \dfrac{\stilde{Y} (y)}{Y(y)} &= \alpha \\[1em]
\dfrac{\stilde{Z} (z)}{Z(z)} &= - \alpha
\end{align*}
\end{frame}
\begin{frame}
\frametitle{Continua el método}
Tenemos entonces que:
\begin{align*}
\dfrac{\stilde{X} (x)}{X(x)} + \dfrac{\stilde{Y} (y)}{Y(y)} = \alpha
\end{align*}
\pause
Entonces es cierto que:
\begin{align}
\dfrac{\stilde{X} (x)}{X(x)} = \alpha - \dfrac{\stilde{Y} (y)}{Y(y)}
\label{eq:ecuacion_Kejer_02}
\end{align}
\end{frame}
\begin{frame}
\frametitle{Segunda constante de Separación}
Siguiendo el razonamiento anterior, cada lado de la ec. (\ref{eq:ecuacion_Kejer_02}) debe ser igual a una constante.
\\
\bigskip
\pause
Sea $\beta$ la segunda constante de separación.
\end{frame}
\begin{frame}
\frametitle{Ecuaciones separadas}
Entonces hacemos:
\begin{align*}
\beta = \dfrac{\stilde{X} (x)}{X(x)} 
\end{align*}
\pause
Por tanto
\begin{align*}
\stilde{X} (x) - \beta \, X(x) = 0
\end{align*}
\end{frame}
\begin{frame}
\frametitle{Ecuaciones separadas}
Además:
\begin{align*}
\dfrac{\stilde{Y} (y)}{Y(y)} = \alpha - \dfrac{\stilde{X} (x)}{X(x)} = \alpha - \beta
\end{align*}
\pause
Entonces:
\begin{align*}
\stilde{Y} (y) - (\alpha - \beta) Y(y) = 0
\end{align*}
\end{frame}
\begin{frame}
\frametitle{Usando las CDF}
Tomamos las EDO2H para cada variable y ocupamos las CDF:
\begin{eqnarray}
\stilde{X} (x) - \beta \, X(x) &=& 0, \hspace{0.5cm} X(0) = X(\pi) = 0 \label{eq:ecuacion_Kejer_03} \\[0.5em] \pause
\stilde{Y} (y) - (\alpha - \beta) Y(y) &=& 0, \hspace{0.5cm} Y(0) = Y(\pi) = 0 \label{eq:ecuacion_Kejer_04} \\[0.5em] \pause
\stilde{Z} (z) - \alpha \, Z(z) &=& 0, \hspace{0.5cm} Z(\pi) = 0 \label{eq:ecuacion_Kejer_05}
\end{eqnarray}
\end{frame}
\begin{frame}
\frametitle{Encontrando las constantes}
Vamos a determinar el valor apropiado para las constantes. \pause De la ecuación
\begin{align*}
\stilde{X} (x) - \beta \, X(x) = 0, \hspace{0.5cm} X(0) = X(\pi) = 0
\end{align*}
debe de ocurrir que $-\beta > 0$ por la condiciones de frontera.
\end{frame}
\subsection{Solución general}
\begin{frame}
\frametitle{Resolviendo la EDO}
Proponemos que $\beta = - n^{2}$.\pause  Por lo que la ec. (\ref{eq:ecuacion_Kejer_03}) es:
\begin{align*}
\stilde{X} (x) + n^{2} \, X(x) = 0, \hspace{0.5cm} X(0) = X(\pi) = 0
\end{align*}
\pause
La solución a esta EDO2H (que puedes corroborar) es:
\begin{align*}
X_{n} (x) =  \sin (n \, x)
\end{align*}
\end{frame}
\begin{frame}
\frametitle{Resolviendo la EDO}
De manera similar, para la ecuación
\begin{align*}
\stilde{Y} (y) - (\alpha - \beta) \, Y(y) = 0, \hspace{0.5cm} Y(0) = Y(\pi) = 0
\end{align*}
\pause
debe de ocurrir que $-(\alpha - \beta) > 0$. \pause Hacemos que $\alpha - \beta = - m^{2}$.
\end{frame}
\begin{frame}
\frametitle{Resolviendo la EDO}
Entonces tenemos que para la ec. (\ref{eq:ecuacion_Kejer_04}) es:
\begin{align*}
\stilde{Y} (y) + m^{2} \, Y(y) &=& 0, \hspace{0.5cm} Y(0) = Y(\pi) = 0
\end{align*}
\pause
que tiene por solución:
\begin{align*}
Y_{m} (y) = \sin (m \, y)
\end{align*}
\end{frame}
\begin{frame}
\frametitle{Resolviendo la EDO}
Notemos que
\begin{align*}
\alpha = - \beta - m^{2} = m^{2} - n^{2}
\end{align*}
\pause
Por lo que la ecuación
\begin{align*}
\stilde{Z} (z) - \alpha \, Z(z) = 0, \hspace{0.5cm} Z(\pi) = 0
\end{align*}
pasa a ser:
\end{frame}
\begin{frame}
\frametitle{Resolviendo la EDO}
\begin{align*}
\stilde{Z} (z) - (m^{2} + n^{2}) \, Z(z) = 0, \hspace{0.5cm} Z(\pi) = 0
\end{align*}
\pause
Entonces la solución (que también puedes corroborar) para esta ecuación es:
\begin{align*}
Z_{mn} (z) = \sinh \sqrt{m^{2} + n^{2}} (\pi - z)
\end{align*}
\end{frame}
\begin{frame}
\frametitle{Solución general}
La solución general al problema de la ec. de Laplace en un cubo es:
\begin{align*}
u(x, y, z) &= \sum_{m=1}^{\infty} \sum_{n=1}^{\infty} a_{mn} \sin (n \, x) \, \sin (m \, y) \times \\[0.5em]
&\times \sinh \sqrt{m^{2} + n^{2}}(\pi - z)
\end{align*}
\end{frame}
\subsection{Solución con las CI}
\begin{frame}
\frametitle{Calculando las constantes $a_{mn}$}
Para determinar las constantes $a_{mn}$, ocupamos la CI:
\begin{align*}
u(x, y, 0) = f(x, y)
\end{align*}
Entonces tenemos que:
\end{frame}
\begin{frame}
\frametitle{Solución con las CI}
Así se tiene:
\begin{eqnarray*}
u(x, y, 0) &= \displaystyle \sum_{m=1}^{\infty} \sum_{n=1}^{\infty} a_{mn} \sin (n \, x) \, \sin (m \, y) \times \\[0.5em]
&\times \sinh \sqrt{m^{2} + n^{2}}(\pi - 0) = \\[0.5em] \pause
&= \displaystyle \sum_{m=1}^{\infty} \sum_{n=1}^{\infty} a_{mn} \sin (n \, x) \, \sin (m \, y) \times \\[0.5em]
&\times \sinh \pi \, \sqrt{m^{2} + n^{2}} 
\end{eqnarray*}
\end{frame}
\begin{frame}
\frametitle{Solución con las CI}
Hagamos
\begin{align*}
c_{mn} = a_{mn} \, \sinh \pi \, \sqrt{m^{2} + n^{2}}
\end{align*}
\pause
Entonces se tendrá:
\begin{align*}
f(x, y) = u(x, y, 0) = \sum_{m=1}^{\infty} \sum_{n=1}^{\infty} c_{mn} \sin (n \, x) \, \sin (m \, y)
\end{align*}
\end{frame}
\begin{frame}
\frametitle{Condición inicial}
Para satisfacer la condición de frontera $u(x, y, 0) = f(x, y)$, debemos elegir
\begin{align*}
c_{mn} = \dfrac{4}{\pi^{2}} \int_{0}^{\pi} \left( \int_{0}^{\pi} \sin (n \, x) \dd{x} \right) \, \sin (m \, y) \dd{y}
\end{align*}
Entonces:
\end{frame}
\begin{frame}
\frametitle{Los coeficientes $c_{mn}$}
Entonces:
\begin{eqnarray*}
a_{mn} &=& \dfrac{c_{mn}}{\sinh \pi \, \sqrt{m^{2} + n^{2}}} = \\[1em] \pause
&=& \dfrac{\displaystyle \dfrac{4}{\pi^{2}} \int_{0}^{\pi} \left( \int_{0}^{\pi} \sin (n \, x) \dd{x} \right) \, \sin (m \, y) \dd{y}}{\sinh \pi \, \sqrt{m^{2} + n^{2}}}
\end{eqnarray*}
\end{frame}
\begin{frame}
\frametitle{Solución al problema}
Entonces la solución general al problema de la ec. de Laplace para un cubo, con las CDF y la CI dadas en el enunciado, es:
\pause
\begin{align*}
u(x, y, z) &= \displaystyle \sum_{m=1}^{\infty} \sum_{n=1}^{\infty} a_{mn} \sin (n \, x) \, \sin (m \, y) \times \\[0.5em]
&\times \sinh \sqrt{m^{2} + n^{2}}(\pi - z) =
\end{align*}
\end{frame}
\begin{frame}
\frametitle{Solución al problema}
\fontsize{12}{12}\selectfont
\begin{align*}
u(x, y, z)  &= \sum_{m=1}^{\infty} \sum_{n=1}^{\infty} \dfrac{\displaystyle \dfrac{4}{\pi^{2}} \int_{0}^{\pi} \left( \int_{0}^{\pi} \sin (n \, x) \dd{x} \right) \, \sin (m \, y) \dd{y}}{\sinh \pi \, \sqrt{m^{2} + n^{2}}} \times \\[0.5em]
&\times \sin (n \, x) \, \sin (m \, y) \, \sinh \sqrt{m^{2} + n^{2}}(\pi - z)
\end{align*}
\end{frame}
\end{document}