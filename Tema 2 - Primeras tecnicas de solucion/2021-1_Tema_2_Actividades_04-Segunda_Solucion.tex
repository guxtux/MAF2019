\documentclass[12pt]{article}
\usepackage[left=0.25cm,top=1cm,right=0.25cm,bottom=1cm]{geometry}
\textwidth = 20cm
\hoffset = -1cm
\usepackage[utf8]{inputenc}
\usepackage[spanish,es-tabla]{babel}
\usepackage[autostyle,spanish=mexican]{csquotes}
\usepackage[tbtags]{amsmath}
\usepackage{nccmath}
\usepackage{amsthm}
\usepackage{amssymb}
\usepackage{graphicx}
\usepackage{standalone}
\usepackage[outdir=./]{epstopdf}
\usepackage{siunitx}
\usepackage{physics}
\usepackage{color}
\usepackage{float}
\usepackage{multicol}
%\usepackage{milista}
\usepackage{enumitem}
\usepackage{anyfontsize}
\usepackage{anysize}
\usepackage{enumitem}
\usepackage{capt-of}
\usepackage{bm}
\usepackage{relsize}
\usepackage{placeins}
\usepackage{empheq}
\usepackage{cancel}
\usepackage{wrapfig}
\spanishdecimal{.}
\renewcommand{\baselinestretch}{1.5} 
\renewcommand\labelenumii{\theenumi.{\arabic{enumii}}}
\newcommand{\ptilde}[1]{\ensuremath{{#1}^{\prime}}}
\newcommand{\stilde}[1]{\ensuremath{{#1}^{\prime \prime}}}
\newcommand{\ttilde}[1]{\ensuremath{{#1}^{\prime \prime \prime}}}
\newcommand{\ntilde}[2]{\ensuremath{{#1}^{(#2)}}}


\usepackage{apacite}
\title{Ejercicios opcionales - Segunda solución\\[0.3em]  \large{Tema 2 - Matemáticas Avanzadas de la Física}\vspace{-3ex}}
\author{M. en C. Gustavo Contreras Mayén}
\date{ }
\begin{document}
\vspace{-4cm}
\maketitle
\fontsize{14}{14}\selectfont
Recuerda que en esta semana tendrás habilitado el espacio para respuestas, el próximo día domingo 1 de noviembre se cerrará la recepción a las 12 del día.
\par
Te recomendamos que descargues el pdf y resuelvas cada inciso, cuando ya tengas la respuesta, anótala en la plataforma.
\par
En los siguientes ejercicios (1-4) determina si el par de funciones dadas son linealmente independientes o dependientes.
\begin{enumerate}
\item $f(x) = x^{2} + 5 \, x \hspace{2cm} g(x) = x^{2} - 5 \, x$ 
\item $f(x) = \cos (3 \, x) \hspace{2cm} g(x) = 4 \, \cos x - 3 \, \cos x$
\item $f(x) = e^{\lambda x} \, \cos (\mu \, x) \hspace{1.2cm} g(x) = e^{\lambda x} \, \sin \mu x \hspace{0.4cm} \mu \neq 0 $
\item $f(x) = e^{3 x} \hspace{3.1cm} g(x) = e^{3(x-1)}$
% \item $f(x) = 3 \, x - 5 \hspace{2.3cm} g(x) = 9 \, x - 15$
% \item $f(x) = x \hspace{3.5cm} g(x) = x^{-1}$
\item El Wronskiano de dos funciones es
\begin{align*}
W(x) = x \, \sin^{2} x
\end{align*}
¿Las funciones son linealmente independientes o dependientes?¿Por qué?
\item Calcula al menos los primeros tres términos no nulos en el desarrollo en serie de potencias alrededor de $x=0$ para obtener una solución general de:
\begin{align*}
9 \, x^{2} \, \stilde{y} + 9 \, x^{2} \, \ptilde{y} + 2 \, y = 0
\end{align*}
con $x > 0$. Deberás de obtener la ecuación de índices, así como la primera solución mediante el método de Frobenius, posteriormente, expresar la segunda solución linealmente independiente.
\end{enumerate}
\end{document}