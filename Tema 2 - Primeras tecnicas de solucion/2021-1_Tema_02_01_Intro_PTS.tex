\documentclass[hidelinks,12pt]{article}
\usepackage[left=0.25cm,top=1cm,right=0.25cm,bottom=1cm]{geometry}
%\usepackage[landscape]{geometry}
\textwidth = 20cm
\hoffset = -1cm
\usepackage[utf8]{inputenc}
\usepackage[spanish,es-tabla]{babel}
\usepackage[autostyle,spanish=mexican]{csquotes}
\usepackage[tbtags]{amsmath}
\usepackage{nccmath}
\usepackage{amsthm}
\usepackage{amssymb}
\usepackage{mathrsfs}
\usepackage{graphicx}
\usepackage{subfig}
\usepackage{standalone}
\usepackage[outdir=./Imagenes/]{epstopdf}
\usepackage{siunitx}
\usepackage{physics}
\usepackage{color}
\usepackage{float}
\usepackage{hyperref}
\usepackage{multicol}
%\usepackage{milista}
\usepackage{anyfontsize}
\usepackage{anysize}
%\usepackage{enumerate}
\usepackage[shortlabels]{enumitem}
\usepackage{capt-of}
\usepackage{bm}
\usepackage{relsize}
\usepackage{placeins}
\usepackage{empheq}
\usepackage{cancel}
\usepackage{wrapfig}
\usepackage[flushleft]{threeparttable}
\usepackage{makecell}
\usepackage{fancyhdr}
\usepackage{tikz}
\usepackage{bigints}
\usepackage{scalerel}
\usepackage{pgfplots}
\usepackage{pdflscape}
\pgfplotsset{compat=1.16}
\spanishdecimal{.}
\renewcommand{\baselinestretch}{1.5} 
\renewcommand\labelenumii{\theenumi.{\arabic{enumii}})}
\newcommand{\ptilde}[1]{\ensuremath{{#1}^{\prime}}}
\newcommand{\stilde}[1]{\ensuremath{{#1}^{\prime \prime}}}
\newcommand{\ttilde}[1]{\ensuremath{{#1}^{\prime \prime \prime}}}
\newcommand{\ntilde}[2]{\ensuremath{{#1}^{(#2)}}}

\newtheorem{defi}{{\it Definición}}[section]
\newtheorem{teo}{{\it Teorema}}[section]
\newtheorem{ejemplo}{{\it Ejemplo}}[section]
\newtheorem{propiedad}{{\it Propiedad}}[section]
\newtheorem{lema}{{\it Lema}}[section]
\newtheorem{cor}{Corolario}
\newtheorem{ejer}{Ejercicio}[section]

\newlist{milista}{enumerate}{2}
\setlist[milista,1]{label=\arabic*)}
\setlist[milista,2]{label=\arabic{milistai}.\arabic*)}
\newlength{\depthofsumsign}
\setlength{\depthofsumsign}{\depthof{$\sum$}}
\newcommand{\nsum}[1][1.4]{% only for \displaystyle
    \mathop{%
        \raisebox
            {-#1\depthofsumsign+1\depthofsumsign}
            {\scalebox
                {#1}
                {$\displaystyle\sum$}%
            }
    }
}
\def\scaleint#1{\vcenter{\hbox{\scaleto[3ex]{\displaystyle\int}{#1}}}}
\def\bs{\mkern-12mu}


\usepackage{apacite}
\title{Tema 2 - Primeras técnicas de solución \\[0.3em]  \large{Matemáticas Avanzadas de la Física}\vspace{-3ex}}
\author{M. en C. Gustavo Contreras Mayén}
\date{ }
\begin{document}
\vspace{-4cm}
\maketitle
\fontsize{14}{14}\selectfont
\tableofcontents
\newpage
\section{Ecuaciones diferenciales parciales.}
\subsection{Introducción.}
La mayoría de los fenómenos físicos, ya sea en el dominio de la dinámica de fluidos, la electricidad, el magnetismo, la mecánica clásica o cuántica, la óptica o el flujo de calor, pueden describirse en general mediante \emph{ecuaciones diferenciales parciales} (EDP).
\par
Encontraremos que la mayoría de la física matemática son EDP. Es cierto que se pueden hacer simplificaciones que reduzcan las ecuaciones en cuestión a ecuaciones diferenciales ordinarias, sin embargo, la descripción completa de estos sistemas reside en el área general de las EDP.
\subsection*{Definición.}
Una ecuación diferencial parcial es una ecuación que contiene derivadas parciales. En contraste con las ecuaciones diferenciales ordinarias (EDO), donde la función desconocida depende solo de una variable, en las EDP, la función desconocida depende de varias variables (como la temperatura $u (x, t)$ depende tanto de la posición $x$ como del tiempo $t$).
\par
Para simplificar la escritura, haremos uso de la siguiente notación:
\begin{align*}
u_{t} = \pdv{u}{t} \hspace{1cm} u_{x} = \pdv{u}{x} \hspace{1cm} u_{xx} = \pdv[2]{u}{x} \hspace{1cm} u_{xy} = \pdv[2]{u}{x}{y}
\end{align*}
Enumeremos algunas EDP conocidas:
\begin{itemize}
\item $u_{t} = u_{xx}$ \hspace{3.5cm} Ecuación de calor en una dimensión.
\item $u_{t} = u_{xx} + u_{yy}$ \hspace{2.3cm} Ecuación de calor en dos dimensiones.
\item $u_{rr} + \dfrac{1}{r} \, u_{r} + \dfrac{1}{r^{2}} \, u_{\theta \theta}$ \hspace{2cm} Ecuación de Laplace en coordenadas polares.
\item $u_{tt} = u_{xx} + u_{yy} + u_{zz}$ \hspace{1cm} Ecuación de onda en tres dimensiones.
\item $u_{tt} = u_{xx} + \alpha \, u_{t} + \beta \, u$ \hspace{0.9cm} Ecuación del telégrafo.
\end{itemize}
Veamos de los ejemplos anteriores que la función desconocida $u$ siempre depende de más de una variable. La variable $u$ (que diferenciamos) se llama \textbf{variable dependiente}, mientras que aquellas con respecto a las que diferenciamos se llaman \textbf{variables independientes}. Por ejemplo, de la ecuación
\begin{align*}
u_{t} = u_{xx}
\end{align*}
la variable dependiente $u(x, t)$ es una función de dos variables independientes $x$ y $t$; mientras que para la ecuación
\begin{align*}
u_{rr} + \dfrac{1}{r} \, u_{r} + \dfrac{1}{r^{2}} \, u_{\theta \theta}
\end{align*}
se tiene que $u(r, \theta, t)$ depende de las variables $r$, $\theta$ y $t$.
\end{document}