\input{../Preambulos/preambulo_presentacion_CambridgeUS_beaver}
\title{\large{Ejercicios con Separación de variables}}
\subtitle{Tema 2 - Primeras técnicas de solución}
\author{M. en C. Gustavo Contreras Mayén}
\date{}
\institute{Facultad de Ciencias - UNAM}
\titlegraphic{\includegraphics[width=1.75cm]{../Imagenes/escudo-facultad-ciencias}\hspace*{4.75cm}~%
   \includegraphics[width=1.75cm]{../Imagenes/escudo-unam}
}
\setbeamertemplate{navigation symbols}{}
\begin{document}
\maketitle
\fontsize{14}{14}\selectfont
\spanishdecimal{.}
\section*{Contenido}
\frame{\tableofcontents[currentsection, hideallsubsections]}
% \section{Demostrando una separación}
% \frame{\tableofcontents[currentsection, hideothersubsections]}
% \subsection{Separación de variables}
% \begin{frame}
% \frametitle{Base de la separación de variables}
% Hemos revisado que cuando se nos plantea un problema de la física, debemos de elegir un sistema coordenado de modo que se adapte al problema, con la finalidad de aprovechar la simetría del mismo.
% \end{frame}
% \begin{frame}
% \frametitle{Base de la separación de variables}
% De tal manera que el problema pudiera ser más fácil de resolver que cuando lo intentamos bajo una geometría cartesiana.
% \end{frame}
% \begin{frame}
% \frametitle{Base de la separación de variables}
% El que mencionemos que pueda ser más fácil de resolver, implica pasar de una EDP a un sistema de EDO más sencillas y manejables.
% \end{frame}
% \subsection{Ecuación relevante}
% \begin{frame}
% \frametitle{Ecuación importante}
% Consideremos la siguiente ecuación:
% \begin{align}
% \laplacian{\psi} + k^{2} \, \psi = 0
% \label{eq:ecuacion_A_02_01}
% \end{align}
% \pause
% Esta ecuación es más general de lo que en apariencia se nota, como veremos a continuación:
% \end{frame}
% \begin{frame}
% \frametitle{Ecuación importante}
% De la ec. (\ref{eq:ecuacion_A_02_01}) si cambiamos el valor de $k$, obtenemos:
% \setbeamercolor{item projected}{bg=blue!70!black,fg=yellow}
% \setbeamertemplate{enumerate items}[circle]
% \begin{enumerate}[<+->]
% \item Si $k^{2} = 0$ \hspace{1.5cm} Ec. de Laplace.
% \item Si $k^{2} = (+)$ constante \hspace{1cm} Ec. de Helmholtz.
% \item Si $k^{2} = (-)$ constante \hspace{1cm} Ec. de difusión (parte espacial)
% \item Si $k^{2} =$ (constante * energía cinética) \hspace{1cm} Ec. de Schrödinger.
% \end{enumerate} 
% \end{frame}
% \begin{frame}
% \frametitle{Once sistemas coordenados}
% Se ha demostrado que existen once sistemas coordenados en donde la ec. (\ref{eq:ecuacion_A_02_01}), siendo la ec. de Helmholtz, ésta puede separarse.
% \\
% \bigskip
% \pause
% En este ejercicio ocuparemos uno de esos sistemas coordenados.
% \end{frame}
% \subsection{Sistemas coordenados esferoidales}
% \begin{frame}
% \frametitle{titulo}
% Los sistemas de coordenadas esferoidales están definidos por hiperboloides y elipsoides de revolución confocal.
% \\
% \bigskip
% \pause
% Si la revolución es alrededor del eje mayor, el sistema de coordenadas se llama \emph{esferoidal prolato} (alargado), y si la revolución se realiza alrededor del eje menor, se llama \emph{esferoidal oblato}.
% \end{frame}
\section{Problema más elaborado}
\frame{\tableofcontents[currentsection, hideothersubsections]}
\subsection{Planteamiento}
%Ref. Zamora - Notas ecuaciones diferenciales parciales. 2.1.7
\begin{frame}
\frametitle{Planteamiento del problema}
Consideremos el problema de conducción de calor en tres dimensiones en un bloque sólido, es decir, tomemos la ecuación de calor:
\begin{align}
u_{t} = \alpha^{2} \, \laplacian{u}
\label{eq:ecuacion_02_214}
\end{align}
\begin{figure}
    \centering
    \includestandalone[scale=1.3]{Figuras/Figura_Cubo}
    \caption{Cubo sólido para el problema}
\end{figure}
\end{frame}
\begin{frame}
\frametitle{Condiciones de frontera}
Con las condiciones de frontera
\\
\begin{minipage}{0.4\textwidth}
\begin{figure}
    \centering
    \includestandalone[scale=1.3]{Figuras/Figura_Cubo}
\end{figure}
\end{minipage}
\hspace{1cm}
\begin{minipage}{0.4\textwidth}
\begin{align}
u(0, y, z, t) &= 0 \label{eq:ecuacion_0215} \\ 
u(a, y, z, t) &= 0 \label{eq:ecuacion_0216} \\ 
u(x, 0, z, t) &= 0 \label{eq:ecuacion_0217} \\
u(x, b, z, t) &= 0 \label{eq:ecuacion_0218} \\
u(x, y, 0, t) &= 0 \label{eq:ecuacion_0219} \\
u(x, y, c, t) &= 0 \label{eq:ecuacion_0220}
\end{align}
\end{minipage}
\end{frame}
\begin{frame}
\frametitle{Condición inicial}
La condición inicial está dada por la expresión:
\begin{align}
u(x, y, z, 0) = f(x, y, z)
\label{eq:ecuacion_02_221}
\end{align}
definidas para 
\begin{align*}
0 < &x < a \\[0.5em]
0 < &y < b \\[0.5em]
0 < &z < c \\[0.5em]
&t > 0
\end{align*}
\end{frame}
\begin{frame}
\frametitle{Laplaciano en tres dimensiones}
El operador diferencial Laplaciano en tres dimensiones está definido por:
\begin{align}
\laplacian{u} = u_{xx} +u_{yy} + u_{zz}
\end{align}
El Laplaciano actúa solamente en las coordenadas espaciales.
\end{frame}
\begin{frame}
\frametitle{El problema}
Como está definido, este problema puede interpretarse como la difusión de calor a través de las caras del bloque $[0, a] \times [0, b] \times [0,c]$ desde su interior.
\end{frame}
\begin{frame}
\frametitle{El problema}
Las condiciones de frontera (\ref{eq:ecuacion_0215}) - (\ref{eq:ecuacion_0220}) para este problema son de temperatura constante cero en cada una de las caras del bloque.
\end{frame}
\begin{frame}
\frametitle{El problema}
Esto es, para este caso no se tiene ninguna cara aislada térmicamente.
\\
\bigskip
Si se quisiera aislar la cara $x_{i}$ = constante, habría que imponer la condición $u_{x_{i}} = 0$ en dicha cara.
\end{frame}
\begin{frame}
\frametitle{El método de separación de variables}
Usamos la propuesta del método de separación de variables:
\begin{align}
u(x, y, z, t) = X(x) \, Y(y) \, Z(z) \, T(t)
\label{eq:ecuacion_02_223}
\end{align}
\pause
En donde realizamos el procedimiento de sustitución en la EDP de calor, dividir por el producto de funciones que dependen de una sola variable.
\end{frame}
\begin{frame}
\frametitle{Sistema de EDO}
Entonces obtenemos un sistema de ecuaciones diferenciales ordinarias:
\begin{align}
\ptilde{T} - A \, T &= 0 \label{eq:ecuacion_02_224} \\[0.5em]
\stilde{X} - C \, X &= 0 \label{eq:ecuacion_02_225} \\[0.5em]
\stilde{Y} - D \, Y &= 0 \label{eq:ecuacion_02_226} \\[0.5em]
\stilde{Z} - E \, Z &= 0 \label{eq:ecuacion_02_227} \end{align}
\end{frame}
\begin{frame}
\frametitle{Constantes de separación}
Donde las constantes $A$, $B$, $C$ y $D$ satisfacen:
\begin{align*}
B &= C + D + E \\[0.5em]
B &= \dfrac{A}{\alpha^{2}}
\end{align*}
\end{frame}
\begin{frame}
\frametitle{Condiciones de frontera}
Las condiciones de frontera que se obtienen son:
\\
\begin{minipage}[t]{0.4\textwidth}
\begin{align}
X(0) &= 0 \label{eq:ecuacion_02_228} \\
X(a) &= 0 \label{eq:ecuacion_02_229} \\
Y(0) &= 0 \label{eq:ecuacion_02_230} \\
Y(b) &= 0 \label{eq:ecuacion_02_231}
\end{align}
\end{minipage}
\hspace{1cm}
\begin{minipage}[t]{0.4\textwidth}
\begin{align}
Z(0) &= 0 \label{eq:ecuacion_02_232} \\
Z(c) &= 0 \label{eq:ecuacion_02_233}
\end{align}
\end{minipage}
\end{frame}
\begin{frame}
\frametitle{Soluciones a las EDO}
Las soluciones de las EDO para $X, Y$ y $Z$, son:
\begin{eqnarray}
X(x) &=& \sin \left( \dfrac{n \, \pi \, x}{a} \right) \label{eq:ecuacion_02_234} \\[0.5em] \pause
Y(y) &=& \sin \left( \dfrac{m \, \pi \, y}{b} \right) \label{eq:ecuacion_02_235} \\[0.5em] \pause
Z(z) &=& \sin \left( \dfrac{l \, \pi \, z}{c} \right) \label{eq:ecuacion_02_236}
\end{eqnarray}
para toda
\begin{align*}
n, m, l = \pm 1, \pm 2, \ldots
\end{align*}
\end{frame}
\begin{frame}
\frametitle{Constantes de separación}
Esto es, las tres constantes de separación son todas negativas:
\begin{eqnarray*}
C &=& - \dfrac{n^{2} \, \pi^{2}}{a^{2}} \\[0.5em]
D &=& - \dfrac{m^{2} \, \pi^{2}}{b^{2}} \\[0.5em]
E &=& - \dfrac{l^{2} \, \pi^{2}}{c^{2}}
\end{eqnarray*}
\end{frame}
\begin{frame}
\frametitle{Constantes de separación}
Se tiene entonces que:
\begin{align*}
B = - \left( \dfrac{n^{2}}{a^{2}} + \dfrac{m^{2}}{b^{2}} + \dfrac{l^{2}}{c^{2}} \right) \, \pi^{2}
\end{align*}
\pause
Y como $B = A / \alpha^{2}$, ocurre que:
\pause
\begin{align}
A = - \left( \dfrac{n^{2}}{a^{2}} + \dfrac{m^{2}}{b^{2}} + \dfrac{l^{2}}{c^{2}} \right) \, \pi^{2} \, \alpha^{2}
\label{eq:ecuacion_02_237}
\end{align}
\end{frame}
\begin{frame}
\frametitle{Solución a la EDO de $T(t)$}
Ahora ya podemos expresar la solución fundamental de la ec. (\ref{eq:ecuacion_02_224}), siendo:
\begin{align}
T(t) = \exp \left( - \left[ \dfrac{n^{2}}{a^{2}} + \dfrac{m^{2}}{b^{2}} + \dfrac{l^{2}}{c^{2}} \right] \, \pi^{2} \, \alpha^{2}\, t \right)
\label{eq:ecuacion_02_238}
\end{align}
\end{frame}
\begin{frame}
\frametitle{Soluciones al problema}
Ahora podemos presentar el conjunto de soluciones para el problema de la ecuación de calor en un bloque cúbico:
\end{frame}
\begin{frame}
\frametitle{Soluciones al problema}
\begin{align}
\begin{aligned}
u_{nml}(x, y, z, t) &= \sin \left(\dfrac{n \, \pi \, x}{a} \right) \, \sin \left( \dfrac{m \, \pi \, y}{b} \right) \, \sin \left( \dfrac{l \, \pi \, z}{c} \right) \times \\[1em]
&\times \exp \left( - \left[ \dfrac{n^{2}}{a^{2}} + \dfrac{m^{2}}{b^{2}} + \dfrac{l^{2}}{c^{2}} \right] \, \pi^{2} \, \alpha^{2}\, t \right)
\end{aligned}
\label{eq:ecuacion_02_239}
\end{align}
\end{frame}
\begin{frame}
\frametitle{Principio de superposición}
Hacemos uso del principio de superposición: en donde, tomando la combinación lineal más general e intercambiando la suma sobre los enteros negativos por una sobre positivos, se encuentra lo siguiente:
\end{frame}
\begin{frame}
\frametitle{Solución al problema}
Tendremos que:
\fontsize{12}{12}\selectfont
\begin{align}
\begin{aligned}
u(x, y, z, t) &= \sum_{n=1}^{\infty} \sum_{m=1}^{\infty} \sum_{l=1}^{\infty} b_{nml} \, u_{nml}(x, y, z, t) \\[1em]
&= \sum_{n,m,l} b_{nml} \, \sin \left(\dfrac{n \, \pi \, x}{a} \right) \, \sin \left( \dfrac{m \, \pi \, y}{b} \right) \, \sin \left( \dfrac{l \, \pi \, z}{c} \right) \times \\[1em]
&\times \exp \left( - \left[ \dfrac{n^{2}}{a^{2}} + \dfrac{m^{2}}{b^{2}} + \dfrac{l^{2}}{c^{2}} \right] \, \pi^{2} \, \alpha^{2}\, t \right)
\end{aligned}
\label{eq:ecuacion_02_241}
\end{align}
\end{frame}
\begin{frame}
\frametitle{Condición inicial}
Los coeficientes $b_{nml}$ quedan determinados por la condición inicial (\ref{eq:ecuacion_02_221}), por lo tanto:
\begin{align}
\begin{aligned}
f(x, y, z) &= \sum_{p=1}^{\infty} \sum_{q=1}^{\infty} \sum_{r=1}^{\infty} b_{pqr} \sin \left( \dfrac{p \, \pi \, x}{a} \right) \, \sin \left( \dfrac{q \, \pi \, y}{b} \right) \times \\[1em]
&\times \sin \left( \dfrac{r \, \pi \, z}{c} \right)
\end{aligned}
\label{eq:ecuacion_02_242}
\end{align}
\end{frame}
\begin{frame}
\frametitle{Calculando los coeficientes}
Lo que tenemos que hacer ahora es:
\setbeamercolor{item projected}{bg=blue!70!black,fg=yellow}
\setbeamertemplate{enumerate items}[circle]
\begin{enumerate}[<+->]
\item Multiplicar la función $f(x,y,z)$ por las funciones seno.
\item Integrar sobre la región
\begin{align*}
[0, a] \times [0, b] \times [0, c]
\end{align*}
\item Calcular las sumas.
\end{enumerate}
\end{frame}
\begin{frame}
\frametitle{Calculando los coeficientes}
Tenemos entonces que:
% \fontsize{12}{12}\selectfont
\begin{align}
\begin{aligned}
b_{nml} &= \dfrac{8}{abc} \int_{0}^{a} \dd{x} \int_{0}^{b} \dd{y} \int_{0}^{c} \dd{z} f(x, y, z) \times \\[1em]
&\times \sin \left( \dfrac{n \, \pi \, x}{a} \right) \, \sin \left( \dfrac{m \, \pi \, y}{b} \right) \, \sin \left( \dfrac{l \, \pi \, z}{c} \right)
\end{aligned}
\label{eq:ecuacion_02_243}
\end{align}
\end{frame}
\begin{frame}
\frametitle{Solución completa}
De esta forma, la solución completa a este problema está dada por la función $u(x, y, z, t)$ de la ec. (\ref{eq:ecuacion_02_241}) con los coeficientes $b_{nml}$ dados por la ec. (\ref{eq:ecuacion_02_243}), válida para toda 
\begin{align}
(x,y,z) \in [0,a] \times [0,b] \times [0,c]
\end{align} 
y para todo $t \geq 0$.
\end{frame}
\section{Ec. de calor en una esfera}
\frame{\tableofcontents[currentsection, hideothersubsections]}
\subsection{Planteamiento}
\begin{frame}
\frametitle{Problema de estudio}
Ahora abordaremos la ecuación de calor en una simetría esférica, en donde también se estudiará junto con las tres coordenadas espaciales, la evolución temporal.
\end{frame}
\begin{frame}
\frametitle{El Laplaciano}
En coordenadas esféricas $(r, \theta, \phi)$ el operador Laplaciano es:
\begin{align}
\begin{aligned}
\laplacian{u} &= \dfrac{1}{r^{2} \, \sin \theta} \, \bigg[ \sin \theta \left( r^{2} \, u_{r} \right)_{r} + \left( \sin \theta \, u_{\theta} \right)_{\theta} + \\[1em]
&+ \dfrac{1}{\sin \theta} \, u_{\phi \phi} \bigg] 
\end{aligned}
\label{eq:ecuacion_02_286}
\end{align}
\end{frame}
\begin{frame}
\frametitle{Ec. de calor en coord. esféricas}
La ecuación de calor escrita en coordenadas esféricas es:
\begin{align}
\begin{aligned}
u_{t} &= \dfrac{\alpha^{2}}{r^{2} \, \sin \theta} \, \bigg[ \sin \theta \left( r^{2} \, u_{r} \right)_{r} + \left( \sin \theta \, u_{\theta} \right)_{\theta} + \\[1em]
&+ \dfrac{1}{\sin \theta} \, u_{\phi \phi} \bigg] 
\end{aligned}
\label{eq:ecuacion_02_287}
\end{align}
\end{frame}
\subsection{Separación de variables}
\begin{frame}
\frametitle{Método de separación de variables}
La propuesta de separación de variables es:
\begin{align}
u(r, \theta, \phi, t) = R(r) \, \Theta (\theta) \, \Phi(\phi) \, T(t)
\label{eq:ecuacion_02_288}
\end{align}
\end{frame}
\begin{frame}
\frametitle{Aplicando el método}
Tenemos entonces que:
\begin{align}
u_{t} = R \, \Theta \, \Phi \, \ptilde{T} \label{eq:ecuacion_02_289}
\end{align}
\begin{align}
\begin{aligned}
\laplacian{u} &= \dfrac{\Theta \, \Phi \, T}{r^{2}} \left( r^{2} \, \ptilde{T} \right)^{\prime} + \dfrac{R \, \, \Phi \, T}{r^{2} \, \sin \theta} \left( \sin \theta \, \ptilde{\Theta} \right)^{\prime} + \\[1em]
&+ \dfrac{R \, \, \Theta \, T}{r^{2} \, \sin^{2} \theta} \, \stilde{\Phi}
\end{aligned}
\label{eq:ecuacion_02_290}
\end{align}
\end{frame}
\begin{frame}
\frametitle{Aplicando el método}
De tal modo que al sustituir la ecuación de calor (\ref{eq:ecuacion_02_287}) y dividir ambos lados por la solución propuesta, ec. (\ref{eq:ecuacion_02_288}) nos lleva a la primera separación:
\end{frame}
\begin{frame}
\frametitle{Primera separación}
Primera separación
\begin{align}
\begin{aligned}
\dfrac{\ptilde{T}}{T} &= \alpha^{2} \bigg[ \dfrac{1}{r^{2} \, R} \, \left( r^{2} \, \ptilde{R} \right)^{\prime} + \dfrac{1}{r^{2} \, \sin \theta \, \Theta} \, \left( \sin \theta \, \ptilde{\Theta} \right)^{\prime} + \\[1em]
&+ \dfrac{1}{r^{2} \, \sin \theta \, \Phi} \, \stilde{\Phi} \bigg]
\end{aligned}
\label{eq:}
\end{align}
\end{frame}
\begin{frame}
\frametitle{Primer sistema de EDO}
Que nos lleva a las siguientes ecuaciones:
\begin{align}
\ptilde{T} - A \, T = 0
\label{eq:ecuacion_02_292}
\end{align}
\begin{align}
\begin{aligned}
\dfrac{1}{r^{2} \, R} \, \left( r^{2} \, \ptilde{R} \right)^{\prime} &+ \dfrac{1}{r^{2} \, \sin \theta \, \Theta} \, \left( \sin \theta \, \ptilde{\Theta} \right)^{\prime} + \\[1em]
&+ \dfrac{1}{r^{2} \, \sin \theta \, \Phi} \, \stilde{\Phi} \bigg] = B
\end{aligned}
\label{eq:ecuacion_02_293}
\end{align}
con $B = A / \alpha^{2}$.
\end{frame}
\begin{frame}
\frametitle{Segunda separación}
Multiplicando la ec. (\ref{eq:ecuacion_02_293}) por $r^{2}$ y al acomodar los términos, obtenemos la segunda separación:
\begin{align}
\begin{aligned}
\dfrac{1}{R} \, \left( r^{2} \, \ptilde{R} \right)^{\prime} - B \, r^{2} &=  - \dfrac{1}{\sin \theta \, \Theta} \, \left( \sin \theta \, \ptilde{\Theta} \right)^{\prime} + \\[1em]
&- \dfrac{1}{\sin^{2} \theta \, \Phi} \, \stilde{\Phi} = \\[1em]
&= C
\end{aligned}
\label{eq:ecuacion_02_294}
\end{align}
\end{frame}
\begin{frame}
\frametitle{Segunda separación}
Es decir:
\begin{align}
\dfrac{1}{R} \, \left( r^{2} \, \ptilde{R} \right)^{\prime} - B \, r^{2} - C = 0
\label{eq:ecuacion_02_295}
\end{align}
\begin{align}
\dfrac{1}{\sin \theta \, \Theta} \, \left( \sin \theta \, \ptilde{\Theta} \right)^{\prime} + \dfrac{1}{\sin^{2} \theta \, \Phi} \, \stilde{\Phi} = - C
\label{eq:ecuacion_02_296}
\end{align}
\end{frame}
\begin{frame}
\frametitle{Tercera separación}
Multiplicando por $\sin^{2} \theta$ la ec. (\ref{eq:ecuacion_02_296}), para luego reagrupar los términos, tendremos la separación final:
\begin{align}
\dfrac{\stilde{\Phi}}{\Phi} = - \dfrac{\sin \theta}{\Theta} \, \left( \sin \theta \, \ptilde{\Theta} \right)^{\prime} - C \, \sin^{2} \theta = D
\label{eq:ecuacion_02_297}
\end{align}
es decir:
\end{frame}
\begin{frame}
\frametitle{Tercera separación}
Tendremos entonces:
\begin{align}
\stilde{\Phi} - D \, \Phi = 0
\label{eq:ecuacion_02_298}
\end{align}
\begin{align}
\dfrac{\sin \theta}{\Theta} \, \left( \sin \theta \, \ptilde{\Theta} \right)^{\prime} + C \, \sin^{2} \theta + D = 0
\label{eq:ecuacion_02_299}
\end{align}
\end{frame}
\subsection{Sistema de 4 EDO}
\begin{frame}
\frametitle{Sistema de 4 EDO}
Después de haber calculado las derivadas y simplificar las EDO, llegamos al siguiente sistema:
\fontsize{12}{12}\selectfont
\begin{eqnarray}
\ptilde{T} - A \, T &=&0 \label{eq:ecuacion_02_300} \\[0.5em] \pause
r^{2} \, \stilde{R} + 2 \, r \, \ptilde{R} - \left( B \, r^{2} + C \right) \, R &=& 0 \label{eq:ecuacion_02_301} \\[0.5em] \pause
\stilde{\Phi} - D \, \Phi &=& 0 \label{eq:ecuacion_02_302} \\[0.5em] \pause
\sin^{2} \theta \, \stilde{\Theta} + \sin \theta \, \cos \theta \, \ptilde{\Theta} + \left( C \, \sin^{2} \theta + D \right) \, \Theta &=& 0 \label{eq:ecuacion_02_303}
\end{eqnarray}
\end{frame}
\subsection*{Constantes de separación}
\begin{frame}
\frametitle{Las constantes de separación}
Donde las constantes de separación son $A, B, C$ y $D$, que en el caso más general pueden ser valores complejos.
\end{frame}
\subsection{Ecuaciones más complicadas}
\begin{frame}
\frametitle{Primera ecuación \enquote{especial}}
La ecuación para $R$ puede reducirse a una ecuación de tipo Bessel de la forma:
\begin{align}
x^{2} \, \stilde{y} + x \, \ptilde{y} + \left[ x^{2} - \left( n + \dfrac{1}{2} \right) \right] \, y = 0
\label{eq:ecuacion_02_304}
\end{align}
donde $y = y(x)$
\end{frame}
\begin{frame}
\frametitle{Valores para adoptar la forma}
Si hacemos que:
\begin{align*}
B &= -k^{2} \\[0.5em]
C &= n(n + 1) \\[0.5em]
k \, r &= x \\[0.5em]
R &= \dfrac{y}{\sqrt{x}}
\end{align*}
\end{frame}
\begin{frame}
\frametitle{Segunda ecuación \enquote{especial}}
Considerando la ecuación para $\Theta$, también es posible llevarla a una forma de la ecuación asociada de Legendre:
\begin{align*}
(1 - x^{2}) \, \stilde{z} - 2 \, x \, \ptilde{z} + \left( \lambda - \dfrac{m^{2}}{1 -x^{2}} \right) \, z = 0
\label{eq:ecuacion_02_305}
\end{align*}
con $z = z(x)$
\end{frame}
\begin{frame}
\frametitle{Valores para adoptar la forma}
Si hacemos que
\begin{align*}
C &= \lambda \\[0.5em]
D &= - m^{2} \\[0.5em]
\cos \theta &= x \\[0.5em]
\Theta &= z
\end{align*}
\end{frame}
\begin{frame}
\frametitle{Resultados}
Hemos logrado separar la ecuación de calor en una geometría esférica, ocupando el método de separación de variables.
\end{frame}
\begin{frame}
\frametitle{Resultados}
Encontramos, sin embargo, que el sistema de EDO es considerablemente más complicado que en el caso ya sea de un sistema de coordenadas cartesianas y de  coordenadas cilíındricas.
\end{frame}
\begin{frame}
\frametitle{Solución mediante serie de potencias}
Veremos como una segunda técnica de solución, el obtener mediante una serie de potencias, la solución a una ecuación diferencial.
\\
\bigskip
\pause
A ese método le llamaremos \emph{método de Frobenius}.
\end{frame}
\end{document}