\input{../Preambulos/preambulo_presentacion_CambridgeUS_beaver}
\title{\large{Ejercicio con Separación de variables}}
\subtitle{Tema 2 - Primeras técnicas de solución}
\author{M. en C. Gustavo Contreras Mayén}
\date{}
\institute{Facultad de Ciencias - UNAM}
\titlegraphic{\includegraphics[width=1.75cm]{../Imagenes/escudo-facultad-ciencias}\hspace*{4.75cm}~%
   \includegraphics[width=1.75cm]{../Imagenes/escudo-unam}
}
\setbeamertemplate{navigation symbols}{}
\begin{document}
\maketitle
\fontsize{14}{14}\selectfont
\spanishdecimal{.}
\section*{Contenido}
\frame{\tableofcontents[currentsection, hideallsubsections]}
\section{Problema más elaborado}
\frame{\tableofcontents[currentsection, hideothersubsections]}
\subsection{Planteamiento}
%Ref. Zamora - Notas ecuaciones diferenciales parciales. 2.1.7
\begin{frame}
\frametitle{Planteamiento del problema}
Consideremos el problema de conducción de calor en tres dimensiones en un bloque sólido, es decir, tomemos la ecuación de calor:
\begin{align}
u_{t} = \alpha^{2} \, \laplacian{u}
\label{eq:ecuacion_02_214}
\end{align}
\begin{figure}
    \centering
    \includestandalone[scale=1.3]{Figuras/Figura_Cubo}
    \caption{Cubo sólido para el problema}
\end{figure}
\end{frame}
\begin{frame}
\frametitle{Condiciones de frontera}
Con las condiciones de frontera
\\
\begin{minipage}{0.4\textwidth}
\begin{figure}
    \centering
    \includestandalone[scale=1.3]{Figuras/Figura_Cubo}
\end{figure}
\end{minipage}
\hspace{1cm}
\begin{minipage}{0.4\textwidth}
\begin{align}
u(0, y, z, t) &= 0 \label{eq:ecuacion_0215} \\ 
u(a, y, z, t) &= 0 \label{eq:ecuacion_0216} \\ 
u(x, 0, z, t) &= 0 \label{eq:ecuacion_0217} \\
u(x, b, z, t) &= 0 \label{eq:ecuacion_0218} \\
u(x, y, 0, t) &= 0 \label{eq:ecuacion_0219} \\
u(x, y, c, t) &= 0 \label{eq:ecuacion_0220}
\end{align}
\end{minipage}
\end{frame}
\begin{frame}
\frametitle{Condición inicial}
La condición inicial está dada por la expresión:
\begin{align}
u(x, y, z, 0) = f(x, y, z)
\label{eq:ecuacion_02_221}
\end{align}
definidas para 
\begin{align*}
0 < &x < a \\[0.5em]
0 < &y < b \\[0.5em]
0 < &z < c \\[0.5em]
&t > 0
\end{align*}
\end{frame}
\begin{frame}
\frametitle{Laplaciano en tres dimensiones}
El operador diferencial Laplaciano en tres dimensiones está definido por:
\begin{align}
\laplacian{u} = u_{xx} +u_{yy} + u_{zz}
\end{align}
El Laplaciano actúa solamente en las coordenadas espaciales.
\end{frame}
\begin{frame}
\frametitle{El problema}
Como está definido, este problema puede interpretarse como la difusión de calor a través de las caras del bloque $[0, a] \times [0, b] \times [0,c]$ desde su interior.
\end{frame}
\begin{frame}
\frametitle{El problema}
Las condiciones de frontera (\ref{eq:ecuacion_0215}) - (\ref{eq:ecuacion_0220}) para este problema son de temperatura constante cero en cada una de las caras del bloque.
\end{frame}
\begin{frame}
\frametitle{El problema}
Esto es, para este caso no se tiene ninguna cara aislada térmicamente.
\\
\bigskip
Si se quisiera aislar la cara $x_{i}$ = constante, habría que imponer la condición $u_{x_{i}} = 0$ en dicha cara.
\end{frame}
\begin{frame}
\frametitle{El método de separación de variables}
Usamos la propuesta del método de separación de variables:
\begin{align}
u(x, y, z, t) = X(x) \, Y(y) \, Z(z) \, T(t)
\label{eq:ecuacion_02_223}
\end{align}
\pause
En donde realizamos el procedimiento de sustitución en la EDP de calor, dividir por el producto de funciones que dependen de una sola variable.
\end{frame}
\begin{frame}
\frametitle{Sistema de EDO}
Entonces obtenemos un sistema de ecuaciones diferenciales ordinarias:
\begin{align}
\ptilde{T} - A \, T &= 0 \label{eq:ecuacion_02_224} \\[0.5em]
\stilde{X} - C \, X &= 0 \label{eq:ecuacion_02_225} \\[0.5em]
\stilde{Y} - D \, Y &= 0 \label{eq:ecuacion_02_226} \\[0.5em]
\stilde{Z} - E \, Z &= 0 \label{eq:ecuacion_02_227} \end{align}
\end{frame}
\begin{frame}
\frametitle{Constantes de separación}
Donde las constantes $A$, $B$, $C$ y $D$ satisfacen:
\begin{align*}
B &= C + D + E \\[0.5em]
B &= \dfrac{A}{\alpha^{2}}
\end{align*}
\end{frame}
\begin{frame}
\frametitle{Condiciones de frontera}
Las condiciones de frontera que se obtienen son:
\\
\begin{minipage}[t]{0.4\textwidth}
\begin{align}
X(0) &= 0 \label{eq:ecuacion_02_228} \\
X(a) &= 0 \label{eq:ecuacion_02_229} \\
Y(0) &= 0 \label{eq:ecuacion_02_230} \\
Y(b) &= 0 \label{eq:ecuacion_02_231}
\end{align}
\end{minipage}
\hspace{1cm}
\begin{minipage}[t]{0.4\textwidth}
\begin{align}
Z(0) &= 0 \label{eq:ecuacion_02_232} \\
Z(c) &= 0 \label{eq:ecuacion_02_233}
\end{align}
\end{minipage}
\end{frame}
\begin{frame}
\frametitle{Soluciones a las EDO}
Las soluciones de las EDO para $X, Y$ y $Z$, son:
\begin{eqnarray}
X(x) &=& \sin \left( \dfrac{n \, \pi \, x}{a} \right) \label{eq:ecuacion_02_234} \\[0.5em] \pause
Y(y) &=& \sin \left( \dfrac{m \, \pi \, y}{b} \right) \label{eq:ecuacion_02_235} \\[0.5em] \pause
Z(z) &=& \sin \left( \dfrac{l \, \pi \, z}{c} \right) \label{eq:ecuacion_02_236}
\end{eqnarray}
para toda
\begin{align*}
n, m, l = \pm 1, \pm 2, \ldots
\end{align*}
\end{frame}
\begin{frame}
\frametitle{Constantes de separación}
Esto es, las tres constantes de separación son todas negativas:
\begin{eqnarray*}
C &=& - \dfrac{n^{2} \, \pi^{2}}{a^{2}} \\[0.5em]
D &=& - \dfrac{m^{2} \, \pi^{2}}{b^{2}} \\[0.5em]
E &=& - \dfrac{l^{2} \, \pi^{2}}{c^{2}}
\end{eqnarray*}
\end{frame}
\begin{frame}
\frametitle{Constantes de separación}
Se tiene entonces que:
\begin{align*}
B = - \left( \dfrac{n^{2}}{a^{2}} + \dfrac{m^{2}}{b^{2}} + \dfrac{l^{2}}{c^{2}} \right) \, \pi^{2}
\end{align*}
\pause
Y como $B = A / \alpha^{2}$, ocurre que:
\pause
\begin{align}
A = - \left( \dfrac{n^{2}}{a^{2}} + \dfrac{m^{2}}{b^{2}} + \dfrac{l^{2}}{c^{2}} \right) \, \pi^{2} \, \alpha^{2}
\label{eq:ecuacion_02_237}
\end{align}
\end{frame}
\begin{frame}
\frametitle{Solución a la EDO de $T(t)$}
Ahora ya podemos expresar la solución fundamental de la ec. (\ref{eq:ecuacion_02_224}), siendo:
\begin{align}
T(t) = \exp \left( - \left[ \dfrac{n^{2}}{a^{2}} + \dfrac{m^{2}}{b^{2}} + \dfrac{l^{2}}{c^{2}} \right] \, \pi^{2} \, \alpha^{2}\, t \right)
\label{eq:ecuacion_02_238}
\end{align}
\end{frame}
\begin{frame}
\frametitle{Soluciones al problema}
Ahora podemos presentar el conjunto de soluciones para el problema de la ecuación de calor en un bloque cúbico:
\end{frame}
\begin{frame}
\frametitle{Soluciones al problema}
\begin{align}
\begin{aligned}
u_{nml}(x, y, z, t) &= \sin \left(\dfrac{n \, \pi \, x}{a} \right) \, \sin \left( \dfrac{m \, \pi \, y}{b} \right) \, \sin \left( \dfrac{l \, \pi \, z}{c} \right) \times \\[1em]
&\times \exp \left( - \left[ \dfrac{n^{2}}{a^{2}} + \dfrac{m^{2}}{b^{2}} + \dfrac{l^{2}}{c^{2}} \right] \, \pi^{2} \, \alpha^{2}\, t \right)
\end{aligned}
\label{eq:ecuacion_02_239}
\end{align}
\end{frame}
\begin{frame}
\frametitle{Principio de superposición}
Hacemos uso del principio de superposición: en donde, tomando la combinación lineal más general e intercambiando la suma sobre los enteros negativos por una sobre positivos, se encuentra lo siguiente:
\end{frame}
\begin{frame}
\frametitle{Solución al problema}
Tendremos que:
\fontsize{12}{12}\selectfont
\begin{align}
\begin{aligned}
u(x, y, z, t) &= \sum_{n=1}^{\infty} \sum_{m=1}^{\infty} \sum_{l=1}^{\infty} b_{nml} \, u_{nml}(x, y, z, t) \\[1em]
&= \sum_{n,m,l} b_{nml} \, \sin \left(\dfrac{n \, \pi \, x}{a} \right) \, \sin \left( \dfrac{m \, \pi \, y}{b} \right) \, \sin \left( \dfrac{l \, \pi \, z}{c} \right) \times \\[1em]
&\times \exp \left( - \left[ \dfrac{n^{2}}{a^{2}} + \dfrac{m^{2}}{b^{2}} + \dfrac{l^{2}}{c^{2}} \right] \, \pi^{2} \, \alpha^{2}\, t \right)
\end{aligned}
\label{eq:ecuacion_02_241}
\end{align}
\end{frame}
\begin{frame}
\frametitle{Condición inicial}
Los coeficientes $b_{nml}$ quedan determinados por la condición inicial (\ref{eq:ecuacion_02_221}), por lo tanto:
\begin{align}
\begin{aligned}
f(x, y, z) &= \sum_{p=1}^{\infty} \sum_{q=1}^{\infty} \sum_{r=1}^{\infty} b_{pqr} \sin \left( \dfrac{p \, \pi \, x}{a} \right) \, \sin \left( \dfrac{q \, \pi \, y}{b} \right) \times \\[1em]
&\times \sin \left( \dfrac{r \, \pi \, z}{c} \right)
\end{aligned}
\label{eq:ecuacion_02_242}
\end{align}
\end{frame}
\begin{frame}
\frametitle{Calculando los coeficientes}
Lo que tenemos que hacer ahora es:
\setbeamercolor{item projected}{bg=blue!70!black,fg=yellow}
\setbeamertemplate{enumerate items}[circle]
\begin{enumerate}[<+->]
\item Multiplicar la función $f(x,y,z)$ por las funciones seno.
\item Integrar sobre la región
\begin{align*}
[0, a] \times [0, b] \times [0, c]
\end{align*}
\item Calcular las sumas.
\end{enumerate}
\end{frame}
\begin{frame}
\frametitle{Calculando los coeficientes}
Tenemos entonces que:
% \fontsize{12}{12}\selectfont
\begin{align}
\begin{aligned}
b_{nml} &= \dfrac{8}{abc} \int_{0}^{a} \dd{x} \int_{0}^{b} \dd{y} \int_{0}^{c} \dd{z} f(x, y, z) \times \\[1em]
&\times \sin \left( \dfrac{n \, \pi \, x}{a} \right) \, \sin \left( \dfrac{m \, \pi \, y}{b} \right) \, \sin \left( \dfrac{l \, \pi \, z}{c} \right)
\end{aligned}
\label{eq:ecuacion_02_243}
\end{align}
\end{frame}
\begin{frame}
\frametitle{Solución completa}
De esta forma, la solución completa a este problema está dada por la función $u(x, y, z, t)$ de la ec. (\ref{eq:ecuacion_02_241}) con los coeficientes $b_{nml}$ dados por la ec. (\ref{eq:ecuacion_02_243}), válida para toda 
\begin{align}
(x,y,z) \in [0,a] \times [0,b] \times [0,c]
\end{align} 
y para todo $t \geq 0$.
\end{frame}
\end{document}