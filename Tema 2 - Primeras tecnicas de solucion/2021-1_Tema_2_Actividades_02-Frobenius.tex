\documentclass[hidelinks,12pt]{article}
\usepackage[left=0.25cm,top=1cm,right=0.25cm,bottom=1cm]{geometry}
%\usepackage[landscape]{geometry}
\textwidth = 20cm
\hoffset = -1cm
\usepackage[utf8]{inputenc}
\usepackage[spanish,es-tabla]{babel}
\usepackage[autostyle,spanish=mexican]{csquotes}
\usepackage[tbtags]{amsmath}
\usepackage{nccmath}
\usepackage{amsthm}
\usepackage{amssymb}
\usepackage{mathrsfs}
\usepackage{graphicx}
\usepackage{subfig}
\usepackage{standalone}
\usepackage[outdir=./Imagenes/]{epstopdf}
\usepackage{siunitx}
\usepackage{physics}
\usepackage{color}
\usepackage{float}
\usepackage{hyperref}
\usepackage{multicol}
%\usepackage{milista}
\usepackage{anyfontsize}
\usepackage{anysize}
%\usepackage{enumerate}
\usepackage[shortlabels]{enumitem}
\usepackage{capt-of}
\usepackage{bm}
\usepackage{relsize}
\usepackage{placeins}
\usepackage{empheq}
\usepackage{cancel}
\usepackage{wrapfig}
\usepackage[flushleft]{threeparttable}
\usepackage{makecell}
\usepackage{fancyhdr}
\usepackage{tikz}
\usepackage{bigints}
\usepackage{scalerel}
\usepackage{pgfplots}
\usepackage{pdflscape}
\pgfplotsset{compat=1.16}
\spanishdecimal{.}
\renewcommand{\baselinestretch}{1.5} 
\renewcommand\labelenumii{\theenumi.{\arabic{enumii}})}
\newcommand{\ptilde}[1]{\ensuremath{{#1}^{\prime}}}
\newcommand{\stilde}[1]{\ensuremath{{#1}^{\prime \prime}}}
\newcommand{\ttilde}[1]{\ensuremath{{#1}^{\prime \prime \prime}}}
\newcommand{\ntilde}[2]{\ensuremath{{#1}^{(#2)}}}

\newtheorem{defi}{{\it Definición}}[section]
\newtheorem{teo}{{\it Teorema}}[section]
\newtheorem{ejemplo}{{\it Ejemplo}}[section]
\newtheorem{propiedad}{{\it Propiedad}}[section]
\newtheorem{lema}{{\it Lema}}[section]
\newtheorem{cor}{Corolario}
\newtheorem{ejer}{Ejercicio}[section]

\newlist{milista}{enumerate}{2}
\setlist[milista,1]{label=\arabic*)}
\setlist[milista,2]{label=\arabic{milistai}.\arabic*)}
\newlength{\depthofsumsign}
\setlength{\depthofsumsign}{\depthof{$\sum$}}
\newcommand{\nsum}[1][1.4]{% only for \displaystyle
    \mathop{%
        \raisebox
            {-#1\depthofsumsign+1\depthofsumsign}
            {\scalebox
                {#1}
                {$\displaystyle\sum$}%
            }
    }
}
\def\scaleint#1{\vcenter{\hbox{\scaleto[3ex]{\displaystyle\int}{#1}}}}
\def\bs{\mkern-12mu}


\usepackage{apacite}
\title{Ejercicios opcionales - Método de Frobenius\\[0.3em]  \large{Tema 2 - Matemáticas Avanzadas de la Física}\vspace{-3ex}}
\author{M. en C. Gustavo Contreras Mayén}
\date{ }
\begin{document}
\vspace{-4cm}
\maketitle
\fontsize{14}{14}\selectfont
Recuerda que en esta semana tendrás habilitado el espacio para respuestas, el próximo día viernes 23 de octubre se cerrará la recepción a las 18 pm.
\par
Te recomendamos que descargues el pdf y resuelvas cada inciso, cuando ya tengas la respuesta, anótala en la plataforma.
\begin{enumerate}
\item Determina los puntos singulares de las siguientes ecuaciones diferenciales. Clasifica el(los) punto(s) como singular o irregular.
\begin{enumerate}
\item $x^{2} \, (x - 5) \stilde{y} - 2 \, x \, \ptilde{y} + 6 \, y = 0$
\item $(x^{2} + x - 6) \, \stilde{y} + (x + 3)  \, \ptilde{y} + (x - 2) \, y = 0$
\item $x^{3} \, (x^{2} - 25) \, (x - 2)^{2} \, \stilde{y} + 3 \, x \, (x - 2) \, \ptilde{y} + 7 \, (x + 5) \, y = 0$
\end{enumerate}
\item Ocupa el método de Frobenius y presenta: la ecuación de índices, las raíces así como la(s) solución(es) de la EDO2 de los problemas:
\begin{enumerate}
\item $3 \, x \, \stilde{y} + (2 - x) \, \ptilde{y} - y = 0$
\item $x^{2} \, \stilde{y} + x \, \ptilde{y} + (x^{2} - \dfrac{1}{4}) \, y = 0$
\end{enumerate}
\end{enumerate}
\end{document}