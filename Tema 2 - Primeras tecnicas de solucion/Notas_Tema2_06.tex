\documentclass[12pt]{article}
\usepackage[utf8]{inputenc}
\usepackage[spanish,es-lcroman, es-tabla]{babel}
\usepackage[autostyle,spanish=mexican]{csquotes}
\usepackage{amsmath}
\usepackage{amssymb}
\usepackage{nccmath}
\numberwithin{equation}{section}
\usepackage{amsthm}
\usepackage{graphicx}
\usepackage{epstopdf}
\DeclareGraphicsExtensions{.pdf,.png,.jpg,.eps}
\usepackage{color}
\usepackage{float}
\usepackage{multicol}
\usepackage{enumerate}
\usepackage[shortlabels]{enumitem}
\usepackage{anyfontsize}
\usepackage{anysize}
\usepackage{array}
\usepackage{multirow}
\usepackage{enumitem}
\usepackage{cancel}
\usepackage{tikz}
\usepackage{circuitikz}
\usepackage{tikz-3dplot}
\usetikzlibrary{babel}
\usetikzlibrary{shapes}
\usepackage{bm}
\usepackage{mathtools}
\usepackage{esvect}
\usepackage{hyperref}
\usepackage{relsize}
\usepackage{siunitx}
\usepackage{physics}
%\usepackage{biblatex}
\usepackage{standalone}
\usepackage{mathrsfs}
\usepackage{bigints}
\usepackage{bookmark}
\spanishdecimal{.}

\setlist[enumerate]{itemsep=0mm}

\renewcommand{\baselinestretch}{1.5}

\let\oldbibliography\thebibliography

\renewcommand{\thebibliography}[1]{\oldbibliography{#1}

\setlength{\itemsep}{0pt}}
%\marginsize{1.5cm}{1.5cm}{2cm}{2cm}


\newtheorem{defi}{{\it Definición}}[section]
\newtheorem{teo}{{\it Teorema}}[section]
\newtheorem{ejemplo}{{\it Ejemplo}}[section]
\newtheorem{propiedad}{{\it Propiedad}}[section]
\newtheorem{lema}{{\it Lema}}[section]

\usepackage{mathrsfs}
\spanishdecimal{.}
%\usepackage{enumerate}
%\author{M. en C. Gustavo Contreras Mayén. \texttt{curso.fisica.comp@gmail.com}}
\title{Función Gamma y Beta \\ {\large Matemáticas Avanzadas de la Física}}
\date{ }
\begin{document}
%\renewcommand\theenumii{\arabic{theenumii.enumii}}
\renewcommand\labelenumii{\theenumi.{\arabic{enumii}}}
\maketitle
\fontsize{14}{14}\selectfont
\section{Definiciones.}
La definición de la función Gamma es
\[ \Gamma(x) = \int_{0}^{\infty} t^{x-1}e^{-t} dt \hspace{1cm x > 0} \]
y una de sus propiedades es
\[ \Gamma(x+1) = x \Gamma(x) \]
La función Gamma es particularmente útil en problemas de probabilidad, en especial, en problemas que involucran el factorial de números muy grandes, se define la función Gamma incompleta
\[ \gamma(x, \tau ) = \int_{0}^{\tau} t^{x-1}e^{-t} dt, \hspace{1cm} x>0, \tau >0  \]
La función Beta $B(x,y)$ se define como
\[ B(x,y) = \int_{0}^{1} t^{x-1} (1-t)^{y-1} dt, \hspace{1cm} x>0, y>0 \]
Las funciones Beta y Gamma están relacionadas por
\[ B(x,y) = \dfrac{\Gamma(x) \Gamma(y)}{\Gamma(x+y)} \]
\section{Ejercicios de la función Gamma.}
\begin{enumerate}
\item Demostrar que
\[ \Gamma(x) = \int_{0}^{1} \left[ \ln \left( \dfrac{1}{u} \right) \right]^{x-1} du \hspace{1cm} x>0 \]
Empezamos con
\[ \Gamma(x) = \int_{0}^{\infty} t^{x-1} e^{-t} dt \hspace{1cm} x>0 \]
Hacemos el cambio de variable $u = e^{-t}$ por lo que
\[ \dfrac{1}{u} = e^{t}, \hspace{1cm} ln \left(\dfrac{1}{u} \right) = t, \hspace{1cm} -\left(\dfrac{1}{u} \right) du = dt, \hspace{1cm} \left[ ln \left( \dfrac{1}{u} \right) \right]^{x-1} = t^{x-1} \]
Los límites de integración quedan: cuando $t=0$ entonces $u=1$ y cuando $t = \infty$ entonces $u=0$, así
\[ \int_{0}^{\infty} t^{x-1} e^{-t} dt = - \int_{1}^{0} \left[  \ln \left( \dfrac{1}{u} \right) \right]^{x-1} u \dfrac{1}{u} du = \int_{0}^{1} \left[ ln \left( \dfrac{1}{u} \right) \right]^{x-1} du  \]
\item Demostrar que 
 \[ \Gamma(x) = 2 \int_{0}^{\infty} m^{2x-1} e^{-m^{2}} dm, \hspace{1cm} x>0 \]
\\
Partimos de la definición $\int_{0}^{\infty} t^{x-1} e^{-t} dt, \hspace{1cm} x>0$.
\\
Hacemos $t=m^{2}$, entonces $dt= 2m dm$. Los límites de integración quedan en los mismos valores, por  lo que
\[ \begin{split}
\Gamma(x) &= \int_{0}^{\infty} t^{x-1} e^{-t} dt = \int_{0}^{\infty} m^{2x-1} e^{-m^{2}} 2m dm \\
 &= 2 \int_{0}^{\infty} m^{2x-1} e^{-m^{2}} dm, \hspace{1cm} x>0
\end{split}  \]
\end{enumerate}
Algunas propiedades:
\begin{eqnarray}
\Gamma (x) &=& \dfrac{\Gamma(x+1)}{x} \nonumber \\
\Gamma (x) &=& (x-1) \Gamma (x-1) \nonumber \\
\Gamma (-x) &=& \dfrac{\Gamma (1-x)}{-x} \hspace{0.5cm} x \neq 0,1,2, \ldots \nonumber 
\end{eqnarray}
\begin{enumerate}
\item Evaluar $\Gamma(0.37)$
\\
Lo que podemos hacer en este caso es incrementar el argumento de $0.37$ a $1.37$, y aprovechar la identidad $\Gamma(x) = \Gamma(x+1)/x$ por lo que
\[ \Gamma(0.37) = \dfrac{\Gamma(0.37 + 1)}{0.37} = \dfrac{\Gamma(1.37)}{0.37} =\dfrac{0.889314}{0.37} = 2.40355 \]
\item Evaluar $\Gamma(\frac{9}{4})$.
\\
Ahora lo que hacemos es reducir el argumento de $\frac{9}{4}$ a $\frac{5}{4}$, usando la identidad $\Gamma(x) = (x-1) \Gamma(x-1)$
\[ \Gamma \left( \dfrac{9}{4} \right) = \left( \dfrac{5}{4} \right) \Gamma \left( \dfrac{5}{4} \right) = \left( \dfrac{5}{4} \right) \Gamma(1.25) = \left( \dfrac{5}{4} \right) (0.906402) = 1.133 \]
\item Evaluar $\Gamma(4.6)$
\\
Aplicamos hasta en tres ocasiones la identidad $\Gamma(x) = (x-1) \Gamma(x-1)$
\[ \begin{split}
\Gamma(4.6) &= (3.6) \Gamma(3.6) = (3.6) (2.6) \Gamma(2.6) = (3.6)(2.6)(1.6) \Gamma(1.6) \\
&= 14.976 (0.893515) = 13.3813
\end{split} \]
\item Evaluar $\Gamma(-1.3)$
Aplicamos un incremento al argumento en tres ocasiones, y usamos la identidad $\Gamma(x)= \Gamma(x+1) /x$
\[ \begin{split}
\Gamma(-1.3) &= \dfrac{\Gamma(-0.3)}{-1.3} = \dfrac{\Gamma(0.7)}{(-1.3)(-0.3)} = \dfrac{\Gamma(1.7)}{(-1.3)(-0.3)(0.7)} \\
&= \dfrac{0.908639}{0.273} = 3.32835
\end{split} \]
\end{enumerate}
\section{La función Beta.}
Usnado la definición integral, podemos escribir el producto de dos factoriales como el producto de dos integrales, y si tomamos las integrales sobre intervalos finitos, se nos facilita el cambio de variables:
\begin{equation}
m! n! = \lim_{a^{2} \to \infty} \int_{0}^{a^{2}} e^{-u} u^{m} du \int_{0}^{a^{2}} e^{-v} v^{n} dv \hspace{1cm} \mathbb{R}(m) > -1, \mathbb{R}(n) > -1
\label{eq:ecuacion_10_57a}
\end{equation}
Re-emplazando $u$ por $x^{2}$ y $v$ por $y^{2}$, tenemos
\begin{equation}
m! n! =  \lim_{a \to \infty} 4 \int_{0}^{a} e^{-x^{2}} x ^{2m+1} dx \int_{0}^{a} e^{-y^{2}} y^{2n+1} dy
\label{eq:ecuacion_10_57b}
\end{equation}
Transformando a coordenadas polares
\begin{eqnarray}
\begin{aligned}
m! n! &= \lim_{a \to \infty} 4 \int_{0}^{a} e^{-r^{2}} r^{2m+2n+3} dr \int_{0}^{\pi/2} \cos^{2m+1} \theta \sin^{2n+1} \theta d\theta \\
&= (m+n+1)! 2 \int_{0}^{\pi/2} \cos^{2m+1} \theta \sin^{2n+1} \theta d \theta
\label{eq:ecuacion_10_58}
\end{aligned}
\end{eqnarray}
donde el elemento de área cartesiano $dxdy$ fue re-emplazado por $r dr d\theta$.
\\
A la integral definida junto con el factor $2$, se le llama \emph{función Beta}
\begin{eqnarray}
\begin{aligned}
B(m+1,n+1) &\equiv \int_{0}^{\pi/2} \cos^{2m+1} \theta \sin^{2n+1} \theta d \theta \\
&= \dfrac{m! n!}{(m+n+1)!} = B(n+1,m+1)
\label{eq:ecuacion_10_59a}
\end{aligned}
\end{eqnarray}
Expresada en términos de la función Gamma
\begin{equation}
B(p,q) = \dfrac{\Gamma(p) \Gamma(q)}{\Gamma(p+q)}
\label{eq:ecuacion_10_59b}
\end{equation}
\subsection{Formas alternas: integrales definidas}
La función Beta es útil para la evaluación de una amplia variedad de integrales definidas.\\
Al sustituir $t= \cos^{2} \theta$, la ecuación (\ref{eq:ecuacion_10_59a}) queda como
\begin{equation}
B(m+1,n+1) = \dfrac{m! n!}{(m+n+1)} = \int_{0}^{1} t^{m} (1-t)^{n} dt
\label{eq:ecuacion_10_60a}
\end{equation}
Re-emplazando $t$ por $x^{2}$, obtenemos
\begin{equation}
\dfrac{m! n!}{2(m+n+1)!} = \int_{0}^{1} x^{2m+1} (1-x^{2})^{n} dx
\label{eq:ecuacion_10_60a}
\end{equation}
Al sustituir $t= u/(1+u)$ en la ecuación (\ref{eq:ecuacion_10_60a}), nos proporciona otra expresión útil
\begin{equation}
\dfrac{m! n!}{(m+n+1)!} = \int_{0}^{\infty} \dfrac{u^{m}}{(1+u)^{m+n+2}} du
\label{eq:ecuacion_10_61}
\end{equation}
\subsection{La función Beta incompleta.}
Veremos que como la función Gamma, la función Beta también tiene una definición ''incompleta''
\begin{eqnarray}
\begin{aligned}
B_{x}(p,q) = \int_{0}^{x} t^{p-1} (1-p)^{q-1} dt \hspace{1.5cm} & 0 \leq x \leq 1 \\
& p > 0 \\
& q > 0 (\text{ si } x=1)
\label{eq:ecuacion_10_67}
\end{aligned}
\end{eqnarray}
Cuando $B_{x=1}(p,q)$ se recupera la forma regular (completa) de la función Beta.
\section{La función Gamma incompleta.}
De la definición de Euler para la función Gamma 
\[  \Gamma(z) \equiv \int_{0}^{\infty} e^{-t} t^{z-1} dt \hspace{1.5cm} \mathbb{R}(z) > 0 \]
se define la función Gamma incompleta por los límites variables de la integral
\[ \gamma(a,x) = \int_{0}^{x} e^{-t} t^{a-1} dt, \hspace{1.5cm} \mathbb{R} (a) > 0 \]
y
\begin{equation}
\Gamma(a,x) = \int_{x}^{\infty} e^{-t} t^{a-1} dt 
\label{eq:ecuacion_10_68}
\end{equation}
Las dos funciones están relacionadas por
\begin{equation}
\gamma(a,x) + \Gamma(a,x) = \Gamma(a)
\end{equation}
\end{document}