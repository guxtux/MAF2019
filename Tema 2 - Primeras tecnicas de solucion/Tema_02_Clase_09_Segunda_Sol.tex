\documentclass[12pt]{beamer}
\usepackage{../Estilos/BeamerMAF}
\usetheme{Warsaw}
\usecolortheme{crane}
%\useoutertheme{default}
\setbeamercovered{invisible}
% or whatever (possibly just delete it)
\setbeamertemplate{section in toc}[sections numbered]
\setbeamertemplate{subsection in toc}[subsections numbered]
\setbeamertemplate{subsection in toc}{\leavevmode\leftskip=3.2em\rlap{\hskip-2em\inserttocsectionnumber.\inserttocsubsectionnumber}\inserttocsubsection\par}
\setbeamercolor{section in toc}{fg=blue}
\setbeamercolor{subsection in toc}{fg=blue}
\setbeamercolor{frametitle}{fg=blue}
\setbeamertemplate{caption}[numbered]

\setbeamertemplate{footline}
\beamertemplatenavigationsymbolsempty
\setbeamertemplate{headline}{}


\makeatletter
\setbeamercolor{section in foot}{bg=gray!30, fg=black!90!orange}
\setbeamercolor{subsection in foot}{bg=blue!30}
\setbeamercolor{date in foot}{bg=black}
\setbeamertemplate{footline}
{
  \leavevmode%
  \hbox{%
  \begin{beamercolorbox}[wd=.333333\paperwidth,ht=2.25ex,dp=1ex,center]{section in foot}%
    \usebeamerfont{section in foot} \insertsection
  \end{beamercolorbox}%
  \begin{beamercolorbox}[wd=.333333\paperwidth,ht=2.25ex,dp=1ex,center]{subsection in foot}%
    \usebeamerfont{subsection in foot}  \insertsubsection
  \end{beamercolorbox}%
  \begin{beamercolorbox}[wd=.333333\paperwidth,ht=2.25ex,dp=1ex,right]{date in head/foot}%
    \usebeamerfont{date in head/foot} \insertshortdate{} \hspace*{2em}
    \insertframenumber{} / \inserttotalframenumber \hspace*{2ex} 
  \end{beamercolorbox}}%
  \vskip0pt%
}
\makeatother

\makeatletter
\patchcmd{\beamer@sectionintoc}{\vskip1.5em}{\vskip0.8em}{}{}
\makeatother

\newlength{\depthofsumsign}
\setlength{\depthofsumsign}{\depthof{$\sum$}}
\newcommand{\nsum}[1][1.4]{% only for \displaystyle
    \mathop{%
        \raisebox
            {-#1\depthofsumsign+1\depthofsumsign}
            {\scalebox
                {#1}
                {$\displaystyle\sum$}%
            }
    }
}
\def\scaleint#1{\vcenter{\hbox{\scaleto[3ex]{\displaystyle\int}{#1}}}}
\def\scaleoint#1{\vcenter{\hbox{\scaleto[3ex]{\displaystyle\oint}{#1}}}}
\def\bs{\mkern-12mu}


% \makeatletter
% \setbeamertemplate{footline}
% {
%   \leavevmode%
%   \hbox{%
%   \begin{beamercolorbox}[wd=.333333\paperwidth,ht=2.25ex,dp=1ex,center]{section in foot}%
%     \usebeamerfont{section in foot} \insertsection
%   \end{beamercolorbox}%
%   \begin{beamercolorbox}[wd=.333333\paperwidth,ht=2.25ex,dp=1ex,center]{subsection in foot}%
%     \usebeamerfont{subsection in foot}  \insertsubsection
%   \end{beamercolorbox}%
%   \begin{beamercolorbox}[wd=.333333\paperwidth,ht=2.25ex,dp=1ex,right]{date in head/foot}%
%     \usebeamerfont{date in head/foot} \insertshortdate{} \hspace*{2em}
%     \insertframenumber{} / \inserttotalframenumber \hspace*{2ex} 
%   \end{beamercolorbox}}%
%   \vskip0pt%
% }
% \makeatother
\date{20 de octubre de 2021}

\title{\large{Segunda solución linealmente independiente}}
\subtitle{Tema 2 - Primeras técnicas de solución}
\author{M. en C. Gustavo Contreras Mayén}

\begin{document}
\maketitle
\fontsize{14}{14}\selectfont
\spanishdecimal{.}

\section*{Contenido}
\frame{\tableofcontents[currentsection, hideallsubsections]}

%Ref. Makarenko. 17. Integración de las ED mediante series
\section{Solución mediante series}
\frame{\tableofcontents[currentsection, hideothersubsections]}
\subsection{Ecuaciones lineales de 2o. orden}

\begin{frame}
\frametitle{Tipo de EDO}
Este método resulta muy común al aplicarlo a las ecuaciones diferenciales lineales, como se ha visto en el tema, seguiremos manejando ecuaciones de segundo orden.
\\
\bigskip
\pause
Consideremos la EDO2H lineal:
\pause
\begin{align}
    \stilde{y} + p(x) \, \ptilde{y} + q(x) \, y = 0
    \label{eq:ecuacion_17_01}
\end{align}
\end{frame}
\begin{frame}
\frametitle{Funciones analíticas}
Haremos la suposición que los coeficientes $p(x)$ y $q(x)$ son funciones analíticas, \pause es decir, se expresan en forma de series de potencias, dispuestas según las potencias enteras positivas de $x$.
\end{frame}
\begin{frame}
\frametitle{EDO como serie de potencias}
De modo que la ec. (\ref{eq:ecuacion_17_01}) se puede escribir de la forma:
\pause
\begin{align}
\begin{aligned}[b]
\stilde{y} &+ \big( a_{0} + a_{1} \, x + a_{2} \, x^{2} + \ldots \big) \, \ptilde{y} + \\[0.5em]
&+ (b_{0} + b_{1} \, x + b_{2} \, x^{2} + \ldots) \, y = 0
\end{aligned}
\label{eq:ecuacion_17_02}
\end{align}
\end{frame}

\section{Serie de potencias generalizada}
\frame{\tableofcontents[currentsection, hideothersubsections]}
\subsection{Definición}

\begin{frame}
\frametitle{Serie de potencias generalizada}
Definimos la serie de la forma:
\pause
\begin{align}
x^{\rho} \nsum_{n=0}^{\infty} c_{n} \, x^{n}, \hspace{1.5cm} c_{0} \neq 0
\label{eq:ecuacion_17_14}
\end{align}
donde $\rho$ es un número dado \pause y la serie de potencias:
\pause
\begin{align*}
\nsum_{n=0}^{\infty} c_{n} \, x^{n}
\end{align*}
converge para $\abs{x} < R$, se le denomina \emph{\textcolor{red}{serie de potencias generalizada.}}
\end{frame}
\begin{frame}
\frametitle{Serie de potencias ordinaria}
Si $\rho$ es un número entero no negativo, la serie de potencias generalizada (\ref{eq:ecuacion_17_14}) se convierte en una \emph{\textcolor{blue}{serie de potencias ordinaria}}.
\end{frame}
\begin{frame}
\frametitle{Serie con puntos regulares}
Si $x = 0$ es un punto singular regular de la ec. (\ref{eq:ecuacion_17_01}), cuyos coeficientes $p(x)$ y $q(x)$ admiten desarrollos:
\pause
\begin{align}
p(x) = \dfrac{\displaystyle \nsum_{n=0}^{\infty} a_{n} \, x^{n}}{x} \hspace{1cm} q(x) = \dfrac{\displaystyle \nsum_{n=0}^{\infty} b_{n} \, x^{n}}{x^{2}}
\label{eq:ecuacion_17_15}
\end{align}
\pause
donde las series que figuran en los numeradores son convergentes en $\abs{x} < R$, y los coeficientes $a_{0}$, $b_{0}$, $c_{0}$ no son simultáneamente iguales a cero
\end{frame}
\begin{frame}
\frametitle{Series convergentes}
Entonces la ec. (\ref{eq:ecuacion_17_01}) posee al menos una solución en forma de serie de potencias generalizada:
\pause
\begin{align}
y(x) = x^{\rho} \, \nsum_{n=0}^{\infty} C_{n} \, x^{n}, \hspace{1.5cm} C_{n} \neq 0
\label{eq:ecuacion_17_16}
\end{align}
que es convergente al menos en el mismo intervalo $\abs{x} < R$.
\end{frame}
\begin{frame}
\frametitle{Obteniendo $\rho$}
Para hallar el exponente $\rho$ y los coeficientes $C_{n}$, es necesario:
\setbeamercolor{item projected}{bg=blue!70!black,fg=yellow}
\setbeamertemplate{enumerate items}[circle]
\begin{enumerate}[<+->]
\item Expresar la serie (\ref{eq:ecuacion_17_16}) en la ec. (\ref{eq:ecuacion_17_01}).
\item Simplificar para $x^{\rho}$.
\item Igualar a cero los coeficientes en distintas potencias de $x$ (método de coeficientes indeterminados).
\end{enumerate}
\end{frame}

\subsection{Obteniendo el exponente \texorpdfstring{$\rho$}{r}.}

\begin{frame}
\frametitle{Solución propuesta}
Al considerar la solución propuesta:
\pause
\begin{align*}
y = \nsum_{n=0}^{\infty} c_{n} \, x^{n+\rho}
\end{align*}
\pause
se requiere conocer la primera y  segunda derivada de $y$ con respecto a $x$:
\begin{align*}
\ptilde{y} &= \sum_{n=0}^{\infty} (n + \rho) \, c_{n} \, x^{n+\rho-1} \\[1em]
\stilde{y} &= \sum_{n=0}^{\infty} (n + \rho)(n + \rho - 1) \, c_{n} \, x^{n+\rho-2}
\end{align*}
\end{frame}
\begin{frame}
\frametitle{Sustituyendo en la EDO}
Sustituimos las derivadas en la ec. (\ref{eq:ecuacion_17_01}) en conjunto con las definiciones de $p(x)$ y $q(x)$ dadas en la ec. (\ref{eq:ecuacion_17_15}):
\end{frame}
\begin{frame}
\begin{align*}
&\nsum_{n=0}^{\infty} (n {+} \rho)(n {+} \rho {+} 1) \, c_{n} \, x^{n+\rho-2} + \\[0.5em] 
&+ \left[ \dfrac{\displaystyle \nsum_{n=0}^{\infty} a_{n} \, x^{n}}{x} \right] \, \left[ \nsum_{n=0}^{\infty} (n {+} \rho) \, c_{n} \, x^{n+\rho-1} \right] + \\[1em]
&+ \left[ \dfrac{\displaystyle \nsum_{n=0}^{\infty} b_{n} \, x^{n}}{x^{2}} \right] \, \nsum_{n=0}^{\infty} c_{n} \, x^{n+\rho} = 0
\end{align*}
\end{frame}
\begin{frame}
\frametitle{Simplificando la expresión}
\begin{align*}
&\nsum_{n=0}^{\infty} (n {+} \rho)(n {+} \rho {+} 1) \, c_{n} \, x^{n+\rho-2} + \\[0.5em]
&+ \nsum_{n=0}^{\infty} a_{n} \, x^{n-1} \, \nsum_{n=0}^{\infty} (n {+} \rho) \, c_{n} \, x^{n+\rho-1} + \\[1em]
&+ \nsum_{n=0}^{\infty} b_{n} \, x^{n-2} \, \nsum_{n=0}^{\infty} c_{n} \, x^{n+\rho} = 0
\end{align*}
\end{frame}
\begin{frame}
\frametitle{Abreviando el producto de las sumas}
\begin{align*}
&\nsum_{n=0}^{\infty} \big[ (n {+} \rho)(n {+} \rho {+} 1) \, c_{n} \big] \, x^{n+\rho-2} + \\[0.5em] 
&+ \nsum_{n=0}^{\infty} \big[ (n {+} \rho) \, a_{n} \, c_{n} \big] \, x^{2n+\rho-2} + \\[1em]
&+ \nsum_{n=0}^{\infty} \big[ b_{n} \, c_{n} \big] \, x^{2n+\rho-2} = 0
\end{align*}
\end{frame}
\begin{frame}
\frametitle{Factorizando para el exponente mayor de $x$}
\begin{align*}
&\nsum_{n=0}^{\infty} \big[ (n {+} \rho)(n {+} \rho {+} 1) \, c_{n} \big] \, x^{n+\rho-2} + \\[1em]
&+ \nsum_{n=0}^{\infty} \big[ (n {+} \rho) \, a_{n} \, c_{n} + b_{n} \, c_{n} \big] \, x^{2n+\rho-2} = 0
\end{align*}
\end{frame}
\begin{frame}
\frametitle{Haciendo $n = 0$ obtenemos}
\begin{align*}
\big[ \rho (\rho - 1) c_{0} + \rho \, a_{0} \, c_{0} + b_{0} \, c_{0} \big] \, x^{\rho-2} = 0
\end{align*}
\end{frame}

% En este producto los coeficientes deben de anularse y como $c_{0} \neq 0$, se tiene que:
% \begin{align}
% \setlength{\fboxsep}{3\fboxsep}\boxed{
% \rho (\rho - 1) + \rho \, a_{0} + b_{0} = 0 }
% \label{eq:ecuacion_17_17}
% \end{align}
% a esta expresión se le denomina \emph{ecuación determinativa}.
% \par
% Donde:
% \begin{align}
% a_{0} = \lim_{x \to 0} x \, p(x) \hspace{1cm} \mbox{y} \hspace{1cm} b_{0} = \lim_{x \to 0} x^{2} \, q(x)
% \label{eq:ecuacion_17_18}
% \end{align}

% \section{Tres casos.}

% Supongamos que $\rho_{1} > \rho_{2}$ son las raíces de la ec. (\ref{eq:ecuacion_17_17}), se distinguen tres casos en la solución de la ec. (\ref{eq:ecuacion_17_01}):

% \subsection{Caso 1.}\label{caso1}

% Si la diferencia $\rho_{1} - \rho_{2}$ no es un número entero o cero, se pueden construir dos soluciones de la forma:
% \begin{align*}
% y_{1}(x) &= x^{\rho_{1}} \sum_{n=0}^{\infty} c_{n} \, x^{n} \hspace{1.5cm} c_{0} \neq 0 \\ 
% y_{2}(x) &= x^{\rho_{2}} \sum_{n=0}^{\infty} a_{n} \, x^{n} \hspace{1.5cm} a_{0} \neq 0
% \end{align*}

% \subsection{Caso 2.} \label{caso2}

% Si la diferencia $\rho_{1} - \rho_{2}$ es un número entero positivo, por lo general, solo se puede construir una serie como solución de la ec. (\ref{eq:ecuacion_17_01}):
% \begin{align}
% y_{1}(x) = x^{\rho_{1}} \sum_{n=0}^{\infty} c_{n} \, x^{n}
% \label{eq:ecuacion_17_19}
% \end{align}

% Se puede demostrar que si la diferencia $\rho_{1} - \rho_{2}$ es un número entero positivo o cero, además de la solución (\ref{eq:ecuacion_17_19}), habrá una solución de la forma:
% \begin{align}
% y_{2}(x) = A \, y_{1}(x) \, \ln x + x^{\rho_{2}} \sum_{n=0}^{\infty} A_{n} \, x^{n}
% \label{eq:ecuacion_17_20}
% \end{align}

% Vemos que $y_{2}(x)$ contiene un término complementario de la forma:
% \begin{align*}
% A \, y_{1} (x) \, \ln x
% \end{align*}
% \noindent
% donde $y_{1}(x)$ es de la forma (\ref{eq:ecuacion_17_19}). Como punto importante hay que considerar que la constante $A$ en la ec. (\ref{eq:ecuacion_17_20}) sea nula, por lo que la segunda solución $y_{2}(x)$ resulta una expresión de la forma de una serie de potencias generalizada.


% \subsection{Caso 3.}

% Si la ec. (\ref{eq:ecuacion_17_17}) posee una raíz múltiple $\rho_{1} = \rho_{2}$, también se construye solo una serie como solución a la ec. \ref{eq:ecuacion_17_01}.
% \begin{align*}
% y_{1}(x) = x^{\rho_{1}} \sum_{n=0}^{\infty} c_{n} \, x^{n}
% \end{align*}

% También es posible demostrar que en este caso, se puede obtener una segunda solución de la forma:
% \begin{align}
% y_{2}(x) = y_{1}(x) \, \ln x + x^{\rho_{2}} \sum_{n=1}^{\infty} b_{n} \, x^{n+\rho_{1}}
% \label{eq:ecuacion_22_Zill}
% \end{align}

% Comparando con la ec. (\ref{eq:ecuacion_17_20}) se tiene que $A = 1$.
% \\[0.5em]
% \noindent
% Queda claro que en el caso \ref{caso1}, las soluciones $y_{1}(x)$ e $y_{2}(x)$ son linealmente independientes. En el segundo y tercer caso, se ha obtenido una única solución tomando la raíz $\rho_{1}$ de la ecuación de índices, como se indica en la ec. (\ref{eq:ecuacion_17_19}).


\end{document}