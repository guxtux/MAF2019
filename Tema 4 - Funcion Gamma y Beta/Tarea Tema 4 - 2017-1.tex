\documentclass[12pt]{article}
\usepackage[utf8]{inputenc}
\usepackage[spanish,es-lcroman, es-tabla]{babel}
\usepackage[autostyle,spanish=mexican]{csquotes}
\usepackage{amsmath}
\usepackage{amssymb}
\usepackage{nccmath}
\numberwithin{equation}{section}
\usepackage{amsthm}
\usepackage{graphicx}
\usepackage{epstopdf}
\DeclareGraphicsExtensions{.pdf,.png,.jpg,.eps}
\usepackage{color}
\usepackage{float}
\usepackage{multicol}
\usepackage{enumerate}
\usepackage[shortlabels]{enumitem}
\usepackage{anyfontsize}
\usepackage{anysize}
\usepackage{array}
\usepackage{multirow}
\usepackage{enumitem}
\usepackage{cancel}
\usepackage{tikz}
\usepackage{circuitikz}
\usepackage{tikz-3dplot}
\usetikzlibrary{babel}
\usetikzlibrary{shapes}
\usepackage{bm}
\usepackage{mathtools}
\usepackage{esvect}
\usepackage{hyperref}
\usepackage{relsize}
\usepackage{siunitx}
\usepackage{physics}
%\usepackage{biblatex}
\usepackage{standalone}
\usepackage{mathrsfs}
\usepackage{bigints}
\usepackage{bookmark}
\spanishdecimal{.}

\setlist[enumerate]{itemsep=0mm}

\renewcommand{\baselinestretch}{1.5}

\let\oldbibliography\thebibliography

\renewcommand{\thebibliography}[1]{\oldbibliography{#1}

\setlength{\itemsep}{0pt}}
%\marginsize{1.5cm}{1.5cm}{2cm}{2cm}


\newtheorem{defi}{{\it Definición}}[section]
\newtheorem{teo}{{\it Teorema}}[section]
\newtheorem{ejemplo}{{\it Ejemplo}}[section]
\newtheorem{propiedad}{{\it Propiedad}}[section]
\newtheorem{lema}{{\it Lema}}[section]

\usepackage{standalone}
\usepackage{enumerate}
\usepackage{hyperref}
\usepackage[left=1.5cm,top=1.5cm,right=1.5cm,bottom=1.5cm]{geometry}
\title{Problemas para la Tarea - Examen del Tema 4 \\ \large{Matemáticas Avanzadas de la Física} \vspace{-1.5\baselineskip}}
\date{ }
\begin{document}
\vspace{-4cm}
%\renewcommand\theenumii{\arabic{theenumii.enumii}}
\renewcommand\labelenumii{\theenumi.{\arabic{enumii})}}
\maketitle
\fontsize{14}{14}\selectfont
\begin{enumerate}
\item En una distribución tipo Maxwell la fracción de partículas moviéndose con velocidad $v$ y $v +\dd{v}$ es
\begin{align*}
\dfrac{\dd{N}}{N} = 4 \, \pi \left( \dfrac{m}{2 \, \pi \, k \, T} \right)^{3/2} \: \exp \left( \dfrac{-m \, v^{2}}{k \, T} \right) \: v^{2} \dd{v}
\end{align*}
donde $N$ es el número total de partículas. El promedio o valor esperado de $v^{n}$ se define como $\displaystyle \expval{v^{n}} = N^{-1} \int v^{n} \dd{N}$. Demostrar que
\begin{align*}
\expval{v^{n}} = \left( \dfrac{2 \, k \, T}{m} \right)^{n/2} \dfrac{\left( \dfrac{n + 1}{2} \right) !} { \left( \dfrac{1}{2} \right) !}
\end{align*}
\item Demostrar que
\begin{align*}
\int_{0}^{\infty} e^{-x^{4}} \dd{x} = \left( \dfrac{1}{4} \right) !
\end{align*}
\item Comprueba las siguientes identidades de la función Beta:
\begin{enumerate}
\setlength{\itemsep}{15pt}
\item $B(a, b) = B(a + 1, b) + B(a, b + 1)$
\item $B(a, b) = \dfrac{a + b}{b} B(a, b + 1)$ 
\item $B(a, b) = \dfrac{b - 1}{a} B(a + 1, b - 1)$
\item $B(a, b) B(a + b, c) = B(b, c) B(a, b + c)$
\end{enumerate}
\item Demostrar que
\begin{align*}
\int_{-1}^{1} (1-x^{2})^{1/2} \: x^{2 \, n} \dd{x} =  
\begin{cases}
\pi/2 & n = 0 \\[1em]
\pi \dfrac{(2 \, n - 1)!!}{(2 \, n + 2)!!} & n = 1,2,3,\ldots  \end{cases}
\end{align*} 
\item Demuestra que 
\begin{align*}
\Gamma \left( \dfrac{1}{2} - n \right) \: \Gamma \left( \dfrac{1}{2} + n \right) = (-1)^{n} \: \pi
\end{align*}
\end{enumerate}
\end{document}