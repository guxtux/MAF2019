\documentclass[12pt]{article}
\usepackage[utf8]{inputenc}
\usepackage[spanish,es-lcroman, es-tabla]{babel}
\usepackage[autostyle,spanish=mexican]{csquotes}
\usepackage{amsmath}
\usepackage{amssymb}
\usepackage{nccmath}
\numberwithin{equation}{section}
\usepackage{amsthm}
\usepackage{graphicx}
\usepackage{epstopdf}
\DeclareGraphicsExtensions{.pdf,.png,.jpg,.eps}
\usepackage{color}
\usepackage{float}
\usepackage{multicol}
\usepackage{enumerate}
\usepackage[shortlabels]{enumitem}
\usepackage{anyfontsize}
\usepackage{anysize}
\usepackage{array}
\usepackage{multirow}
\usepackage{enumitem}
\usepackage{cancel}
\usepackage{tikz}
\usepackage{circuitikz}
\usepackage{tikz-3dplot}
\usetikzlibrary{babel}
\usetikzlibrary{shapes}
\usepackage{bm}
\usepackage{mathtools}
\usepackage{esvect}
\usepackage{hyperref}
\usepackage{relsize}
\usepackage{siunitx}
\usepackage{physics}
%\usepackage{biblatex}
\usepackage{standalone}
\usepackage{mathrsfs}
\usepackage{bigints}
\usepackage{bookmark}
\spanishdecimal{.}

\setlist[enumerate]{itemsep=0mm}

\renewcommand{\baselinestretch}{1.5}

\let\oldbibliography\thebibliography

\renewcommand{\thebibliography}[1]{\oldbibliography{#1}

\setlength{\itemsep}{0pt}}
%\marginsize{1.5cm}{1.5cm}{2cm}{2cm}


\newtheorem{defi}{{\it Definición}}[section]
\newtheorem{teo}{{\it Teorema}}[section]
\newtheorem{ejemplo}{{\it Ejemplo}}[section]
\newtheorem{propiedad}{{\it Propiedad}}[section]
\newtheorem{lema}{{\it Lema}}[section]

\numberwithin{equation}{section}
\usepackage{tikz-3dplot}
%\author{M. en C. Gustavo Contreras Mayén. \texttt{curso.fisica.comp@gmail.com}}
\title{Matemáticas Avanzadas de la Física \\ {\large Sistemas coordenados especiales}}
\date{ }
\begin{document}
%\renewcommand\theenumii{\arabic{theenumii.enumii}}
\renewcommand\labelenumii{\theenumi.{\arabic{enumii}}}
\maketitle
\fontsize{14}{14}\selectfont
\section{Sistema coordenado cartesiano}
Los factores de escala son:
\begin{eqnarray}
\begin{aligned}
h_{1} &=& h_{x} = 1 \\
h_{2} &=& h_{y} = 1 \\
h_{3} &=& h_{z} = 1 \\
\end{aligned}
\end{eqnarray}
Las siguientes expresiones corresponden para el gradiente, divergencia, laplaciano y rotacional:
\begin{eqnarray}
\nabla \psi &=& \mathbf{i} \dfrac{\partial \psi}{\partial x} + \mathbf{j} \dfrac{\partial \psi}{\partial y} + \mathbf{k} \dfrac{\partial \psi}{\partial z} \\
\nabla \cdot \mathbf{V} &=& \dfrac{\partial \psi}{\partial x} + \dfrac{\partial \psi}{\partial y} + \dfrac{\partial \psi}{\partial z} \\
\nabla \cdot \nabla \psi &=& \dfrac{\partial^{2} \psi}{\partial x^{2}} + \dfrac{\partial^{2} \psi}{\partial y^{2}} +  \dfrac{\partial^{2} \psi}{\partial z^{2}} \\
\nabla \times \mathbf{V} &=& \begin{vmatrix}
\mathbf{i} & \mathbf{j} & \mathbf{k} \\
\dfrac{\partial}{\partial x} & \dfrac{\partial}{\partial y} & \dfrac{\partial}{\partial z} \\
V_{x} & V_{y} & V_{z}
\end{vmatrix}
\end{eqnarray}
\section{Sistema coordenado cilíndrico ($\rho,\varphi,z$)}
En el sistema coordenado cilíndrico las tres coordenadas curvilíneas del sistema generalizado $(q_{1},q_{2},q_{3})$, se renombran por $(\rho, \varphi, z)$. Las superficies coordenadas:
\begin{enumerate}
\item Los cilindros circulares tienen al eje $z$ como eje común, tal que:
\[ \rho =  (x^{2} + y^{2})^{1/2} = \text{constante} \]
\item Los semiplanos a través del eje $z$
\[ \varphi = \tan^{-1} \left(\dfrac{y}{x} \right) = \text{constante} \]
\item Los planos paralelos al plano $x-y$, como en el sistema cartesiano:
\[ z = \text{constante} \]
\end{enumerate}
\begin{figure}[!h]
\centering
\tdplotsetmaincoords{70}{120}
\begin{tikzpicture}[tdplot_main_coords][scale=0.75]
\tikzstyle{every node}=[font=\small]
\draw[thick,-latex] (0,0,0) -- (6,0,0) node[anchor=north east]{$x$};
\draw[thick,-latex] (0,0,0) -- (0,6,0) node[anchor=north west]{$y$};
\draw[thick,-latex] (0,0,0) -- (0,0,6) node[anchor=south]{$z$};
\draw [thick](0,0,0) circle (3);
\draw [thick](0,0,4) circle (3);
\draw [thick](1.9,-2.35,0) -- (1.9,-2.35,4) node[midway, left]{$\rho=\rho_1$ superficie};
\draw [thick](-1.9,2.35,0) -- (-1.9,2.35,4);
\filldraw[fill=orange, nearly transparent] (-4,-4,4) -- (4,-4,4) --  (4,5,4) -- (-4,5,4) -- (-4,-4,4);
\filldraw[fill=blue, nearly transparent] (0,0,4) -- (5.2,6,4) --  (5.2,6,0) -- (0,0,0) -- (0,0,4);
\filldraw [color=blue](2,2.25,4) circle (0.075cm) ;
\draw (-4,5,4) node[anchor=south]{$z=z_1$ plano};
\draw (5.2,6,0) node[anchor=south west]{$\varphi=\varphi_1$ plano};
\node at (1.8,1,4)  { $\rho_{1}(\rho_{1},\varphi_{1},z_{1})$};
\draw[ultra thick,-latex](2,2.25,4) -- (3,3.45,4) node[anchor=north] {$\bm{\rho}_{0}$};
\draw[ultra thick,-latex](2,2.25,4) -- (1,2.5,4) node[anchor=north west] {$\bm{\varphi}_{0}$};
\draw[ultra thick,-latex](2,2.25,4) -- (2,2.25,4.75) node[anchor=north west] {$\mathbf{k}$};
\draw [thick,->](4,0,0) arc (0:45:4 and 4.5);
\draw (3.6,2,0) node[anchor=north] {$\varphi_1$};
\draw[ultra thick,-latex](0,0,0) -- (2,2.35,0);
\draw (1,1,0) node[anchor=north] {$\rho_1$};
\draw [ultra thick] (2,2.25,4)--(1.95,2.25,0);
\draw[ultra thick](0.1,0,4) -- (-0.1,0,4) node[anchor=south west] {$z_1$};
\end{tikzpicture}
\end{figure}
Los límites de $\rho, \varphi$ y $z$, son:
\[ 0 \leq \rho < \infty, \hspace{1cm} 0 \leq \varphi \leq 2 \pi, \hspace{1cm} -\infty < z < \infty \]
Podemos recuperar las relaciones de transformación:
\begin{eqnarray}
\begin{aligned}
x &= \rho \cos \varphi \\
y &= \rho \sin \varphi \\
z &= z
\end{aligned}
\end{eqnarray}
Considerando los elementos de longitud $ds_{1}$, revisamos que los factores de escala son:
\begin{eqnarray}
\begin{aligned}
h_{1} &=& h_{\rho} = 1 \\
h_{2} &=& h_{\varphi} = \rho \\
h_{3} &=& h_{z} = 1
\end{aligned}
\end{eqnarray}
Los vectores unitarios $(\mathbf{e}_{1},\mathbf{e}_{2},\mathbf{e}_{3})$ se renombran $(\bm{\rho}_{0},\bm{\varphi}_{0},\mathbf{k})$:
\begin{enumerate}
\item El vector unitario $\bm{\rho}_{0}$ es normal a la superficie cilíndrica que apunta en la dirección del incremento del radio $\rho$.
\item El vector unitario $\bm{\varphi}_{0}$ es tangencial a la superficie cilíndrica y además, perpendicular al semiplano $\varphi=\text{constante}$, y el vector apunta en la dirección del incremento del ángulo azimutal $\varphi$.
\item El vector $k$, es el vector unitario que conocemos del sistema cartesiano.
\end{enumerate}
Un elemento diferencial de desplazamineto $d\mathbf{r}$ se puede escribir como
\begin{eqnarray}
\begin{aligned}
d \mathbf{r} &= \bm{\rho}_{0} ds_{\rho} + \bm{\varphi}_{0} ds_{\varphi} + \mathbf{k} dz \\
&= \bm{\rho}_{0} d \rho + \bm{\varphi}_{0} \rho d \varphi + \mathbf{k} dz
\end{aligned}
\end{eqnarray}
Los operadores diferenciales, resultan ser:
\begin{eqnarray}
\nabla \varphi (\rho, \varphi, z) &=& \bm{\rho}_{0} \dfrac{\partial \varphi}{\partial \rho} + \bm{\varphi}_{0} \dfrac{1}{\rho} \dfrac{\partial \varphi}{\partial \varphi} + \mathbf{k} \dfrac{\partial \varphi}{\partial z} \\
\bm{\nabla} \cdot \mathbf{V} &=& \dfrac{1}{\rho} \dfrac{\partial}{\partial \rho} (\rho V_{\rho}) + \dfrac{1}{\rho} \dfrac{\partial V_{\varphi}}{\partial \varphi} + \dfrac{\partial V_{z}}{\partial z} \\
\nabla^{2} &=& \dfrac{1}{\rho} \dfrac{\partial}{\partial \rho} \left( \rho \dfrac{\partial \varphi}{\partial \rho} \right) + \dfrac{1}{\rho^{2}} \dfrac{\partial^{2} \varphi}{\partial \varphi^{2}} + \dfrac{\partial^{2} \varphi}{\partial z^{2}} \\
\mathbf{\nabla} \times \mathbf{V} &=& \dfrac{1}{\rho} \begin{vmatrix}
\bm{\rho}_{0} & \bm{\varphi}_{0} & \mathbf{k} \\
\dfrac{\partial}{\partial \rho} & \dfrac{\partial}{\partial \varphi} & \dfrac{\partial}{\partial z} \\
V_{\rho} & \rho V_{\varphi} & V_{z}
\end{vmatrix}
\end{eqnarray}
\section{Coordenadas esféricas polares $(r,\theta, \varphi)$}
Renombrando las coordenadas $(q_{1},q_{2},q_{3})$ como $(r, \theta, \varphi)$, vemos que el sistema coordenado esférico es consistente con:
\begin{enumerate}
\item Tenemos esferas concéntricas en el origen:
\[ r = (x^{2} + y^{2} + z^{2})^{1/2} =  \text{constante} \]
\item Hay conos circulares concéntricos en el eje $z$-polar, con vértices en el origen:
\[ \theta = \arccos \dfrac{z}{(x^{2} +y^{2} + z^{2})^{1/2}} = \text{constante}\]
\item Tenemos que los semiplanos pasan a través del eje $z$-polar:
\[ \varphi = \arctan\left(\dfrac{y}{x} \right) =  \text{constante}\]
\end{enumerate}
Dado que la elección del ángulo polar $\theta$, el ángulo azimutal $\varphi$, el eje $z$ merece un manejo especial. Las ecuaciones de transformación son:
\begin{eqnarray}
\begin{aligned}
x &= r \sin \theta \cos \varphi \\
y &= r \sin \theta \sin \varphi \\
z &= r \cos \theta
\end{aligned}
\end{eqnarray}
Midiendo $\theta$ en el cuadrante positivo del eje $z$ y $\varphi$ en el plano $x-y$ sobre el eje positivo $x$. Los rangos donde varían las coordenadas son:
\begin{eqnarray}
\begin{aligned}
h_{1} &= h_{r} = 1 \\
h_{2} &= h_{\theta} = r \\
h_{3} &= h_{\varphi} = r \sin \theta 
\end{aligned}
\end{eqnarray}
Lo que nos da un elemento de línea
\[ d \mathbf{r} = \mathbf{r}_{0} dr + \bm{\theta}_{0} r d\theta + \bm{\varphi}_{0} r \sin \theta d \varphi \]
En este sistema coordenado esférico, el elemtneo de área (para $r=\text{constante}$) es:
\[ dA = d\sigma_{\theta \varphi} = r^{2} \sin \theta d\theta d\varphi \]
Integrando sobre la coordenada azimutal $\varphi$, se tiene que el elemento de área, genera un anillo de ancho $d\theta$
\[ dA = 2 \pi r^{2} \sin \theta d \theta \]
Esta expresión se presenta frecuentemente en problemas con coordenadas esféricas polares con simetría azimutal, tales como la dispersión de un haz no polarizado de partículas nucleares.
\\
Por definición de estereoradianes, un elemento de ángulo sólido $d\Omega$ está dado por:
\[ d \Omega = \dfrac{dA}{r^{2}} = \sin \theta d \theta d \varphi \]
Integrnado sobre toda la superficie esférica, se obtiene
\[ \int d \Omega = 4 \pi \]
El elemento de volumen es:
\begin{eqnarray}
\begin{aligned}
d \tau &= r^{2} dr \sin \theta d \theta d\varphi \\
&= r^{2} d r d \Omega
\end{aligned}
\end{eqnarray}
Los vectores unitarios del sistema polar esféricos se muestran en la siguiente figura:
\begin{figure}[!h]
\centering
\tdplotsetmaincoords{60}{110}
%
\pgfmathsetmacro{\rvec}{.8}
\pgfmathsetmacro{\thetavec}{30}
\pgfmathsetmacro{\phivec}{60}
%
\begin{tikzpicture}[scale=5,tdplot_main_coords]
    \coordinate (O) at (0,0,0);
    \draw[thick,->] (0,0,0) -- (1,0,0) node[anchor=north east]{$x$};
    \draw[thick,->] (0,0,0) -- (0,1,0) node[anchor=north west]{$y$};
	\draw[thick,->] (0,0,0) -- (0,0,1) node[anchor=south]{$z$};
    \tdplotsetcoord{P}{\rvec}{\thetavec}{\phivec}
    \draw[-stealth,color=red] (O) -- (P) node[above right] {$P$};
    \draw[dashed, color=red] (O) -- (Pxy);
    \draw[dashed, color=red] (P) -- (Pxy);
    %\draw[dashed, color=red] (0,0,0.7) -- (O);
    \tdplotdrawarc{(O)}{0.4}{0}{\phivec}{anchor=north}{$\phi$}
    \tdplotsetthetaplanecoords{\phivec}
    \tdplotdrawarc[tdplot_rotated_coords]{(0,0,0)}{0.5}{0}%
        {\thetavec}{anchor=south west}{$\theta$}


\end{tikzpicture}
\end{figure}
Se insiste en que los vectores unitarios $\mathbf{r}_{0}, \bm{\theta}_{0}, \bm{\varphi}_{0}$ cambian de dirección conforme cambian los ángulos $\theta$ y $\varphi$. En términos de los vectores unitarios cartesianos $(\mathbf{i},\mathbf{j},\mathbf{k})$ cuya dirección es fija, tenemos que
\begin{eqnarray}
\begin{aligned}
\mathbf{r}_{0} &= \mathbf{i}\sin \theta \cos \varphi + \mathbf{j} \sin \theta \sin \varphi + \mathbf{k} \cos \theta \\
\bm{\theta}_{0} &= \mathbf{i}\cos \theta \cos \varphi + \mathbf{j} \cos \theta \sin \varphi - \mathbf{k} \sin \theta \\
\bm{\varphi}_{0} &= - \mathbf{i} \sin \varphi + \mathbf{j} \cos \varphi
\end{aligned}
\end{eqnarray}
Renombrando los vectores unitarios del sistema de coordenadas curvilineas $(\mathbf{e}_{1}, \mathbf{e}_{2}, \mathbf{e}_{3}) $ como $(\mathbf{r}_{0}, \bm{\theta}_{0}, \bm{\varphi}_{0})$, los operadores diferenciales son:
\begin{eqnarray}
\nabla \psi &=& \mathbf{r}_{0} \dfrac{\partial \psi}{\partial r} + \bm{\theta}_{0} \dfrac{1}{r} \dfrac{\partial \psi}{\partial \theta} + \bm{\varphi}_{0} \dfrac{1}{r \sin \theta} \dfrac{\partial \psi}{\partial \varphi} \\
\mathbf{\nabla \cdot V} &=& \dfrac{1}{r^{2} \sin \theta} \left[ \sin \theta \dfrac{\partial}{\partial r}(r^{2} V_{r}) + r \dfrac{\partial}{\partial \theta} (\sin \theta V_{\theta}) + r \dfrac{\partial V_{\varphi}}{\partial \varphi} \right] \\
\mathbf{\nabla \cdot \nabla \psi} &=& \dfrac{1}{r^{2} \sin \theta} \left[  \sin \theta \dfrac{\partial}{\partial r} \left(  r^{2} \dfrac{\partial \psi}{\partial r} \right) + \dfrac{\partial}{\partial \theta} \left( \sin \theta \dfrac{\partial \psi}{\partial \theta} \right) + \dfrac{1}{\sin \theta} \dfrac{\partial^{2} \psi}{\partial \varphi^{2}}   \right] \\
\mathbf{\nabla \times V} &=& \dfrac{1}{r^{2} \sin \theta} \begin{vmatrix}
\mathbf{r}_{0} & r \bm{\theta}_{0} & r \sin \theta \bm{\varphi}_{0} \\
\dfrac{\partial}{\partial r} & \dfrac{\partial}{\partial \theta} & \dfrac{\partial}{\partial \varphi} \\
V_{r} & r V_{\theta} & r \sin \theta V_{\varphi}
\end{vmatrix}
\end{eqnarray}
\section{Coordenadas cilíndricas esféricas $(u, \theta, z)$}
Están dadas por:
\begin{eqnarray}
\begin{aligned}
x &= a \cosh u \cos \theta \\
y &= a \sinh u \sin \theta \\
z &= z
\end{aligned}
\end{eqnarray}
donde $a$ es una constante.
\section{Coordenadas toroidales $(\xi,\chi,\varphi)$}
Están definidas por las ecuaciones de transformación:
\begin{eqnarray}
\begin{aligned}
x &=& \dfrac{a \sinh \xi \cos \varphi}{\cosh \xi - \cos \chi} \\
y &=& \dfrac{a \sin  \xi \sin \varphi}{\cosh \xi - \cos \chi} \\
z &=& \dfrac{a \sin \chi}{\cosh \xi - \cos \chi}
\end{aligned}
\end{eqnarray}

\end{document}