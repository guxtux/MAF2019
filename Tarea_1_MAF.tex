\documentclass[12pt]{article}
\usepackage[left=0.5cm,top=1cm,right=0.5cm,bottom=1cm]{geometry}
\usepackage[utf8]{inputenc}
\usepackage[spanish,es-tabla]{babel}
\usepackage{amsmath}
\usepackage{amsthm}
\usepackage{graphicx}
\usepackage{color}
\usepackage{float}
\usepackage{multicol}
\usepackage{enumerate}
\usepackage{anyfontsize}
\usepackage{anysize}
\usepackage{enumitem}
\usepackage{capt-of}
\usepackage{bm}
\usepackage{relsize}
\spanishdecimal{.}
\setlist[enumerate]{itemsep=0mm}
\renewcommand{\baselinestretch}{1.2}
\let\oldbibliography\thebibliography
\renewcommand{\thebibliography}[1]{\oldbibliography{#1}
\setlength{\itemsep}{0pt}}
%\marginsize{1.5cm}{1.5cm}{0cm}{2cm}
\title{Tarea 1 - Matemáticas Avanzadas de la Física}
\date{ }
\begin{document}
\vspace{-4cm}
%\renewcommand\theenumii{\arabic{theenumii.enumii}}
\renewcommand\labelenumii{\theenumi.{\arabic{enumii}}}
\maketitle
\fontsize{14}{14}\selectfont
Para las respuestas de la tarea, te pedimos sea lo más claro posible, presentando de manera organizada tu solución. La calificación de esta tarea está en función del número de ejercicios que entregues, cuentas con el suficiente tiempo para resolverla y hacer un buen esfuerzo.
\begin{enumerate}
\item Si $\mathbf{A}= 2 \mathbf{i} - \mathbf{j} - \mathbf{k}$, $\mathbf{b}= 2 \mathbf{i} - 3 \mathbf{j} + \mathbf{k}$, $\mathbf{C}= \mathbf{j} + \mathbf{k}$. Calcula $(\mathbf{A} \cdot \mathbf{B}) \mathbf{C}$, $\mathbf{A}(\mathbf{B} \cdot \mathbf{C})$, $(\mathbf{A \times \mathbf{B}) \cdot C}$, $\mathbf{A} \cdot (\mathbf{B} \times \mathbf{C})$, $(\mathbf{A} \times \mathbf{B}) \times \mathbf{C})$, $\mathbf{A} \times (\mathbf{B} \times \mathbf{C})$
\item Demostrar la identidad de Lagrange
\[ (\mathbf{A} \times \mathbf{B}) \cdot (\mathbf{C} \times \mathbf{D}) = (\mathbf{A} \cdot \mathbf{C})(\mathbf{B} \cdot \mathbf{D}) - (\mathbf{A} \cdot \mathbf{D})(\mathbf{B} \cdot \mathbf{C}) \]
\item Demostrar la identidad de Jacobi
\[ \mathbf{A} \times (\mathbf{B} \times \mathbf{C}) + \mathbf{B} \times (\mathbf{C} \times \mathbf{A}) + \mathbf{C} \times (\mathbf{A} \times \mathbf{B}) = 0 \]
\item Una carga $q$ se coloca en las coordenadas cilíndricas $(a, \pi/4, 2a)$. Determina las componentes \emph{cartesianas} del campo electrostático de esta carga en el punto $P$ con coordenadas cilíndricas $(2a, \pi/6, a)$. Escribe tu respuesta en múltiplos de $k_{e} q /a^{2}$. Calcula el potencial electrostático en $P$ y escribe la respuesta en múltiplos de $k_{e} q/a$.
\item Una carga $q$ se está moviendo a velocidad constante $v$ sobre el eje positivo $x$. Otras dos cargas $-q$ y $2q$ se mueven a velocidad constante $v$ y $2v$ a lo largo del eje positivo $y$ y el eje negativo $z$, respectivamente. Suponemos que en $t=0$, la carga $q$ está en el origen, $-q$ está en $(0,a,0)$ y $2q$ está en $(0,0,a)$.
\begin{enumerate}
\item Calcula las componentes cartesianas del campo magnético en el punto $(x,y,z)$ para $t>0$.
\item Calcula las componentes cilíndricas del campo magnético en el punto $(\rho, \varphi, z)$ para $t>0$.
\item Calcula las componentes esféricas del campo magnético en el punto $(r, \theta, \varphi)$ para $t>0$.
\end{enumerate}
\item Las coordenadas elípticas $(u,\theta)$ están dadas por
\[  \begin{split}
x &= a \cosh u \cos \theta, \\
y &= a \sinh u \sin \theta
\end{split} \]
donde $a$ es una constante.
\begin{enumerate}
\item ¿Cuál(es) son las curvas con $u$ constante?
\item ¿Cuál(es) son las curvas con $\theta$ constante?
\item Calcula $\mathbf{\widehat{e}}_{u}$ y $\mathbf{\widehat{e}}_{\theta}$ en términos de vectores cartesianos unitarios, revisa su ortogonalidad.
\end{enumerate}
\item Las coordenadas esferoidales prolatas $(u, \theta, \varphi)$ están dadas por
\[  \begin{split}
x &= a \sinh u \sin \theta \cos \varphi, \\
y &= a \sinh u \sin \theta \sin \varphi, \\
z &= a \cosh u \cos \theta
\end{split} \]
\begin{enumerate}
\item ¿Cuál(es) son las superficies con $u$ constante?
\item ¿Cuál(es) son las sueprficies con $\theta$ constante?
\item Calcula $\mathbf{\widehat{e}}_{u}$,  $\mathbf{\widehat{e}}_{\theta}$ y $\mathbf{\widehat{e}}_{\varphi}$ en términos de vectores cartesianos unitarios, revisa su ortogonalidad mutua.
\end{enumerate}
\item Dado el vector $\mathbf{A} = (x^{2} - y^{2}) \mathbf{i} +  2xy \mathbf{j}$
\begin{enumerate}
\item Calcula $\bm{\nabla} \times \mathbf{A}$.
\item Evalúa $\mathlarger{\iint} (\bm{\nabla} \times \mathbf{A}) \cdot d \sigma $ en un rectángulo en el plano $x-y$ cercado por las líneas $x=0$, $x=a$, $y=0$, $y=b$.
\item Evalúa $\mathlarger{\oint} \mathbf{A} \cdot d \mathbf{r}$ alrededor de la frontera del rectángulo y verifica el teorema de Stokes para este ejemplo.
\end{enumerate}
\item Si la temperatura es $T = x^{2} - xy + z^{2}$ calcula
\begin{enumerate}
\item la dirección del flujo de calor en $(2,1,-1)$.
\item la razón de cambio de la temperatura en la dirección $\mathbf{j}-\mathbf{k}$ en $(2,1,-1)$
\end{enumerate}
\item Demuestra que
\[ \mathbf{F} = y^{2} z \sinh (2xz) \mathbf{i} + 2y \cosh^{2} (xz) \mathbf{j} +  y^{2} \sinh(2xz) \mathbf{k} \]
es conservativa, y encuentra un potencial escalar $\phi$ tal que $\mathbf{F} = - \bm{\nabla} \phi$.
\item Las ecuaciones de Navier-Stokes para el flujo de un fluido incompresible
\[ - \bm{\nabla} \times ( \mathbf{v} \times (\bm{\nabla} \times \mathbf{v} )) =  \dfrac{\eta}{\rho_{0}} \bm{\nabla}^{2} (\bm{\nabla} \times \mathbf{v}) \]
Donde $\eta$ es la viscosidad y $\rho_{0}$ la densidad del fluido. Para un flujo axial dentro de un cilindro, consideremos que la velocidad $\mathbf{v}$ está dada por
\[ \mathbf{v} =  \mathbf{k} v (\rho) \]
Considera que
\[ \bm{\nabla} \times (\mathbf{v} \times (\bm{\nabla} \times \mathbf{v})) = 0 \]
para este valor de $\mathbf{v}$.
\\
Demuestra que
\[ \bm{\nabla}^{2} ( \bm{\nabla} \times \mathbf{v}) = 0  \]
nos lleva a la ecuación diferencial
\[ \dfrac{1}{\rho} \dfrac{d}{d \rho} \left( \rho \dfrac{d^{2} v}{d \rho^{2}} \right) =  \dfrac{1}{\rho} \dfrac{d v}{d \rho} = 0 \]
y que la siguiente expresión es solución de la misma ecuación diferencial
\[ v = v_{0} + a_{2} \rho^{2} \]
\item Un almabre conductor sobre el eje $z$ conduce una corriente $I$. El vector de potencial magnético resultante está dado por
\[ \mathbf{A} = \mathbf{k} \dfrac{\mu I}{2 \pi} ln\left(\dfrac{1}{\rho} \right) \]
Demuestra que el campo de inducción magnético $B$ está dado por
\[ \mathbf{B} = \bm{\varphi}_{0} \dfrac{\mu I}{2 \pi \rho} \]
\item Una fuerza está descrita por
\[ \mathbf{F} = - \mathbf{i} \dfrac{y}{x^{2} + y^{2}} + \mathbf{j} \dfrac{x}{x^{2} + y^{2}} \]
\begin{enumerate}
\item Expresa la fuerza $F$ en coordenadas cilíndricas.
\item En coordenadas cilíndricas, calcula el rotacional de $F$. \label{2}
\item Calcula el trabajo realizado por $F$ al rodear un círculo unitario en el sentido de las manecillas del reloj (en coordenadas cilíndricas) \label{3}
\item ¿Cómo es que son compatibles los resultados de los incisos ($7.2$) y ($7.3$)? 
\end{enumerate}
\item El cálculo de efecto ''pinch'' en magnetohidrodinámica, involucra la evaluación de $(\mathbf{B} \cdot \bm{\nabla}) \mathbf{B}$. Si la inducción magnética $\mathbf{B}$ se toma como $\mathbf{B} = \bm{\varphi}_{0} B_{\varphi} (\rho)$, demostrar que
\[ (\mathbf{B} \cdot \bm{\nabla}) \mathbf{B} = - \dfrac{\bm{\rho_{0}} B_{\varphi}^{2}}{\rho}	 \]
\item Una partícula se mueve en el espacio, calcua las componentes de la velocidad y aceleración en coordenadas esféricas:
\[ \begin{split}
v_{r} &= \dot{r}, \\
v_{\theta} &= r \dot{\theta}, \\
v_{\varphi} &= r \sin \theta \dot{\varphi}, \\
a_{r} &= \ddot{r} - r \dot{\theta}^{2} - r \sin^{2} \theta \dot{\varphi}^{2}, \\
a_{\theta} &= r \ddot{\theta} + 2 \dot{r} \dot{\theta} - r \sin \theta \cos \theta \dot{\varphi}^{2}, \\
a_{\varphi} &= r \sin \theta \ddot{\varphi} + \sin \dot{r} \sin \theta \dot{\varphi} + 2 r \cos \theta \dot{\theta} \dot{\varphi}
\end{split} \]
Considera que
\[ \begin{split}
\mathbf{r}(t) &= \mathbf{r}_{0}(t) r(t) \\
&= [ \mathbf{i} \sin \theta (t) \cos \varphi (t) + \mathbf{j} \sin \theta (t) \sin \varphi (t) + \mathbf{k} \cos \theta (t) ] r(t)
\end{split} \]
\item Una partícula $m$ se mueve en respuesta a una fuerza central de acuerdo a la segunda ley de Newton
\[ m \mathbf{\ddot{r}} =  \mathbf{r}_{0} f(\mathbf{r}) \]
Demuestra que $\mathbf{r} \times \mathbf{\dot{r}} =  c$, es una constante y que la interpretación geométrica, nos conduce a la segunda ley de Kepler.
\item Se define el operador de momento angular tomado de la mecánica cuántica por $\mathbf{L} =  -i (\mathbf{r} \times \bm{\nabla})$, ahora demuestra que
\begin{enumerate}
\item $L_{x} + i L_{y} = e^{i \varphi} \left( \dfrac{\partial}{\partial \theta} + i \cot \theta \dfrac{\partial}{\partial \varphi} \right)$
\item $L_{x} - i L_{y} = - e^{-i\varphi} \left( \dfrac{\partial}{\partial \theta} - i \cot \theta \dfrac{\partial}{\partial \varphi} \right) $
\end{enumerate}
A estas expresiones se les conoce como operadores de ascenso y descenso, respectivamente.
\item De la expresión
\[ \bm{\nabla} \psi = \mathbf{r}_{0} \dfrac{\partial \psi}{\partial r} + \bm{\theta}_{0} \dfrac{1}{r} \dfrac{\partial \psi}{\partial \theta} + \bm{\varphi}_{0} \dfrac{1}{r \sin \theta} \dfrac{\partial \psi}{\partial \varphi} \]
demostrar que
\begin{enumerate}
\item \[ \mathbf{L} = -i (\mathbf{r} \times \bm{\nabla}) = i \left( \bm{\theta}_{0} \dfrac{1}{\sin \theta} \dfrac{\partial}{\partial \varphi} - \bm{\varphi}_{0} \dfrac{\partial}{\partial \theta} \right)\]
\item Resolviendo para $\bm{\theta}_{0}$ y para $\bm{\varphi}_{0}$ en sus componentes cartesianas, expresa $L_{x}$, $L_{y}$ y $L_{z}$ en términos de $\theta,\varphi$ y sus derivadas.
\item De $L^{2} =  L_{x}^{2} + L_{y}^{2} + L_{z}^{2}$ demostrar que
\[ \begin{split}
\mathbf{L}^{2} &= - \dfrac{1}{\sin \theta} \dfrac{\partial}{\partial \theta} \left( \sin \theta \dfrac{\partial}{\partial \theta} \right) - \dfrac{1}{\sin^{2} \theta} \dfrac{\partial^{2}}{\partial \varphi^{2}} \\
&= -r^{2} \bm{\nabla}^{2} + \dfrac{\partial}{\partial r} \left( r^{2} \dfrac{\partial}{\partial r} \right)
\end{split} \]
\end{enumerate}
\item Un cierto campo de fuerza está dado por
\[ \mathbf{F} = \mathbf{r}_{0} \dfrac{2 P \cos \theta}{r^{3}} + \bm{\theta}_{0} \dfrac{P}{r^{3}} \sin \theta, \hspace{1.5cm} r \geq P/2 \]
en coordenadas polares.
\begin{enumerate}
\item Revisa $\bm{\nabla} \times \mathbf{F}$ para checar si existe un potencial.
\item Calcular $\mathlarger{\oint} \mathbf{F} \cdot d \bm{\lambda}$ para un círculo unitario en el plano $\theta = \pi/2$.\\
¿Qué nos dice esto sobre la fuerza?¿Es conservativa o no conservativa?
\item Si considera que $\mathbf{F}$ se puede describir por $\mathbf{F} = - \bm{\nabla}\psi$, encuentra $\psi$. De otra manera argumenta que no es posible que un potencial exista.
\end{enumerate}
\item Un vector de potencial magnético está dado por
\[ \mathbf{A} = \dfrac{\mu_{0}}{4 \pi} \dfrac{\mathbf{m} \times \mathbf{r}}{r^{3}} \]
Demuestra que esto nos conduce a un campo de inducción magnética $\mathbf{B}$ de un dipolo magnético, de momento $\mathbf{m}$.
\end{enumerate}
\end{document}