\documentclass[12pt]{article}
\usepackage[utf8]{inputenc}
\usepackage[spanish,es-lcroman, es-tabla]{babel}
\usepackage[autostyle,spanish=mexican]{csquotes}
\usepackage{amsmath}
\usepackage{amssymb}
\usepackage{nccmath}
\numberwithin{equation}{section}
\usepackage{amsthm}
\usepackage{graphicx}
\usepackage{epstopdf}
\DeclareGraphicsExtensions{.pdf,.png,.jpg,.eps}
\usepackage{color}
\usepackage{float}
\usepackage{multicol}
\usepackage{enumerate}
\usepackage[shortlabels]{enumitem}
\usepackage{anyfontsize}
\usepackage{anysize}
\usepackage{array}
\usepackage{multirow}
\usepackage{enumitem}
\usepackage{cancel}
\usepackage{tikz}
\usepackage{circuitikz}
\usepackage{tikz-3dplot}
\usetikzlibrary{babel}
\usetikzlibrary{shapes}
\usepackage{bm}
\usepackage{mathtools}
\usepackage{esvect}
\usepackage{hyperref}
\usepackage{relsize}
\usepackage{siunitx}
\usepackage{physics}
%\usepackage{biblatex}
\usepackage{standalone}
\usepackage{mathrsfs}
\usepackage{bigints}
\usepackage{bookmark}
\spanishdecimal{.}

\setlist[enumerate]{itemsep=0mm}

\renewcommand{\baselinestretch}{1.5}

\let\oldbibliography\thebibliography

\renewcommand{\thebibliography}[1]{\oldbibliography{#1}

\setlength{\itemsep}{0pt}}
%\marginsize{1.5cm}{1.5cm}{2cm}{2cm}


\newtheorem{defi}{{\it Definición}}[section]
\newtheorem{teo}{{\it Teorema}}[section]
\newtheorem{ejemplo}{{\it Ejemplo}}[section]
\newtheorem{propiedad}{{\it Propiedad}}[section]
\newtheorem{lema}{{\it Lema}}[section]

\setlength{\jot}{12pt}
\title{Funciones expresadas por una hipergeométrica \\ {\large Tema 5 - Matemáticas Avanzadas de la Física}\vspace{-1.5\baselineskip}}
\author{}
\date{}
\begin{document}
\newgeometry{left=1cm,right=1cm, top=1.5cm, bottom=1.5cm}
\maketitle
\fontsize{14}{14}\selectfont
\section{Funciones especiales en términos de la hipergeométrica.}
Muchas funciones especiales puedes expresarse en términos de estas nuevas funciones, como por ejemplo:
\begin{align*}
P_{\ell} (x) &= {}_{2}F_{1} \left( -\ell , \ell + 1; 1; \dfrac{1 - x}{2} \right) \\[1em]
P_{\ell}^{m} &= \dfrac{(\ell + m)!}{(\ell - m)!} \, \dfrac{(1 - x^{2})^{m/2}}{2^{m} \, m!} \, {}_{2}F_{1} \left( m - \ell, m + \ell + 1; m + 1; \dfrac{1 - x}{2} \right) \\[1em]
J_{n} (x) &= \dfrac{e^{-i x}}{n!} \, \left( \dfrac{x}{2} \right)^{n} \, {}_1 F_{1} \left( n + \dfrac{1}{2}; 2 \, n; 2 \, n + 1; 2 \, i \, x \right) \\[1em]
H_{2 n} (x) &= (-)^{n} \, \dfrac{(2 \, n)!}{n!} \, {}_1 F_{1} \left( -n; \dfrac{1}{2}; x^{2} \right) \\[1em]
H_{2 n+1} (x) &= x \, (-)^{n} \, \dfrac{2 (2 \, n + 1)!}{n!} \, {}_1 F_{1} \left( -n; \dfrac{3}{2}; x^{2}\right) \\[1em]
L_{n} (x) &= {}_{1} F_{1} (-n; 1; x)\\[1em]
L_{n}^{k} (x) &= \dfrac{\Gamma (n +  k + 1)}{n! \, \Gamma (k + 1)} \, {}_{1} F_{1} (-n; k + 1; x) \\[1em]
T_{\ell} (x) &= {}_{2} F_{1} \left( -\ell, \ell; \dfrac{1}{2}; \dfrac{1 - x}{2} \right) \\[1em]
U_{\ell}(x) &= \ell \,\sqrt{1- x^{2}} \, {}_{2} F_{1} \left( -\ell + 1, \ell+1; \dfrac{3}{2}; \dfrac{1-x}{2} \right)
\end{align*}
\section{Funciones en términos de la hipergeométrica.}
Las funciones hipergeométrica son útiles para expresar otras funciones:
\begin{align*}
\dfrac{\ln (1 + x)}{x} &= {}_{2} F_{1}(1, 1; 2; -x) \\[1em]
(1 + x)^{n} &= {}_{2} F_{1}(-n, b; b; -x) \\[1em]
- \dfrac{\ln (1 + x)}{x} &= {}_{2} F_{1}(1, 1; 2; x) \\[1em]
\dfrac{1}{2 \, x} \, \ln \left( \dfrac{1 + x}{1 - x} \right)&= {}_{2} F_{1}\left( \dfrac{1}{2}, 1; \dfrac{3}{2}; x^{2} \right) \\[1em]
\dfrac{\arcsin x}{x} &= {}_{2} F_{1} \left( \dfrac{1}{2}, \dfrac{1}{2}; \dfrac{3}{2}, x^{2} \right) \\[1em]
\dfrac{\arctan x}{x} &= {}_{2} F_{1} \left( \dfrac{1}{2}, 1; \dfrac{3}{2}, -x^{2} \right) \\[1em]
\cos (2 \, a \, x) &= {}_{2} F_{1} \left( -a \, a, \dfrac{1}{2}; \sin^{2} x \right)
\end{align*}
\end{document}