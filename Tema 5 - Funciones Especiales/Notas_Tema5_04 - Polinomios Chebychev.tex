\documentclass[12pt]{article}
\usepackage[utf8]{inputenc}
\usepackage[spanish,es-lcroman, es-tabla]{babel}
\usepackage[autostyle,spanish=mexican]{csquotes}
\usepackage{amsmath}
\usepackage{amssymb}
\usepackage{nccmath}
\numberwithin{equation}{section}
\usepackage{amsthm}
\usepackage{graphicx}
\usepackage{epstopdf}
\DeclareGraphicsExtensions{.pdf,.png,.jpg,.eps}
\usepackage{color}
\usepackage{float}
\usepackage{multicol}
\usepackage{enumerate}
\usepackage[shortlabels]{enumitem}
\usepackage{anyfontsize}
\usepackage{anysize}
\usepackage{array}
\usepackage{multirow}
\usepackage{enumitem}
\usepackage{cancel}
\usepackage{tikz}
\usepackage{circuitikz}
\usepackage{tikz-3dplot}
\usetikzlibrary{babel}
\usetikzlibrary{shapes}
\usepackage{bm}
\usepackage{mathtools}
\usepackage{esvect}
\usepackage{hyperref}
\usepackage{relsize}
\usepackage{siunitx}
\usepackage{physics}
%\usepackage{biblatex}
\usepackage{standalone}
\usepackage{mathrsfs}
\usepackage{bigints}
\usepackage{bookmark}
\spanishdecimal{.}

\setlist[enumerate]{itemsep=0mm}

\renewcommand{\baselinestretch}{1.5}

\let\oldbibliography\thebibliography

\renewcommand{\thebibliography}[1]{\oldbibliography{#1}

\setlength{\itemsep}{0pt}}
%\marginsize{1.5cm}{1.5cm}{2cm}{2cm}


\newtheorem{defi}{{\it Definición}}[section]
\newtheorem{teo}{{\it Teorema}}[section]
\newtheorem{ejemplo}{{\it Ejemplo}}[section]
\newtheorem{propiedad}{{\it Propiedad}}[section]
\newtheorem{lema}{{\it Lema}}[section]

\usepackage{mathrsfs}
\usepackage{bigints}
\usepackage{array}
\newcolumntype{C}[1]{>{\centering\let\newline\\\arraybackslash\hspace{0pt}}m{#1}}
\usepackage{booktabs}
\usepackage{caption}
\renewcommand\spanishtablename{Tabla}
%\usepackage{tabularx}
\spanishdecimal{.}
%\usepackage{enumerate}
%\author{M. en C. Gustavo Contreras Mayén. \texttt{curso.fisica.comp@gmail.com}}
\title{Polinomios de Chebychev \\ {\large Matemáticas Avanzadas de la Física}\vspace{-1.5\baselineskip}}
\date{ }
\begin{document}
%\renewcommand\theenumii{\arabic{theenumii.enumii}}
\renewcommand\labelenumii{\theenumi.{\arabic{enumii}}}
\maketitle
\fontsize{14}{14}\selectfont
%Referencia Sepulveda 8.7 Ecuación de Gegenbauer
\section{Ecuación de Gegenbauer.}
Esta ecuación tiene la siguiente forma
\begin{align*}
(1 - x^{2}) \, \ddot{y} - (2 \, \lambda + 1) \, x \,\dot{y} + \ell (\ell + 2 \, \lambda) \, y = 0
\end{align*}
La forma de Sturm-Lioville puede obtenerse, tomando en cuenta que
\begin{align*}
p &= (1 -x^{2})^{\lambda - 1/2} \\
q &= (1- x^{2})^{\lambda + 1/2}
\end{align*}
se tiene entonces
\begin{align*}
\dv{x} \left[ (1 - x^{2})^{\lambda+1/2} \, \dot{y} \right] + \ell (\ell + 2 \, \lambda) (1 - x^{2})^{\lambda-1/2} \, y = 0
\end{align*}
Las soluciones a esta ecuación forma una base ortogonal si $[ q \, W]\eval_{x=a}^{x=b} = 0$, es decir, si
\begin{align*}
[(1 - x^{2})^{\lambda+1/2} \, W]\eval_{x=a}^{x=b} = 0
\end{align*}
lo cual se cumple si $a = -1, b = 1$. A estas soluciones les llamaremos $T_{\ell}^{\lambda} (x)$, son ortogonales, con una función de peso $(1 - x^{2})^{\lambda-1/2}$, con $\ell$ entero:
\begin{align*}
\setlength{\fboxsep}{3\fboxsep}\boxed{\int_{-1}^{1} (1 - x^{2})^{\lambda-1/2} \, T_{\ell}^{\lambda} (x) \, T_{m}^{\lambda} (x) \, \dd{x}
 = \dfrac{2^{1-2 \lambda} \, \Gamma (\ell + 2 \, \lambda)}{(\ell + \lambda) [ \Gamma (\lambda + 1)^{2}])} \, \delta_{\ell m}}
\end{align*}
Los polinomios de Gegenbauer (también llamados \emph{ultraesféricos}) $T_{\ell}^{\lambda}$ de grado $\ell$ son los coeficientes $t^{\ell}$ en la expansión en series de la función
\begin{align*}
g(x, t) = (1 - 2 \, x \, t + t^{2})^{-\lambda} = \sum_{\ell=0}^{\infty} T_{\ell}^{\lambda} (x) \, t^{\ell}
\end{align*}
Así, lo polinomios $T_{\ell}^{\lambda} (x)$ son una generalización de los polinomios de Legendre $P_{\ell}(x)$ que se presentan cuando $\lambda = 1/2$.
\par
Es posible demostrar que
\begin{equation}
T_{\ell}^{\lambda} (x) = \sum_{r=0}^{n/2} \dfrac{\Gamma (n - r + \lambda)}{\Gamma (\lambda) \, r! \, (n - 2 \, r)!} \, (2 \, x)^{n-2r}
\label{eq:ecuacion_08_86}
\end{equation}
 \section{Polinomios de Chebychev.}
 Si en la ecuación de Gegenbauer, hacemos que $\lambda = 0$, se obtiene la ecuación de \textbf{Chebyshev I}:
\begin{equation}
(1 - x^{2}) \, \ddot{y} - x \, \dot{y} + \ell \, y = 0
\label{eq:ecuacion_08_87}
\end{equation}
La relevancia de los polinomios de Chebychev radica en el análisis numérico.
\par
La solución de la ec. de Chebychev (\ref{eq:ecuacion_08_87}), a partir de la ec. (\ref{eq:ecuacion_08_86}) es:
\begin{align*}
T_{\ell}^{0} (x) = T_{\ell} (x) &= \sum_{r=0}^{N} \dfrac{(-)^{r} \, \Gamma (\ell - r)}{r! \, (\ell - 2 \, r)!} \, (2 \, x)^{\ell - 2r} \\
&= \sum_{r=0}^{N} \dfrac{(-)^{r} \, (\ell - r - 1)!}{r! \, (\ell - 2 \, r)!} \, (2 \, x)^{\ell - 2r}
\end{align*}
donde $N=\ell/2$ si $\ell=$ par, y $N=(\ell-1)/2$ si $N=$ impar.
\par
Una forma equivalente de definir los polinomios de Chebyshev $T_{\ell} (x)$ y $U_{\ell} (x)$ de primera y segunda clase, es la siguiente:
\begin{align*}
T_{\ell} (x) &= \cos (\ell \, \arccos x) \\
U_{\ell} (x) &= \sin (\ell \, \arcsin x) \\
\end{align*}
Que las funciones $T_{\ell} (x)$ y $U_{\ell} (x)$ son soluciones independientes, se sigue de observar que:
\begin{enumerate}[label=\alph*)]
\item Son funciones \emph{seno} y \emph{coseno}.
\item $T_{\ell}(1) = 1$ mientras que $U_{\ell}(1) = 0$, por lo que $U_{\ell}(x)$ \textbf{no puede ser múltiplo constante} de $T_{\ell}(x)$.
\end{enumerate}
La solución $U_{\ell}(x)$ también puede escribirse de la forma:
\begin{align*}
U_{\ell} (x) = \sqrt{1 - x^{2}} \, \sum_{r=0}^{N} (-)^{r} \, \dfrac{(\ell - r - 1)!}{r! \, (\ell -2 \, r - 1)!} \, (2 \, x)^{\ell-2r-1}
\end{align*}
\par
Las primeras funciones de Chebyshev tienen la siguiente expresión:
\begin{table}[H]
\centering
\begin{tabular}{>{\raggedright\arraybackslash}p{5cm} >{\raggedright\arraybackslash}p{5cm}}
$T_{0}=1$ & $U_{0}(x)=0$  \\
$T_{1}(x) = x$ & $U_{1}(x) = \sqrt{1 - x^{2}}$ \\
$T_{2}(x) = 2 \, x^{2} - 1$ & $U_{2}(x) = 2 \, x \, \sqrt{1 - x^{2}}$ \\
$T_{3}(x) = 4 \, x^{3} - 3 \, x$ & $U_{3}(x) = (4 \, x^{2} - 1) \, \sqrt{1 - x^{2}}$ \\ 
\end{tabular}
\end{table}
Las funciones generatrices de los polinomios de Chebyshev son:
\begin{align}
\begin{aligned}
g(x, t) &= \dfrac{1 - t^{2}}{1 - 2 \, x \, t + t^{2}} = \sum_{\ell=0}^{\infty} \varepsilon_{\ell} \, T_{\ell} (x) \, t^{\ell} \\[0.5em]
g(x, t) &= \dfrac{\sqrt{1 - t^{2}}}{1 - 2 \, x \, t + t^{2}} = \sum_{\ell=0}^{\infty} U_{\ell} (x) \, t^{\ell}
\end{aligned}
\label{eq:ecuacion_08_88}
\end{align}
donde los valores de $\varepsilon_{0} = 1, \varepsilon_{2} = \varepsilon_{3} = \ldots = 2$.
\par
Las condiciones de ortogonalidad, son de la forma:
\begin{align*}
\int_{-1}^{1} \dfrac{T_{\ell} (x) \, T_{m} (x)}{\sqrt{1 - x^{2}}} \, \dd{x} = 
\begin{cases}
0 & \ell \neq m \\
\pi/2 & \ell = m \neq 0 \\
\pi & \ell = m = 0
\end{cases}
\end{align*}
Para los polinomios de Chebyshev de segunda clase:
\begin{align*}
\int_{-1}^{1} \dfrac{U_{\ell} (x) \, U_{m} (x)}{\sqrt{1 - x^{2}}} \, \dd{x} = 
\begin{cases}
0     & \ell \neq m     \\
\pi/2 & \ell = m \neq 0 \\
0   & \ell = m = 0    
\end{cases}
\end{align*}
Algunos valores especiales:
\begin{table}[H]
\centering
\begin{tabular}{>{\raggedright\arraybackslash}p{5cm} >{\raggedright\arraybackslash}p{5cm}}
$T_{\ell} (1) = 1$ & $U_{\ell}(1)=0$ \\
$T(-1) = (-1)^{\ell}$ & $U_{\ell}(-1) = 0$ \\
$T_{2 \ell}(0) = (-1)^{\ell}$ & $U_{2 \ell}(0) = 0$ \\
$T_{2 \ell+1}(0) = 0$ & $U_{2 \ell+1}(0) = (-1)^{\ell}$ \\ 
\end{tabular}
\end{table}
A continuación se muestran algunas relaciones de recurrencia para $T_{\ell}(x)$, y que son válidas también para $U_{\ell}(x)$:
\begin{align*}
T_{\ell+1}(x) &= 2 \, x \, T_{\ell} (x) - T_{\ell-1} (x) \\[1em]
(1 - x^{2}) \, \dot{T}_{\ell}(x) &= -\ell \, x \, T_{\ell}(x) + \ell \, T_{\ell-1} (x) \\[1em]
T_{\ell-1} &= \left[ x + \dfrac{1 - x^{2}}{\ell} \, \dv{x} \right] \, T_{\ell} \\[1em]
T_{\ell+1} &= \left[ x - \dfrac{1 - x^{2}}{\ell} \, \dv{x} \right] \, T_{\ell} \\[1em]
\end{align*}
\subsection{Polinomios de Chebyshev de tipo II.}
La ecuación de Chebyshev de tipo II, satisface la siguiente expresión:
\begin{equation}
(1 - x^{2}) \, \ddot{V}_{\ell} - 3 \, x \, \dot{V}_{\ell} + \ell (\ell + 2) \, V_{\ell} = 0
\label{eq:ecuacion_08_89}
\end{equation}
y es una ecuación de la familia de Chebyshev, en tanto que:
\begin{equation}
V_{\ell} (x) = \dfrac{U_{\ell+1}(x)}{\sqrt{1 - x^{2}}}
\label{eq:ecuacion_08_90}
\end{equation}
\section{Polinomios de Jacobi.}
Los polinomios y la ecuación de Legendre pueden ser generalizados aún más, mediante las funciones
\begin{align*}
y(x) = P_{\ell}^{\alpha, \beta} (x)
\end{align*}
que satisfacen la \textbf{ecuación de Jacobi.}
\begin{align*}
(1 - x^{2}) \, \ddot{y} + \left\{ \beta - \alpha - (\alpha + \beta + 2) \, x \right\} \, \dot{y} + \ell (\ell + \alpha + \beta + 1) \, y = 0
\end{align*}
Nótese que si $\alpha = \beta = 0$, se obtiene
\begin{align*}
y = P_{\ell}^{0,0} (x) = P_{\ell}(x)
\end{align*}
Las funciones de Gegenbauer, Chebyshev, Laguerre y Hermite se expresan mediante los polinomios de Jacobi de la forma:
\begin{align*}
T_{\ell}^{\lambda} &= \dfrac{\Gamma (\lambda + 1/2) \, \Gamma (\lambda + 2 \, \lambda)}{\Gamma (2 \, \lambda) \, \Gamma (\ell + \lambda) + 1/2)} \, P_{\ell}^{\lambda-1/2, \lambda+1/2} (x) \\[1em]
T_{\ell}(x) &= \dfrac{n}{2} \, \lim_{\lambda \to 0} \left[ \dfrac{T_{\ell}^{\lambda}(x)}{\lambda} \right], \hspace{1cm} n \geq 1 \\[1em]
U_{\ell} &= \sqrt{1 - x^{2}} \, T_{\ell-1}^{1} (x) \\[1em]
L_{n}^{\alpha} (x) &= \lim_{\beta \to \infty} \left[ P_{n}^{\alpha, \beta} (1 - 2 \, x / \beta) \right] \\[1em]
H_{n}(x) &= n! \, \lim_{\lambda \to \infty} \left[ \lambda^{-n/2} \, T_{n}^{\lambda} \left( \dfrac{x}{\sqrt{\lambda}} \right) \right]
\end{align*}
Se tiene entonces que de la ecuación de Jacobi, surgen las de Gegenbauer, Laguerre y Hermite; mientras que de la ecuación de Gegenbauer, surgen las de Chebyshev y Legendre ordinaria.
\par
La fórmula de Rodrigues es
\[ P_{\ell}^{\alpha \beta} (x) = \dfrac{(-)^{n}}{2^{n} \, n!} \, (1 - x)^{-\alpha} \, (1 + x)^{- \beta} \, \dv[n]{x} \left[ (1 -x)^{\alpha+n} \, (1+ x)^{\beta+n}  \right] \]
que equivale a la expansión en series:
\begin{align*}
P_{\ell}^{\alpha \beta} (x) = \sum_{r=0}^{\ell} &\dfrac{\Gamma (\ell + \alpha + 1)}{\Gamma (\alpha + r + 1) \, \Gamma (\alpha + \beta - r + 1) \, (\ell -r)! \, r!} \times \\
&{} \times \left( \dfrac{x - 1}{2} \right)^{r} \, \left( \dfrac{x+1}{2} \right)^{\ell-r}
\end{align*}
La condición de ortogonalidad es:
\begin{align*}
\int_{-1}^{1} &(1 - x)^{\alpha} (1 + x)^{\beta} \, P_{\ell}^{\alpha, \beta} (x) \, P_{m}^{\alpha, \beta} (x) \, \dd{x} \\
&= \dfrac{2^{\alpha+\beta+1} \Gamma (\ell + \alpha + 1) \, \Gamma (\ell + \beta + 1)}{(2 \, \ell + \alpha +\beta + 1) \, \ell! \, \Gamma (\ell + \alpha + \beta + 1)} \, \delta_{\ell  m}
\end{align*}
Relaciones de recurrencia:
\begin{align*}
\dot{P}_{\ell}^{\alpha, \beta} (x) &= \dfrac{1}{2} (1 + \alpha + \beta + \ell) \, P_{\ell-1}^{\alpha+1, \beta+1}(x) \\[1em]
(x + 1) \, \dot{P}_{\ell}^{\alpha, \beta} (x) &= \ell \, P_{\ell}^{\alpha, \beta} (x) + (\beta + \ell) P_{\ell-1}^{\alpha+1, \beta} (x) \\[1em]
(x - 1) \, \dot{P}_{\ell}^{\alpha, \beta} (x) &= \ell \, P_{\ell}^{\alpha, \beta} (x) - (\alpha + \ell) P_{\ell-1}^{\alpha, \beta+1} (x) \\[1em]
\dot{P}_{\ell}^{\alpha, \beta} (x) &= \dfrac{1}{2} \, P_{\ell-1}^{\alpha+1, \beta}(x) + (\alpha + \ell) \, P_{\ell-1}^{\alpha, \beta+1}
\end{align*}
\section{La ecuación hipergeométrica.}
Se definen los símbolos de Pochhammer $(\alpha)_{n}$ como:
\[ (\alpha)_{n} = \alpha \, (\alpha + 1) \ldots (\alpha + n -1) = \dfrac{\Gamma (\alpha + n)}{\Gamma (\alpha)}, \hspace{1.5cm} (\alpha)_{0} = 1 \]
donde $n$ es un entero positivo.
\par
Se define la \textbf{función hipergeométrica general} de la forma:
\begin{equation}
{}_{m}F_{n} (\alpha_{1}, \alpha_{2}, \ldots, \alpha_{m}; \beta_{1}, \beta_{2}, \ldots, \beta_{n},; x) = \sum_{r=0}^{\infty} \dfrac{(\alpha_{1})_{r}, (\alpha_{2})_{r}, \ldots, (\alpha_{m})_{r}}{(\beta_{1})_{r}, (\beta_{2})_{r}, \ldots, (\beta_{n})_{r}} \, x^{r}
\label{eq:ecuacion_08_91}
\end{equation}
Muchas funciones especiales puedes expresarse en términos de estas nuevas funciones, como por ejemplo:
\begin{align*}
P_{\ell} (x) &= {}_{2}F_{1} \left( -\ell , \ell + 1; 1; \dfrac{1 - x}{2} \right) \\[1em]
P_{m}^{\ell} &= \dfrac{(\ell + m)!}{(\ell - m)!} \, \dfrac{(1 - x^{2})^{m/2}}{2^{m} \, m!} \, {}_{2}F_{1} \left( m - \ell, m + \ell + 1; m + 1; \dfrac{1 - x}{2} \right) \\[1em]
J_{n} (x) &= \dfrac{e^{-i x}}{n!} \, \left( \dfrac{x}{2} \right)^{n} \, {}_1 F_{1} \left( n + \dfrac{1}{2}; 2 \, n; 2 \, n + 1; 2 \, i \, x \right) \\[1em]
H_{2 n} (x) &= (-)^{n} \, \dfrac{(2 \, n)!}{n!} \, {}_1 F_{1} \left( -n; \dfrac{1}{2}; x^{2} \right) \\[1em]
H_{2 n+1} (x) &= x \, (-)^{n} \, \dfrac{2 (2 \, n + 1)!}{n!} \, {}_1 F_{1} \left( -n; \dfrac{3}{2}; x^{2}\right) \\[1em]
L_{n} (x) &= {}_{1} F_{1} (-n; 1; x)\\[1em]
L_{n}^{k} (x) &= \dfrac{\Gamma (n +  k + 1)}{n! \, \Gamma (k + 1)} \, {}_{1} F_{1} (-n; k + 1; x) \\[1em]
T_{\ell} (x) &= {}_{2} F_{1} \left( -\ell, \ell; \dfrac{1}{2}; \dfrac{1 - x}{2} \right) \\[1em]
U_{\ell}(x) &= \ell \,\sqrt{1- x^{2}} \, {}_{2} F_{1} \left( -\ell + 1, \ell+1; \dfrac{3}{2}; \dfrac{1-x}{2} \right)
\end{align*}
\section{La ecuación hipergeométrica (de Gauss).}
La ecuación hipergeométrica o ecuación de Gauss, tiene la forma
\begin{align*}
x \, (1 - x) \, \ddot{y} + [c - (a + b + 1) \, x] \, \dot{y} - a \, b \, y = 0
\end{align*}
Expresada en la forma de Sturm-Liouville, resulta ser
\begin{align*}
\dv{x} \left[ (1 - x^{2})^{a+b-c+1} \, x^{c} \, \dot{y} \right] - a \, b \, x^{c-1} \, (1 - x)^{a+b-c} \, y = 0
\end{align*}
donde
\begin{align*}
p (x) &= x^{c-1} \, (1 - x)^{a+b-c} \\
q (x) &= x^{c} \, (1 - x)^{a+b-c+1}
\end{align*}
de tal manera que la solución forma una base ortogonal.
\par
Esta ecuación tiene tres puntos singulares: $x = 0, 1, \infty$. La primera solución, conocida como \emph{serie hipergeométrica}, tiene la forma de la ec. (\ref{eq:ecuacion_08_91}) con $m=2, n=1, (\alpha_{1})_{r}, (\alpha_{2})_{r} = (b)_{r}, (\beta_{1})_{r} = (c)_{r}$:
\begin{align*}
y (x) &= {}_{2} F_{1} (a, b, c ; x) = 1 + \dfrac{a \, b \, x}{c} + \dfrac{a (a + 1) \, b (b + 1)}{2! \, c \, (c + 1)} \, x^{2} + \\
&+ \dfrac{a (a + 1)(a + 2) \, b (b + 1) (b + 2)}{3! \, c \, (c + 1)(c + 2)} \, x^{3} + \ldots + \dfrac{(a)_{r} \, (b)_{r}}{r! \, (c)_{r}} \, x^{r} + \ldots \\
&= \sum_{r=0}^{\infty} \dfrac{(a)_{r} \, (b)_{r}}{r! \, (c)_{r}} \, x^{r} \hspace{2cm} c \neq 0, -1, -2, -3, \ldots
\end{align*}
Si $c$ es no entero, se tiene una segunda solución independiente:
\begin{align*}
y(x) = {}_{2} F_{1} (a + 1 - c, b + 1 - c, 2 - c; x) \, x^{1-c}, \hspace{1.5cm} c \neq 2, 3, 4, \ldots
\end{align*}
Si $c$ es entero, las dos soluciones coinciden o (si además $a$ o $b$ es entero) una de ellas diverge. En tal caso, la segunda solución incluye un término logarítmico.
\par
El caso especial $a = b = 1$ genera la serie geométrica $\displaystyle{\sum_{n=0}^{\infty} x^{n}}$. Por esta razón, la solución general se llama \emph{hipergeométrica}; fue Gauss quien realizó un estudio sistemático de esta función.
\par
Si $a$ o $b$ es cero o entero negativo, la serie infinita se convierte en un polinomio.
\par
La serie converge para $-1 < x \leq 1$ si $c > a + b$ y converge en $x = -1$ si $c > a + b - 1$.
\par
Para $c = -n$ con $n = 0, 1, 2, \ldots$ la serie es indeterminada si $a \neq -m$ y $b \neq -m$ con $m < n, m$ entero positivo. Excluyendo estos valores de $a, b, c$, la serie converge para $-1 < x < 1$ si cumple las siguientes reglas:
\begin{enumerate}[label=\alph*)]
\item Si $a + b - c > 1$, la serie converge en $x = 1$.
\item Si $a + b - c < 0$, la serie converge absolutamente en $x = 1$.
\item Si Si $a + b - c \geq 1$, la serie diverge en $x = 1$.
\end{enumerate}
Las funciones hipergeométricas son útiles para expresar otras funciones:
\begin{align*}
\dfrac{\ln (1 + x)}{x} &= {}_{2} F_{1}(1, 1; 2; -x) \\[1em]
(1 + x)^{n} &= {}_{2} F_{1}(-n, b; b; -x) \\[1em]
- \dfrac{\ln (1 + x)}{x} &= {}_{2} F_{1}(1, 1; 2; x) \\[1em]
\dfrac{1}{2 \, x} \, \ln \left( \dfrac{1 + x}{1 - x} \right)&= {}_{2} F_{1}\left( \dfrac{1}{2}, 1; \dfrac{3}{2}; x^{2} \right) \\[1em]
\dfrac{\arcsin x}{x} &= {}_{2} F_{1} \left( \dfrac{1}{2}, \dfrac{1}{2}; \dfrac{3}{2}, x^{2} \right) \\[1em]
\dfrac{\arctan x}{x} &= {}_{2} F_{1} \left( \dfrac{1}{2}, 1; \dfrac{3}{2}, -x^{2} \right)
\end{align*}
así como las integrales elípticas:
\begin{align*}
\int_{0}^{\pi/2} (1 - k^{2} \, \sin^{2} \theta)^{-1/2} \, \dd \theta = \dfrac{\pi}{2} \, {}_{2} F_{1} \left( \dfrac{1}{2}, \dfrac{1}{2}; 1; k^{2}  \right) \\[1em]
\int_{0}^{\pi/2} (1 - k^{2} \, \sin^{2} \theta)^{1/2} \, \dd \theta = \dfrac{\pi}{2} \, {}_{2} F_{1} \left( \dfrac{1}{2}, -\dfrac{1}{2}; 1; k^{2}  \right)
\end{align*}
\section{La ecuación hipergeométrica confluente.}
La ecuación hipergeométrica confluente (o ecuación de Kummer) es:
\begin{align*}
x^{2} \, \ddot{y} + (\beta - x) \, \dot{y} - \alpha \, y = 0
\end{align*}
tiene por soluciones:
\begin{align*}
{}_{1} F_{1} (\alpha; \beta; x)
\end{align*}
Si $\beta$ es no entero, la segunda solución independiente está dada por
\begin{align*}
x^{1-\beta} \, {}_{1} F_{1} (\alpha - \beta + 1; 2 - \beta; x)
\end{align*}
%Referencia. Seaborn. Hypergeometric functions and their applications. 2.5 The simple pendulum
\subsection{Ejemplo: El péndulo simple.}
Un péndulo simple consiste de un cuerpo de masa $m$ atado a una cuerda sin masa de longitud $L$. El otro extremo de la cuerda está fijo en un punto de tal manera que el sistema puede moverse bajo la acción de la gravedad como se muestra en la siguiente figura:
\begin{figure}[H]
    \centering
    \includestandalone{Figuras/pendulo_01}
    \caption{Péndulo simple.}
\end{figure}
Las fuerzas que actúan sobre la masa $m$, son la tensión $T$ de la cuerda y el peso $mg$; usando la segunda ley de Newton, tenemos que
\begin{align}
\text{Centrípeta} \hspace{0.5cm} T - m \, g \, \cos \theta &= \dfrac{m \, v^{2}}{L}  \label{eq:ecuacion_pendulo_01} \\
\text{Tangencial} \hspace{0.5cm} -m \, g \, \sin \theta &= m \dv[2]{t} (L \, \theta) \label{eq:ecuacion_pendulo_02}
\end{align}
Considerando la ecuación tangencial (\ref{eq:ecuacion_pendulo_02}) y luego de acomodar términos, tenemos que
\begin{equation} \dv[2]{\theta}{t} + \dfrac{g}{L} \, \sin \theta = 0
\label{eq:ecuacion_pendulo_03}
\end{equation}
La aproximación usual para oscilaciones de pequeña amplitud es tomar el término principal en la expansión de la serie del seno, en cuyo caso el movimiento descrito es el de un oscilador armónico simple y el período es
\begin{align*}
T = 2 \, \pi \,\sqrt{\dfrac{L}{g}}
\end{align*}
\par
Para obtener una solución exacta de la ec. (\ref{eq:ecuacion_pendulo_03}), se multiplica por $\displaystyle{\dv{\theta}{t}}$, para obtener
\begin{equation}
\dv{t} \left[ \dfrac{1}{2} \, \left( \dv{\theta}{t} \right)^{2} - \dfrac{g}{L} \, \cos \theta \right] = 0
\label{eq:ecuacion_pendulo_04}
\end{equation}
La ecuación anterior nos dice que la cantidad entre los corchetes es una constante\footnote{De hecho, si multiplicamos por $m \, L^{2}$, vemos que el primer término es la energía cinética y el segundo es la energía potencial gravitatoria. La ecuación comprueba entonces el teorema de conservación de energía.}. Para obtener el valor de la constante, suponemos que liberamos el péndulo del reposo, por lo que el péndulo se desplaza del ángulo $\theta$ a $\theta_{0}$, esto nos conduce a
\begin{align*}
\dv{\theta}{t} = \sqrt{\dfrac{2 \, g}{L} \, (\cos \theta - \cos \theta_{0})}
\end{align*}
Podemos utilizar la siguiente identidad trigonométrica $\cos 2 \, x = 1 - 2 \, \sin^{2} x$, para entonces ordenar los términos de la ecuación, con lo cual
\begin{align*}
2 \, \sqrt{\dfrac{g}{L}} \int \, \dd{t} = \int \dfrac{\dd{\theta}}{[k^{2} - \sin^{2} (\theta/2)]^{1/2}}
\end{align*}
En la integral, $k = \sin (\theta_{0}/2)$. Como el ángulo $\theta \leq \theta_{0}$, podemos hacer el cambio de variable $\sin \phi = k^{-1} \, \sin (\theta/2)$, para entonces obtener
\begin{align*}
\sqrt{\dfrac{g}{L}} \int \, \dd{t} = \int \dfrac{\dd{\phi}}{\sqrt{1 - k^{2} - \sin^{2} \phi}}
\end{align*}
La integral de la derecha es una \emph{interal elíptica de primera clase}, descrita por
\begin{align*}
F(k , \phi) = \int_{0}^{\phi} \dfrac{\dd{\phi^{\prime}}}{\sqrt{1 - k^{2} \, \sin^{2} \phi^{\prime} }}
\end{align*}
El tiempo necesario para que el ángulo cambie de $0$ a $\theta_{0}$ es un cuarto del período $T$. En este tiempo, el ángulo $\phi$ cambia de $0$ a $\pi/2$. Con esos límites, se tiene que la integral es
\begin{equation}
F(k) = \int_{0}^{\pi/2} \dfrac{\dd \phi}{\sqrt{1 - k^{2} \, \sin^{2} \phi} }
\label{eq:ecuacion_pendulo_05}
\end{equation}
y se le conoce como la \emph{integral elíptica completa} de primera clase.
\par
Utilizando el resultado que expresa una serie
\begin{equation}
(1 - z)^{s} = \sum_{k=0}^{\infty} \dfrac{(-s)_{k}}{k!} \, z^{k}
\label{eq:ecuacion_pendulo_06}
\end{equation}
Para evaluar la integral $F(k)$, usamos el resultado de la ec. (\ref{eq:ecuacion_pendulo_06}) para expandir el integrando
\begin{align*}
[ 1 - k^{2} \, \sin^{2} \phi ]^{-1/2} = \sum_{n=0}^{\infty} \dfrac{(\frac{1}{2})_{n}}{n!} \, k^{2 n} \, \sin^{2 n} \phi
\end{align*}
Al sustituir esta expansión en la ec. (\ref{eq:ecuacion_pendulo_05}) es posible evaluar una integral del tipo
\begin{align*}
\int \sin^{p} \, x \, \dd{x} = \int \sin^{2 n} \, \phi \, \dd{\phi}
\end{align*}
Integrando por partes
\begin{align*}
\int \sin^{p} \, x \, \dd{x} &= - \sin^{p-1} \, x \, \cos x + (p - 1) \, \int \sin^{p-1} \, x \, \cos^{2} \, x \, \dd{x} \\
&= -\sin^{p-1} x \, \cos x + (p - 1) \, \int \sin^{p-2} x \, \dd{x} - (p - 1) \, \int \sin^{p} x \dd{x} 
\end{align*}
Resolviendo para $\displaystyle{\int \sin^{p} x \, \dd{x}}$, llegamos a la fórmula
\begin{align*}
\int \sin^{p} x \, \dd{x} = - \dfrac{\sin^{p-1} \, x \, \cos x}{p} + \dfrac{p - 1}{p} \int \sin^{p-2} x \, \dd{x}
\end{align*}
Ocupando la fórmula nuevamente en la integral de la derecha, se tiene que
\begin{align*}
\int \sin^{p} x \, \dd{x} &= - \dfrac{\sin^{p-1} \, x \, \cos x}{p} + \dfrac{p - 1}{p} \, \dfrac{\sin^{p-3} \, x \, \cos x}{(p - 2)} + \\
&+ \dfrac{(p - 1)(p - 3)}{p \, (p - 2)} \int \sin^{p-4} \, x \, \dd{x}
\end{align*}
Repitiendo el uso de la fórmula siempre se tendrá como resultado una suma de términos, cada uno de los cuales involucra un producto del $\cos x$ con alguna de potencia del $\sin x$, más una integral con una potencia de $\sin x$. Si tomamos el rango de integración de $0$ a $\pi / 2$, entonces el $\cos x$ se anula en un límite y $\sin x$ se anula en el otro. Sólo el término con la integral permanece. El resultado es entonces
\begin{align*}
&\int_{0}^{\pi/2} \sin^{p} \, x \, \dd{x} = \\
&= \begin{dcases}
\dfrac{(p-1)(p-3) \ldots (p - (p-1))}{p(p-2) \ldots (p -(p-2))} \int_{0}^{\pi/2} \, \dd{x},  & p = 0, 2, 4, \ldots \\[1em]
\dfrac{(p-1)(p-3) \ldots (p - (p-2))}{p(p-2) \ldots (p -(p-3))} \int_{0}^{\pi/2} \sin x \, \dd{x}, & p = 1, 3, 5, \ldots
\end{dcases}
\end{align*}
Las integrales de la derecha se pueden evaluar directamente para obtener $\pi/2$ y $1$, respectivamente. La notación más compacta de Pochhammer para los productos queda expresada como
\begin{align*}
\int_{0}^{\pi/2} \sin^{p} x \, \dd{x} = \begin{cases}
\dfrac{\pi}{2} \, \dfrac{\left( 1/2 \right)_{p/2}}{(p/2)!}, & p = 0, 2, 4, \ldots \\[1em]
\dfrac{((p-1)/2)!}{(3/2)_{(p-1)/2}}. & p = 1, 3, 5, \ldots
\end{cases}
\end{align*}
Si el resultado para valores pares de $p$ se incluyen en la expansión del integrando en la ecuación (\ref{eq:ecuacion_pendulo_05}), la integral elíptica completa es
\begin{align*}
F(k) &= \dfrac{\pi}{2} \sum_{n=0}^{\infty} \dfrac{\left(\frac{1}{2} \right)_{n} \, \left(\frac{1}{2} \right)_{n}}{n! \, (1)_{n}} \, k^{2n} \\
&= \dfrac{\pi}{2} \, {}_{2} F_{1} (\frac{1}{2}, \frac{1}{2}; 1; k^{2})
\end{align*}
Así, el período de oscilación de un péndulo simple está dado por
\begin{align*}
T = 2 \, \pi \, \sqrt{\dfrac{L}{g}} \, {}_{2} F_{1} \left(\frac{1}{2}, \frac{1}{2}; 1; \sin^{2} \left( \dfrac{\theta_{0}}{2} \right) \right)
\end{align*}
Claramente, el período depende de la amplitud de la oscilación. Sin embargo, si la amplitud es pequeña, podemos usar sólo el término principal en las series hipergeométricas. Entonces, el período se reduce a la aproximación habitual de ángulo pequeño:
\begin{align*}
T = 2 \, \pi (L / g)^{1/2} \hspace{2cm} \theta_{0} \ll 1
\end{align*}
La integral elíptica de segunda clase está definida por
\begin{align*}
E (k, \phi) = \int_{0}^{\phi} \sqrt{1 - k^{2} \, \sin^{2} \phi^{\prime}} \, \dd{\phi^{\prime}}
\end{align*}
La integral elíptica completa de segunda clase se relaciona con la función hipergeométrica por
\begin{equation}
E(k) = \int_{0}^{\pi/2} \sqrt{1 - k^{2} \, \sin^{2} \phi} \, \dd{\phi} =  \dfrac{\pi}{2} \, F \left( -\frac{1}{2}, \frac{1}{2}; 1; k^{2} \right)
\end{equation}

% \\
% Considerando la función generatriz para los polinomios de Gegenbauer
% \begin{equation}
% \dfrac{1}{(1 - 2xt + t^{2})^{\alpha}} = \sum_{n=0}^{\infty} C_{n}^{(\alpha)} \; (x) \; t^{n}, \hspace{1cm} \vert x \vert < 1, \hspace{0.5cm} \vert t \vert < 1
% \label{eq:ecuacion_13_092}
% \end{equation}
% en el caso de que $\alpha = \frac{1}{2}$, se obtienen los polinomios de Legendre. Cuando $\alpha=1$ y luego $\alpha=0$ se generan dos conjuntos de polinomios llamados: Polinomios de Chebychev.
% \subsection{Polinomios de tipo II.}
% Con $\alpha=1$ y
% \[ C_{n}^{(1)} (x) = U_{n} \]
% la ecuación (\ref{eq:ecuacion_13_92}) conduce a
% \begin{equation}
% \dfrac{1}{(1 - 2xt + t^{2})} = \sum_{n=0}^{\infty} U_{n} \; (x) \; t^{n} \hspace{1cm} \vert x \vert < 1, \hspace{0.5cm} \vert t \vert < 1
% \label{eq:ecuacion_13_093}
% \end{equation} 
% Esas funciones $U_{n}(x)$ generadas por el término $(1-2xt+t^{2})^{-1}$ se conocen como polinomios de tipo  II. Son útiles en el estudio de la teoría de momento angular con el desarrollo de armónicos esféricos en cuatro dimensiones.
% \subsection{Polinomios de Tipo I.}
% Con $\alpha=0$ tenemos complicaciones, de hecho la función generatriz se reduce a la constante $1$. Para evitar este problema, diferenciamos la ecuación (\ref{eq:ecuacion_13_092}) respecto a $t$, lo que nos devuelve
% \begin{equation}
% \dfrac{-\alpha(-2x + 2t)}{(1 - 2xt + t^{2})^{\alpha+1}} = \sum_{n=1}^{\infty} n \; C_{n}^{(\alpha)} \; (x) \; t^{n-1}
% \label{eq:ecuacion_13_094}
% \end{equation}
% o equivalentemente
% \begin{equation}
% \dfrac{x - t}{(1 - 2xt + t^{2})^{\alpha + 1)}} = \sum_{n=1}^{\infty} \dfrac{n}{2} \left[ \dfrac{C_{n}^{(\alpha)} (x)}{\alpha} \right] \; t^{n-1}
% \label{eq:ecuacion_13_095}
% \end{equation}
% Definimos $C_{n}^{(0)} (x)$ por
% \begin{equation}
% C_{n}^{(0)} (x) = \lim_{\alpha \to 0} \dfrac{C_{n}^{(\alpha)}}{\alpha}
% \label{eq:ecuacion_13_096}
% \end{equation}
% Hemos diferenciado con respecto a $t$ para dejar a $\alpha$ en el denominador y con esto, crear una indeterminación. Ahora multiplicamos la ecuación (\ref{eq:ecuacion_13_095}) por $2t$ le sumamos $1 = (1 - 2xt + t^{2}) / (1 - 2xt + t^{2})$ para obtener
% \begin{equation}
% \dfrac{1 - t^{2}}{1 - 2xt + t^{2}} =  1 + 2 \sum_{n=1}^{\infty} \dfrac{n}{2} C_{n}^{(0)} \; (x) \; t^{n}
% \label{eq:ecuacion_13_097}
% \end{equation}
% Definimos $T_{n}(x)$ por
% \begin{equation}
% T_{n}(x) = \begin{cases}
% 1, & n=0 \\
% \dfrac{n}{2} C_{n}^{(0)} (x), & n>0
% \end{cases}
% \label{eq:ecuacion_13_098} 
% \end{equation}
% Hay que señalar que el valor de $C_{n}^{(0)}$ es el límite indicado en la ecuación (\ref{eq:ecuacion_13_096}) y no una sustitución literal de $\alpha=0$ en la función generatriz. Con esta indicación, tenemos pues
% \begin{equation}
% \dfrac{1 - t^{2}}{1 - 2xt + t^{2}} = T_{0} (x) +  2 \sum_{n=1}^{\infty} T_{n} (x) t^{n}, \hspace{1cm} \vert x \vert \leq 1, \hspace{0.5cm} \vert t \vert < 1
% \label{eq:ecuacion_13_099}
% \end{equation}
% A los $T_{n}(x)$ se les denomina polinomios de Chebychev de tipo I. Tomen en cuenta de que el nombre puede cambiar en algunas referencias.
% \\
% Si diferenciamos la función generatriz (ec. \ref{eq:ecuacion_13_099}) con respecto a $t$ y multiplicamos el denominador por $1 - 2xt + t^{2}$, se obtiene
% \[ \begin{split}
% - t - (t - x) \left[ T_{0}(x) +  2 \sum_{n=1}^{\infty} T_{n} (x) t^{n} \right] &= (1 - 2xt + t^{2}) \sum_{n=1}^{\infty} n \; T_{n}(x) \; t^{n-1} \\
% &= \sum_{n=1}^{\infty} [ n \; T_{n} \; t^{n-1} - 2 \; x \; n \; T_{n} \; t^{n} + n \; T_{n} \; t^{n+1} ]
% \end{split} \]
% de donde recuperamos la relación de recurrencia
% \begin{equation}
% T_{n+1} (x) - 2 x T_{n} (x) + T_{n-1} (x) = 0
% \label{eq:ecuacion_13_100}
% \end{equation}
% que se sigue desplazando el índice de suma con el fin de conseguir la misma potencia, $t^{n}$ en cada término y comparando los coeficientes de $t^{n}$. Con el mismo proceso en la ecuación (\ref{eq:ecuacion_13_093}), encontramos
% \[ - \dfrac{2 (t - x)}{1 - 2xt + t^{2}} = (1 - 2xt + t^{2}) \sum_{n=1}^{\infty} n \; U_{n} (x) \; t^{n-1} \]
% de la que se genera la siguiente relación de recurrencia
% \begin{equation}
% U_{n+1}(x) - 2 x U_{n} + U_{n-1} (x) = 0
% \label{eq:ecuacion_13_101}
% \end{equation}
% a la que se llega comparando los coeficientes de potencias de $t$ similares. En la tabla (\ref{tab:tabla_01}) se muestra la relación entre los coeficientes y algunos polinomios
% \begin{center}
% \begin{tabular}{ p{4cm} | C{2cm} | C{2cm} | C{2cm} | C{2cm} }
% \toprule
%  & $P_{n}$ & $A_{n}$ & $B_{n}$ & $C_{n}$ \\ \hline
% Legendre & $P_{n}(x)$ & $\frac{2n+1}{n+1}$ & $0$ & $\frac{1}{n+1}$ \\
% Chebychev I & $T_{n}$ & $2$ & $0$ & $1$ \\
% Chebychev II & $U_{n}$ & $2$ & $0$ & $1$ \\
% Asoc. Laguerre & $L_{n}^{(k)} $ & $-\frac{1}{n+1}$ & $-\frac{2n+k+1}{n+1}$ & $\frac{n+k}{n+1}$ \\
% Hermite & $H_{n}(x)$ & $2$ & $0$ & $2n$ \\
% \bottomrule
% \end{tabular}
% \captionof{table}{Relación de recursiva $P_{n+1}(x) = (A_{n} x + B_{n}) P_{n}(x) - C_{n} P_{n-1}(x)$.}
% \label{tab:tabla_01}
% \end{center}
% Usando las funciones generatrices para los primeros valores de $n$ y las relaciones de recurrencia para polinomios de orden mayor, obtenemos las siguientes expresiones tablas (\ref{tab:Poli_tipoI}) y (\ref{tab:Poli_tipoII}):
% \\
% \begin{center}
% \begin{minipage}{5cm}
% \captionof{table}{Polinomios de Chebychev Tipo I.}
% \begin{tabular}{ l }
% \toprule
% $T_{0} = 1$ \\
% $T_{1} = x$ \\
% $T_{2} = 2x^{2} -1$ \\
% $T_{3} = 4x^{3} - 3x$ \\
% $T_{4} = 8x^{4} - 8x^{2} + 1$ \\
% \bottomrule
% \end{tabular}
% \label{tab:Poli_tipoI}
% \end{minipage}
% \hspace{2cm}
% \begin{minipage}{5cm}
% \captionof{table}{Polinomios de Chebychev Tipo II.}
% \begin{tabular}{ l }
% \toprule
% $U_{0} = 1$ \\
% $U_{1} = 2x$ \\
% $U_{2} = 4x^{2} - 1$ \\
% $U_{3} = 8x^{3} - 4x$ \\
% $U_{4} = 16x^{4} - 12x^{2} + 1$ \\
% \bottomrule
% \end{tabular}
% \label{tab:Poli_tipoII}
% \end{minipage}
% \end{center}
% Nuevamente de las funciones generatrices, podemos obtener algunos valores especiales para los polinomios:
% \begin{eqnarray}
% \begin{aligned}
% T_{n}(1) &= 1 \\
% T_{n}(-1) &= (-1)^{n} \\
% T_{1n}(0) &= (-1)^{n} \\
% T_{2n+1}(0) &= 0 \\
% U_{n}(1) &= n + 1 \\
% U_{n}(-1) &= (-1)^{n} (n+1) \\
% U_{2n}(0) &= (-1)^{n} \\
% U_{2n+1}(0) &= 0
% \end{aligned}
% \end{eqnarray}
% Las relaciones de paridad para $T_{n}$ y $U_{n}$ se obtiene de sus respectivas funciones generatrices, con las sustituciones $t \to -t$ , $x \to -x$, con lo que quedan como invariantes, y son
% \begin{equation}
% T_{n}(x) = (-1)^{n} \; T_{n}(-x) \hspace{1.5cm} U_{n}(x) = (-1)^{n} \; U_{n}(-x)
% \label{eq:ecuacion_13_104}
% \end{equation}
% Las fórmulas de Rodrigues para $T_{n}$ y $U_{n}$ son
% \begin{equation}
% T_{n}(x) = \dfrac{(-1)^{n} \pi^{1/2}(1 - x^{2})^{1/2}}{2^{n} (n - \frac{1}{2})!} \; \dfrac{d^{n}}{d x^{n}}\left[ (1 - x^{2})^{n - 1/2} \right]
% \label{eq:ecuacion_13_105}
% \end{equation}
% y
% \begin{equation}
% U_{n}(x) = \dfrac{(-1)^{n} (n + 1)  \pi^{1/2}}{2^{n + 1} (n + \frac{1}{2})! \; (1 - x^{2})^{1/2}} \; \dfrac{d^{n}}{d x^{n}}\left[ (1 - x^{2})^{n + 1/2} \right]
% \label{eq:ecuacion_13_105}
% \end{equation}
% \subsection{Relaciones de recurrencia.}
% Al diferenciar las funciones generatrices de $T_{n}$ y $U_{n}$ con respecto a $x$, se obtiene una variedad de relaciones de recurrencia que involucra las derivadas, por ejemplo, de la ecuación (\ref{eq:ecuacion_13_99}), tenemos
% \[ (1 - 2xt +  t^{2}) \; 2 \; \sum_{n=1}^{\infty} T_{n}^{\prime} (x) \; t^{n} =  2 t \left[ T_{0}(x) + 2 \sum_{n=1}^{\infty} T_{n}(x) \; t^{n} \right] \]
% de la que se genera la recurrencia
% \begin{equation}
% 2 T_{n-1} (x) = T_{n}^{\prime} (x) - 2 x \; T_{n-1}^{\prime} (x) +  T_{n-2}^{\prime}(x)
% \label{eq:ecuacion_13_107}
% \end{equation}
% que es la derivada de la ecuación (\ref{eq:ecuacion_13_100}) para $n \to n-1$.
% \\
% Otras relaciones de recurrencia útiles son
% \begin{equation}
% (1 - x^{2}) T_{n}^{\prime} (x) = -n \; x \; T_{n}(x) + n \; T_{n-1}(x)
% \label{eq:ecuacion_13_108}
% \end{equation}
% y 
% \begin{equation}
% (1 - x^{2}) U_{n}^{\prime} (x) = -n \; x \; U_{n}(x) + (n + 1) \; U_{n-1}(x)
% \label{eq:ecuacion_13_109}
% \end{equation}
% Manejando este conjunto de relaciones de recurrencia, podemos eliminar el índice $n-1$ y ocupar $T_{n}^{\prime \prime} (x)$ así como $T_{n}(x)$, por lo que los polinomios de Chebychev de tipo I satisfacen la EDO2H
% \begin{equation}
% (1 - x^{2}) T_{n}^{\prime \prime} (x) - x T_{n}^{\prime} (x) + n^{2} T_{n} (x) = 0
% \label{eq:ecuacion_13_110}
% \end{equation}
% Los polinomios de Chebychev de tipo II, satisfacen
% \begin{equation}
% (1 - x^{2}) U_{n}^{\prime \prime} (x) - 3 x U_{n}^{\prime} (x) + n(n +2) U_{n} (x) = 0
% \label{eq:ecuacion_13_111}
% \end{equation}
% La ecuación \emph{ultraesférica}
% \begin{equation}
% (1 - x^{2}) \dfrac{d^{2}}{d x^{2}} C_{n}^{(\alpha)} (x) - (2 \alpha + 1) x \dfrac{d}{dx} C_{n}^{(\alpha)} (x) + n(n + 2\alpha) C_{n}^{(\alpha)} (x) = 0
% \label{eq:ecuacion_13_112}
% \end{equation}
% es una generalización de esas ecuaciones diferenciales, que se reduce a la ecuación (\ref{eq:ecuacion_13_110}) cuando $\alpha=0$ y la ecuación (\ref{eq:ecuacion_13_111}) cuando $\alpha = 1$, en el caso de que $\alpha = \frac{1}{2}$ se recupera la ecuación de Legendre.
% \subsection{Forma trigonométrica.}
% En el desarrollo de las propiedades de las soluciones de Chebychev, es conveniente hacer un cambio de variable, de $x$ a $\cos \theta$.
% \\
% Con $x = \cos \theta$ y $d /dx = (-1 / \sin \theta) (d / d \theta)$, se comprueba que
% \[ (1 -x^{2}) \dfrac{d^{2} T_{n}}{d x^{2}} = \dfrac{d^{2} T_{n}}{d \theta^{2}} - \cot \theta \dfrac{d T_{n}}{d \theta}, \hspace{1.5cm} x T_{n}^{prime} = - \cot \theta \dfrac{d T_{n}}{d \theta} \]
% agregando esos términos a la ecuación (\ref{eq:ecuacion_13_110}), resulta
% \begin{equation}
% \dfrac{d^{2} T_{n}}{d \theta^{2}} + n^{2} T_{n} = 0
% \label{eq:ecuacion_13_113}
% \end{equation}
% que es la ecuación de un oscilador armónico simple, con soluciones $\cos n \theta$ y $\sin n \theta$. Con las condiciones de frontera en $x=0,1$, se tiene
% \begin{equation}
% T_{n} = \cos n \theta = \cos n (\arccos x)
% \label{eq:ecuacion_13_114a}
% \end{equation}
% Una segunda solución independiente de las ecuaciones (\ref{eq:ecuacion_13_110}) y (\ref{eq:ecuacion_13_113}) es
% \begin{equation}
% V_{n} = \sin n \theta = \sin n (\arccos x)
% \label{eq:ecuacion_13_114b}
% \end{equation}
% Las correspondientes soluciones para la ecuación de Chebychev de tipo II, (\ref{eq:ecuacion_13_111}) son
% \begin{eqnarray}
% U_{n} &= \dfrac{\sin (n + 1) \theta}{\sin \theta} \label{eq:ecuacion_13_115a} \\
% W_{n} &= \dfrac{\cos (n + 1) \theta}{\sin \theta} \label{eq:ecuacion_13_115b}
% \end{eqnarray}
% Los dos conjuntos de soluciones de tipo I y Tipo II, están relacionadas por
% \begin{eqnarray}
% V_{n} &=  (1 - x^{2})^{1/2} \; U_{n-1} (x) \label{eq:ecuacion_13_116a} \\
% W_{n} &=  (1 - x^{2})^{1/2} \; T_{n-1} (x) \label{eq:ecuacion_13_116b} 
% \end{eqnarray}
% De las funciones generatrices se ha visto que $T_{n}$ y $U_{n}$ son polinomios, claramente $V_{n}$ y $W_{n}$ \textbf{no} son polinomios. 
% \subsection{Ortogonalidad.}
% Si la ecuación (\ref{eq:ecuacion_13_110}) se deja en una forma autoadjunta, obtenemos el factor de peso $w (x) = (1 - x^{2})^{-1/2}$, para la ecuación (\ref{eq:ecuacion_13_111}), la correspondiente función de peso es $(1 - x^{2})^{+1/2}$. Resultando las integrales de ortogonalidad
% \begin{eqnarray}
% \int_{-1}^{1} T_{m} (x) T_{n} (x) (1 - x^{2})^{-1/2} dx = 
% \begin{cases}
% 0, & m \neq n \\
% \dfrac{\pi}{2}, & m = n \neq 0 \\
% \pi, & m = n = 0
% \end{cases}
% \label{eq:ecuacion_13_120} \\
% \int_{-1}^{1} V_{m} (x) V_{n} (x) (1 - x^{2})^{-1/2} dx = 
% \begin{cases}
% 0, & m \neq n \\
% \dfrac{\pi}{2}, & m = n \neq 0 \\
% 0, & m = n = 0
% \end{cases}
% \label{eq:ecuacion_13_121}
% \end{eqnarray}
% y
% \begin{equation}
% \int_{-1}^{1} W_{m} (x) W_{n} (x) (1 - x^{2})^{-1/2} dx = \dfrac{\pi}{2} \delta_{mn}
% \label{eq:ecuacion_13_123}
% \end{equation}


\end{document}