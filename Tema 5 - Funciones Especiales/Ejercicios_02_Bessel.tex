\documentclass[12pt]{article}
\usepackage[utf8]{inputenc}
\usepackage[spanish,es-lcroman, es-tabla]{babel}
\usepackage[autostyle,spanish=mexican]{csquotes}
\usepackage{amsmath}
\usepackage{amssymb}
\usepackage{nccmath}
\numberwithin{equation}{section}
\usepackage{amsthm}
\usepackage{graphicx}
\usepackage{epstopdf}
\DeclareGraphicsExtensions{.pdf,.png,.jpg,.eps}
\usepackage{color}
\usepackage{float}
\usepackage{multicol}
\usepackage{enumerate}
\usepackage[shortlabels]{enumitem}
\usepackage{anyfontsize}
\usepackage{anysize}
\usepackage{array}
\usepackage{multirow}
\usepackage{enumitem}
\usepackage{cancel}
\usepackage{tikz}
\usepackage{circuitikz}
\usepackage{tikz-3dplot}
\usetikzlibrary{babel}
\usepackage{bm}
\usepackage{mathtools}
\usepackage{esvect}
\usepackage{hyperref}
\usepackage{relsize}
\usepackage{siunitx}
\usepackage{physics}
%\usepackage{biblatex}
\usepackage{standalone}
\usepackage{mathrsfs}
\usepackage{bigints}
\usepackage{bookmark}
\spanishdecimal{.}

\setlist[enumerate]{itemsep=0mm}

\renewcommand{\baselinestretch}{1.5}

\let\oldbibliography\thebibliography

\renewcommand{\thebibliography}[1]{\oldbibliography{#1}

\setlength{\itemsep}{0pt}}
%\marginsize{1.5cm}{1.5cm}{2cm}{2cm}


\newtheorem{defi}{{\it Definición}}[section]
\newtheorem{teo}{{\it Teorema}}[section]
\newtheorem{ejemplo}{{\it Ejemplo}}[section]
\newtheorem{propiedad}{{\it Propiedad}}[section]
\newtheorem{lema}{{\it Lema}}[section]

\title{Ejercicios Funciones de Bessel\\ \large {Matemáticas Avanzadas de la Física}  \vspace{-1.5\baselineskip}}
\date{}
\author{}
\begin{document}
\newgeometry{top=2cm, bottom=2cm, left=1cm, right=1cm,headsep=0pt}
\renewcommand\labelenumii{\theenumi.{\arabic{enumii})}}
\maketitle
\fontsize{14}{14}\selectfont
%Referencia Hassani - Chapter 12
\begin{enumerate}
\item \textbf{(1 punto.) } Un cilindro largo conductor de calor de radio $a$ se compone de dos mitades (con secciones transversales semicirculares) con un espacio infinitesimal entre ellas. Las mitades superior e inferior del cilindro están en contacto con baños térmicos $+T_{0}$ y $-T_{0}$, respectivamente. Encuentra la temperatura tanto dentro como fuera del cilindro.
\item \textbf{(1 punto.) } Un cilindro largo conductor de calor de radio $a$ se compone de dos mitades (con secciones transversales semicirculares) con un espacio infinitesimal entre ellas. Las mitades superior e inferior del cilindro están en contacto con baños térmicos $+T_{1}$ y $-T_{1}$, respectivamente. El cilindro está dentro de otro cilindro de radio $b$ más grande ( $a < b$ y coaxial con él) que se mantiene a la temperatura $T_{2}$. Encuentra la temperatura dentro del cilindro interno, entre los dos cilindros y fuera del cilindro externo.
%Referencia Andrews - Chapter 6, Problem 26
\item \textbf{(1 punto.) } Una onda con distorsión modulada de fase puede representarse por
\begin{align*}
s(t) = R \, \cos [ \omega_{0} \, t +  \epsilon (t)]
\end{align*}
donde $R$ es la amplitud de la onda y $\epsilon(t)$ representa el \enquote{término de distorsión}. A menudo es suficiente aproximar $\epsilon (t)$ por el primer término de su serie de Fourier, es decir,
\begin{align*}
\epsilon (t) \cong a \, \sin \omega_{m} \, t
\end{align*}
donde $a$ es el pico de la fase y $\omega_{m}$ es la frecuencia, ambas de la distorsión. Por lo que la onda original es
\begin{align*}
s(t) \cong R \, \cos (\omega_{0} \, t + a \, \sin \omega_{m} \, t)
\end{align*}
Demuestra que $s(t)$ puede descomponerse en sus componentes armónicos de acuerdo a la siguiente expresión:
\begin{align*}
s(t) \cong R \, J_{0}(a) \, \cos \omega_{0} \, t + R \sum_{n=1}^{\infty} J_{n} (a) [\cos (\omega_{0} \, t + n \, \omega_{m} \, t) + (-1)^{n} \cos (\omega_{0} \, t - n \, \omega_{m} \, t) ]
\end{align*}
Recuerda que si ocupas una expresión que te sirva para la solución, deberás de demostrar cada expresión que utilices.
\end{enumerate}
\end{document}