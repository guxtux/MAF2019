\documentclass[12pt]{article}
\usepackage[left=0.25cm,top=1cm,right=0.25cm,bottom=1cm]{geometry}
\textwidth = 20cm
\hoffset = -1cm
\usepackage[utf8]{inputenc}
\usepackage[spanish,es-tabla]{babel}
\usepackage[autostyle,spanish=mexican]{csquotes}
\usepackage[tbtags]{amsmath}
\usepackage{nccmath}
\usepackage{amsthm}
\usepackage{amssymb}
\usepackage{graphicx}
\usepackage{standalone}
\usepackage[outdir=./]{epstopdf}
\usepackage{siunitx}
\usepackage{physics}
\usepackage{color}
\usepackage{float}
\usepackage{multicol}
%\usepackage{milista}
\usepackage{enumitem}
\usepackage{anyfontsize}
\usepackage{anysize}
\usepackage{enumitem}
\usepackage{capt-of}
\usepackage{bm}
\usepackage{relsize}
\usepackage{placeins}
\usepackage{empheq}
\usepackage{cancel}
\usepackage{wrapfig}
\spanishdecimal{.}
\renewcommand{\baselinestretch}{1.5} 
\renewcommand\labelenumii{\theenumi.{\arabic{enumii}}}
\newcommand{\ptilde}[1]{\ensuremath{{#1}^{\prime}}}
\newcommand{\stilde}[1]{\ensuremath{{#1}^{\prime \prime}}}
\newcommand{\ttilde}[1]{\ensuremath{{#1}^{\prime \prime \prime}}}
\newcommand{\ntilde}[2]{\ensuremath{{#1}^{(#2)}}}


\title{Enunciados del Tema 4 para el Segundo Examen \\[0.3em]  \large{Matemáticas Avanzadas de la Física}\vspace{-3ex}}
\author{M. en C. Gustavo Contreras Mayén}
\date{ }
\begin{document}
\vspace{-4cm}
\maketitle
\fontsize{14}{14}\selectfont

\textbf{Indicaciones: } Deberás de resolver cada ejercicio de la manera más completa, ordenada y clara posible, anotando cada paso así como las operaciones involucradas. El puntaje de cada ejercicio es de \textbf{1 punto}, con excepción en donde se indica.
\par
La calificación del examen se hará considerando 6 ejercicios, por lo que resolver el adicional, favorecerá un puntaje que mejoraría el promedio.

\begin{enumerate}
%Ref. Andrews (1998) Special Functions Chap. 5.3 Problem 6
\item Polinomios asociados y ordinarios de Laguerre.
\begin{enumerate}
\item Demuestra que:
\begin{align*}
L_{n}^{k} (0) = \dfrac{(n + k)!}{n! \, k!}
\end{align*}
\item Verifica la fórmula integral:
\begin{align*}
\scaleint{6ex}_{\bs 0}^{\infty} e^{-x} \, x^{k} \, L_{n} (x) \dd{x} = \begin{cases}
0 & k < n \\
(-1)^{n} \, n! & k = n
\end{cases}
\end{align*}
\end{enumerate}
\item Polinomios de Hermite.
\begin{enumerate}
%Ref. Arfken(2006) 13.1.7 (b)
\item Demuestra que:
\begin{align*}
\scaleint{6ex}_{\bs - \infty}^{\infty} x \, H_{n}(x) \, \exp \bigg[ - \dfrac{x^{2}}{2} \bigg] \dd{x} = \begin{cases}
0 & n \mbox{ par} \\
2 \, \pi \dfrac{(n + 1)!}{\big[ (n + 1)/2 \big]!} & n \mbox{ impar}
\end{cases}
\end{align*}
%Ref. Andrews (1998) Exercises 5.8 Problem 21
\item Demuestra que las funciones $\psi_{n} (x) = \exp(-x^{2}/2) \, H_{n} (x)$ satisfacen:
\begin{align*}
\scaleint{6ex}_{\bs -\infty}^{\infty} x^{2} \, \abs{\psi_{n} (x)}^{2} \dd{x} = \sqrt{\pi} \, 2^{n} \, n! \big( n + \dfrac{1}{2} \big) \hspace{1.5cm} n = 0, 1, 2, \ldots
\end{align*}
\end{enumerate}
\item Funciones de Bessel.
\begin{enumerate}
\item (\textbf{1 punto}. ) Deducir una expresión para la distribución estacionaria de temperaturas $u(r, z)$ en el cilindro macizo limitado por las superficies $r = 1$, $z = 0$ y $z = 1$, si $u = 0$ en la superficie $r= 1$, $u = 1$ en la superficie $z = 1$ y la base $z = 0$ está aislada.
\item Usando al función generatriz demuestra que:
\begin{enumerate}
%Ref. Andrews (1998) - Chap. 6 Problem 9 (a) y (b)
\item (\textbf{1 punto}. )
\begin{align*}
\exp(i \, x \, \sin \theta) = \nsum_{n=-\infty}^{\infty} J_{n}(x) \, e^{i n \theta}
\end{align*}
\item (\textbf{1 punto}. )
\begin{align*}
\exp(i \, x \, \cos \theta) = J_{0} (x) + 2 \, \nsum_{n=1}^{\infty} i^{n} \, J_{n}(x) \, \cos n \theta
\end{align*}
\end{enumerate}
\end{enumerate}


%Ref. Mason (2003) 1.3.2
\item Polinomios de Chebyshev. Los polinomios de Chebyshev están definidos en el intervalo $[- 1, 1]$, siendo posible definirlos en cualquier rango finito $[a, b]$ para la variable $x$, haciendo que éste rango corresponda al rango $[-1, 1]$ con una nueva variable $s$, se ocupa la siguiente transformación lineal:
\begin{align*}
s = \dfrac{2 \, x - (a + b)}{(b - a)}
\end{align*}

Por lo que los polinomios de Chebyshev de primer tipo ajustados al intervalo $[a, b]$ son $T_{n}(s)$, de manera similar se hace el ajuste para los polinomios de segundo tipo  $U_{n} (s)$. 
\noindent
\par
Desarrolla la expresión para los polinomios de grado $n = 0, 1, 2, 3, 4$ en el rango indicado para $x$:
\begin{enumerate}
\item De Chebyshev de primer tipo $T_{n} (s)$   ajustados al rango $[-4, 6]$.
\item De Chebyshev de segundo tipo $U_{n} (s)$   ajustados al rango $[-5, 5]$.
\end{enumerate}
\item Función hipergeométrica. Identifica la serie para la siguiente función hipergeométrica, escribiéndola en términos de una función conocida.
\begin{enumerate}
%Ref. Riley (2006) 18.11 (b) y (c)
\item 
\begin{align*}
F \big( 1, 1; 2; - x \big) = \quad ?
\end{align*}
\item 
\begin{align*}
F \bigg( \dfrac{1}{2}, 1; \dfrac{3}{2}; - x^{2} \bigg) = \quad ?
\end{align*}

\end{enumerate}
\end{enumerate}


\end{document}