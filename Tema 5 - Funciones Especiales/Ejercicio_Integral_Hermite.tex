\documentclass[12pt]{article}
\usepackage[utf8]{inputenc}
\usepackage[T1]{fontenc}
\usepackage[spanish,es-lcroman]{babel}
\usepackage{amsmath}
\usepackage{amsthm}
\usepackage{amsfonts}
\usepackage{amssymb}
\usepackage{physics}
\usepackage{tikz}
\usepackage{float}
\usepackage{calc}
\usepackage[autostyle,spanish=mexican]{csquotes}
\usepackage[per-mode=symbol]{siunitx}
\usepackage{textcomp, gensymb}
\usepackage{multicol}
\usepackage{enumitem}
\usepackage{hyperref}
\usepackage{setspace}
\usepackage[left=2.00cm, right=2.00cm, top=2.00cm, 
     bottom=2.00cm]{geometry}
% \usepackage{Estilos/ColoresLatex}
\usepackage{makecell}
\usepackage{subcaption}
\usepackage[skip=10pt, indent=30pt]{parskip}
% \usepackage{scalerel}
\usepackage{scalerel}[2016-12-29]
\usepackage{biblatex}
\usepackage{cancel}
\usepackage{tcolorbox}

\definecolor{ao}{rgb}{0.0, 0.0, 1.0}

\hypersetup{
    colorlinks=true,
    linkcolor=ao,
    filecolor=magenta,      
    urlcolor=ao,
}

\newcommand{\ptilde}[1]{\ensuremath{{#1}^{\prime}}}
\newcommand{\stilde}[1]{\ensuremath{{#1}^{\prime \prime}}}
\newcommand{\ttilde}[1]{\ensuremath{{#1}^{\prime \prime \prime}}}
\newcommand{\ntilde}[2]{\ensuremath{{#1}^{(#2)}}}
\newcommand{\pderivada}[1]{\ensuremath{{#1}^{\prime}}}
\newcommand{\sderivada}[1]{\ensuremath{{#1}^{\prime \prime}}}
\newcommand{\tderivada}[1]{\ensuremath{{#1}^{\prime \prime \prime}}}
\newcommand{\nderivada}[2]{\ensuremath{{#1}^{(#2)}}}

\def\stretchint#1{\vcenter{\hbox{\stretchto[440]{\displaystyle\int}{#1}}}}
\def\scaleint#1{\vcenter{\hbox{\scaleto[3ex]{\displaystyle\int}{#1}}}}
\def\scaleiint#1{\vcenter{\hbox{\scaleto[6ex]{\displaystyle\iint}{#1}}}}
\def\scaleiiint#1{\vcenter{\hbox{\scaleto[6ex]{\displaystyle\iiint}{#1}}}}
\def\scaleoint#1{\vcenter{\hbox{\scaleto[3ex]{\displaystyle\oint}{#1}}}}
\def\bs{\mkern-12mu}

% \newcommand{\textbf}[2]{\textbf{\textcolor{#1}{#2}}}
\sisetup{per-mode=symbol}
\decimalpoint
\sisetup{bracket-numbers = false}
\newlength{\depthofsumsign}
\setlength{\depthofsumsign}{\depthof{$\sum$}}
\newcommand{\nsum}[1][1.4]{% only for \displaystyle
    \mathop{%
        \raisebox
            {-#1\depthofsumsign+1\depthofsumsign}
            {\scalebox
                {#1}
                {$\displaystyle\sum$}%
            }
    }
}

\AtBeginDocument{\RenewCommandCopy\qty\SI}
\ExplSyntaxOn
\msg_redirect_name:nnn { siunitx } { physics-pkg } { none }
\ExplSyntaxOff

\numberwithin{equation}{section}

\linespread{1.25}

\renewcommand{\labelenumii}{\theenumii}
\renewcommand{\theenumii}{\theenumi.\arabic{enumii}.}

\emergencystretch=1em


\title{\large{Ejercicio con los polinomios de Hermite}}
\author{M. en C. Gustavo Contreras Mayén}
\date{}

\spanishdecimal{.}

\begin{document}
\maketitle
\fontsize{14}{14}\selectfont

Demuestra que:
\begin{align*}
\scaleint{6ex}_{\bs - \infty}^{\infty} x^{2} \, e^{-x^{2}} \, H_{n}(x) \, H_{n}(x) \dd{x} = \pi^{\frac{1}{2}} \, 2^{n} \, n! \, \bigg( n + \dfrac{1}{2} \bigg)
\end{align*}
Esta integral se presenta en el cálculo del desplazamiento medio cuadrado del oscilador armónico cuántico.

\vspace{1cm}
\noindent
\textbf{Solución:} Reescribimos la integral de la siguiente forma:
\begin{align*}
\scaleint{6ex}_{\bs - \infty}^{\infty} x^{2} \, e^{-x^{2}} \, H_{n}(x) \, H_{n}(x) \dd{x} = \scaleint{6ex}_{\bs - \infty}^{\infty} e^{-x^{2}} \, \left[ x \, H_{n} (x) \right]^{2} \dd{x}
\end{align*}
tenemos una relación de recurrencia tal que:
\begin{align*}
    H_{n+1} (x) &= 2 \, x \, H_{n} (x) - 2 \, n \, H_{n-1} (x) \\[0.5em]
    \Rightarrow \quad x \, H_{n} (x) &= n \, H_{n-1} (x) + \dfrac{1}{2} \, H_{n+1} (x)
\end{align*}
entonces la integral inicial la escribimos como:
\begin{align*}
\scaleint{6ex}_{\bs - \infty}^{\infty} e^{-x^{2}} \, \left[ x \, H_{n} (x) \right]^{2} \dd{x} = \scaleint{6ex}_{\bs - \infty}^{\infty} e^{-x^{2}} \, \left[ n \, H_{n-1} (x) + \dfrac{1}{2} \, H_{n+1} (x) \right]^{2} \dd{x}
\end{align*}
procedemos ahora a desarrollar el binomio al cuadrado que está a la derecha de la igualdad:
\begin{align*}
&\scaleint{6ex}_{\bs - \infty}^{\infty} e^{-x^{2}} \, \left[ n \, H_{n-1} (x) + \dfrac{1}{2} \, H_{n+1} (x) \right]^{2} \dd{x} = \\[1em]
&= \scaleint{6ex}_{\bs - \infty}^{\infty} e^{-x^{2}} \, \left[ n^{2} \, H_{n-1} (x) \, H_{n-1} (x) + n \, H_{n-1} (x) \, H_{n+1} (x) + \dfrac{1}{2^{2}} \, H_{n+1} (x) \, H_{n+1} (x) \right] \dd{x}
\end{align*}
separamos en una suma de integrales:
\begin{align*}
&n^{2} \, \scaleint{6ex}_{\bs - \infty}^{\infty} e^{-x^{2}} \, H_{n-1} (x) \, H_{n-1} (x) \dd{x} + n \, \scaleint{6ex}_{\bs - \infty}^{\infty} e^{-x^{2}} \, H_{n-1} (x) \, H_{n+1} (x) \dd{x} + \\[1em]
&+ \dfrac{1}{2^{2}} \, \scaleint{6ex}_{\bs - \infty}^{\infty} e^{-x^{2}} \, H_{n+1} (x) \, H_{n+1} (x) \dd{x} \\
\end{align*}
que al usar la propiedad de ortogonalidad de los polinomios de Hermite:
\begin{align*}
    \scaleint{6ex}_{-\infty}^{\infty} e^{-x^{2}} \, &H_{n} (x) \, H_{m} (x) \dd{x} = 2^{n} \, \sqrt{\pi} \, n! \, \delta_{m n} \\
    n &= 0, 1, 2, \ldots
\end{align*}
la segunda integral se anula, dejando entonces el resultado para la primera y tercera integral:
\begin{align*}
    n^{2} \left[ 2^{n-1} \, \sqrt{\pi} \, \left( n - 1 \right)! \right] + \dfrac{1}{2^{2}} \left[ 2^{n+1} \, \sqrt{\pi} \, \left( n + 1 \right)! \right] =
\end{align*}
simplificamos cada uno de los términos:
\begin{align*}
    &= 2^{n-1} \, \sqrt{\pi} \, \left( n \cdot n \right) \, \left( n - 1 \right)! + 2^{n-1} \, \sqrt{\pi} \, \left( n + 1 \right) \, n! = \\[1em]
    &= 2^{n-1} \, \sqrt{\pi} \, n! \, \left( n + n + 1 \right) = \\[1em]
    &= 2^{n-1} \, \sqrt{\pi} \, n! \, \left( 2 \, n + 1 \right) = \\[1em]
    &= 2^{n} \, \sqrt{\pi} \, n! \, \left( n + \dfrac{1}{2} \right) \quad \qed
\end{align*}
\end{document}