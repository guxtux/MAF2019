\documentclass[12pt]{article}
\usepackage[utf8]{inputenc}
\usepackage[spanish,es-lcroman, es-tabla]{babel}
\usepackage[autostyle,spanish=mexican]{csquotes}
\usepackage{amsmath}
\usepackage{amssymb}
\usepackage{nccmath}
\numberwithin{equation}{section}
\usepackage{amsthm}
\usepackage{graphicx}
\usepackage{epstopdf}
\DeclareGraphicsExtensions{.pdf,.png,.jpg,.eps}
\usepackage{color}
\usepackage{float}
\usepackage{multicol}
\usepackage{enumerate}
\usepackage[shortlabels]{enumitem}
\usepackage{anyfontsize}
\usepackage{anysize}
\usepackage{array}
\usepackage{multirow}
\usepackage{enumitem}
\usepackage{cancel}
\usepackage{tikz}
\usepackage{circuitikz}
\usepackage{tikz-3dplot}
\usetikzlibrary{babel}
\usetikzlibrary{shapes}
\usepackage{bm}
\usepackage{mathtools}
\usepackage{esvect}
\usepackage{hyperref}
\usepackage{relsize}
\usepackage{siunitx}
\usepackage{physics}
%\usepackage{biblatex}
\usepackage{standalone}
\usepackage{mathrsfs}
\usepackage{bigints}
\usepackage{bookmark}
\spanishdecimal{.}

\setlist[enumerate]{itemsep=0mm}

\renewcommand{\baselinestretch}{1.5}

\let\oldbibliography\thebibliography

\renewcommand{\thebibliography}[1]{\oldbibliography{#1}

\setlength{\itemsep}{0pt}}
%\marginsize{1.5cm}{1.5cm}{2cm}{2cm}


\newtheorem{defi}{{\it Definición}}[section]
\newtheorem{teo}{{\it Teorema}}[section]
\newtheorem{ejemplo}{{\it Ejemplo}}[section]
\newtheorem{propiedad}{{\it Propiedad}}[section]
\newtheorem{lema}{{\it Lema}}[section]

\usepackage{mathrsfs}
\usepackage{standalone}
\usepackage{tikz}
\usetikzlibrary{shapes}
\usepackage{bigints}
\newcommand{\saltosin}{\nonumber \\}
\spanishdecimal{.}
%\usepackage{enumerate}
%\author{M. en C. Gustavo Contreras Mayén. \texttt{curso.fisica.comp@gmail.com}}
\title{Problemas con las Funciones de Bessel \\ {\large Matemáticas Avanzadas de la Física}}
\date{ }
\begin{document}
%\renewcommand\theenumii{\arabic{theenumii.enumii}}
\renewcommand\labelenumii{\theenumi.{\arabic{enumii}}}
\maketitle
\fontsize{14}{14}\selectfont
\section*{Distribución de temperaturas en cilindros.}
Considérese un largo cilindro macizo de radio $r = a$ cuya superficie lateral se matiene a temperatura cero. Inicialmente, la temperatura en el cilindro está dada por $f(r)$. Suponiendo que no hay variación con $z$ y que la temperatura para cada radio $r$ es la misma en toda esa superficie (independiente de $\theta$), se desea hallar una expresión para la distribución de temperaturas $u(r, t)$ en el cilindro.
\\
El modelo matemático del problema está constituido por la ecuación de calor
\begin{equation}
\dfrac{\partial u}{\partial t} = k \left( \dfrac{\partial^{2} u }{\partial r^{2}} + \dfrac{1}{r} \dfrac{\partial u}{\partial r} \right) \hspace{1cm} 0 < r < a, \hspace{0.5cm} t > 0
\label{eq:ecuacion_039}
\end{equation}
junto con las condiciones
\begin{equation}
u(a,t) = 0, \hspace{2cm} u(r,0) =  f(r)
\label{eq:ecuacion_040}
\end{equation}
Además, se exige que la función $u(r,t)$ debe ser acotada en todo el volumen del cilindro.
\\
Usando el método de separación de variables, proponemos una solución del tipo
\begin{equation}
u(r,t) = R(r) T(t)
\label{eq:ecuacion_041}
\end{equation}
susituyendo en la ec. (\ref{eq:ecuacion_039}), se tiene que
\begin{eqnarray*}
R(t) \; T'(t) &=& k \left[ R^{\prime \prime}(r) T(t) + \dfrac{1}{r} R^{\prime}(r) T(t) \right] \saltosin
\dfrac{R^{\prime \prime} + \frac{1}{r} R^{\prime}(r)}{R(r)} &=& \dfrac{1}{k} \; \dfrac{T^{\prime}}{T(t)} =  \lambda^{2}
\end{eqnarray*}
Adicionalmente, de la condición $u(a,t) = 0$, se deduce que
\[ R(a) T(t) = 0 \Rightarrow R(a) = 0 \]
De esta manera, se obtiene los siguientes dos problemas:
\begin{eqnarray}
R^{\prime \prime} + \dfrac{1}{r} R^{\prime} (r) + \lambda^{2} R(r) &=& 0, \hspace{1.5cm} R(a) = 0 \label{eq:ecuacion_042} \\
T^{\prime}(t) + k \lambda^{2} T(t) &=& 0 \label{eq:ecuacion_043} 
\end{eqnarray}
La ec. (\ref{eq:ecuacion_042}) corresponde a una ecuación de Bessel con parámetro $\lambda$ del tipo
\begin{equation}
y^{\prime \prime} + \dfrac{1}{x} y^{\prime} + \left( \lambda^{2} - \dfrac{v^{2}}{x^{2}} \right) y = 0 
\label{eq:ecuacion_011}
\end{equation}
que al hacer el cambio de variable $z = \lambda x$, se deduce que la solución general de esta ecuación está dada por
\begin{equation}
y = C_{1} J_{v} (\lambda x) + C_{2} N_{n} (\lambda x)
\label{eq:ecuacion_012}
\end{equation}
Entonces para el problema del cilindro, se tiene que $v=0$, por lo tanto, la solución general es de la forma
\[ R(r) = C_{1} J_{0}(\lambda r) + C_{2} N_{0} (\lambda r) \]
La función $R(r)$ debe de estar acotada en $r=0$, por lo que se reduce que $C_{2} =0$, ya que $N_{0}(0) \to \infty$. Quedando entonces
\[ R(r) =C_{1} J_{0} (\lambda r) \]
Ahora
\[ R(a) = 0 \Rightarrow C_{1} J_{0} (\lambda a) = 0 \]
Para obtener una solución no trivial, se debe de tomar $C_{1} \neq 0$, entonces
\begin{equation}
J_{0}(\lambda a) = 0
\label{eq:ecuacion_044}
\end{equation}
Si $\lambda_{i}$ con $i = 1,2, \ldots$, son las raíces positivas de la ec. (\ref{eq:ecuacion_044}), se tiene que, aparte del factor constante, las funciones
\[ R_{i}(r) = J_{0} (\lambda_{i} r) \hspace{1cm} i = 1, 2, \ldots \]
son soluciones de la ec. (\ref{eq:ecuacion_042}).
Para los valores de $\lambda$ considerados, la ec. (\ref{eq:ecuacion_043}) tiene las soluciones particulares
\[ T_{i}(t) = e^{-k \lambda_{i}^{2} t} \hspace{1.5cm} i = 1, 2, \ldots \]
Así la sucesión de funciones
\[ u_{i}(r,t) = J_{0} (\lambda_{i} r) e^{-k \lambda_{i}^{2} t} \hspace{1.5cm} i= 1, 2, \ldots \]
satisfacen la ec. (\ref{eq:ecuacion_039}) y las condiciones de frontera consideradas, y de acuerdo al principio de superposición, la función
\begin{equation}
u(r,t) =  \sum_{i=1}^{\infty} C_{i} J_{0} (\lambda_{i} r)e^{-k \lambda_{i}^{2} t}
\label{eq:ecuacion_045}
\end{equation}
también las satisface.
\\
Aplicando la condición $u(r,0) = f(r)$, se obtiene la expansión de Fourier-Bessel
\[ f(r) = \sum_{i=1}^{\infty} C_{i} J_{0}(\lambda_{i} r) \]
de donde
\[ C_{i} = \dfrac{1}{a^{2} J_{1}^{2} (\lambda_{i} a)} \; \int_{0}^{a} r \; f(r) \; J_{0}(\lambda_{i} r) dr \hspace{1cm} i = 1,2, \ldots \]
finalmente, la distribución de temperaturas en el cilindro viene dada por
\begin{equation}
u(r,t) = \dfrac{2}{a^{2}} \sum_{i = 1}^{\infty } \left[ \int_{0}^{a} r \; f(r) \; J_{0}(\lambda_{i} r) dr \right] \; \dfrac{J_{0}(\lambda_{i} r)} {J_{1}^{2} (\lambda_{i} a)} e^{-k \lambda_{i}^{2} t}
\label{eq:ecuacion_046}
\end{equation}
donde la suma es tomada sobre todas las raíces positivas de la ec. (\ref{eq:ecuacion_044}).
\\
\textbf{Ejercicio a cuenta de examen.} Deducir una expresión para la distribución estacionaria de temperaturas $u(r, z)$ en el cilindro macizo limitado por las superficies $r = 1$, $z = 0$ y $z = 1$, si $u = 0$ en la superficie $r= 1$, $u = 1$ en la superficie $z = 1$ y la base $z = 0$ está aislada.
\section*{Oscilaciones en una membrana circular.}
Una onda que viaja en el espacio puede ser expresada como una función $\mathbf{u}( \overrightarrow{r}, t)$ donde cada componente deberá satisfacer la ecuación de onda
\begin{equation}
\nabla \mathbf{u} = \dfrac{1}{v^{2}} \dfrac{\partial^{2} \mathbf{u}}{\partial t^{2}}
\label{eq:ecuacion_01_01}
\end{equation}
donde $\overrightarrow{r}$ representa el vector de movimiento en el espacio tridimensional en coordenadas euclidianas $(x, y, z)$ y $v$ es la velocidad de propagación de la onda en el medio.
\\
Una característica muy significativa de la ecuación (\ref{eq:ecuacion_01_01}) es que es una ecuación lineal y por lo tanto, si $\mathbf{u}_{1}(\overrightarrow{r}, t)$, ; $\mathbf{u}_{2}(\overrightarrow{r}, t)$, $\mathbf{u}_{1}( \ldots, \overrightarrow{n}, t)$, son soluciones individuales de dicha ecuación, cualquier combinación lineal de éstas también será una solución.
\\
Por lo tanto, una solución de la forma
\begin{equation}
\mathbf{u}(\overrightarrow{r},t) = \sum_{i=1}^{n} C_{i} \mathbf{u}_{i} (\overrightarrow{r},t)
\label{eq:ecuacion_01_02}
\end{equation}
donde los coeficientes $C_{i}$ son constantes arbitrarias, deberá satisfacer la ecuación de onda (\ref{eq:ecuacion_01_01}).
\\
Esta propiedad de linealidad se denomina principio de superposición y sugiere que en sistemas lineales, la perturbación resultante debida a cualquier número de ondas en un punto cualquiera del espacio será la suma algebráica de sus ondas
constituyentes por separado.
\\
Para cualquier sistema de coordenadas, los desplazamientos de una superficie en su dirección normal, están determinados
por la ecuación de onda

\begin{equation}
\nabla U = \dfrac{1}{v^{2}} \; \dfrac{\partial^{2} U }{\partial t^{2}}
\label{eq:ecuacion_01_15}
\end{equation}
donde $v$ representa la velocidad de propagación de la onda en la superficie. Para desarrollar la ecuación (\ref{eq:ecuacion_01_15}) se hicieron algunas suposiciones como que la membrana es delgada y uniforme con rigidez despreciable, que es perfectamente elástica sin amortiguamiento y que vibra con desplazamientos de amplitud pequeños. Además se supuso que la tensión sobre la superficie de la membrana está distribuida uniformemente.
\\
Para calcular los modos normales, se vuelve a escribir (\ref{eq:ecuacion_01_15}) suponiendo una solución de la forma
\begin{equation}
U = \Psi e^{i \omega t}
\label{eq:ecuacion_01_16}
\end{equation}
donde $\Psi =  \Psi(x,y)$ es una función que depende sólo de la posición. Sustituyendo la solución (\ref{eq:ecuacion_01_16}) en (\ref{eq:ecuacion_01_15}) se obtiene
\begin{equation}
\nabla^{2} \Psi + k^{2} \Psi = 0
\label{eq:ecuacion_01_17}
\end{equation}
con $k^{2} = \sqrt{\omega/v}$, que es la ecuación de onda independiente del tiempo o ecuación de Helmholtz.
\\
Las soluciones a esta ecuación representan un conjunto de líneas nodales, es decir, regiones donde el desplazamiento es constante. Éstas serán zonas en las que los desplazamientos de la membrana son nulos y que permanecen así en cualquier instante. Las soluciones a la ecuación (\ref{eq:ecuacion_01_17}) para una membrana de forma y condiciones específicas representará sus modos normales de vibración.
\\
\\
Para una membrana circular, la ecuación de onda apropiada para las vibraciones transversales $U(r, \theta, t)$ en coordenadas polares es
\begin{equation}
\dfrac{\partial^{2} \mathbf{U}}{\partial r^{2}} + \dfrac{1}{r} \; \dfrac{\partial \mathbf{U}}{\partial r} + \dfrac{1}{r^{2}} \; \dfrac{\partial^{2} \mathbf{U}}{\partial \theta^{2}} = \dfrac{1}{v^{2}} \; \dfrac{\partial^{2} \mathbf{U}}{\partial t^{2}}
\label{eq:ecuacion_01_24}  
\end{equation}
La ecuación (\ref{eq:ecuacion_01_24}) se convierte en la ecuación de Helmholtz en coordenadas polares,
\begin{equation}
\dfrac{r^{2}}{R} \left( \dfrac{d^{2} R}{d r^{2}} + \dfrac{1}{r} \; \dfrac{d R}{d r} \right) + k^{2} r^{2} = \dfrac{1}{\Theta} \; \dfrac{d^{2} \Theta}{d \theta^{2}}
\label{eq:ecuacion_01_25}
\end{equation}
La solución a la parte angular tiene una forma armónica
\begin{equation}
\Theta (\theta) = \cos (m \theta + \gamma)
\label{eq:ecuacion_01_26}
\end{equation}
donde $\gamma$ es el ángulo de fase inicial y los valores de la constante $m$ están restringidos a valores enteros $m = 0, 1,2, \ldots $. La ecuación (\ref{eq:ecuacion_01_25}) se convierte en la ecuación de Bessel
\begin{equation}
\dfrac{d^{2} R}{d r^{2}} + \dfrac{1}{r} \; \dfrac{d R}{d r}  + \left( k^{2} - \dfrac{m^{2}}{r^{2}} \right) R = 0
\label{eq:ecuacion_01_27}
\end{equation}
Las soluciones de esta ecuación son las funciones de Bessel de primera clase $J_{m}(kr)$ y segunda clase $N_{m}(kr)$ de orden $m$ respectivamente,
\begin{equation}
R(r) = A J_{m} (kr) + B N_{m} (kr)
\label{eq:ecuacion_01_28}
\end{equation}
Las funciones de Bessel de primera clase $J_{m}(kr)$ son funciones oscilantes cuyas amplitudes disminuyen conforme aumenta $kr$, y las funciones de Bessel de segunda clase $N_{m}(kr)$ crecen sin límite cuando $kr \to 0$.
\\
Si el radio $r$ de la membrana circular es $a$ y ésta se encuentra fija en todo su contorno, la condición de frontera requerida será
\begin{equation}
U(a, \theta, t) = 0
\label{eq:ecuacion_01_29}
\end{equation}

Aunque  la ec. (\ref{eq:ecuacion_01_28}) es la solución general de (\ref{eq:ecuacion_01_27}), dado que la membrana se extiende a través del origen $(r = 0)$, la solución debe ser finita en el origen $(r = 0)$ y esto requiere que la constante $B = 0$ de tal manera que la solución será la parte real de
\begin{equation}
U_{m,n} (r, \theta, t) =  A_{mn} J_{m} (k_{mn} r) \cos (m \theta) e^{i \omega_{mn}t}
\label{eq:ecuacion_01_30}
\end{equation}
donde $A_{mn}$ es la amplitud máxima del modo dado por los enteros $(m,n)$ y la constante $k_{mn}$ viene dada por
\begin{equation}
k_{mn} = \dfrac{j_{mn}}{a}
\label{eq:ecuacion_01_31}
\end{equation}
donde $j_{mn}$ representa el $n$-ésimo cero de la función de Bessel $J_{m}$ de primera clase de orden $m$. Las frecuencias naturales de vibración estarán dadas por
\begin{equation}
f_{mn} = \dfrac{v}{2 \pi} \; \dfrac{j_{mn}}{a}
\label{eq:ecuacion_01_32}
\end{equation}
El valor de $m$ restringe el número de líneas nodales radiales mientras que el valor de $n$, el número de circunferencias nodales.


\end{document}