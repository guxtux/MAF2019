\documentclass[12pt]{beamer}
\usepackage{../Estilos/BeamerMAF}
\input{../Preambulos/preambulo_Beamer_Copenhagen_wolverine}

\date{15 diciembre de 2021}

\title{\large{Propiedades de las funciones de Bessel}}
\author{M. en C. Gustavo Contreras Mayén}

\begin{document}
\maketitle
\fontsize{14}{14}\selectfont
\spanishdecimal{.}

\section*{Contenido}
\frame[allowframebreaks]{\tableofcontents[currentsection, hideallsubsections]}

%Referencia: Funciones de Bessel pdf
\section{Funciones de Bessel}
\frame{\tableofcontents[currentsection, hideothersubsections]}
\subsection{Relaciones de recurrencia}

\begin{frame}
\frametitle{Definición de las funciones de Bessel}
Partiendo de la definición de las funciones de Bessel de orden $n$, se tiene que:
\pause
\begin{eqnarray*}
\begin{aligned}
x^{n} \, J_{n}(x) &= x^{n} \, \nsum_{k=0}^{\infty} \dfrac{(-1)^{k} \, x^{n+2k}}{k! \, \Gamma (n + k + 1)} = \\[0.5em] \pause
&=  \nsum_{k=0}^{\infty} \dfrac{(-1)^{k} \, x^{2n+2k}}{ 2^{n+2k} \, k! \, \Gamma (n + k + 1)}
\end{aligned}
\end{eqnarray*}
\end{frame}
\begin{frame}
\frametitle{Diferenciando la expresión}
Al diferenciar con respecto a $x$ la expresión anterior resulta:
\pause
\begin{eqnarray*}
\begin{aligned}
\dv{x} \big[ x^{n} \, J_{n}(x) \big] &= \nsum_{k=0}^{\infty} \dfrac{(-1)^{k} \, (2 \, n + 2 \, k) \, x^{2n+2k-1}}{ 2^{n+2k} \, k! \, \Gamma (n + k + 1)} = \\[0.5em] \pause
&= \nsum_{k=0}^{\infty} \dfrac{(-1)^{k} \, 2(n + k) \, x^{2n+2k-1}}{ 2^{n+2k} \, k! \, \big( n + k\big) \, \Gamma (n + k)} = \\[0.5em] \pause
&= x^{n} \, \nsum_{k=0}^{\infty} \dfrac{(-1)^{k} \, (x/2)^{(n-1)+2k}}{ k! \, \Gamma \big[ (n - 1) + k + 1 \big]}
\end{aligned}
\end{eqnarray*}
\end{frame}
\begin{frame}
\frametitle{Primer resultado}
Lo que significa que:
\pause
\begin{align}
\dv{x} \big[ x^{n} \, J_{n}(x) \big] = x^{n} \, J_{n-1}(x)
\label{eq:ecuacion_14}
\end{align}
\end{frame}
\begin{frame}
\frametitle{Segundo resultado}
De manera análoga, se demuestra que:
\pause
\begin{align}
\dv{x} \big[ x^{-n} \, J_{n}(x) \big] = -x^{-n} \, J_{n+1}(x)
\label{eq:ecuacion_15}
\end{align}    
\end{frame}
\begin{frame}
\frametitle{Derivando un resultado}
Al derivar el lado izquierdo de la ec. (\ref{eq:ecuacion_14}) como un producto, se tiene que:
\pause
\begin{align*}
n \, x^{n-1} \, J_{n}(x) + x^{n} \, \pderivada{J}_{n}(x) = x^{n} \, J_{n-1}(x)
\end{align*}
\end{frame}
\begin{frame}
\frametitle{Manejando la expresión}
De donde, al multiplicar por $x^{-n}$, se obtiene:
\begin{align}
\dfrac{n}{x} \, J_{n}(x) + \pderivada{J}_{n}(x) = J_{n-1}(x)
\label{eq:ecuacion_16}
\end{align}
\end{frame}
\begin{frame}
\frametitle{Obteniendo otro resultado}
De la misma manera, desarrollando la derivada en la ec. (\ref{eq:ecuacion_15}) y multiplicando luego por $x^{n}$, se llega a:
\pause
\begin{align}
-\dfrac{n}{x} \, J_{n}(x) + \pderivada{J}_{n}(x) = - J_{n+1}(x)
\label{eq:ecuacion_17}    
\end{align}
\end{frame}
\begin{frame}
\frametitle{Operando nuevamente las expresiones}
Sumando y restando las ecs. (\ref{eq:ecuacion_16}) y (\ref{eq:ecuacion_17}), resultan las relaciones de recurrencia:
\begin{eqnarray}
\begin{aligned}
2 \, \pderivada{J}_{n}(x) &= J_{n-1} (x) - J_{n+1}(x) \label{eq:ecuacion_18} \\[0.5em] \pause
\dfrac{2 \, n}{x} \, \pderivada{J}_{n}(x) &= J_{n-1} (x) + J_{n+1}(x) \label{eq:ecuacion_19}
\end{aligned}
\end{eqnarray}
respectivamente.
\end{frame}
\begin{frame}
\frametitle{Expresión en términos de $J_{0}(x)$ y $J_{1}(x)$}
Notemos que utilizando repetidamente la ec. (\ref{eq:ecuacion_19}), cualquier función de Bessel $J_{n}(x) (n = 2, 3, \ldots)$ puede expresarse en términos de $J_{0}(x)$ y $J_{1}(x)$.
\end{frame}\begin{frame}
\frametitle{Expresión integral}
Las ecs. (\ref{eq:ecuacion_14}) y (\ref{eq:ecuacion_15}) también resultan útiles cuando se escriben de la forma integral:
\pause
\begin{eqnarray}
\begin{aligned}
\scaleint{6ex} x^{n} \, J_{n-1} (x) \dd{x} &= x^{n} \, J_{n}(x) + C \label{eq:ecuacion_20} \\[0.5em] \pause
\scaleint{6ex} x^{-n} \, J_{n+1} (x) \dd{x} &= - x^{-n} \, J_{n}(x) + C \label{eq:ecuacion_21}
\end{aligned}
\end{eqnarray}
\end{frame}
\begin{frame}
\frametitle{Caso particular con $n = 0$}
Haciendo que $n = 0$ en las ecs. (\ref{eq:ecuacion_15}) y (\ref{eq:ecuacion_21}), se obtiene el caso particular:
\pause
\begin{align*}
\pderivada{J}_{0} (x) = - J_{1}(x) \hspace{1.5cm} \scaleint{6ex} J_{1}(x) \dd{x} = - J_{0}(x) + C
\end{align*}
que se ocupa frecuentemente.
\end{frame}
\begin{frame}
\frametitle{Relaciones de recurrencia válidas}
Las funciones $Y_{n}(x)$, $H_{n}^{(1)} (x)$ y $H_{n}^{(2)} (x)$ satisfacen las mismas fórmulas de recurrencia.
\end{frame}

\section{Ejercicios con las relaciones de recurrencia}
\frame{\tableofcontents[currentsection, hideothersubsections]}
\subsection{Ejercicio 1}

\begin{frame}
\frametitle{Enunciado del ejercicio}
Calcular:
\begin{align*}
\dv{x} \big[ x^{2} \, J_{3}(2 x) \big]
\end{align*}
en términos de funciones de Bessel.
\end{frame}
\begin{frame}
\frametitle{Punto importante}
Al derivar la función de Bessel, se multiplica por la derivada del argumento, así:
\pause
\begin{align*}
\dv{x} \big[ x^{2} \, J_{3}(2 x) \big] = 2 \, x \, J_{3} (2 x) + x^{2} \cdot 2 \, \pderivada{J}_{3} (2 x)
\end{align*}
\end{frame}
\begin{frame}
\frametitle{Usando las relaciones mencionadas}
Usando la ec. (\ref{eq:ecuacion_16}) con $n = 3$ y $2 x$ en lugar de $x$, se tiene:
\pause
\begin{eqnarray*}
\begin{aligned}
\dfrac{3}{2 x} \, J_{3}(2 x) + \pderivada{J}_{3}(2 x) &= J_{2} (2 x) \\[0.5em] \pause
\Rightarrow \hspace{0.2cm} \pderivada{J}_{3} (2 x) &= J_{2}(x) - \dfrac{3}{2 x} \, J_{3} (2 x)
\end{aligned}
\end{eqnarray*}
\end{frame}
\begin{frame}
\frametitle{Manejando la expresión}
Sustituyendo $\pderivada{J}_{3}(2 x)$, se obtiene:
\pause
\begin{eqnarray*}
\begin{aligned}
\dv{x} \big[ x^{2} \, J_{3}(2 x) \big] &=  2 \, x \, J_{3} (2 x) + 2 \, x^{2} \bigg[ J_{2}(x) - \dfrac{3}{2 x} \, J_{3} (2 x) \bigg] = \\[0.5em] \pause
 &= 2 \, x^{2} \, J_{2} (2 x) - x \, J_{3} (2 x) \qed
\end{aligned}
\end{eqnarray*}
\end{frame}

\subsection{Ejercicio 2}

\begin{frame}
\frametitle{Enunciado del ejercicio 2}
Demostrar que:
\begin{align*}
\pderivada{J}_{2}(x) = \big( 1 - 4 \, x^{-2} \big) \, J_{1}(x) + 2 \, x^{-1} \, J_{0} (x)
\end{align*}
\end{frame}
\begin{frame}
\frametitle{Ocupando resultados previos}
Usando la ec. (\ref{eq:ecuacion_16}) con $n = 2$, se tiene que:
\pause
\begin{eqnarray*}
\begin{aligned}
\dfrac{2}{x} \, J_{2}(x) + \pderivada{J}_{2}(x) &= J_{1}(x) \\[0.5em] \pause
\Rightarrow \hspace{0.2cm} \pderivada{J}_{2}(x) &= {J}_{1} (x) - \dfrac{2}{x} \, J_{2}(x)
\end{aligned}
\end{eqnarray*}
\end{frame}
\begin{frame}
\frametitle{Ocupando valores determinados}
Haciendo que $n = 1$ en la ec. (\ref{eq:ecuacion_19}), llegamos a:
\pause
\begin{align*}
\dfrac{2}{x} \, J_{1}(x) &= J_{0} (x) + J_{2}(x) \\[0.5em] \pause
\Rightarrow \hspace{0.2cm} J_{2}(x) &= \dfrac{2}{x} \, J_{1}(x) - J_{0}(x)
\end{align*}
\end{frame}
\begin{frame}
\frametitle{Completando la solución}
Luego:
\pause
\begin{eqnarray*}
\begin{aligned}
\pderivada{J}_{2}(x) &= J_{1}(x) - \dfrac{2}{x} \bigg[ \dfrac{2}{x} \, J_{1}(x) - J_{0}(x) \bigg] \\[0.5em] \pause
\Rightarrow \hspace{0.2cm} \pderivada{J}_{2} (x) &= \big( 1 - 4 \, x^{-2} \big) \, J_{1}(x) - 2 \, x^{-1} \, J_{0}(x) \qed
\end{aligned}
\end{eqnarray*}
\end{frame}

\subsection{Ejercicio 3}

\begin{frame}
\frametitle{Enunciado del ejercicio 3}
Hallar:
\begin{align*}
I = \displaystyle \scaleint{6ex} x^{4} \, J_{1}(x) \dd{x}
\end{align*}
en términos de $J_{0}(x)$ y $J_{1}(x)$.
\end{frame}
\begin{frame}
\frametitle{Cambiando la expresión de la integral}
La integral $I$ la podemos ver como:
\pause
\begin{align*}
I = \scaleint{6ex} x^{2} \cdot x^{2} \, J_{1}(x) \dd{x}
\end{align*}
\end{frame}
\begin{frame}
\frametitle{Resolviendo la integral}
Que al integrar por partes:
\pause
\begin{eqnarray*}
\begin{aligned}
u = x^{2} \hspace{0.5cm} &\Rightarrow \hspace{0.5cm} \dd{u} = 2\, x \, \dd{u} \\[0.5em] \pause
\dd{v} = x^{2} \, J_{1} (x) \hspace{0.5cm} &\Rightarrow \hspace{0.5cm} \scaleint{6ex} x^{2} \, J_{1}(x) \dd{x} = x^{2} \, J_{2}(x) \\[0.5em]
& \hspace{1.5cm} \mbox{-según la ec. (\ref{eq:ecuacion_20})-}
\end{aligned}
\end{eqnarray*}
\end{frame}
\begin{frame}
\frametitle{Resolviendo la integral}
Entonces se tiene que:
\pause
\begin{align*}
I = u \, v - \scaleint{6ex} v \dd{u} = x^{4} \, J_{2}(x) - 2 \scaleint{6ex} x^{3} \, J_{2} (x) \dd{x}
\end{align*}
\end{frame}
\begin{frame}
\frametitle{Usando la expresión integral}
Usando de nuevo la ec. (\ref{eq:ecuacion_20}), llegamos a:
\pause
\begin{align*}
I = x^{4} \, J_{2}(x) - 2 \, x^{3} \, J_{3}(x) + C
\end{align*}
\end{frame}
\begin{frame}
\frametitle{Ocupando la expresión integral}
Ocupando la ec. (\ref{eq:ecuacion_19}), finalmente tenemos que:
\pause
\begin{align*}
I = \scaleint{6ex} x^{4} J_{1}(x) \dd{x} &= \big( 8 \, x^{2} - x^{4} \big) \, J_{0} (x) + \\[0.5em]
&+ \big( 4 \, x^{3} - 16 \, x \big) \, J_{1}(x) + C \qed
\end{align*}
\end{frame}

\end{document}