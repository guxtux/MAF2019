\documentclass[hidelinks,12pt]{article}
\usepackage[left=0.25cm,top=1cm,right=0.25cm,bottom=1cm]{geometry}
%\usepackage[landscape]{geometry}
\textwidth = 20cm
\hoffset = -1cm
\usepackage[utf8]{inputenc}
\usepackage[spanish,es-tabla]{babel}
\usepackage[autostyle,spanish=mexican]{csquotes}
\usepackage[tbtags]{amsmath}
\usepackage{nccmath}
\usepackage{amsthm}
\usepackage{amssymb}
\usepackage{mathrsfs}
\usepackage{graphicx}
\usepackage{subfig}
\usepackage{standalone}
\usepackage[outdir=./Imagenes/]{epstopdf}
\usepackage{siunitx}
\usepackage{physics}
\usepackage{color}
\usepackage{float}
\usepackage{hyperref}
\usepackage{multicol}
%\usepackage{milista}
\usepackage{anyfontsize}
\usepackage{anysize}
%\usepackage{enumerate}
\usepackage[shortlabels]{enumitem}
\usepackage{capt-of}
\usepackage{bm}
\usepackage{relsize}
\usepackage{placeins}
\usepackage{empheq}
\usepackage{cancel}
\usepackage{wrapfig}
\usepackage[flushleft]{threeparttable}
\usepackage{makecell}
\usepackage{fancyhdr}
\usepackage{tikz}
\usepackage{bigints}
\usepackage{scalerel}
\usepackage{pgfplots}
\usepackage{pdflscape}
\pgfplotsset{compat=1.16}
\spanishdecimal{.}
\renewcommand{\baselinestretch}{1.5} 
\renewcommand\labelenumii{\theenumi.{\arabic{enumii}})}
\newcommand{\ptilde}[1]{\ensuremath{{#1}^{\prime}}}
\newcommand{\stilde}[1]{\ensuremath{{#1}^{\prime \prime}}}
\newcommand{\ttilde}[1]{\ensuremath{{#1}^{\prime \prime \prime}}}
\newcommand{\ntilde}[2]{\ensuremath{{#1}^{(#2)}}}

\newtheorem{defi}{{\it Definición}}[section]
\newtheorem{teo}{{\it Teorema}}[section]
\newtheorem{ejemplo}{{\it Ejemplo}}[section]
\newtheorem{propiedad}{{\it Propiedad}}[section]
\newtheorem{lema}{{\it Lema}}[section]
\newtheorem{cor}{Corolario}
\newtheorem{ejer}{Ejercicio}[section]

\newlist{milista}{enumerate}{2}
\setlist[milista,1]{label=\arabic*)}
\setlist[milista,2]{label=\arabic{milistai}.\arabic*)}
\newlength{\depthofsumsign}
\setlength{\depthofsumsign}{\depthof{$\sum$}}
\newcommand{\nsum}[1][1.4]{% only for \displaystyle
    \mathop{%
        \raisebox
            {-#1\depthofsumsign+1\depthofsumsign}
            {\scalebox
                {#1}
                {$\displaystyle\sum$}%
            }
    }
}
\def\scaleint#1{\vcenter{\hbox{\scaleto[3ex]{\displaystyle\int}{#1}}}}
\def\bs{\mkern-12mu}


%\usepackage{showframe}
\title{Propiedades funciones de Bessel \\ \large {Tema 5 - Funciones especiales} \vspace{-3ex}}
\author{M. en C. Gustavo Contreras Mayén}
\date{ }
\begin{document}
\vspace{-4cm}
\maketitle
\fontsize{14}{14}\selectfont
\tableofcontents
\newpage

%Referencia: Funciones de Bessel pdf
\section{Relaciones de recurrencia.}

Partiendo de la definición de las funciones de Bessel de orden $n$, se tiene que:
\begin{align*}
x^{n} \, J_{n}(x) &= x^{n} \, \nsum_{k=0}^{\infty} \dfrac{(-1)^{k} \, x^{n+2k}}{k! \, \Gamma (n + k + 1)} = \\[0.5em]
&=  \nsum_{k=0}^{\infty} \dfrac{(-1)^{k} \, x^{2n+2k}}{ 2^{n+2k} \, k! \, \Gamma (n + k + 1)}
\end{align*}

Al diferenciar con respecto a $x$ la expresión anterior resulta:
\begin{align*}
\dv{x} \big[ x^{n} \, J_{n}(x) \big] &= \nsum_{k=0}^{\infty} \dfrac{(-1)^{k} \, (2 \, n + 2 \, k) \, x^{2n+2k-1}}{ 2^{n+2k} \, k! \, \Gamma (n + k + 1)} = \\[0.5em]
&= \nsum_{k=0}^{\infty} \dfrac{(-1)^{k} \, 2(n + k) \, x^{2n+2k-1}}{ 2^{n+2k} \, k! \, \big( n + k\big) \, \Gamma (n + k)} = \\[0.5em]
&= x^{n} \, \nsum_{k=0}^{\infty} \dfrac{(-1)^{k} \, (x/2)^{(n-1)+2k}}{ k! \, \Gamma \big[ (n - 1) + k + 1 \big]}
\end{align*}
lo que significa que:
\begin{align}
\dv{x} \big[ x^{n} \, J_{n}(x) \big] = x^{n} \, J_{n-1}(x)
\label{eq:ecuacion_14}
\end{align}

De manera análoga, se demuestra que:
\begin{align}
\dv{x} \big[ x^{-n} \, J_{n}(x) \big] = -x^{-n} \, J_{n+1}(x)
\label{eq:ecuacion_15}
\end{align}    

Al derivar el lado izquierdo de la ec. (\ref{eq:ecuacion_14}) como un producto, se tiene que:
\begin{align*}
n \, x^{n-1} \, J_{n}(x) + x^{n} \, \pderivada{J}_{n}(x) = x^{n} \, J_{n-1}(x)
\end{align*}
de donde, al multiplicar por $x^{-n}$, se obtiene:
\begin{align}
\dfrac{n}{x} \, J_{n}(x) + \pderivada{J}_{n}(x) = J_{n-1}(x)
\label{eq:ecuacion_16}
\end{align}
De la misma manera, desarrollando la derivada en la ec. (\ref{eq:ecuacion_15}) y multiplicando luego por $x^{n}$, se llega a:
\begin{align}
-\dfrac{n}{x} \, J_{n}(x) + \pderivada{J}_{n}(x) = - J_{n+1}(x)
\label{eq:ecuacion_17}    
\end{align}
Sumando y restando las ecs. (\ref{eq:ecuacion_16}) y (\ref{eq:ecuacion_17}), resultan las relaciones de recurrencia:
\begin{align}
2 \, \pderivada{J}_{n}(x) &= J_{n-1} (x) - J_{n+1}(x) \label{eq:ecuacion_18} \\[0.5em]
\dfrac{2 \, n}{x} \, \pderivada{J}_{n}(x) &= J_{n-1} (x) + J_{n+1}(x) \label{eq:ecuacion_19}
\end{align}
respectivamente.
\par
Notemos que utilizando repetidamente la ec. (\ref{eq:ecuacion_19}), cualquier función de Bessel $J_{n}(x) (n = 2, 3, \ldots)$ puede expresarse en términos de $J_{0}(x)$ y $J_{1}(x)$.
\par
Las ecs. (\ref{eq:ecuacion_14}) y (\ref{eq:ecuacion_15}) también resultan útiles cuando se escriben de la forma integral:
\begin{align}
\scaleint{6ex} x^{n} \, J_{n-1} (x) \dd{x} &= x^{n} \, J_{n}(x) + C \label{eq:ecuacion_20} \\[0.5em]
\scaleint{6ex} x^{-n} \, J_{n+1} (x) \dd{x} &= - x^{-n} \, J_{n}(x) + C \label{eq:ecuacion_21}
\end{align}
Haciendo que $n = 0$ en las ecs. (\ref{eq:ecuacion_15}) y (\ref{eq:ecuacion_21}), se obtiene el caso particular:
\begin{align*}
\pderivada{J}_{0} (x) = - J_{1}(x) \hspace{1.5cm} \scaleint{6ex} J_{1}(x) \dd{x} = - J_{0}(x) + C
\end{align*}
que se ocupa frecuentemente.
\par
Las funciones $Y_{n}(x)$, $H_{n}^{(1)} (x)$ y $H_{n}^{(2)} (x)$ satisfacen las mismas fórmulas de recurrencia.

\newpage
\section{Ejercicios con las relaciones de recurrencia.}

\subsection{Ejemplo 1.}

\noindent
Calcular $\displaystyle \dv{x} \big[ x^{2} \, J_{3}(2 x) \big]$ en términos de funciones de Bessel.
\par
Al derivar la función de Bessel, se multiplica por la derivada del argumento, así:
\begin{align*}
\dv{x} \big[ x^{2} \, J_{3}(2 x) \big] = 2 \, x \, J_{3} (2 x) + x^{2} \cdot 2 \, \pderivada{J}_{3} (2 x)
\end{align*}
Usando la ec. (\ref{eq:ecuacion_16}) con $n = 3$ y $2 x$ en lugar de $x$, se tiene:
\begin{align*}
\dfrac{3}{2 x} \, J_{3}(2 x) + \pderivada{J}_{3}(2 x) &= J_{2} (2 x) \\[0.5em]
\Rightarrow \hspace{0.2cm} \pderivada{J}_{3} (2 x) &= J_{2}(x) - \dfrac{3}{2 x} \, J_{3} (2 x)
\end{align*}
Sustituyendo $\pderivada{J}_{3}(2 x)$, se obtiene:
\begin{align*}
\dv{x} \big[ x^{2} \, J_{3}(2 x) \big] &=  2 \, x \, J_{3} (2 x) + 2 \, x^{2} \bigg[ J_{2}(x) - \dfrac{3}{2 x} \, J_{3} (2 x) \bigg] = \\[0.5em]
&= 2 \, x^{2} \, J_{2} (2 x) - x \, J_{3} (2 x) \qed
\end{align*}

\subsection{Ejemplo 2.}

\noindent

Demostrar que:
\begin{align*}
\pderivada{J}_{2}(x) = \big( 1 - 4 \, x^{-2} \big) \, J_{1}(x) + 2 \, x^{-1} \, J_{0} (x)
\end{align*}

Usando la ec. (\ref{eq:ecuacion_16}) con $n = 2$, se tiene que:
\begin{align*}
\dfrac{2}{x} \, J_{2}(x) + \pderivada{J}_{2}(x) &= J_{1}(x) \\[0.5em]
\Rightarrow \hspace{0.2cm} \pderivada{J}_{2}(x) &= {J}_{1} (x) - \dfrac{2}{x} \, J_{2}(x)
\end{align*}
Haciendo que $n = 1$ en la ec. (\ref{eq:ecuacion_19}), llegamos a:
\begin{align*}
\dfrac{2}{x} \, J_{1}(x) &= J_{0} (x) + J_{2}(x) \\[0.5em]
\Rightarrow \hspace{0.2cm} J_{2}(x) &= \dfrac{2}{x} \, J_{1}(x) - J_{0}(x)
\end{align*}
Luego:
\begin{align*}
\pderivada{J}_{2}(x) &= J_{1}(x) - \dfrac{2}{x} \bigg[ \dfrac{2}{x} \, J_{1}(x) - J_{0}(x) \bigg] \\[0.5em]
\Rightarrow \hspace{0.2cm} \pderivada{J}_{2} (x) &= \big( 1 - 4 \, x^{-2} \big) \, J_{1}(x) - 2 \, x^{-1} \, J_{0}(x) \qed
\end{align*}

\subsection{Ejemplo 3.}

\noindent
Hallar $I = \displaystyle \scaleint{6ex} x^{4} \, J_{1}(x) \dd{x}$ en términos de $J_{0}(x)$ y $J_{1}(x)$.
\\[0.5em]
La integral $I$ la podemos ver como:
\begin{align*}
I = \scaleint{6ex} x^{2} \cdot x^{2} \, J_{1}(x) \dd{x}
\end{align*}
que al integrar por partes:
\begin{align*}
u = x^{2} \hspace{0.5cm} &\Rightarrow \hspace{0.5cm} \dd{u} = 2\, x \, \dd{u} \\[0.5em]
\dd{v} = x^{2} \, J_{1} (x) \hspace{0.5cm} &\Rightarrow \hspace{0.5cm} \scaleint{6ex} x^{2} \, J_{1}(x) \dd{x} = x^{2} \, J_{2}(x) \\[0.5em]
 & \hspace{1.5cm} \mbox{-según la ec. (\ref{eq:ecuacion_20})-}
\end{align*}
Entonces se tiene que:
\begin{align*}
I = u \, v - \scaleint{6ex} v \dd{u} = x^{4} \, J_{2}(x) - 2 \scaleint{6ex} x^{3} \, J_{2} (x) \dd{x}
\end{align*}
Usando de nuevo la ec. (\ref{eq:ecuacion_20}), llegamos a:
\begin{align*}
I = x^{4} \, J_{2}(x) - 2 \, x^{3} \, J_{3}(x) + C
\end{align*}
Ocupando la ec. (\ref{eq:ecuacion_19}), finalmente tenemos que:
\begin{align*}
I = \scaleint{6ex} x^{4} J_{1}(x) \dd{x} = \big( 8 \, x^{2} - x^{4} \big) \, J_{0} (x) + \big( 4 \, x^{3} - 16 \, x \big) \, J_{1}(x) + C \qed
\end{align*}


\end{document}