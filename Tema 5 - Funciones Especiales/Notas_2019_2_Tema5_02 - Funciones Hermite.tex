\documentclass[12pt]{article}
\usepackage[utf8]{inputenc}
\usepackage[spanish,es-lcroman, es-tabla]{babel}
\usepackage[autostyle,spanish=mexican]{csquotes}
\usepackage{amsmath}
\usepackage{amssymb}
\usepackage{nccmath}
\numberwithin{equation}{section}
\usepackage{amsthm}
\usepackage{graphicx}
\usepackage{epstopdf}
\DeclareGraphicsExtensions{.pdf,.png,.jpg,.eps}
\usepackage{color}
\usepackage{float}
\usepackage{multicol}
\usepackage{enumerate}
\usepackage[shortlabels]{enumitem}
\usepackage{anyfontsize}
\usepackage{anysize}
\usepackage{array}
\usepackage{multirow}
\usepackage{enumitem}
\usepackage{cancel}
\usepackage{tikz}
\usepackage{circuitikz}
\usepackage{tikz-3dplot}
\usetikzlibrary{babel}
\usetikzlibrary{shapes}
\usepackage{bm}
\usepackage{mathtools}
\usepackage{esvect}
\usepackage{hyperref}
\usepackage{relsize}
\usepackage{siunitx}
\usepackage{physics}
%\usepackage{biblatex}
\usepackage{standalone}
\usepackage{mathrsfs}
\usepackage{bigints}
\usepackage{bookmark}
\spanishdecimal{.}

\setlist[enumerate]{itemsep=0mm}

\renewcommand{\baselinestretch}{1.5}

\let\oldbibliography\thebibliography

\renewcommand{\thebibliography}[1]{\oldbibliography{#1}

\setlength{\itemsep}{0pt}}
%\marginsize{1.5cm}{1.5cm}{2cm}{2cm}


\newtheorem{defi}{{\it Definición}}[section]
\newtheorem{teo}{{\it Teorema}}[section]
\newtheorem{ejemplo}{{\it Ejemplo}}[section]
\newtheorem{propiedad}{{\it Propiedad}}[section]
\newtheorem{lema}{{\it Lema}}[section]

%\author{M. en C. Gustavo Contreras Mayén. \texttt{curso.fisica.comp@gmail.com}}
\title{Funciones de Hermite \\ {\large Tema 5 - Matemáticas Avanzadas de la Física}\vspace{-1.5\baselineskip}}
\date{ }
\begin{document}
\maketitle
\fontsize{14}{14}\selectfont
%Referencia:: Sepúlveda - Lecciones de Física Matemática Cap. 8.5 Polinomios de Hermite.
\section{Funciones de Hermite.}
La ecuación de Hermite, tiene una aplicación en física, quizá la más importante que es la del oscilador armónico en mecánica cuántica, tiene la forma:
\begin{equation}
\ddot{H}(x) - 2 \, x \,  \dot{H}(x) + 2\, n \, H(x) = 0
\label{eq:ecuacion_08_58}
\end{equation}
La forma autoadjunta para una ED sabemos que es de la forma
\[ \dv{x} \left[ q(x) \, \dv{H(x)}{x} \right] + r(x) \, H(x) + \lambda \, p(x) \, H(x) = 0 \]
por lo que para la ec. de Hermite, resulta ser
\begin{equation}
\dv{x} \left[ \exp(-x^{2}) \, \dv{H(x)}{x} \right] + 2 \, n \, \exp(-x^{2}) \, H(x) = 0
\label{eq:ecuacion_08_59}
\end{equation}
La solución a esta ecuación forma una base ortogonal con factor de peso $p(x) = \exp(-x^{2})$, en el intervalo $(-\infty, \infty)$, ya que en dicho intervalo
\[ [q \, W]\eval_{-\infty}^{\infty} = [ \exp(-x^{2}) \, W ]\eval_{-\infty}^{\infty} = 0 \]
La ortogonalidad queda expresada por
\[ \int_{-\infty}^{\infty} \exp(-x^{2}) \, H_{n}(x) \, H_{m}(y) \, \dd{x} = 0, \hspace{1cm} \mbox{ si } n \neq m \]
Con lo que vemos que la ortogonalidad de las funciones propias está asociada a la elección particular del dominio de la variable independiente.
\par
Aplicamos el método de Frobenius para encontrar una solución:
\begin{align*}
H(x) &= \sum_{\alpha = 0}^{\infty} a_{\alpha} \, x^{\alpha+k} \\[1em]
\dot{H}(x) &= \sum_{\alpha = 0}^{\infty} a_{\alpha} \, (\alpha + k) \, x^{\alpha+k-1} \\[1em]
\ddot{H}(x) &= \sum_{\alpha = 0}^{\infty} a_{\alpha} \, (\alpha + k) \, (\alpha + k - 1) \, x^{\alpha+k-2}
\end{align*}
re-emplazando en la ecuación (\ref{eq:ecuacion_08_58}) y factorizando los términos
\[ \sum_{\alpha = 0}^{\infty} a_{\alpha} \, (\alpha + k) \, (\alpha + k - 1) \, x^{\alpha+k-2} - \sum_{\alpha=0}^{\infty} a_{\alpha} \, [ 2 \, (\alpha + k) - 2 \, n] \, x^{\alpha+k} = 0 \]
que puede escribirse como
\[ \sum_{\alpha=-2}^{\infty} a_{\alpha + 2} \, ( \alpha + k + 2) \, (\alpha + k + 1)\, x^{\alpha+k} - \sum_{\alpha=0}^{\infty} a_{\alpha} \, [2 \, (\alpha + k) - 2 \, n] \, x^{\alpha+k} = 0 \]
escribiendo los dos primeros términos de la serie
\begin{align*}
 a_{0}(k) \, (k - 1)\, x^{k-2} &+ a_{1} \, (k + 1)(k)^{\alpha-1} + \\
&+ \sum_{\alpha=0}^{\infty} \left\{ a_{\alpha + 2} \, (\alpha + k +2) \, (\alpha + k + 1) + \right. \\
&- \left. a_{0} \, [2 (\alpha + k ) - 2 \, n] \right\} \, x^{\alpha+k} = 0 
\end{align*}
donde reconocemos las ecuaciones de índices:
\begin{align*}
a_{0}\, k \, (k-1) &= 0 \\
a_{1} \, k\, (k+1) &= 0 \\
a_{\alpha+2}\, (\alpha + k + 2) \, (\alpha + k + 1) - a_{\alpha} \, [2 \, (\alpha + k) - 2 \, n] &= 0, \hspace{1cm} \alpha=0, 1, 2, \ldots
\end{align*}
Revisamos que:
\begin{enumerate}
\item De la primera expresión: si $a_{0} \neq 0$, entonces $k=0$ ó $k=1$.
\item De la segunda expresión: con $k=0$ se sigue que $a_{1} \neq 0$ y de $k=1$, se concluye que $a_{1} = 0$.
\item De la tercera ecuación de índices con $k=0$, se obtiene
\[ a_{\alpha+2} = \dfrac{2a_{\alpha}(\alpha - n)}{(\alpha + 2)(\alpha + 1)} \hspace{1cm} \alpha = 0, 1, 2, \ldots \]
\end{enumerate}
Explícitamente los primeros coeficientes son
\begin{align*}
a_{2} &= \dfrac{(-) \, 2 \, n}{2!} \, a_{0} \\
a_{3} &= \dfrac{(-) \, 2 \, (n-1)}{3!} \, a_{1} \\
a_{4} &= \dfrac{(-) \, 2 \, a_{2} \, (2 - n)}{4 \times 3} = \dfrac{2^{2} \, (-)^{2} \, (n) \, (n - 2)}{4!} a_{0} \\
\vdots
\end{align*}
Entonces tenemos
\begin{align*}
H(x) &= a_{0} \, \left[ 1 + \dfrac{(-) 2 \, n}{2!}\, x^{2} + \dfrac{(-) \, 2^{2} \, n \, (n - 2)}{4!}\, x^{4} + \right. \\
&+ \left. \dfrac{(-)^{3} \, 2^{3} \, n \, (n-2)(n-4) }{6!}\, x^{6} + \ldots \right] + \\
&+ a_{1} \, \left[ x + \dfrac{(-) \, 2 \, (n-1)}{3!} \, x^{3} + \dfrac{(-)^{2} \, 2^{2} \, (n-1)(n-3)}{5!}\, x^{5} + \right.\\
&+ \left. \dfrac{(-)^{3} \, 2^{3} \, (n-1)(n-3)(n-5)}{7!} \, x^{7} + \ldots \right]
\end{align*}
Ambas series son divergentes en $x \to \pm \infty$. Si se incluyen estos dos extremos y se quiere lograr convergencia es necesario cortar las series y convertirlas en polinomios.
\par
La serie en $a_{0}$ requiere que $n$ = par positivo y la serie en $a_{1}$ requiere $n$ = impar positivo. Puesto que $n$ no puede ser simultáneamente par e impar en las series para $a_{0}$ y $a_{1}$, si $n$ = par, la segunda serie es divergente y si $n$ = impar la primera diverge.
\par
Las series convergentes serán las de nuestro interés, y conformarán los \emph{Polinomios de Hermite}.
\\
Puede demostrarse que con $n \neq$ entero, $H(x) \propto x^{2} \, \exp(x^{2})$ para $x \to \pm \infty$, lo que muestra la no convergencia para $n$ no entero.
\par
Consideremos $n$ = par. La serie para $a_{0}$, con $n=2\, m, \; m = 0, 1, 2, \ldots$ será
\begin{align*}
H(x) &= a_{0} \, \left[ 1 + \dfrac{(-) \, 2^{2} \, m}{2!} \, x^{2} + \dfrac{(-)^{2} \, 2^{4} \, m \, (m - 1)}{4!} \, x^{4} + \right. \\
&+ \left. \dfrac{(-)^{3} \, 2^{6} \, m \, (m-1)(m-2)}{6!} \, x^{6} + \ldots \right] \\[1em]
H(x) &= a_{0}\, \left[ 1 + \dfrac{(-)\, 2^{2}\, m}{2!} \, x^{2} + \ldots + \dfrac{(-)^{3} \, 2^{6}\, m!}{(m-3)! \, 6!} \, x^{6} + \ldots \right. \\[0.5em]
&+ \left. \dfrac{(-)^{p} \, (2 \, x)^{2\, p}\, m!}{(m-p)! \, (2\, p)!} + \ldots \right] \\[0.5em]
&= m!\, a_{0} \, \sum_{p=0}^{\infty} \dfrac{(-)^{p}\, (2\, x)^{2\, p}}{(m-p)! \, (2\, p)!}
\end{align*}
Se definen los polinomios de Hermite de orden par, en la forma
\begin{equation}
H_{2\, m} (x) = (-)^{m} \, (2\, m)! \sum_{p=0}^{\infty} \dfrac{(-)^{p} \, (2\, x)^{2\, p}}{(m-p)!\, (2p)!}
\label{eq:ecuacion_08_60}
\end{equation}
Revisemos que efectivaente la ecuación (\ref{eq:ecuacion_08_60}) es un polinomio, ya que para $p > m : 1 / (m-p)! \to 0$. Esto significa que la suma se extiende entre $0$ y $m$.
\par
De manera análoga, para $n=$ impar, con $n = 2 \, m + 1, \hspace{0.5cm} m=0, 1, 2, \ldots$
\begin{align*}
H(x) & = a_{1} \, \left[ x + \dfrac{(-) \, 2^{2} \, m}{3!} \, x^{2} + \dfrac{(-)^{2} \, 2^{4} \, m \, (m - 1)}{5!} \, x^{5} + \right. \\
& + \left. \dfrac{(-)^{3} \, 2^{6} \, m \, (m-1)(m-2)}{7!} \, x^{7} + \ldots \right] \\[1em]
H(x) & = a_{1}\, \left[ x + \dfrac{(-)\, 2^{2}\, m}{3!} \, x^{3} + \ldots + \dfrac{(-)^{3} \, 2^{6}\, m!}{(m-3)! \, 7!} \, x^{7} + \ldots + \right. \\[0.5em]
&+  \left. \dfrac{(-)^{p} \, 2^{2 \, p} \, m!}{(m-p)! \, (2\, p + 1)!} \, x^{2p+1} + \ldots \right] \\[0.5em]
& = a_{1} \, m! \sum_{p=0}^{\infty} \dfrac{(-)^{p}\, (2\, x)^{2\, p+1}}{(m-p)! \, (2\, p + 1)!}  
\end{align*}
Definimos los polinomios de Hermite de orden impar como:
\begin{equation}
H_{2m+1} (x) = (-)^{m} \, (2 \, m + 1)! \, \sum_{p=0}^{\infty} \dfrac{(-)^{p} \, (2 \, x)^{2p+1}}{(m-p)! \, (2 \, p + 1)!}
\label{eq:ecuacion_08_61}
\end{equation}
donde la suma da términos diferentes de cero sólo entre $0$ y $m$. En forma compacta se tiene la expresión:
\begin{equation}
\boxed{H_{n}(x) = n! \, \sum_{p=0}^{N} \dfrac{(-)^{p}}{(n - 2 \, p)! \, p!} \, (2 \, x)^{n-2p}}
\label{eq:ecuacion_08_62}
\end{equation}
con $N=n/2$ si $n$ es par, o $N=n-1/2$ si $n$ es impar.
\par
La primera ecuación de índices provee para el exponente $k$ en la serie de Frobenius un segundo valor: $k = 1$. Es directo comprobar que nada nuevo se añade al desarrollo anterior.
\par
Los primeros polinomios de Hermite son:
\begin{align*}
H_{0}(x) &= 1 \\
H_{1}(x) &= 2 \, x \\
H_{2}(x) &= 4 \, x^{2} - 2 \\
H_{3}(x) &= 8 \, x^{3} - 12 \, x \\
H_{4}(x) &= 16 \, x^{4} - 48 \, x^{2} + 12 \\
H_{5}(x) &= 32 \, x^{5} - 160 \, x^{3} + 120 \, x
\end{align*}
La normalización de los polinomios ha sifo elegida de modo tal que $H_{0}(x) = 1$.
\par
Algunas de las propiedades de los polinomios de Hermite son:
\begin{align*}
H_{n} (x) &= (-)^{n} \, e^{x^{2}} \, \dv[n]{n} \left( e^{-x^{2}} \right) \\
H_{n+1} (x) &= 2 \, x \, H_{n} (x) - 2 \, n \, H_{n-1} (x) \\
\dot{H}_{n} (x) &= 2 \, n \, H_{n-1} \\
H_{n} (x) &= (-)^{n} \, H_{n} (-x) \\
H_{2n} (0) &= (-)^{n} \, \dfrac{(2 \, n)!}{n!} \\
H_{2n+1} (0) &= 0
\end{align*}
La primera de estas se conoce como fórmula de Rodrigues.
\par
La normalización de la integral de ortogonalidad de los polinomios $\left\{ H_{n} (x) \right\}$ puede hacerse utilizando la siguiente identidad, que define la función generatriz de los polinomios de Hermite:
\[ \exp \left( -t^{2} + 2 \, x \, t \right) = \sum_{n=0}^{\infty} \dfrac{H_{n} (x) \, t^{n}}{n!} \]
se sigue que:
\begin{align*}
\exp \left( -x^{2} \right) \, & \exp \left( -t^{2} + 2 \, x \, t \right) \, \exp \left( -s^{2} + 2 \, x \, s \right) = \\
&= \sum_{n=0}^{\infty} \, \sum_{m=0}^{\infty} \dfrac{e^{-x^{2}}}{n! \, m!} \, t^{n} \, s^{m} \, H_{n} (x) \, H_{m} (x)
\end{align*}
Integrando con respecto a $x$ en el intervalo $(-\infty, \infty)$ y considerando que
\[ \exp \left( -x^{2} \right) \, \exp \left( -t^{2} + 2 x t \right) \, \exp \left( -s^{2} + 2 x s \right) = \exp \left( - (x - s - t)^{2} \right) \, \exp \left( 2 s t \right) \]
y con el resultado
\[ \int_{-\infty}^{\infty} e^{-x^{2}} \, H_{n} (x) \, H_{m} (x) \, \dd x = \delta_{m n} \int_{-\infty}^{\infty} e^{-x^{2}} \, H_{n}^{2} (x) \, \dd x \]
se obtiene:
\[ \sum_{n=0}^{\infty} \dfrac{(s t)^{n}}{(n!)^{2}} \int_{-\infty}^{\infty} e^{-x^{2}} \, H_{n}^{2} (x) \, \dd x = \sqrt{\pi} \, e^{2 s t } = \sqrt{\pi} \, \sum_{n=0}^{\infty} \sum_{n=0}^{\infty} \dfrac{2^{n} \, (s t)^{n}}{n!} \]
de donde:
\[ \int_{-\infty}^{\infty} e^{-x^{2}} \, H_{n}^{2} (x) \, \dd x = 2^{n} \, \sqrt{\pi} \, n! \]
En consecuencia, la condición de ortogonalidad es:
\begin{equation}
\boxed{ \int_{-\infty}^{\infty} e^{-x^{2}} \, H_{n} (x) \, H_{m} (x) \dd x = 2^{n} \, \sqrt{\pi} \, n! \, \delta_{m n} \hspace{1cm} n = 0, 1, 2, \ldots }
\end{equation}
El conjunto $\left\{ H_{n }(x) \right\}$ es completo; por tanto cualquier función $f(x)$ definida en el intervalo $(-\infty, \infty)$ puede expandirse en polinomios de Hermite:
\[ f(x) = \sum_{n=0}^{\infty} C_{n} \, H_{n} (x) \]
En forma general puede afirmarse que cualquier función definida en $(-\infty, \infty)$ puede expandirse en cualquier base ortogonal definida en $(-\infty, \infty)$.
\par
Expresar la función en una u otra es cambiar de base: la misma función puede expandirse, por ejemplo, en la base de Hermite $\left\{ H_{n }(x) \right\}$, o en la de Fourier $\left\{ e^{i k x} \right\}$.
\section{La familia de la ecuación de Hermite.}
A partir de la ecuación de Hermite y utilizando la transformación
\[ H_{n} (x) = e^{\alpha x^{2}} \, \psi_{n} (\mu), \hspace{2cm} \mu = x^{b} \]
se obtiene la primera familia de Hermite
\begin{align*}
&{} b^{2} \, \dv[2]{\psi_{n} (\mu)}{\mu} \, \mu^{2 (b-1)/b} + \dv{\psi_{n}(\mu)}{\mu} \left[ b \, (b-1) \mu^{\frac{b-2}{2}}  + \right. \\
&+\left. 2 \, b \, (2 \, a - 1) \, \mu \right] + \left[ 4 \, a \, (a - 1) \, \mu^{2/b} + 2 \, n + 2 \, a \right] \, \psi_{n} (\mu) = 0
\end{align*}
cuya solución es
\[ \psi_{n} (\mu) = e^{-\alpha x^{2}} \, H_{n} (x) = \exp \left( - \dfrac{a x^{2}}{b} \right) \, H_{n} (\mu^{1/b}) \]
\begin{enumerate}[label=\roman*.)]
\item Con $a = 0, b = 1$, se recupera la ecuación de Hermite.
\item Si $b = 1, a = 1/2$, se obtiene la ecuación de \textbf{Weber-Hermite}:
\begin{equation}
\dv[2]{\psi_{n} (x)}{x} + [ 1 + 2 \, n - x^{2} ] \, \psi_{n} (x) = 0
\label{eq:ecuacion_08_65}
\end{equation}
\end{enumerate}
Con $\mu = x$ y $\psi_{n} (x) = e^{-x^{2}/2} / H_{n} (x)$. Esta última describe el oscilador armónicos unidimensional en mecánica cuántica. Nótese que la base $\left\{ \psi_{n} (x) \right\}$ es ortonormal.
\section{El oscilador armónico cuántico.}
Como una aplicación importante de los polinomios de Hermite consideremos la cuantización de la energía del oscilador armónico.
\par
De acuerdo con la mecánica cuántica, un oscilador armónico unidimensional, cuya energía potencial es $V = \frac{1}{2} k \, x^{2}$ puede describirse mediante la ecuación de Schrödinger:
\[ - \dfrac{\hbar^{2}}{2 \, m} \, \dv[2]{\psi (x)}{x} + V (x) \, \psi (x) =  E \, \psi (x) \]
donde $\psi (x)$ representa la función de onda, $\hbar$ es la constante de Planck dividida entre $2 \, \pi$, $m$ es la masa del oscilador y $E$ su energía total. 
\par
Cambiando a la nueva variable adimensional: $y = x/\alpha$, donde $\alpha$ tendrá la misma dimensión que $x$, y utilizando $\omega^{2} = k/m$, siendo $\omega$ la frecuencia angular del oscilador y $k$ la constante del resorte, podremos escribir:
\[ \dv[2]{\psi}{y} - \left( \dfrac{\omega \, m \, \alpha^{2}}{\hbar}  \right)^{2} \, y^{2} \, \psi  + \left( \dfrac{2 \, m \, E \, \alpha^{2}}{\hbar^{2}} \right) \, \psi = 0 \] 
La adimensionalidad de $y$ y la homogeneidad en $\psi$ de la ecuación permiten escoger un valor para $\alpha$, tal que el primer paréntesis tenga el valor de $1$, esto es:
\[ \dfrac{\omega \, m \, \alpha}{\hbar} = 1 \hspace{1.5cm} \mbox{de donde } \alpha^{2} = \dfrac{\hbar}{m \, \omega} \]
El segundo paréntesis (adimensional), lo llamaremos $\lambda$:
\[ \lambda = \dfrac{2 \, m \, E \, \alpha^{2}}{\hbar} = \dfrac{2 \, E}{\hbar \, \omega} \]
Por lo que la ecuación del oscilador toma la forma:
\begin{equation}
\dv[2]{\psi}{y} + (\lambda - y^{2}) \, \psi = 0
\label{eq:ecuacion_08_66}
\end{equation}
Esta expresión, conocida como \emph{ecuación de Weber-Hermite}, corresponde, a la primera familia de Hermite con $b = 1$ y $a = 1/2$, tal que su solución es la función de Weber-Hermite de orden $n$ entero:
\begin{equation}
\psi_{n} (y) = e^{y^{2}/2} \, H_{n} (y)
\label{eq:ecuacion_08_67}
\end{equation}
y por tanto $1 + 2 \, n = \lambda$
\par
Dado que $\lambda = 2 \, E / \hbar \, \omega$, se tiene que
\begin{align}
\dfrac{2 \, E}{\hbar \, \omega} &= 1 + 2 \, n \nonumber \\
E &= \left(n + \dfrac{1}{2} \right) \, \hbar \, \omega. \hspace{1.5cm} n \geq 0
\label{eq:ecuacion_08_68}
\end{align}
Como $n$ toma valores enteros, se sigue que la energía del oscilador está cuantizada y que hay una energía mínima o de punto cero: $E_{min} = \hbar \, \omega /2$.
\par
La secuencia de niveles de energía (\ref{eq:ecuacion_08_68}) tiene el espaciamiento $\Delta E = \hbar \, \omega$ postulado por Planck en 1900. Lo notable es que hay un mínimo en la energía que no aparece en la teoría de Planck y que es exigido por el principio de incertidumbre: \emph{un oscilador armónico no puede estar en reposo}. Si lo estuviera, sería en $x = 0$ que es el punto de equilibrio; en consecuencia podríamos conocer simultáneamente su posición y velocidad, lo que no es compatible con el principio de incertidumbre de Heisenberg.
\par
La función de onda normalizada del oscilador será entonces, de acuerdo con la ec. (\ref{eq:ecuacion_08_67}):
\begin{align}
\psi_{n} (y) = \dfrac{1}{(2^{2} \, \sqrt{\pi} \, n!)^{1/2}} \, e^{-y^{2}/2} \, H_{n} (y) \hspace{1.5cm} y = \dfrac{x}{\alpha} = x \, \sqrt{\dfrac{m \, \omega}{\hbar}}
\label{eq:ecuacion_08_69}
\end{align}
Explícitamente, los primeros niveles de energía tienen la forma:
\begin{align*}
\psi_{0} (y) &= e^{-y^{2}/2} \hspace{2.6cm} E = \hbar \, \omega /2 \\
\psi_{1} (y) &= 2 \, x \, e^{-y^{2}/2} \hspace{2cm} E = 3 \, \hbar \, \omega /2 \\
\psi_{2} (y) &= (4 \, y^{2} - 2) \, e^{-y^{2}/2} \hspace{0.7cm} E = 5 \, \hbar \, \omega /2 \\
\end{align*}
\section{Operadores escalera.}
Una de las identidades que satisface la función de Hermite es:
\begin{equation}
H_{n-1} (x) = \dfrac{1}{2 \, n} \, \dv{x} H_{n} (x)
\label{eq:ecuacion_08_70}
\end{equation}
Según esta expresión, dado un $H_{n} (x)$ todos los anteriores pueden ser deducidos de él. Otra de las identidades es
\[ H_{n+1} (x) = 2 \, x \, H_{n} (x) - 2 \, n \, H_{n-1} (x) \]
Si de estas dos ecuaciones se elimina $H_{n-1} (x)$, se obtiene
\begin{equation}
H_{n+1} (x) = \left( 2 \, x - \dv{x} \right) \, H_{n} (x)
\label{eq:ecuacion_08_71}
\end{equation}
expresión que nos dice: dado un $H_{n} (x)$ todos los que le siguen pueden deducirse de él. Basta entonces con un $Hn(x)$ para generar los demás. Los operadores 
\[ \dfrac{1}{2 \, n} \dv{x} \hspace{2cm} 2 \, x - \dv{x} \]
son los operadores escalera de los polinomios de Hermite.
\par
En lo que sigue introduciremos los operadores escalera asociados a las funciones de onda del oscilador armónico cuántico dadas por la ecuación (\ref{eq:ecuacion_08_69}). Reemplazando $H_{n} (y)$ de la ecuación (\ref{eq:ecuacion_08_69}) en las ecuaciones (\ref{eq:ecuacion_08_70}) y (\ref{eq:ecuacion_08_71}) obtenemos:
\begin{align*}
\sqrt{2 \, n} \, \psi_{n-1} (y) &= \left( y + \dv{x} \right) \, \psi_{n} (y) \\
\sqrt{2 \, (n + 1)} \, \psi_{n+1} (y) &= \left( y - \dv{x} \right) \, \psi_{n} (y)
\end{align*}
que pueden ser escritos como:
\begin{align}
\begin{aligned}
\hat{a} \, \psi_{n} &= \sqrt{n} \, \psi_{n-1} \\
\hat{a}^{\dagger} \, \psi_{n} &= \sqrt{n+1} \, \psi_{n+1}
\end{aligned}
\label{eq:ecuacion_08_72}
\end{align}
donde los operadores $\hat{a}$ y $\hat{a}^{\dagger}$ están definidos por las ecuaciones
\[ \hat{a} = \dfrac{1}{\sqrt{2}} \, \left( y + \dv{y} \right) \hspace{1.5cm} \hat{a}^{\dagger} = \dfrac{1}{\sqrt{2}} \, \left( y - \dv{y} \right) \]
Estos operadores bajan y suben, respectivamente, estados cuánticos del oscilador. De la primera ecuación con $n = 0$ se obtiene la función de onda normalizada del estado base:
\[ \psi_{0} (y) = \pi^{-1/4} \, e^{-y^{2}/2} \]
A partir de ésta, pueden calcularse todas las funciones de onda del oscilador armónico mediante la expresión:
\begin{equation}
\psi_{n} = \dfrac{1}{\sqrt{n!}} \, \left( a^{\dagger} \right)^{n} \, \psi_{0}
\label{eq:ecuacion_08_73}
\end{equation}
\end{document}