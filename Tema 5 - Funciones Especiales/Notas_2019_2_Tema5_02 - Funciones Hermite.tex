\documentclass[12pt]{article}
\usepackage[utf8]{inputenc}
\usepackage[spanish,es-lcroman, es-tabla]{babel}
\usepackage[autostyle,spanish=mexican]{csquotes}
\usepackage{amsmath}
\usepackage{amssymb}
\usepackage{nccmath}
\numberwithin{equation}{section}
\usepackage{amsthm}
\usepackage{graphicx}
\usepackage{epstopdf}
\DeclareGraphicsExtensions{.pdf,.png,.jpg,.eps}
\usepackage{color}
\usepackage{float}
\usepackage{multicol}
\usepackage{enumerate}
\usepackage[shortlabels]{enumitem}
\usepackage{anyfontsize}
\usepackage{anysize}
\usepackage{array}
\usepackage{multirow}
\usepackage{enumitem}
\usepackage{cancel}
\usepackage{tikz}
\usepackage{circuitikz}
\usepackage{tikz-3dplot}
\usetikzlibrary{babel}
\usepackage{bm}
\usepackage{mathtools}
\usepackage{esvect}
\usepackage{hyperref}
\usepackage{relsize}
\usepackage{siunitx}
\usepackage{physics}
%\usepackage{biblatex}
\usepackage{standalone}
\usepackage{mathrsfs}
\usepackage{bigints}
\usepackage{bookmark}
\spanishdecimal{.}

\setlist[enumerate]{itemsep=0mm}

\renewcommand{\baselinestretch}{1.5}

\let\oldbibliography\thebibliography

\renewcommand{\thebibliography}[1]{\oldbibliography{#1}

\setlength{\itemsep}{0pt}}
%\marginsize{1.5cm}{1.5cm}{2cm}{2cm}


\newtheorem{defi}{{\it Definición}}[section]
\newtheorem{teo}{{\it Teorema}}[section]
\newtheorem{ejemplo}{{\it Ejemplo}}[section]
\newtheorem{propiedad}{{\it Propiedad}}[section]
\newtheorem{lema}{{\it Lema}}[section]

\author{}
%\author{M. en C. Gustavo Contreras Mayén. \texttt{curso.fisica.comp@gmail.com}}
\title{Polinomios de Hermite - El oscilador armónico cuántico\\ {\large Tema 5 - Matemáticas Avanzadas de la Física}\vspace{-1.5\baselineskip}}
\date{ }
\begin{document}
\maketitle
\fontsize{14}{14}\selectfont
%Referencia: Luis de la Peña. Introducción a la mecánica cuántica. Cap. 11
\section{Introducción.}
Pasaremos ahora a estudiar con detenimiento uno de los sistemas físicos de
mayor interés teórico, \textbf{el oscilador armónico cuántico}. Para simplificar el tratamiento
nos restringimos al caso unidimensional, que carece de degeneración (el
oscilador armónico en más de una dimensión es un problema degenerado que,
para su tratamiento cabal, requiere en la teoría clásica de la teoría del momento
angular.
\par
El oscilador armónico ocurre en el caso cuántico de manera similar a como se da clásicamente:
\emph{aparece cuando potenciales más o menos complicados se aproximan en la región
de un mínimo con curvas parabólicas}. Por lo tanto, se trata esencialmente de la
teoría de oscilaciones poco energéticas en torno a un mínimo de potencial, situación
que se da con frecuencia con átomos, moléculas, etc. Además, el empleo frecuente
del análisis de Fourier para estudiar sistemas complejos en términos de osciladores
elementales hace del oscilador armónico uno de los problemas más importantes de
la física teórica, sin considerar su interés metodológico debido al hecho de poseer
solución exacta.
\par
En esta sección nuestro interés primordial se centra en el estudio de la dinámica
del sistema. El problema que vamos a abordar ahora (la solución de la ecuación
completa de Schrödinger) es más complicado que el de determinar los eigenestados
y el espectro del oscilador armónico cuántico, pero preferimos iniciar el estudio del
oscilador mostrando que se trata en efecto de osciladores, exhibiendo y analizando
las oscilaciones. Con esto, tendremos oportunidad de ver en qué son similares y
en qué se distinguen los osciladores clásicos de los cuánticos. Con esta idea en
mente, iniciamos el tema con el estudio de un paquete de osciladores armónicos,
posponiendo un poco el de los estados propios del oscilador. Por otro lado sucede
que, de todos los posibles paquetes de osciladores, hay algunos muy particulares
cuyo comportamiento es el más simple. Puesto que nos interesa —por claridad—
mantener la máxima simplicidad posible, nos limitaremos al estudio de estos últimos,
que son los paquetes (gaussianos) de mínima dispersión y máxima coherencia, como
tendremos oportunidad de comprobar.
El potencial de un oscilador armónico cuántico unidimensional está dado por la
expresión usual
%Referencia: Graton. Introducción a la mecánica cuántica. Cap. 9
\begin{equation}
V(x) = \dfrac{1}{2} \, m  \, \omega^{2} \, x^{2}
\label{eq:ecuacion_09_068}
\end{equation}
donde $\omega$ es la frecuencia clásica del oscilador y $m$ la masa. La forma de la ec. (\ref{eq:ecuacion_09_068}) del potencial $V (x)$ es de gran importancia práctica, pues es una aproximación para cualquier energía potencial arbitraria en el entorno de un punto de equilibrio estable. 
\par
El oscilador armónico simple es también importante porque el comportamiento de sistemas tales como las vibraciones de un medio elástico y del campo electromagnético en una cavidad se pueden describir como la superposición de un número infinito de osciladores armónicos simples. Al cuantificar esos sistemas nos encontramos entonces con la mecánica cuántica de muchos osciladores armónicos lineales de diferentes frecuencias. Por tal motivo, todas las teorías de campos modernas utilizan los resultados que vamos a obtener.
\par
El Hamiltoniano del oscilador armónico simple es
\begin{equation}
H = \dfrac{p^{2}}{2 m} + \dfrac{1}{2} \, m \, \omega^{2} \, x^{2}
\label{eq:ecuacion_09_069}
\end{equation}
por lo que la ecuación de Schrödinger independiente del tiempo es:
\begin{equation}
- \dfrac{h^{2}}{2 m} \dv[2]{\psi}{x} + \dfrac{1}{2} m \, \omega^{2} \, x^{2} \, \psi = E \, \psi
\label{eq:ecuacion_09_070}
\end{equation}
Para un manejo más sencillo de las expresiones, sustituimos las variables $x \to \xi$ y $p \to \eta$, definidos por
\begin{align}
\begin{aligned}
x &= \xi \, \sqrt{\dfrac{\hbar}{m \omega}} \\[1em]
p &= \eta \, \sqrt{m \, \hbar \, \omega}
\end{aligned}
\label{eq:ecuacion_07_071}
\end{align}
Se puede comprobar fácilmente que $\xi$ y $\eta = - i \dv*{\xi}$ cumplen la relación de conmutación
\begin{equation}
[\xi, \eta] = \xi \, \eta - \eta \, \xi = i
\label{eq:ecuacion_09_072}
\end{equation}
En términos de  $\xi$ y $\eta$, el Hamiltoniano se escribe
\begin{equation}
H =  \hbar \, \omega \mathcal{H} \hspace{1.5cm} \mathcal{H} = \dfrac{1}{2} (\eta^{2} + \xi^{2})
\label{eq:ecuacion_09_073}
\end{equation} 
entonces la ec. (\ref{eq:ecuacion_09_070}) toma la forma
\begin{equation}
\dv[2]{\psi}{\xi} + (2 \, \varepsilon - \xi^{2}) \, \psi = 0 \hspace{1.5cm} E = \hbar \, \omega \, \varepsilon
\label{eq:ecuacion_09_074}
\end{equation}
Veamos el comportamiento de $\psi (\xi)$ cuando $\xi \to \pm \infty$. Para valores finitos de $\varepsilon$, es fácil verificar que
\begin{equation}
\psi (\xi \to \pm \infty) \rightarrow \exp(-\xi^{2}/2)
\label{eq:ecuacion_09_075}
\end{equation}
de manera que $\psi$ tiene el comportamiento de una gaussiana.
\par
Es inmediato verificar por sustitución directa que
\begin{equation}
\psi_{0} (\xi) = \exp(-\xi^{2}/2)
\label{eq:ecuacion_09_076}
\end{equation}
es una soluci{on de la ec. (\ref{eq:ecuacion_09_074}) y corresponde al valor propio $\varepsilon = 1/2$. En efecto, si $\varepsilon = 1/2$, se cumplen
\begin{equation}
\dv[2]{\psi_{0}}{\xi} + (2 \, \varepsilon - \xi^{2}) \, \psi_{0} = - \psi_{0} + \xi^{2} \, \psi_{0} + (2 \, \varepsilon - \xi^{2}) \, \psi_{0} = 0
\label{eq:ecuacion_09_077}
\end{equation}
Para encontrar las demás funciones propias y valores propios, vamos a usar una sencilla y elegante técnica de operadores, que es diferente de los métodos que se emplean habitualmente en los textos elementales de Mecánica Cuántica. Hacemos así porque esta técnica es el prototipo de otras semejantes que se aplican en una variedad de problemas.
\par
El método se funda en las propiedades de conmutación de ciertos operadores no Hermitianos oportunamente definidos, y permite encontrar sistemáticamente mediante un procedimiento recursivo todas las funciones propias y sus correspondientes valores propios a partir de $\psi_{0}$ y de $\varepsilon$. Para eso definimos el operador
\begin{equation}
a = \dfrac{1}{\sqrt{2}} (\xi + i \, \eta) = \dfrac{1}{\sqrt{2}} \left( \xi + \dv{\xi} \right)
\label{eq:ecuacion_09_078}
\end{equation}
que no es Hermitiano, y su adjunto es
\begin{equation}
a^{\dagger} = \dfrac{1}{\sqrt{2}} (\xi - i \, \eta) = \dfrac{1}{\sqrt{2}} \left( \xi - \dv{\xi} \right)
\label{eq:ecuacion_09_079}
\end{equation}
En términos de $a$ y $a^{\dagger}$, el operador $\mathcal{H}$ se expresa como
\begin{equation}
\mathcal{H} = a^{\dagger} \, a + \dfrac{1}{2}
\label{eq:ecuacion_09_080}
\end{equation}
El conmutador $[a, a^{\dagger}]$ es
\begin{equation}
[a, a^{\dagger}] = a \, a^{\dagger} - a^{\dagger} \, a = 1
\label{eq:ecuacion_09_081}
\end{equation}
Puesto que $\mathcal{H}$ y $a^{\dagger} \, a$ conmutan, las funciones propias de $\mathcal{H}$ y $a^{\dagger} \, a$ son las mismas, de modo que para encontrar los estados estacionarios es suficiente resolver el problema de valores propios de $a^{\dagger} \, a$. Si llamamos $\lambda_{n}\, (n = 0, 1, 2, \ldots)$ a los valores propios y $\psi_{n}$ las correspondientes funciones propias, la ecuación que queremos resolver es
\begin{equation}
a^{\dagger} \, a \, \psi_{n} = \lambda_{n} \, \psi_{n}
\label{eq:ecuacion_09_082}
\end{equation}
Primero veremos que los valores propios no pueden ser negativos. De la ec. (\ref{eq:ecuacion_09_082}) se tiene que
\begin{equation}
(\psi_{n}, a^{\dagger} \, a \, \psi_{n}) = ( a \, \psi_{n}, a \, \psi_{n} ) = \lambda_{n} (\psi_{n}, \psi_{n})
\label{eq:ecuacion_09_083}
\end{equation}
donde usamos la definición de operador adjunto. Puesto que la norma de una función no puede
ser negativa, concluimos que
\begin{equation}
\lambda_{n} \geq 0
\label{eq:ecuacion_09_084}
\end{equation}
Si $\psi_{k}$ es una función propia de $a^{\dagger} \, a$, entonces $a^{\dagger} \, \psi_{k}$ es también una función propia; en efecto usando la relación de conmutación (\ref{eq:ecuacion_09_081}), vemos que:
\begin{equation}
(a^{\dagger} \, a) \, a^{\dagger} \, \psi_{k} = a^{\dagger} (a^{\dagger} \, a + 1) \, \psi_{k} = (\lambda_{k} + 1) \, a^{\dagger} \, \psi_{k}
\label{eq:ecuacion_09_085}
\end{equation}
Por lo tanto $a^{\dagger} \, \psi_{k}$ es una función propia con valor propio $\lambda_{k} + 1$. Del mismo modo se obtiene
\begin{equation}
(a^{\dagger} \, a) \, a \, \psi_{k} = a (a^{\dagger} \, a + 1) \, \psi_{k} = (\lambda_{k} + 1) \, a \, \psi_{k}
\label{eq:ecuacion_09_086}
\end{equation}
que muestra que $a \, \psi_{k}$ es una autofunción de $a^{\dagger} \, a$ con autovalor $\lambda_{k} - 1$ . Debido a estas propiedades $a^{\dagger}$ y $a$ se denominan \emph{operador de subida y operador de bajada}, respectivamente.
\par
Operando reiteradamente con $a^{\dagger}$ y $a$ sobre una función propia $\psi_{k}$ dada, podemos generar nuevas funciones propias correspondientes a diferentes valores propios, del mismo modo como se suben o se bajan los peldaños de una escalera.
\par
Sin embargo, la condición (\ref{eq:ecuacion_09_084}) limita la cantidad de veces que se
puede aplicar el operador de bajada, porque cuando se llega un valor propio $0 leq \lambda_{0} < 1$, la aplicación del operador de bajada no permite ya encontrar una nueva función propia, pues sería una función propia correspondiente a un valor propio que viola la condición (\ref{eq:ecuacion_09_084}). Por lo tanto para el peldaño más bajo de la escalera $(n = 0)$ se debe cumplir
\begin{equation}
a^{\dagger} \, a \, \psi_{0} = \lambda_{0} \, \psi_{0}, \hspace{1.5cm} 0 \leq \lambda_{0} < 1
\label{eq:ecuacion_09_087}
\end{equation}
y también
\begin{equation}
a \, \psi_{0} = 0
\label{eq:ecuacion_09_088}
\end{equation}
y por consiguiente, el menor valor propio de $a^{\dagger} \, a$ es
\begin{equation}
\lambda_{0} = 0
\label{eq:ecuacion_09_089}
\end{equation}
Partiendo entonces de $\psi_{0}$ y de $\lambda_{0} = 0$ podemos obtener todas las demás funciones propias y valores propios por aplicación reiterada del operador de subida $a^{\dagger}$. Pero nosotros ya conocemos $\psi_{0}$, que está dado por la ec. (\ref{eq:ecuacion_09_076}):
\begin{equation}
\psi_{0} (\xi) = \exp(- \xi^{2}/2) 
\label{eq:ecuacion_09_090}
\end{equation}
Por consiguiente la $n-$ésima función propia, y su correspondiente valor propio son
\begin{equation}
\psi_{n} \propto \left( a^{\dagger} \right)^{n} \, \psi_{0} (\xi) = \left[ \dfrac{1}{\sqrt{2}} \left( \xi - \dv{\xi} \right) \right]^{n} \, \exp(-\xi^{2}/2), \hspace{1cm} \lambda_{n} = n
\label{eq:ecuacion_09_091}
\end{equation}
Usando las ecs. (\ref{eq:ecuacion_09_074}) y (\ref{eq:ecuacion_09_080}), obtenemos que
\begin{equation}
H \, \psi_{n} = \hbar \, \omega (n + \dfrac{1}{2}) \, \psi_{n}
\label{eq:ecuacion_09_092}
\end{equation}
por lo que los valores propios de la energía son
\begin{equation}
E_{n} = \hbar \, \omega (n + \dfrac{1}{2}) \hspace{1cm} n = 0, 1, 2, \ldots
\label{eq:ecuacion_09_093}
\end{equation}
Observamos que a diferencia del caso clásico, la energía del oscilador \emph{no es nula} en el estado fundamental $(n = 0)$ sino que todavía vale $\hbar \omega / 2$ . Este resultado de la teoría de Schrödinger difiere del que se obtuvo en la Teoría Cuántica Antigua a partir de los postulados de cuantificación Planck y de Wilson-Sommerfeld. La energía $\hbar \omega / 2$ se denomina energía de punto cero del oscilador armónico y su existencia es un fenómeno cuántico que se puede entender en base al principio de incertidumbre.

%Referencia:: Sepúlveda - Lecciones de Física Matemática Cap. 8.5 Polinomios de Hermite.
% \section{Funciones de Hermite.}
% La ecuación de Hermite, tiene una aplicación en física, quizá la más importante que es la del oscilador armónico en mecánica cuántica, tiene la forma:
% \begin{equation}
% \ddot{H}(x) - 2 \, x \,  \dot{H}(x) + 2\, n \, H(x) = 0
% \label{eq:ecuacion_08_58}
% \end{equation}
% La forma autoadjunta para una ED sabemos que es de la forma
% \[ \dv{x} \left[ q(x) \, \dv{H(x)}{x} \right] + r(x) \, H(x) + \lambda \, p(x) \, H(x) = 0 \]
% por lo que para la ec. de Hermite, resulta ser
% \begin{equation}
% \dv{x} \left[ \exp(-x^{2}) \, \dv{H(x)}{x} \right] + 2 \, n \, \exp(-x^{2}) \, H(x) = 0
% \label{eq:ecuacion_08_59}
% \end{equation}
% La solución a esta ecuación forma una base ortogonal con factor de peso $p(x) = \exp(-x^{2})$, en el intervalo $(-\infty, \infty)$, ya que en dicho intervalo
% \[ [q \, W]\eval_{-\infty}^{\infty} = [ \exp(-x^{2}) \, W ]\eval_{-\infty}^{\infty} = 0 \]
% La ortogonalidad queda expresada por
% \[ \int_{-\infty}^{\infty} \exp(-x^{2}) \, H_{n}(x) \, H_{m}(y) \, \dd{x} = 0, \hspace{1cm} \mbox{ si } n \neq m \]
% Con lo que vemos que la ortogonalidad de las funciones propias está asociada a la elección particular del dominio de la variable independiente.
% \par
% Aplicamos el método de Frobenius para encontrar una solución:
% \begin{align*}
% H(x) &= \sum_{\alpha = 0}^{\infty} a_{\alpha} \, x^{\alpha+k} \\[1em]
% \dot{H}(x) &= \sum_{\alpha = 0}^{\infty} a_{\alpha} \, (\alpha + k) \, x^{\alpha+k-1} \\[1em]
% \ddot{H}(x) &= \sum_{\alpha = 0}^{\infty} a_{\alpha} \, (\alpha + k) \, (\alpha + k - 1) \, x^{\alpha+k-2}
% \end{align*}
% re-emplazando en la ecuación (\ref{eq:ecuacion_08_58}) y factorizando los términos
% \[ \sum_{\alpha = 0}^{\infty} a_{\alpha} \, (\alpha + k) \, (\alpha + k - 1) \, x^{\alpha+k-2} - \sum_{\alpha=0}^{\infty} a_{\alpha} \, [ 2 \, (\alpha + k) - 2 \, n] \, x^{\alpha+k} = 0 \]
% que puede escribirse como
% \[ \sum_{\alpha=-2}^{\infty} a_{\alpha + 2} \, ( \alpha + k + 2) \, (\alpha + k + 1)\, x^{\alpha+k} - \sum_{\alpha=0}^{\infty} a_{\alpha} \, [2 \, (\alpha + k) - 2 \, n] \, x^{\alpha+k} = 0 \]
% escribiendo los dos primeros términos de la serie
% \begin{align*}
%  a_{0}(k) \, (k - 1)\, x^{k-2} &+ a_{1} \, (k + 1)(k)^{\alpha-1} + \\
% &+ \sum_{\alpha=0}^{\infty} \left\{ a_{\alpha + 2} \, (\alpha + k +2) \, (\alpha + k + 1) + \right. \\
% &- \left. a_{0} \, [2 (\alpha + k ) - 2 \, n] \right\} \, x^{\alpha+k} = 0 
% \end{align*}
% donde reconocemos las ecuaciones de índices:
% \begin{align*}
% a_{0}\, k \, (k-1) &= 0 \\
% a_{1} \, k\, (k+1) &= 0 \\
% a_{\alpha+2}\, (\alpha + k + 2) \, (\alpha + k + 1) - a_{\alpha} \, [2 \, (\alpha + k) - 2 \, n] &= 0, \hspace{1cm} \alpha=0, 1, 2, \ldots
% \end{align*}
% Revisamos que:
% \begin{enumerate}
% \item De la primera expresión: si $a_{0} \neq 0$, entonces $k=0$ ó $k=1$.
% \item De la segunda expresión: con $k=0$ se sigue que $a_{1} \neq 0$ y de $k=1$, se concluye que $a_{1} = 0$.
% \item De la tercera ecuación de índices con $k=0$, se obtiene
% \[ a_{\alpha+2} = \dfrac{2a_{\alpha}(\alpha - n)}{(\alpha + 2)(\alpha + 1)} \hspace{1cm} \alpha = 0, 1, 2, \ldots \]
% \end{enumerate}
% Explícitamente los primeros coeficientes son
% \begin{align*}
% a_{2} &= \dfrac{(-) \, 2 \, n}{2!} \, a_{0} \\
% a_{3} &= \dfrac{(-) \, 2 \, (n-1)}{3!} \, a_{1} \\
% a_{4} &= \dfrac{(-) \, 2 \, a_{2} \, (2 - n)}{4 \times 3} = \dfrac{2^{2} \, (-)^{2} \, (n) \, (n - 2)}{4!} a_{0} \\
% \vdots
% \end{align*}
% Entonces tenemos
% \begin{align*}
% H(x) &= a_{0} \, \left[ 1 + \dfrac{(-) 2 \, n}{2!}\, x^{2} + \dfrac{(-) \, 2^{2} \, n \, (n - 2)}{4!}\, x^{4} + \right. \\
% &+ \left. \dfrac{(-)^{3} \, 2^{3} \, n \, (n-2)(n-4) }{6!}\, x^{6} + \ldots \right] + \\
% &+ a_{1} \, \left[ x + \dfrac{(-) \, 2 \, (n-1)}{3!} \, x^{3} + \dfrac{(-)^{2} \, 2^{2} \, (n-1)(n-3)}{5!}\, x^{5} + \right.\\
% &+ \left. \dfrac{(-)^{3} \, 2^{3} \, (n-1)(n-3)(n-5)}{7!} \, x^{7} + \ldots \right]
% \end{align*}
% Ambas series son divergentes en $x \to \pm \infty$. Si se incluyen estos dos extremos y se quiere lograr convergencia es necesario cortar las series y convertirlas en polinomios.
% \par
% La serie en $a_{0}$ requiere que $n$ = par positivo y la serie en $a_{1}$ requiere $n$ = impar positivo. Puesto que $n$ no puede ser simultáneamente par e impar en las series para $a_{0}$ y $a_{1}$, si $n$ = par, la segunda serie es divergente y si $n$ = impar la primera diverge.
% \par
% Las series convergentes serán las de nuestro interés, y conformarán los \emph{Polinomios de Hermite}.
% \\
% Puede demostrarse que con $n \neq$ entero, $H(x) \propto x^{2} \, \exp(x^{2})$ para $x \to \pm \infty$, lo que muestra la no convergencia para $n$ no entero.
% \par
% Consideremos $n$ = par. La serie para $a_{0}$, con $n=2\, m, \; m = 0, 1, 2, \ldots$ será
% \begin{align*}
% H(x) &= a_{0} \, \left[ 1 + \dfrac{(-) \, 2^{2} \, m}{2!} \, x^{2} + \dfrac{(-)^{2} \, 2^{4} \, m \, (m - 1)}{4!} \, x^{4} + \right. \\
% &+ \left. \dfrac{(-)^{3} \, 2^{6} \, m \, (m-1)(m-2)}{6!} \, x^{6} + \ldots \right] \\[1em]
% H(x) &= a_{0}\, \left[ 1 + \dfrac{(-)\, 2^{2}\, m}{2!} \, x^{2} + \ldots + \dfrac{(-)^{3} \, 2^{6}\, m!}{(m-3)! \, 6!} \, x^{6} + \ldots \right. \\[0.5em]
% &+ \left. \dfrac{(-)^{p} \, (2 \, x)^{2\, p}\, m!}{(m-p)! \, (2\, p)!} + \ldots \right] \\[0.5em]
% &= m!\, a_{0} \, \sum_{p=0}^{\infty} \dfrac{(-)^{p}\, (2\, x)^{2\, p}}{(m-p)! \, (2\, p)!}
% \end{align*}
% Se definen los polinomios de Hermite de orden par, en la forma
% \begin{equation}
% H_{2\, m} (x) = (-)^{m} \, (2\, m)! \sum_{p=0}^{\infty} \dfrac{(-)^{p} \, (2\, x)^{2\, p}}{(m-p)!\, (2p)!}
% \label{eq:ecuacion_08_60}
% \end{equation}
% Revisemos que efectivaente la ecuación (\ref{eq:ecuacion_08_60}) es un polinomio, ya que para $p > m : 1 / (m-p)! \to 0$. Esto significa que la suma se extiende entre $0$ y $m$.
% \par
% De manera análoga, para $n=$ impar, con $n = 2 \, m + 1, \hspace{0.5cm} m=0, 1, 2, \ldots$
% \begin{align*}
% H(x) & = a_{1} \, \left[ x + \dfrac{(-) \, 2^{2} \, m}{3!} \, x^{2} + \dfrac{(-)^{2} \, 2^{4} \, m \, (m - 1)}{5!} \, x^{5} + \right. \\
% & + \left. \dfrac{(-)^{3} \, 2^{6} \, m \, (m-1)(m-2)}{7!} \, x^{7} + \ldots \right] \\[1em]
% H(x) & = a_{1}\, \left[ x + \dfrac{(-)\, 2^{2}\, m}{3!} \, x^{3} + \ldots + \dfrac{(-)^{3} \, 2^{6}\, m!}{(m-3)! \, 7!} \, x^{7} + \ldots + \right. \\[0.5em]
% &+  \left. \dfrac{(-)^{p} \, 2^{2 \, p} \, m!}{(m-p)! \, (2\, p + 1)!} \, x^{2p+1} + \ldots \right] \\[0.5em]
% & = a_{1} \, m! \sum_{p=0}^{\infty} \dfrac{(-)^{p}\, (2\, x)^{2\, p+1}}{(m-p)! \, (2\, p + 1)!}  
% \end{align*}
% Definimos los polinomios de Hermite de orden impar como:
% \begin{equation}
% H_{2m+1} (x) = (-)^{m} \, (2 \, m + 1)! \, \sum_{p=0}^{\infty} \dfrac{(-)^{p} \, (2 \, x)^{2p+1}}{(m-p)! \, (2 \, p + 1)!}
% \label{eq:ecuacion_08_61}
% \end{equation}
% donde la suma da términos diferentes de cero sólo entre $0$ y $m$. En forma compacta se tiene la expresión:
% \begin{equation}
% \boxed{H_{n}(x) = n! \, \sum_{p=0}^{N} \dfrac{(-)^{p}}{(n - 2 \, p)! \, p!} \, (2 \, x)^{n-2p}}
% \label{eq:ecuacion_08_62}
% \end{equation}
% con $N=n/2$ si $n$ es par, o $N=n-1/2$ si $n$ es impar.
% \par
% La primera ecuación de índices provee para el exponente $k$ en la serie de Frobenius un segundo valor: $k = 1$. Es directo comprobar que nada nuevo se añade al desarrollo anterior.
% \par
% Los primeros polinomios de Hermite son:
% \begin{align*}
% H_{0}(x) &= 1 \\
% H_{1}(x) &= 2 \, x \\
% H_{2}(x) &= 4 \, x^{2} - 2 \\
% H_{3}(x) &= 8 \, x^{3} - 12 \, x \\
% H_{4}(x) &= 16 \, x^{4} - 48 \, x^{2} + 12 \\
% H_{5}(x) &= 32 \, x^{5} - 160 \, x^{3} + 120 \, x
% \end{align*}
% La normalización de los polinomios ha sifo elegida de modo tal que $H_{0}(x) = 1$.
% \par
% Algunas de las propiedades de los polinomios de Hermite son:
% \begin{align*}
% H_{n} (x) &= (-)^{n} \, e^{x^{2}} \, \dv[n]{n} \left( e^{-x^{2}} \right) \\
% H_{n+1} (x) &= 2 \, x \, H_{n} (x) - 2 \, n \, H_{n-1} (x) \\
% \dot{H}_{n} (x) &= 2 \, n \, H_{n-1} \\
% H_{n} (x) &= (-)^{n} \, H_{n} (-x) \\
% H_{2n} (0) &= (-)^{n} \, \dfrac{(2 \, n)!}{n!} \\
% H_{2n+1} (0) &= 0
% \end{align*}
% La primera de estas se conoce como fórmula de Rodrigues.
% \par
% La normalización de la integral de ortogonalidad de los polinomios $\left\{ H_{n} (x) \right\}$ puede hacerse utilizando la siguiente identidad, que define la función generatriz de los polinomios de Hermite:
% \[ \exp \left( -t^{2} + 2 \, x \, t \right) = \sum_{n=0}^{\infty} \dfrac{H_{n} (x) \, t^{n}}{n!} \]
% se sigue que:
% \begin{align*}
% \exp \left( -x^{2} \right) \, & \exp \left( -t^{2} + 2 \, x \, t \right) \, \exp \left( -s^{2} + 2 \, x \, s \right) = \\
% &= \sum_{n=0}^{\infty} \, \sum_{m=0}^{\infty} \dfrac{e^{-x^{2}}}{n! \, m!} \, t^{n} \, s^{m} \, H_{n} (x) \, H_{m} (x)
% \end{align*}
% Integrando con respecto a $x$ en el intervalo $(-\infty, \infty)$ y considerando que
% \[ \exp \left( -x^{2} \right) \, \exp \left( -t^{2} + 2 x t \right) \, \exp \left( -s^{2} + 2 x s \right) = \exp \left( - (x - s - t)^{2} \right) \, \exp \left( 2 s t \right) \]
% y con el resultado
% \[ \int_{-\infty}^{\infty} e^{-x^{2}} \, H_{n} (x) \, H_{m} (x) \, \dd x = \delta_{m n} \int_{-\infty}^{\infty} e^{-x^{2}} \, H_{n}^{2} (x) \, \dd x \]
% se obtiene:
% \[ \sum_{n=0}^{\infty} \dfrac{(s t)^{n}}{(n!)^{2}} \int_{-\infty}^{\infty} e^{-x^{2}} \, H_{n}^{2} (x) \, \dd x = \sqrt{\pi} \, e^{2 s t } = \sqrt{\pi} \, \sum_{n=0}^{\infty} \sum_{n=0}^{\infty} \dfrac{2^{n} \, (s t)^{n}}{n!} \]
% de donde:
% \[ \int_{-\infty}^{\infty} e^{-x^{2}} \, H_{n}^{2} (x) \, \dd x = 2^{n} \, \sqrt{\pi} \, n! \]
% En consecuencia, la condición de ortogonalidad es:
% \begin{equation}
% \boxed{ \int_{-\infty}^{\infty} e^{-x^{2}} \, H_{n} (x) \, H_{m} (x) \dd x = 2^{n} \, \sqrt{\pi} \, n! \, \delta_{m n} \hspace{1cm} n = 0, 1, 2, \ldots }
% \end{equation}
% El conjunto $\left\{ H_{n }(x) \right\}$ es completo; por tanto cualquier función $f(x)$ definida en el intervalo $(-\infty, \infty)$ puede expandirse en polinomios de Hermite:
% \[ f(x) = \sum_{n=0}^{\infty} C_{n} \, H_{n} (x) \]
% En forma general puede afirmarse que cualquier función definida en $(-\infty, \infty)$ puede expandirse en cualquier base ortogonal definida en $(-\infty, \infty)$.
% \par
% Expresar la función en una u otra es cambiar de base: la misma función puede expandirse, por ejemplo, en la base de Hermite $\left\{ H_{n }(x) \right\}$, o en la de Fourier $\left\{ e^{i k x} \right\}$.
% \section{La familia de la ecuación de Hermite.}
% A partir de la ecuación de Hermite y utilizando la transformación
% \[ H_{n} (x) = e^{\alpha x^{2}} \, \psi_{n} (\mu), \hspace{2cm} \mu = x^{b} \]
% se obtiene la primera familia de Hermite
% \begin{align*}
% &{} b^{2} \, \dv[2]{\psi_{n} (\mu)}{\mu} \, \mu^{2 (b-1)/b} + \dv{\psi_{n}(\mu)}{\mu} \left[ b \, (b-1) \mu^{\frac{b-2}{2}}  + \right. \\
% &+\left. 2 \, b \, (2 \, a - 1) \, \mu \right] + \left[ 4 \, a \, (a - 1) \, \mu^{2/b} + 2 \, n + 2 \, a \right] \, \psi_{n} (\mu) = 0
% \end{align*}
% cuya solución es
% \[ \psi_{n} (\mu) = e^{-\alpha x^{2}} \, H_{n} (x) = \exp \left( - \dfrac{a x^{2}}{b} \right) \, H_{n} (\mu^{1/b}) \]
% \begin{enumerate}[label=\roman*.)]
% \item Con $a = 0, b = 1$, se recupera la ecuación de Hermite.
% \item Si $b = 1, a = 1/2$, se obtiene la ecuación de \textbf{Weber-Hermite}:
% \begin{equation}
% \dv[2]{\psi_{n} (x)}{x} + [ 1 + 2 \, n - x^{2} ] \, \psi_{n} (x) = 0
% \label{eq:ecuacion_08_65}
% \end{equation}
% \end{enumerate}
% Con $\mu = x$ y $\psi_{n} (x) = e^{-x^{2}/2} / H_{n} (x)$. Esta última describe el oscilador armónicos unidimensional en mecánica cuántica. Nótese que la base $\left\{ \psi_{n} (x) \right\}$ es ortonormal.
% \section{El oscilador armónico cuántico.}
% Como una aplicación importante de los polinomios de Hermite consideremos la cuantización de la energía del oscilador armónico.
% \par
% De acuerdo con la mecánica cuántica, un oscilador armónico unidimensional, cuya energía potencial es $V = \frac{1}{2} k \, x^{2}$ puede describirse mediante la ecuación de Schrödinger:
% \[ - \dfrac{\hbar^{2}}{2 \, m} \, \dv[2]{\psi (x)}{x} + V (x) \, \psi (x) =  E \, \psi (x) \]
% donde $\psi (x)$ representa la función de onda, $\hbar$ es la constante de Planck dividida entre $2 \, \pi$, $m$ es la masa del oscilador y $E$ su energía total. 
% \par
% Cambiando a la nueva variable adimensional: $y = x/\alpha$, donde $\alpha$ tendrá la misma dimensión que $x$, y utilizando $\omega^{2} = k/m$, siendo $\omega$ la frecuencia angular del oscilador y $k$ la constante del resorte, podremos escribir:
% \[ \dv[2]{\psi}{y} - \left( \dfrac{\omega \, m \, \alpha^{2}}{\hbar}  \right)^{2} \, y^{2} \, \psi  + \left( \dfrac{2 \, m \, E \, \alpha^{2}}{\hbar^{2}} \right) \, \psi = 0 \] 
% La adimensionalidad de $y$ y la homogeneidad en $\psi$ de la ecuación permiten escoger un valor para $\alpha$, tal que el primer paréntesis tenga el valor de $1$, esto es:
% \[ \dfrac{\omega \, m \, \alpha}{\hbar} = 1 \hspace{1.5cm} \mbox{de donde } \alpha^{2} = \dfrac{\hbar}{m \, \omega} \]
% El segundo paréntesis (adimensional), lo llamaremos $\lambda$:
% \[ \lambda = \dfrac{2 \, m \, E \, \alpha^{2}}{\hbar} = \dfrac{2 \, E}{\hbar \, \omega} \]
% Por lo que la ecuación del oscilador toma la forma:
% \begin{equation}
% \dv[2]{\psi}{y} + (\lambda - y^{2}) \, \psi = 0
% \label{eq:ecuacion_08_66}
% \end{equation}
% Esta expresión, conocida como \emph{ecuación de Weber-Hermite}, corresponde, a la primera familia de Hermite con $b = 1$ y $a = 1/2$, tal que su solución es la función de Weber-Hermite de orden $n$ entero:
% \begin{equation}
% \psi_{n} (y) = e^{y^{2}/2} \, H_{n} (y)
% \label{eq:ecuacion_08_67}
% \end{equation}
% y por tanto $1 + 2 \, n = \lambda$
% \par
% Dado que $\lambda = 2 \, E / \hbar \, \omega$, se tiene que
% \begin{align}
% \dfrac{2 \, E}{\hbar \, \omega} &= 1 + 2 \, n \nonumber \\
% E &= \left(n + \dfrac{1}{2} \right) \, \hbar \, \omega. \hspace{1.5cm} n \geq 0
% \label{eq:ecuacion_08_68}
% \end{align}
% Como $n$ toma valores enteros, se sigue que la energía del oscilador está cuantizada y que hay una energía mínima o de punto cero: $E_{min} = \hbar \, \omega /2$.
% \par
% La secuencia de niveles de energía (\ref{eq:ecuacion_08_68}) tiene el espaciamiento $\Delta E = \hbar \, \omega$ postulado por Planck en 1900. Lo notable es que hay un mínimo en la energía que no aparece en la teoría de Planck y que es exigido por el principio de incertidumbre: \emph{un oscilador armónico no puede estar en reposo}. Si lo estuviera, sería en $x = 0$ que es el punto de equilibrio; en consecuencia podríamos conocer simultáneamente su posición y velocidad, lo que no es compatible con el principio de incertidumbre de Heisenberg.
% \par
% La función de onda normalizada del oscilador será entonces, de acuerdo con la ec. (\ref{eq:ecuacion_08_67}):
% \begin{align}
% \psi_{n} (y) = \dfrac{1}{(2^{2} \, \sqrt{\pi} \, n!)^{1/2}} \, e^{-y^{2}/2} \, H_{n} (y) \hspace{1.5cm} y = \dfrac{x}{\alpha} = x \, \sqrt{\dfrac{m \, \omega}{\hbar}}
% \label{eq:ecuacion_08_69}
% \end{align}
% Explícitamente, los primeros niveles de energía tienen la forma:
% \begin{align*}
% \psi_{0} (y) &= e^{-y^{2}/2} \hspace{2.6cm} E = \hbar \, \omega /2 \\
% \psi_{1} (y) &= 2 \, x \, e^{-y^{2}/2} \hspace{2cm} E = 3 \, \hbar \, \omega /2 \\
% \psi_{2} (y) &= (4 \, y^{2} - 2) \, e^{-y^{2}/2} \hspace{0.7cm} E = 5 \, \hbar \, \omega /2 \\
% \end{align*}
% \section{Operadores escalera.}
% Una de las identidades que satisface la función de Hermite es:
% \begin{equation}
% H_{n-1} (x) = \dfrac{1}{2 \, n} \, \dv{x} H_{n} (x)
% \label{eq:ecuacion_08_70}
% \end{equation}
% Según esta expresión, dado un $H_{n} (x)$ todos los anteriores pueden ser deducidos de él. Otra de las identidades es
% \[ H_{n+1} (x) = 2 \, x \, H_{n} (x) - 2 \, n \, H_{n-1} (x) \]
% Si de estas dos ecuaciones se elimina $H_{n-1} (x)$, se obtiene
% \begin{equation}
% H_{n+1} (x) = \left( 2 \, x - \dv{x} \right) \, H_{n} (x)
% \label{eq:ecuacion_08_71}
% \end{equation}
% expresión que nos dice: dado un $H_{n} (x)$ todos los que le siguen pueden deducirse de él. Basta entonces con un $Hn(x)$ para generar los demás. Los operadores 
% \[ \dfrac{1}{2 \, n} \dv{x} \hspace{2cm} 2 \, x - \dv{x} \]
% son los operadores escalera de los polinomios de Hermite.
% \par
% En lo que sigue introduciremos los operadores escalera asociados a las funciones de onda del oscilador armónico cuántico dadas por la ecuación (\ref{eq:ecuacion_08_69}). Reemplazando $H_{n} (y)$ de la ecuación (\ref{eq:ecuacion_08_69}) en las ecuaciones (\ref{eq:ecuacion_08_70}) y (\ref{eq:ecuacion_08_71}) obtenemos:
% \begin{align*}
% \sqrt{2 \, n} \, \psi_{n-1} (y) &= \left( y + \dv{x} \right) \, \psi_{n} (y) \\
% \sqrt{2 \, (n + 1)} \, \psi_{n+1} (y) &= \left( y - \dv{x} \right) \, \psi_{n} (y)
% \end{align*}
% que pueden ser escritos como:
% \begin{align}
% \begin{aligned}
% \hat{a} \, \psi_{n} &= \sqrt{n} \, \psi_{n-1} \\
% \hat{a}^{\dagger} \, \psi_{n} &= \sqrt{n+1} \, \psi_{n+1}
% \end{aligned}
% \label{eq:ecuacion_08_72}
% \end{align}
% donde los operadores $\hat{a}$ y $\hat{a}^{\dagger}$ están definidos por las ecuaciones
% \[ \hat{a} = \dfrac{1}{\sqrt{2}} \, \left( y + \dv{y} \right) \hspace{1.5cm} \hat{a}^{\dagger} = \dfrac{1}{\sqrt{2}} \, \left( y - \dv{y} \right) \]
% Estos operadores bajan y suben, respectivamente, estados cuánticos del oscilador. De la primera ecuación con $n = 0$ se obtiene la función de onda normalizada del estado base:
% \[ \psi_{0} (y) = \pi^{-1/4} \, e^{-y^{2}/2} \]
% A partir de ésta, pueden calcularse todas las funciones de onda del oscilador armónico mediante la expresión:
% \begin{equation}
% \psi_{n} = \dfrac{1}{\sqrt{n!}} \, \left( a^{\dagger} \right)^{n} \, \psi_{0}
% \label{eq:ecuacion_08_73}
% \end{equation}
\end{document}