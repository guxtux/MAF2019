\documentclass[12pt]{article}
\usepackage[utf8]{inputenc}
\usepackage[spanish,es-lcroman, es-tabla]{babel}
\usepackage[autostyle,spanish=mexican]{csquotes}
\usepackage{amsmath}
\usepackage{amssymb}
\usepackage{nccmath}
\numberwithin{equation}{section}
\usepackage{amsthm}
\usepackage{graphicx}
\usepackage{epstopdf}
\DeclareGraphicsExtensions{.pdf,.png,.jpg,.eps}
\usepackage{color}
\usepackage{float}
\usepackage{multicol}
\usepackage{enumerate}
\usepackage[shortlabels]{enumitem}
\usepackage{anyfontsize}
\usepackage{anysize}
\usepackage{array}
\usepackage{multirow}
\usepackage{enumitem}
\usepackage{cancel}
\usepackage{tikz}
\usepackage{circuitikz}
\usepackage{tikz-3dplot}
\usetikzlibrary{babel}
\usetikzlibrary{shapes}
\usepackage{bm}
\usepackage{mathtools}
\usepackage{esvect}
\usepackage{hyperref}
\usepackage{relsize}
\usepackage{siunitx}
\usepackage{physics}
%\usepackage{biblatex}
\usepackage{standalone}
\usepackage{mathrsfs}
\usepackage{bigints}
\usepackage{bookmark}
\spanishdecimal{.}

\setlist[enumerate]{itemsep=0mm}

\renewcommand{\baselinestretch}{1.5}

\let\oldbibliography\thebibliography

\renewcommand{\thebibliography}[1]{\oldbibliography{#1}

\setlength{\itemsep}{0pt}}
%\marginsize{1.5cm}{1.5cm}{2cm}{2cm}


\newtheorem{defi}{{\it Definición}}[section]
\newtheorem{teo}{{\it Teorema}}[section]
\newtheorem{ejemplo}{{\it Ejemplo}}[section]
\newtheorem{propiedad}{{\it Propiedad}}[section]
\newtheorem{lema}{{\it Lema}}[section]

%\author{M. en C. Gustavo Contreras Mayén. \texttt{curso.fisica.comp@gmail.com}}
\title{Funciones de Hermite \\ {\large Matemáticas Avanzadas de la Física}}
\date{ }
\begin{document}
\maketitle
\fontsize{14}{14}\selectfont
%Referencia:: Sepúlveda - Lecciones de Física Matemática Cap. 8.5 Polinomios de Hermite.
\section{Funciones de Hermite.}
La ecuación de Hermite, tiene una aplicación en física, quizá la más importante que es la del oscilador armónico en mecánica cuántica, tiene la forma:
\begin{equation}
\ddot{H}(x) - 2 \, x \,  \dot{H}(x) + 2\, n\ , H(x) = 0
\label{eq:ecuacion_08_58}
\end{equation}
La forma autoadjunta
\[ \dv{x} \left[ q(x) \, \dv{H(x)}{x} \right] + r(x) \, H(x) + \lambda \, p(x) \, H(x) = 0 \]
es entonces
\begin{equation}
\dv{x} \left[ \exp(-x^{2}) \dv{H(x)}{x} \right] + 2 \, n \, \exp(-x^{2}) \, H(x) = 0
\label{eq:ecuacion_08_59}
\end{equation}
La solución a esta ecuación forma una base ortogonal con factor de peso $p(x) = \exp(-x^{2})$, en el intervalo $(-\infty, \infty)$, ya que en dicho intervalo
\[ [q \, W]_{-\infty}^{\infty} = [ \exp(-x^{2}) \, W ]_{-\infty}^{\infty} = 0 \]
Por lo que la ortogonalidad queda expresada por
\[ \int_{-\infty}^{\infty} \exp(-x^{2}) \, H_{n}(x) \, H_{m}(y) \, \dd x = 0, \hspace{1cm} \mbox{ si } n \neq m \]
Con lo que vemos que la ortogonalidad de las funciones propias está asociada a la elección particular del dominio de la variable independiente.
\par
Aplicando el método de Frobenius:
\begin{align*}
H(x) &= \sum_{\alpha = 0}^{\infty} a_{\alpha} \, x^{\alpha+k} \\
\dot{H}(x) &= \sum_{\alpha = 0}^{\infty} a_{\alpha} \, (\alpha + k) \, x^{\alpha+k-1} \\
\ddot{H}(x) &= \sum_{\alpha = 0}^{\infty} a_{\alpha} \, (\alpha + k) \, (\alpha + k - 1) \, x^{\alpha+ k-2}
\end{align*}
re-emplazando en la ecuación (\ref{eq:ecuacion_08_58}) y factorizando los términos
\[ \sum_{\alpha = 0}^{\infty} a_{\alpha} \, (\alpha + k) \, (\alpha + k - 1) \, x^{\alpha+k-2} - \sum_{\alpha=0}^{\infty} a_{\alpha} \, [ 2 \, (\alpha + k) - 2 \, n] \, x^{\alpha+k} = 0 \]
que puede escribirse como
\[ \sum_{\alpha=-2}^{\infty} a_{\alpha + 2} \, ( \alpha + k + 2) \, (\alpha + k + 1)\, x^{\alpha+k} - \sum_{\alpha=0}^{\infty} a_{\alpha} \, [2 \, (\alpha + k) - 2 \, n] \, x^{\alpha+k} = 0 \]
escribiendo los dos primeros términos de la serie
\begin{align*}
 a_{0}(k) \, (k - 1)\, x^{k-2} &+ a_{1} \, (k + 1)(k)^{\alpha-1} + \\
&+ \sum_{\alpha=0}^{\infty} \left[ a_{\alpha + 2} \, (\alpha + k +2) \, (\alpha + k + 1) - a_{0} \, [2 (\alpha + k ) - 2 \, n] \right] \, x^{\alpha+k} = 0 
\end{align*}
donde reconocemos las ecuaciones de índices:
\begin{align*}
a_{0}\, k \, (k-1) &= 0 \\
a_{1} \, k\, (k+1) &= 0 \\
a_{\alpha+2}\, (\alpha + k + 2) \, (\alpha + k + 1) - a_{\alpha} \, [2 \, (\alpha + k) - 2 \, n] &= 0, \hspace{1cm} \alpha=0, 1, 2, \ldots
\end{align*}
Revisamos que:
\begin{enumerate}
\item De la primera expresión: si $a_{0} \neq 0$, entonces $k=0$ ó $k=1$.
\item De la segunda expresión: con $k=0$ se sigue que $a_{1} \neq 0$ y de $k=1$, se concluye que $a_{1} = 0$.
\item De la tercera ecuación de índices con $k=0$, se obtiene
\[ a_{\alpha+2} = \dfrac{2a_{\alpha}(\alpha - n)}{(\alpha + 2)(\alpha + 1)} \hspace{1cm} \alpha = 0, 1, 2, \ldots \]
\end{enumerate}
Explícitamente los primeros coeficientes son
\begin{align*}
a_{2} &= \dfrac{(-) \, 2 \, n}{2!} \, a_{0} \\
a_{3} &= \dfrac{(-) \, 2 \, (n-1)}{3!} \, a_{1} \\
a_{4} &= \dfrac{(-) \, 2 \, a_{2} \, (2 - n)}{4 \times 3} = \dfrac{2^{2} \, (-)^{2} \, (n) \, (n - 2)}{4!} a_{0} \\
\vdots
\end{align*}
Entonces tenemos
\begin{align*}
H(x) &= a_{0} \, \left[ 1 + \dfrac{(-) 2 \, n}{2!}\, x^{2} + \dfrac{(-) \, 2^{2} \, n \, (n - 2)}{4!}\, x^{4} + \dfrac{(-)^{3} \, 2^{3} \, n \, (n-2)(n-4) }{6!}\, x^{6} + \ldots \right] + \\
&+ a_{1} \, \left[ x + \dfrac{(-) \, 2 \, (n-1)}{3!} \, x^{3} + \dfrac{(-)^{2} \, 2^{2} \, (n-1)(n-3)}{5!}\, x^{5} + \right.\\
&+ \left. \dfrac{(-)^{3} \, 2^{3} \, (n-1)(n-3)(n-5)}{7!} \, x^{7} + \ldots \right]
\end{align*}
Ambas series son divergentes en $x \to \pm \infty$. Si se incluyen estos dos extremos y se quiere lograr convergencia es necesario cortar las series y convertirlas en polinomios.
\par
La serie en $a_{0}$ requiere que $n$ = par positivo y la serie en $a_{1}$ requiere $n$ = impar positivo. Puesto que $n$ no puede ser simultáneamente par e impar en las series para $a_{0}$ y $a_{1}$, si $n$ = par, la segunda serie es divergente y si $n$ = impar la primera diverge.
\par
Las series convergentes serán las de nuestro interés, y conformarán los \emph{Polinomios de Hermite}.
\\
Puede demostrarse que con $n \neq$ entero, $H(x) \propto x^{2} \, \exp(x^{2})$ para $x \to \pm \infty$, lo que muestra la no convergencia para $n$ no entero.
\par
Consideremos $n$ = par. La serie para $a_{0}$, con $n=2\, m, \; m = 0, 1, 2, \ldots$ será
\begin{align*}
H(x) &= a_{0} \, \left[ 1 + \dfrac{(-) \, 2^{2} \, m}{2!} \, x^{2} + \dfrac{(-)^{2} \, 2^{4} \, m \, (m - 1)}{4!} \, x^{4} + \right. \\
&+ \left. \dfrac{(-)^{3} \, 2^{6} \, m \, (m-1)(m-2)}{6!} \, x^{6} + \ldots \right] \\[1em]
H(x) &= a_{0}\, \left[ 1 + \dfrac{(-)\, 2^{2}\, m}{2!} \, x^{2} + \ldots + \dfrac{(-)^{3} \, 2^{6}\, m!}{(m-3)! \, 6!} \, x^{6} + \ldots + \dfrac{(-)^{p} \, (2 \, x)^{2\, p}\, m!}{(m-p)! \, (2\, p)!} + \ldots \right] \\
&= m!\, a_{0} \, \sum_{p=0}^{\infty} \dfrac{(-)^{p}\, (2\, x)^{2\, p}}{(m-p)! \, (2\, p)!}
\end{align*}
Se definen los polinomios de Hermite de orden par, en la forma
\begin{equation}
H_{2\, m} (x) = (-)^{m} \, (2\, m)! \sum_{p=0}^{\infty} \dfrac{(-)^{p} \, (2\, x)^{2\, p}}{(m-p)!\, (2p)!}
\label{eq:ecuacion_08_60}
\end{equation}
Revisemos que efectivaente la ecuación (\ref{eq:ecuacion_08_60}) es un polinomio, ya que para $p > m : 1 / (m-p)! \to 0$. Esto significa que la suma se extiende entre $0$ y $m$.
\par
De manera análoga, para $n=$ impar, con $n = 2 \, m + 1, \hspace{0.5cm} m=0, 1, 2, \ldots$
\begin{align*}
H(x) & = a_{1} \, \left[ x + \dfrac{(-) \, 2^{2} \, m}{3!} \, x^{2} + \dfrac{(-)^{2} \, 2^{4} \, m \, (m - 1)}{5!} \, x^{5} + \right. \\
& + \left. \dfrac{(-)^{3} \, 2^{6} \, m \, (m-1)(m-2)}{7!} \, x^{7} + \ldots \right] \\[1em]
H(x) & = a_{1}\, \left[ x + \dfrac{(-)\, 2^{2}\, m}{3!} \, x^{3} + \ldots + \dfrac{(-)^{3} \, 2^{6}\, m!}{(m-3)! \, 7!} \, x^{7} + \ldots + \right. \\
&+  \left. \dfrac{(-)^{p} \, 2^{2 \, p} \, m!}{(m-p)! \, (2\, p + 1)!} \, x^{2p+1} + \ldots \right] \\
& = a_{1} \, m! \sum_{p=0}^{\infty} \dfrac{(-)^{p}\, (2\, x)^{2\, p+1}}{(m-p)! \, (2\, p + 1)!}  
\end{align*}
Definimos los polinomios de Hermite de orden impar como:
\begin{equation}
H_{2m+1} (x) = (-)^{m} \, (2 \, m + 1)! \, \sum_{p=0}^{\infty} \dfrac{(-)^{p} \, (2 \, x)^{2p+1}}{(m-p)! \, (2 \, p + 1)!}
\label{eq:ecuacion_08_61}
\end{equation}
donde la suma da términos diferentes de cero sólo entre $0$ y $m$. En forma compacta se tiene la expresión:
\begin{equation}
\boxed{H_{n}(x) = n! \, \sum_{p=0}^{N} \dfrac{(-)^{p}}{(n - 2 \, p)! \, p!} \, (2 \, x)^{n-2p}}
\label{eq:ecuacion_08_62}
\end{equation}
con $N=n/2$ si $n$ es par, o $N=n-1/2$ si $n$ es impar.
\par
La primera ecuación de índices provee para el exponente $k$ en la serie de Frobenius un segundo valor: $k = 1$. Es directo comprobar que nada nuevo se añade al desarrollo anterior.
\par
Los primeros polinomios de Hermite son:
\begin{align*}
H_{0}(x) &= 1 \\
H_{1}(x) &= 2 \, x \\
H_{2}(x) &= 4 \, x^{2} - 2 \\
H_{3}(x) &= 8 \, x^{3} - 12 \, x \\
H_{4}(x) &= 16 \, x^{4} - 48 \, x^{2} + 12 \\
H_{5}(x) &= 32 \, x^{5} - 160 \, x^{3} + 120 \, x
\end{align*}
\end{document}