\documentclass[12pt]{article}
\usepackage[utf8]{inputenc}
\usepackage[spanish,es-lcroman, es-tabla]{babel}
\usepackage[autostyle,spanish=mexican]{csquotes}
\usepackage{amsmath}
\usepackage{amssymb}
\usepackage{nccmath}
\numberwithin{equation}{section}
\usepackage{amsthm}
\usepackage{graphicx}
\usepackage{epstopdf}
\DeclareGraphicsExtensions{.pdf,.png,.jpg,.eps}
\usepackage{color}
\usepackage{float}
\usepackage{multicol}
\usepackage{enumerate}
\usepackage[shortlabels]{enumitem}
\usepackage{anyfontsize}
\usepackage{anysize}
\usepackage{array}
\usepackage{multirow}
\usepackage{enumitem}
\usepackage{cancel}
\usepackage{tikz}
\usepackage{circuitikz}
\usepackage{tikz-3dplot}
\usetikzlibrary{babel}
\usetikzlibrary{shapes}
\usepackage{bm}
\usepackage{mathtools}
\usepackage{esvect}
\usepackage{hyperref}
\usepackage{relsize}
\usepackage{siunitx}
\usepackage{physics}
%\usepackage{biblatex}
\usepackage{standalone}
\usepackage{mathrsfs}
\usepackage{bigints}
\usepackage{bookmark}
\spanishdecimal{.}

\setlist[enumerate]{itemsep=0mm}

\renewcommand{\baselinestretch}{1.5}

\let\oldbibliography\thebibliography

\renewcommand{\thebibliography}[1]{\oldbibliography{#1}

\setlength{\itemsep}{0pt}}
%\marginsize{1.5cm}{1.5cm}{2cm}{2cm}


\newtheorem{defi}{{\it Definición}}[section]
\newtheorem{teo}{{\it Teorema}}[section]
\newtheorem{ejemplo}{{\it Ejemplo}}[section]
\newtheorem{propiedad}{{\it Propiedad}}[section]
\newtheorem{lema}{{\it Lema}}[section]

\usepackage{mathrsfs}
\usepackage{bigints}
\spanishdecimal{.}
%\usepackage{enumerate}
%\author{M. en C. Gustavo Contreras Mayén. \texttt{curso.fisica.comp@gmail.com}}
\title{Funciones de Hermite \\ {\large Matemáticas Avanzadas de la Física}}
\date{ }
\begin{document}
%\renewcommand\theenumii{\arabic{theenumii.enumii}}
\renewcommand\labelenumii{\theenumi.{\arabic{enumii}}}
\maketitle
\fontsize{14}{14}\selectfont
\section{Funciones de Hermite.}
La ecuación de Hermite, tiene una aplicación en física, quizá la más importante que es la del oscilador armónico en mecánica cuántica, tiene la forma:
\begin{equation}
\ddot{H}(x) - 2 x \dot{H}(x) + 2n H(x) = 0
\label{eq:ecuacion_001}
\end{equation}
La forma autoadjunta
\[ \dfrac{d}{dx} \left( q(x) \dfrac{d H(x)}{dx} \right) + r(x) H(x) + \lambda p(x) H(x) = 0 \]
es entonces
\begin{equation}
\dfrac{d}{dx} \left( \exp(-x^{2}) \dfrac{d H(x)}{dx} \right) + 2n \exp(-x^{2}) H(x) = 0
\label{eq:ecuacion_002}
\end{equation}
La solución a esta ecuación forma una base ortogonal con factor de peso $p(x) = \exp(-x^{2})$, en el intervalo $(-\infty, \infty)$, ya que en dicho intervalo
\[ [q W]_{-\infty}^{\infty} = [ \exp(-x^{2}) W ]_{-\infty}^{\infty} = 0 \]
Por lo que la ortogonalidad queda expresada por
\[ \int_{-\infty}^{\infty} \exp(-x^{2}) H_{n}(x) H_{m}(y) dx = 0, \hspace{1cm} \mbox{ si } n \neq m \]
Con lo que vemos que la ortogonalidad de las autofunciones está asociada a la elección particular del dominio de la variable independiente.
\\
Aplicando el método de Frobenius:
\begin{eqnarray}
H(x) &=& \sum_{\alpha = 0}^{\infty} a_{\alpha} x^{\alpha + k} \nonumber \\
\dot{H}(x) &=& \sum_{\alpha = 0}^{\infty} a_{\alpha}(\alpha + k) x^{\alpha + k-1} \nonumber \\
\ddot{H}(x) &=& \sum_{\alpha = 0}^{\infty} a_{\alpha}(\alpha + k)(\alpha + k - 1) x^{\alpha + k-2} \nonumber
\end{eqnarray}
re-emplazando en la ecuación (\ref{eq:ecuacion_001}) y factorizando los términos
\[ \sum_{\alpha = 0}^{\infty} a_{\alpha}(\alpha + k)(\alpha + k - 1) x^{\alpha + k-2} - \sum_{\alpha=0}^{\infty} a_{\alpha} [ 2 (\alpha + k) - 2n ] x^{\alpha +k} = 0 \]
que puede escribirse como
\[ \sum_{\alpha=-2}^{\infty} a_{\alpha + 2}( \alpha + k + 2)(\alpha + k + 1) x^{\alpha + k} - \sum_{\alpha=0}^{\infty} a_{\alpha} [2 (\alpha + k) - 2n ] x^{\alpha + k} = 0 \]
escribiendo los dos primeros términos de la serie
\[ \begin{split}
 a_{0}(k)(k - 1) x^{k-2} &+ a_{1} (k + 1)(k)^{\alpha-1} \\
&+ \sum_{\alpha=0}^{\infty} \left[ a_{\alpha + 2} (\alpha + k +2)(\alpha + k + 1) - a_{0} [2 (\alpha + k ) - 2n] \right] x^{\alpha + k} = 0 
\end{split} \]
donde reconocemos las ecuaciones indiciales
\begin{eqnarray}
a_{0} k (k-1) &=& 0 \nonumber \\
a_{1} k (k+1) &=& 0 \nonumber \\
a_{\alpha+2} (\alpha + k + 2) (\alpha + k + 1) - a_{\alpha} [2 (\alpha + k) - 2n] &=& 0, \hspace{1cm} \alpha=0,1,2,\ldots \nonumber
\end{eqnarray}
Revisamos que:
\begin{enumerate}
\item De la primera expresión: si $a_{0} \neq 0$, entonces $k=0$ ó $k=1$.
\item De la segunda expresión: con $k=0$ se sigue que $a_{1} \neq 0$ y de $k=1$, se concluye que $a_{1} = 0$.
\item De la tercera ecuación indicial con $k=0$, se obtiene
\[ a_{\alpha+2} = \dfrac{2a_{\alpha}(\alpha - n)}{(\alpha + 2)(\alpha + 1)} \hspace{1cm} \alpha = 0, 1, 2, \ldots \]
\end{enumerate}
Explícitamente los primeros coeficientes son
\begin{eqnarray}
a_{2} &=& \dfrac{(-) 2n}{2!} a_{0} \nonumber \\
a_{3} &=& \dfrac{(-) 2(n-1)}{3!} a_{1} \nonumber \\
a_{4} &=& \dfrac{(-) 2 a_{2} (2 - n)}{4 \times 3} = \dfrac{2^{2} (-)^{2}(n)(n - 2)}{4!} a_{0} \nonumber \\
\vdots \nonumber
\end{eqnarray}
Entonces tenemos
\[ \begin{split}
H(x) &= a_{0} \left[ 1 + \dfrac{(-) 2n}{2!} x^{2} + \dfrac{(-) 2^{2} n (n - 2)}{4!} x^{4} + \dfrac{(-)^{3}2^{3} n (n-2)(n-4) }{6!} x^{6} + \ldots \right] + \\
&+ a_{1} \left[ x + \dfrac{(-) 2 (n-1)}{3!} x^{3} + \dfrac{(-)^{2}2^{2}(n-1)(n-3)}{5!} x^{5} + \right.\\
&+ \left. \dfrac{(-)^{3}2^{3}(n-1)(n-3)(n-5)}{7!} x^{7} + \ldots \right]
\end{split} \]
Ambas series son divergentes en $x \to \pm \infty$. Si se incluyen estos dos extremos y se quiere lograr convergencia es necesario cortar las series y convertirlas en polinomios.
\\
La serie en $a_{0}$ requiere que $n$ = par positivo y la serie en $a_{1}$ requiere $n$ = impar positivo. Puesto que $n$ no puede ser simultáneamente par e impar en las series para $a_{0}$ y $a_{1}$, si $n$ = par, la segunda serie es divergente y si $n$ = impar la primera diverge.
\\
Las series convergentes serán las de nuestro interés, y conformarán los \emph{Polinomios de Hermite}.
\\
Puede demostrarse que con $n$ no entero, $H(x) \propto x^{2} \exp(x^{2})$ para $x \to \pm \infty$, lo que muestra la no convergencia para $n$ no entero.
\\
Consideremos $n$ = par. La serie para $a_{0}$, con $n=2m, \; m=0,1,2\ldots$ será
\[ \begin{split}
H(x) &= a_{0} \left[ 1 + \dfrac{(-) 2^{2} m}{2!} x^{2} + \dfrac{(-) 2^{4} m (m - 1)}{4!} x^{4} + \right. \\
&+ \left. \dfrac{(-)^{3}2^{6} m (m-1)(m-2) }{6!} x^{6} + \ldots \right] \\
H(x) &= a_{0} \left[ 1 + \dfrac{(-) 2^{2} m}{2!} x^{2} + \ldots + \dfrac{(-)^{3} 2^{6} m!}{(m-3)! 6!} x^{6} + \ldots + \dfrac{(-)^{p}(2x)^{2p} m!}{(m-p)!(2p)!} + \ldots \right]
\end{split} \]
Se definen los polinomios de Hermite de orden par, en la forma
\begin{equation}
H_{2m} (x) = (-)^{m} (2m)! \sum_{p=0}^{\infty} \dfrac{(-)^{p}(2x)^{2p}}{(m-p)!(2p)!}
\label{eq:ecuacion_003}
\end{equation}
Revisemos que efectivaente la ecuación (\ref{eq:ecuacion_003}) es un polinomio, ya que para $p > m : 1 / (m-p)! \to 0$. Esto significa que la suma se extiende entre $0$ y $m$.
\\
De manera análoga, para $n$ = impar, con $n=2m+1, \hspace{0.5cm} m=0,1,2,\ldots$

\end{document}