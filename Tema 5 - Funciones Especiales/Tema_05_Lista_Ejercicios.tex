\documentclass[12pt]{article}
\usepackage[left=0.25cm,top=1cm,right=0.25cm,bottom=1cm]{geometry}
\textwidth = 20cm
\hoffset = -1cm
\usepackage[utf8]{inputenc}
\usepackage[spanish,es-tabla]{babel}
\usepackage[autostyle,spanish=mexican]{csquotes}
\usepackage[tbtags]{amsmath}
\usepackage{nccmath}
\usepackage{amsthm}
\usepackage{amssymb}
\usepackage{graphicx}
\usepackage{standalone}
\usepackage[outdir=./]{epstopdf}
\usepackage{siunitx}
\usepackage{physics}
\usepackage{color}
\usepackage{float}
\usepackage{multicol}
%\usepackage{milista}
\usepackage{enumitem}
\usepackage{anyfontsize}
\usepackage{anysize}
\usepackage{enumitem}
\usepackage{capt-of}
\usepackage{bm}
\usepackage{relsize}
\usepackage{placeins}
\usepackage{empheq}
\usepackage{cancel}
\usepackage{wrapfig}
\spanishdecimal{.}
\renewcommand{\baselinestretch}{1.5} 
\renewcommand\labelenumii{\theenumi.{\arabic{enumii}}}
\newcommand{\ptilde}[1]{\ensuremath{{#1}^{\prime}}}
\newcommand{\stilde}[1]{\ensuremath{{#1}^{\prime \prime}}}
\newcommand{\ttilde}[1]{\ensuremath{{#1}^{\prime \prime \prime}}}
\newcommand{\ntilde}[2]{\ensuremath{{#1}^{(#2)}}}


\title{Ejercicios del Tema 5 \\[0.3em]  \large{Matemáticas Avanzadas de la Física}\vspace{-3ex}}
\author{M. en C. Gustavo Contreras Mayén}
\date{ }
\begin{document}
\vspace{-4cm}
\maketitle
\fontsize{14}{14}\selectfont

\textbf{Indicaciones: } Deberás de resolver cada ejercicio de la manera más completa, ordenada y clara posible, anotando cada paso así como las operaciones involucradas. El puntaje de cada ejercicio es de \textbf{1 punto}, con excepción en donde se indica.

\begin{enumerate}
%Ref. Andrews (1998) Special Functions Chap. 5.3 Problem 6
\item Polinomios asociados de Laguerre.
\par
\noindent
Demuestra que:
\begin{align*}
L_{n}^{k} (0) = \dfrac{(n + k)!}{n! \, k!}
\end{align*}
%Ref. Arfken (2006) 13.2.9
\item Polinomios asociados de Laguerre.
\par
\noindent
La función de onda normalizada para el átomo de hidrógeno es:
\begin{align*}
R_{n l} (r) = \left[ \alpha^{3} \dfrac{(n -l -1)!}{2 \, n \, (n + l)!} \right]^{1/2} \, \exp \left( \dfrac{-\alpha \, r}{2} \right) \, (\alpha \, r)^{l} \, L_{n - l +1}^{2 l +1} \, (\alpha \, r) 
\end{align*}
en donde: 
\begin{align*}
\alpha = \dfrac{2 \, Z}{n \, a_{0}} = \dfrac{2 \, Z \, m \, e^{2}}{4 \, \pi \, \epsilon_{0} \, \hbar^{2}}
\end{align*}
La cantidad $\expval{r}$ es el desplazamiento promedio del electrón con respecto al núcleo.
\par
Demuestra que al evaluar la siguiente integral se obtiene el valor indicado:
\begin{align*}
\expval{r} = \scaleint{6ex}_{\bs 0}^{\infty} r \, R_{n l} (\alpha \, r) \, R_{n l} (\alpha \, r) \, r^{2} \dd{r} = \dfrac{a_{0}}{2} [3 \, n^{2} - l (l + 1)]
\end{align*}
%Ref. Arfken(2006) 13.1.7 (b)
\item Polinomios de Hermite.
\par
\noindent
Demuestra que:
\begin{align*}
\scaleint{6ex}_{\bs - \infty}^{\infty} x \, H_{n}(x) \, \exp \bigg[ - \dfrac{x^{2}}{2} \bigg] \dd{x} = \begin{cases}
0 & n \mbox{ par} \\
2 \, \pi \dfrac{(n + 1)!}{\big[ (n + 1)/2 \big]!} & n \mbox{ impar}
\end{cases}
\end{align*}

\item Funciones de Bessel.
%Ref. Arfken(2006) 11.2.7
\par
\noindent
Un cilindro circular recto tiene un potencial electrostático $\psi (\rho, \varphi)$ en los extremos, mientras que el potencial en la superficie curva del cilindro es cero. Calcula el potencial en los puntos interiores del cilindro. Nota: Elige debidamente el sistema coordenado para que ajustes la dependencia de $z$ y explotes al máximo la simetría del potencial.
%Ref. Andrews (1998) - Chap. 6 Problem 9 (a) y (b)
\item Funciones de Bessel.
\par
\noindent
Usando al función generatriz demuestra que:
\begin{align*}
\exp(i \, x \, \sin \theta) = \nsum_{n=-\infty}^{\infty} J_{n}(x) \, e^{i n \theta}
\end{align*}
%Ref. Mason (2003) 1.3.2
\item Polinomios de Chebyshev.
\par
\noindent
Los polinomios de Chebyshev están definidos en el intervalo $[- 1, 1]$, siendo posible definirlos en cualquier rango finito $[a, b]$ para la variable $x$, haciendo que éste rango corresponda al rango $[-1, 1]$ con una nueva variable $s$, se ocupa la siguiente transformación lineal:
\begin{align*}
s = \dfrac{2 \, x - (a + b)}{(b - a)}
\end{align*}

Por lo que los polinomios de Chebyshev de primer tipo ajustados al intervalo $[a, b]$ son $T_{n}(s)$, de manera similar se hace el ajuste para los polinomios de segundo tipo  $U_{n} (s)$. 
\noindent
\par
Desarrolla la expresión para los polinomios de Chebyshev de segundo tipo $U_{n} (s)$ de grado $n = 0, 1, 2, 3, 4$ en el rango $[-5, 5]$ para $x$.
\item Polinomios de Chebyshev.
\par
\noindent
%Ref. Arfken (2006) 13.3.4
Demuestra que:
\begin{align*}
W_{n} (x) = (1 - x^{2})^{\frac{1}{2}} \, T_{n+1} (x)
\end{align*}
es una solución de:
\begin{align*}
(1 - x^{2}) \, \sderivada{W}_{n} (x) - 3 \, x \, \pderivada{W}_{n} (x) + n (n + 2) \, W_{n} (x) = 0
\end{align*}
%Ref. Riley (2006) 18.11 (b) y (c)
\item Función hipergeométrica.
\par
\noindent
Identifica la serie para la siguiente función hipergeométrica, escribiéndola en términos de una función conocida.
\begin{align*}
F \big( 1, 1; 2; - x \big) = \quad ?
\end{align*}
%Ref. Arfken (2006) 13.4.3 (b)-(c)
\item Función Hipergeométrica.
\par
\noindent
Demuestra que el siguiente polinomio transformado a funciones hipergeométricas con argumento $x^{2}$, es:
\begin{align*}
x^{-1} \, T_{2n+1} (x) = (-1)^{n} \, (2 \, n + 1 ) \, \, {}_{2}F_{1} \bigg(-n, n+1; \dfrac{3}{2}; x^{2} \bigg)
\end{align*}
\item Desarrolla una monografía de las funciones especiales que se indican a continuación, utilizando el formato que se presenta.
\begin{enumerate}
\item Polinomios asociados de Legendre.
\item Armónicos esféricos.
\item Funciones ordinarias de Laguerre.
\item Funciones asociadas de Laguerre.
\item Funciones de Bessel de primera clase.
\item Funciones de Bessel de segunda clase.
\item Funciones de Hermite.
\item Funciones de Chebychev de primera clase.
\item Funciones de Chebychev de segunda clase.
\item Funciones hipergeométrica ordinaria.
\item Funciones hipergeométrica confluente.
\item Funciones de Gegenbauer.
\end{enumerate}

La idea de este ejercicio es tener de manera concentrada, las propiedades de varias funciones especiales.

\begin{table}[H]
    \centering
\begin{tabular}{| p{5cm} | p{12cm} |} \hline
\multicolumn{2}{|c|}{\textbf{Polinomios ordinarios de Legendre}} \\ \hline
Problema(s) de la Física & \makecell[l]{Ecuación radial del átomo de hidrógeno \\ Ecuación de Laplace \\ Desarrollo multipolar eléctrico} \\ \hline
Geometría & Sistema coordenado esférico \\ \hline
Ecuación diferencial & \(\displaystyle
(1 - x^{2}) \stilde{y} - 2 \, x \, \ptilde{y} + \ell (\ell + 1) \, y = 0
\) \\ \hline
Soluciones & \makecell[l]{\( y_{1}(x) = 1 - \ell (\ell + 1) \dfrac{x^{2}}{2!} + (\ell - 2)\; \ell \; (\ell + 1)\;(\ell + 3) \dfrac{x^{4}}{4!} - \ldots \) \\ \( y_{2}(x) = x - (\ell - 1)(\ell + 2) \dfrac{x^{3}}{3!} + (\ell - 3) (\ell - 1)(\ell + 2)(\ell + 4) \dfrac{x^{5}}{5!} - \ldots \)}\\ \hline
Solución general & \( \displaystyle y(x) = c_{1} \, P_{\ell}(x) + c_{2} \, Q_{\ell} (x)  \)\\ \hline
Función generatriz & \(\displaystyle G(x ,h) = (1 - 2 \, x \, h + h^{2})^{-1/2} =  \sum_{n=0}^{\infty} P_{n}(x) \, h^{n} \) \\ \hline
Ortogonalidad & \makecell[l]{ \( \displaystyle \int_{-1}^{1} P_{\ell}(x) P_{k}(x) \dd{x} = 0 \hspace{1cm} \mbox{ si $\ell \neq k$} \) \\
\( \displaystyle \int_{-1}^{1} P_{\ell}(x) P_{k}(x) \dd{x} = \dfrac{2}{2 \, \ell + 1} \hspace{1cm} \mbox{ si $\ell = k$} \) } \\ \hline
Relaciones de recurrencia & \makecell[l]{ \( \displaystyle \ptilde{P}_{n+1} + \ptilde{P}_{n-1} =  P_{n} + 2 \; x \; \ptilde{P}_{n} \) \\ \( \displaystyle \ptilde{P}_{n+1} = (n+1) \; P_{n} + x \; \ptilde{P}_{n}\) \\
\( \displaystyle \ptilde{P}_{n-1} = -n \; P_{n} + x \; \ptilde{P}_{n}\) \\
\( \displaystyle (1 - x^{2}) \, \ptilde{P}_{n+1} = n \; (P_{n-1} - x \; P_{n})\) \\
\( \displaystyle (2 \, n + 1) \, P_{n} = \ptilde{P}_{n+1} - \ptilde{P}_{n-1} \)} \\ \hline
Paridad & \( \displaystyle P_{\ell} (-u) = (-1)^{\ell} \, P_{\ell} (u) \) \\ \hline
Expresión integral & \\ \hline
\end{tabular}
\end{table}
\end{enumerate}
\end{document}