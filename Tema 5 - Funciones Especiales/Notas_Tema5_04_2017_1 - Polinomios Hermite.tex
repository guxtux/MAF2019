\documentclass[12pt]{article}
\usepackage[utf8]{inputenc}
\usepackage[spanish,es-lcroman, es-tabla]{babel}
\usepackage[autostyle,spanish=mexican]{csquotes}
\usepackage{amsmath}
\usepackage{amssymb}
\usepackage{nccmath}
\numberwithin{equation}{section}
\usepackage{amsthm}
\usepackage{graphicx}
\usepackage{epstopdf}
\DeclareGraphicsExtensions{.pdf,.png,.jpg,.eps}
\usepackage{color}
\usepackage{float}
\usepackage{multicol}
\usepackage{enumerate}
\usepackage[shortlabels]{enumitem}
\usepackage{anyfontsize}
\usepackage{anysize}
\usepackage{array}
\usepackage{multirow}
\usepackage{enumitem}
\usepackage{cancel}
\usepackage{tikz}
\usepackage{circuitikz}
\usepackage{tikz-3dplot}
\usetikzlibrary{babel}
\usetikzlibrary{shapes}
\usepackage{bm}
\usepackage{mathtools}
\usepackage{esvect}
\usepackage{hyperref}
\usepackage{relsize}
\usepackage{siunitx}
\usepackage{physics}
%\usepackage{biblatex}
\usepackage{standalone}
\usepackage{mathrsfs}
\usepackage{bigints}
\usepackage{bookmark}
\spanishdecimal{.}

\setlist[enumerate]{itemsep=0mm}

\renewcommand{\baselinestretch}{1.5}

\let\oldbibliography\thebibliography

\renewcommand{\thebibliography}[1]{\oldbibliography{#1}

\setlength{\itemsep}{0pt}}
%\marginsize{1.5cm}{1.5cm}{2cm}{2cm}


\newtheorem{defi}{{\it Definición}}[section]
\newtheorem{teo}{{\it Teorema}}[section]
\newtheorem{ejemplo}{{\it Ejemplo}}[section]
\newtheorem{propiedad}{{\it Propiedad}}[section]
\newtheorem{lema}{{\it Lema}}[section]

\usepackage{mathrsfs}
\usepackage{bigints}
\usepackage{standalone}
\usetikzlibrary{decorations}
\usetikzlibrary{decorations.pathreplacing}
\spanishdecimal{.}
\newcommand{\saltosin}{\nonumber \\}
\newtheorem{teorema}{{\it Teorema}}[section]
%\usepackage{enumerate}
%\author{M. en C. Gustavo Contreras Mayén. \texttt{curso.fisica.comp@gmail.com}}
\title{Polinomios de Hermite \\ {\large Matemáticas Avanzadas de la Física}}
\date{ }
\begin{document}
%\renewcommand\theenumii{\arabic{theenumii.enumii}}
\renewcommand\labelenumii{\theenumi.{\arabic{enumii}}}
\maketitle
\fontsize{14}{14}\selectfont
\section{Polinomios de Hermite.}
Se ha visto que los polinomios de Hermite son de la forma
\begin{eqnarray*}
h_{N} (y) &=& a_{0} + a_{2} x^{2} + \ldots + a_{N} x^{N}, \hspace{1cm} \text{para } N \text{ par}, \saltosin
h_{N} (y) &=& a_{1} + a_{3} x^{3} + \ldots + a_{N} x^{N}, \hspace{1cm} \text{para } N \text{ impar}
\end{eqnarray*}
donde los coeficientes satisfacen las relaciones de recurrencia
\begingroup\makeatletter\def\f@size{12}\check@mathfonts
\begin{eqnarray}
a_{2k} &=& (-2)^{k} \; \dfrac{N (N - 2)(N - 4) \ldots (N - 2k + 4)(N - 2k + 2)}{(2k)!} \; a_{0} \hspace{1cm} N \text{ par} \label{eq:ecuacion_06_28} \\
a_{2k+1} &=& (-2)^{k} \; \dfrac{(N - 1)(N - 3) \ldots (N - 2k + 3)(N - 2k + 1)}{(2k + 1)!} \; a_{1} \hspace{1cm} N \text{ impar} \label{eq:ecuacion_06_29}
\end{eqnarray}
\endgroup
Por lo que los primeros polinomios tienen la forma
\begin{eqnarray*}
h_{0} &=& a_{0} \saltosin
h_{1} &=& a_{1} x \saltosin
h_{2} &=& a_{0} - 2 a_{0} x^{2} \saltosin
h_{3} &=& a_{1} - 2 a_{0} x^{2} + \dfrac{4}{3} a_{0} x^{4} \saltosin
\vdots \nonumber
\end{eqnarray*}
Ahora nos podemos plantear la siguiente pregunta: ¿cómo encontramos los coeficientes $a_{0}$ y $a_{1}$? La respuesta es: normalizando.
\\
Recordemos en el caso de los polinomios de Legendre, la normalización se escoge de tal manera que $P_{n}(1)=1$, con lo que
\[ \int_{-1}^{1} P_{n}(x) P_{m}(x) dx = \dfrac{2}{2n + 1} \delta_{nm} \]
En particular,
\[  P_{0}(x) = a_{0} \hspace{0.5cm} P_{0}(1) = 1 \Rightarrow a_{0} = 1 \Rightarrow P_{0} (x) = 1 \]
Que también se puede ver de la relación de ortogonalidad
\[ \int_{-1}^{1} a_{0}^{2} dx = 2 \Rightarrow a_{0}^{2} \; 2 = 2 \Rightarrow a_{0}^{2} = 1 \Rightarrow a_{0} = 1 \]
pero para los polinomios de Hermite
\[ \int_{- \infty}^{\infty} h_{0}^{2} (x) dx = \int_{- \infty}^{\infty} a_{0}^{2} dx = a_{0}^{2} x \bigg\vert_{-\infty}^{\infty} \rightarrow \infty \hspace{1.5cm} \text{la integral diverge!} \]
¿Qué sucede? El punto es que si $f(x)$ y $g(x)$ son funciones continuas arbitrarias por tramos en el intervalo $(-\infty, \infty)$, no se puede garantizar que la integral impropia
\begin{equation}
\int_{-\infty}^{\infty} f(x) g(x) dx = \lim_{a \to \infty \\ b \to \infty} \int_{-a}^{b} f(x)g(x) dx \rightarrow \text{ converja}
\label{eq:ecuacion_06_30}
\end{equation}
De hecho esta integral es indefinida aún cuando las funciones $f$ y $g$ sean polinomios, entonces la definición usual del producto interior no es válida!
\\
¿Ahora qué es lo que se hace? Se debe de considerar otro producto interior!
\begin{equation}
\int_{-\infty}^{\infty} \omega(x) \; P(x) g(x) dx
\label{eq:ecuacion_06_31}
\end{equation}
donde $\omega(x)$ se le denomina \emph{función de peso}. Es decir, estamos insistiendo en que cualquier espacio de funciones que consideremos contiene todos los polinomios y que su producto interior esté definido por medio de una integral impropia sobre toda la recta real. Es claro que $\omega (x)$ no puede ser cualquier función, ya que debe hacer el trabajo de que la integral (\ref{eq:ecuacion_06_31}) converja. La función $\omega (x)$ debe entonces satisfacer algunos requisitos.
\begin{enumerate}
\item $\omega(x)$ debe ser continua en toda la recta real.
\item el $\lim_{\vert x \vert \to \infty} \omega (x) \rightarrow 0$
\item Más aún $\lim_{\vert x \vert \to \infty} \omega (x) x^{n} \rightarrow 0$
\end{enumerate}
La experiencia que se tiene del curso de cálculo, sugiere que
\[  \omega(x) = e^{-x^{2}} \]
De hecho se puede demostrar que
\begin{equation}
\int_{-\infty}^{\infty} e^{-x^{2}} \; x^{n} d x = \begin{cases}
\dfrac{(2n)!}{2^{n}n!} \sqrt{\pi} & n = 0,2, 4, \ldots, 2n \\
0 & n = 1, 3, 5, \ldots, 2n - 1 \end{cases}
\label{eq:ecuacion_06_32}
\end{equation}
Nos podemos preguntar el espacio en el que se encuentran los polinomios de Hermite. Vamos a mencionar los siguientes teoremas.
\begin{teorema}
$L_{2}^{\omega}$ es un espacio euclidiano en el que se satisfacen las definiciones usuales de adición y multiplicación por un escalar de las funciones continuas por tramos y del producto interior.
\begin{equation}
\langle f \vert g \rangle = \int_{-\infty}^{\infty} e^{-x^{2}} \; f(x)g(x) dx
\label{eq:ecuacion_06_33}
\end{equation}
\end{teorema}
\begin{teorema}
$L_{2}^{\omega}$ es el conjunto de todas las funciones continuas por tramos en $(-\infty, \infty)$ que pueden expresarse como límites, o fronteras, en la media, de sucesiones de polinomios.
\end{teorema}
La base del espacio $L_{2}^{\omega}$ son los polinomios de Hermite, cuyo producto interior está dado por
\begin{equation}
\int_{-\infty}^{\infty} e^{-x^{2}} \; H_{n}(x) H_{m}(x) dx = 2^{n} n! \; \sqrt{\pi} \; \delta_{nm}
\label{eq:ecuacion_06_34}
\end{equation}
Nótese que hemos escrito en esta relación $H_{n}(x)$ en vez de $h_{n}(x)$. La razón de este cambio de notación se debe a que comunmente los polinomios se denotan por la letra $H$ una vez normalizados. Podemos calcular el valor de cada una de las constantes de los diferentes polinomios $h_{n}$ que tenemos y obtener así los polinomios de Hermite $H_{n}$.
\\
Una manera alternativa es calcularlos a través de una función generadora análoga a la fórmula de Rodrigues que se estudió para los polinomios de Legendre. Se defien así los polinomios de Hermite como los polinomios que se generan por la relación
\begin{equation}
H_{n} (x) = (-1)^{n} \; e^{x^{2}} \; \dfrac{d^{n}}{d x^{n}} e^{-x^{2}} \hspace{1.5cm} n = 0, 1, 2, \ldots
\label{eq:ecuacion_06_35}
\end{equation}
Por cálculo directo se tiene
\begin{eqnarray}
H_{0}(x) &=& 1 \label{eq:ecuacion_06_36} \\
H_{1}(x) &=& (-1) \; e^{x^{2}} \; \dfrac{d}{dx} e^{-x^{2}} =  2x \label{eq:ecuacion_06_36} \\
H_{2}(x) &=& (-1)^{2} \; e^{x^{2}} \; \dfrac{d^{2}}{d x^{2}} e^{-x^{2}} =  -2 e^{x^{2}} \; \dfrac{d}{d x} \left( x e^{-x^{2}} \right) =  4x^{2} - 2 \label{eq:ecuacion_06_36} \\
\end{eqnarray}
Los polinomios de Hermite satisfacen la relación de paridad
\begin{equation}
H_{n}(x) = (-1)^{n} H_{n}(-x)
\label{eq:ecuacion_06_41}
\end{equation}
y las siguientes relaciones de recurrencia
\begin{eqnarray*}
H_{n + 1}(x) - 2 x \; H_{n} + 2n \; H_{n-1}(x) &=& 0 \saltosin
H_{n + 1}(x) + \dfrac{d H_{n}}{dx} - 2x \; H_{n}(x) &=& 0 \nonumber
\end{eqnarray*}
\section{La familia de la ecuación de Hermite.}
%notas de Sepulveda.
A partir de la ecuación de Hermite y usando la transformación
\[ H_{n} (x) = e^{ax^{2}} \psi_{n}(\mu), \hspace{1.5cm} \mu = x^{b} \]
se obtiene la primera familia de Hermite:
\[ \begin{split}
 b^{2} \dfrac{d^{2} \psi_{n} (\mu)}{d \mu^{2}} \mu^{2(b-1)/b} &+ \dfrac{d \psi_{n}(\mu)}{d \mu} \left[ b(b - 1) \mu^{\frac{b - 2}{2}} +  \right. \\
&+ \left. 2b(2a - 1) \mu \right] + \left[ 4a (a - 1) \mu^{2/b} + 2n + 2a \right] \psi_{n} (\mu) = 0 \end{split}  \]
cuya solución es
\[ \psi_{n}(\mu) =  e^{-ax^{2}} \; H_{n}(x) = e^{-a \mu^{2} / b} \; H_{n}(\mu^{1/b}) \]
\begin{enumerate}
\item Con $a=0, b=1$, se recupera la ecuación de Hermite.
\item Si $b = 1, a = 1/2$, se obtiene la ecuación de \textbf{Weber-Hermite:}
\begin{equation}
\dfrac{d^{2} \psi_{n}(x)}{d x^{2}} + [1 + 2n - x^{2}] \psi_{n} (x) = 0
\label{eq:ecuacion_08_65}
\end{equation}
con $\mu = x$ y $\psi_{n}(x) = e^{-x^{2}/2} \; H_{n}(x)$. Esta última ecuación describe el oscilador armónico unidimensional en mecánica cuántica. La base $\{\psi_{n}(x)\}$ es ortonormal.
\end{enumerate}
\section{De nuevo el oscilador armónico cuántico.}
Para el oscilador armónico unidimensional con energía potencial $V = \frac{1}{2} k x^{2}$ se describe con la ecuación de Schrödinger
\[ - \dfrac{\hbar^{2}}{2m} \dfrac{d^{2} \psi(x)}{d x^{2}} +  V(x) \psi(x) =  E \psi(x) \]
donde
\begin{itemize}
\item $\psi(x)$ representa la función de onda.
\item $\hbar$ es la constante de Planck $h$ dividida por $2 \pi$.
\item $m$ es la masa del oscilador.
\item $E$ su energía total.
\end{itemize}
Cambiando la nueva variable adimensional $y = x / \alpha$ donde $\alpha$ tendrá la misma dimensión que $x$, y utilizando $\omega^{2} =  k /m$, siendo $\omega$ la frecuencia angular del oscilador y $k$ la constante del resorte, podemos escribir:
\[ \dfrac{d^{2} \psi}{d y^{2}} - \left( \dfrac{\omega m \alpha^{2}}{\hbar} \right)^{2} y^{2} \psi  + \left( \dfrac{2m E \alpha^{2}}{\hbar^{2}} \right) \psi = 0 \]
la adimensionalidad de $y$ y la homogeneidad de $\psi$ de la ecuación, permiten escoger para $\alpha$ un valor tal que el primer paréntesis tenga el valor de $1$, esto es
\[ \dfrac{\omega m \alpha^{2}}{\hbar} = 1 \hspace{1.5cm} \text{de donde } \alpha^{2} = \dfrac{\hbar}{m \omega} \]
el segundo paréntesis (adimensional) se le denominará $\lambda$:
\[ \lambda = \dfrac{2 m E \alpha^{2}}{\hbar^{2}} = \dfrac{2 E}{\hbar \omega} \]
la ecuación del oscilaor toma la forma:
\begin{equation}
\dfrac{d^{2} \psi}{d y^{2}} + (\lambda - y^{2}) \psi = 0
\label{eq:ecuacion_08_66}
\end{equation}
Esta ecuación conocida como \emph{ecuación de Weber-Hermite}, corresponde a la primera familia de Hermite con $b=1$ y $a=1/2$, tal que su solución es la función de Weber-Hermite de orden $n$ entero:
\begin{equation}
\psi_{n}(y) = e^{-y^{2}} \; H_{n}(y)
\label{eq:ecuacion_08_67}
\end{equation}
y por tanto $1 + 2n = \lambda$.
\\
Puesto que
\[ \lambda = \dfrac{2 E}{\hbar \omega} \Rightarrow \dfrac{2 E }{\hbar \omega} = 1 + 2n \]
de donde
\begin{equation}
E = \left(n + \dfrac{1}{2} \right) \hbar \omega, \hspace{1.5cm} n \geq 0
\label{eq:ecuacion_08_68}
\end{equation}
Como $n$ toma valores enteros, se sigue que la energía del oscilador está cuantizada y que hay una energía mínima o de punto cero
\[ E_{\text{mín}} = \dfrac{\hbar \omega}{2} \]
La secuencia de niveles de energía (\ref{eq:ecuacion_08_68}) tiene el espaciamiento $\Delta E = \hbar \omega$ postulado por Planck en 1900. Lo notable es que hay un mínimo de energía que no aparece en la teoría de Planck y que es exigido por el principio de incertidumebre: \emph{un oscilador armónico no puede estar en reposo}. Si lo estuviera, sería en $x=0$ que es el punto de equilibrio; en consecuencia podríamos conocer simultáneamente su posición y velocidad, lo que no es compatible con el principio de incertidumbre de Heinsenberg.
\begin{figure}[H]
\centering
\includestandalone{funcion_onda_oscilador}
\label{fig:figura_08_11}
\caption{Funciones de onda del oscilador armónico cuántico. La barra gruesa indica el rango permitido en el oscilador clásico con la misma energía.}
\end{figure}
La función de onda normalizada del oscilador será entonces, de acuerdo con la ec. (\ref{eq:ecuacion_08_67})
\begin{equation}
\psi_{n}(y) = \dfrac{1}{(2^{n} \sqrt{\pi} \; n!)^{1/2}} \; e^{-y^{2}/2} \; H_{n}(y), \hspace{1.5cm} y = \dfrac{x}{\alpha} = x \sqrt{\dfrac{m \omega}{\hbar}}
\label{eq:ecuacion_08_69}
\end{equation}
Explícitamente los primeros niveles de energía tienen la forma:
\begin{center}
\begin{tabular}{l l}
$y_{0}(y) = e^{-y^{2}/2}$ & $E = \dfrac{\hbar \omega}{2}$ \\
[0em] \\
$y_{1}(y) = 2 y e^{-y^{2}/2}$ & $E = \dfrac{3 \hbar \omega}{2}$ \\
[0em] \\
$y_{2}(y) = (4 y^{2} - 2) e^{-y^{2}/2}$ & $E = \dfrac{5 \hbar \omega}{2}$ 
\end{tabular}
\end{center}
La función $\vert \psi_{n}(y) \vert^{2}$ describe la probabilidad por unidad de volumen de encontrar la partícula en la posición $y$. La forma de la función de onda para los primeros valores de $n$ es mostrada en las figura (\ref{fig:figura_08_11}). Como $H_{n}(y)$ es un polinomio de grado $n$, $\psi_{n}(y)$ tendrá $n$ ceros. La probabilidad de encontrar la partícula en estos puntos es cero. En los extremos  $\psi_{n}(y)$ decrece rápidamente, en forma exponencial.
\section{Operadores escalera.}
Una de las identidades que satisface la función de Hermite es
\begin{equation}
H_{n-1}(x) = \dfrac{1}{2} \; \dfrac{d}{d x} H_{n}(x)\
\label{eq:ecuacion_08_70}
\end{equation}
Dado un $H_{n}(x)$ todos los anteriores pueden deducrise de él. Otra de las identidades  tiene la forma
\[ H_{n+1}(x) =  2x \; H_{n}(x) - 2 n \; H_{n-1}(x) \]
Al elimininar de ambas el término $H_{n-1}(x)$ se obtiene
\begin{equation}
H_{n+1}(x) = \left(2 x - \dfrac{d}{dx} \right) H_{n}(x)
\label{eq:ecuacion_08_71}
\end{equation}
expresión que nos dice que, dado un $H_{n}(x)$ todos los que le siguen pueden deducirse de él. Basta entonces con un $H_{n}(x)$ para generar los demás.
\\
Los operadores
\[ \dfrac{1}{2n} \dfrac{d}{dx} \hspace{1cm} \text{y} \hspace{1cm} 2x - \dfrac{d}{dx} \]
son los \emph{operadores escalera} de los polinomios de Hermite.
\\
En lo siguiente se introducen los operadores escalera osciados a las funciones de onda del oscilador armónico cuántico dada por la ecuación (\ref{eq:ecuacion_08_69}). Reemplazando $H_{n}(y)$ de la ecuación (\ref{eq:ecuacion_08_69}) en las ecuaciones (\ref{eq:ecuacion_08_70}) y (\ref{eq:ecuacion_08_71}), se obtiene:
\begin{eqnarray*}
\sqrt{2n} \; \psi_{n - 1}(y) &=& \left( y + \dfrac{d}{dy} \right) \; \psi_{n}(y) \saltosin
\sqrt{2(n + 1)} \psi_{n + 1}(y) &=& \left( y - \dfrac{d}{dy} \right) \; \psi_{n}(y) \nonumber
\end{eqnarray*}
que pueden escribire como
\begin{eqnarray}
\begin{aligned}
\widehat{a} \psi_{n} &=& \sqrt{n} \; \psi_{n-1} 
\widehat{a}^{\dagger} \psi &=& \sqrt{n + 1} \; \psi_{n + 1}
\end{aligned}
\label{eq:ecuacion_08_72}
\end{eqnarray}
donde los operadores $\widehat{a}$ y $\widehat{a}^{\dagger}$ son definidos por las ecuaciones
\begin{eqnarray*}
\widehat{a} &=& \dfrac{1}{\sqrt{2}} \left( y + \dfrac{d}{dy} \right) \saltosin
\widehat{a}^{\dagger} &=& \dfrac{1}{\sqrt{2}} \left( y - \dfrac{d}{dy} \right) \nonumber
\end{eqnarray*}
Estos operadores bajan y suben, respectivamente, estados cuánticos del oscilador. De la primera ecuación con $n=0$, se obtiene la función de onda normalizada del estado base
\[ \psi_{0}(y) = \pi^{-1/4} \; e^{-y^{2}/2} \]
A partir de esta expresión pueden calcularse todas las funciones de onda del oscilador armónico mediante la expresión:
\begin{equation}
\psi_{n} = \dfrac{1}{\sqrt{n}} \left( a^{\dagger} \right)^{n} \; \psi_{0}
\label{eq:ecuacion_08_73}
\end{equation}

\end{document}