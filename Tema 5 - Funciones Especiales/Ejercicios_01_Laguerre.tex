\documentclass[12pt]{article}
\usepackage[utf8]{inputenc}
\usepackage[spanish,es-lcroman, es-tabla]{babel}
\usepackage[autostyle,spanish=mexican]{csquotes}
\usepackage{amsmath}
\usepackage{amssymb}
\usepackage{nccmath}
\numberwithin{equation}{section}
\usepackage{amsthm}
\usepackage{graphicx}
\usepackage{epstopdf}
\DeclareGraphicsExtensions{.pdf,.png,.jpg,.eps}
\usepackage{color}
\usepackage{float}
\usepackage{multicol}
\usepackage{enumerate}
\usepackage[shortlabels]{enumitem}
\usepackage{anyfontsize}
\usepackage{anysize}
\usepackage{array}
\usepackage{multirow}
\usepackage{enumitem}
\usepackage{cancel}
\usepackage{tikz}
\usepackage{circuitikz}
\usepackage{tikz-3dplot}
\usetikzlibrary{babel}
\usetikzlibrary{shapes}
\usepackage{bm}
\usepackage{mathtools}
\usepackage{esvect}
\usepackage{hyperref}
\usepackage{relsize}
\usepackage{siunitx}
\usepackage{physics}
%\usepackage{biblatex}
\usepackage{standalone}
\usepackage{mathrsfs}
\usepackage{bigints}
\usepackage{bookmark}
\spanishdecimal{.}

\setlist[enumerate]{itemsep=0mm}

\renewcommand{\baselinestretch}{1.5}

\let\oldbibliography\thebibliography

\renewcommand{\thebibliography}[1]{\oldbibliography{#1}

\setlength{\itemsep}{0pt}}
%\marginsize{1.5cm}{1.5cm}{2cm}{2cm}


\newtheorem{defi}{{\it Definición}}[section]
\newtheorem{teo}{{\it Teorema}}[section]
\newtheorem{ejemplo}{{\it Ejemplo}}[section]
\newtheorem{propiedad}{{\it Propiedad}}[section]
\newtheorem{lema}{{\it Lema}}[section]

\title{Ejercicios Polinomios de Laguerre\\ \large {Matemáticas Avanzadas de la Física}  \vspace{-1.5\baselineskip}}
\date{}
\author{}
\begin{document}
\newgeometry{top=2cm, bottom=2cm, left=1.5cm, right=1.5cm,headsep=0pt}
\renewcommand\labelenumii{\theenumi.{\arabic{enumii})}}
\maketitle
\fontsize{14}{14}\selectfont
\begin{enumerate}
%Referencia Arfken Problema 13.2.9 
\item De acuerdo con la ecuación
\begin{align*}
\psi_{n l m} (r , \theta, \varphi) &= \left[ \left( \dfrac{2 \, Z}{n \, a_{0}} \right)^{3} \, \dfrac{(n - l -1)!}{2 \, n \, (n + l)!} \right]^{1/2} \, \exp \left( - \dfrac{\alpha \, r}{2} \right) * \\[1em]
&* (\alpha \, r)^{L} \, L_{n - l +1}^{2 l +1}(\alpha \, r) \, Y_{l}^{m} (\theta, \varphi)
\end{align*}
la parte normalizada de la función de onda para el átomo de hidrógeno es
\begin{align*}
R_{n l} (r) = \left[ \alpha^{3} \dfrac{(n -l -1)!}{2 \, n \, (n + l)!} \right]^{1/2} \, \exp \left( \dfrac{-\alpha \, r}{2} \right) \, (\alpha \, r)^{l} \, L_{n - l +1}^{2 l +1}(\alpha \, r) 
\end{align*}
en donde 
\begin{align*}
\alpha = \dfrac{2 \, Z}{n \, a_{0}} = \dfrac{2 \, Z \, m \, e^{2}}{4 \, \pi \, \epsilon_{0} \, \hbar^{2}}
\end{align*}
La cantidad $\expval{r}$ es el desplazamiento promedio del electrón con respecto al núcleo, mientras que $\expval{r^{-1}}$ es el promedio del movimiento recíproco.
\par
Demuestra que al evaluar las siguientes integrales se obtienen los valores indicados:
\begin{enumerate}
\item \textbf{(1 punto.) } $\displaystyle \expval{r} = \int_{0}^{\infty} r \, R_{n l} (\alpha \, r) \, R_{n l} (\alpha \, r) \, r^{2} \dd{r} = \dfrac{a_{0}}{2} [3 \, n^{2} - l (l + 1)]$
\item \textbf{(1 punto.) } $\displaystyle \expval{r^{-1}} = \int_{0}^{\infty} r^{-1} \, R_{n l} (\alpha \, r) \, R_{n l} (\alpha \, r) \, r^{2} \dd{r} = \dfrac{1}{n^{2} \, a_{0}}$
\end{enumerate}
\item Mediante el teorema del desarrollo, demuestra que la expansión de la función $f(x) = \exp(- a \, x)$ en la base $L_{n}^{k} (x)$ dejando $k$ fijo mientras que $n$ cambia de $0$ a $\infty$, es
\begin{align*}
\exp(-a \, x) = \dfrac{1}{(1 + a)^{1 + k}} \, \sum_{n=0}^{\infty} \left( \dfrac{a}{1 + a} \right)^{n} \, L_{n}^{k} (x) \hspace{1.5cm} 0 \leq x < \infty
\end{align*}
\begin{enumerate}
\item \textbf{(1 punto) } Evalúa directamente los coeficientes en el desarrollo supuesto.
\item \textbf{(1 punto) } Desarrolla la expansión a partir de la función generatriz.
\end{enumerate}
\end{enumerate}
\end{document}