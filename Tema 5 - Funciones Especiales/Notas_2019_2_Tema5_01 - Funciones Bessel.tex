\documentclass[12pt]{article}
\usepackage[utf8]{inputenc}
\usepackage[spanish,es-lcroman, es-tabla]{babel}
\usepackage[autostyle,spanish=mexican]{csquotes}
\usepackage{amsmath}
\usepackage{amssymb}
\usepackage{nccmath}
\numberwithin{equation}{section}
\usepackage{amsthm}
\usepackage{graphicx}
\usepackage{epstopdf}
\DeclareGraphicsExtensions{.pdf,.png,.jpg,.eps}
\usepackage{color}
\usepackage{float}
\usepackage{multicol}
\usepackage{enumerate}
\usepackage[shortlabels]{enumitem}
\usepackage{anyfontsize}
\usepackage{anysize}
\usepackage{array}
\usepackage{multirow}
\usepackage{enumitem}
\usepackage{cancel}
\usepackage{tikz}
\usepackage{circuitikz}
\usepackage{tikz-3dplot}
\usetikzlibrary{babel}
\usetikzlibrary{shapes}
\usepackage{bm}
\usepackage{mathtools}
\usepackage{esvect}
\usepackage{hyperref}
\usepackage{relsize}
\usepackage{siunitx}
\usepackage{physics}
%\usepackage{biblatex}
\usepackage{standalone}
\usepackage{mathrsfs}
\usepackage{bigints}
\usepackage{bookmark}
\spanishdecimal{.}

\setlist[enumerate]{itemsep=0mm}

\renewcommand{\baselinestretch}{1.5}

\let\oldbibliography\thebibliography

\renewcommand{\thebibliography}[1]{\oldbibliography{#1}

\setlength{\itemsep}{0pt}}
%\marginsize{1.5cm}{1.5cm}{2cm}{2cm}


\newtheorem{defi}{{\it Definición}}[section]
\newtheorem{teo}{{\it Teorema}}[section]
\newtheorem{ejemplo}{{\it Ejemplo}}[section]
\newtheorem{propiedad}{{\it Propiedad}}[section]
\newtheorem{lema}{{\it Lema}}[section]

\title{Funciones de Bessel \\ {\large Tema 5 - Matemáticas Avanzadas de la Física}\vspace{-1.5\baselineskip}}
\date{ }
\author{}
\begin{document}
\maketitle
\fontsize{14}{14}\selectfont
%Sadri Hassani - Mathematical methods for students of physics. Chap. 27 Laplace equation: cylindrical coordinates
Antes de desarrollar un problema con una geometría cilíndrica, consideremos una pregunta que tiene implicaciones más generales. Vimos en el Tema 2 que la separación de variables condujo a un sistema de EDO en las que aparecían ciertas constantes de separación, y que elegir el signo de esa constante nos lleva a una forma funcional diferente de la solución general.
\par
Por ejemplo, para una ecuación como
\begin{align*}
\dv[2]{x}{t} - k \, x = 0
\end{align*}
se pueden tener soluciones exponenciales si $k > 0$ y soluciones trigonométricas si $k < 0$ Uno no puede asignar a priori un signo específico a $k$. Por lo tanto, la forma general de la solución es indeterminada. Sin embargo, \emph{una vez que se imponen las CDF}, las soluciones únicas surgirán independientemente de la forma funcional inicial de las soluciones.
\par
El siguiente argumento ilustra este punto en la ED angular resultante de la separación de variables para la ecuación de Laplace en coordenadas cilíndricas.
\section{Funciones de Bessel.}
La separación de variables en coordenadas cilíndricas para la ecuación de Laplace
\begin{align*}
\laplacian{\Phi} = 0
\end{align*}
mediante la solución propuesta $\Phi (\rho, \varphi, z) = R(\rho) S(\varphi) Z(z)$, nos lleva a un sistema de tres EDO:
\begin{align}
\begin{aligned}
\dv{\rho} \left( \rho \dv{R}{\rho} \right) + \left( \lambda \, \rho + \dfrac{\mu}{\rho} \right) \, R &= 0 \\
\dv[2]{S}{\varphi} - \mu \, S &= 0 \\
\dv[2]{Z}{z} - \lambda \, Z &= 0
\end{aligned}
\label{eq:ecuacion_27_01}
\end{align}
Si nos fijamos en la segunda ecuación cuya solución más general podemos escribir como
\begin{align}
S (\varphi) = \begin{cases}
A \, \exp(\sqrt{\mu} \, \varphi) + B \, \exp(-\sqrt{\mu} \, \varphi) & \mbox{ si } \mu \neq 0 \\
C \,\varphi + D & \mbox { si } \mu = 0
\end{cases}
\label{eq:ecuacion_27_02}
\end{align}
No importa que tipo de CDF se impongan al potencial $\Phi$, ya que debe de devolver el mismo valor en $\varphi$ y en $\varphi + 2 \pi$, mientras que se mantengan las otras dos variables fijas\footnote{Este argumento es válido solo para los casos físicos definidos para todo el rango de $\varphi$. Si la región de interés restringe a $\varphi$ en un subconjunto del intervalo $[0, 2 \pi]$, el argumento ya no funciona.}. Esto debido a que $(\rho, \varphi, z)$ y $(\rho, \varphi + 2 \pi, z)$ representan físicamente al mismo punto en el espacio. Se sigue entonces que
\begin{align*}
R(\rho) S(\varphi) Z(z) = R(\rho) S(\varphi + 2 \pi) Z(z) \Longrightarrow S(\varphi + 2 \pi) = S (\varphi)
\end{align*}
ya que la identidad se mantiene para todos los valores de $\rho$ y $z$. Si la última relación es verdadera para el caso de $\mu = 0$, entonces tenemos que $C = 0$ y $S(\varphi) = D$. Para $\mu neq 0$, la ec. (\ref{eq:ecuacion_27_02}) es
\begin{align*}
A \, \exp(\sqrt{\mu}(\varphi + 2 \pi)) + B \, \exp(- \sqrt{\mu}(\varphi + 2 \pi)) = A \, \exp(\sqrt{\mu} \, \varphi) 
\end{align*}
o también
\begin{align*}
A \, \exp(\sqrt{\mu} \, \varphi) \, (\exp(\sqrt{\mu} \, 2 \pi) - 1) + B \, \exp(- \sqrt{\mu} \, \varphi) \, (\exp(- \sqrt{\mu} \, 2 \pi) - 1) = 0
\end{align*}
que debe de cumplirse para todo $\varphi$; la única manera en la que esto puede ocurrir (cuidando de que $A$ y $B$ no sean nulos) es mediante
\begin{align*}
\exp(\sqrt{\mu} \, 2 \pi) - 1 = 0 \hspace{1cm} y \hspace{1cm} \exp(- \sqrt{\mu} \, 2 \pi) - 1 = 0
\end{align*}
En ambos casos se tiene que $\exp(\sqrt{\mu} 2 \pi) = 1$
\par
Si nos limitamos a valores reales de $\mu$, obtendremos soluciones triviales. Para prevenir esto, tenemos que hacer
\begin{align*}
\sqrt{\mu} = i \, m \hspace{1cm} m = 0, \pm 1, \pm 2, \ldots
\end{align*}
o
\begin{align*}
\mu = -m^{2}  \hspace{1cm} m = 0, \pm 1, \pm 2, \ldots
\end{align*}
Con esta elección de $\mu$, la ED para $S(\varphi)$ es
\begin{align*}
S^{\prime \prime} + m^{2} \, S = 0
\end{align*}
que tiene como solución general: una suma de funciones trigonométricas.
\begin{teo}
Para todos los problemas físicos para los cuales el ángulo azimutal varíe entre $0$ y $2 \pi$, uno está forzado a restringir el valor de $\mu$ al negativo de la raíz cuadrada de un entero. La solución para la parte angular es entonces
\begin{equation}
S (\varphi) = A_{m} \, \cos m \varphi + B_{m} \, \sin m \varphi, \hspace{1cm} m = 0, 1, 2, \ldots
\label{eq:ecuacion_27_03}
\end{equation}
donde $A_{m}$ y $B_{m}$ son constantes que pueden ser distintas para diferentes valores de $m$.
\end{teo}
Los valores negativos de $m$ no darán lugar a nuevas soluciones, por lo que no se incluyen en el rango de $m$. El caso de $\mu = 0$ no se necesita tratar por separado, ya que la solución aceptable para este caso es $S = D = \mbox{ constante}$, la cual es la que se obtiene en la ec. (\ref{eq:ecuacion_27_03}) cuando $m = 0$.
\par
La ED para $Z(z)$ es independiente de $m$ y tiene una solución exponencial si $\lambda > 0$ y una solución trigonométrica si $\lambda < 0$. Asumiendo ésta forma y escribiendo $\lambda \equiv l^{2}$, se tiene
\begin{equation}
Z(z) = A \, e^{l \, z} + B \, e^{- li \, z}
\label{eq:ecuacion_27_04}
\end{equation}
La más familiar de las ED es la ecuación radial, en términos de $l = \sqrt{\lambda}$, se puede escribir como
\begin{equation}
\dv[2]{R}{\rho} + \dfrac{1}{\rho} \, \dv{R}{\rho} + \left( l^{2} - \dfrac{m^{2}}{\rho^{2}} \right) \, R = 0
\label{eq:ecuacion_27_05}
\end{equation}
Además, si definimos la variable $v = l\, \rho$, podemos transformar la ec. (\ref{eq:ecuacion_27_05}) en la forma:
\begin{equation}
\dv[2]{R}{\rho} + \dfrac{1}{v} \, \dv{R}{v} + \left( 1 - \dfrac{m^{2}}{v^{2}} \right) \, R = 0
\label{eq:ecuacion_27_06}
\end{equation}
Tanto la ec. (\ref{eq:ecuacion_27_05}) o (\ref{eq:ecuacion_27_06}), es una de las ED de la física matemática más famosas: \textbf{la ecuación diferencial de Bessel}.
\section{Soluciones para la ED de Bessel.}
El método de Frobenius es una manera efectiva para encontrar las soluciones de las EDO. Al reescribir la ec. (\ref{eq:ecuacion_27_06}) multiplicando por $v^{2}$ para convertir todos sus coeficientes en polinomios como lo sugiere la ecuación (\ref{eq:ecuacion_26_07}).
\begin{equation}
p_{2} \, \dv[2]{y}{x} + p_{1}(x) \, \dv{y}{x} + p_{0} \, y
 = 0
\label{eq:ecuacion_26_07}
\end{equation}
Esto nos lleva a
\begin{equation}
v^{2} \, \dv[2]{R}{v} + v \dv{R}{v} + (v^{2} - m^{2}) \, R = 0
\label{eq:ecuacion_27_07}
\end{equation}
Como $v^{2}$ se anula cuando $v = 0$, podemos proponer una solución de la forma
\begin{align*}
R(v) &= v^{s} \sum_{k=0}^{\infty} c_{k} \, v^{k} = \\
&= \sum_{k=0}^{\infty} c_{k} \, v^{k+s}
\end{align*}
de la cual obtenemos al diferenciar en dos ocasiones:
\begin{align*}
v \, \dv{R}{v} &= \sum_{k=0}^{\infty} c_{k} (k + s) \, v^{k+s} \\
v^{2} \, \dv[2]{R}{v} &= \sum_{k=0}^{\infty} c_{k} (k + s)(k + s - 1) \, v^{k+s} 
\end{align*}
sustituyendo estos, como
\begin{align*}
(v^{2} - m^{2}) \sum_{k=0}^{\infty} c_{k} \, v^{k+s}
\end{align*}
en la ED, nos conduce a
\begin{align*}
\sum_{k=0}^{\infty} c_{k} [\underbrace{k + s + (k + s)(k + s - 1)}_{=(k+s)^{2}} - m^{2} ] \, v^{k+s} + \sum_{k=0}^{\infty} c_{k} \, v^{k+s+2} = 0
\end{align*}
Para encontrar la relación de recurrencia, necesitamos que el valor de la potencia de $v$ sea el mismo en las sumas, por lo que re-escribimos la primera suma como
\begin{align*}
c_{0}(s^{2} &- m^{2}) v^{s} + c_{1} [(s + 1)^{2}- m^{2}] v^{s+1} + \sum_{k=0}^{\infty} c_{k} [(k + s)^{2} - m^{2}] v^{k+s} \\
&= c_{0}(s^{2} - m^{2}) v^{s} + c_{1} [(s + 1)^{2}- m^{2}] v^{s+1} \\
&+ \sum_{n=0}^{\infty} c_{n+2} [(n + 2 + s)^{2} - m^{2}] v^{n+2+s}
\end{align*}
donde en la segunda línea, usamos $n = k -2$. Como el índice $n$ es mudo, lo podemos regresar de nuevo al índice $k$. Entonces, seguimos con que
\begin{align*}
c_{0}(s^{2} &- m^{2}) v^{s} + c_{1} [(s + 1)^{2}- m^{2}] v^{s+1} + \\
&+ \sum_{k=0}^{\infty} \left\{ c_{k+2} [(k + 2 + s)^{2} - m^{2}] + c_{k} \right\} \, v^{k+2+s} = 0
\end{align*}
Suponemos que $c_{0} \neq 0$ y que todos los demás coeficinetes de las potencias de $v$ se anulan, obtenemos
\begin{align*}
s^{2} &= m^{2} \\
c_{1} [(s + 1)^{2}- m^{2}] &= 0\\
c_{k+2} [(k + 2 + s)^{2} - m^{2}] + c_{k} &= 0
\end{align*}
La primera ecuación nos dice que $m = \pm s$. Por lo que al usarlo en la segunda ecuación, resulta en
\begin{align*}
c_{1} (2 \, s + 1) = 0 \hspace{0.5cm} \Rightarrow \hspace{0.5cm} c_{1} = 0 \hspace{1cm} \mbox{ o } \hspace{0.5cm} s = - \dfrac{1}{2}
\end{align*}
La elección $s = -1/2$ nos devuelve que $m = \mp 1/2$ la cual no es aceptable\footnote{En realidad, los problemas que surgen de otras áreas de la física más allá de la electrostática y la transferencia de calor en estado estacionario permiten valores de $m$ no enteros. Sin embargo, no trataremos tales problemas aquí.}, por lo que decidimos que $m$ sea un entero positivo. Por lo tanto concluimos que $s = \pm m$ y $c_{1} = 0$. 
\par
Se sigue que la regla de recurrencia para todos los coeficientes impares son cero. La serie de Frobenius entonces será
\begin{equation}
R(v) = v^{\pm m} \sum_{k=0}^{\infty} c_{2k} \, v^{2k}, \hspace{1.5cm} \dfrac{c_{2k+2}}{c_{2k}} = - \dfrac{1}{(2k + 2 + s)^{2} - m^{2}}
\label{eq:ecuacion_27_08}
\end{equation}
La prueba del cociente para la convergencia de la serie nos lleva a
\begin{align*}
\lim_{k \to \infty} \abs{\dfrac{c_{2k+2} \, v^{2k+2}}{c_{2k} \, v^{2k}}} = \lim_{k \to \infty} \abs{\dfrac{1}{(2k + 2 + s)^{2} -m^{2}}} \, v^{2} = 0
\end{align*}
Lo que nos indica que la serie de la ec. (\ref{eq:ecuacion_27_08}) es convergente para todos los valores de $v$.
\par
Ahora usaremos la relación de recurrencia para obtener los coeficientes de la expansión. Reescribimos la relación de recuerrencia como
\begin{align*}
c_{k+2} = - \dfrac{1}{(k + 2 + s)^{2} - s^{2}} \, c_{k} = \dfrac{1}{(k + 2)(2 s + k + 2)} \, c_{k}
\end{align*}
donde hemos sustituido $s^{2}$ por $m^{2}$. Lo que nos devuelve
\begin{align*}
c_{2} &= - \dfrac{1}{2 (2 s + 2)} c_{0} \\
c_{4} &= - \dfrac{1}{4 (2 s + 4)} c_{2} = (-1)^{2} \dfrac{1}{4 (2 s + 4)} \, \dfrac{1}{2 (2 s + 2)} \, c_{0} \\
c_{6} &= - \dfrac{1}{6 (2 s + 6)} c_{4} = (-1)^{3} \dfrac{1}{6 (2 s + 6)} \, \dfrac{1}{4 (2 s + 4)} \, \dfrac{1}{2 (2 s + 2)} \, c_{0}
\end{align*}
y de manera general
\begin{align*}
c_{2k} = (-1)^{k} \dfrac{1}{\underbrace{2 k \cdot (2 k - 2) \ldots 2}_{=2^{k} \, k!} \underbrace{(2 s + 2 k)[2 s + (2 k - 2)] \ldots (2 s + 2)}_{=2^{k} (s + k)(s + k + 1) \ldots (s + 1)}} \, c_{0}
\end{align*}
Multiplicando el numerador y el denominador por $s!$, llegamos a
\begin{equation}
c_{2k} = (-1)^{k} \dfrac{s!}{2^{2k} \, k! \, (s+k)!} \, c_{0}
\label{eq:ecuacion_27_09}
\end{equation}
Sustituyendo la ec. (\ref{eq:ecuacion_27_09}) en la (\ref{eq:ecuacion_27_08}), nos conduce abs
\begin{align*}
R(v) &= c_{0} \, s! \, v^{s} \sum_{k=0}^{\infty} \dfrac{(-1)^{k}}{2^{2k} \, k! \, (s + k)!} \, v^{2k} \\
&= c_{0} \, s! \, 2^{s} \, \left( \dfrac{v}{2} \right)^{s} \sum_{k=0}^{\infty} \dfrac{(-1)^{k}}{k! \, (s + k)!} \, \left( \dfrac{v}{2} \right)^{2 k} 
\end{align*}
donde sustituimos $s$ por $\pm m$ en el exponente de $v$ fuera de la suma. También absorbimos las potencias de $2$ en el denominador de la suma en las potencias de $v$, y fuera de la suma, multiplicamos y dividimos por $2^{s}$. Es habitual elegir que la constante arbitraria $c_{0}$ sea igual a $1/(s! \, 2^{s})$. Esto nos lleva a 
\begin{tcolorbox}
Las \textbf{funciones de Bessel} de orden $s$, que se escribe como $J_{s}$ y está dada por la serie
\begin{equation}
J_{s}(x) = \left( \dfrac{x}{2} \right)^{s} \sum_{k=0}^{\infty} \dfrac{(-1)^{k}}{k!\, (s + k)!} \, \left( \dfrac{x}{2} \right)^{2k}
\label{eq:ecuacion_27_10}
\end{equation}
la cual es convergente para todos los valores de $x$.
\end{tcolorbox}
Aunque la Ecuación (\ref{eq:ecuacion_27_10}) se obtuvo asumiendo que $m$, y por lo tanto $s$, era un número entero, al levantar esta restricción se obtendría una serie que es convergente en todas partes, y uno puede definir funciones de Bessel cuyas órdenes son números reales o incluso complejos. La única dificultad es interpretar correctamente $(s + n)!$ para $s$ no enteros. Pero esto es precisamente para lo que se inventó la función gamma. Por lo tanto, dejaremos que la ecuación (\ref{eq:ecuacion_27_10}) represente las funciones de Bessel de todos los órdenes.

\section{Segunda solución a la ED de Bessel.}
Se puede obtener una segunda solución de la ED de Bessel, usando la ec. (\ref{eq:ecuacion_24_06}), donde $p(x) = 1/x$.
\begin{equation}
f_{2}(x) =  f_{1}(x) \left\{ C + K \int_{\alpha}^{x} \dfrac{1}{f_{1}^{2} (s)} \, \exp( - \int_{c}^{s} p(t) \dd{t}) \dd{s} \right\}
\label{eq:ecuacion_24_06}
\end{equation}
Usando $J_{m}(x)$ como entrada, podemos generar la segunda solución; con $C = 0$ en la ec. (\ref{eq:ecuacion_24_06}), obtenemos
\begin{align*}
Z_{m}(x) = K \, J_{m}(x) \, \int_{\alpha}^{x} \dfrac{1}{J_{m}^{2}(x)} \exp(- \int_{c}^{u} \dfrac{\dd{t}}{t}) \dd{u} = A_{m} \, J_{m}(x) \int_{\alpha}^{x} \dfrac{\dd{u}}{u \, J_{m}^{2}(u)}
\end{align*}
donde $A_{m} \equiv K , c$ y $\alpha$ son constantes arbitrarias que se establecen por convención.
\par
Nótese que, contrario a $J_{m}(x)$, la solución $Z_{m}(x)$ no se comporta bien en $x = 0$, debido a la presencia de $u$ en el denominador del integrando.
\par
Aunque el procedimiento anterior genera una segunda solución para la ED  de Bessel, no es el procedimiento habitual. Resulta que para los no enteros $s$, la función de Bessel $J_{-s} (x)$ es independiente de $J_{s} (x)$ y se puede usar como una segunda solución. Sin embargo, una segunda solución más común es la combinación lineal
\begin{equation}
Y_{s} (x) = \dfrac{J_{s}(x) , \cos s \pi - J_{-s}(x)}{\sin s\pi}
\label{eq:ecuacion_27_11}
\end{equation}
que es llamada la función de Bessel de segunda clase, o \textbf{Funciones de Neumann}.
\par
Para enteros $s$ la función es indeterminada debido a que
\begin{align}
\begin{aligned}
J_{-m}(x) &= \left(\dfrac{x}{2} \right)^{-m} \sum_{n=0}^{\infty} \dfrac{(-1)^{m+n}}{(m+n)! \, \Gamma(n+1)} \, \left( \dfrac{x}{2} \right)^{2m +2n} = \\
&= (-1)^{m} \left( \dfrac{x}{2} \right)^{m} \sum_{n=0}^{\infty} \dfrac{(-1)^{n}}{\Gamma(m+n+1) \, n!} \, \left( \dfrac{x}{2} \right)^{2n} = \\
&= (-1)^{m} \, J_{m}(x)    
\end{aligned}
\label{ecuacion_11_32}
\end{align}
y por la identidad $\cos n \pi = (-1)^{n}$. Por lo tanto, usamos la regla de l'Hôpital y definimos
\begin{align*}
Y_{n}(x) &\equiv \lim_{s \to n} Y_{s} (x) = \lim_{s \to n} \dfrac{\displaystyle \pdv{s} \left[ J_{s}(x) \,\cos s \pi - J_{-s}(x) \right]}{\pi \, \cos n \pi} \\[1em]
&= \dfrac{1}{\pi} \lim_{s \to n} \left[ \pdv{J_{s}}{s} - (-1)^{n} \pdv{J_{-s}}{s} \right]
\end{align*}
De la ec. (\ref{eq:ecuacion_27_10}) se obtiene que
\begin{align*}
\pdv{J_{s}}{s} = J_{s} (x) \, \ln \left( \dfrac{x}{2} \right) - \left( \dfrac{x}{2} \right)^{s} \sum_{k=0}^{\infty} (-1)^{k} \, \dfrac{\Psi(s + k + 1)}{k! \, \Gamma(s + k + 1)} \, \left( \dfrac{x}{2} \right)^{2k}
\end{align*}
donde
\begin{align*}
\Psi (x) \equiv \dv{x} \ln[(x - 1)!] = \dv{x} \ln \Gamma (x) = \dfrac{\dv*{\Gamma(x)}{x}}{\Gamma (x)}
\end{align*}
De manera similar
\begin{align*}
\pdv{J_{-s}}{s} = -J_{-s} (x) \, \ln \left( \dfrac{x}{2} \right) + \left( \dfrac{x}{2} \right)^{-s} \sum_{k=0}^{\infty} (-1)^{k} \, \dfrac{\Psi(-s + k + 1)}{k! \, \Gamma(-s + k + 1)} \, \left( \dfrac{x}{2} \right)^{2k}
\end{align*}
Al sustituir estas expresiones en la definición de $Y_{n}(x)$ y usando la identidad $J_{-n} = (-1)^{n} J_{n}(x)$, se obtiene
\begin{align}
\begin{aligned}
Y_{n} (x) &= \dfrac{2}{\pi} \, J_{n} (x) \, \ln \left(\dfrac{x}{2} \right) - \dfrac{1}{\pi} \left( \dfrac{x}{2} \right)^{n} \sum_{k=0}^{\infty} (-1)^{k} \dfrac{\Psi (n + k + 1)}{k! \, \Gamma (n + k + 1)} \left(\dfrac{x}{2} \right)^{2k} + \\
&- \dfrac{1}{\pi} (-1)^{n} \left( \dfrac{x}{2} \right)^{-n} \sum_{k=0}^{\infty} (-1)^{k} \dfrac{\Psi (k - n + 1)}{k! \, \Gamma (k - n + 1)} \left(\dfrac{x}{2} \right)^{2k}
\end{aligned}
\label{eq:ecuacion_27_12}
\end{align}
Debe de quedar claro con la ec. (\ref{eq:ecuacion_27_12}) que la función de Neumann $Y_{s} (x)$ está mal definida en $x = 0$, como se espera de la segunda solución para la ED de Bessel, como la función $Z_{m} (x)$ discutida anteriormente.
\par
Como $Y_{s} (x)$ es linealmente independiente de $J_{s} (x)$ para cualquier $s$, entero o no entero, es conveniente considerar $\left\{ Js (x), Ys (x) \right\}$ como base de las soluciones para la ED de Bessel. En particular, la solución de la ecuación radial en coordenadas cilíndricas, es decir, la primera ecuación en (\ref{eq:ecuacion_27_01}), se convierte en
\begin{equation}
R(\rho) = A \, J_{m}(v) + B \, Y_{m}(v) = A \, J_{m} (l \, \rho) + B \, Y_{m} (l \, \rho)
\label{eq:ecuacion_27_13}
\end{equation}
\section{Propiedades de las funciones de Bessel.}
\end{document}