\input{../Preambulos/preambulo_presentacion_Montepellier_default}
\title{\large{Tema 5 - Funciones Especiales}}
\subtitle{Objetivos}
\author{M. en C. Gustavo Contreras Mayén}
\date{}
\institute{Facultad de Ciencias - UNAM}
\titlegraphic{\includegraphics[width=1.75cm]{../Imagenes/escudo-facultad-ciencias}\hspace*{4.75cm}~%
   \includegraphics[width=1.75cm]{../Imagenes/escudo-unam}
}
\setbeamertemplate{navigation symbols}{}
\begin{document}
\maketitle
\fontsize{14}{14}\selectfont
\spanishdecimal{.}
\section*{Contenido}
\frame[allowframebreaks]{\tableofcontents[currentsection, hideallsubsections]}
\section{Funciones Especiales}
\frame{\tableofcontents[currentsection, hideothersubsections]}
\subsection{Otra vez: ¿para qué las FE?}
\begin{frame}
\frametitle{Pensamiento}
Alberto Grünbaum del Departamento de Matemáticas de la Universidad de Berkeley, es uno de los mayores expertos mundiales en aplicaciones de funciones especiales y polinomios ortogonales, en particular en probabilidad, procesos estocásticos, sistemas integrables, imagenología médica (tomografía computarizada), biomatemáticas y mecánica estadística, entre otros.
\end{frame}
\begin{frame}
\frametitle{Pensamiento}
El Dr. Grünbaum expresó en una ocasión:
\\
\bigskip
\pause
\begin{quote}
\enquote{Las funciones especiales son para las matemáticas lo que las tuberías son para una casa: nadie quiere exhibirlas abiertamente, pero nada funciona sin ellas.}
\end{quote}
\end{frame}
\section{Lo que veremos}
\frame{\tableofcontents[currentsection, hideothersubsections]}
\begin{frame}
\frametitle{Contenido importante}
Esta parte del curso incluye un contenido muy relevante ya que se trata del manejo y formalismo matemático que tendremos que utilizar como físicos, a partir de este sexto semetre y en lo que resta de nuestra vida profesional.
\end{frame}
\subsection{Funciones de Laguerre}
\begin{frame}
\frametitle{Laguerre - Motivación}
Se tomará como punto de partida la ecuación radial del átomo de hidrógeno para obtener los \emph{polinomios ordinarios de Laguerre} y \emph{los polinomios asociados de Lagurre}.
\end{frame}
\begin{frame}
\frametitle{Laguerre - Descripción matemática}
Así mismo, se revisarán las características y propiedades de los polinomios, buscando conectar su uso con las aplicaciones en problemas de la física matemática.
\end{frame}
\subsection{Funciones de Hermite}
\begin{frame}
\frametitle{Hermite - Motivación}
Continuando con ejemplos de mecánica cuántica, ahora tomaremos el estudio del oscilador armónico cuántico, lo que nos conducirá a los \emph{polinomios de Hermite}.
\end{frame}
\subsection{Funciones de Bessel}
\begin{frame}
\frametitle{Bessel - Motivación y geometría}
La consideración de la ecuación de Laplace en un sistema coordenado cilíndrico polar, nos llevará a establecer las \emph{funciones de Bessel}, conocidas como funciones de Bessel de primera clase, y las funciones de Bessel de segunda clase llamadas \emph{funciones de Neumann}.
\end{frame}
\begin{frame}
\frametitle{Bessel - Geometría esférica}
Como un caso particular del estudio con una geometría esférica, tendremos la oportunidad de revisar las \emph{funciones de Bessel esféricas}.
\\
\bigskip
En el Tema 4 nos concentramos a revisar las funciones de Legendre, siendo más oportuno dejar su estudio para este momento.
\end{frame}
\subsection{Funciones de Chebychev}
\begin{frame}
\frametitle{Funciones de Chevychev}
Para este par de funciones especiales: Chebychev de tipo I y Chebychev de Tipo II, tomaremos como punto de partida la ecuación diferencial correspondiente.
\\
\bigskip
\pause
Se revisarán las propiedades y características de cada una de ellas.
\end{frame}
\subsection{Funciones hipergeométricas}
\begin{frame}
\frametitle{La función hipergeométrica}
Estudiaremos la ecuación diferencial como punto de partida y encontraremos que habrá casos especiales o límite, en donde recuperamos otras funciones especiales.
\\
\bigskip
Revisaremos la \emph{función hipergeométrica confluente}.
\end{frame}
\subsection{Funciones de Gegenbauer}
\begin{frame}
\frametitle{Funciones de Gegenbauer}
Este tipo de funciones que se obtienen a partir de la serie hipergeométrica para los casos en donde ésta, es finita, además, son la solución de la ecuación diferencial de Gegenbauer, una generalización de los polinomios de Legendre.
\end{frame}
\section{Calendarización}
\frame{\tableofcontents[currentsection, hideothersubsections]}
\subsection{Distribución de tiempos}
\begin{frame}
\frametitle{Trabajo por semanas}
Considerando que ya tenemos en puerta el período vacacional de fin de año, que comienza a partir del 12 de diciembre, se propone el siguiente esquema de trabajo:
\end{frame}
\begin{frame}
\frametitle{Semanas 11 y 12}
\setbeamercolor{item projected}{bg=blue!70!black,fg=yellow}
\setbeamertemplate{enumerate items}[circle]
\begin{enumerate}[<+->]
\item Del 2 al 4 de diciembre, se trabajaría con el material correspondiente a las funciones de Laguerre.
\item Del 7 al 11 de diciembre, se revisaría el material correspondiente a las funciones de Hermite y de Bessel.
\end{enumerate}
\end{frame}
\begin{frame}
\frametitle{Semana 13 y 14}
Del 4 al 15 de enero del 2021, se trabajaría con los materiales correspondientes a las funciones de Chebychev, la hipergeométrica y las funciones de Gengenbauer.
\end{frame}
\begin{frame}
\frametitle{Sesiones de trabajo}
Se programan sesiones de trabajo síncronas para los días miércoles:
\begin{itemize}
\item 9 de diciembre de 2020.
\item 6 de enero de 2021.
\item 13 de enero de 2021.
\end{itemize}
En el acostumbrado horario de las 3 pm.
\end{frame}
\begin{frame}
\frametitle{Sesiones de los viernes}
Se mantienen programadas las sesiones de los viernes a las 3 pm:
\begin{itemize}
\item 11 de diciembre de 2020.
\item 8 de enero de 2021.
\item 15 de enero de 2021.
\end{itemize}
\end{frame}
\end{document}