\documentclass[12pt]{article}
\usepackage[left=0.25cm,top=1cm,right=0.25cm,bottom=1cm]{geometry}
\textwidth = 20cm
\hoffset = -1cm
\usepackage[utf8]{inputenc}
\usepackage[spanish,es-tabla]{babel}
\usepackage[autostyle,spanish=mexican]{csquotes}
\usepackage[tbtags]{amsmath}
\usepackage{nccmath}
\usepackage{amsthm}
\usepackage{amssymb}
\usepackage{graphicx}
\usepackage{standalone}
\usepackage[outdir=./]{epstopdf}
\usepackage{siunitx}
\usepackage{physics}
\usepackage{color}
\usepackage{float}
\usepackage{multicol}
%\usepackage{milista}
\usepackage{enumitem}
\usepackage{anyfontsize}
\usepackage{anysize}
\usepackage{enumitem}
\usepackage{capt-of}
\usepackage{bm}
\usepackage{relsize}
\usepackage{placeins}
\usepackage{empheq}
\usepackage{cancel}
\usepackage{wrapfig}
\spanishdecimal{.}
\renewcommand{\baselinestretch}{1.5} 
\renewcommand\labelenumii{\theenumi.{\arabic{enumii}}}
\newcommand{\ptilde}[1]{\ensuremath{{#1}^{\prime}}}
\newcommand{\stilde}[1]{\ensuremath{{#1}^{\prime \prime}}}
\newcommand{\ttilde}[1]{\ensuremath{{#1}^{\prime \prime \prime}}}
\newcommand{\ntilde}[2]{\ensuremath{{#1}^{(#2)}}}


%\usepackage{showframe}
\title{Oscilaciones en una cadena vertical \\ \large {Tema 5 - Funciones especiales} \vspace{-3ex}}
\author{M. en C. Gustavo Contreras Mayén}
\date{ }
\begin{document}
\vspace{-4cm}
\maketitle
\fontsize{14}{14}\selectfont
\tableofcontents
\newpage

%Ref. Bowman - Introduction to Bessel Functions 27

\section{Oscilaciones pequeñas en una cadena colgante uniforme y flexible.}

Tomando el origen $O$ como punto de equilibrio en el extremo inferior de una cadena (Fig. 7). Sea $l$ su longitud, $\lambda$ su masa por unidad de longitud.

Considere el movimiento de un elemento PQ de longitud dx, con su punto medio a una altura x por encima de 0. Sea T la tensión en el punto medio del elemento. La componente horizontal de esta tensión es aproximadamente — T por J dx, y las componentes correspondientes en los extremos P, Q son


Estos actúan en direcciones opuestas sobre el elemento y causan su movimiento; la ecuación de movimiento es por lo tanto:

los puntos al final indican términos desconocidos de mayor orden de pequeñez.

Ahora T = gXx aproximadamente, ignorando el movimiento vertical; por lo tanto

Dividiendo por X dx, dado que A es constante, tenemos, en el límite, la ecuación diferencial

Para encontrar los modos normales de vibración, hacemos la sustitución

donde X es una función de x solamente, y después de dividir por cos (out — e) encontramos que X debe satisfacer la ecuación

Esto, tal como está, no es una ecuación de Bessel, sino la sustitución

se transforma en una ED de Bessel:

La interpretación de tau es la siguiente. La velocidad de onda en una cuerda con tensión uniforme igual a la que se obtiene en el punto x sería sqrt(P/ro) o sqrt(g x). El tiempo que tardaría un punto geométrico que se mueve siempre con esta velocidad local en viajar desde el extremo inferior hasta el punto x es int_{0}^}{x} dx/sqrt(gx) = tau
\end{document}