%\RequirePackage[l2tabu, orthodox]{nag}
\documentclass[12pt]{article}
\usepackage[utf8]{inputenc}
\usepackage[spanish,es-lcroman, es-tabla]{babel}
\usepackage[autostyle,spanish=mexican]{csquotes}
\usepackage{amsmath}
\usepackage{amssymb}
\usepackage{nccmath}
\numberwithin{equation}{section}
\usepackage{amsthm}
\usepackage{graphicx}
\usepackage{epstopdf}
\DeclareGraphicsExtensions{.pdf,.png,.jpg,.eps}
\usepackage{color}
\usepackage{float}
\usepackage{multicol}
\usepackage{enumerate}
\usepackage[shortlabels]{enumitem}
\usepackage{anyfontsize}
\usepackage{anysize}
\usepackage{array}
\usepackage{multirow}
\usepackage{enumitem}
\usepackage{cancel}
\usepackage{tikz}
\usepackage{circuitikz}
\usepackage{tikz-3dplot}
\usetikzlibrary{babel}
\usepackage{bm}
\usepackage{mathtools}
\usepackage{esvect}
\usepackage{hyperref}
\usepackage{relsize}
\usepackage{siunitx}
\usepackage{physics}
%\usepackage{biblatex}
\usepackage{standalone}
\usepackage{mathrsfs}
\usepackage{bigints}
\usepackage{bookmark}
\spanishdecimal{.}

\setlist[enumerate]{itemsep=0mm}

\renewcommand{\baselinestretch}{1.5}

\let\oldbibliography\thebibliography

\renewcommand{\thebibliography}[1]{\oldbibliography{#1}

\setlength{\itemsep}{0pt}}
%\marginsize{1.5cm}{1.5cm}{2cm}{2cm}


\newtheorem{defi}{{\it Definición}}[section]
\newtheorem{teo}{{\it Teorema}}[section]
\newtheorem{ejemplo}{{\it Ejemplo}}[section]
\newtheorem{propiedad}{{\it Propiedad}}[section]
\newtheorem{lema}{{\it Lema}}[section]

\marginsize{1cm}{1cm}{2cm}{2cm}
% \pagestyle{fancy}
% \fancyhf{}
% \rhead{Examen - Tarea 3}
% \rfoot{\thepage}
\renewcommand{\headrulewidth}{0.5pt}
\setlength{\headheight}{30pt} 
%\usepackage[left=1.5cm,top=1.5cm,right=1.5cm,bottom=1.5cm]{geometry}
\author{}
\date{}
\title{Examen Reposición - Funciones especiales \\ \large{Matemáticas Avanzadas de la Física}\vspace{-20pt}}
\begin{document}
\newgeometry{margin=1.5cm}
\maketitle
\fontsize{14}{14}\selectfont
\begin{enumerate}
\item Un cilindro largo conductor de calor de radio $a$ se compone de dos mitades (con secciones transversales semicirculares) con un espacio infinitesimal entre ellas. Las mitades superior e inferior del cilindro están en contacto con baños térmicos $+T_{1}$ y $-T_{1}$, respectivamente. El cilindro está dentro de otro cilindro de radio $b$ más grande ( $a < b$ y coaxial con él) que se mantiene a la temperatura $T_{2}$. Encuentra la temperatura en: a) dentro del cilindro interno, b) entre los dos cilindros y c) fuera del cilindro externo. Las respuestas deberán de expresarse con las funciones de Bessel.
\item Demuestra que:
\begin{align*}
\int_{-\infty}^{\infty} x^{2} \, \exp(-x^{2}) \, H_{n} (x) \, H_{n} (x) \dd{x} = \pi^{1/2} \, 2^{n} \, n! \,  \left(n + \dfrac{1}{2} \right)
\end{align*}
\item Considera la función de onda
\begin{align*}
\psi_{n} (x) = \dfrac{\exp(x^{2}/2) \, H_{n} (x)}{2^{n} \, n! \, \pi^{1/2}}
\end{align*}
Demuestra que  los operadores $a_{n}$ y $a_{n}^{\dagger}$ definidos como
\begin{align*}
a_{n} &= \dfrac{1}{\sqrt{2}} \left( x + \dv{x} \right) \\
a_{n}^{\dagger} &= \dfrac{1}{\sqrt{2}} \left( x - \dv{x} \right)
\end{align*}
satisfacen 
\begin{align*}
a_{n} \, \psi_{n} &= n^{1/2} \, \psi_{n-1} (x) \\[1em]
a_{n}^{\dagger} \, \psi_{n} &= \left(n + \dfrac{1}{2} \right)^{1/2} \, \psi_{n+1} (x)
\end{align*}
\item Con los resultados del problema anterior, demuestra que el Hamiltoniano dado por
\begin{align*}
H = - \dfrac{1}{2} \, \dv[2]{x} + \dfrac{x^{2}}{2}
\end{align*}
es equivalente a
\begin{align*}
\mathcal{H} = \dfrac{1}{2} \, \left( a \, a^{\dagger} + a^{\dagger} \, a \right)
\end{align*}
\item 
\begin{enumerate}
\item Demuestra la siguiente igualdad entre operadores
\begin{align*}
x - \dv{x} = - \exp(x^{2}/2) \, \dv{x} \, \exp(-x^{2}/2)
\end{align*}
\item Para una función de onda
\begin{align*}
\psi_{n} (x) = \left( \pi^{1/2} \, 2^{n} \, n! \right) \, \exp(-x^{2}/2) \, H_{n} (x)
\end{align*}
ocupando el resultado del inciso anterior, demuestra que la función de onda se puede escribir como
\begin{align*}
\psi_{n} (x) = \left( \pi^{1/2} \, 2^{n} \, n! \right) \, \left( x - \dv{x} \right)^{n} \, \exp(-x^{2}/2) 
\end{align*}
\end{enumerate}
\end{enumerate}
\end{document}