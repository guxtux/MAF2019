\documentclass[12pt]{article}
\usepackage[left=0.25cm,top=1cm,right=0.25cm,bottom=1cm]{geometry}
\textwidth = 20cm
\hoffset = -1cm
\usepackage[utf8]{inputenc}
\usepackage[spanish,es-tabla]{babel}
\usepackage[autostyle,spanish=mexican]{csquotes}
\usepackage[tbtags]{amsmath}
\usepackage{nccmath}
\usepackage{amsthm}
\usepackage{amssymb}
\usepackage{graphicx}
\usepackage{standalone}
\usepackage[outdir=./]{epstopdf}
\usepackage{siunitx}
\usepackage{physics}
\usepackage{color}
\usepackage{float}
\usepackage{multicol}
%\usepackage{milista}
\usepackage{enumitem}
\usepackage{anyfontsize}
\usepackage{anysize}
\usepackage{enumitem}
\usepackage{capt-of}
\usepackage{bm}
\usepackage{relsize}
\usepackage{placeins}
\usepackage{empheq}
\usepackage{cancel}
\usepackage{wrapfig}
\spanishdecimal{.}
\renewcommand{\baselinestretch}{1.5} 
\renewcommand\labelenumii{\theenumi.{\arabic{enumii}}}
\newcommand{\ptilde}[1]{\ensuremath{{#1}^{\prime}}}
\newcommand{\stilde}[1]{\ensuremath{{#1}^{\prime \prime}}}
\newcommand{\ttilde}[1]{\ensuremath{{#1}^{\prime \prime \prime}}}
\newcommand{\ntilde}[2]{\ensuremath{{#1}^{(#2)}}}


%\usepackage{showframe}
\usepackage{apacite}
\title{Vibraciones en una membrana circular \\ \large {Tema 5 - Funciones de Bessel} \vspace{-3ex}}
\author{M. en C. Gustavo Contreras Mayén}
\date{ }
\begin{document}
\vspace{-4cm}
\maketitle
\fontsize{14}{14}\selectfont
\tableofcontents
\newpage
\section{Introducción.}
Consideramos ondas estacionarias en un sistema bidimensional con simetría circular: la de una membrana circular delgada y flexible (por ejemplo, un parche circular de tambor idealizado) de radio $R$, la ecuación de onda en coordenadas cilíndricas en dos dimensiones $(x , y \Rightarrow r, \varphi)$ para la amplitud del desplazamiento, $\psi (r, \varphi, t)$ viene dada por:
\begin{align*}
\laplacian{\psi} (r, \varphi, t) - \dfrac{1}{v^{2}} \, \pdv[2]{\psi (r, \varphi, t)}{t} = 0
\end{align*}
Donde la velocidad longitudinal de propagación de ondas transversales en una membrana circular bidimensional estirada está (también) dada por 
\begin{align*}
v = \sqrt{\dfrac{T_{\ell}}{\sigma}}
\end{align*}
donde $T_{\ell}$ es la tensión superficial de la membrana (en $\si{\newton\per\metre}$) y $\sigma \equiv M/A = M / \pi \, R^{2}$ es la densidad de masa superficial de la membrana (en $\si{\kilo\gram\per\square\metre}$). Con $x = r \, \cos \varphi$, $y = r \, \sin \varphi$, $\dd[2]{r} = r \, \dd{r} \dd{\varphi}$, el operador Laplaciano, $\laplacian$ en coordenadas cilíndricas en 2D viene dado por:
\begin{align*}
\laplacian = \dfrac{1}{r} \, \pdv{r} \left( r \, \pdv{r} \right) + \dfrac{1}{r^{2}} \, \pdv[2]{\phi}
\end{align*}
Así, la ecuación de onda bidimensional que describe el comportamiento de las ondas en una membrana cilíndrica viene dada por:
\begin{align*}
\pdv[2]{\psi}{r} + \dfrac{1}{r} \pdv{\psi}{r} + \dfrac{1}{r^{2}} \, \pdv[2]{\psi}{\varphi} = \dfrac{1}{v^{2}} \, \pdv[2]{\psi}{t}
\end{align*}
Notemos que el lado izquierdo de la igualdad (lado derecho de la igualda) contiene solo funciones dependientes del espacio (dependientes del tiempo), respectivamente. Por lo tanto, podemos usar la técnica de separación de variables, con
\begin{align*}
\psi (r, \varphi, t) = U(r, \varphi) \, T(t)
\end{align*}
donde $U (r, \varphi)$ contiene solo términos espacialmente dependientes de $r$ y $\varphi$ y $T (t)$ contiene solo el término dependiente del tiempo.
\par
Tenemos la relación
\begin{align*}
v = f \lambda = \left( \dfrac{\omega}{2 \, \pi} \right) \left( \dfrac{2 \, \pi}{k} \right) = \dfrac{\omega}{k}
\end{align*}
entonces $v \, k = \omega$.
\par
Obtenemos una constante de separación $-k^{2}$, y después de algunas manipulaciones algebraicas simples, obtenemos las siguientes dos ED2H:
\begin{align*}
\pdv[2]{\psi}{r} + \dfrac{1}{r} \pdv{\psi}{r} + \dfrac{1}{r^{2}} \, \pdv[2]{\psi}{\varphi} + k^{2} \, U(r, \varphi) &= 0\\[0.5em]
\dv[2]{T(t)}{t} + \omega \, T(t) &= 0
\end{align*}
Podemos usar nuevamente la técnica de separación de variables en la ecuación espacial anterior, con una solución de producto de la forma
\begin{align*}
U(r, \varphi) = R(r) \, \Phi (\varphi)
\end{align*}
Incorporando este producto en la ecuación espacial anterior y realizando las diferenciaciones (parciales), dividiendo por $U(r, \varphi)$ y realizando una manipulación algebraica simple, obtenemos la siguiente ecuación:
\begin{align*}
\dfrac{1}{R} \left( r^{2} \, \dv[2]{R}{r} +  r \, \dv{R}{r} + k^{2} \, r^{2} \, R(r) \right) = - \dfrac{1}{\Phi} \, \dv[2]{\Phi}{\varphi}
\end{align*}
El lado izquierdo de la igualdad (el lado derecho) de esta ecuación depende solo de $r$ (solo de $\varphi$), respectivamente.
\par
Nuevamente, esto solo puede ser cierto para todos los valores posibles de $(r, \varphi)$, si tanto el lado izquierdo de la expresión como el lado derecho, son iguales a una constante (adimensional).
\par
Sabemos que las soluciones$\Phi(\varphi)$ deben ser periódicas y univaluadas, es decir:
\begin{align*}
\Phi (\varphi = 0) = \Phi (\varphi = 2 \, \pi)
\end{align*}
o de manera general
\begin{align*}
\Phi (\varphi = \varphi_{0}) = \Phi (\varphi = \varphi_{0} + 2 \, m \, \pi), \hspace{1.5cm} m = 0, \pm 1, \pm 2,\pm 3, \ldots
\end{align*}
Por lo tanto, elegiremos esta constante de separación para que sea $m^{2}$.
\par
Las funciones propias de $\Phi_{m} (\varphi)$ son (una de) las dos siguientes formas equivalentes:
7La ecuación radial es la conocida ecuación diferencial de Bessel:
\begin{align*}
\dv[2]{R}{r} + \dfrac{1}{r} \, \dv{R}{r} + \left( k^{2} - \dfrac{m^{2}}{r^{2}} \right) \, R = 0
\end{align*}
La solución más general a la ecuación de Bessel, con $m$ entero (que es el caso que aquí tenemos), es de la forma:
\begin{align*}
R_{m} (r) = A_{m} \, J_{m} (k \, r) + B_{m} \, Y_{m} (k \, r)
\end{align*}
donde $J_{m} (x)$ son las funciones ordinarias de Bessel de orden $m$, y las $Y_{m} (x)$ son las funciones de Bessel de segundo orden.
\par
Las funciones $J_{m} (x)$ son finitas en $x = 0$ y se expresan normalmente en términos de una serie de potencias en $x$:
\begin{align*}
J_{m} (x) = \sum_{r=0}^{\infty} \dfrac{(-1)^{r}}{r! \, \Gamma(m + r + 1)} \, \left( \dfrac{x}{2} \right)^{m+2r}
\end{align*}
Sabemos que para $m$ entero: $J_{-m}(x) = (-1)^{m} \, J_{m}(x)$. Las funciones $Y_{m}(x)$ se pueden expresar de diversas maneras, entre ellas:
\begin{align*}
Y_{m}(x) = \dfrac{J_{m}(x) \, \cos (m \, \pi) - J_{-m}(x)}{\sin (m \, \pi)}
\end{align*}
Para $m$ entero, también se pueden escribir como:
\begin{align*}
Y_{m}(x) = \dfrac{1}{\pi} \left[ \pdv{J_{m}(x)}{m} - (-1)^{m} \, \pdv{J_{-m}(x)}{m} \right]
\end{align*}
Las funciones $Y_{m} (x)$ son singulares (se vuelven (negativas) infinitas) en $x = 0$. Sin embargo, debido a que usamos coordenadas cilíndricas para nuestra membrana circular, el origen $(r = 0)$ \textbf{se incluye} en este problema.
\par
Físicamente, \textbf{NO} se permiten desplazamientos de amplitud infinitos tal que $R(r) \to \infty$ para cualquier valor de $r$, ya que una suposición inicial implícita eran oscilaciones de \emph{amplitud pequeñas}. Por tanto, todos los coeficientes $B_{m}$ para el $Y_{m} (x)$ deben ser $B_{m} = 0$ para soluciones en valores propios físicamente permitidas de la membrana circular bidimensional.
\par
La condición de frontera radial para ondas estacionarias transversales en una membrana circular con borde fijo (es decir, desplazamiento transversal cero) en $r = R$ es $R_{m} (r = R) = 0$, es decir, $J_{m} (k \, R) = 0$.
\par
Dado que $r = R > 0$, esto significa que buscamos los ceros de $J_{m} (k \, R)$, es decir, $J_{m} (x = k \,R) = 0$. Debido a la complejidad de la forma de $J_{m} (x)$, los ceros de $J_{m} (x)$ (y $Y_{m} (x)$) son de naturaleza no analítica, más bien, están tabulados en muchos libros de matemáticas, o pueden determinarse mediante técnicas numéricas gráficas y/o computacionales. Presentamos los primeros ceros del $J_{m} (x)$ de orden inferior en la siguiente tabla:
\begin{table}[H]
\centering
\begin{tabular}{c c r c c c}
 & & & $n = 1$ & $n = 2$ & $n = 3$ \\
$m = 0$: & $J_{0}(x) = 0$ & $x \approx$ & $2.40$ & $5.52$ & $8.65$ \\
$m = 1$: & $J_{1}(x) = 0$ & $x \approx$ & $3.83$ & $7.02$ & $10.17$ \\
$m = 2$: & $J_{2}(x) = 0$ & $x \approx$ & $5.14$ & $8.42$ & $11.62$ \\
\end{tabular}
\end{table}
Como $x = k \, R$, entonces $k = x / R$ y notando que nuevamente, para este problema de valores propios de onda estacionaria bidimensional tenemos dos índices, $m$ y $n$ para denotar:
\begin{enumerate}
\item Los números de onda propios 
\begin{align*}
k_{m, n} = \dfrac{x_{m, n}}{R}
\end{align*}
\item Las frecuencias propias
\begin{align*}
\omega_{m,n} = v \, k_{m, n} \hspace{1cm} f_{m,n} = \dfrac{v}{\lambda_{m,n}}
\end{align*}
con $v = \sqrt{T_{\ell} / \sigma}$
\item Las energías propias 
\begin{align*}
E_{m,n} = 1/4 \, M \, \omega_{m,n}^{2} \, A_{m,n}^{2}
\end{align*}
\item Las funciones propias 
\begin{align*}
\psi_{m,n} (r, \varphi, t) = R_{m,n}(r) \, \Phi_{m}(\varphi) \, T_{m,n}(t)
\end{align*}
\end{enumerate}
Las soluciones de modos propios de la ecuación de onda temporal asociada para ondas estacionarias bidimensionales en una membrana circular tienen las siguientes formas equivalentes:
\begin{align*}
\mathbf{i)} \hspace{0.2cm} T_{m,n} (t) &= b_{m,n} \, \sin (\omega_{m,n} \, t) + c_{m,n} \, \cos (\omega_{m,n} \, t) \\[0.5em]
-1 &\leq b_{m,n} \leq 1 \hspace{1cm} -1 \leq c_{m,n} \leq 1 \hspace{1cm} \sqrt{b_{m,n}^{2} + c_{m,n}^{2}} = 1 \\[0.5em]
\mathbf{ii)} \hspace{0.2cm} T_{m,n} (t) &= \sin (\omega_{m,n} \, t + \delta_{m,n}) = \cos(\omega_{m,n} \, t + \varphi_{m,n}) \hspace{1cm} \delta_{m,n} = \varphi_{m,n} + \dfrac{\pi}{2} \\[0.5em]
\mathbf{ii)} \hspace{0.2cm} T_{m,n} (t) &= \exp(i(\omega_{m,m} \, t + \varphi_{m,n}))
\end{align*}
Las soluciones completas en modos propios para ondas estacionarias bidimensionales en una membrana circular de radio, $R$ con bordes fijos están dadas por alguna de las siguientes expresiones que son equivalentes:
\begin{align*}
\psi_{m,n} (r ,\varphi, t) &= R_{m,n} (r) \, \Phi_{m} (\varphi) \, T_{m,n} (t) \\[0.5em]
\psi_{m,n} (r ,\varphi, t) &= A_{m,n} \, J_{m,n} (k_{m,n} \, R) \, \exp(i(m \, \varphi + \delta_{m,n})) \, \exp(i(\omega_{m,n} \, t + \varphi_{m,n})) \\[0.5em]
\psi_{m,n} (r ,\varphi, t) &= A_{m,n} \, J_{m,n} (k_{m,n} \, R) \, \big[ \alpha_{m,n} \,  \cos (m \, \varphi) + \beta_{m} \, \sin (m \, \varphi) \big] \times \\[0.5em]
&\times \big[ b_{m,n} \,  \cos (\omega_{m,n} \, t) + c_{m,n} \, \sin (\omega_{m,n} \, t) \big]
\end{align*}
Con:
\begin{enumerate}[label=\alph*)]
\item Frecuencias propias
\begin{align*}
f_{m,n} = \dfrac{\omega_{m,n}}{2 \, \pi} = \dfrac{v \, k_{m,n}}{2 \, \pi} = \dfrac{v}{\lambda_{m,n}}
\end{align*}
\item Longitudes de onda propias
\begin{align*}
\lambda_{m,n} = \dfrac{2 \, \pi}{k_{m,n}} = \dfrac{2 \, \pi \, R}{x_{m,n}}
\end{align*}
\item Energías propias
\begin{align*}
E_{m,n} = \dfrac{1}{4} \, M \, \omega_{m,n}^{2} \, A_{m,n}^{2}
\end{align*}
\begin{align*}
m = 0, 1, 2, 3, \ldots \hspace{1cm} n = 1, 2, 3, \ldots
\end{align*}
\end{enumerate}
Los modos más bajos de ondas estacionarias transversales en una membrana circular se enumeran a continuación:
\begin{table}[H]
\centering
\begin{tabular}{c c c c l}
$m {=} 0$ & $n {=} 1$ & $k_{0,1} {\approx} 2.40/R$ & $\omega_{0,1} {\approx} 2.40 \, v/R$ & $\psi_{0,1}(r, \varphi, t) {=} A_{0,1} \, J_{0}(k_{0,1} \, R) \, T_{0,1}(t)$ \\[0.5em] \hline
$m {=} 1$ & $n {=} 1$ & $k_{1,1} {\approx} 3.83/R$ & $\omega_{1,1} {\approx} 3.83 v/R$ & \\[0.5em]
\multicolumn{5}{l}{$\psi_{1,1}(r, \varphi, t) {=} A_{1,1} J_{1}(k_{1,1} R) \big[ \alpha_{1} \cos \varphi {+} \beta_{1} \sin \varphi \big] T_{0,1}(t)$} \\[0.5em] \hline
$m {=} 2$ & $n {=} 1$ & $k_{2,1} {\approx} 5.14/R$ & $\omega_{2,1} {\approx} 5.14 v/R$ & \\[0.5em]
\multicolumn{5}{l}{$\psi_{2,1}(r, \varphi, t) {=} A_{2,1} J_{2}(k_{2,1} R) \big[ \alpha_{2} \cos 2 \varphi {+} \beta_{2} \sin 2 \varphi \big] T_{2,1}(t)$} \\[0.5em] \hline
$m {=} 0$ & $n {=} 2$ & $k_{0,2} {\approx} 5.52/R$ & $\omega_{0,1} {\approx} 5.52 \, v/R$ & $\psi_{0,2}(r, \varphi, t) {=} A_{0,2} \, J_{0}(k_{0,2} \, R) \, T_{0,2}(t)$
\end{tabular}
\end{table}
Algunos de los modos de vibración propios de orden inferior para ondas estacionarias transversales en una membrana circular (con líneas nodales discontinuas) se muestran en la siguiente figura:
\begin{table}[H]
\centering
\begin{tabular}{c@{\hskip 0.5cm} c@{\hskip 0.5cm} c@{\hskip 0.5cm}}
\includestandalone{Figuras/Modos_Vibracion_Membrana_0_1} & \includestandalone{Figuras/Modos_Vibracion_Membrana_0_2} & 
\includestandalone{Figuras/Modos_Vibracion_Membrana_0_3} \\
\multicolumn{3}{c}{} \\
\includestandalone{Figuras/Modos_Vibracion_Membrana_1_1} & \includestandalone{Figuras/Modos_Vibracion_Membrana_1_2} & 
\includestandalone{Figuras/Modos_Vibracion_Membrana_1_3} \\
\multicolumn{3}{c}{} \\
\includestandalone{Figuras/Modos_Vibracion_Membrana_2_1} & \includestandalone{Figuras/Modos_Vibracion_Membrana_2_2} & 
\includestandalone{Figuras/Modos_Vibracion_Membrana_2_3}
\end{tabular}
Primeros modos de vibración en una membrana circular fija.
\end{table}
Las degeneraciones de orden 2 para $m > 0$ surgen de nuevo debido a los 2 grados espaciales de libertad ($x$ e $y$, $r$ y $\varphi$). La simetría rotacional de la membrana circular: es invariante en rotaciones arbitrarias.
\par
Una siguiente parte de estudio sobre los modos de vibración de una membrana circular, es sobre la manera de visualizar las oscilaciones, si de hecho en la figura anterior, se nota la descripción por zonas en la membrana, se puede extender el estudio, siendo necesario el apoyo de herramientas ya sea de programación o de software matemático. En la siguiente figura se presenta el mismo arreglo descriptivo, pero ahora con más información, se utilizó el  software \texttt{Mathematica}, para resolver la ecuación diferencial, obtener los ceros de las funciones de Bessel y finalmente, ocupar una rutina de graficación para visualizar de otra manera el mismo resultado:
\begin{figure}[H]
    \centering
    \includegraphics[scale=0.8]{Imagenes/Modos_Vibracion_Membrana_Circular_01.eps}
\end{figure}
En donde ahora nos encontramos con zonas de color, las zonas claras refieren a un posición de la membrana contraria a las zonas oscuras, cuya posición es opuesta a las zonas claras.
\end{document}