\documentclass[12pt]{beamer}
\usepackage{../Estilos/BeamerMAF}
\usepackage{../Estilos/ColoresLatex}
\input{../Preambulos/preambulo_Beamer_Copenhagen_wolverine}

\setbeamercolor{section in foot}{bg=peridot, fg=black}
\setbeamercolor{subsection in foot}{bg=powderblue(web), fg=black}


\makeatletter
\setbeamertemplate{footline}
{
\leavevmode%
\hbox{%
\begin{beamercolorbox}[wd=.333333\paperwidth,ht=2.25ex,dp=1ex,center]{section in foot}%
  \usebeamerfont{section in foot} \insertsection
\end{beamercolorbox}%
\begin{beamercolorbox}[wd=.333333\paperwidth,ht=2.25ex,dp=1ex,center]{subsection in foot}%
  \usebeamerfont{subsection in foot}  \insertsubsection
\end{beamercolorbox}%
\begin{beamercolorbox}[wd=.333333\paperwidth,ht=2.25ex,dp=1ex,right]{date in head/foot}%
  \usebeamerfont{date in head/foot} \insertshortdate{} \hspace*{1.5em}
  \insertframenumber{} / \inserttotalframenumber \hspace*{2ex} 
\end{beamercolorbox}}%
\vskip0pt%
}
\makeatother
\usefonttheme{serif}
\setbeamercolor{frametitle}{bg=palecerulean}
\resetcounteronoverlays{saveenumi}

\date{12 de mayo de 2022}

\title{\large{Tema 5 - Funciones Especiales}}
\subtitle{Funciones Especiales Parte II}
\author{M. en C. Gustavo Contreras Mayén}

\begin{document}
\maketitle
\fontsize{14}{14}\selectfont
\spanishdecimal{.}

\section*{Contenido}
\frame{\frametitle{Temas a revisar} \tableofcontents[currentsection, hideallsubsections]}

\section{Más funciones especiales}
\frame{\frametitle{Puntos dentro del tema} \tableofcontents[currentsection, hideothersubsections]}
\subsection{Formalismo y requisitos}

\begin{frame}
\frametitle{Contenido importante}
Esta parte del curso incluye un contenido muy relevante ya que se trata del manejo y formalismo matemático que tendremos que utilizar como físicos, a partir de este sexto semestre y en lo que resta de nuestra vida profesional.
\end{frame}
\begin{frame}
\frametitle{Sobre los temas en los ejemplos}
Nuevamente tendremos que apoyarnos con distintas áreas de la física para presentar problemas que nos conduzcan a una función especial.
\\
\bigskip
\pause
La construcción completa del ejercicio, es decir, la base teórica en cada tema, será algo que daremos que ya manejan en el sexto semestre.
\end{frame}
\begin{frame}
\frametitle{Sobre los temas en los ejemplos}
En caso de que tengan alguna complicación para el planteamiento de un ejercicio en un tema particular, deberán de apoyarse con las referencias en cada tema.
\\
\bigskip
\pause
De esta manera tendrán completo el contexto del ejercicio, de tal manera que la solución del ejercicio y su interpretación será mucho más fácil de realizar.
\end{frame}

\section{Objetivos}
\frame[allowframebreaks]{\frametitle{Objetivos del Tema 5} \tableofcontents[currentsection, hideothersubsections]}
\subsection{Conocimiento y habilidades}

\begin{frame}
\frametitle{Objetivos}
Al concluir el Tema 5, se espera que el alumno:
\setbeamercolor{item projected}{bg=darkcoral,fg=white}
\setbeamertemplate{enumerate items}{%
\usebeamercolor[bg]{item projected}%
\raisebox{1.5pt}{\colorbox{bg}{\color{fg}\footnotesize\insertenumlabel}}%
}
\begin{enumerate}[<+->]
\item Reconozca que las funciones especiales se obtienen de las soluciones de EDO2 que se recuperan luego de haber planteado un problema físico bajo ciertas condiciones en la geometría asociada, así como en las CDF y condiciones iniciales.                            
\seti
\end{enumerate}
\end{frame}
\begin{frame}
\frametitle{Objetivos}
\setbeamercolor{item projected}{bg=darkcoral,fg=white}
\setbeamertemplate{enumerate items}{%
\usebeamercolor[bg]{item projected}%
\raisebox{1.5pt}{\colorbox{bg}{\color{fg}\footnotesize\insertenumlabel}}%
}
\begin{enumerate}[<+->]
\conti
\item Identifique el conjunto de propiedades para las funciones especiales, en particular:
\begin{itemize}
\item La función generatriz.
\item Las relaciones de recurrencia.
\item La condición de ortogonalidad y normalización.
\item La fórmula de Rodrigues.
\item La condición de paridad.
\end{itemize}
\seti
\end{enumerate}
\end{frame}
\begin{frame}
\frametitle{Objetivos}
\setbeamercolor{item projected}{bg=darkcoral,fg=white}
\setbeamertemplate{enumerate items}{%
\usebeamercolor[bg]{item projected}%
\raisebox{1.5pt}{\colorbox{bg}{\color{fg}\footnotesize\insertenumlabel}}%
}
\begin{enumerate}[<+->]
\conti
\item Utilice el conjunto de propiedades de las funciones especiales para resolver problemas con determinada geometría.
\item Adquiera una metodología de trabajo para estudiar una función especial que no se haya revisado en el curso.
\end{enumerate}
\end{frame}

\section{Las funciones especiales}
\frame[allowframebreaks]{\frametitle{Funciones especiales a revisar} \tableofcontents[currentsection, hideothersubsections]}

\subsection{Funciones de Laguerre}

\begin{frame}
\frametitle{Laguerre - Motivación}
Se tomará como punto de partida la ecuación radial del átomo de hidrógeno para obtener los \emph{\textbf{\textcolor{darkslateblue}{los polinomios asociados de Laguerre}}} \pause y como un caso especial de éstos, los \emph{\textbf{\textcolor{darktan}{polinomios ordinarios de Laguerre}}}.
\end{frame}
\begin{frame}
\frametitle{Laguerre - Descripción matemática}
Así mismo, se revisarán las características y propiedades de los polinomios, buscando conectar su uso con las aplicaciones en problemas de la física matemática.
\end{frame}

\subsection{Funciones de Hermite}

\begin{frame}
\frametitle{Hermite - Motivación}
Continuando con ejemplos de mecánica cuántica, ahora tomaremos el estudio del oscilador armónico cuántico, lo que nos conducirá a los \emph{\textbf{\textcolor{darkterracotta}{polinomios de Hermite}}}.
\\
\bigskip
\pause
Ya hemos revisado previamente el caso del oscilador cuántico, por lo que se completará el tema con el estudio de las propiedades de los polinomios de Hermite.
\end{frame}

\subsection{Funciones de Bessel}

\begin{frame}
\frametitle{Bessel - Motivación y geometría}
La consideración de la ecuación de Laplace en un sistema coordenado cilíndrico polar, nos llevará a establecer las \emph{\textbf{\textcolor{debianred}{funciones de Bessel}}}, conocidas como funciones de Bessel de primera clase, \pause y las funciones de Bessel de segunda clase llamadas \emph{\textbf{\textcolor{darkslateblue}{funciones de Neumann}}}.
\end{frame}

\subsection{Funciones de Chebyshev}

\begin{frame}
\frametitle{Funciones de Chevychev}
Para este par de funciones especiales: Chebyshev de tipo I y Chebyshev de Tipo II, tomaremos como punto de partida la ecuación diferencial correspondiente.
\\
\bigskip
\pause
Se revisarán las propiedades y características de cada una de ellas. Veremos su utilidad en la interpolación numérica.
\end{frame}

\subsection{Funciones hipergeométricas}

\begin{frame}
\frametitle{La función hipergeométrica}
Estudiaremos la ecuación diferencial como punto de partida y encontraremos que habrá casos especiales o límite, en donde recuperamos otras funciones especiales.
\\
\bigskip
Revisaremos la \emph{\textbf{\textcolor{darkslateblue}{función hipergeométrica confluente}}} \pause y la \emph{\textbf{\textcolor{denim}{función hipergeométrica ordinaria}}}.
\end{frame}

\subsection{Funciones de Gegenbauer}

\begin{frame}
\frametitle{Funciones de Gegenbauer}
Las \emph{\textbf{\textcolor{flame}{funciones de Gegenbauer}}} se obtienen a partir de la serie hipergeométrica para los casos en donde ésta, es finita, \pause además, son la solución de la ecuación diferencial de Gegenbauer, una generalización de los polinomios de Legendre.
\end{frame}

\section{Evaluación del Tema}
\frame{\frametitle{Esquema de evaluación} \tableofcontents[currentsection, hideothersubsections]}
\subsection{Ejercicios y enunciados para resolver}

\begin{frame}
\frametitle{Ejercicios a cuenta}
Los ejercicios a cuenta se presentarán en la sesión del martes 17 de mayo durante la clase.
\\
\bigskip
\pause
Debiendo entregarse la solución el día \textbf{Jueves 2 de junio} a más tardar a 6 pm, por la plataforma Moodle.
\end{frame}
\begin{frame}
\frametitle{Entrega de enunciados del Tema 5}
En la misma sesión del martes 17 de mayo, se entregarán los enunciados del tema 5 que forman parte del segundo examen del curso.
\end{frame}
\begin{frame}
\frametitle{Fechas importantes}
Considerando que ya nos encontramos en la recta final del semestre 2022-2, el segundo examen del curso se entregará en dos partes:
\end{frame}
\begin{frame}
\frametitle{Fechas importantes}    
\setbeamercolor{item projected}{bg=darkcoral,fg=white}
\setbeamertemplate{enumerate items}{%
\usebeamercolor[bg]{item projected}%
\raisebox{1.5pt}{\colorbox{bg}{\color{fg}\footnotesize\insertenumlabel}}%
}
\begin{enumerate}[<+->]
\item Los enunciados del Tema 4 y del Tema 5 se entregarán el \textbf{martes 7 de junio}, a más tardar 6 pm por la plataforma Moodle.
\item Los enunciados del Tema 6 se entregarán en las semanas 15 y 16, \pause teniendo solo un fin de semana para resolverlos y enviarlos por Moodle.
\end{enumerate}
\end{frame}

\section{Calendarización}
\frame{\frametitle{Cronograma de trabajo} \tableofcontents[currentsection, hideothersubsections]}
\subsection{Distribución de tiempos}

\begin{frame}
\frametitle{Trabajo por semanas}
Como ya se comentó, ya estamos por completar el semestre, por lo que se presenta la siguiente distribución de tiempos para cubrir los materiales de trabajo y las sesiones presenciales.
\end{frame}
\begin{frame}
\frametitle{Programa de trabajo}
\setbeamercolor{item projected}{bg=darkcoral,fg=white}
\setbeamertemplate{enumerate items}{%
\usebeamercolor[bg]{item projected}%
\raisebox{1.5pt}{\colorbox{bg}{\color{fg}\footnotesize\insertenumlabel}}%
}
\begin{enumerate}[<+->]
\item Semanas 12 y 13: Funciones de Laguerre, Hermite y Bessel.
\item Semanas 13 y 14: Funciones de Chebyshev, hipergeométricas y Gegenbauer.
\end{enumerate}
\end{frame}

\subsection{Sesiones presenciales}

\begin{frame}
\frametitle{Sesiones de trabajo}
Tendremos las clases presenciales los martes y jueves en el horario de 3:30 pm a 6:00 pm
\begin{itemize}[<+->]
\item Jueves 12 de mayo.
\item Martes 17 y jueves 19 de mayo.
\item Martes 24 y jueves 26 de mayo.
\end{itemize}
\end{frame}

\end{document}