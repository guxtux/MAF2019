\documentclass[12pt]{beamer}
\usepackage{../Estilos/BeamerMAF}
\input{../Preambulos/preambulo_Beamer_Copenhagen_wolverine}

\date{diciembre de 2021}

\title{\large{Tema 5 - Funciones Especiales}}
\subtitle{Funciones Especiales Parte II}
\author{M. en C. Gustavo Contreras Mayén}

\begin{document}
\maketitle
\fontsize{14}{14}\selectfont
\spanishdecimal{.}

\section*{Contenido}
\frame[allowframebreaks]{\tableofcontents[currentsection, hideallsubsections]}

\section{Funciones Especiales}
\frame{\tableofcontents[currentsection, hideothersubsections]}
\subsection{Otra vez: ¿para qué las FE?}

\begin{frame}
\frametitle{Pensamiento}
Alberto Grünbaum del Departamento de Matemáticas de la Universidad de Berkeley, es uno de los mayores expertos mundiales en aplicaciones de funciones especiales y polinomios ortogonales.
\end{frame}
\begin{frame}
\frametitle{Pensamiento}
En particular en probabilidad, procesos estocásticos, sistemas integrables, imagenología médica (tomografía computarizada), biomatemáticas y mecánica estadística, entre otros.
\end{frame}
\begin{frame}
\frametitle{Pensamiento}
El Dr. Grünbaum expresó en una ocasión:
\\
\bigskip
\pause
\begin{quote}
\enquote{Las funciones especiales son para las matemáticas lo que las tuberías son para una casa: nadie quiere exhibirlas abiertamente, pero nada funciona sin ellas.}
\end{quote}
\end{frame}

\section{Objetivos}
\frame{\tableofcontents[currentsection, hideothersubsections]}
\subsection{Conocimiento y habilidades}

\begin{frame}
\frametitle{Objetivos}
Al concluir el Tema 5, se espera que el alumno:
\setbeamercolor{item projected}{bg=blue!70!black,fg=yellow}
\setbeamertemplate{enumerate items}[circle]
\begin{enumerate}[<+->]
\item Reconozca que las funciones especiales se obtienen de las soluciones de EDO2 que se recuperan luego de haber planteado un problema físico bajo ciertas condiciones en la geometría asociada, así como en las CDF y condiciones iniciales.                            
\seti
\end{enumerate}
\end{frame}
\begin{frame}
\frametitle{Objetivos}
\setbeamercolor{item projected}{bg=blue!70!black,fg=yellow}
\setbeamertemplate{enumerate items}[circle]
\begin{enumerate}[<+->]
\conti
\item Identifique el conjunto de propiedades para las funciones especiales, en particular:
\begin{itemize}
\item La función generatriz.
\item Las relaciones de recurrencia.
\item La condición de ortogonalidad y normalización.
\item La fórmula de Rodrigues.
\item La condición de paridad.
\end{itemize}
\seti
\end{enumerate}
\end{frame}
\begin{frame}
\frametitle{Objetivos}
\setbeamercolor{item projected}{bg=blue!70!black,fg=yellow}
\setbeamertemplate{enumerate items}[circle]
\begin{enumerate}[<+->]
\conti
\item Utilice el conjunto de propiedades de las funciones especiales para resolver problemas con determinada geometría.
\item Adquiera una metodología de trabajo para estudiar una función especial que no se haya revisado en el curso.
\end{enumerate}
\end{frame}

\section{Contenido}
\frame[allowframebreaks]{\tableofcontents[currentsection, hideothersubsections]}
\subsection{Formalismo y requisitos}


\begin{frame}
\frametitle{Contenido importante}
Esta parte del curso incluye un contenido muy relevante ya que se trata del manejo y formalismo matemático que tendremos que utilizar como físicos, a partir de este sexto semestre y en lo que resta de nuestra vida profesional.
\end{frame}
\begin{frame}
\frametitle{Sobre los temas en los ejemplos}
Nuevamente tendremos que apoyarnos con distintas áreas de la física para presentar problemas que nos conduzcan a una función especial.
\\
\bigskip
\pause
La construcción completa del ejercicio, es decir, la base teórica en cada tema, será algo que daremos que ya manejan en el sexto semestre.
\end{frame}
\begin{frame}
\frametitle{Sobre los temas en los ejemplos}
En caso de que tengan alguna complicación para el planteamiento de un ejercicio en un tema particular, deberán de apoyarse con las referencias en cada tema.
\\
\bigskip
\pause
De esta manera tendrán completo el contexto del ejercicio, de tal manera que la solución del ejercicio y su interpretación será mucho más fácil de realizar.
\end{frame}

\subsection{Funciones de Laguerre}

\begin{frame}
\frametitle{Laguerre - Motivación}
Se tomará como punto de partida la ecuación radial del átomo de hidrógeno para obtener los \emph{polinomios ordinarios de Laguerre} y \emph{los polinomios asociados de Laguerre}.
\end{frame}
\begin{frame}
\frametitle{Laguerre - Descripción matemática}
Así mismo, se revisarán las características y propiedades de los polinomios, buscando conectar su uso con las aplicaciones en problemas de la física matemática.
\end{frame}

\subsection{Funciones de Hermite}

\begin{frame}
\frametitle{Hermite - Motivación}
Continuando con ejemplos de mecánica cuántica, ahora tomaremos el estudio del oscilador armónico cuántico, lo que nos conducirá a los \emph{polinomios de Hermite}.
\\
\bigskip
\pause
Ya hemos revisado previamente el caso del oscilador cuántico, por lo que se completará el tema con el estudio de las propiedades de los polinomios de Hermite.
\end{frame}

\subsection{Funciones de Bessel}

\begin{frame}
\frametitle{Bessel - Motivación y geometría}
La consideración de la ecuación de Laplace en un sistema coordenado cilíndrico polar, nos llevará a establecer las \emph{funciones de Bessel}, conocidas como funciones de Bessel de primera clase, y las funciones de Bessel de segunda clase llamadas \emph{funciones de Neumann}.
\end{frame}
\begin{frame}
\frametitle{Bessel - Geometría esférica}
Como un caso particular del estudio con una geometría esférica, tendremos la oportunidad de revisar las \emph{funciones de Bessel esféricas}.
\\
\bigskip
En el Tema 2, con la construcción de la solución con el método de Frobenius, se presentó el caso de la EDO de Bessel.
\end{frame}

\subsection{Funciones de Chebychev}

\begin{frame}
\frametitle{Funciones de Chevychev}
Para este par de funciones especiales: Chebychev de tipo I y Chebychev de Tipo II, tomaremos como punto de partida la ecuación diferencial correspondiente.
\\
\bigskip
\pause
Se revisarán las propiedades y características de cada una de ellas. Veremos su utilidad en la interpolación numérica.
\end{frame}

\subsection{Funciones hipergeométricas}

\begin{frame}
\frametitle{La función hipergeométrica}
Estudiaremos la ecuación diferencial como punto de partida y encontraremos que habrá casos especiales o límite, en donde recuperamos otras funciones especiales.
\\
\bigskip
Revisaremos la \emph{función hipergeométrica confluente} y la \emph{función hipergeométrica ordinaria}.
\end{frame}

\subsection{Funciones de Gegenbauer}

\begin{frame}
\frametitle{Funciones de Gegenbauer}
Este tipo de funciones que se obtienen a partir de la serie hipergeométrica para los casos en donde ésta, es finita, además, son la solución de la ecuación diferencial de Gegenbauer, una generalización de los polinomios de Legendre.
\end{frame}

\section{Evaluación del Tema}
\frame{\tableofcontents[currentsection, hideothersubsections]}
\subsection{Ejercicios para resolver}

\begin{frame}
\frametitle{Ejercicios a cuenta}
La distribución de ejercicios es la siguiente:
\pause
\renewcommand{\arraystretch}{1.15}
\begin{table}
\centering
\begin{tabular}{l c c}
Material & Semanales & Opcionales \\ \hline
Laguerre & 2 & 1 \\ \hline
Hermite  & 2 & 1 \\ \hline
Bessel & 2 & 1 \\ \hline    
Chebychev & 2 & 1 \\ \hline    
Hipergeométrica & 2 & 1 \\ \hline
Gegenbauer & 2 & 1
\end{tabular}
\end{table}
\end{frame}

\subsection{Examen Tarea}

\begin{frame}
\frametitle{Entrega del examen tarea 5}
En la sesión del 17 de diciembre se entregarán los enunciados del examen tarea 5, \pause para enviar la solución el día 16 de enero de 2022.
\end{frame}

\section{Calendarización}
\frame{\tableofcontents[currentsection, hideothersubsections]}
\subsection{Distribución de tiempos}

\begin{frame}
\frametitle{Trabajo por semanas}
Considerando que ya tenemos en puerta el período vacacional de fin de año, que comienza a partir del 19 de diciembre, se propone el siguiente esquema de trabajo:
\end{frame}
\begin{frame}
\frametitle{Semanas 11 y 12}
\setbeamercolor{item projected}{bg=blue!70!black,fg=yellow}
\setbeamertemplate{enumerate items}[circle]
\begin{enumerate}[<+->]
\item Semana 12: Del 6 al 10 de diciembre, funciones de Laguerre, Hermite y Bessel.
\item Semana 13: Del 13 al 17 de diciembre, funciones de Bessel, Chebychev e hipergeométricas.
\item Semanas 14 y 15: 6 y 7 de enero y del 10 al 12 de enero de 2022. Funciones hipergeométricas y Gegenbauer.
\end{enumerate}
\end{frame}

\subsection{Sesiones síncronas}

\begin{frame}
\frametitle{Sesiones de trabajo}
Se programan sesiones de trabajo síncronas para los días miércoles y viernes:
\begin{itemize}
\item Viernes 10 de diciembre de 2021.
\item Miércoles 15 y viernes 17 de diciembre de 2021.
\item Jueves 6 de enero de 2022.
\item Lunes 10 de enero de 2022.
\end{itemize}
En el acostumbrado horario de las 3 pm.
\end{frame}

\end{document}