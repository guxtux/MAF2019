\documentclass[12pt]{beamer}
\usepackage{../Estilos/BeamerMAF}
\usepackage{../Estilos/ColoresLatex}
\input{../Preambulos/preambulo_Beamer_Warsaw_seahorse}

\title{\large{Evaluación Semanal Tema 4}}
\subtitle{Tema 4 - Funciones Especiales}

\author{M. en C. Gustavo Contreras Mayén}

\date{}

\begin{document}
\maketitle
\fontsize{14}{14}\selectfont
\spanishdecimal{.}

\section*{Contenido}
\frame[allowframebreaks]{\tableofcontents[currentsection, hideallsubsections]}

\section{Enunciados}
\frame{\tableofcontents[currentsection, hideothersubsections]}
\subsection{Funciones de Bessel}

\begin{frame}
\frametitle{Enunciado 1}
%Ref. Hassani
Un cilindro largo conductor de calor de radio $a$ se compone de dos mitades (con secciones transversales semicirculares) con un espacio infinitesimal entre ellas.
\\
\bigskip
\pause
Las mitades superior e inferior del cilindro están en contacto con baños térmicos $+T_{1}$ y $-T_{1}$, respectivamente.
\end{frame}

\begin{frame}
\frametitle{Enunciado 1}
El cilindro está dentro de otro cilindro de radio $b$ más grande ( $a < b$ y coaxial con él) que se mantiene a la temperatura $T_{2}$. 
\\
\bigskip
\pause
Calcula la temperatura en puntos:
\setbeamercolor{item projected}{bg=bananayellow,fg=ao}
\setbeamertemplate{enumerate items}{%
\usebeamercolor[bg]{item projected}%
\raisebox{1.5pt}{\colorbox{bg}{\color{fg}\footnotesize\insertenumlabel}}%
}
\begin{enumerate}[<+->]
\item Dentro del cilindro interno.
\item Entre los dos cilindros.
\item Fuera del cilindro externo.
\end{enumerate}
\end{frame}

\begin{frame}
\frametitle{Punto extra}
Si entregas la solución del ejercicio anterior, este ejercicio contará como ejercicio adicional: puedes sumar 4 puntos y se promedia sobre 3.
\\
\bigskip
\pause
Deducir una expresión para la distribución estacionaria de temperaturas $u(r, z)$ en el cilindro macizo limitado por las superficies $r = 1$, $z = 0$ y $z = 1$, si $u = 0$ en la superficie $r= 1$, $u = 1$ en la superficie $z = 1$ y la base $z = 0$ está aislada.
\end{frame}

\subsection{Funciones de Hermite}

\begin{frame}
\frametitle{Enunciado 2}
Demuestra que las funciones $\psi_{n} (x) = \exp(-x^{2}/2) \, H_{n} (x)$ satisfacen:
\pause
\begin{align*}
\scaleint{6ex}_{\bs -\infty}^{\infty} x^{2} \, &\abs{\psi_{n} (x)}^{2} \dd{x} = \sqrt{\pi} \, 2^{n} \, n! \bigg( n + \dfrac{1}{2} \bigg) \\[0.5em]
 n = 0, &1, 2, \ldots
\end{align*}
\end{frame}

\subsection{Funciones de Chebyshev}

\begin{frame}
\frametitle{Enunciado 3}
Los polinomios de Chebyshev están definidos en el intervalo $[- 1, 1]$, siendo posible definirlos en cualquier rango finito $[a, b]$ para la variable $x$, haciendo que éste rango corresponda al rango $[-1, 1]$ con una nueva variable $s$, se ocupa la siguiente transformación lineal:
\pause
\begin{align*}
s = \dfrac{2 \, x - (a + b)}{(b - a)}
\end{align*}
\end{frame}
\begin{frame}
\frametitle{Enunciado 3}    
Por lo que los polinomios de Chebyshev de primer tipo ajustados al intervalo $[a, b]$ son $T_{n}(s)$, de manera similar se hace el ajuste para los polinomios de segundo tipo  $U_{n} (s)$. 
\\
\bigskip
\pause
Desarrolla la expresión para los polinomios de Chebyshev de segundo tipo $U_{n} (s)$ de grado $n = 0, 1, 2, 3, 4$ en el rango $[-5, 5]$ para $x$.
\end{frame}
\end{document}