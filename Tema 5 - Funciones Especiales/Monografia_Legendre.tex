\documentclass[12pt]{article}
\usepackage[left=0.25cm,top=1cm,right=0.25cm,bottom=1cm]{geometry}
\textwidth = 20cm
\hoffset = -1cm
\usepackage[utf8]{inputenc}
\usepackage[spanish,es-tabla]{babel}
\usepackage[autostyle,spanish=mexican]{csquotes}
\usepackage[tbtags]{amsmath}
\usepackage{nccmath}
\usepackage{amsthm}
\usepackage{amssymb}
\usepackage{graphicx}
\usepackage{standalone}
\usepackage[outdir=./]{epstopdf}
\usepackage{siunitx}
\usepackage{physics}
\usepackage{color}
\usepackage{float}
\usepackage{multicol}
%\usepackage{milista}
\usepackage{enumitem}
\usepackage{anyfontsize}
\usepackage{anysize}
\usepackage{enumitem}
\usepackage{capt-of}
\usepackage{bm}
\usepackage{relsize}
\usepackage{placeins}
\usepackage{empheq}
\usepackage{cancel}
\usepackage{wrapfig}
\spanishdecimal{.}
\renewcommand{\baselinestretch}{1.5} 
\renewcommand\labelenumii{\theenumi.{\arabic{enumii}}}
\newcommand{\ptilde}[1]{\ensuremath{{#1}^{\prime}}}
\newcommand{\stilde}[1]{\ensuremath{{#1}^{\prime \prime}}}
\newcommand{\ttilde}[1]{\ensuremath{{#1}^{\prime \prime \prime}}}
\newcommand{\ntilde}[2]{\ensuremath{{#1}^{(#2)}}}


%\usepackage{showframe}
\title{Monografía funciones Lagrange \\ \large {Matemáticas Avanzadas de la Física} \vspace{-3ex}}
\author{M. en C. Gustavo Contreras Mayén}
\date{ }
\begin{document}
\maketitle
\fontsize{14}{14}\selectfont
\renewcommand\arraystretch{2}
\vspace*{-3cm}
\begin{table}[H]
    \centering
\begin{tabular}{| p{5cm} | p{12cm} |} \hline
\multicolumn{2}{|c|}{\textbf{Polinomios ordinarios de Legendre}} \\ \hline
Problema(s) de la Física & \makecell[l]{Ecuación radial del átomo de hidrógeno \\ Ecuación de Laplace \\ Desarrollo multipolar eléctrico} \\ \hline
Geometría & Sistema coordenado esférico \\ \hline
Ecuación diferencial & \(\displaystyle
(1 - x^{2}) \stilde{y} - 2 \, x \, \ptilde{y} + \ell (\ell + 1) \, y = 0
\) \\ \hline
Soluciones & \makecell[l]{\( y_{1}(x) = 1 - \ell (\ell + 1) \dfrac{x^{2}}{2!} + (\ell - 2)\; \ell \; (\ell + 1)\;(\ell + 3) \dfrac{x^{4}}{4!} - \ldots \) \\ \( y_{2}(x) = x - (\ell - 1)(\ell + 2) \dfrac{x^{3}}{3!} + (\ell - 3) (\ell - 1)(\ell + 2)(\ell + 4) \dfrac{x^{5}}{5!} - \ldots \)}\\ \hline
Solución general & \( \displaystyle y(x) = c_{1} \, P_{\ell}(x) + c_{2} \, Q_{\ell} (x)  \)\\ \hline
Función generatriz & \(\displaystyle G(x ,h) = (1 - 2 \, x \, h + h^{2})^{-1/2} =  \sum_{n=0}^{\infty} P_{n}(x) \, h^{n} \) \\ \hline
Ortogonalidad & \makecell[l]{ \( \displaystyle \int_{-1}^{1} P_{\ell}(x) P_{k}(x) \dd{x} = 0 \hspace{1cm} \mbox{ si $\ell \neq k$} \) \\
\( \displaystyle \int_{-1}^{1} P_{\ell}(x) P_{k}(x) \dd{x} = \dfrac{2}{2 \, \ell + 1} \hspace{1cm} \mbox{ si $\ell = k$} \) } \\ \hline
Relaciones de recurrencia & \makecell[l]{ \( \displaystyle \ptilde{P}_{n+1} + \ptilde{P}_{n-1} =  P_{n} + 2 \; x \; \ptilde{P}_{n} \) \\ \( \displaystyle \ptilde{P}_{n+1} = (n+1) \; P_{n} + x \; \ptilde{P}_{n}\) \\
\( \displaystyle \ptilde{P}_{n-1} = -n \; P_{n} + x \; \ptilde{P}_{n}\) \\
\( \displaystyle (1 - x^{2}) \, \ptilde{P}_{n+1} = n \; (P_{n-1} - x \; P_{n})\) \\
\( \displaystyle (2 \, n + 1) \, P_{n} = \ptilde{P}_{n+1} - \ptilde{P}_{n-1} \)} \\ \hline
Paridad & \( \displaystyle P_{\ell} (-u) = (-1)^{\ell} \, P_{\ell} (u) \) \\ \hline
Expresión integral & \\ \hline
\end{tabular}
\end{table}
\end{document}