\documentclass[hidelinks,12pt]{article}
\usepackage[left=0.25cm,top=1cm,right=0.25cm,bottom=1cm]{geometry}
%\usepackage[landscape]{geometry}
\textwidth = 20cm
\hoffset = -1cm
\usepackage[utf8]{inputenc}
\usepackage[spanish,es-tabla]{babel}
\usepackage[autostyle,spanish=mexican]{csquotes}
\usepackage[tbtags]{amsmath}
\usepackage{nccmath}
\usepackage{amsthm}
\usepackage{amssymb}
\usepackage{mathrsfs}
\usepackage{graphicx}
\usepackage{subfig}
\usepackage{standalone}
\usepackage[outdir=./Imagenes/]{epstopdf}
\usepackage{siunitx}
\usepackage{physics}
\usepackage{color}
\usepackage{float}
\usepackage{hyperref}
\usepackage{multicol}
%\usepackage{milista}
\usepackage{anyfontsize}
\usepackage{anysize}
%\usepackage{enumerate}
\usepackage[shortlabels]{enumitem}
\usepackage{capt-of}
\usepackage{bm}
\usepackage{relsize}
\usepackage{placeins}
\usepackage{empheq}
\usepackage{cancel}
\usepackage{wrapfig}
\usepackage[flushleft]{threeparttable}
\usepackage{makecell}
\usepackage{fancyhdr}
\usepackage{tikz}
\usepackage{bigints}
\usepackage{scalerel}
\usepackage{pgfplots}
\usepackage{pdflscape}
\pgfplotsset{compat=1.16}
\spanishdecimal{.}
\renewcommand{\baselinestretch}{1.5} 
\renewcommand\labelenumii{\theenumi.{\arabic{enumii}})}
\newcommand{\ptilde}[1]{\ensuremath{{#1}^{\prime}}}
\newcommand{\stilde}[1]{\ensuremath{{#1}^{\prime \prime}}}
\newcommand{\ttilde}[1]{\ensuremath{{#1}^{\prime \prime \prime}}}
\newcommand{\ntilde}[2]{\ensuremath{{#1}^{(#2)}}}

\newtheorem{defi}{{\it Definición}}[section]
\newtheorem{teo}{{\it Teorema}}[section]
\newtheorem{ejemplo}{{\it Ejemplo}}[section]
\newtheorem{propiedad}{{\it Propiedad}}[section]
\newtheorem{lema}{{\it Lema}}[section]
\newtheorem{cor}{Corolario}
\newtheorem{ejer}{Ejercicio}[section]

\newlist{milista}{enumerate}{2}
\setlist[milista,1]{label=\arabic*)}
\setlist[milista,2]{label=\arabic{milistai}.\arabic*)}
\newlength{\depthofsumsign}
\setlength{\depthofsumsign}{\depthof{$\sum$}}
\newcommand{\nsum}[1][1.4]{% only for \displaystyle
    \mathop{%
        \raisebox
            {-#1\depthofsumsign+1\depthofsumsign}
            {\scalebox
                {#1}
                {$\displaystyle\sum$}%
            }
    }
}
\def\scaleint#1{\vcenter{\hbox{\scaleto[3ex]{\displaystyle\int}{#1}}}}
\def\bs{\mkern-12mu}


%\usepackage{showframe}
\title{Funciones hipergeométricas generalizadas \\ \large {Tema 5 - Funciones especiales} \vspace{-3ex}}
\author{M. en C. Gustavo Contreras Mayén}
\date{ }
\begin{document}
\vspace{-4cm}
\maketitle
\fontsize{14}{14}\selectfont
\tableofcontents
\newpage


%Ref. Slater (1996) - Generalized hypergeometric functions
\section{La función de Gauss.}

La serie:
\begin{align}
1 + \dfrac{a b}{c} \dfrac{z}{1!} + \dfrac{a (a {+} 1) b (b {+} 1)}{c( c {+} 1)} \dfrac{z^{2}}{2!} + \dfrac{a (a {+} 1)(a {+} 2) b (b {+} 1)(b {+} 2)}{c (c {+} 1)(c {+} 2)} \dfrac{z^{3}}{3!} + \ldots
\label{eq:ecuacion_01_01_01}
\end{align}
es llamada \emph{serie de Gauss} o \emph{serie hipergeométrica ordinaria}. Normalmente se representa con la notación:
\begin{align*}
{}_{2} F_{1} \big[ a, b; c; z \big]
\end{align*}
La variable es $z$, y $a$, $b$ y $c$ son los parámetros  de la función. Si alguna de las cantidades $a$ o $b$ es un entero negativo $-n$, la serie tiene solo un número finito de términos y se convierte de hecho en un polinomio:
\begin{align*}
{}_{2} F_{1} \big[ -n, b; c; z \big]
\end{align*}

Por ejemplo, si $a = -2$, entonces la serie es:
\begin{align*}
{}_{2} F_{1} \big[ -2, b: c; z \big] &= 1 + \dfrac{(-2) b}{c} \dfrac{z}{1!} + \dfrac{(-2) (-1) b (b + 1)}{c( c + 1)} \dfrac{z^{2}}{2!} + 0 
\end{align*}    
esto es:
\begin{align}
{}_{2} F_{1} \big[ -2, b; c; z \big] = 1 - \dfrac{2 b z}{c} + \dfrac{b( b + 1) z^{2}}{c (c + 1)}
\label{eq:ecuacion_01_01_02}
\end{align}
ya que los restantes términos se anulan.
\par
En su trabajo \emph{Arithmetica Infinitorum} (1655), el profesor de Oxford John Wallis (1616 - 1703) fue el primero en utilizar el término \enquote{hipergeométrica} (del griego $\acute{\nu} \pi \epsilon \rho$, por arriba o más allá de) para denotar cualquier serie que estuviera más allá de las serie geométrica ordinaria:
\begin{align*}
1 + x + x^{2} + x^{3} + \ldots
\end{align*}
En particular, él estudió las series:
\begin{align*}
1 + a + a( a + 1) + a (a + 1)(a + 2) + \ldots
\end{align*}

Durante los siguientes ciento cincuenta años, muchos otros
matemáticos estudiaron series similares, notablemente el suizo Leonard Euler (1783) dio, entre muchos otros resultados, la famosa relación:
\begin{align}
{}_{2} F_{1} \big[ -n, b; c; z \big] = (1 - z)^{c+n-b} \, {}_{2} F_{1} \big[ c + n, c - b; c; z \big]
\label{eq:ecuacion_01_01_03} 
\end{align}

En 1770, el francés A. T. Vandermonde (1735-1796) estableció su teorema, una extensión del teorema del binomio, en la forma:
\begin{align}
{}_{2} F_{1} \big[ -n, b; c; 1 \big] = \dfrac{(c {-} b)(c {-} b {+} 1)(c {-} b {+} 2) \cdots (c {-} b {+} n {-} 1)}{c (c {+} 1)(c {+} 2)(c {+} 3) \cdots (c {+} n {-} 1)}
\label{eq:ecuacion_01_01_04}
\end{align}
pero durante los siguientes cuarenta años, la escuela de Göttingen bajo C. F. Hindenberg (1741-1808) desperdició mucho esfuerzo en varias extensiones complicadas de los teoremas binomial y multinomial. Todo esto cambió drásticamente cuando el 20 de enero de 1812 C. F. Gauss (1777-1855) pronunció su famosa tesis \enquote{Disquisitiones generales circa seriem infinitam} ante la Real Sociedad de Göttingen. En el, este brillante matemático definió la serie infinita moderna de la ec. (\ref{eq:ecuacion_01_01_01}) e introdujo la notación $F \big[a, b; c; z \big]$ para ello. También demostró su famoso teorema de suma:
\begin{align}
{}_{2} F_{1} \big[ a, b; c; 1 \big] = \dfrac{\Gamma (c) \Gamma (c - a - b)}{\Gamma(c - a) \Gamma (c - b)}
\label{eq:ecuacion_01_01_05}
\end{align}
y dio muchas relaciones entre dos o más de estas series. Él
mostró claramente que ya estaba considerando ${}_{2} F_{1} \big[ a, b; c; z \big]$ como una función de cuatro variables, en lugar de una serie en $z$, y en una nota agregada el 10 de febrero de 1812, dio una discusión notablemente completa de la convergencia de tales series.
\par
El siguiente gran avance fue realizado en 1836 por E. E. Kummer (1810 - 1893), quien utilizó por primera vez el término \enquote{hipergeométrico} únicamente para series del tipo (\ref{eq:ecuacion_01_01_01}). Mostró que la ecuación diferencial:
\begin{align}
z (1 - z) \dv[2]{y}{z} + \big[ c - \big( 1 + a + b \big) z \big] \dv{y}{z} - a b y = 0
\label{eq:ecuacion_01_01_06}
\end{align}
se satisface por la función:
\begin{align*}
{}_{2} F_{1} \big[ a, b; c; z \big]
\end{align*}
y tiene en total veinticuatro soluciones en términos de funciones de Gauss similares. En 1857, G. F. B. Riemann (1826 - 1866) amplió esta teoría mediante la introducción de sus funciones $P$, que en cierto modo, son generalizaciones de la Gaussiana:
\begin{align*}
{}_{2} F_{1} \big[ a, b; c; z \big]
\end{align*}    

Riemann también discutió la teoría general de la transformación de la variable en una ecuación diferencial y esta teoría fue aplicada al trabajo de Kummer por J. Thomae quien, en 1879, elaboró en detalle las relaciones entre las veinticuatro soluciones de Kummer.
\par
La primera representación integral de la función de Gauss se remonta a Euler, quien demostró que:
\begin{align}
\begin{aligned}[b]
{}_{2} F_{1} \big[ -n, b; c; z \big] &= \dfrac{n!}{c (c {+} 1)(c {+} 2) \cdots (c {+} {-} 1)} \times \\[0.5em]
&\times \scaleint{6ex}_{\bs 0}^{1} t^{-n-1} \, (1 {-} t)^{c+n-1} \, (1 - t z)^{-b} \dd{t}
\end{aligned}
\label{eq:ecuacion_01_01_07}
\end{align}

La idea básica de representar una función mediante una integral de contorno con funciones Gamma en el integrando parece deberse a S. Pincherle (1853 - 1936), quien utilizó contornos de un tipo que proviene del trabajo de Riemann. Este lado del tema fue desarrollado extensamente por R. Mellin y E. W. Barnes. En 1907, Barnes publicó sus representaciones integrales de contorno de las veinticuatro funciones de Kummer, y más tarde, en 1910, demostró el análogo integral del teorema de Gauss.
\begin{align}
\begin{aligned}[b]
\dfrac{1}{2 \pi i} \scaleint{6ex}_{\bs -\infty}^{\infty} \, &\Gamma (a {+} s) \, \Gamma (b {+} s) \, \Gamma(c {-} s) \, \Gamma (d {-} s) \dd{s} = \\[0.5em]
&= \dfrac{\Gamma (a {+} c) \, \Gamma (a {+} d) \, \Gamma(b {+} c) \, \Gamma (b {+} d)}{\Gamma (a {+} b {+} c) {+ d}}
\end{aligned}
\label{eq:ecuacion_01_01_08}
\end{align}

\newpage
\section{Las series de Gauss y su convergencia.}

Escribamos lo siguiente:
\begin{align}
\big( a \big)_{n} = a ( a + 1)(a + 2)(a + 3) \cdots (a + n - 1)
\label{eq:ecuacion_01_01_01_01}
\end{align}
y en particular $\big( a \big)_{0} \equiv 1$, por ejemplo:
\begin{align*}
\big( 3 \big)_{5} = 3 \cdot 4 \cdot 5 \cdot 6 \cdot 7 = 2520
\end{align*}
y también se tiene que $\big( 1 \big) = n!$. Entonces:
\begin{align}
\big( a \big)_{n} = \dfrac{\Gamma (a + n)}{\Gamma (a)}
\label{eq:ecuacion_01_01_01_02}
\end{align}
y
\begin{align}
\lim_{n \to \infty} \big( a \big)_{n} = \dfrac{1}{\Gamma (a)}
\label{eq:ecuacion_01_01_01_03}
\end{align}

Si $a$ es un entero negativo $-m$, entonces:
\begin{align*}
\big( a \big)_{n} = \begin{cases}
\big( -m \big)_{n} & \mbox{si } m \geq n \\
0 & \mbox{si } m < n
\end{cases}
\end{align*}
por lo que:
\begin{align*}
\big( -3 \big)_{3} = (-3)(-2)(-1) = -6 \hspace{1cm} \mbox{pero} \hspace{1cm} \big( -3 \big)_{4} = 0
\end{align*}

En esta notación, la función de Gauss es:
\begin{align}
{}_{2} &F_{1} \big[ a, b; c; z \big] = \nsum_{n=1}^{\infty} \dfrac{\big( a \big)_{n} \big( b \big)_{n} \, z^{n}}{\big( c \big)_{n} \, n!}
\label{eq:ecuacion_01_01_01_04}
\end{align}
donde $a$, $b$, $c$ y $z$ pueden ser reales o complejos. A partir de esto, vemos que si cualquiera de los números $a$ o $b$ es cero o un número entero negativo, la función se reduce a un polinomio, pero si $c$ es cero o un número entero negativo, la función no está definida, ya que todos menos un número finito de los términos de la serie se vuelven infinitos. También tenemos inmediatamente que:
\begin{align}
\dv{z} \bigg( {}_{2} F_{1} \big[ a, b; c; z \big] \bigg) = \dfrac{a b}{c} \, {}_{2} F_{1} \big[ a + 1, b + 1; c + 1; z \big]
\label{eq:ecuacion_01_01_01_05}
\end{align}

La función de Gauss se puede encontrar en distintas notaciones, por ejemplo:
\begin{enumerate}[label=\alph*)]
\item Appell (1926):
\begin{align}
{}_{2} F_{1} \bigg[
\begin{tabular}{c c}
a, b; & \multirow{2}{*}{z} \\
c; &
\end{tabular} \bigg] = {}_{2} F_{1} \big[ a, b; c; z \big]
\label{eq:ecuacion_01_01_01_06}
\end{align}
\item Bailey (1935):
\begin{align}
F \big( a, b; c; z \big) = {}_{2} F_{1} \big[ a, b; c; z \big]
\label{eq:ecuacion_01_01_01_07}
\end{align}
\item Meijer (1953):
\begin{align}
\Phi \big( a, b; c; z \big) = \dfrac{{}_{2} F_{1} \big[ a, b; c; z \big]}{\Gamma (c)}
\label{eq:ecuacion_01_01_01_08}
\end{align}
\item MacRobert (1947):
\begin{align}
E \big( 2; a, b; 1; c; -\dfrac{1}{z} \big) = \dfrac{\Gamma (a) \Gamma (b)}{\Gamma (c)} \, {}_{2} F_{1} \big[ a, b; c; z \big]
\label{eq:ecuacion_01_01_01_9}
\end{align}
\end{enumerate}

Hagamos que:
\begin{align*}
u_{n} = \dfrac{\big( a \big)_{n} \big( b \big)_{n}}{\big( c \big)_{n} \big( 1 \big)_{n}}
\end{align*}
entonces se tiene que:
\begin{align}
(1 + n)(c + n) \, u_{n+1} = (a + n)(b +  n) \, u_{n}
\label{eq:ecuacion_01_01_01_12}
\end{align}
la razón de dos términos sucesivos $u_{n}$ y $u_{n+1}$ de la serie Gaussiana es:
\begin{align}
\dfrac{(a + n)(b +  n)}{(c + n)(1 +  n)} \, z = \dfrac{(1 + a/n)(1 + b/n)}{(1 + c/n)(1 + 1/n)} \, z
\label{eq:ecuacion_01_01_01_13}
\end{align}
tal que, cuando $n \to \infty$, la razón:
\begin{align*}
\abs{\dfrac{u_{n+1}}{u_{n}}} \to \abs{z}
\end{align*}
Por tanto, según la prueba de D'Alembert, la serie es convergente para todos los valores de $z$, reales o complejos tales que $\abs{z} < 1$, y divergente para todos los valores de $z$ reales o complejos, tales que $\abs{z} > 1$.
\par
Cuando $\abs{z} = 1$:
\begin{align}
\begin{aligned}[b]
\abs{\dfrac{u_{n+1}}{u_{n}}} &= \abs{\left\{ 1 + \dfrac{a + b}{n} + \order{\dfrac{1}{n^{2}}} \right\} \left\{ 1 - \dfrac{1 + c}{n} + \order{\dfrac{1}{n^{2}}} \right\}} = \\[0.5em]
&= \abs{1 + \dfrac{a + b - c - 1}{n} + \order{\dfrac{1}{n^{2}}}} \\[0.5em]
&\leq 1 + \left\{ \Re (a + b - c - 1) / n \right\} + \order{\dfrac{1}{n^{2}}}
\end{aligned}
\label{eq:eq:ecuacion_01_01_01_14}
\end{align}
Entonces, cuando $z = 1$, por la prueba de Raabe, la serie es convergente si $\Re (c - a - b) > 0$, y divergente si $\Re (c - a - b) < 0$.
\par
También es divergente cuando $\Re (c - a - b) = 0$, en este caso:
\begin{align*}
\abs{\dfrac{u_{n+1}}{u_{n}}} > 1 - \dfrac{1}{n} - \dfrac{C}{n^{2}}
\end{align*}
donde $C$ es una constante.
\par
Cuando $\abs{z} = 1$, pero $z \neq 1$, la serie es absolutamente convergente cuando $\Re (c - a - b) > 0$, convergente pero no de manera absoluta solo cuando:
\begin{align*}
- 1 < \Re (c - a - b) \leq 0
\end{align*}
y divergente cuando $\Re (c - a - b) < -1$. Si $\Re (c - a - b) = -1$, se requieren pruebas de convergencia más delicadas. En este caso, se tiene que:
\begin{align}
\abs{\dfrac{u_{n+1}}{u_{n}}} = 1 - \dfrac{\Re (a + b - a b + 1)}{n^{2}} + \order{\dfrac{1}{n^{3}}}
\label{eq:ecuacion_01_01_01_15}
\end{align}
De aquí la serie es convergente si $\Re (a + b) > \Re \, a \, b$, y divergente si \break \hfill $\Re (a + b) \leq \Re \, a \, b$.
\par
Por ejemplo, la serie:
\begin{align}
1 - \dfrac{2}{3} + \dfrac{3}{4} - \dfrac{4}{5} + \dfrac{5}{6} - \dfrac{6}{7} + \cdots = \dfrac{1}{2} \left\{ 1 + \, {}_{2} F_{1} \big[ 2, 2; 3; -1 \big] \right\}
\label{eq:ecuacion_01_01_01_16}
\end{align}
es divergente.
\par
Notemos que:
\begin{align*}
\dfrac{\big( a \big)_{n} \big( b \big)_{n}}{\big( c \big)_{n}} \to 0 \hspace{1cm} \mbox{cuando} \hspace{0.2cm} n \to \infty, \hspace{0.2cm} \mbox{si } 0 < \Re (1 + c - a - b) < 1
\label{eq:ecuacion_01_01_01_17}
\end{align*}
\end{document}