\documentclass[hidelinks,12pt]{article}
\usepackage[left=0.25cm,top=1cm,right=0.25cm,bottom=1cm]{geometry}
%\usepackage[landscape]{geometry}
\textwidth = 20cm
\hoffset = -1cm
\usepackage[utf8]{inputenc}
\usepackage[spanish,es-tabla]{babel}
\usepackage[autostyle,spanish=mexican]{csquotes}
\usepackage[tbtags]{amsmath}
\usepackage{nccmath}
\usepackage{amsthm}
\usepackage{amssymb}
\usepackage{mathrsfs}
\usepackage{graphicx}
\usepackage{subfig}
\usepackage{standalone}
\usepackage[outdir=./Imagenes/]{epstopdf}
\usepackage{siunitx}
\usepackage{physics}
\usepackage{color}
\usepackage{float}
\usepackage{hyperref}
\usepackage{multicol}
%\usepackage{milista}
\usepackage{anyfontsize}
\usepackage{anysize}
%\usepackage{enumerate}
\usepackage[shortlabels]{enumitem}
\usepackage{capt-of}
\usepackage{bm}
\usepackage{relsize}
\usepackage{placeins}
\usepackage{empheq}
\usepackage{cancel}
\usepackage{wrapfig}
\usepackage[flushleft]{threeparttable}
\usepackage{makecell}
\usepackage{fancyhdr}
\usepackage{tikz}
\usepackage{bigints}
\usepackage{scalerel}
\usepackage{pgfplots}
\usepackage{pdflscape}
\pgfplotsset{compat=1.16}
\spanishdecimal{.}
\renewcommand{\baselinestretch}{1.5} 
\renewcommand\labelenumii{\theenumi.{\arabic{enumii}})}
\newcommand{\ptilde}[1]{\ensuremath{{#1}^{\prime}}}
\newcommand{\stilde}[1]{\ensuremath{{#1}^{\prime \prime}}}
\newcommand{\ttilde}[1]{\ensuremath{{#1}^{\prime \prime \prime}}}
\newcommand{\ntilde}[2]{\ensuremath{{#1}^{(#2)}}}

\newtheorem{defi}{{\it Definición}}[section]
\newtheorem{teo}{{\it Teorema}}[section]
\newtheorem{ejemplo}{{\it Ejemplo}}[section]
\newtheorem{propiedad}{{\it Propiedad}}[section]
\newtheorem{lema}{{\it Lema}}[section]
\newtheorem{cor}{Corolario}
\newtheorem{ejer}{Ejercicio}[section]

\newlist{milista}{enumerate}{2}
\setlist[milista,1]{label=\arabic*)}
\setlist[milista,2]{label=\arabic{milistai}.\arabic*)}
\newlength{\depthofsumsign}
\setlength{\depthofsumsign}{\depthof{$\sum$}}
\newcommand{\nsum}[1][1.4]{% only for \displaystyle
    \mathop{%
        \raisebox
            {-#1\depthofsumsign+1\depthofsumsign}
            {\scalebox
                {#1}
                {$\displaystyle\sum$}%
            }
    }
}
\def\scaleint#1{\vcenter{\hbox{\scaleto[3ex]{\displaystyle\int}{#1}}}}
\def\bs{\mkern-12mu}


%\usepackage{showframe}
\usepackage{apacite}
\title{La ecuación de Jacobi y de Gengenbauer \\ \large {Tema 5 - Funciones especiales} \vspace{-3ex}}
\author{M. en C. Gustavo Contreras Mayén}
\date{ }
\begin{document}
\vspace{-4cm}
\maketitle
\fontsize{14}{14}\selectfont
\tableofcontents
\newpage
\section{Ecuación de Jacobi.}

Cuando se estudian las funciones hipergeométricas notamos que si $a$ o $b$ es un entero negativo por ejemplo $-n$ (con $n$ natural) la serie queda truncada y resulta un polinomio de grado $n$.
\par
Consideremos entonces la ecuación hipergeométrica con las siguientes particularidades:
\begin{align*}
a &= -n \\[0.5em]
b &= n + \alpha + \beta + 1 \\[0.5em]
c &= \alpha + 1
\end{align*}
La ecuación queda modificada es:
\begin{align*}
z (1 - z) \, \stilde{y} (z) + \big[ 1 + \alpha - (\alpha + \beta + 2) \, z \big] \, \ptilde{y} (z) + n(n + \alpha + \beta + 1) \, y(z) = 0
\end{align*}
Al hacer el cambio de variable $z = (1 - x)/2$, se obtiene la llamada \emph{ecuación de Jacobi}:
\begin{align*}
(1 {-} x^{2}) \, \dv[2]{x} y(x) + \big[ \beta {-} \alpha {-} (\alpha {+} \beta {+} 2) \, x \big] \, \dv{x} y(x) + n(n {+} \alpha {+} \beta {+} 1) \, y(x) = 0
\end{align*}
Para cada $n \in \mathbb{N}$ y para cada $\alpha$ y $\beta$, la solución de esta ecuación que satisfaga la condición:
\begin{align*}
P_{n}^{\alpha \beta} \, (1) = \dfrac{\Gamma (\alpha + n + 1)}{\Gamma (\alpha + 1) \, n!}
\end{align*}
son los denominados \emph{polinomios de Jacobi}. Con esta condición entre los polinomios de Jacobi y la función hipergeométrica será:
\begin{align*}
P_{n}^{\alpha \beta} \, (x) = \dfrac{\Gamma (\alpha + n + 1)}{\Gamma (\alpha + 1) \, n!} \, F \left( -n, n + \alpha + \beta + 1, \alpha + 1; \dfrac{1 - x}{2} \right)
\end{align*}

\subsection{Relación de recurrencia.}

Dada la relación entre los polinomios de Jacobi y la función hipergeométrica, se pueden aplicar las relaciones contiguas para establecer las relaciones entre polinomios de Jacobi de distintos grados. A partir de las relaciones contiguas:
\begin{align*}
(c - a) \, F(a - 1, b, c; z) &+ \big[ 2 \, a - c - (a - b) \, z \big] \, F(a, b, c; z) + \\[0.5em]
&+ a(z - 1) \, F( a + 1, b, c; z) = 0
\end{align*}
y con la relación:
\begin{align*}
(c - b) \, F(a, b - 1, c; z) &+ \big[ 2 \, b - c - (b - a) \, z \big] \, F(a, b, c; z) + \\[0.5em]
&+ b(z - 1) \, F( a, b + 1, c; z) = 0
\end{align*}
y teniendo en cuenta que:
\begin{align*}
P_{n+1}^{\alpha \beta} \, (x) = \dfrac{\Gamma (\alpha + n + 2)}{\Gamma (\alpha + 1) \, (n + 1)!} \, F \left( -n + 1, n + \alpha + \beta + 2, \alpha + 1; \dfrac{1 - x}{2} \right)
\end{align*}
además de:
\begin{align*}
P_{n-1}^{\alpha \beta} \, (x) = \dfrac{\Gamma (\alpha + n)}{\Gamma (\alpha + 1) \, (n + 1)!} \, F \left( -n + 1, n + \alpha + \beta + 2, \alpha + 1; \dfrac{1 - x}{2} \right)
\end{align*}
podemos establecer la relación entre los polinomios:
\begin{align*}
&(\alpha + \beta + 2 \, n + 1) \big[ \alpha^{2} - \beta^{2} + (\alpha + \beta + 2 \, n)(\alpha + \beta + 2 \, n + 2) \big] \, P_{n}^{\alpha \beta} \, (x) = \\[0.5em]
&= 2 (n + 1)(\alpha + \beta + n + 1)(\alpha + \beta +  2 \, n) \, P_{n+1}^{\alpha \beta} \, (x) + \\[0.5em]
&+ 2 (\alpha + n)(\beta + n)(\alpha + \beta +  2 \, n + 2) \, P_{n-1}^{\alpha \beta} \, (x)
\end{align*}
Existen otras relaciones de recurrencia que vinculan diferentes valores de $\alpha$ y $\beta$, las cuales se pueden obtener a partir de las relaciones contiguas.

\subsection{Función generatriz.}

Los polinomios de Jacobi admiten como función generatriz:
\begin{align*}
g(x, t) = \dfrac{2^{\alpha+\beta}}{\vb{R}} \, \big[ 1 - t + \vb{R} \big]^{-\alpha} \, \big[ 1 + t + \vb{R} \big]^{-\beta}
\end{align*}
donde
\begin{align*}
\vb{R} = \sqrt{1 - 2 \, x \, t + t^{2}}
\end{align*}
de manera tal que al desarrollar $g(x, t)$ en potencias de $t$:
\begin{align*}
g(x, t) = \sum_{\ell=0}^{\infty} P_{\ell}^{\alpha \beta} \, (x) \, t^{\ell}
\end{align*}
siendo un desarrollo válido para $\abs{t} < 1$.

\subsection{Fórmula de Rodrigues.}

Además de la función generatriz, los polinomios de Jaconi pueden calcularse directamente a partir de la fórmula de Rodrigues, la cual viene dada por la expresión:
\begin{align*}
P_{n}^{\alpha \beta} \, (x) = \dfrac{(-1)^{n}}{2^{n} \, n!} (1 - x)^{-\alpha} \, (1 + x)^{-\beta} \, \dv[n]{x} \big[ (1 - x)^{\alpha+n} \, (1 + x)^{\beta+n} \big]
\end{align*}
Como ejemplo calculemos el polinomio $P_{1}^{\alpha \beta} \, (x)$, como:
\begin{align*}
P_{1}^{\alpha \beta} \, (x) = - \dfrac{1}{2} (1 - x)^{- \alpha} \, (1 + x)^{-\beta} \, \dv{x} \big[ (1 - x)^{\alpha+1} \, (1 + x)^{\beta+1} \big]
\end{align*}
entonces:
\begin{align*}
P_{1}^{\alpha \beta} \, (x) &= \dfrac{1}{2} (\alpha + 1)(1 + x) - (\beta + 1) \, \dfrac{1}{2} (1 - x) = \\[0.5em]
&= \dfrac{\alpha - \beta}{2} + \dfrac{\alpha + \beta + 2}{2} \, x
\end{align*}
En particular, si $\alpha$ y $\beta$ son nulos, se tiene que:
\begin{align*}
P_{1}^{0 \, 0} \, (x) =  x
\end{align*}

\section{Ecuación de Gegenbauer.}

Si la ecuación de Jacobi, consideramos los valores para $\alpha$ y $\beta$:
\begin{align*}
\alpha = \beta = \lambda - \dfrac{1}{2}
\end{align*}
entonces se obtiene la denominada \emph{ecuación de Gegenbauer}:
\begin{align*}
(1 - x^{2}) \, \dv[2]{x} \omega (x) - (2 \, \lambda + 1) \, x \, \dv{x} \omega (x) +  n(n + 2 \, \lambda) \, \omega (x) = 0
\end{align*}
Cuya solución normalizada:
\begin{align*}
C_{n} (1) = \dfrac{\Gamma (2 \, \lambda + n)}{\Gamma (2 \, \lambda) \, n!}
\end{align*}
son los denominados \emph{polinomios de Gegenbauer} o también conocidos como \emph{funciones ultraesféricas}:
\begin{align*}
C_{n}^{\lambda} = \dfrac{\Gamma (2 \, \lambda + n)}{\Gamma (2 \, \lambda) \, n!} \, F \left( -n, n + 2 \, \lambda, \lambda + \dfrac{1}{2}; \dfrac{1 - x}{2} \right)
\end{align*}
Claramente, al ser los polinomios de Gegenbauer propiamente unos polinomios de Jacobi, éstos satisfacerán las relaciones de recurrencia, la fórmula de Rodrigues y generatriz, solo es necesario determinar los valores de $\alpha$ y $\beta$.

\subsection{Caso particular \texorpdfstring{$\lambda = 1/2$}{}.}

Partiendo de la ecuación de Gegenbauer, haciendo que $\lambda = 1/2$, por lo que en términos de los polinomios de Jacobi se tiene que $\alpha = \beta = 0$, se obtiene la \emph{ecuación de Legendre}:
\begin{align*}
(1 - x^{2}) \, \dv[2]{x} \omega(x) - 2 \, x \, \dv{x} \omega (x) + n(n + 1) \, \omega (x) = 0
 \end{align*}
cuya solución son los \emph{polinomios de Legendre}, que en términos de la función hipergeométrica es:
\begin{align*}
P_{n} (x) =  F \left( -n, n + 1, 1; \dfrac{1 - x}{2} \right)
\end{align*}
\end{document}