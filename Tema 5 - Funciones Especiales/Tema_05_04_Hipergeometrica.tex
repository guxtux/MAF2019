\documentclass[12pt]{beamer}
\usepackage{../Estilos/BeamerMAF}
\usepackage{../Estilos/ColoresLatex}
\usepackage[absolute, overlay]{textpos}

\input{../Preambulos/preambulo_Beamer_Copenhagen_wolverine}

\resetcounteronoverlays{saveenumi}

\AtBeginDocument{\RenewCommandCopy\qty\SI}
\ExplSyntaxOn
\msg_redirect_name:nnn { siunitx } { physics-pkg } { none }
\ExplSyntaxOff

% \documentclass[12pt]{beamer}
% \usepackage{../Estilos/BeamerMAF}
% \input{../Preambulos/preambulo_Beamer_Copenhagen_wolverine}

\date{7 de noviembre de 2024}

\title{\large{Funciones hipergeométricas generalizadas}}
\author{M. en C. Gustavo Contreras Mayén}

\begin{document}
\maketitle
\fontsize{14}{14}\selectfont
\spanishdecimal{.}

\section*{Contenido}
\frame[allowframebreaks]{\frametitle{Contenido} \tableofcontents[currentsection, hideallsubsections]}

%Ref. Slater (1996) - Generalized hypergeometric functions
\section{La función de Gauss.}
\frame[allowframebreaks]{\frametitle{Temas a revisar} \tableofcontents[currentsection, hideothersubsections]}
\subsection{Definiciones}

\begin{frame}
\frametitle{Definición}
La serie:
\pause
\begin{align}
\begin{aligned}[b]
1 &+ \dfrac{a b}{c} \dfrac{z}{1!} + \dfrac{a (a {+} 1) b (b {+} 1)}{c( c {+} 1)} \dfrac{z^{2}}{2!} + \\[0.5em]
&+ \dfrac{a (a {+} 1)(a {+} 2) b (b {+} 1)(b {+} 2)}{c (c {+} 1)(c {+} 2)} \dfrac{z^{3}}{3!} + \ldots
\end{aligned}
\label{eq:ecuacion_01_01_01}
\end{align}
es llamada \textocolor{red}{serie de Gauss} o \textocolor{ao}{serie hipergeométrica ordinaria}.
\end{frame}
\begin{frame}
\frametitle{Notación utilizada}
Normalmente se representa con la notación:
\pause
\begin{align*}
{}_{2} F_{1} \big[ a, b; c; z \big]
\end{align*}
La variable es $z$, y $a$, $b$ y $c$ son los parámetros  de la función.
\end{frame}
\begin{frame}
\frametitle{De serie a polinomio}
Si alguna de las cantidades $a$ o $b$ es un entero negativo $-n$, la serie tiene solo un número finito de términos y se convierte de hecho en un polinomio:
\begin{align*}
{}_{2} F_{1} \big[ -n, b; c; z \big]
\end{align*}
\end{frame}
\begin{frame}
\frametitle{De serie a polinomio}
Por ejemplo, si $a = -2$, entonces la serie es:
\pause
\begin{align*}
{}_{2} F_{1} \big[ -2, b: c; z \big] &= 1 + \dfrac{(-2) b}{c} \dfrac{z}{1!} + \\[0.5em]
&+ \dfrac{(-2) (-1) b (b + 1)}{c( c + 1)} \dfrac{z^{2}}{2!} + 0 
\end{align*}    
esto es:
\pause
\begin{align}
{}_{2} F_{1} \big[ -2, b; c; z \big] = 1 - \dfrac{2 b z}{c} + \dfrac{b( b + 1) z^{2}}{c (c + 1)}
\label{eq:ecuacion_01_01_02}
\end{align}
ya que los restantes términos se anulan.
\end{frame}

\subsection{Desarrollo histórico}

\begin{frame}
\frametitle{Uso del término \emph{hiper}}
En su trabajo \textocolor{carmine}{Arithmetica Infinitorum} (1655), el profesor de Oxford John Wallis (1616 - 1703) fue el primero en utilizar el término \enquote{hipergeométrica} (del griego $\acute{\nu} \pi \epsilon \rho$, por arriba o más allá de...)
\end{frame}
\begin{frame}
\frametitle{Series de trabajo de Wallis}
Para denotar cualquier serie que estuviera más allá de las serie geométrica ordinaria:
\pause
\begin{align*}
1 + x + x^{2} + x^{3} + \ldots
\end{align*}
\pause
En particular, estudió las series:
\begin{align*}
1 + a + a( a + 1) + a (a + 1)(a + 2) + \ldots
\end{align*}
\end{frame}
\begin{frame}
\frametitle{Siguientes estudios}
Durante los siguientes ciento cincuenta años, muchos otros matemáticos estudiaron series similares, notablemente Leonard Euler (1783) dio, entre muchos otros resultados, la famosa relación:
\pause
\begin{align}
\begin{aligned}[b]
{}_{2} &F_{1} \big[ -n, b; c; z \big] = \\[0.5em]
&= (1 - z)^{c+n-b} \, {}_{2} F_{1} \big[ c + n, c - b; c; z \big]
\end{aligned}
\label{eq:ecuacion_01_01_03} 
\end{align}
\end{frame}
\begin{frame}
\frametitle{Extensión del teorema del binomio}
En 1770, el francés A. T. Vandermonde (1735-1796) estableció su teorema, una extensión del teorema del binomio, en la forma:
\pause
\begin{align}
\begin{aligned}[b]
{}_{2} &F_{1} \big[ -n, b; c; 1 \big] = \\[0.5em]
&= \dfrac{(c {-} b)(c {-} b {+} 1)(c {-} b {+} 2) \cdots (c {-} b {+} n {-} 1)}{c (c {+} 1)(c {+} 2)(c {+} 3) \cdots (c {+} n {-} 1)}
\end{aligned}
\label{eq:ecuacion_01_01_04}
\end{align}
\end{frame}
\begin{frame}
\frametitle{Avances infructuosos}
Pero durante los siguientes cuarenta años, la escuela de Göttingen bajo C. F. Hindenberg (1741-1808) desperdició mucho esfuerzo en varias extensiones complicadas de los teoremas binomial y multinomial.
\end{frame}
\begin{frame}
\frametitle{Avances infructuosos}
Todo esto cambió drásticamente cuando el 20 de enero de 1812 C. F. Gauss (1777-1855) pronunció su famosa tesis \enquote{Disquisitiones generales circa seriem infinitam} ante la Real Sociedad de Göttingen.
\end{frame}
\begin{frame}
\frametitle{El avance de Gauss}
En el, Gauss definió la serie infinita moderna de la ec. (\ref{eq:ecuacion_01_01_01}) e introdujo la notación $F \big[a, b; c; z \big]$ para ello.
\end{frame}
\begin{frame}
\frametitle{El avance de Gauss}
También demostró su famoso teorema de suma:
\pause
\begin{align}
{}_{2} F_{1} \big[ a, b; c; 1 \big] = \dfrac{\Gamma (c) \Gamma (c - a - b)}{\Gamma(c - a) \Gamma (c - b)}
\label{eq:ecuacion_01_01_05}
\end{align}
y dio muchas relaciones entre dos o más de estas series.
\end{frame}
\begin{frame}
\frametitle{Más aportaciones de Gauss}
Gauss mostró claramente que ya estaba considerando ${}_{2} F_{1} \big[ a, b; c; z \big]$ como \textocolor{byzantium}{una función de cuatro variables}, en lugar de una serie en $z$.
\\
\bigskip
\pause
En una nota agregada el 10 de febrero de 1812, dio una discusión notablemente completa de la convergencia de tales series.
\end{frame}
\begin{frame}
\frametitle{Los trabajos de Kummer}
El siguiente gran avance fue realizado en 1836 por E. E. Kummer (1810 - 1893), quien utilizó por primera vez el término \enquote{hipergeométrico} únicamente para series del tipo (\ref{eq:ecuacion_01_01_01}).
\end{frame}
\begin{frame}
\frametitle{Los trabajos de Kummer}
Mostró que la ecuación diferencial:
\pause
\begin{align}
z (1 - z) \dv[2]{y}{z} + \big[ c - \big( 1 + a + b \big) z \big] \dv{y}{z} - a b y = 0
\label{eq:ecuacion_01_01_06}
\end{align}
se satisface por la función:
\pause
\begin{align*}
{}_{2} F_{1} \big[ a, b; c; z \big]
\end{align*}
y tiene en total veinticuatro soluciones en términos de funciones de Gauss similares.
\end{frame}
\begin{frame}
\frametitle{La aportación de Riemann}
En 1857, G. F. B. Riemann (1826 - 1866) amplió esta teoría mediante la introducción de sus funciones $P$, que en cierto modo, son generalizaciones de la Gaussiana:
\pause
\begin{align*}
{}_{2} F_{1} \big[ a, b; c; z \big]
\end{align*}    
\end{frame}
\begin{frame}
\frametitle{La aportación de Riemann}
Riemann también discutió la teoría general de la transformación de la variable en una ecuación diferencial y esta teoría fue aplicada al trabajo de Kummer por J. Thomae quien, en 1879, elaboró en detalle las relaciones entre las veinticuatro soluciones de Kummer.
\end{frame}
\begin{frame}
\frametitle{Representación integral}
La primera representación integral de la función de Gauss se remonta a Euler, quien demostró que:
\pause
\begin{align}
\begin{aligned}[b]
{}_{2} &F_{1} \big[ -n, b; c; z \big] = \dfrac{n!}{c (c {+} 1)(c {+} 2) \cdots (c {+} {-} 1)} \times \\[0.5em]
&\times \scaleint{6ex}_{\bs 0}^{1} t^{-n-1} \, (1 {-} t)^{c+n-1} \, (1 - t z)^{-b} \dd{t}
\end{aligned}
\label{eq:ecuacion_01_01_07}
\end{align}
\end{frame}
\begin{frame}
\frametitle{Funciones como integrales}
La idea básica de representar una función mediante una integral de contorno con funciones Gamma en el integrando parece deberse a S. Pincherle (1853 - 1936), quien utilizó contornos de un tipo que proviene del trabajo de Riemann. 
\end{frame}
\begin{frame}
\frametitle{Funciones como integrales}
Este lado del tema fue desarrollado extensamente por R. Mellin y E. W. Barnes.
\\
\bigskip
\pause
En 1907, Barnes publicó sus representaciones integrales de contorno de las veinticuatro funciones de Kummer.
\end{frame}
\begin{frame}
\frametitle{La demostración de Barnes}
Más tarde, en 1910, demostró el análogo integral del teorema de Gauss.
\begin{align}
\begin{aligned}[b]
\dfrac{1}{2 \pi i} \scaleint{6ex}_{\bs -\infty}^{\infty} \, &\Gamma (a {+} s) \, \Gamma (b {+} s) \, \Gamma(c {-} s) \, \Gamma (d {-} s) \dd{s} = \\[0.5em]
&= \dfrac{\Gamma (a {+} c) \, \Gamma (a {+} d) \, \Gamma(b {+} c) \, \Gamma (b {+} d)}{\Gamma (a {+} b {+} c) {+ d}}
\end{aligned}
\label{eq:ecuacion_01_01_08}
\end{align}
\end{frame}


\section{Las series de Gauss y su convergencia.}
\frame[allowframebreaks]{\frametitle{Temas a revisar} \tableofcontents[currentsection, hideothersubsections]}
\subsection{Función de apoyo}

\begin{frame}
\frametitle{Símbolos de Pochhamer}
Escribamos lo siguiente:
\pause
\begin{align}
\big( a \big)_{n} = a ( a + 1)(a + 2)(a + 3) \cdots (a + n - 1)
\label{eq:ecuacion_01_01_01_01}
\end{align}
y en particular $\big( a \big)_{0} \equiv 1$
\end{frame}
\begin{frame}
\frametitle{Símbolos de Pochhamer}
Por ejemplo:
\pause
\begin{align*}
\big( 3 \big)_{5} = 3 \cdot 4 \cdot 5 \cdot 6 \cdot 7 = 2520
\end{align*}
\pause
y también se tiene que $\big( 1 \big) = n!$.
\end{frame}
\begin{frame}
\frametitle{Representación con la función Gamma}
Entonces:
\pause
\begin{align}
\big( a \big)_{n} = \dfrac{\Gamma (a + n)}{\Gamma (a)}
\label{eq:ecuacion_01_01_01_02}
\end{align}
y: 
\pause
\begin{align}
\lim_{n \to \infty} \big( a \big)_{n} = \dfrac{1}{\Gamma (a)}
\label{eq:ecuacion_01_01_01_03}
\end{align}
\end{frame}
\begin{frame}
\frametitle{Caso especial}
Si $a$ es un entero negativo $-m$, entonces:
\pause
\begin{align*}
\big( a \big)_{n} = \begin{cases}
\big( -m \big)_{n} & \mbox{si } m \geq n \\
0 & \mbox{si } m < n
\end{cases}
\end{align*}
por lo que: \pause
\begin{align*}
\big( -3 \big)_{3} = (-3)(-2)(-1) = -6 \hspace{0.5cm} \mbox{pero} \hspace{0.5cm} \big( -3 \big)_{4} = 0
\end{align*}
\end{frame}
\begin{frame}
\frametitle{Función de Gauss con notación Pochhamer}
En esta notación, la función de Gauss es:
\pause
\begin{align}
{}_{2} &F_{1} \big[ a, b; c; z \big] = \nsum_{n=1}^{\infty} \dfrac{\big( a \big)_{n} \big( b \big)_{n} \, z^{n}}{\big( c \big)_{n} \, n!}
\label{eq:ecuacion_01_01_01_04}
\end{align}
donde $a$, $b$, $c$ y $z$ pueden ser reales o complejos.
\end{frame}
\begin{frame}
\frametitle{De serie a polinomio}
A partir de esto, vemos que si cualquiera de los números $a$ o $b$ es cero o un número entero negativo, \pause la función se reduce a un polinomio.
\end{frame}
\begin{frame}
\frametitle{De serie a polinomio}
Pero si $c$ es cero o un número entero negativo, \pause la función no está definida, ya que todos menos un número finito de los términos de la serie se vuelven infinitos.
\end{frame}
\begin{frame}
\frametitle{Un resultado importante}
También tenemos inmediatamente que:
\pause
\begin{align}
\begin{aligned}[b]
\dv{z} \bigg( {}_{2} &F_{1} \big[ a, b; c; z \big] \bigg) = \\[0.5em]
&= \dfrac{a b}{c} \, {}_{2} F_{1} \big[ a + 1, b + 1; c + 1; z \big]
\end{aligned}
\label{eq:ecuacion_01_01_01_05}
\end{align}
\end{frame}

\subsection{Notación para la función de Gauss}

\begin{frame}
\frametitle{Diferentes notaciones}
La función de Gauss se puede encontrar en distintas notaciones, por ejemplo:
\pause
\setbeamercolor{item projected}{bg=bananayellow,fg=ao}
\setbeamertemplate{enumerate items}{%
\usebeamercolor[bg]{item projected}%
\raisebox{1.5pt}{\colorbox{bg}{\color{fg}\footnotesize\insertenumlabel}}%
}
\begin{enumerate}[<+->]
\item Appell (1926):
\pause
\begin{align}
{}_{2} F_{1} \bigg[
\begin{tabular}{c c}
a, b; & \multirow{2}{*}{z} \\
c; &
\end{tabular} \bigg] = {}_{2} F_{1} \big[ a, b; c; z \big]
\label{eq:ecuacion_01_01_01_06}
\end{align}
\seti 
\end{enumerate}
\end{frame}
\begin{frame}
\frametitle{Diferentes notaciones}
\setbeamercolor{item projected}{bg=bananayellow,fg=ao}
\setbeamertemplate{enumerate items}{%
\usebeamercolor[bg]{item projected}%
\raisebox{1.5pt}{\colorbox{bg}{\color{fg}\footnotesize\insertenumlabel}}%
}
\begin{enumerate}[<+->]    
\conti
\item Bailey (1935):
\begin{align}
F \big( a, b; c; z \big) = {}_{2} F_{1} \big[ a, b; c; z \big]
\label{eq:ecuacion_01_01_01_07}
\end{align}
\seti 
\end{enumerate}
\end{frame}
\begin{frame}
\frametitle{Diferentes notaciones}
\setbeamercolor{item projected}{bg=bananayellow,fg=ao}
\setbeamertemplate{enumerate items}{%
\usebeamercolor[bg]{item projected}%
\raisebox{1.5pt}{\colorbox{bg}{\color{fg}\footnotesize\insertenumlabel}}%
}
\begin{enumerate}[<+->]    
\conti
\item Meijer (1953):
\begin{align}
\Phi \big( a, b; c; z \big) = \dfrac{{}_{2} F_{1} \big[ a, b; c; z \big]}{\Gamma (c)}
\label{eq:ecuacion_01_01_01_08}
\end{align}
\seti 
\end{enumerate}
\end{frame}
\begin{frame}
\frametitle{Diferentes notaciones}
\setbeamercolor{item projected}{bg=bananayellow,fg=ao}
\setbeamertemplate{enumerate items}{%
\usebeamercolor[bg]{item projected}%
\raisebox{1.5pt}{\colorbox{bg}{\color{fg}\footnotesize\insertenumlabel}}%
}
\begin{enumerate}[<+->]    
\conti
\item MacRobert (1947):
\begin{align}
\begin{aligned}[b]
E \big( 2; a, &b; 1; c; -\dfrac{1}{z} \big) = \\[0.5em]
&= \dfrac{\Gamma (a) \Gamma (b)}{\Gamma (c)} \, {}_{2} F_{1} \big[ a, b; c; z \big]
\label{eq:ecuacion_01_01_01_09}
\end{aligned}
\end{align}
\end{enumerate}
\end{frame}

\subsection{Pruebas de convergencia}

\begin{frame}
\frametitle{Estableciendo pruebas de convergencia}
Hagamos que:
\pause
\begin{align*}
u_{n} = \dfrac{\big( a \big)_{n} \big( b \big)_{n}}{\big( c \big)_{n} \big( 1 \big)_{n}}
\end{align*}
\end{frame}
\begin{frame}
\frametitle{Pruebas de convergencia}
Entonces se tiene que:
\pause
\begin{align}
(1 + n)(c + n) \, u_{n+1} = (a + n)(b +  n) \, u_{n}
\label{eq:ecuacion_01_01_01_12}
\end{align}
\pause
la razón de dos términos sucesivos $u_{n}$ y $u_{n+1}$ de la serie Gaussiana es:
\pause
\begin{align}
\dfrac{(a + n)(b +  n)}{(c + n)(1 +  n)} \, z = \dfrac{(1 + a/n)(1 + b/n)}{(1 + c/n)(1 + 1/n)} \, z
\label{eq:ecuacion_01_01_01_13}
\end{align}
\end{frame}
\begin{frame}
\frametitle{Tomando un límite}
Tal que, cuando $n \to \infty$, la razón:
\pause
\begin{align*}
\abs{\dfrac{u_{n+1}}{u_{n}}} \to \abs{z}
\end{align*}
\pause
Por tanto, según la prueba de D'Alembert, la serie es convergente para todos los valores de $z$, reales o complejos tales que $\abs{z} < 1$, y divergente para todos los valores de $z$ reales o complejos, tales que $\abs{z} > 1$.
\end{frame}
\begin{frame}
\frametitle{Caso especial}
Cuando $\abs{z} = 1$:
\pause
\begin{eqnarray}
\begin{aligned}[b]
\abs{\dfrac{u_{n+1}}{u_{n}}} &= \abs{\left\{ 1 {+} \dfrac{a {+} b}{n} {+} \order{\dfrac{1}{n^{2}}} \right\} \left\{ 1 {-} \dfrac{1 {+} c}{n} {+} \order{\dfrac{1}{n^{2}}} \right\}} = \\[0.5em] \pause
&= \abs{1 + \dfrac{a + b - c - 1}{n} + \order{\dfrac{1}{n^{2}}}} \\[0.5em] \pause
&\leq 1 {+} \left\{ \dfrac{\Re (a {+} b {-} c {-} 1)}{n} \right\} {+} \order{\dfrac{1}{n^{2}}}
\end{aligned}
\label{eq:eq:ecuacion_01_01_01_14}
\end{eqnarray}
\end{frame}
\begin{frame}
\frametitle{Convergencia de la serie}
Entonces, cuando $z = 1$, por la prueba de Raabe, la serie es convergente si $\Re (c - a - b) > 0$, y divergente si $\Re (c - a - b) < 0$.
\end{frame}
\begin{frame}
\frametitle{Divergencia de la serie}
También es divergente cuando $\Re (c - a - b) = 0$, en este caso:
\pause
\begin{align*}
\abs{\dfrac{u_{n+1}}{u_{n}}} > 1 - \dfrac{1}{n} - \dfrac{C}{n^{2}}
\end{align*}
donde $C$ es una constante.
\end{frame}
\begin{frame}
\frametitle{Convergencia absoluta}
Cuando $\abs{z} = 1$, pero $z \neq 1$, la serie es absolutamente convergente cuando $\Re (c - a - b) > 0$, convergente pero no de manera absoluta solo cuando:
\pause
\begin{align*}
- 1 < \Re (c - a - b) \leq 0
\end{align*}
\pause
y divergente cuando $\Re (c - a - b) < -1$.
\end{frame}
\begin{frame}
\frametitle{Convergencia absoluta}
Si $\Re (c - a - b) = -1$, se requieren pruebas de convergencia más delicadas.
\end{frame}
\begin{frame}
\frametitle{Caso especial}
En este caso, se tiene que:
\pause
\begin{align}
\abs{\dfrac{u_{n+1}}{u_{n}}} = 1 - \dfrac{\Re (a + b - a b + 1)}{n^{2}} + \order{\dfrac{1}{n^{3}}}
\label{eq:ecuacion_01_01_01_15}
\end{align}
\pause
De aquí la serie es convergente si $\Re (a + b) > \Re \, a \, b$, y divergente si $\Re (a + b) \leq \Re \, a \, b$.
\end{frame}
\begin{frame}
\frametitle{Ejemplo de serie divergente}
Por ejemplo, la serie:
\pause
\begin{align}
\begin{aligned}
1 - \dfrac{2}{3} &+ \dfrac{3}{4} - \dfrac{4}{5} + \dfrac{5}{6} - \dfrac{6}{7} + \cdots = \\[0.5em]
&= \dfrac{1}{2} \left\{ 1 + \, {}_{2} F_{1} \big[ 2, 2; 3; -1 \big] \right\}
\end{aligned}
\label{eq:ecuacion_01_01_01_16}
\end{align}
es divergente.
\end{frame}
\begin{frame}
\frametitle{Cociente divergente}
Notemos que:
\pause
\begin{align}
\begin{aligned}
\dfrac{\big( a \big)_{n} \big( b \big)_{n}}{\big( c \big)_{n}} &\to 0 \hspace{1cm} \mbox{cuando} \hspace{0.2cm} n \to \infty, \hspace{0.2cm} \\[0.5em]
&\mbox{si } 0 < \Re (1 + c - a - b) < 1
\end{aligned}
\label{eq:ecuacion_01_01_01_17}
\end{align}
\end{frame}

\section{Función hipergeométrica}
\frame[allowframebreaks]{\frametitle{Temas a revisar} \tableofcontents[currentsection, hideothersubsections]}
\subsection{Definición}

\begin{frame}
\frametitle{Ecuación diferencial de partida}
La ecuación diferencial hipergeométrica tiene la forma:
\pause
\begin{align}
x (1 - x) \, \sderivada{y} + \big[ c - (a + b - 1) \, x \big] \, \pderivada{y} -  a \, b \, y = 0
\label{eq:ecuacion_18_136}
\end{align}
tiene tres puntos singulares regulares, en $x = 0, 1, \infty$, pero sin singularidades esenciales. Los parámetros $a$, $b$ y $c$ son números reales.
\end{frame}
\begin{frame}
\frametitle{Ecuación diferencial de partida}
%En la revisión previa sobre las funciones de Legendre, las funciones de Legendre asociadas y las funciones de Chebyshev, se observó que en cada caso la ecuación diferencial de segundo orden correspondiente tenía tres puntos singulares regulares, en $x = -1, 1 , \infty$, y sin singularidades esenciales.
La ecuación hipergeométrica puede considerarse como la \textocolor{red}{forma canónica} de las ecuaciones diferenciales de segundo orden con este número de singularidades.
\end{frame}
\begin{frame}
\frametitle{Ecuación diferencial de partida}
Se puede demostrar que, al hacer los cambios apropiados de las variables independientes y dependientes, cualquier ecuación diferencial de segundo orden con tres singularidades regulares y un punto ordinario en el infinito se puede transformar en la ecuación hipergeométrica (\ref{eq:ecuacion_18_136}) con las singularidades en $x= -1 , 1, \infty$.
\end{frame}
\begin{frame}
\frametitle{Ecuación diferencial de partida}
Esto permite que las funciones de Legendre, las funciones de Legendre asociadas y las funciones de Chebyshev, por ejemplo, se escriban como \textocolor{cobalt}{casos particulares de funciones hipergeométricas}, que son las soluciones a la ec. (\ref{eq:ecuacion_18_136}).
\end{frame}
\begin{frame}
\frametitle{Singularidades en la ED}
Ya que el punto $x = 0$ es una singularidad regular de la ec. (\ref{eq:ecuacion_18_136}), podemos encontrar al menos una solución en una forma de serie de Frobenius:
\pause
\begin{align}
y (x) = \nsum_{0}^{\infty} a_{n} \, x^{n+\sigma}
\label{eq:ecuacion_18_137}
\end{align}
\end{frame}
\begin{frame}
\frametitle{Solución en serie de potencias}
Sustituyendo esta serie en la ec. (\ref{eq:ecuacion_18_136}) para luego dividir entre $x^{\sigma-1}$, se obtiene:
\pause
\begin{align}
\begin{aligned}[b]
\nsum_{n=0}^{\infty} &\big[ (1 - x)(n + \sigma)(n + \sigma - 1) + \\[0.5em]
&+ [c - (a + b + 1) \, x] \, (n + \sigma) - a \, b \, x \big] a_{n} \, x^{n} = 0
\end{aligned}
\label{eq:ecuacion_18_138}
\end{align}
\end{frame}
\begin{frame}
\frametitle{Solución en serie de potencias}
Estableciendo $x = 0$, de modo que solo quede el término $n = 0$, obtenemos la ecuación de índices:
\begin{align*}
\sigma (\sigma - 1) + c \, \sigma = 0
\end{align*}
que tiene las raíces $\sigma = 0$ y $\sigma = 1 - c$.
\end{frame}
\begin{frame}
\frametitle{Soluciones a la ED}
Por tanto, siempre que $c$ no sea un número entero, se pueden obtener dos soluciones linealmente independientes de la ecuación hipergeométrica en la forma (\ref{eq:ecuacion_18_137}).
\\
\bigskip
\pause
Para $\sigma = 0$, la solución correspondiente es una serie de potencias simple.
\end{frame}
\begin{frame}
\frametitle{Relación de recurrencia}
Sustituyendo $\sigma = 0$ en la ec. (\ref{eq:ecuacion_18_138}) y exigiendo que el coeficiente de $x^{n}$ se anule, encontramos la relación de recurrencia:
\pause
\begin{align}
\begin{aligned}
&n \big[ (n - 1) + c \big] \, a_{n} - \big[ (n - 1)(a + b + n - 1) + \\[0.5em]
&+ a \, b \big] \, a_{n-1} = 0
\end{aligned}
\label{eq:ecuacion_18_139}
\end{align}
\end{frame}
\begin{frame}
\frametitle{Relación de recurrencia}
Que al simplificar y reemplazar $n$ por $n + 1$, nos lleva a la relación de recurrencia:
\pause
\begin{align}
a_{n+1} = \dfrac{(a + n)(b + n)}{(n + 1)(c + n)} \, a_{n}
\label{eq:ecuacion_18_140}
\end{align}
Es convencional hacer la elección simple de $a_{0} = 1$.
\end{frame}
\begin{frame}
\frametitle{Presentando la solución}
Por lo tanto, siempre que $c$ no sea un número entero negativo o cero, podemos escribir la solución de la siguiente manera:
\pause
\begin{align}
\begin{aligned}
&F(a, b, c; x) {=} 1 {+} \dfrac{a \, b}{c} \, \dfrac{x}{1!} + \\[1em]
&+ \dfrac{a (a {+} 1)\, b (b {+} 1)}{c (c {+} 1)} \, \dfrac{x^{2}}{2!} + \ldots \label{eq:ecuacion_18_141}
\end{aligned}
\end{align}
\end{frame}
\begin{frame}
\frametitle{Presentando la solución}
\begin{align}
F(a, b, c; x) {=} \dfrac{\Gamma (c)}{\Gamma (a) \, \Gamma (b)} \nsum_{n=0}^{\infty} \dfrac{\Gamma (a {+} n) \, \Gamma (b {+} n)}{\Gamma (c {+} n)} \, \dfrac{x^{n}}{n!} \label{eq:ecuacion_18_142}
\end{align}
donde $F(a, b, x; x)$ se le conoce como la \textocolor{burgundy}{función hipergeométrica} o \textocolor{bole}{serie hipergeométrica}.
\end{frame}
\begin{frame}
\frametitle{Convergencia de la solución}
Es sencillo demostrar que la serie hipergeométrica converge en el rango $\abs{x} < 1$.
\\
\bigskip
\pause
También converge en $x = 1$ si $ c > a + b$ y en $x = -1$ si $c > a + b - 1$.
\end{frame}
\begin{frame}
\frametitle{Simetría en la solución}
También observamos que $F (a, b, c; x)$ es simétrico en los parámetros $a$ y $b$, es decir, $F (a, b, c; x) = F (b, a, c; x)$.
\end{frame}
\begin{frame}
\frametitle{Solución parcial}
La función hipergeométrica $y (x) = F (a, b, c; x)$ claramente no es la solución general de la ecuación hipergeométrica (\ref{eq:ecuacion_18_136}), \pause ya que también debemos considerar la segunda raíz de la ecuación de índices.
\end{frame}
\begin{frame}
\frametitle{Segunda solución}
Sustituyendo la raíz $\sigma = 1 - c$ en la ec.  (\ref{eq:ecuacion_18_138}) y exigiendo que el coeficiente de $x^{n}$ se anule, encontramos que debemos tener:
\pause 
\begin{align*}
n (n + 1 - &c) \, a_{n} - \big[ (n - c)(a + b + n - c) + \\[1em]
&+ a \, b \big] \, a_{n-1} = 0
\end{align*}
\end{frame}
\begin{frame}
\frametitle{Segunda solución}
Que al comparar con la ec. (\ref{eq:ecuacion_18_139}) y al reemplazar $n$ por $n + 1$, se obtiene la relación de recurrencia:
\pause
\begin{align*}
a_{n+1} = \dfrac{(a - c + 1 + n)(b -c + 1 + n)}{(n + 1)(2 - c + n)} \, a_{n}
\end{align*}
Vemos que esta relación de recurrencia tiene la misma forma que la ec. (\ref{eq:ecuacion_18_140}) si se hacen los reemplazos $a \to a - c + 1$, $b \to b - c + 1$ y $c \to 2 - c$.
\end{frame}
\begin{frame}
\frametitle{Solución general a la ED}
Por lo tanto, siempre que $c$, $a - b$ y $c - a - b$ sean no enteros, la solución general de la ecuación hipergeométrica, válida para el intervalo $\abs{x} < 1$, puede escribirse como:
\pause
\begin{align}
\begin{aligned}
y (x) &= A \, F (a, b, c; x) + \\[1em]
&+ B \, x^{1-c} \, F (a-c+1, b - c + 1, 2 - c; x)
\end{aligned}
\label{eq:ecuacion_18_143}
\end{align}
\end{frame}
\begin{frame}
\frametitle{Solución general a la ED}
Donde $A$ y $B$ son constantes arbitrarias que quedarán determinadas por las CDF en la solución.
\\
\bigskip
\pause
Para que sea solución en $x = 0$, se requiere que $B = 0$.
\end{frame}

\end{document}