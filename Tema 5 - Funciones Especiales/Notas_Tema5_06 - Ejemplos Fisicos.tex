\documentclass[12pt]{article}
\usepackage[utf8]{inputenc}
\usepackage[spanish,es-lcroman, es-tabla]{babel}
\usepackage[autostyle,spanish=mexican]{csquotes}
\usepackage{amsmath}
\usepackage{amssymb}
\usepackage{nccmath}
\numberwithin{equation}{section}
\usepackage{amsthm}
\usepackage{graphicx}
\usepackage{epstopdf}
\DeclareGraphicsExtensions{.pdf,.png,.jpg,.eps}
\usepackage{color}
\usepackage{float}
\usepackage{multicol}
\usepackage{enumerate}
\usepackage[shortlabels]{enumitem}
\usepackage{anyfontsize}
\usepackage{anysize}
\usepackage{array}
\usepackage{multirow}
\usepackage{enumitem}
\usepackage{cancel}
\usepackage{tikz}
\usepackage{circuitikz}
\usepackage{tikz-3dplot}
\usetikzlibrary{babel}
\usepackage{bm}
\usepackage{mathtools}
\usepackage{esvect}
\usepackage{hyperref}
\usepackage{relsize}
\usepackage{siunitx}
\usepackage{physics}
%\usepackage{biblatex}
\usepackage{standalone}
\usepackage{mathrsfs}
\usepackage{bigints}
\usepackage{bookmark}
\spanishdecimal{.}

\setlist[enumerate]{itemsep=0mm}

\renewcommand{\baselinestretch}{1.5}

\let\oldbibliography\thebibliography

\renewcommand{\thebibliography}[1]{\oldbibliography{#1}

\setlength{\itemsep}{0pt}}
%\marginsize{1.5cm}{1.5cm}{2cm}{2cm}


\newtheorem{defi}{{\it Definición}}[section]
\newtheorem{teo}{{\it Teorema}}[section]
\newtheorem{ejemplo}{{\it Ejemplo}}[section]
\newtheorem{propiedad}{{\it Propiedad}}[section]
\newtheorem{lema}{{\it Lema}}[section]

\title{Ejemplos Físicos \\ {\large Matemáticas Avanzadas de la Física}}
\date{ }
\begin{document}
\maketitle
\fontsize{14}{14}\selectfont
\section{Expansion en Polinomios de Legendre.}
La ortogonalidad de los polinomios de Legendre nos es de utilidad para expandir una función definida en el intervalo $(-1,1)$. Sea $f(x)$ esa función, entonces escribimos:
\begin{equation}
f(x) = \sum_{n=0}^{\infty} c_{n} \, P_{n} (x)
\label{eq:ecuacion_26_46}
\end{equation}
quedando por definir los coeficientes $c_{n}$.
\par
Los coeficientes $c_{n}$ se obtienen al multiplicar en ambos lados de la igualdad por $P_{m}$, para luego integrar de $-1$ a $1$. En la expresión del lado izquierdo tenemos
\[ \int_{-1}^{1} f(x) \, P_{m}(x) \, \dd x  \]
y del lado derecho
\[ \int_{-1}^{1} \left( \sum_{n=0}^{\infty} c_{n} P_{n} (x) \right) \, P_{m} (x) \, \dd x = \sum_{n=0}^{\infty} \int_{-1}^{1} P_{n}(x) \, P_{m}(x) \, \dd x \]
pero el valor de la integral es tal que
\[ \int_{-1}^{1} P_{n}(x) \, P_{m}(x) \, \dd x = \dfrac{2}{(2 \, n +1)} \, \delta_{nm} \]
por tanto
\[ \int_{-1}^{1} \left( \sum_{n=0}^{\infty} c_{n} P_{n} (x) \right) \, P_{m} (x) \, \dd x = c_{m} \, \dfrac{2}{(2 \, n +1)} \]
igualando las expresiones, se tiene que
\begin{align}
\begin{aligned}
c_{m} &= \dfrac{2 \, m + 1}{2} \, \int_{-1}^{1} f(x) \, P_{m} (x) \, \dd x \hspace{0.5cm} \mbox{ o } \\
c_{n} &= \dfrac{2 \, n + 1}{2} \, \int_{-1}^{1} f(x) \, P_{n} (x)\, \dd x
\end{aligned}
\label{eq:ecuacion_26_47}
\end{align}
Las ecs. (\ref{eq:ecuacion_26_46}) y (\ref{eq:ecuacion_26_47}) nos proporcionan una manera para expandir una función arbitraria en el intervalo $(-1, 1)$ en términos de polinomios de Legendre.
\par
Si $f(x)$ es un polinomio de grado $k$, entonces se puede escribir como una suma finita de polinomios de Legendre de grado $k$ y menor.
\par
De hecho, para $f(x) = x^{k}$, se tiene
\[ c_{n} = \dfrac{2 \, n + 1}{2} \int_{-1}^{1} x^{k} \, P_{n} (x) \, \dd x = 0, \hspace{1.5cm} \mbox{para } n > k \]
Por lo tanto, los coeficientes en la suma (\ref{eq:ecuacion_26_46}) más allá de $k$ son todos cero.
\subsection{Ejemplo. Expansión de $f(x)$.}
Queremos encontrar la expansión de Legendra de la función $f(x)$ definida por
\begin{align*}
f(x) = \begin{cases}
V_{0} & \mbox{ si } 0 < x \leq 1 \\
-V_{0} & \mbox { si } -1 \leq x < 1
\end{cases}
\end{align*}
Para encontrar los coeficientes de la expansión, usamos la ec. (\ref{eq:ecuacion_26_47}):
\begin{align}
\begin{aligned}
c_{n} &= \dfrac{2 \, n + 1}{2} \int_{-1}^{1} f(x) \, P_{n} (x) \, \dd x \\
&= \dfrac{2 \, n + 1}{2} \int_{-1}^{0} \underbrace{f(x)}_{=-V_{0}} \, P_{n} (x) \, \dd x + \dfrac{2 \, n + 1}{2} \int_{0}^{1} \underbrace{f(x)}_{=+V_{0}} \, P_{n} (x) \, \dd x \\
&= \dfrac{2 \, n + 1}{2} \, V_{0} \, \left[ - \int_{-1}^{0} P_{n} (x) \, \dd x + \int_{0}^{1} P_{n} (x) \, \dd x  \right]
\end{aligned}
\label{eq:ecuacion_26_48}
\end{align}
Para la primera integral, hacemos el cambio de variable $x = -y$, y usando la fórmula de paridad
\begin{equation}
P_{k}(-u) = (-)^{k} \, P_{k}(u)
\label{eq:ecuacion_26_35}
\end{equation}
tenemos que
\begin{align*}
\int_{-1}^{0} P_{n}(x) \, \dd x = \int_{+1}^{0} P_{n} (-y) \, (- \dd y) = \int_{0}^{1} P_{n} (-y) \, \dd y = (-)^{n} \, \int_{0}^{1} P_{n} (x) \, \dd x  
\end{align*}
en la última igualdad, se ha cambiado la variable muda de integración de $y$ a $x$.
\par
Ocupando este resultado en la ec. (\ref{eq:ecuacion_26_48}), se obtiene:
\begin{align*}
c_{n} &= \dfrac{2 \, n + 1}{2} \, V_{0} \left[ 1 - (-)^{n} \right] \, \int_{0}^{1} P_{n} \, \dd x \\
&= \dfrac{2 \, n + 1}{2} \, V_{0} \begin{dcases}
0 & \mbox{ si } n \mbox{ es par} \\
2 \, \int_{0}^{1} P_{2k+1} (x) \, \dd x& \mbox{ si } n = 2 \, k + 1
\end{dcases}
\end{align*}
donde se ha escrito el valor impar de $n = 2 \, k + 1$ para $k = 0, 1, \ldots$
\par
Quedando pendiente evaluar la integral en el intervalo $(0,1)$ con los polinomios de Legendre de orden impar, para ello, usaremos la fórmula de Rodrigues:
\begin{align*}
\int_{0}^{1} &P_{2k+1} (x) \, \dd x
 = \dfrac{1}{2^{2k+1} \, (2k+1)!} \int_{0}^{1} \dv[2k+1]{x} \left[ (x^{2} - 1)^{2k+1} \right] \, \dd x \\[0.5em]
&= \dfrac{1}{2^{2k+1} \, (2k+1)!} \, \dv[2k+1]{x} \left[ (x^{2} - 1)^{2k+1} \right] \eval_{0}^{1} \\[0.5em]
&= \dfrac{1}{2^{2k+1} (2k+1)!} \left\{ \dv[2k+1]{x} \left[ (x^{2} - 1)^{2k+1} \right] \eval_{x=1} + \right. \\[0.5em]
&- \left. \dv[2k+1]{x} \left[ (x^{2} - 1)^{2k+1} \right] \eval_{x=0} \right\}
\end{align*}
El primer término da cero porque no hay un número suficiente de diferenciaciones para deshacerse de todos los factores de $(x^{2} - 1)$. Para el segundo término, notamos que $(x^{2} - 1)^{2k+1}$ es un polinomio en $x$ cuyas derivadas de varios órdenes consisten en potencias de $x$. Estas potencias darán cero en $x = 0$, excepto por el término constante (de potencia cero). Entonces, usemos la expansión binomial para $(x^{2} - 1)^{2k+1}$ que es igual a $-(1 - x^{2})^{2k+1}$:
\begin{align*}
\dv[2k]{x} \left[ (x^{2} - 1)^{2k+1} \right] \eval_{x=0} &= - \dv[2k]{x} \left[ \sum_{j=0}^{2k+1} \dfrac{(2 \, k + 1)!}{j! \, (2 \, k + 1 - j)!} \, (-x^{2})^{j} \right] \eval_{x=0} \\[0.5em]
&= - \sum_{j=0}^{2k+1} \dfrac{(2 \, k + 1)!}{j! \, (2 \, k + 1 - j)!} \, (-)^{j} \, \dv[2k]{x} \left( x^{2j} \right) \eval_{x=0}
\end{align*}
cuyo término constante se obtiene cuando $k = j$, todos los demás términos de la suma se anulan, debido a que hay demasiadas diferenciaciones (cuando $j < k$, terminamos diferenciando constantes) o muy pocas diferenciaciones (cuando $j > k$, una potencia de $x$ se mantendrá, y cuando se evalúa $x = 0$ se anula). Por lo tanto,
\begin{align*}
\dv[2k]{x} \left[ (x^{2} - 1)^{2k+1} \right] \eval_{x=0} &= - \dfrac{(2 \, k + 1)!}{k! \, (k + 1)!} \, (-)^{k} \, \dv[2k]{x} \left( x^{2k} \right) \eval_{x=0} \\
&= \dfrac{(2 \, k + 1)!}{k! \, (k + 1)!} \, (-)^{k+1} \, (2 \, k )!
\end{align*}
entonces la integral
\begin{align}
\begin{aligned}
\int_{0}^{1} P_{2k+1} (x) \, \dd x &= - \dfrac{1}{2^{2k+1} \, (2 \, k + 1)!} \, \left[ \dfrac{(2 \, k + 1)!}{k! \, (k + 1)!} (-)^{k+1} \, (2 \, k)! \right] = \\
&= \dfrac{(-)^{k} \, (2 \, k )!}{2^{2k+1} \, k! \, (k+1)!}
\end{aligned}
\label{eq:ecuacion_26_49}
\end{align}
Con esto, ya podemos escribir los coeficientes $c_{2k+1}$ como
\begin{align*}
c_{2k+1} &=  2 \, \dfrac{2 (2 \, k + 1) + 1}{2} \, V_{0} \, \int_{0}^{1} P_{2k+1} (x) \, \dd x = \\
&= \dfrac{(-)^{k} \, (4 \, k + 3)(2 \, k)!}{2^{2k+1} \,k! \, (k + 1)!} \, V_{0}
\end{align*}
con $c_{n}=0$ para valores pares de $n$.
\par
La expansión por series queda entonces como:
\begin{align*}
f(x) &= \begin{cases}
V_{0} & \mbox{ si } 0 < x \leq 1 \\
-V_{0} & \mbox { si } -1 \leq x < 0
\end{cases} = \\
&= V_{0} \, \sum_{k=0}^{\infty} \dfrac{(-)^{k} \, (4 \, k + 3)(2 \, k)!}{2^{2k+1} \,k! \, (k + 1)!} \, P_{2k+1} (x) = \\
&= V_{0} \left[ \dfrac{3}{2} P_{1} (x) - \dfrac{7}{8} P_{3} (x) + \dfrac{11}{16} P_{5} (x) - \ldots \right]
\end{align*}
\section{Funciones de Legendre.}
Los problemas físicos más comunes que involucran la ecuación de Laplace son los de la electrostática en un espacio vacío y la transferencia de calor en estado estable. En cada caso, una superficie se mantiene a un potencial o temperatura (no necesariamente uniforme) y el potencial o la temperatura se busca en regiones alejadas de la superficie. En el presente contexto, estas superficies son típicamente (porciones de) esferas.
\subsection{Ejemplo 1. Semiesferas a diferentes temperaturas.}\label{sec:sub_seccion_01}
Dos hemisferios sólidos conductores de calor de radio $a$, separados por un pequeño espacio aislante, forman una esfera. Las dos mitades de la esfera están en contacto, en el exterior, con dos baños de calor (infinitos) a temperaturas $T_{0}$ y $-T_{0} $, ver la figura (\ref{fig:figura_esfera_01}).
\begin{figure}[H]
    \centering
    \includestandalone[scale=0.8]{Figuras/Ejemplo_Esfera_01}.
    \caption{Dos hemisferios a distintas temperaturas.}
    \label{fig:figura_esfera_01}
\end{figure}
Queremos encontrar la distribución de temperatura $T(r, \theta, \varphi)$ dentro de la esfera.
\par
Elegimos un sistema de coordenadas esféricas en el que el origen coincide con el centro de la esfera y el eje polar es perpendicular al plano ecuatorial. Se supone que el hemisferio con temperatura $T_{0}$ constituye el hemisferio norte.
\par
Dado que el problema tiene simetría azimutal, $T$ es independiente de $\varphi$, y podemos escribir inmediatamente la solución general con simetría azimutal de la ecuación Laplace en coordenadas esféricas: que multiplica la solución radial con la solución angular, para cada valor de $k$ y luego suma sobre todos los posibles valores de $k$:
\begin{equation}
\Phi = (r, \theta) = \sum_{k=0}^{\infty} \left( A_{k} \, r^{k} + \dfrac{B_{k}}{r^{k+1}} \right) \, P_{k} (\cos \theta)
\label{eq:ecuacion_26_29}
\end{equation}
Sin embargo, dado que el origen está en la región de interés, debemos excluir todos los potencias negativas de $r$. Esto se logra haciendo que todos los coeficientes $B$ sean iguales a cero. Así, tenemos
\begin{equation}
T(r, \theta) = \sum_{n=0}^{\infty} A_{n} \, r^{n} \, P_{n} (\cos \theta)
\label{eq:ecuacion_26_50}
\end{equation}
Quedando por calcular los coeficientes $A_{n}$.
\par
Para resolver esta parte, revisemos que
\begin{align*}
T (a, \theta) = \begin{cases}
T_{0} & \mbox{ si } 0 \leq \theta < \pi / 2 \\
-T_{0} & \mbox{ si } \dfrac{\pi}{2} < \theta \leq \pi
\end{cases}
\end{align*}
En términos de $u = \cos \theta$, queda expresado por
\begin{align*}
T (a, u) = \begin{cases}
-T_{0} & \mbox{ si } -1 \leq u < 0 \\
T_{0} & \mbox{ si } 0 < u \leq 1
\end{cases}
\end{align*}
Sustituyendo este resultado en la ecuación (\ref{eq:ecuacion_26_50}), se obtiene
\begin{align*}
T (a, \theta) = \begin{cases}
- T_{0}  & \mbox{ si } -1 \leq u < 0 \\
T_{0} & \mbox{ si } 0 < u \leq 1 
\end{cases}
\hspace{0.5cm} = \sum_{n=0}^{\infty} A_{n} \, a^{n} \, P_{n}(u)
\label{eq:ecuacion_26_51}
\end{align*}
Haciendo una expansión de $c_{n} = A_{n} \, a^{n}$ se encuentra que los coeficientes pares están ausentes y por tanto
\[ c_{2k+1} \equiv A_{2k+1} \, a^{2k+1} = \dfrac{(-)^{k} (4 \, k + 3)(2 \, k)!}{2^{2k+1} \, k! \, (k+1)!} \, T_{0} \]
Despejando $A_{2k+1}$ de la ecuación anterior, para utilizarlo dentro de la ec. (\ref{eq:ecuacion_26_50}) nos lleva a:
\begin{equation}
T(r, \theta) = T_{0} \, \sum_{k=0}^{\infty} \dfrac{(-)^{k} (4 \, k + 3)(2 \, k)!}{2^{2k+1} \, k! \, (k+1)!} \, \left( \dfrac{r}{a} \right)^{2k+1} \, P_{2k+1} (\cos \theta)
\label{eq:ecuacion_26_52}
\end{equation}
donde se ha sustituido $\cos \theta$ por $u$.
\subsection{Ejercicio 2. Semiesferas a distintos potenciales.}
Considere dos hemisferios de radio conductores de electricidad separados por un pequeño espacio de aislamiento en el ecuador. El hemisferio superior se mantiene al potencial $V_{0}$ y el inferior a $-V_{0}$ como se muestra en la figura (\ref{fig:figura_esfera_02}).
\begin{figure}[H]
    \centering
    \includestandalone{Figuras/Ejemplo_Esfera_03}
    \caption{Dos semiesferas con distintos potenciales eléctricos.}
    \label{fig:figura_esfera_02}
\end{figure}
Queremos encontrar el potencial en puntos fuera de la esfera completa. Como el potencial debe desvanecerse en el infinito, esperamos que el primer término en la ec. (\ref{eq:ecuacion_26_29}) esté ausente, es decir, $A_{k} = 0$. Para encontrar $B_{k}$, sustituimos $a$ por $r$ en la ec. (\ref{eq:ecuacion_26_29}), y dejamos $\cos \theta \equiv u$. Entonces:
\begin{align*}
\Phi (a, u) = \sum_{k=0}^{\infty} \dfrac{B_{k}}{a^{k+1}} \, P_{k} (u)
\end{align*}
en donde
\[ c_{k} \equiv \dfrac{B_{k}}{a^{k+1}} \]
y además
\begin{align*}
\Phi (a, u) = \begin{cases}
-V_{0} & \mbox{ si } -1 < u < 0 \\
+V_{0} & \mbox{ si } 0 < u < 1
\end{cases}
\end{align*}
El cálculo de los coeficientes se realiza de la misma manera que en el \nameref{sec:sub_seccion_01}.
\par
Por lo que $c_{k} = 0$ para valores pares de $k$ y
\[ c_{2m+1} = \dfrac{B_{2m+1}}{a^{2m+2}} = (-)^{m} \, \dfrac{(4 \, m + 3)(2 \, m)!}{2^{2m+1} \, (m + 1)! \, m!} \, V_{0} \]
o equivalentemente
\[ B_{2m+1} = \dfrac{(-)^{m} \, (4 \, m + 3)(2 \, m)!}{2^{2m+1} \, (m + 1)! \, m!} \, a^{2m+2} \, V_{0} \]
Una vez hallados los coeficientes, escribimos el potencial:
\begin{equation}
\setlength{\fboxsep}{3\fboxsep}\boxed{\Phi (r, \theta) = V_{0} \, \sum_{m=0}^{\infty} (-)^{m} \, \dfrac{(4 \, m + 3)(2 \, m)!}{2^{2m+1} \, (m + 1)! \, m!} \, \left( \dfrac{a}{r} \right)^{2m+2} \, P_{2m+1} (\cos \theta)}
\label{eq:ecuacion_26_53}
\end{equation}

\end{document}