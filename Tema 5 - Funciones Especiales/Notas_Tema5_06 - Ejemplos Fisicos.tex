\documentclass[12pt]{article}
\usepackage[utf8]{inputenc}
\usepackage[spanish,es-lcroman, es-tabla]{babel}
\usepackage[autostyle,spanish=mexican]{csquotes}
\usepackage{amsmath}
\usepackage{amssymb}
\usepackage{nccmath}
\numberwithin{equation}{section}
\usepackage{amsthm}
\usepackage{graphicx}
\usepackage{epstopdf}
\DeclareGraphicsExtensions{.pdf,.png,.jpg,.eps}
\usepackage{color}
\usepackage{float}
\usepackage{multicol}
\usepackage{enumerate}
\usepackage[shortlabels]{enumitem}
\usepackage{anyfontsize}
\usepackage{anysize}
\usepackage{array}
\usepackage{multirow}
\usepackage{enumitem}
\usepackage{cancel}
\usepackage{tikz}
\usepackage{circuitikz}
\usepackage{tikz-3dplot}
\usetikzlibrary{babel}
\usepackage{bm}
\usepackage{mathtools}
\usepackage{esvect}
\usepackage{hyperref}
\usepackage{relsize}
\usepackage{siunitx}
\usepackage{physics}
%\usepackage{biblatex}
\usepackage{standalone}
\usepackage{mathrsfs}
\usepackage{bigints}
\usepackage{bookmark}
\spanishdecimal{.}

\setlist[enumerate]{itemsep=0mm}

\renewcommand{\baselinestretch}{1.5}

\let\oldbibliography\thebibliography

\renewcommand{\thebibliography}[1]{\oldbibliography{#1}

\setlength{\itemsep}{0pt}}
%\marginsize{1.5cm}{1.5cm}{2cm}{2cm}


\newtheorem{defi}{{\it Definición}}[section]
\newtheorem{teo}{{\it Teorema}}[section]
\newtheorem{ejemplo}{{\it Ejemplo}}[section]
\newtheorem{propiedad}{{\it Propiedad}}[section]
\newtheorem{lema}{{\it Lema}}[section]

\title{Ejemplos Físicos \\ {\large Matemáticas Avanzadas de la Física}}
\date{ }
\begin{document}
\maketitle
\fontsize{14}{14}\selectfont
\section{Funciones de Legendre.}
Los problemas físicos más comunes que involucran la ecuación de Laplace son los de la electrostática en un espacio vacío y la transferencia de calor en estado estable. En cada caso, una superficie se mantiene a un potencial o temperatura (no necesariamente uniforme) y el potencial o la temperatura se busca en regiones alejadas de la superficie. En el presente contexto, estas superficies son típicamente (porciones de) esferas.
\subsection{Ejemplo 1. Semiesferas a diferentes temperaturas.}
Dos hemisferios sólidos conductores de calor de radio $a$, separados por un pequeño espacio aislante, forman una esfera. Las dos mitades de la esfera están en contacto, en el exterior, con dos baños de calor (infinitos) a temperaturas $T_{0}$ y $-T_{0} $, ver la figura (\ref{fig:figura_esfera_01}).
\begin{figure}[H]
    \centering
    \includestandalone[scale=0.8]{Figuras/Ejemplo_Esfera_01}.
    \caption{Dos hemisferios a distintas temperaturas.}
    \label{fig:figura_esfera_01}
\end{figure}
Queremos encontrar la distribución de temperatura $T(r, \theta, \varphi)$ dentro de la esfera.
\par
Elegimos un sistema de coordenadas esféricas en el que el origen coincide con el centro de la esfera y el eje polar es perpendicular al plano ecuatorial. Se supone que el hemisferio con temperatura $T_{0}$ constituye el hemisferio norte.
\par
Dado que el problema tiene simetría azimutal, $T$ es independiente de $\varphi$, y podemos escribir inmediatamente la solución general con simetría azimutal de la ecuación Laplace en coordenadas esféricas: que multiplica la solución radial con la solución angular, para cada valor de $k$ y luego suma sobre todos los posibles valores de $k$:
\begin{equation}
\Phi = (r, \theta) = \sum_{k=0}^{\infty} \left( A_{k} \, r^{k} + \dfrac{B_{k}}{r^{k+1}} \right) \, P_{k} (\cos \theta)
\label{eq:ecuacion_26_29}
\end{equation}
Sin embargo, dado que el origen está en la región de interés, debemos excluir todos los potencias negativas de $r$. Esto se logra haciendo que todos los coeficientes $B$ sean iguales a cero. Así, tenemos
\begin{equation}
T(r, \theta) = \sum_{n=0}^{\infty} A_{n} \, r^{n} \, P_{n} (\cos \theta)
\label{eq:ecuacion_26_50}
\end{equation}
Quedando por calcular los coeficientes $A_{n}$.
\par
Para resolver esta parte, revisemos que
\begin{align*}
T (a, \theta) = \begin{cases}
T_{0} & \mbox{ si } 0 \leq \theta < \pi / 2 \\
-T_{0} & \mbox{ si } \dfrac{\pi}{2} < \theta \leq \pi
\end{cases}
\end{align*}
En términos de $u = \cos \theta$, queda expresado por
\begin{align*}
T (a, u) = \begin{cases}
-T_{0} & \mbox{ si } -1 \leq u < 0 \\
T_{0} & \mbox{ si } 0 < u \leq 1
\end{cases}
\end{align*}
Sustituyendo este resultado en la ecuación (\ref{eq:ecuacion_26_50}), se obtiene
\begin{align*}
T (a, \theta) = \begin{cases}
- T_{0}  & \mbox{ si } -1 \leq u < 0 \\
T_{0} & \mbox{ si } 0 < u \leq 1 
\end{cases}
\hspace{0.5cm} = \sum_{n=0}^{\infty} A_{n} \, a^{n} \, P_{n}(u)
\label{eq:ecuacion_26_51}
\end{align*}
Haciendo una expansión de $c_{n} = A_{n} \, a^{n}$ se encuentra que los coeficientes pares están ausentes y por tanto
\[ c_{2k+1} \equiv A_{2k+1} \, a^{2k+1} = \dfrac{(-)^{k} (4 \, k + 3)(2 \, k)!}{2^{2k+1} \, k! \, (k+1)!} \, T_{0} \]
Despejando $A_{2k+1}$ de la ecuación anterior, para utilizarlo dentro de la ec. (\ref{eq:ecuacion_26_50}) nos lleva a:
\begin{equation}
T(r, \theta) = T_{0} \, \sum_{k=0}^{\infty} \dfrac{(-)^{k} (4 \, k + 3)(2 \, k)!}{2^{2k+1} \, k! \, (k+1)!} \, \left( \dfrac{r}{a} \right)^{2k+1} \, P_{2k+1} (\cos \theta)
\label{eq:ecuacion_26_52}
\end{equation}
donde se ha sustituido $\cos \theta$ por $u$.
\end{document}