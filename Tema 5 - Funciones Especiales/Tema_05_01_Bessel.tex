\documentclass[12pt]{beamer}
\usepackage{../Estilos/BeamerMAF}
\usepackage{../Estilos/ColoresLatex}
\usepackage[absolute, overlay]{textpos}

\usetheme{Copenhagen}
\usecolortheme{wolverine}
%\useoutertheme{default}
\setbeamercovered{invisible}
% or whatever (possibly just delete it)
\setbeamertemplate{section in toc}[sections numbered]
\setbeamertemplate{subsection in toc}[subsections numbered]
\setbeamertemplate{subsection in toc}{\leavevmode\leftskip=3.2em\rlap{\hskip-2em\inserttocsectionnumber.\inserttocsubsectionnumber}\inserttocsubsection\par}
% \setbeamercolor{section in toc}{fg=blue}
% \setbeamercolor{subsection in toc}{fg=blue}
% \setbeamercolor{frametitle}{fg=blue}
\setbeamertemplate{caption}[numbered]

\setbeamertemplate{footline}
\beamertemplatenavigationsymbolsempty
\setbeamertemplate{headline}{}


\makeatletter
% \setbeamercolor{section in foot}{bg=gray!30, fg=black!90!orange}
% \setbeamercolor{subsection in foot}{bg=blue!30}
% \setbeamercolor{date in foot}{bg=black}
\setbeamertemplate{footline}
{
  \leavevmode%
  \hbox{%
  \begin{beamercolorbox}[wd=.333333\paperwidth,ht=2.25ex,dp=1ex,center]{section in foot}%
    \usebeamerfont{section in foot} \insertsection
  \end{beamercolorbox}%
  \begin{beamercolorbox}[wd=.333333\paperwidth,ht=2.25ex,dp=1ex,center]{subsection in foot}%
    \usebeamerfont{subsection in foot}  \insertsubsection
  \end{beamercolorbox}%
  \begin{beamercolorbox}[wd=.333333\paperwidth,ht=2.25ex,dp=1ex,right]{date in head/foot}%
    \usebeamerfont{date in head/foot} \insertshortdate{} \hspace*{2em}
    \insertframenumber{} / \inserttotalframenumber \hspace*{2ex} 
  \end{beamercolorbox}}%
  \vskip0pt%
}
\makeatother

\makeatletter
\patchcmd{\beamer@sectionintoc}{\vskip1.5em}{\vskip0.8em}{}{}
\makeatother

% %\newlength{\depthofsumsign}
% \setlength{\depthofsumsign}{\depthof{$\sum$}}
% \newcommand{\nsum}[1][1.4]{% only for \displaystyle
%     \mathop{%
%         \raisebox
%             {-#1\depthofsumsign+1\depthofsumsign}
%             {\scalebox
%                 {#1}
%                 {$\displaystyle\sum$}%
%             }
%     }
% }
% \def\scaleint#1{\vcenter{\hbox{\scaleto[3ex]{\displaystyle\int}{#1}}}}
% \def\scaleoint#1{\vcenter{\hbox{\scaleto[3ex]{\displaystyle\oint}{#1}}}}
% \def\bs{\mkern-12mu}


\makeatletter
\setbeamertemplate{footline}
{
\leavevmode%
\hbox{%
\begin{beamercolorbox}[wd=.333333\paperwidth,ht=2.25ex,dp=1ex,center]{section in foot}%
  \usebeamerfont{section in foot} \insertsection
\end{beamercolorbox}%
\begin{beamercolorbox}[wd=.333333\paperwidth,ht=2.25ex,dp=1ex,center]{subsection in foot}%
  \usebeamerfont{subsection in foot}  \insertsubsection
\end{beamercolorbox}%
\begin{beamercolorbox}[wd=.333333\paperwidth,ht=2.25ex,dp=1ex,right]{date in head/foot}%
  \usebeamerfont{date in head/foot} \insertshortdate{} \hspace*{1.5em}
  \insertframenumber{} / \inserttotalframenumber \hspace*{2ex} 
\end{beamercolorbox}}%
\vskip0pt%
}
\makeatother
% \usefonttheme{serif}
\setbeamercolor{frametitle}{bg=palecerulean}
\resetcounteronoverlays{saveenumi}

\AtBeginDocument{\RenewCommandCopy\qty\SI}
\ExplSyntaxOn
\msg_redirect_name:nnn { siunitx } { physics-pkg } { none }
\ExplSyntaxOff

\date{}

\title{\large{Funciones de Bessel}}
\subtitle{Tema 4 - Funciones Especiales}
\author{M. en C. Gustavo Contreras Mayén}

\resetcounteronoverlays{saveenumi}

\begin{document}
\maketitle
\fontsize{14}{14}\selectfont
\spanishdecimal{.}

\section*{Contenido}
\frame[allowframebreaks]{\frametitle{Contenido} \tableofcontents[currentsection, hideallsubsections]}

\section{Iniciando el estudio}
\frame[allowframebreaks]{\frametitle{Temas a revisar} \tableofcontents[currentsection, hideothersubsections]}
\subsection{Introducción}

%Sadri Hassani - Mathematical methods for students of physics. Chap. 27 Laplace equation: cylindrical coordinates
\begin{frame} 
\frametitle{Consideración}
Antes de desarrollar un problema con una geometría cilíndrica, consideremos una pregunta que tiene implicaciones más generales.
\end{frame}
\begin{frame}
\frametitle{Ocupando lo ya revisado}
Vimos en el Tema 2 que la separación de variables condujo a un sistema de EDO en las que aparecían ciertas constantes de separación, y que elegir el signo de esa constante nos lleva a una forma funcional diferente de la solución general.
\end{frame}
\begin{frame}
\frametitle{Problema tipo}
Por ejemplo, para una ecuación como:
\pause
\begin{align*}
\dv[2]{x}{t} - k \, x = 0
\end{align*}
\pause
se pueden tener soluciones exponenciales si $k > 0$ y soluciones trigonométricas si $k < 0$. \pause Uno no puede asignar a priori un signo específico a $k$.
\end{frame}
\begin{frame}
\frametitle{Solución indeterminada}
Por lo tanto, la forma general de la solución es indeterminada. 
\\
\bigskip
\pause
Sin embargo, \textocolor{cobalt}{una vez que se imponen las CDF}, las soluciones únicas surgirán independientemente de la forma funcional inicial de las soluciones.
\end{frame}
\begin{frame}
\frametitle{Uso de las CDF}
El siguiente argumento ilustra este punto en la ED angular resultante de la separación de variables para la \textocolor{red}{ecuación de Laplace} en \textocolor{ao}{coordenadas cilíndricas}.
\end{frame}


\section{Funciones de Bessel}
\frame{\tableofcontents[currentsection, hideothersubsections]}
\subsection{El problema de estudio}

\begin{frame}
\frametitle{Ecuación de Laplace}
La separación de variables en coordenadas cilíndricas para la ecuación de Laplace:
\pause
\begin{align*}
\laplacian{\Phi} (\rho, \varphi, z) = 0
\end{align*}
\end{frame}
\begin{frame}
\frametitle{Ecuación de Laplace}    
Mediante la solución propuesta:
\begin{align*}
\Phi (\rho, \varphi, z) = R (\rho) \, S (\varphi) \, Z (z)
\end{align*}
nos lleva a un sistema de tres EDO:
\end{frame}
\begin{frame}
\frametitle{Sistema de EDO}
\begin{eqnarray}
\begin{aligned}
\dv{\rho} \left( \rho \dv{R}{\rho} \right) + \left( \lambda \, \rho + \dfrac{\mu}{\rho} \right) \, R &= 0 \label{eq:ecuacion_27_01a} \\[1em] \pause
\dv[2]{S}{\varphi} - \mu \, S &= 0 \label{eq:ecuacion_27_01b}\\[1em] \pause
\dv[2]{Z}{z} - \lambda \, Z &= 0 \label{eq:ecuacion_27_01c}
\end{aligned}
\end{eqnarray}
\end{frame}
\begin{frame}
\frametitle{Ecuación en particular}
Si nos fijamos en la segunda ecuación (\ref{eq:ecuacion_27_01b}) cuya solución más general podemos escribir como:
\pause
\begin{align}
S (\varphi) = \begin{cases}
A \, \exp(\sqrt{\mu} \, \varphi) {+} B \, \exp(-\sqrt{\mu} \, \varphi) & \mbox{ si } \mu \neq 0 \\
C \,\varphi + D & \mbox { si } \mu = 0
\end{cases}
\label{eq:ecuacion_27_02}
\end{align}
\end{frame}
\begin{frame}
\frametitle{Solución a la ecuación}
No importa que tipo de CDF se impongan al potencial $\Phi$, \pause ya que debe de devolver el mismo valor en $\varphi$ y en $\varphi + 2 \pi$, mientras que se mantengan las otras dos variables fijas.
\end{frame}
\begin{frame}
\frametitle{Validez del argumento}
Este argumento es válido solo para los casos físicos definidos para todo el rango de $\varphi$. 
\\
\bigskip
\pause
Si la región de interés restringe a $\varphi$ en un subconjunto del intervalo $[0, 2 \pi]$, el argumento ya no funciona.
\end{frame}
\begin{frame}
\frametitle{Validez del argumento}
Esto es debido a que $(\rho, \varphi, z)$ y $(\rho, \varphi + 2 \pi, z)$ representan físicamente al mismo punto en el espacio.
\end{frame}
\begin{frame}
\frametitle{Resultado}
Se sigue entonces que:
\pause
\begin{eqnarray*}
&R(\rho) \, S(\varphi) \, Z(z) = \pause R(\rho)  \, S(\varphi + 2 \pi) , Z(z) \\[0.5em] \pause
&\Longrightarrow S(\varphi + 2 \pi) = \pause S (\varphi)
\end{eqnarray*}
ya que la identidad se mantiene para todos los valores de $\rho$ y $z$.
\end{frame}
\begin{frame}
\frametitle{Resultado}
Si la última relación es verdadera para el caso de $\mu = 0$, \pause entonces tenemos que $C = 0$ y $S(\varphi) = D$.
\\
\bigskip
\pause
Para $\mu \neq 0$, la ec. (\ref{eq:ecuacion_27_02}) es:
\pause
\begin{align*}
S (\varphi) &= A \exp(\sqrt{\mu}(\varphi {+} 2 \pi)) {+} B \exp(- \sqrt{\mu}(\varphi {+} 2 \pi)) = \\[0.5em] \pause
&= A \exp(\sqrt{\mu} \varphi) + B \exp(-\sqrt{\mu} \varphi)
\end{align*}
\end{frame}
\begin{frame}
\frametitle{Versión alterna}
O también:
\pause
\begin{align*}
&S (\varphi) = A \, \exp(\sqrt{\mu} \, \varphi) \, (\exp(\sqrt{\mu} \, 2 \pi) - 1) + \\[0.5em]
&+ B \, \exp(- \sqrt{\mu} \, \varphi) \, (\exp(- \sqrt{\mu} \, 2 \pi) - 1) = 0
\end{align*}
\pause
que debe de cumplirse para todo $\varphi$.
\end{frame}
\begin{frame}
\frametitle{Cumplimiento de la condición}
La única manera en la que esto puede ocurrir (cuidando de que $A$ y $B$ no sean nulos) es mediante:
\pause
\begin{align*}
\exp(\sqrt{\mu} \, 2 \pi) - 1 = 0 \hspace{0.7cm} \mbox{y} \hspace{0.7cm} \exp(- \sqrt{\mu} \, 2 \pi) - 1 = 0
\end{align*}
\pause
En ambos casos se tiene que $\exp(\sqrt{\mu} \, 2 \, \pi) = 1$.
\end{frame}
\begin{frame}
\frametitle{Tomando valores reales}
Si nos limitamos a valores reales de $\mu$, obtendremos soluciones triviales.
\\
\bigskip
\pause
Para prevenir esto, tenemos que hacer:
\pause
\begin{align*}
\sqrt{\mu} = i \, m \hspace{1cm} m = 0, \pm 1, \pm 2, \ldots
\end{align*}
\end{frame}
\begin{frame}
\frametitle{De manera equivalente}
\begin{align*}
\mu = -m^{2}  \hspace{1cm} m = 0, \pm 1, \pm 2, \ldots
\end{align*}
\pause
Con esta elección de $\mu$, la ED para $S(\varphi)$ es:
\pause
\begin{align*}
\sderivada{S} + m^{2} \, S = 0
\end{align*}
\pause
que tiene como solución general: una suma de funciones trigonométricas.
\end{frame}
\begin{frame}
\frametitle{Teorema importante}
Para todos los problemas físicos para los cuales el ángulo azimutal varíe entre $0$ y $2 \pi$, uno está forzado a restringir el valor de $\mu$ al negativo de la raíz cuadrada de un entero.
\end{frame}
\begin{frame}
\frametitle{Teorema importante}
La solución para la parte angular es entonces:
\pause
\begin{equation}
\begin{aligned}
S (\varphi) = A_{m} \, \cos m \varphi + B_{m} \, \sin m \varphi, \\[0.5em]
m = 0, 1, 2, \ldots
\end{aligned}
\label{eq:ecuacion_27_03}
\end{equation}
donde $A_{m}$ y $B_{m}$ son constantes que pueden ser distintas para diferentes valores de $m$.
\end{frame}
\begin{frame}
\frametitle{Valores negativos}
Los valores negativos de $m$ no darán lugar a nuevas soluciones, por lo que no se incluyen en el rango de $m$.
\\
\bigskip
El caso de $\mu = 0$ no se necesita tratar por separado, \pause ya que la solución aceptable para este caso es $S = D = \mbox{ constante}$, la cual es la que se obtiene en la ec. (\ref{eq:ecuacion_27_03}) cuando $m = 0$.
\end{frame}
\begin{frame}
\frametitle{Solución para $Z (z)$}
La ED para $Z (z)$ es independiente de $m$ \pause y tiene una solución exponencial si $\lambda > 0$ y una solución trigonométrica si $\lambda < 0$.
\\
\bigskip
\pause
Asumiendo ésta forma y escribiendo $\lambda \equiv l^{2}$, se tiene:
\pause
\begin{equation}
Z (z) = A \, e^{l \, z} + B \, e^{- l \, z}
\label{eq:ecuacion_27_04}
\end{equation}
\end{frame}
\begin{frame}
\frametitle{Solución radial}
La más familiar de las ED es la ecuación radial, \pause en términos de $l = \sqrt{\lambda}$, se puede escribir como:
\pause
\begin{equation}
\dv[2]{R}{\rho} + \dfrac{1}{\rho} \, \dv{R}{\rho} + \left( l^{2} - \dfrac{m^{2}}{\rho^{2}} \right) \, R = 0
\label{eq:ecuacion_27_05}
\end{equation}
\end{frame}
\begin{frame}
\frametitle{Cambiando la variable}
Además, si definimos la variable $v = l\, \rho$, podemos transformar la ec. (\ref{eq:ecuacion_27_05}) en la forma:
\pause
\begin{equation}
\dv[2]{R}{\rho} + \dfrac{1}{v} \, \dv{R}{v} + \left( 1 - \dfrac{m^{2}}{v^{2}} \right) \, R = 0
\label{eq:ecuacion_27_06}
\end{equation}
\end{frame}
\begin{frame}
\frametitle{Ecuación Diferencial de Bessel}
Tanto la ec. (\ref{eq:ecuacion_27_05}) o (\ref{eq:ecuacion_27_06}), es una de las ED de la física matemática más famosas: \textocolor{carmine}{la ecuación diferencial de Bessel}.
\end{frame}

\section{Soluciones para la ED de Bessel}
\frame{\tableofcontents[currentsection, hideothersubsections]}
\subsection{Solución en series}

\begin{frame}
\frametitle{Usando el método de Frobenius}
El método de Frobenius es una manera efectiva para encontrar las soluciones de las EDO.
\end{frame}
\begin{frame}
\frametitle{Usando el método de Frobenius}
Al reescribir la ec. (\ref{eq:ecuacion_27_06}) multiplicando por $v^{2}$ para convertir todos sus coeficientes en polinomios como lo sugiere la ecuación (\ref{eq:ecuacion_26_07}):
\pause
\begin{equation}
p_{2}(x) \, \dv[2]{y}{x} + p_{1}(x) \, \dv{y}{x} + p_{0} (x) \, y
 = 0
\label{eq:ecuacion_26_07}
\end{equation}
\end{frame}
\begin{frame}
\frametitle{Manejando la expresión}
Esto nos lleva a:
\pause
\begin{equation}
v^{2} \, \dv[2]{R}{v} + v \dv{R}{v} + (v^{2} - m^{2}) \, R = 0
\label{eq:ecuacion_27_07}
\end{equation}
\end{frame}
\begin{frame}
\frametitle{Manejando la expresión}    
Como $v^{2}$ se anula cuando $v = 0$, podemos proponer una solución de la forma:
\pause
\begin{eqnarray*}
R(v) &= v^{s} \displaystyle \nsum_{k=0}^{\infty} c_{k} \, v^{k} = \\[0.5em] \pause
&= \displaystyle  \nsum_{k=0}^{\infty} c_{k} \, v^{k+s}
\end{eqnarray*}
\end{frame}
\begin{frame}
\frametitle{Diferenciando la expresión}
De la cual obtenemos al diferenciar en dos ocasiones:
\pause
\begin{eqnarray*}
\begin{aligned}
v \, \dv{R}{v} &= \displaystyle  \nsum_{k=0}^{\infty} c_{k} (k + s) \, v^{k+s} \\[0.5em] \pause
v^{2} \, \dv[2]{R}{v} &= \displaystyle  \nsum_{k=0}^{\infty} c_{k} (k + s)(k + s - 1) \, v^{k+s} 
\end{aligned}
\end{eqnarray*}
\end{frame}
% \begin{frame}
% \frametitle{Sustituyendo en la expresión}
% Sustituyendo los términos así como:
% \pause
% \begin{align*}
% (v^{2} - m^{2}) \nsum_{k=0}^{\infty} c_{k} \, v^{k+s}
% \end{align*}
% \pause
% en la ED inicial, nos conduce a:
% \end{frame}
% \begin{frame}
% \frametitle{Sustituyendo en la expresión}
% \begin{align*}
% &\nsum_{k=0}^{\infty} c_{k} [\underbrace{k + s + (k + s)(k + s - 1)}_{=(k+s)^{2}} - m^{2} ] \, v^{k+s} + \\[0.5em] 
% &+ \nsum_{k=0}^{\infty} c_{k} \, v^{k+s+2} = 0
% \end{align*}
% \end{frame}
% \begin{frame}
% \frametitle{Relación de recurrencia}
% Para encontrar la relación de recurrencia, necesitamos que el valor de la potencia de $v$ sea el mismo en las sumas, por lo que reescribimos la primera suma como:
% \end{frame}
% \begin{frame}
% \frametitle{Relación de recurrencia}
% \begin{eqnarray*}
% \begin{aligned}
% c_{0}(s^{2} &- m^{2}) v^{s} + c_{1} [(s + 1)^{2}- m^{2}] v^{s+1} + \\
% &\nsum_{k=2}^{\infty} c_{k} [(k + s)^{2} - m^{2}] v^{k+s} = \\ \pause
% &= c_{0}(s^{2} - m^{2}) v^{s} + c_{1} [(s + 1)^{2}- m^{2}] v^{s+1} + \\
% &+ \nsum_{n=0}^{\infty} c_{n+2} [(n + 2 + s)^{2} - m^{2}] v^{n+2+s}
% \end{aligned}
% \end{eqnarray*}
% donde en la segunda línea, usamos $n = k -2$.
% \end{frame}
% \begin{frame}
% \frametitle{Relación de recurrencia}    
% Como el índice $n$ es mudo, lo podemos regresar de nuevo al índice $k$.
% \\
% \bigskip
% \pause
% Entonces, seguimos con que:
% \pause
% \begin{align*}
% c_{0}(s^{2} &{-} m^{2}) v^{s} {+} c_{1} [(s {+} 1)^{2} {-} m^{2}] v^{s+1} + \\
% &+ \nsum_{k=0}^{\infty} \left\{ c_{k+2} [(k {+} 2 {+} s)^{2} {-} m^{2}] {+} c_{k} \right\} \, v^{k+2+s} = 0
% \end{align*}
% \end{frame}
% \begin{frame}
% \frametitle{Relación de recurrencia}
% Suponiendo que $c_{0} \neq 0$ y que todos los demás coeficientes de las potencias de $v$ se anulan, obtenemos:
% \pause
% \begin{eqnarray*}
% \begin{aligned}
% s^{2} &= m^{2} \\ \pause 
% c_{1} [(s + 1)^{2}- m^{2}] &= 0\\ \pause
% c_{k+2} [(k + 2 + s)^{2} - m^{2}] + c_{k} &= 0
% \end{aligned}
% \end{eqnarray*}
% \end{frame}
% \begin{frame}
% \frametitle{Relación de recurrencia}
% La primera ecuación nos dice que $m = \pm s$.
% \\
% \bigskip
% \pause
% Por lo que al usarlo en la segunda ecuación, resulta en:
% \pause
% \begin{align*}
% c_{1} (2 \, s + 1) = 0 \hspace{0.5cm} \Rightarrow \hspace{0.5cm} c_{1} = 0 \hspace{0.5cm} \mbox{ o } \hspace{0.5cm} s = - \dfrac{1}{2}
% \end{align*}
% \end{frame}
% \begin{frame}
% \frametitle{Relación de recurrencia}
% La elección $s = -1/2$ nos devuelve que $m = \mp 1/2$ la cual no es aceptable, por lo que decidimos que $m$ sea un entero positivo.
% \\
% \bigskip
% \pause
% Por lo tanto concluimos que $s = \pm m$ y $c_{1} = 0$.
% \end{frame}
% \begin{frame}
% \frametitle{Sobre los valores no enteros}
% En realidad, los problemas que surgen de otras áreas de la física más allá de la electrostática y la transferencia de calor en estado estacionario permiten valores de $m$ no enteros.
% \\
% \bigskip
% \pause
% Sin embargo, no trataremos tales problemas en el curso.
% \end{frame}
% \begin{frame}
% \frametitle{Relación de recurrencia}
% Se sigue que la regla de recurrencia para todos los coeficientes impares son cero.
% \\
% \bigskip
% \pause
% La serie de Frobenius entonces será:
% \pause
% \begin{equation}
% \begin{aligned}
% R (v) &= v^{\pm m} \nsum_{k=0}^{\infty} c_{2k} \, v^{2k}, \\[0.5cm]
% \dfrac{c_{2k+2}}{c_{2k}} &= - \dfrac{1}{(2k + 2 + s)^{2} - m^{2}}
% \end{aligned}
% \label{eq:ecuacion_27_08}
% \end{equation}
% \end{frame}
% \begin{frame}
% \frametitle{Convergencia de la serie}
% La prueba del cociente para la convergencia de la serie nos lleva a:
% \pause
% \begin{align*}
% \lim_{k \to \infty} \abs{\dfrac{c_{2k+2} \, v^{2k+2}}{c_{2k} \, v^{2k}}} = \lim_{k \to \infty} \abs{\dfrac{1}{(2k + 2 + s)^{2} -m^{2}}} \, v^{2} = 0
% \end{align*}
% \pause
% Lo que nos indica que la serie de la ec. (\ref{eq:ecuacion_27_08}) es convergente para todos los valores de $v$.
% \end{frame}
% \begin{frame}
% \frametitle{Uso de la relación de recurrencia}
% Ahora usaremos la relación de recurrencia para obtener los coeficientes de la expansión.
% \end{frame}
% \begin{frame}
% \frametitle{Uso de la relación de recurrencia}
% Reescribimos la relación de recurrencia como:
% \pause
% \begin{eqnarray*}
% \begin{aligned}
% c_{k+2} &= - \dfrac{1}{(k + 2 + s)^{2} - s^{2}} \, c_{k} = \\ \pause
% &= \dfrac{1}{(k + 2)(2 s + k + 2)} \, c_{k}
% \end{aligned}
% \end{eqnarray*}
% donde hemos sustituido $s^{2}$ por $m^{2}$.
% \end{frame}
% \begin{frame}
% \frametitle{Obteniendo los términos}
% Lo que nos devuelve:
% \pause
% \begin{eqnarray*}
% \begin{aligned}
% c_{2} &{=} - \dfrac{1}{2 (2 s {+} 2)} c_{0} \\ \pause
% c_{4} &{=} - \dfrac{1}{4 (2 s {+} 4)} c_{2} = (-1)^{2} \dfrac{1}{4 (2 s {+} 4)} \, \dfrac{1}{2 (2 s {+} 2)} \, c_{0} \\ \pause
% c_{6} &{=} - \dfrac{1}{6 (2 s + 6)} c_{4} = (-1)^{3} \dfrac{1}{6 (2 s {+} 6)} \, \dfrac{1}{4 (2 s {+} 4)} \, \dfrac{1}{2 (2 s {+} 2)} \, c_{0} 
% \end{aligned}
% \end{eqnarray*}
% \end{frame}
% \begin{frame}
% \frametitle{Obteniendo los términos}
% De manera general:
% \pause
% \begin{align*}
% &c_{2k} = (-1)^{k} \times \\
% &\times \dfrac{c_{0}}{\underbrace{2 k \cdot (2 k {-} 2) \ldots 2}_{=2^{k} \, k!} \underbrace{(2 s {+} 2 k)[2 s {+} (2 k {-} 2)] \ldots (2 s {+} 2)}_{=2^{k} (s {+} k)(s {+} k {+} 1) \ldots (s {+} 1)}}
% \end{align*}
% \end{frame}
% \begin{frame}
% \frametitle{Obteniendo los términos}
% Multiplicando el numerador y el denominador por $s!$, llegamos a:
% \pause
% \begin{equation}
% c_{2k} = (-1)^{k} \dfrac{s!}{2^{2k} \, k! \, (s+k)!} \, c_{0}
% \label{eq:ecuacion_27_09}
% \end{equation}
% \end{frame}
% \begin{frame}
% \frametitle{La solución en series}
% Sustituyendo la ec. (\ref{eq:ecuacion_27_09}) en la (\ref{eq:ecuacion_27_08}), nos conduce a:
% \pause
% \begin{eqnarray*}
% R(v) &= c_{0} \, s! \, v^{s} \nsum_{k=0}^{\infty} \dfrac{(-1)^{k}}{2^{2k} \, k! \, (s + k)!} \, v^{2k} = \\ \pause
% &= c_{0} \, s! \, 2^{s} \, \left( \dfrac{v}{2} \right)^{s} \nsum_{k=0}^{\infty} \dfrac{(-1)^{k}}{k! \, (s + k)!} \, \left( \dfrac{v}{2} \right)^{2 k} 
% \end{eqnarray*}
% donde sustituimos $s$ por $\pm m$ en el exponente de $v$ fuera de la suma. 
% \end{frame}
% \begin{frame}
% \frametitle{La solución en series}
% También absorbimos las potencias de $2$ en el denominador de la suma en las potencias de $v$, y fuera de la suma, multiplicamos y dividimos por $2^{s}$.
% \\
% \bigskip
% \pause
% Es habitual elegir que la constante arbitraria $c_{0}$ sea igual a $1/(s! \, 2^{s})$.
% \end{frame}
\begin{frame}
\frametitle{La solución}
Esto nos lleva a:
\pause
\begin{tcolorbox}
Las \textbf{funciones de Bessel} de orden $s$, que se escribe como $J_{s}$ y está dada por la serie:
\pause
\begin{equation}
J_{s}(x) = \left( \dfrac{x}{2} \right)^{s} \nsum_{k=0}^{\infty} \dfrac{(-1)^{k}}{k!\, (s + k)!} \, \left( \dfrac{x}{2} \right)^{2k}
\label{eq:ecuacion_27_10}
\end{equation}
la cual es convergente para todos los valores de $x$.
\end{tcolorbox}
\end{frame}
\begin{frame}
\frametitle{Consideración sobre $m$}
Aunque la Ecuación (\ref{eq:ecuacion_27_10}) se obtuvo asumiendo que $m$, y por lo tanto $s$, era un número entero, \pause al levantar esta restricción se obtendría una serie que es convergente en todas partes, y uno puede definir funciones de Bessel cuyas órdenes son números reales o incluso complejos.
\end{frame}
\begin{frame}
\frametitle{Consideración sobre $m$}
La única dificultad es interpretar correctamente $(s + n)!$ para $s$ no enteros.
\\
\bigskip
\pause
Pero esto es precisamente para lo que se inventó la función Gamma $\Gamma (x)$. \pause Por lo tanto, dejaremos que la ecuación (\ref{eq:ecuacion_27_10}) represente las funciones de Bessel de todos los órdenes.
\end{frame}

\section{Segunda solución a la ED de Bessel}
\frame[allowframebreaks]{\frametitle{Temas a revisar} \tableofcontents[currentsection, hideothersubsections]}
\subsection{Obteniendo la 2a. solución}

\begin{frame}
\frametitle{Recuperando la 2a. solución}
Se puede obtener una segunda solución de la ED de Bessel, usando la ec. (\ref{eq:ecuacion_24_06}), donde $p(x) = 1/x$.\pause
\begin{eqnarray}
\begin{aligned}
f_{2}(x) &=  f_{1}(x) \bigg[ C + K \scaleint{6ex}_{\bs \alpha}^{x} \dfrac{1}{f_{1}^{2} (s)}  \times \\[0.5em]
&\times \exp( - \scaleint{6ex}_{\bs c}^{s} p(t) \dd{t}) \dd{s} \bigg]
\end{aligned}
\label{eq:ecuacion_24_06}
\end{eqnarray}
\end{frame}
\begin{frame}
\frametitle{Usando la primera solución}
Usando $J_{m}(x)$ como entrada, podemos generar la segunda solución; con $C = 0$ en la ec. (\ref{eq:ecuacion_24_06}), obtenemos:
\pause
\begin{eqnarray*}
\begin{aligned}
Z_{m}(x) &= K \, J_{m}(x) \, \scaleint{6ex}_{\bs \alpha}^{x} \dfrac{1}{J_{m}^{2}(x)} \exp(- \scaleint{6ex}_{\bs c}^{u} \dfrac{\dd{t}}{t}) \dd{u} = \\[0.5em] \pause
&= A_{m} \, J_{m}(x) \scaleint{6ex}_{\bs \alpha}^{x} \dfrac{\dd{u}}{u \, J_{m}^{2}(u)}
\end{aligned}
\end{eqnarray*}
\pause
donde $A_{m} \equiv K , c$ y $\alpha$ son constantes arbitrarias que se establecen por convención.
\end{frame}
\begin{frame}
\frametitle{Particularidad de la 2a. solución}
Notemos que, contrario a $J_{m}(x)$, la solución $Z_{m}(x)$ no se comporta bien en $x = 0$, debido a la presencia de $u$ en el denominador del integrando.
\\
\bigskip
\pause
Aunque el procedimiento anterior genera una segunda solución para la ED de Bessel, no es el procedimiento habitual.
\end{frame}
\begin{frame}
\frametitle{Particularidad de la 2a. solución}
Resulta que para los no enteros $s$, la función de Bessel $J_{-s} (x)$ es independiente de $J_{s} (x)$ y se puede usar como una segunda solución.
\\
\bigskip
\end{frame}
\begin{frame}
\frametitle{La 2a. solución}
Sin embargo, una segunda solución más común es la combinación lineal:
\pause
\begin{equation}
Y_{s} (x) = \dfrac{J_{s}(x) \, \cos s \pi - J_{-s}(x)}{\sin s\pi}
\label{eq:ecuacion_27_11}
\end{equation}
que es llamada la función de Bessel de segunda clase, o \textocolor{byzantine}{Funciones de Neumann}.
\end{frame}
\begin{frame}
\frametitle{Caso de valores enteros}
Para valores enteros de $s$, la función es indeterminada debido a que:
\pause
\begin{eqnarray}
\begin{aligned}[b]
&J_{-m}(x) = \left(\dfrac{x}{2} \right)^{-m} \nsum_{n=0}^{\infty} \dfrac{(-1)^{m+n}}{(m {+} n)! \, \Gamma(n {+} 1)} \, \left( \dfrac{x}{2} \right)^{2m+2n} = \\ \pause
&= (-1)^{m} \left( \dfrac{x}{2} \right)^{m} \nsum_{n=0}^{\infty} \dfrac{(-1)^{n}}{\Gamma(m {+} n {+} 1) \, n!} \, \left( \dfrac{x}{2} \right)^{2n} = \\ \pause
&= (-1)^{m} \, J_{m}(x)    
\end{aligned}
\label{eq:ecuacion_11_32}
\end{eqnarray}
y por la identidad $\cos n \pi = (-1)^{n}$.
\end{frame}
\begin{frame}
\frametitle{Usando un resultado}
Por lo tanto, usamos la regla de l'Hôpital y definimos:
\pause
\begin{eqnarray*}
\begin{aligned}
Y_{n}(x) &\equiv \lim_{s \to n} Y_{s} (x) = \lim_{s \to n} \dfrac{\displaystyle \pdv{s} \bigg[ J_{s}(x) \, \cos s \pi - J_{-s}(x) \bigg]}{\pi \, \cos n \pi} \\[1em] \pause
&= \dfrac{1}{\pi} \lim_{s \to n} \left[ \pdv{J_{s}}{s} - (-1)^{n} \pdv{J_{-s}}{s} \right]
\end{aligned}
\end{eqnarray*}
\end{frame}
\begin{frame}
\frametitle{Lo que se obtiene}
De la ec. (\ref{eq:ecuacion_27_10}) se obtiene que:
\pause
\begin{align*}
&\pdv{J_{s}}{s} = J_{s} (x) \, \ln \left( \dfrac{x}{2} \right) - \left( \dfrac{x}{2} \right)^{s} \times \\
&\times \nsum_{k=0}^{\infty} (-1)^{k} \, \dfrac{\Psi(s + k + 1)}{k! \, \Gamma(s + k + 1)} \, \left( \dfrac{x}{2} \right)^{2k}
\end{align*}
\pause
donde:
\pause
\begin{eqnarray*}
\Psi (x) \equiv \dv{x} \ln[(x - 1)!] = \pause \dv{x} \ln \Gamma (x) = \pause \dfrac{\dv*{\Gamma(x)}{x}}{\Gamma (x)}
\end{eqnarray*}
\end{frame}
\begin{frame}
\frametitle{Resultado análogo}
De manera similar:
\pause
\begin{align*}
&\pdv{J_{-s}}{s} = -J_{-s} (x) \, \ln \left( \dfrac{x}{2} \right) + \\[0.5em]
&+ \left( \dfrac{x}{2} \right)^{-s} \nsum_{k=0}^{\infty} (-1)^{k} \, \dfrac{\Psi(-s + k + 1)}{k! \, \Gamma(-s + k + 1)} \, \left( \dfrac{x}{2} \right)^{2k}
\end{align*}
\end{frame}
\begin{frame}
\frametitle{Expresando el resultado}    
Al sustituir estas expresiones en la definición de $Y_{n}(x)$ y usando la identidad $J_{-n} = (-1)^{n} J_{n}(x)$, se obtiene:
\pause
\begin{align*}
&Y_{n} (x) = \dfrac{2}{\pi} \, J_{n} (x) \, \ln \left(\dfrac{x}{2} \right) - \dfrac{1}{\pi} \left( \dfrac{x}{2} \right)^{n} \times \\[1em]
&\times \nsum_{k=0}^{\infty} (-1)^{k} \dfrac{\Psi (n {+} k {+} 1)}{k! \, \Gamma (n + k + 1)} \left(\dfrac{x}{2} \right)^{2k} =  \\ 
\end{align*}
\end{frame}
\begin{frame}
\frametitle{Expresando el resultado}    
\begin{eqnarray}
\begin{aligned}
Y_{n} (x) &= - \dfrac{1}{\pi} (-1)^{n} \left( \dfrac{x}{2} \right)^{-n} \times \\[0.5em]
&\times \nsum_{k=0}^{\infty} (-1)^{k} \dfrac{\Psi (k {-} n {+} 1)}{k! \, \Gamma (k {-} n {+} 1)} \left(\dfrac{x}{2} \right)^{2k}
\end{aligned}
\label{eq:ecuacion_27_12}
\end{eqnarray}
\end{frame}
\begin{frame}
\frametitle{Punto en particular}
Debe de quedar claro con la ec. (\ref{eq:ecuacion_27_12}) que la función de Neumann $Y_{s} (x)$ está mal definida en $x = 0$, como se espera de la segunda solución para la ED de Bessel, como la función $Z_{m} (x)$ discutida anteriormente.
\end{frame}
\begin{frame}
\frametitle{De la independencia lineal}
Como $Y_{s} (x)$ es linealmente independiente de $J_{s} (x)$ para cualquier $s$, entero o no entero, \pause es conveniente considerar $\left\{ J_{s} (x), Y_{s} (x) \right\}$ como base de las soluciones para la ED de Bessel.
\end{frame}
\begin{frame}
\frametitle{Solución final}
En particular, la solución de la ecuación radial en coordenadas cilíndricas, es decir, la primera ecuación en (\ref{eq:ecuacion_27_01a}), es:
\pause
\begin{eqnarray}
\begin{aligned}
R (\rho) &= A \, J_{m} (v) + B \, Y_{m} (v) = \\[0.5em] \pause 
&= A \, J_{m} (l \, \rho) + B \, Y_{m} (l \, \rho)
\end{aligned}
\label{eq:ecuacion_27_13}
\end{eqnarray}
\end{frame}

\section{Propiedades de las funciones de Bessel}
\frame[allowframebreaks]{\frametitle{Temas a revisar} \tableofcontents[currentsection, hideothersubsections]}
\subsection{Funciones de Bessel de orden entero negativo}

\begin{frame}
\frametitle{Relación entre órdenes}
La ec. (\ref{eq:ecuacion_11_32}) nos proporciona una relación de la función de Bessel de orden entero y la función de Bessel de orden entero negativo:
\pause
\begin{equation}
J_{-m} (x) = (-1)^{m} \, J_{m} (x)
\label{eq:ecuacion_27_14}
\end{equation}
\end{frame}

\subsection{Relaciones de recurrencia}

\begin{frame}
\frametitle{Relaciones importantes}
Una primera relación de recurrencia que no involucra la derivada de la función de Bessel es:
\pause
\begin{equation}
J_{m-1} (x) + J_{m+1} (x) = \dfrac{2 \, m}{x} \, J_{m} (x)
\label{eq:ecuacion_27_15}
\end{equation}
\end{frame}
\begin{frame}
\frametitle{Relaciones importantes}    
La siguiente relación considera la derivada de la función de Bessel:
\pause
\begin{equation}
J_{m-1} (x) - J_{m+1} (x) = 2 \, \pderivada{J}_{m} (x)
\label{eq:ecuacion_27_16}
\end{equation}
\end{frame}
\begin{frame}
\frametitle{Relaciones importantes}    
Combinando las dos ecuaciones anteriores, se obtiene:
\pause
\begin{eqnarray}
\begin{aligned}
J_{m-1}(x) &= \dfrac{m}{x} \, J_{m} (x) + \pderivada{J}_{m} (x) \\[0.5em] \pause
J_{m+1}(x) &= \dfrac{m}{x} \, J_{m} (x) - \pderivada{J}_{m} (x)
\end{aligned}
\label{eq:ecuacion_27_17}
\end{eqnarray}
\end{frame}
\begin{frame}
\frametitle{Más relaciones importantes}    
Podemos utilizar estas ecuaciones para obtener nuevas y útiles relaciones de recurrencia.
\end{frame}
\begin{frame}
\frametitle{Más relaciones importantes}
Por ejemplo, al diferenciar $x^{m} \, J_{m} (x)$, se obtiene:
\pause
\begin{eqnarray*}
\begin{aligned}
[x^{m} \, J_{m} (x)]^{\prime} &= m \, x^{m-1} \, J_{m} (x) + x^{m} \, \pderivada{J}_{m} (x) \\[1em] \pause
&= x^{m} \underbrace{ \left[ \dfrac{m}{x} \, J_{m} (x) + \pderivada{J}_{m} (x)\right]}_{=J_{m-1}(x) \text{ por } (\ref{eq:ecuacion_27_17})}  \\[1em] \pause
&= x^{m} \, J_{m-1} (x)
\end{aligned}
\end{eqnarray*}
\end{frame}
\begin{frame}
\frametitle{Operando la expresión}    
Integrando esta ecuación, nos lleva a:
\pause
\begin{equation}
\scaleint{6ex} x^{m} \, J_{m-1} (x) \dd{x} = x^{m} \, J_{m} (x)
\label{eq:ecuacion_27_18}
\end{equation}
\end{frame}
\begin{frame}
\frametitle{Operando la expresión}    
De manera similar, se obtiene que:
\pause
\begin{equation}
\scaleint{6ex} x^{-m} \, J_{m+1} (x) \dd{x} = -x^{-m} \, J_{m} (x)
\label{eq:ecuacion_27_19}
\end{equation}
\end{frame}

\subsection{Ortogonalidad}

\begin{frame}
\frametitle{Relación de ortogonalidad}
Las funciones de Bessel satisfacen una relación de ortogonalidad, \pause la cantidad que determina esta ortogonalidad de las diferentes funciones de Bessel no es el orden sino \textocolor{armygreen}{un parámetro en su argumento}.
\end{frame}
\begin{frame}
\frametitle{Relación de ortogonalidad}
Considera dos soluciones de la ED de Bessel que corresponden al mismo parámetro azimutal, \pause pero con un parámetro radial diferente.
\\
\bigskip
\pause
Más específicamente, hagamos:
\pause
\begin{eqnarray*}
f(\rho) &= J_{m} (k \rho) \\ \pause
g(\rho) &= J_{m} (l \rho)
\end{eqnarray*}
\end{frame}
\begin{frame}
\frametitle{Relación de ortogonalidad}
Entonces se tiene que:
\pause
\begin{eqnarray*}
\dv[2]{f}{\rho} + \dfrac{1}{\rho} \, \dv{f}{\rho} + \left( k^{2} - \dfrac{m^{2}}{\rho^{2}} \right) \, f &= 0 \\[1em] \pause
\dv[2]{g}{\rho} + \dfrac{1}{\rho} \, \dv{g}{\rho} + \left( l^{2} - \dfrac{m^{2}}{\rho^{2}} \right) \, g &= 0
\end{eqnarray*}
\end{frame}
\begin{frame}
\frametitle{Relación de ortogonalidad}
Al multiplicar la primera ecuación por $\rho \, g$ y la segunda por $\rho \, f$, para luego restarlas, se llega a:
\pause
\begin{align*}
\dv{\rho} [\rho (f \, g^{\prime} - g \, f^{\prime} ) ] = (k^{2} - l^{2}) \rho \, f \, g
\end{align*}
donde el primado indica que la diferenciación es con respecto a la variable $\rho$.
\end{frame}
\begin{frame}
\frametitle{Relación de ortogonalidad}
Al integrar la última ecuación con respecto a $\rho$ de un valor inicial (digamos $a$) a un valor final (digamos $b$), se obtiene:
\pause
\begin{align*}
[\rho (f \, g^{\prime} - g \, f^{\prime} ) ]\eval_{a}^{b} = (k^{2} - l^{2}) \scaleint{6ex}_{\bs a}^{b} \rho \, f(\rho) \, g(\rho) \dd{\rho}
\end{align*}
\pause
En todas las aplicaciones físicas, tanto $a$ como $b$ se escogen de tal manera para hacer que el lado derecho de la igualdad se anule.
\end{frame}
\begin{frame}
\frametitle{Relación de ortogonalidad}
Entonces, al sustituir $f$ y $g$ en términos de las funciones de Bessel, tenemos:
\pause
\begin{align*}
(k^{2} - l^{2}) \scaleint{6ex}_{\bs a}^{b} \rho \, J_{m} (k \, \rho) \, J_{m}(l \, \rho) \dd{\rho} = 0
\end{align*}
\end{frame}
\begin{frame}
\frametitle{Relación de ortogonalidad}
Se sigue que si $k \neq l$, entonces la integral se anula, es decir:
\pause
\begin{equation}
\scaleint{6ex}_{\bs a}^{b} \rho \, J_{m} (k \, \rho) \, J_{m}(l \, \rho) \dd{\rho} = 0 \hspace{1.5cm} \mbox{ si } k \neq l
\label{eq:ecuacion_27_20}
\end{equation}
\pause
Esta es la relación de ortogonalidad para las funciones de Bessel.
\end{frame}
\begin{frame}
\frametitle{Relación de ortogonalidad}
Para completar la relación de ortogonalidad, también debemos abordar el caso cuando $k = l$.
\end{frame}
\begin{frame}
\frametitle{Relación de ortogonalidad}
Esto implica la evaluación de la integral:
\pause
\begin{align*}
\displaystyle \scaleint{6ex} \rho \, J_{m}^{2} ( k \, \rho) \dd{\rho}
\end{align*}
\pause
la cual con el cambio de variable $x \equiv k \, \rho$ se reduce a:
\pause
\begin{align*}
\dfrac{\displaystyle \scaleint{6ex} x \, J_{m}^{2} (x) \dd{x}}{k^{2}}
\end{align*}
\end{frame}
\begin{frame}
\frametitle{Relación de ortogonalidad}
Integrando por partes, se tiene:
\pause
\begin{eqnarray*}
\begin{aligned}
&I \equiv \scaleint{6ex}  \underbrace{J_{m}^{2} (x)}_{u} \, \underbrace{x \dd{x}}_{\dd{v}} = \\[1em] \pause
&= \dfrac{1}{2} x^{2} \, J_{m}^{2} (x) - \scaleint{6ex} J_{m}(x) \, \pderivada{J}_{m} (x) \, x^{2} \dd{x}
\end{aligned}
\end{eqnarray*}
\end{frame}
\begin{frame}
\frametitle{Relación de ortogonalidad}
En la última integral, sustituimos por $x^{2} \, J_{m} (x)$ de la ecuación de Bessel (\ref{eq:ecuacion_27_06}), usando $x$ en lugar de $v$:
\pause
\begin{align*}
x^{2} \, J_{m} (x) =  m^{2} \, J_{m}(x) - x \, \pderivada{J}_{m} (x) - x^{2} \, \sderivada{J}_{m} (x)
\end{align*}
\end{frame}
\begin{frame}
\frametitle{Relación de ortogonalidad}
Por lo tanto:
\begin{eqnarray*}
\begin{aligned}
&I = \dfrac{1}{2} \, x^{2} \, J_{m}^{2} (x) - \scaleint{6ex} \pderivada{J}_{m}(x) \, \big[ m^{2} J_{m}(x) + \\
&{}\overbrace{- x \, \pderivada{J}_{m} (x) - x^{2} \, \sderivada{J}_{m} (x) \big ]}^{=-(\frac{1}{2} x^{2} [\pderivada{J}_{m} (x)]^{2})^{\prime}} \, \dd{x} 
\end{aligned}
\end{eqnarray*}
\end{frame}
\begin{frame}
\frametitle{Relación de ortogonalidad}
\begin{eqnarray*}
\begin{aligned}
&= \dfrac{1}{2} \, x^{2} \, J_{m}^{2} (x) - m^{2} \scaleint{6ex} \overbrace{J_{m} (x) \, \pderivada{J}_{m} (x)}^{=\frac{1}{2} [J_{m}^{2}(x)]^{\prime}} \dd{x} + \\ 
&+ \dfrac{1}{2} \scaleint{6ex} \dv{x} (x^{2} [\pderivada{J}_{m} (x)]^{2}) \dd{x} \\[1em] \pause 
&= \dfrac{1}{2} \, x^{2} \, J_{m}^{2} (x) - \dfrac{1}{2} m^{2} \, J_{m}^{2} (x) + \dfrac{1}{2} \, x^{2} \, [\pderivada{J}_{m}(x)]^{2}
\end{aligned}
\end{eqnarray*}
\end{frame}
\begin{frame}
\frametitle{Relación de ortogonalidad}
Regresando a la variable $\rho$, obtenemos una integral indefinida:
\pause
\begin{eqnarray}
\begin{aligned}[b]
&\scaleint{6ex} \rho \, J_{m}^{2} ( k \, \rho) \dd{\rho} = \pause \dfrac{I}{k^{2}} = \\
&= \dfrac{1}{2} \left(\rho^{2} - \dfrac{m^{2}}{k^{2}} \right) \, J_{m}^{2} (k \, \rho) + \dfrac{1}{2} \rho^{2} [\pderivada{J}_{m}(k \, \rho)]^{2}
\end{aligned}
\label{eq:ecuacion_27_21}
\end{eqnarray}
\end{frame}
\begin{frame}
\frametitle{Relación de ortogonalidad}
En la mayoría de las aplicaciones, el límite inferior de integración es cero y el límite superior es un número positivo $a$. 
\end{frame}
\begin{frame}
\frametitle{Relación de ortogonalidad}
El lado derecho de la igualdad de la ec. (\ref{eq:ecuacion_27_21}) se anula en el límite inferior debido a la siguiente razón: 
\\
\bigskip
\pause
El primer término se desvanece en $\rho = 0$ porque $J_{m} (0) = 0$ para todo $m > 0$, como se desprende de la expansión de la serie (\ref{eq:ecuacion_27_10}). 
\end{frame}
\begin{frame}
\frametitle{Relación de ortogonalidad}
Para $m = 0$ (y $\rho = 0$), los paréntesis en el primer término de la ec. (\ref{eq:ecuacion_27_21}) se anulan. 
\\
\bigskip
\pause
Entonces, el primer término es cero para todo $m \geq 0$ en el límite inferior de integración.
\end{frame}
\begin{frame}
\frametitle{Relación de ortogonalidad}
El segundo término se anula debido a la presencia de $\rho^{2}$. \pause Así, obtenemos:
\pause
\begin{equation}
\begin{aligned}[b]
&\scaleint{6ex}_{\bs 0}^{a} \rho \, J_{m}^{2} ( k \, \rho) \dd{\rho} = \\
&= \dfrac{1}{2} \left( a^{2} - \dfrac{m^{2}}{k^{2}} \right) \, J_{m}^{2} (k \, a) + \dfrac{1}{2} a^{2} [\pderivada{J}_{m} (k \, a)]^{2}
\end{aligned}
\label{eq:ecuacion_27_22}
\end{equation}
\pause
para todo $m \geq 0$ y, por la ec. (\ref{eq:ecuacion_27_14}), también para todos los enteros negativos.
\end{frame}
\begin{frame}
\frametitle{Relación de ortogonalidad}
Como se mencionó anteriormente, limitaremos nuestra discusión a las funciones de Bessel de orden entero.
\end{frame}
\begin{frame}
\frametitle{Relación de ortogonalidad}
Es costumbre simplificar el lado derecho de la igualdad de la ec. (\ref{eq:ecuacion_27_22}) al elegir $k$ de tal manera que $J_{m} (k \, a) = 0$, es decir, que $k \, a$ sea una raíz de la función de Bessel de orden $m$. 
\\
\bigskip
\pause
En general, hay infinitas raíces.
\end{frame}
\begin{frame}
\frametitle{Relación de ortogonalidad}
Entonces, digamos que $x_{mn}$ corresponde a la $n$-ésima raíz de $J_m (x)$.
\\
\bigskip
\pause
Entonces:
\pause
\begin{align*}
k \, a = x_{mn} \hspace{0.75cm} \Longrightarrow \hspace{0.75cm} k = \dfrac{x_{mn}}{a}, \hspace{0.5cm} n = 1, 2, 3, \ldots
\end{align*}
\end{frame}
\begin{frame}
\frametitle{Relación de ortogonalidad}
Si utilizamos la ec. (\ref{eq:ecuacion_27_17}) obtenemos:
\pause
\begin{equation}
\scaleint{6ex}_{\bs 0}^{a} \rho \, J_{m}^{2} \left(\dfrac{x_{mn} \, \rho}{a}   \right) \dd{\rho} = \dfrac{1}{2} a^{2} \big[ J_{m+1} (x_{mn}) \big]^{2}
\label{eq:ecuacion_27_23}
\end{equation}
\end{frame}
\begin{frame}
\frametitle{Relación de ortogonalidad}
Las ecuaciones (\ref{eq:ecuacion_27_20}) y (\ref{eq:ecuacion_27_23}) se pueden combinar en una ecuación sencilla usando una delta de Kronecker:
\end{frame}
\begin{frame}
\frametitle{Relación de ortogonalidad}
\begin{tcolorbox}
Las funciones de Bessel de orden entero satisfacen la relación de ortogonalidad.
\begin{eqnarray}
\begin{aligned}
&\scaleint{6ex}_{\bs 0}^{a} J_{m} \left(\dfrac{x_{mn} \, \rho}{a} \right) \, J_{m} \left(\dfrac{x_{mk} \, \rho}{a} \right) \, \rho \dd{\rho} = \\
&= \dfrac{1}{2} a^{2} \, J_{m+1}^{2} (x_{mn}) \, \delta_{kn}
\end{aligned}
\label{eq:ecuacion_27_24}
\end{eqnarray}
donde $a > 0$ y $x_{mn}$ es la $n$-ésima raíz de $J_{m}(x)$.
\end{tcolorbox}
\end{frame}

\subsection{Función generatriz}

\begin{frame}
\frametitle{Definiendo la función generatriz}
Las funciones de Bessel de orden entero tienen una función generatriz, es decir, existe una función $g(x, t)$, tal que:
\pause
\begin{equation}
g (x, t) = \nsum_{n=-\infty}^{\infty} t^{n} \, J_{n} (x)
\label{eq:ecuacion_27_25}
\end{equation}
\end{frame}
\begin{frame}
\frametitle{Estableciendo $g$}
Para determinar a la función $g$, comenzamos con la siguiente relación de recurrencia:
\pause
\begin{align*}
J_{m-1}(x) + J_{m+1} (x) = \dfrac{2 \, m}{x} \, J_{m} (x)
\end{align*}
\end{frame}
\begin{frame}
\frametitle{Estableciendo $g$}
Multiplicándola por $t^{m}$, para luego sumar sobre todo $m$, tenemos:
\pause
\begin{eqnarray}
\begin{aligned}
&\nsum_{m=-\infty}^{\infty} t^{m} \, J_{m-1} (x) + \nsum_{m=-\infty}^{\infty} t^{m} \, J_{m+1} (x) = \\[0.5em] \pause
=& \dfrac{2}{x} \, \nsum_{m=-\infty}^{\infty} m \, t^{m} \, J_{m} (x)
\label{eq:ecuacion_27_26}
\end{aligned}
\end{eqnarray}
\end{frame}
\begin{frame}
\frametitle{Estableciendo $g$}
La primera suma se puede escribir como:
\pause
\begin{eqnarray*}
\begin{aligned}
&\nsum_{m=-\infty}^{\infty} t^{m} \, J_{m-1} (x) = \pause t \, \nsum_{-\infty}^{\infty} t^{m-1} \, J_{m-1} (x) = \\[0.5em] \pause
&= t \, \nsum_{-\infty}^{\infty} t^{n} \, J_{n} (x) = \\[0.5em] \pause
&= t \, g(x, t)
\end{aligned}
\end{eqnarray*}
donde se ha sustituido el índice mudo $n = m -1$ por $m$.
\end{frame}
\begin{frame}
\frametitle{Estableciendo $g$}
De manera similar:
\pause
\begin{eqnarray*}
\begin{aligned}
&\nsum_{m=-\infty}^{\infty} t^{m} \, J_{m+1} (x) = \pause \dfrac{1}{t} \, \nsum_{-\infty}^{\infty} t^{m+1} \, J_{m+1} (x) = \\[0.5em] \pause
&= \dfrac{1}{t} \,  g(x, t)
\end{aligned}
\end{eqnarray*}
\end{frame}
\begin{frame}
\frametitle{Estableciendo $g$}
Por lo que:
\pause
\begin{eqnarray*}
\begin{aligned}
&\dfrac{2}{m} \, \nsum_{m=-\infty}^{\infty} m \, t^{m} \, J_{m} (x) = \\[0.5em]
&= \dfrac{2 \, t}{x} \, \nsum_{m=-\infty}^{\infty} m \, t^{m-1} \, J_{m} (x) = \dfrac{2 \, t}{x} \, \pdv{g}{t}
\end{aligned}
\end{eqnarray*}
\end{frame}
\begin{frame}
\frametitle{Estableciendo $g$}
Se sigue de la ec. (\ref{eq:ecuacion_27_26}) que:
\pause
\begin{align*}
\left( t + \dfrac{1}{t} \right) \, g(x, t) = \dfrac{2 \, t}{x} \, \pdv{g}{t}
\end{align*}
\pause
o también:
\begin{align*}
\dfrac{x}{2} \left( 1 + \dfrac{1}{t^{2}} \right) \dd{t} = \dfrac{\dd{g}}{g}
\end{align*}
donde se supone que $x$ es una constante porque hemos estado diferenciando con respecto a $t$.
\end{frame}
\begin{frame}
\frametitle{Estableciendo $g$}
Integrando ambos lados resulta:
\pause
\begin{align*}
\underbrace{\scaleint{6ex} \dfrac{x}{2} \left( 1 + \dfrac{1}{t^{2}} \right) \dd{t}}_{=\frac{x}{2} (t - \frac{1}{t})} = \ln g + \ln \phi(x)
\end{align*}
donde el último término es la \enquote{constante} de integración.
\end{frame}
\begin{frame}
\frametitle{Estableciendo $g$}
Entonces:
\pause
\begin{align*}
g (x, t) = \phi (x) \, \exp \left[ \dfrac{x}{2} \left( t - \dfrac{1}{t} \right) \right]
\end{align*}
\end{frame}
\begin{frame}
\frametitle{Determinando $\phi (x)$}
Para encontrar $\phi(x)$, veamos que:
\pause
\begin{eqnarray*}
\begin{aligned}
g(x, t) &= \phi (x) \, \exp(x \, t / 2) \, \exp(-x \, t / 2) = \\[1em] \pause
&= \phi (x) \, \nsum_{n=0}^{\infty} \dfrac{(x \, t / 2)^{n}}{n!} \, \nsum_{m=0}^{\infty} \dfrac{(-x \, t / 2)^{m}}{m!} \\[1em] \pause
&= \phi (x) \, \nsum_{n, m=0}^{\infty} \dfrac{(-1)^{m}}{n! \, m!} \, \left( \dfrac{x}{2} \right)^{n+m} \, t^{n-m}
\end{aligned}
\end{eqnarray*}
\end{frame}
\begin{frame}
\frametitle{Determinando $\phi (x)$}
En la última suma doble, se reúnen todos los términos cuya potencia de $t$ es cero y llamemos a esta suma $S_{0}$. 
\end{frame}
\begin{frame}
\frametitle{Determinando $\phi (x)$}
Esto se obtiene haciendo $n = m$.
\\
\bigskip
\pause
Entonces:
\pause
\begin{align*}
S_{0} = \phi (x) \, \nsum_{n=0}^{\infty} \dfrac{(-1)^{n}}{n! \, n!} \, \left( \dfrac{x}{2} \right)^{2n} = \phi (x) \, J_{0} (x)
\end{align*}
\pause
donde hemos usado la ec. (\ref{eq:ecuacion_27_10}) con $s = 0$. 
\end{frame}
\begin{frame}
\frametitle{Determinando $\phi (x)$}
Pero la ec. (\ref{eq:ecuacion_27_05}) nos dice que al juntar los términos cuya potencia de $t$ es cero, es simplemente $J_{0} (x)$. 
\\
\bigskip
\pause
Por lo tanto, $S_{0} = J_{0} (x)$, y $\phi (x) = 1$.
\end{frame}
\begin{frame}
\frametitle{La función generatriz}
Esto nos conduce a la forma final de la función de generatriz de Bessel:
\pause
\begin{equation}
g (x, t) = \exp \left[ \dfrac{x}{2} \left( t - \dfrac{1}{t} \right) \right] = \nsum_{n=-\infty}^{\infty} t^{n} \, J_{n} (x)
\label{eq:ecuacion_27_27}
\end{equation}
\end{frame}
% \textbf{Problema a cuenta: } A partir de la función generatriz, demuestra el \textbf{teorema de adición} de las funciones de Bessel:
% \begin{equation}
% J_{n} (x + y) = \nsum_{m=-\infty}^{\infty} J_{n-m} (x) \, J_{m} (y) = \nsum_{m=-\infty}^{\infty} J_{m} (x) \, J_{n-m} (y)
% \label{eq:ecuacion_27_28}
% \end{equation}
% Tip: Considera que
% \begin{align*}
% g (x + y, t) =  g(x, t) \, g(y, t)
% \end{align*}
% que tendrías que expandir y realizar las debidas operaciones para llegar al resultado indicado en la ec. (\ref{eq:ecuacion_27_28}).
% \par

\begin{frame}
\frametitle{Uso de la función generatriz}
La función generatriz de Bessel también nos puede conducir a algunas identidades muy importantes. 
\\
\bigskip
\pause
En la ec. (\ref{eq:ecuacion_27_27}), hacemos $t = e^{i \theta}$ y usamos el resultado:
\pause
\begin{equation}
\sin \theta = \dfrac{1}{2 \, i} (e^{i \theta} - e^{-i \theta})
\label{eq:ecuacion_18_14}
\end{equation}
\end{frame}
\begin{frame}
\frametitle{Resultado}
Para obtener:
\pause
\begin{equation}
e^{i  \, x \, \sin \theta} = \nsum_{n=-\infty}^{\infty} e^{i n \theta} \, J_{n} (x)
\label{eq:ecuacion_27_29}
\end{equation}
Esta es una expansión de la serie de Fourier en $\theta$, cuyos coeficientes son las funciones de Bessel.
\end{frame}
\begin{frame}
\frametitle{Calculando los coeficientes}
Para encontrar estos coeficientes, multiplicamos ambos lados por $e^{-i m \theta}$ para luego integrar de $-\pi$ a $\pi$. 
\\
\bigskip
\pause
El lado izquierdo de la igualdad nos devuelve:
\pause
\begin{align*}
\scaleint{6ex}_{\bs -\pi}^{\pi} e^{i \, x \, \sin \theta} \, e^{i \, m \,  \theta} \dd{\theta} = \scaleint{6ex}_{\bs -\pi}^{\pi} e^{i (x \, \sin \theta -  m \,  \theta} \dd{\theta}
\end{align*}
\end{frame}
\begin{frame}
\frametitle{Calculando los coeficientes}
Para el lado derecho de la igualdad, tenemos:
\pause
\begin{align*}
\nsum_{n=-\infty}^{\infty} \left[ \scaleint{6ex}_{\bs -\pi}^{\pi} e^{i(n - m) \theta} \dd{\theta} \right] \, J_{n} (x) = 2 \, \pi \, J_{m}(x)
\end{align*}
\end{frame}
\begin{frame}
\frametitle{Calculando los coeficientes}
En donde hemos utilizado el siguiente resultado (que se puede verificar fácilmente):
\pause
\begin{align*}
\scaleint{6ex}_{\bs -\pi}^{\pi} e^{i(n - m) \theta} \dd{\theta} &= \begin{cases}
0 & \mbox{ si } n \neq m \\
2 \, \pi & \mbox{ si } n = m
\end{cases} \\[1em]
&= 2 \pi \, \delta_{mn}
\end{align*}
\end{frame}
\begin{frame}
\frametitle{Calculando los coeficientes}
Al igualar ambos lados de la expresión, obtenemos:
\pause
\begin{equation}
J_{m} (x) = \dfrac{1}{2 \, \pi} \scaleint{6ex}_{\bs -\pi}^{\pi} e^{i (x \, \sin \theta -  m \, \theta}) \dd{\theta}
\label{eq:ecuacion_27_30}
\end{equation}
\end{frame}
\begin{frame}
\frametitle{Calculando los coeficientes}
Que podemos reducir a:
\pause
\begin{equation}
J_{m} (x) = \dfrac{1}{\pi} \scaleint{6ex}_{\bs -\pi}^{\pi} \cos (x \, \sin \theta -  m \, \theta) \dd{\theta}
\label{eq:ecuacion_27_31}
\end{equation}
que es llamada la \textocolor{cerise}{integral de Bessel.}
\end{frame}

\section{Expansión con funciones de Bessel}
\frame{\tableofcontents[currentsection, hideothersubsections]}
\subsection{Aprovechando la ortogonalidad}

\begin{frame}
\frametitle{Usando la ortogonalidad}
La ortogonalidad de las funciones de Bessel puede ser útil para ampliar otras funciones en términos de ellas.
\\
\bigskip
\pause
La idea básica es similar a la expansión en las series de Fourier.
\end{frame}
\begin{frame}
\frametitle{Usando la ortogonalidad}
Si una función $f (\rho)$ está definida en el intervalo $(0, a)$, entonces podemos escribir:
\pause
\begin{equation}
f(\rho) = \nsum_{n=1}^{\infty} c_{n} \, J_{m} \left( \dfrac{x_{mn}\, \rho}{a} \right)
\label{eq:ecuacion_27_34}
\end{equation}
\end{frame}
\begin{frame}
\frametitle{Calculando los coeficientes}
Los coeficientes $c_{n}$ pueden calcularse al multiplicar ambos lados por la expresión $\rho \, J_{m}(x_{mk} \, \rho/a)$, para luego integrar de $0$ a $a$.
\end{frame}
\begin{frame}
\frametitle{Calculando los coeficientes}
Se puede verificar que esto nos lleva a:
\begin{equation}
c_{n} = \dfrac{2}{a^{2} \, J_{m+1}^{2} (x_{mn})} \, \scaleint{6ex}_{\bs 0}^{a} f (\rho) \, J_{m} \left( \dfrac{x_{mn} \, \rho}{a} \right) \, \rho \dd{\rho}
\label{eq:ecuacion_27_35}
\end{equation}
\end{frame}
\begin{frame}
\frametitle{Expansión de una función}
Con estas ecuaciones, podemos expandir una función en términos de las funciones de Bessel de un orden en particular.
\end{frame}
\begin{frame}
\frametitle{Expansión de una función}
Si elevamos al cuadrado la ec. (\ref{eq:ecuacion_27_34}) y la multiplicamos por $\rho$, para luego integrar de $0$ a $a$, se tiene que:
\pause
\begin{align*}
&\scaleint{6ex}_{\bs 0}^{a} f^{2} (\rho) \, \rho \dd{\rho} = \nsum_{n=1}^{\infty} \nsum_{k=1}^{\infty} c_{n} \, c_{k} \times \\[1em]
&\times \underbrace{\scaleint{6ex}_{\bs 0}^{a} J_{m} \left( \dfrac{x_{mn} \, \rho}{a} \right) \, J_{m} \left( \dfrac{x_{mk} \, \rho}{a} \right) \, \rho \dd{\rho}}_{=\frac{1}{2} a^{2} J_{m+1}^{2} (x_{mn}) \delta_{kn} \mbox{ por } (\ref{eq:ecuacion_27_24})}
\end{align*}
\end{frame}
\begin{frame}
\frametitle{Expansión de una función}
Que nos conduce a la llamada \textocolor{cobalt}{relación de Parseval}.
\begin{equation}
\scaleint{6ex}_{\bs 0}^{a} f^{2} (\rho) \, \rho \dd{\rho} = \dfrac{1}{2} a^{2} \nsum_{n=1}^{\infty} c_{n}^{2} \, J_{m+1}^{2} (x_{mn})
\label{eq:ecuacion_27_38}
\end{equation}
para un valor de $m$.
\\
\bigskip
\pause
Este valor de $m$ se puede escoger para que las integrales sean lo más sencillas posible.
\end{frame}
% \section{Ejemplos de la física.}
% Podemos considerar que las funciones de Bessel son las funciones \enquote{naturales} de la geometría cilíndrica.
% \par
% Como en el caso de las coordenadas cartesianas y esféricas, a menos que la simetría del problema simplifique la situación, la separación de la ecuación de Laplace da como resultado dos parámetros que conducen a una suma doble. Así, esperamos una suma doble en la solución más general de la ecuación de Laplace en geometrías cilíndricas. Una de estas sumas es sobre $m$ que, como muestra la ecuación (\ref{eq:ecuacion_27_03}), aparece en el argumento de las funciones seno y coseno. También designa el orden de la función Bessel (o Neumann).
% \par
% \subsection{Potencial en un cilindro conductor.}
% Para comprender el origen de la segunda suma, considera un cilindro conductor, de radio $a$ y altura $h$ (ver la figura xxx). Supongamos que el potencial en la tapa superior varía como $V (\rho, \varphi)$ mientras que la superficie lateral y la tapa inferior están conectadas a tierra.
% \par
% Calculemos el potencial electrostático $\Phi$ en todos los puntos dentro del cilindro.
% \par
% La solución general es el producto de las soluciones (\ref{eq:ecuacion_27_03}), (\ref{eq:ecuacion_27_04}) y (\ref{eq:ecuacion_27_13}):
% \begin{align*}
% \Phi (\rho, \varphi, z) = R(\rho) \, S(\varphi) \, Z(z)
% \end{align*}
% Como $\Phi (\rho, \varphi, 0) = 0$ para cualquier valor arbitrario de $\rho$ y $\varphi$, tenemos que $Z(0) = 0$, dando una constante $Z(z) = \sinh (l \, z)$.
% \par
% Ya que $\Phi (0, \varphi, z)$ es finito, es finito, no se permite ninguna función de Neumann en la expansión y, dentro de una constante, tenemos $R(\rho) = J_{m}(l \, \rho)$. Además, como $\Phi (a, \varphi, z) = 0$ para cualquier valor arbitrario $\varphi$ y $z$, debemos de tener
% \begin{align*}
% R(a) = J_{m} (l \, a) = 0 \hspace{0.5cm} \Longrightarrow \hspace{0.5cm} l \, a = x_{mn} \hspace{0.5cm} \Longrightarrow \hspace{0.5cm} l = \dfrac{x_{mn}}{a} \hspace{1cm} n = 1, 2, \ldots
% \end{align*}
% donde, $x_{mn}$ son las $n$-ésimas raíces de $J_{m}$.
% \par
% Ahora podemos multiplicar $R$, $S$ y $Z$ y luego sumar todos los valores posibles de $m$ y $n$, teniendo en cuenta que los valores negativos de $m$ dan términos que dependen linealmente de los valores positivos correspondientes. El resultado es la llamada \textbf{serie de Fourier-Bessel}:
% \begin{equation}
% \Phi (\rho, \varphi, z) = \nsum_{m=0}^{\infty} \, \nsum_{n=1}^{\infty} J_{m} \left( \dfrac{x_{mn}}{a}  \, \rho \right) \, \sinh \left( \dfrac{x_{mn}}{a}  \, z \right) \, (A_{mn} \cos m \, \varphi + B_{mn} \sin m \, \varphi )
% \label{eq:ecuacion_27_39}
% \end{equation}
% donde las constantes $A_{mn}$ y $B_{mn}$ deben ser determinadas por las CDF restantes que establecen que $\Phi (\rho, \varphi, z) = V(\rho, \varphi)$ o
% \begin{equation}
% V (\rho, \varphi) = \nsum_{m=0}^{\infty} \, \nsum_{n=1}^{\infty} J_{m} \left( \dfrac{x_{mn}}{a}  \, \rho \right) \, \sinh \left( \dfrac{x_{mn}}{a}  \, h \right) \, (A_{mn} \cos m \, \varphi + B_{mn} \sin m \, \varphi )
% \label{eq:ecuacion_27_40}
% \end{equation}
% Multiplicando ambos lados por
% \begin{align*}
% \rho \, J_{m} \left( \dfrac{x_{mk} \, a}{\rho} \right) \, \cos j \varphi
% \end{align*}
% para luego integrar de $0$ a $2 \, \pi$, y de $0$ a $a$ en $\rho$, se obtiene $A_{jk}$. Cambiando el coseno al seno, y siguiendo el mismo procedimiento, obtenemos $B_{jk}$. Regresamos del índice $m$ al $n$, por lo tanto
% \begin{align}
% \begin{aligned}
% A_{mn} &= \dfrac{\displaystyle 2 \scaleint{6ex}_{\bs 0}^{2 \pi} \dd{\varphi} \scaleint{6ex}_{\bs 0}^{a} \rho \, V (\rho, \varphi) \, J_{m} \, \left( \dfrac{x_{mn}}{a}  \, \rho \right) \, \cos m \varphi \dd{\rho}}{\pi \, a^{2} \, J_{m+1}^{2} (x_{mn}) \, \sinh \left( \dfrac{x_{mn} \, h}{a} \right) } \\[1em]
% B_{mn} &= \dfrac{\displaystyle 2 \scaleint{6ex}_{\bs 0}^{2 \pi} \dd{\varphi} \scaleint{6ex}_{\bs 0}^{a} \rho \, V (\rho, \varphi) \, J_{m} \, \left( \dfrac{x_{mn}}{a}  \, \rho \right) \, \sin m \varphi \dd{\rho}}{\pi \, a^{2} \, J_{m+1}^{2} (x_{mn}) \, \sinh \left( \dfrac{x_{mn} \, h}{a} \right) }
% \end{aligned}
% \label{eq:ecuacion_27_41}
% \end{align}
% en donde se ha utilizado la ec. (\ref{eq:ecuacion_27_24}).
% \par
% El caso importante de simetría azimutal requiere una consideración especial. En tal caso, el potencial de la superficie superior $V (\rho, \varphi)$ debe ser independiente de $\varphi$. Además, como $S (\varphi)$ es constante, su derivada debe anularse. Por lo tanto, la segunda ecuación en (\ref{eq:ecuacion_27_01}) devuelve $\mu = -m^{2} = 0$. Este valor cero de $m$ reduce la suma doble de la ec. (\ref{eq:ecuacion_27_39}) a una sola suma, y obtenemos
% \begin{equation}
% \Phi (\rho, z) = \nsum_{n=1}^{\infty} A_{n} \, J_{0} \left( \dfrac{x_{0n}}{a}  \, \rho \right) \, \sinh \left( \dfrac{x_{0n}}{a}  \, z \right)
% \label{eq:ecuacion_27_42}
% \end{equation}
% Los coeficientes $A_{n}$ pueden obtenerse haciendo $m = 0$ en la primera ecuación de (\ref{eq:ecuacion_27_41}):
% \begin{equation}
% A_{n} = \dfrac{4}{a^{2} \, J_{1}^{2} (x_{0n}) \, \sinh \left( \dfrac{x_{0n} \, h}{a} \right)} \, \scaleint{6ex}_{\bs 0}^{a} \rho \, V (\rho) \, J_{0} \, \left( \dfrac{x_{0n}}{a}  \, \rho \right) \dd{\rho}
% \label{eq:ecuacion_27_43}
% \end{equation}
% donde $V(\rho)$ es el potencial independiente de $\varphi$ en la tapa superior del cilindro.
% \subsection{Cilindro conductor con tapa a un potencial fijo $V_{0}$.}
% Supongamos que la tapa superior de un cilindro conductor se mantiene a un potencial constante $V_{0}$ mientras que la superficie lateral y la tapa inferior están conectadas a tierra.
% \par
% Queremos calcular el potencial electrostático $\Phi$ en todos los puntos dentro del cilindro.
% \par
% Como el potencial de la tapa superior es independiente de $\varphi$, prevalece la simetría azimutal y la ec. (\ref{eq:ecuacion_27_43}) nos devuelve
% \begin{align*}
% A_{n} &= \dfrac{4 \, V_{0}}{a^{2} \, J_{1}^{2} (x_{0n}) \, \sinh \left( \dfrac{x_{0n} \, h}{a} \right)} \, \scaleint{6ex}_{\bs 0}^{a} \rho \,  J_{0} \, \left( \dfrac{x_{0n}}{a}  \, \rho \right) \dd{\rho} \\[1em]
% &= \dfrac{4 \, V_{0}}{x_{0n} \, J_{1} (x_{0n}) \, \sinh \left( \dfrac{x_{0n} \, h}{a} \right)}
% \end{align*}
% donde hemos usado la ec. (\ref{eq:ecuacion_27_18}). El detalle del cálculo de la integral es \textbf{un problema a cuenta}. Se tiene entonces que
% \begin{align*}
% \Phi (\rho, z) = 4 \, V_{0} \, \nsum_{n=1}^{\infty} \dfrac{J_{0} \left( \dfrac{x_{0n} \, \rho}{a} \right) \, \sinh \left( \dfrac{x_{0n} \, z}{a} \right)}{{x_{0n} \, J_{1} (x_{0n}) \, \sinh \left( \dfrac{x_{0n} \, h}{a} \right)}}
% \end{align*}

\end{document}