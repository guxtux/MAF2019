\documentclass[12pt]{article}
\usepackage[utf8]{inputenc}
\usepackage[spanish,es-lcroman, es-tabla]{babel}
\usepackage[autostyle,spanish=mexican]{csquotes}
\usepackage{amsmath}
\usepackage{amssymb}
\usepackage{nccmath}
\numberwithin{equation}{section}
\usepackage{amsthm}
\usepackage{graphicx}
\usepackage{epstopdf}
\DeclareGraphicsExtensions{.pdf,.png,.jpg,.eps}
\usepackage{color}
\usepackage{float}
\usepackage{multicol}
\usepackage{enumerate}
\usepackage[shortlabels]{enumitem}
\usepackage{anyfontsize}
\usepackage{anysize}
\usepackage{array}
\usepackage{multirow}
\usepackage{enumitem}
\usepackage{cancel}
\usepackage{tikz}
\usepackage{circuitikz}
\usepackage{tikz-3dplot}
\usetikzlibrary{babel}
\usepackage{bm}
\usepackage{mathtools}
\usepackage{esvect}
\usepackage{hyperref}
\usepackage{relsize}
\usepackage{siunitx}
\usepackage{physics}
%\usepackage{biblatex}
\usepackage{standalone}
\usepackage{mathrsfs}
\usepackage{bigints}
\usepackage{bookmark}
\spanishdecimal{.}

\setlist[enumerate]{itemsep=0mm}

\renewcommand{\baselinestretch}{1.5}

\let\oldbibliography\thebibliography

\renewcommand{\thebibliography}[1]{\oldbibliography{#1}

\setlength{\itemsep}{0pt}}
%\marginsize{1.5cm}{1.5cm}{2cm}{2cm}


\newtheorem{defi}{{\it Definición}}[section]
\newtheorem{teo}{{\it Teorema}}[section]
\newtheorem{ejemplo}{{\it Ejemplo}}[section]
\newtheorem{propiedad}{{\it Propiedad}}[section]
\newtheorem{lema}{{\it Lema}}[section]

\title{Vibraciones de una membrana circular \\ {\large Tema 5 - Matemáticas Avanzadas de la Física}\vspace{-1.5\baselineskip}}
\date{ }
\author{}
\begin{document}
\maketitle
\fontsize{14}{14}\selectfont
\section{Definición del problema.}
Una membrana elástica uniforme con densidad superficial de masa $\sigma$ está sometida a
una tensión superficial también uniforme $T$. La membrana en equilibrio se encuentra en
el plano $Oxy$, sobre $z = 0$.
\par
La vibración de los puntos de en la superficie de la membrana constituyen una función $z(x, y, t)$ que obedece la ecuación de onda
\begin{equation}
\pdv[2]{z}{x} + \pdv[2]{z}{y} = \dfrac{1}{c^{2}} \, \pdv[2]{z}{t}, \hspace{1.5cm} c^{2} = \dfrac{T}{\sigma}
\label{eq:ecuacion_026}
\end{equation}
Analizaremos el caso de una membrana circular de radio $a$ y fija en el borde. En este
caso, intentamos obtener la función en coordenadas polares $z (r , \varphi, t)$, por lo cual expresamos el laplaciando en coordenadas polares. Esto es
\begin{equation}
\dfrac{1}{r} \pdv{r} \left[ r \, \pdv{z}{r} \right] + \dfrac{1}{r^{2}} \, \pdv[2]{z}{\varphi} = \dfrac{1}{c^{2}} \, \pdv[2]{z}{t}
\label{eq:ecuacion_027}
\end{equation}
Buscamos los modos normales de oscilación de la membrana, por lo que proponemos
una solución oscilatoria con frecuencia $\omega$ de la forma
\begin{equation}
z(r, \varphi, t) = F (r, \varphi) \, \exp(i \, \omega \, t)
\label{eq:ecuacion_028}
\end{equation}
Al sustituir este resultado en la ec. (\ref{eq:ecuacion_027}), se obtiene la expresión
\begin{equation}
\dfrac{1}{r} \pdv{r} \left[ r \, \pdv{F}{r} \right] + \dfrac{1}{r^{2}} \, \pdv[2]{F}{\varphi} = -\dfrac{\omega^{2}}{c^{2}} \, F(r, \varphi)
\label{eq:ecuacion_029}
\end{equation}
Hacemos $k^{2} = \omega^{2}/c^{2}$, para luego multiplicar por $r^{2}$, resulta entonces
\begin{equation}
r \, \pdv{r} \left[ r \, \pdv{F}{r} \right] + \pdv[2]{F}{\varphi} + k^{2} \, r^{2} \, F = 0
\label{eq:ecuacion_030}
\end{equation}
Usando la técnica de separación de variables, proponemos una soluci{on del tipo
\begin{equation}
F(r, \varphi) = R(r) \, \Phi(\varphi)
\label{eq:ecuacion_031}
\end{equation}
que al sustituirla en la ec. (\ref{eq:ecuacion_030}), para luego dividir por $R \, \Phi$, se tiene
\begin{equation}
\dfrac{r}{R} \, \dv{r} \left[ r \, \dv{R}{r} \right] + \dfrac{1}{\Phi} \dv[2]{\Phi}{\varphi} + k^{2} \, r^{2} = 0
\label{eq:ecuacion_032}
\end{equation}
Como el segundo término de la ec,(\ref{eq:ecuacion_032}) es solamente función de $\varphi$ y los otros dos, lo son solamente de la variable $r$, cada uno de ellos será una constante. Es decir,
\begin{equation}
\dfrac{1}{\Phi} \dv[2]{\Phi}{\varphi} = -n^{2}
\label{eq:ecuacion_033}
\end{equation}
donde, en principio, $-n^{2}$ es cualquier número real o complejo. Lo escribimos de esta
forma por las condiciones que se impondrán a continuación.
\par
La solución general de la ec. (\ref{eq:ecuacion_033}) está dada por las dos funciones independientes de la forma
\begin{equation}
\Phi(\varphi) = A \, \exp(i \, n \, \varphi) + B \, \exp(- i \, n \, \varphi)
\label{eq:ecuacion_034}
\end{equation}
La función $\Phi$ debe de ser periódica, puesto que al barrer la coordenada $\varphi$ un ángulo
completo, llega al mismo lugar del espacio, de forma que $\Phi(\varphi + 2 \pi = \Phi (\varphi)$. Entonces la constante $n$ introducida en la ec. (\ref{eq:ecuacion_033}) por la separación de variables debe de ser real y además entero.
\par
El resultado de la ec. (\ref{eq:ecuacion_034}) también puede expresarse como combinación de senos y cosenos, de la forma
\begin{equation}
\Phi (\varphi) = A \, \cos (n \, \varphi) + A \, \sin (n \, \varphi)
\label{eq:ecuacion_035}
\end{equation}
Sustituyendo la ec. (\ref{eq:ecuacion_033}) en la ec. (\ref{eq:ecuacion_032}), obtenemos una ecuación para la función radial $R (r)$, que podemos escribir como
\begin{equation}
r \, \dv{r} \left( r \, \dv{R}{r} \right) + (k^{2} \, r^{2} - n^{2}) \, R = 0
\label{eq:ecuacion_036}
\end{equation}
siendo $n$ un entero arbitrario. Haciendo el cambio de variable $x = k \, r$ vemos que la
ec. (\ref{eq:ecuacion_036}) es la ecuación de Bessel de orden $n$.
\par
Por consiguiente, la solución aceptable que se mantiene finita en el origen es la
función de Bessel de primera clase y de orden $n$. Esto es
\begin{equation}
R(r) = J_{n} (k \, r)
\label{eq:ecuacion_037}
\end{equation}
Por otra parte, la condición de frontera fija de la membrana de radio $a$ implica que
\begin{equation}
J_{n} (k \, a) = 0 \hspace{1cm} \Longrightarrow \hspace{1cm} k \, a = x_{mn}, \hspace{1cm} n = 1, 2, 3,  \ldots
\label{eq:ecuacion_039}
\end{equation}
Esto significa que el producto $k \, a$ debe de ser una raíz de $J_{n}$ , lo cual limita los posibles valores de $m$ y $n$, por consiguiente, las posibles de las frecuencias $\omega = c \, k$ de los modos de oscilación de la membrana.
\par
De la ec. (\ref{eq:ecuacion_039}) entonces resulta
\begin{equation}
k = k_{mn} = \dfrac{x_{mn}}{a}, \hspace{1cm} \omega_{mn} = c \, \dfrac{x_mn}{a}
\label{eq:ecuacion_040}
\end{equation}
Las posibles soluciones para $R(r)$ pueden escribirse como
\begin{equation}
R(r) = J_{n} (k_{mn} \, r) = J_{n} \left( x_{mn} \, \dfrac{r}{a} \right)
\label{eq:ecuacion_041}
\end{equation}
Cada modo de oscilación de la membrana se identifica por dos índices correspondientes a una determinada frecuencia $\omega_{mn}$ dada en la ec. (\ref{eq:ecuacion_040}). El movimiento de la
membrana en un modo dado resulta entonces de multiplicar las funciones introducidas en la separación de variables: $R (r), \Phi (\varphi), exp(i \, \omega \,t)$. Obtenemos entonces, luego de tomar la parte real,
\begin{equation}
z_{mn} (r, \varphi, t) = J_{n} \left(\dfrac{x_{mn} \, r}{a} \right) \left[ A_{mn} \cos (n \, \varphi) \right]
\end{equation}
\end{document}