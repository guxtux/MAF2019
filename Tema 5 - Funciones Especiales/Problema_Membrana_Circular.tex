\documentclass[12pt]{article}
\usepackage[utf8]{inputenc}
\usepackage[spanish,es-lcroman, es-tabla]{babel}
\usepackage[autostyle,spanish=mexican]{csquotes}
\usepackage{amsmath}
\usepackage{amssymb}
\usepackage{nccmath}
\numberwithin{equation}{section}
\usepackage{amsthm}
\usepackage{graphicx}
\usepackage{epstopdf}
\DeclareGraphicsExtensions{.pdf,.png,.jpg,.eps}
\usepackage{color}
\usepackage{float}
\usepackage{multicol}
\usepackage{enumerate}
\usepackage[shortlabels]{enumitem}
\usepackage{anyfontsize}
\usepackage{anysize}
\usepackage{array}
\usepackage{multirow}
\usepackage{enumitem}
\usepackage{cancel}
\usepackage{tikz}
\usepackage{circuitikz}
\usepackage{tikz-3dplot}
\usetikzlibrary{babel}
\usetikzlibrary{shapes}
\usepackage{bm}
\usepackage{mathtools}
\usepackage{esvect}
\usepackage{hyperref}
\usepackage{relsize}
\usepackage{siunitx}
\usepackage{physics}
%\usepackage{biblatex}
\usepackage{standalone}
\usepackage{mathrsfs}
\usepackage{bigints}
\usepackage{bookmark}
\spanishdecimal{.}

\setlist[enumerate]{itemsep=0mm}

\renewcommand{\baselinestretch}{1.5}

\let\oldbibliography\thebibliography

\renewcommand{\thebibliography}[1]{\oldbibliography{#1}

\setlength{\itemsep}{0pt}}
%\marginsize{1.5cm}{1.5cm}{2cm}{2cm}


\newtheorem{defi}{{\it Definición}}[section]
\newtheorem{teo}{{\it Teorema}}[section]
\newtheorem{ejemplo}{{\it Ejemplo}}[section]
\newtheorem{propiedad}{{\it Propiedad}}[section]
\newtheorem{lema}{{\it Lema}}[section]

\title{Ondas estacionarias transversales sobre una membrana circular con borde fijo \\ {\large Tema 5 - Matemáticas Avanzadas de la Física}\vspace{-1.5\baselineskip}}
\date{ }
\author{}
\begin{document}
\maketitle
\fontsize{14}{14}\selectfont
\section{Definición del problema.}
Ahora consideramos ondas estacionarias en un sistema bidimensional con simetría circular: la de una membrana circular delgada y perfectamente compatible (es decir, flexible) (por ejemplo, una cabeza de tambor circular idealizada) de radio $R$, la ecuación de onda en coordenadas bidimensionales cilíndricas $(x , y \rightarrow r, \varphi)$ para la amplitud de desplazamiento, $\psi (r, \varphi, t)$ viene dada por:
\begin{align*}
\laplacian \psi (r, \varphi, t) - \dfrac{1}{v^{2}} \pdv[2]{\psi (r, \varphi, t)}{t} = 0
\end{align*}
Donde la velocidad longitudinal de propagación de las ondas transversales en una membrana circular bidimensional estirada está (también) dada por $v \sqrt{T_{\ell} / \sigma}$ donde $T_{\ell}$ es la tensión superficial de la membrana (en $\si{\newton\per\meter}$) y $\sigma \equiv M / A = M / \pi \, R^{2}$ es la densidad de masa areal\footnote{Masa por unidad de área.} de la membrana (en $\si{\kilo\gram\per\square\meter}$). 
\par
Con 
\begin{align*}
x &= r \cos \varphi \\
y &= r \sin \varphi,
\dd[2]{r} &= r \, \dd{r} \dd{\varphi}
\end{align*}
el operador laplaciano, $\laplacian$ en coordenadas cilíndricas 2-D viene dado por:
\begin{align*}
\laplacian = \dfrac{1}{r} \, \pdv{r} \left( r \, \pdv{r} \right) + \dfrac{1}{r^{2}} \pdv[2]{\varphi}
\end{align*}
Por lo tanto, la ecuación de onda bidimensional que describe el comportamiento de las ondas en una membrana cilíndrica viene dada por:
\begin{align*}
\pdv[2]{\psi (r, \varphi, t)}{r} + \dfrac{1}{r} \pdv{\psi (r, \varphi, t)}{r} + \dfrac{1}{r^{2}} \pdv[2]{\psi (r, \varphi, t)}{\varphi} - \dfrac{1}{v^{2}} \pdv{\psi (r, \varphi, t)}{t^{2}} = 0
\end{align*}
Reescribiendo la ecuación anterior, tenemos que
\begin{align*}
\pdv[2]{\psi (r, \varphi, t)}{r} + \dfrac{1}{r} \pdv{\psi (r, \varphi, t)}{r} + \dfrac{1}{r^{2}} \pdv[2]{\psi (r, \varphi, t)}{\varphi} = \dfrac{1}{v^{2}} \pdv{\psi (r, \varphi, t)}{t^{2}}
\end{align*}
Observemos que el lado izquierdo de la expresión (lado derecho) contiene sólo funciones dependientes del espacio (dependiente del tiempo), respectivamente. Por lo tanto, nuevamente podemos usar la técnica de separación de variables, con
\begin{align*}
\psi (r, \varphi, t) = U(r, \varphi) \, T(t)
\end{align*}
donde $U(r, \varphi)$ contiene solo términos dependientes espacialmente: $r$ y $\varphi$, mientras que $T (t)$ contiene solo el término dependiente del tiempo.
\par
Tenemos la relación 
\begin{align*}
v = f \lambda = \left( \dfrac{\omega}{2 \pi} \right) \left( \dfrac{2 \pi}{k} \right) = \dfrac{\omega}{k} \hspace{1cm} \Longrightarrow \hspace{1cm} v \, k = \omega
\end{align*}
Obtenemos una constante de separación de $-k^{2}$ y, después de algunas manipulaciones algebraicas simples, obtenemos las siguientes dos ecuaciones diferenciales lineales y homogéneas:
\begin{align*}
\pdv[2]{U (r, \varphi)}{r} + \dfrac{1}{r} \pdv{U (r, \varphi)}{r} + \dfrac{1}{r^{2}} \pdv[2]{U (r, \varphi)}{\varphi} + k^{2} \, U (r, \varphi) &= 0 \\
\dv[2]{T(t)}{t} + \omega^{2} \, T &= 0 
\end{align*}
Podemos usar nuevamente la técnica de separación de variables en la ecuación espacial anterior, con una solución de producto de la forma $U (r, \varphi) = R (r) \, \Phi (\varphi)$. Al agregar esto en la ecuación espacial anterior y realizar las diferenciaciones (parciales), dividiendo por $U (r, \varphi) = R (r) \, \Phi (\varphi)$ y realizando una manipulación algebraica simple, obtenemos la siguiente ecuación:
\begin{align*}
\dfrac{1}{R(r)} \left[ r^{2} \dv[2]{R(r)}{r} + r \, \dv{R(r)}{r} + k^{2} \, r^{2} \, R(r) \right] = - \dfrac{1}{\Phi(\varphi)} \, \dv[2]{\Phi(\varphi)}{\varphi}
\end{align*}
El lado izquierdo (lado derecho) de esta ecuación depende sólo de $r$ ($\varphi$), respectivamente. Nuevamente, esto sólo puede ser cierto para todos los valores posibles de $(r, \varphi)$, si tanto el lado derecho como el lado izquierdo de la ecuación son iguales a una constante (adimensional).
\par
Sabemos que las soluciones de $\Phi (\varphi)$ deben tener un valor periódico y univaluado, es decir,
\begin{align*}
\Phi (\varphi = 0) = \Phi (\varphi = 2 \pi)
\end{align*}
o de manera más general
\begin{align*}
\Phi (\varphi = \varphi_{0}) = \Phi (\varphi = \varphi_{0} + 2 \, m \, \pi) \hspace{1.5cm} m = 0, \pm 1, \pm 2, \pm 3, \ldots
\end{align*}
Por lo tanto, elegiremos esta constante de separación, tal que $m^{2}$.
La forma matemática de las soluciones propias $\Phi_{m} (\varphi)$ de la onda estacionaria que buscamos para vibraciones modales en una membrana circular debe satisfacer:
\begin{align*}
\dv[2]{\Phi_{m}(\varphi)}{\varphi} + m^{2} \Phi_{m} (\varphi) = 0
\end{align*}
Por lo tanto, las soluciones propias de $\Phi_{m} (\varphi)$ son (una de) las siguientes dos formas equivalentes:
\begin{align*}
\Phi_{m} (\varphi) &= \alpha_{m} \, \cos m \, \varphi + \beta \, \sin m \, \varphi \hspace{1cm} \begin{cases}
-1 \leq \alpha_{m} \leq 1, \\
-1 \leq \beta_{m} \leq 1 \\
\sqrt{\alpha_{m}^{2} + \beta_{m}^{2}} = 1
\end{cases} \\[1em]
\Phi_{m} (\varphi) &= \exp(i (m \, \varphi + \delta_{m})) \hspace{1cm} \delta_{m} = \tan^{-1} \left( \dfrac{\beta_{m}}{\alpha_{m}}\right) \\[1em]
&\mbox{con } m = 0, \pm 1, \pm 2, \ldots
\end{align*}
La ecuación radial es la conocida ecuación de Bessel:
\begin{align*}
\dv[2]{R(r)}{r} + \dfrac{1}{r} \dv{R(r)}{r} + \left( k^{2} - \dfrac{m^{2}}{r^{2}} \right) \, R(r) = 0
\end{align*}
La solución más general para la ecuación de Bessel, con $m =$ entero (que es el caso que tenemos aquí) tiene la siguiente forma:
\begin{align*}
R_{m} (r) = A_{m} \, J_{m} (k \, r) + B_{m} Y_{m} (k \, r)
\end{align*}
Las funciones $Y_{m} (x)$ son singulares (tienden a un infinito negativo) en $x = 0$. Sin embargo, debido a que usamos coordenadas cilíndricas para nuestra membrana circular, el origen $(r = 0)$ se incluye en este problema. Físicamente, \textbf{NO} permitimos desplazamientos de amplitud infinitos cuando $R (r) \rightarrow \infty$ para ningún valor de $r$, ya que una suposición inicial implícita eran pequeñas oscilaciones de amplitud. Por lo tanto, todos los coeficientes de $B_{m}$ para las funciones $Y_{m} (x)$ deben ser tales que $B_{m} = 0$ para las soluciones de los modos propios de oscilación permitidos físicamente de la membrana circular 2D.
\par
La condición de límite radial para ondas estacionarias transversales en una membrana circular con borde fijo (es decir, desplazamiento transversal cero) en $r = R$ es
\begin{align*}
R_{m} (r = R) = 0, \hspace{1cm} \Longrightarrow J_{m} (k \, R) = 0
\end{align*}
Dado que $r = R > 0$, esto significa que buscamos los ceros de $J_{m} (k \, R)$, es decir, $J_{m} (x = k \,R) = 0$. Debido a la complejidad de la forma de $J_{m} (x)$, los ceros de $J_{m} (x)$ (y de $Y_{m} (x)$) son de naturaleza no analítica, más bien, se tabulan en muchos libros matemáticos, o pueden determinarse mediante técnicas numéricas gráficas y/o computacionales.
\par
Resumimos los primeros ceros de $J_{m} (x)$ de orden inferior en la siguiente tabla:
\begin{table}[H]
\centering
\begin{tabular}{c c c c c c}
 & & & $n=1$ & $n=2$ & $n=3$ \\
$m=0:$ & $J_{0}(x)=0:$ & $x \approx$ & $2.40$ & $5.52$ & $8.65  \ldots$ \\
$m=1:$ & $J_{1}(x)=0:$ & $x \approx$ & $3.83$ & $7.02$ & $10.17 \ldots$ \\
$m=2:$ & $J_{2}(x)=0:$ & $x \approx$ & $5.14$ & $8.42$ & $11.62 \ldots$ \\
\vdots
\end{tabular}
\end{table}
Dado que $x = k \, R$, entonces $k = x / R$ y observando nuevamente, para este problema de valor propio de onda estacionaria bidimensional, \textbf{tenemos dos índices}: $m$ y $n$ para denotar:
\begin{itemize}
\item Los números de onda propios $k_{m, n}= x_{m, n} / R$,
\item Las frecuencias propias $\omega_{m, n}  = v \, k_{m, n}$  $f_{m, n} = v / \lambda_{m, n}$ con $v = \sqrt{T_{\ell} / \sigma}$,
\item  Las energías propias, $E_{m, n} = \dfrac{1}{4} M \, \omega_{m, n}^{2} \, A_{m, n}^{2}$
\item Las funciones propias $\psi_{m, n} (r, \varphi, t) = R_{m,n} (r, \varphi, t) \, \Phi_{m} (\varphi) \, T_{m, n} (t)$
\end{itemize}
Las soluciones en modo propio de la ecuación de onda temporal asociada para ondas estacionarias bidimensionales en una membrana circular son de la siguiente(s) forma(s) equivalente(s):
\begin{align*}
T_{m, n} (t) &= b_{m, n} \, \sin \omega_{m, n} \, t + c_{m , n} \, \cos \omega_{m, n} \, t \hspace{1cm} \begin{cases}
-1 \leq b_{m, n} \leq 1 \\
-1 \leq c_{m, n} \leq 1 \\
\sqrt{b_{m, n}^{2} + c_{m, n}^{2}} = 1
\end{cases} \\[1em]
T_{m, n} (t) &= \sin (\omega_{m, n} \, t + \delta_{m, n}) = \cos (\omega_{m, n} + \varphi_{m, n} ) \hspace{1cm} \delta_{m, n} = \varphi_{m, n} + \dfrac{\pi}{2} \\[1em]
T_{m, n} (t) &= \exp(i (\omega_{m, n} + \varphi_{m, n}))
\end{align*}
Las soluciones completas de los modos propio para ondas estacionarias bidimensionales en una membrana circular de radio, $R$ con el borde fijo, por lo tanto, están dada por:
\begin{align*}
\psi_{m,n} (r, \varphi, t) &= R_{m, n} (r) \, \Phi_{m}(\varphi) \, T_{m, n} (t) \\[1em]
\psi_{m,n} (r, \varphi, t) &= A_{m, n} \, J_{m} (k_{m,n} \, R) \, \exp(i(m \varphi + \delta_{m, n} )) \, \exp(i(\omega_{m,n} \, t + \varphi_{m, n})) \\[1em]
\psi_{m,n} (r, \varphi, t) &= A_{m, n} \, J_{m} (k_{m,n} \, R) *\\
&* [\alpha_{m} \cos m \, \varphi + \beta_{m} \sin m \, \varphi ][b_{m, n}\cos \omega_{m, n} \, t + c_{m, n} \sin \omega_{m, n} \, t]
\end{align*}
con
\begin{itemize}
\item frecuencias propias: $f_{m, n} = \dfrac{\omega_{m, n}}{2 \pi} = \dfrac{v \, k_{m, n}}{2 \pi} = \dfrac{v}{\lambda_{m, n}}$
\item longitud de onda propia: $\lambda_{m, n} = \dfrac{2 \pi}{k_{m, n}} = \dfrac{2 \pi \, R}{x_{m, n}}$
\item energías propias: $E_{m, n} = \dfrac{1}{4} M \, \omega_{m, n}^{2} \, A_{m, n}^{2}$
\item con $m= 0, 1, 2, 3, \ldots$ y $n = 1, 2, 3, \ldots$
\end{itemize}
Los modos más bajos de ondas estacionarias transversales en una membrana circular se enumeran a continuación:
\par
Para $m = 0$ y $n = 1$:
\begin{align*}
k_{0,1} \approx \dfrac{2.40}{R} \hspace{1cm} \omega_{0,1} \approx \dfrac{2.40 \, v}{R} \hspace{1cm}
\psi_{0,1} (r,\varphi,t) = A_{0,1} \, J_{0}(k_{0,1} \, R) \, T_{0,1} (t)
\end{align*}
Para $m = 1$ y $n = 1$:
\begin{align*}
k_{1,1} &\approx \dfrac{3.83}{R} \hspace{1cm} \omega_{1,1} \approx \dfrac{3.83 \, v}{R} \\
\psi_{1,1} (r,\varphi,t) &= A_{1,1} \, J_{1}(k_{1,1} \, R) \, [\alpha_{1}\cos \varphi + \beta_{1} \sin \varphi ] \, T_{1,1} (t)
\end{align*}
Para $m = 2$ y $n = 1$:
\begin{align*}
k_{2,1} &\approx \dfrac{5.14}{R} \hspace{1cm} \omega_{2,1} \approx \dfrac{5.14 \, v}{R} \\
\psi_{2,1} (r,\varphi,t) &= A_{2,1} \, J_{2}(k_{2,1} \, R) \, [\alpha_{2}\cos 2 \,\varphi + \beta_{2} \sin 2 \, \varphi ] \, T_{2,1} (t)
\end{align*}
Para $m = 0$ y $n = 2$:
\begin{align*}
k_{0,2} &\approx \dfrac{5.52}{R} \hspace{1cm} \omega_{0,2} \approx \dfrac{5.52 \, v}{R} \\
\psi_{0,2} (r,\varphi,t) &= A_{0,2} \, J_{0}(k_{0,2} \, R) \, T_{0,2} (t)
\end{align*}
Algunos de los modos propios de vibración de orden inferior para ondas estacionarias transversales en una membrana circular (con líneas nodales discontinuas) se muestran en la siguiente figura:
\newpage
\begin{table}[H]
\begin{tabular}{c@{\hskip 3cm} c@{\hskip 3cm} c@{\hskip 3cm}}
\includestandalone{Figuras/Modos_Vibracion_Membrana_0_1} & \includestandalone{Figuras/Modos_Vibracion_Membrana_1_1} & 
\includestandalone{Figuras/Modos_Vibracion_Membrana_2_1} \\
$m = 0, n = 1$ & $m = 1, n = 1$ & $m = 2, n = 1$\\
$m + n = 1$ & $m + n = 2$ & $m + n = 3$ \\
$J_{0}(k_{0,1} r)$ &  $J_{1}(k_{1,1} r) \, e^{i \varphi}$ & $J_{2}(k_{2,1} r) \, e^{2 \, i \varphi}$ \\ 
\multicolumn{3}{c}{} \\
\includestandalone{Figuras/Modos_Vibracion_Membrana_0_2} & \includestandalone{Figuras/Modos_Vibracion_Membrana_1_2} & 
\includestandalone{Figuras/Modos_Vibracion_Membrana_2_2} \\
$m = 0, n = 2$ & $m = 1, n = 2$ & $m = 2, n = 2$\\
$m + n = 2$ & $m + n = 3$ & $m + n = 4$ \\
$J_{0}(k_{0,2} r)$ &  $J_{1}(k_{1,2} r) \, e^{i \varphi}$ & $J_{2}(k_{2,2} r) \, e^{2 \, i \varphi}$ \\
\multicolumn{3}{c}{} \\
\includestandalone{Figuras/Modos_Vibracion_Membrana_0_3} & \includestandalone{Figuras/Modos_Vibracion_Membrana_1_3} & 
\includestandalone{Figuras/Modos_Vibracion_Membrana_2_3} \\
$m = 0, n = 3$ & $m = 1, n = 3$ & $m = 2, n = 3$\\
$m + n = 3$ & $m + n = 4$ & $m + n = 5$ \\
$J_{0}(k_{0,3} r)$ &  $J_{1}(k_{1,3} r) \, e^{i \varphi}$ & $J_{2}(k_{2,3} r) \, e^{2 \, i \varphi}$
\end{tabular}
\end{table}

\newpage
Las degeneraciones de 2 pliegues para $m > 0$ surgen de nuevo debido a los 2 grados espaciales de libertad ($x$ e $y$, $r$ y $\varphi$). La simetría rotacional de la membrana circular: es invariante en rotaciones arbitrarias.
\end{document}