\documentclass[12pt]{article}
\usepackage[utf8]{inputenc}
\usepackage[spanish,es-lcroman, es-tabla]{babel}
\usepackage[autostyle,spanish=mexican]{csquotes}
\usepackage{amsmath}
\usepackage{amssymb}
\usepackage{nccmath}
\numberwithin{equation}{section}
\usepackage{amsthm}
\usepackage{graphicx}
\usepackage{epstopdf}
\DeclareGraphicsExtensions{.pdf,.png,.jpg,.eps}
\usepackage{color}
\usepackage{float}
\usepackage{multicol}
\usepackage{enumerate}
\usepackage[shortlabels]{enumitem}
\usepackage{anyfontsize}
\usepackage{anysize}
\usepackage{array}
\usepackage{multirow}
\usepackage{enumitem}
\usepackage{cancel}
\usepackage{tikz}
\usepackage{circuitikz}
\usepackage{tikz-3dplot}
\usetikzlibrary{babel}
\usepackage{bm}
\usepackage{mathtools}
\usepackage{esvect}
\usepackage{hyperref}
\usepackage{relsize}
\usepackage{siunitx}
\usepackage{physics}
%\usepackage{biblatex}
\usepackage{standalone}
\usepackage{mathrsfs}
\usepackage{bigints}
\usepackage{bookmark}
\spanishdecimal{.}

\setlist[enumerate]{itemsep=0mm}

\renewcommand{\baselinestretch}{1.5}

\let\oldbibliography\thebibliography

\renewcommand{\thebibliography}[1]{\oldbibliography{#1}

\setlength{\itemsep}{0pt}}
%\marginsize{1.5cm}{1.5cm}{2cm}{2cm}


\newtheorem{defi}{{\it Definición}}[section]
\newtheorem{teo}{{\it Teorema}}[section]
\newtheorem{ejemplo}{{\it Ejemplo}}[section]
\newtheorem{propiedad}{{\it Propiedad}}[section]
\newtheorem{lema}{{\it Lema}}[section]

\usepackage{mathrsfs}
\usepackage{standalone}
\usepackage{tikz}
\usetikzlibrary{shapes}
\usepackage{bigints}
\newtheorem{problema}{{\it Problema}}
\spanishdecimal{.}
%\usepackage{enumerate}
%\author{M. en C. Gustavo Contreras Mayén. \texttt{curso.fisica.comp@gmail.com}}
\title{Armónicos esféricos \\ {\large Matemáticas Avanzadas de la Física}}
\date{ }
\begin{document}
\renewcommand\labelenumii{\theenumi.{\arabic{enumii}}}
\maketitle
\fontsize{14}{14}\selectfont
Las soluciones $\exp(\pm i m \varphi)$ forman una base ortogonal respecto al índice $m$ en $(0, 2\pi)$, mientras $P_{\ell}^{m} (x)$ son ortogonales en $(-1, 1)$ respecto al índice $\ell$.
\\
Es posible definir una nueva base ortogonal en $\theta$ y $\varphi$ respecto a los índices $\ell$ y $m$. Más exactamente, una base bi-ortogonal. Las nuevas funciones, ortonormales sobre una superficie esférica y conocidas como armónicos esféricos se definen como:
\begin{equation}
Y_{\ell m} (\theta, \varphi) = \sqrt{\dfrac{2 \ell + 1}{4 \pi} \dfrac{(\ell - m)!}{(\ell + m) !}} P_{\ell}^{m} (\cos \theta) e^{i m \varphi}
\label{eq:ecuacion_8_36}
\end{equation}
El radical ha sido escogido de forma tal que ${ Y_{\ell m} (\theta, \varphi)}$ sea una base ortonormal:
\[ \int_{\varphi = 0}^{2 \pi} \int_{\theta=0}^{\pi} Y^{*}_{\ell m} (\theta, \varphi) Y_{\ell^{\prime} m^{\prime}} (\theta, \varphi) \sin \theta d \theta d\varphi = \delta_{\ell \ell^{\prime}} \delta_{m m^{\prime}}  \]
y como el ángulo sólido es
\[ d \Omega =  \sin \theta d \theta d \varphi \]
podemos escribir
\begin{equation}
\int_{4 \pi} Y^{*}_{\ell m} (\theta, \varphi) Y_{\ell^{\prime} m^{\prime}} (\theta, \varphi) d \Omega =  \delta_{\ell \ell^{\prime}} \delta_{m m^{\prime}}
\label{eq:ecuacion_8_37}
\end{equation}
Nota: En coordenadas esféricas la ecuación
\[ \nabla^{2} \phi(r, \theta, \varphi) = 0 \]
se escribe
\[ \dfrac{1}{r} \dfrac{\partial^{2}}{\partial r^{2}} (r \phi)  + \dfrac{1}{r^{2} \sin \theta} \dfrac{\partial}{\partial \theta} \left( \sin \theta \dfrac{\partial \phi}{\partial \theta} \right) + \dfrac{1}{r^{2} \sin^{2} \theta} \dfrac{\partial^{2} \phi}{\partial \varphi^{2}}  = 0 \]
Proponemos una separación de variables de la forma
\[ \phi(r, \theta, \varphi) = \dfrac{U(r)}{r} \; Y(\theta, \varphi) \]
La parte radial ha sido escrita $U(r) = r$ con el fin de simplificar el primer término de la ecuación diferencial. Posteriormente separaremos $Y(\theta, \varphi)$ en un producto de funciones en $\theta$ y $\varphi$. Se sigue entonces:
\[ \dfrac{r^{2}}{U} \dfrac{d^{2} U}{d r^{2}} + \dfrac{1}{Y} \left[ \dfrac{1}{\sin \theta} \dfrac{\partial}{\partial \theta} \left( \sin \theta \dfrac{\partial Y}{\partial \theta} \right) + \dfrac{1}{\sin^{2} \theta} \dfrac{\partial^{2} Y}{\partial \varphi^{2}} \right] = 0\]
La ecuación se ha separado en dos componentes: una radial y la otra angular.
El cuadrado del operador
\[ \mathbf{L} = \dfrac{1}{i} \mathbf{r} \times \nabla \]
es el negativo del corchete en la última ecuación, por lo que se puede escribir
\[ \dfrac{r^{2}}{U} \dfrac{d^{2} U}{d r^{2}} -  \dfrac{L^{2} Y}{Y} = 0 \]
De acuerdo a la técnica de separación de variables cada sumando ha de ser una constante. Por razones que serán claras más tarde escribimos la constante de separación en la forma $\ell (\ell + 1)$ lo que resulta en
\[ r^{2} \dfrac{d^{2} U}{d r^{2}} - \ell (\ell + 1) U = 0, \hspace{1cm} L^{2} Y = \ell (\ell + 1) Y \]
La ecuación radial es homogénea en $r$, es decir, es  una ecuación del tipo de Euler cuya solución es
\[ U(r) = A \; r^{\ell + 1} + \dfrac{B}{r^{\ell}}, \hspace{0.5cm} \ell \geq 0 \]
La ecuación $L^{2} Y = \ell (\ell + 1) Y$ tiene la forma explícita
\[ \dfrac{1}{\sin \theta} \dfrac{\partial}{\partial \theta} \left( \sin \theta \dfrac{\partial Y}{\partial \theta}  \right) + \dfrac{1}{\sin^{2} \theta} \dfrac{\partial^{2} Y}{\partial \varphi^{2}} + \ell (\ell + 1) Y = 0 \]
Separando las variables $\theta$ y $\varphi$ en la forma $Y(\theta, \varphi) = P(\theta) G(\varphi)$, lo que nos devuelve, luego de reemplazar y multiplicar por $\sin^{2} \theta / PG$
\[ \dfrac{\sin \theta}{P} \dfrac{d}{\theta} \left( \sin \theta \dfrac{d P}{d \theta} \right) + \ell (\ell + 1) \sin^{2} \theta = - \dfrac{1}{G} \dfrac{d^{2} G}{d \varphi^{2}} \]
En consecuencia, escogiendo la constante de separación de modo que permita soluciones armónicas en $\varphi$, que a la vez garanticen la continuidad en $\varphi$ de la solución, tendremos
\[ \dfrac{1}{G} \dfrac{d^{2} G}{d \varphi^{2}} = - m^{2} \]
cuya solución es
\begin{eqnarray}
G &=& C e^{i m \varphi} + D e^{-i m \varphi}, \hspace{1cm} m = 1, 2, \ldots \nonumber \\
G &=& a \varphi + b, \hspace{1cm} m = 0 \nonumber
\end{eqnarray}
La ecuación en $\theta$ será entonces
\[ \sin \theta \dfrac{d}{d \theta} \left( \sin \theta \dfrac{d P}{d \theta} \right) +  P \; \ell (\ell +  1) \sin^{2} \theta - m^{2} \; P = 0 \]
con la sustitución $x = \cos \theta$, se obtiene la \emph{ecuación diferencial asociada de Legendre}:
\[ \dfrac{d}{d x} \left[ (1 - x^{2}) \dfrac{d P(x)}{d x} \right] - \dfrac{m^{2} P(x)}{1 - x^{2}} + \ell (\ell + 1) P(x) = 0 \]
con $m = 0$ se recupera la ecuación diferencial ordinaria de Legendre.
\\
Tenemos entonces que los armónicos esféricos son autofunciones del operador $ L^{2}$ con autovalores $\ell (\ell + 1)$:
\[ L^{2} \; Y_{\ell m} (\theta, \varphi) = \ell (\ell + 1) Y_{\ell m} (\theta, \varphi) \]
Dado que ${Y_{\ell m} (\theta, \varphi)}$ es una base completa, una función $f(\theta, \varphi)$ puede expandirse en armónicos esféricos
\begin{equation}
f (\theta, \varphi) = \sum_{\ell = 0}^{\infty} \sum_{m = -\ell}^{\ell}  A_{\ell m} Y_{\ell m} (\theta, \varphi)
\label{eq:ecuacion_8_38}
\end{equation}
Así pues la base ${Y_{\ell m} (\theta, \varphi)}$ permite expandir funciones definidas sobre la superficie de una esfera.
\subsection{Armónicos esféricos y operadores escalera.}
El operador momento angular (adimensional), también conocido como \emph{operador de rotación}, se define de la forma:
\[ \mathbf{L} = \dfrac{1}{i} \mathbf{r} \times \bm{\nabla} \]
En coordenadas cartesianas
\[  \mathbf{L} = -i \left[ \widehat{\mathbf{i}} \left( y \dfrac{\partial}{\partial z} - z \dfrac{\partial}{\partial y} \right) + \widehat{\mathbf{j}} \left( z \dfrac{\partial}{\partial x} - x \dfrac{\partial}{\partial z} \right) + \widehat{\mathbf{k}} \left( x \dfrac{\partial}{\partial y} - y \dfrac{\partial}{\partial x} \right) \right]  \]
en coordenadas esféricas:
\[  \mathbf{L} = i \left( \widehat{\mathbf{e}}_{\theta} \dfrac{1}{\sin \theta} \; \dfrac{\partial}{\partial \varphi} - \widehat{\mathbf{e}}_{\varphi} \dfrac{\partial}{\partial \theta} \right)  \]
sustituyendo $\widehat{\mathbf{e}}_{\theta}, \widehat{\mathbf{e}}_{\varphi}$ en términos de $(\mathbf{i, j, k})$, se obtiene:
\begin{eqnarray*}
L_{x} &=& i \left( \sin \varphi \dfrac{\partial}{\partial \theta} + \cot \theta \cos \varphi \dfrac{\partial}{\partial \varphi} \right) \nonumber \\
L_{y} &=& i \left( - \cos \varphi \dfrac{\partial}{\partial \theta} + \cot \theta \sin \varphi \dfrac{\partial}{\partial \varphi} \right) \nonumber \\
L_{z} &=& - i \dfrac{\partial}{\partial \varphi} \nonumber
\end{eqnarray*}
De acuerdo con lo visto previamente 
\[ L^{2} Y_{\ell m} = \ell (\ell + 1) Y_{\ell m} \]
y es fácil demostrar, utilizando la definición de $Y_{\ell m}$ que
\[ L_{z} Y_{\ell m} =  m Y_{ell m} \]
de modo que $Y_{\ell m}$ es función propia y simultánea de $L^{2}$ y $L_{z}$ con valores propios $\ell (\ell + 1)$ y $m$, respectivamente. Este hecho depende crucialmente de que $L^{2}$ y $L_{z}$ conmuten:
\[ [L^{2}, L_{z} ] = L^{2} L_{z} - L_{z} L^{2} = 0 \]
Sin embargo
\[ L_{i}, L_{j} = i \sum_{k=1}^{3} \epsilon_{ijk} L_{k} \]
por lo cual $L_{x}, L_{y}, L_{z}$ no conmutan entre sí, de lo que se sigue que $L^{2}, L_{x}, L_{y}$ no comparten $Y_{\ell m}$ como funciones propias. Los operadores que conmutan tienen el mismo conjunto de funciones propias.
\\
Definamos ahora la pareja de operadores $L_{+}$ y $L_{-}$ en la forma:
\[ L_{+} = L_{x} + i L_{y},\hspace{1.5cm} L_{-} = L_{x} - i L_{y}  \]
que en coordenadas esféricas se presetan como
\begin{eqnarray}
L_{+} &=& e^{i \varphi} \left( \dfrac{\partial}{\partial \theta} + i \cot \theta \dfrac{\partial}{\partial \varphi} \right) \nonumber \\
L_{-} &=& - e^{- i \varphi} \left( \dfrac{\partial}{\partial \theta} - i \cot \theta \dfrac{\partial}{\partial \varphi} \right) \nonumber
\end{eqnarray}
Consideremos los siguientes resultados a modo de tarea moral:
\begin{eqnarray*}
[L_{z}, L_{+}] &=& L_{+}, \hspace{1.5cm} [L_{z}, L_{-}] = - L_{-} \nonumber \\
[L_{+}, L_{-}] &=&  2 L_{z} \nonumber \\
[L^{2}, L_{z}] &=& [L^{2}, L_{+}] = [L^{2}, L_{-}] = 0 \nonumber \\
(L_{+} f, g) &=& (f, L_{-} g), \hspace{1.5cm} (L_{-} f, g) = (f, L_{+} g) \nonumber
\end{eqnarray*}
Las últimas dos ecuaciones, se pueden resumir en
\[ [L^{2}, L_{\pm}] = 0 \]
por lo que
\[ L^{2} (L_{\pm} Y_{\ell m}) =  L_{\pm} (L^{2} Y_{\ell m} ) = \ell (\ell + 1) L_{\pm} Y_{\ell m} \]
En esta notación aparecen, en verdad, dos ecuaciones, una para los signos superiores, y otra para los inferiores. Se sigue que $L_{\pm} Y_{\ell m}$ es función propia de $L^{2}$ con valor propio $\ell (\ell + 1)$.
\\
Además, de las primeras ecuaciones de la tarea moral, se sintetizan en
\[ [L_{z}, L_{\pm}] = \pm L_{\pm} \]
de donde:
\[ L_{z} (L_{\pm} Y_{\ell m}) = (L_{\pm} L_{z} \pm L_{\pm}) Y_{\ell m} = (m \pm 1) L_{\pm} Y_{\ell m} \]
por lo que $L_{\pm} Y_{\ell m}$ es función propia de $L_{z}$ con valor propio $m \pm 1$, de donde se sigue que:
\begin{equation}
L_{\pm} Y_{\ell m} = a_{\pm} Y_{\ell, m \pm 1}
\label{eq:ecuacion_8_45}
\end{equation}
Con el fin de calcular $a_{\pm}$, hay que tomar en cuenta que 
\[ L^{2} = \dfrac{1}{2} (L_{+} \; L_{-} \; + \; L_{-} \; L_{+}) \; +  L_{z}^{2} \]
y de que
\[ L_{z} = \dfrac{1}{2} (L_{+} \; L_{-} \; - \; L_{-} \; L_{+}) \]
obtenemos sumando y restando
\begin{eqnarray*}
L_{+} \; L_{-} &=& L^{2} - L_{z} (L_{z} - 1) \nonumber \\
L_{-} \; L_{+} &=& L^{2} - L_{z} (L_{z} + 1) \nonumber
\end{eqnarray*}
entonces:
\begin{eqnarray}
\begin{aligned}
L_{+} \; L_{-} \; Y_{\ell m} &= [ L^{2} - L_{z} (L_{z} - 1)] Y_{\ell m} \\
&= [\ell (\ell + 1) - m (m -1)] Y_{\ell m} \\
&= (\ell + m)(\ell - m + 1) Y_{\ell m}
\end{aligned}
\label{eq:ecuacion_8_46}
\end{eqnarray}
análogamente
\begin{equation}
L_{-} \; L_{+} \; Y_{\ell m} = (\ell - m)(\ell + m + 1) Y_{\ell m}
\label{eq:ecuacion_8_47}
\end{equation}
teniendo en cuenta que:
\[ [L_{+} \; Y_{\ell m}, L_{+} \; Y_{\ell m}] = [Y_{\ell m}, L_{-} \; L_{+} \; Y_{\ell m}] \]
y por sustitución a la izquierda de (\ref{eq:ecuacion_8_45}) y a la derecha de (\ref{eq:ecuacion_8_47}), se obtiene
\begin{eqnarray}
\begin{aligned}
 \left[ a_{+} \; Y_{\ell, m+1}, a_{+} \; Y_{\ell, m+1} \right] &= (\ell - m) (\ell + m + 1) [Y_{\ell m}, Y_{\ell m}] \\
&= a^{2}_{\pm} [Y_{\ell m}, Y_{\ell m}] 
\end{aligned}
\label{eq:ecuacion_8_48}
\end{eqnarray}
por lo cual:
\[ a_{+} = \sqrt{(\ell - m)(\ell + m + 1)}, \hspace{1.5cm} a_{-} = \sqrt{(\ell - m)(\ell - m + 1)} \]
en consecuencia:
\begin{eqnarray*}
L_{+} \; Y_{\ell m} &= \sqrt{(\ell - m)(\ell + m + 1)} \; Y_{\ell, m+1} \nonumber \\
L_{-} \; Y_{\ell m} &= \sqrt{(\ell + m)(\ell - m + 1)} \; Y_{\ell, m-1} \nonumber
\end{eqnarray*}
La acción del operador $L_{+}$ (o $L_{-}$) sobre $Y_{\ell m}$ da lugar a
\begin{eqnarray*}
L_{+} \; Y_{\ell m} \Rightarrow Y_{\ell, m+1} \nonumber \\
L_{-} \; Y_{\ell m} \Rightarrow Y_{\ell, m-1} \nonumber
\end{eqnarray*} 
es decir sube (o baja) $m$ en 1. Por ello $L_{+}$ y $L_{-}$ se conocen como \emph{operadores escalera}. Su aplicación más importante es en la mecánica cuántica.







\end{document}