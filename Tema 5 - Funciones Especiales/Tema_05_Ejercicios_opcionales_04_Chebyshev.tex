\documentclass[12pt]{article}
\usepackage[left=0.25cm,top=1cm,right=0.25cm,bottom=1cm]{geometry}
\textwidth = 20cm
\hoffset = -1cm
\usepackage[utf8]{inputenc}
\usepackage[spanish,es-tabla]{babel}
\usepackage[autostyle,spanish=mexican]{csquotes}
\usepackage[tbtags]{amsmath}
\usepackage{nccmath}
\usepackage{amsthm}
\usepackage{amssymb}
\usepackage{graphicx}
\usepackage{standalone}
\usepackage[outdir=./]{epstopdf}
\usepackage{siunitx}
\usepackage{physics}
\usepackage{color}
\usepackage{float}
\usepackage{multicol}
%\usepackage{milista}
\usepackage{enumitem}
\usepackage{anyfontsize}
\usepackage{anysize}
\usepackage{enumitem}
\usepackage{capt-of}
\usepackage{bm}
\usepackage{relsize}
\usepackage{placeins}
\usepackage{empheq}
\usepackage{cancel}
\usepackage{wrapfig}
\spanishdecimal{.}
\renewcommand{\baselinestretch}{1.5} 
\renewcommand\labelenumii{\theenumi.{\arabic{enumii}}}
\newcommand{\ptilde}[1]{\ensuremath{{#1}^{\prime}}}
\newcommand{\stilde}[1]{\ensuremath{{#1}^{\prime \prime}}}
\newcommand{\ttilde}[1]{\ensuremath{{#1}^{\prime \prime \prime}}}
\newcommand{\ntilde}[2]{\ensuremath{{#1}^{(#2)}}}


\title{Polinomios de Chebyshev \\[0.3em]  \large{Ejercicio opcional} \vspace{-3ex}}
\author{M. en C. Gustavo Contreras Mayén}
\date{ }

\begin{document}
\vspace{-4cm}
\maketitle
\fontsize{14}{14}\selectfont

%Ref. Mason (2003) Chebyshev polynomials

Los polinomios de Chebyshev están definidos en el intervalo $[- 1, 1]$, siendo posible definirlos en cualquier rango finito $[a, b]$ para la variable $x$, haciendo que éste rango corresponda al rango $[-1, 1]$ con una nueva variable $s$, ocupando la siguiente transformación lineal:
\begin{align*}
s = \dfrac{2 \, x - (a + b)}{(b - a)}
\end{align*}

Por lo que los polinomios de Chebyshev de primer tipo ajustados al intervalo $[a, b]$ son $T_{n}(s)$, de manera similar se hace el ajuste para los polinomios de segundo tipo  $U_{n} (s)$.
\\[0.5em]
\noindent
\textbf{Ejercicio opcional (22).} Desarrolla la expresión para los polinomios de Chebyshev de primer tipo de grados $0, 1, 2, 3, 4$ ajustados al rango $[-4, 6]$ para $x$.



\end{document}