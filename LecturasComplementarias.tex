\documentclass[12pt]{article}
\usepackage[left=0.3cm,top=2cm,right=0.3cm,bottom=2cm]{geometry}
\usepackage[utf8]{inputenc}
\usepackage[spanish,es-tabla]{babel}
\usepackage{amsmath}
\usepackage{amsthm}
\usepackage{graphicx}
\usepackage{color}
\usepackage{float}
\usepackage{multicol}
\usepackage{enumerate}
\usepackage{anyfontsize}
\usepackage{anysize}
\usepackage{enumitem}
\usepackage{capt-of}
\usepackage{cite}
\spanishdecimal{.}
\setlist[enumerate]{itemsep=0mm}
\renewcommand{\baselinestretch}{1.2}
\let\oldbibliography\thebibliography
\renewcommand{\thebibliography}[1]{\oldbibliography{#1}
\setlength{\itemsep}{0pt}}
%\marginsize{1.5cm}{1.5cm}{0cm}{2cm}
\author{M. en C. Gustavo Contreras Mayén. \texttt{curso.fisica.comp@gmail.com}\\
Fís. Abraham Lima Buendía. \texttt{abraham3081@ciencias.unam.mx}}
\title{Bibliografía de consulta para soluciones de problemas de la tarea \\ {\large MAF Semestre 2017-1 Grupo 8183}}
\date{ }
\begin{document}
\vspace{-4cm}
%\renewcommand\theenumii{\arabic{theenumii.enumii}}
\renewcommand\labelenumii{\theenumi.{\arabic{enumii}}}
\maketitle
\fontsize{14}{14}\selectfont

El siguiente material de consulta les permitirá extender y/o repasar algunos temas que debieron de haber revisado en otros cursos, por lo que se considera como adicional; les será de ayuda para resolver algunos problemas de las tareas del curso.
\\
\\
Los libros se dejarán dentro de la carpeta compartida de dropbox para que puedan consultarlos y/o descargarlos.
\\
\\
\textbf{Tarea I (temas 1 y 2)} 
\\
\begin{enumerate}
\item Del libro \cite{Marsden_1991} el capítulo:
\begin{enumerate}
\item Capítulo 8: Teoremas integrales del análisis vectorial.
\end{enumerate}
\item Del libro \cite{Griffits_1995} los capítulos:
\begin{enumerate}
\item Capítulo 2: The time-independent Schrödinger equation.
\item Capítulo 3: Formalism.
\item Capítulo 4: Quantum mechanics in three dimensions.
\end{enumerate}
\item Del libro \cite{Wangsness_2001} los capítulos:
\begin{enumerate}
\item Capítulo 1: Vectores.
\item Capítulo 9: Condiciones de frontera en una sueprficie de discontinuidad.
\item Capítulo 11: Métodos especiales en electrostática.
\end{enumerate}
\item Del libro \cite{Greiner_electro} el capítulo:
\begin{enumerate}
\item Capítulo 3: Green's Theorem.
\end{enumerate}
\item Del libro \cite{Nikolski} los capítulos:
\begin{enumerate}
\item Capítulo 1: Leyes del electromagnetismo. 
\item Capítulo 2: Campos estáticos, estacionarios y cuasiestacionarios.
\item Capítulo 3: Ondas electromagnéticas.
\item Capítulo 5: Ondas guiadas y volúmenes limitados.
\end{enumerate}
\end{enumerate}
\bibliographystyle{apalike}
\inputencoding{latin2}
\bibliography{LecturasComplementarias}
\end{document}