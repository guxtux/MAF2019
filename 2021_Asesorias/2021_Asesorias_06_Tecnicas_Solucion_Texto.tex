\documentclass[hidelinks,12pt]{article}
\usepackage[left=0.25cm,top=1cm,right=0.25cm,bottom=1cm]{geometry}
%\usepackage[landscape]{geometry}
\textwidth = 20cm
\hoffset = -1cm
\usepackage[utf8]{inputenc}
\usepackage[spanish,es-tabla]{babel}
\usepackage[autostyle,spanish=mexican]{csquotes}
\usepackage[tbtags]{amsmath}
\usepackage{nccmath}
\usepackage{amsthm}
\usepackage{amssymb}
\usepackage{mathrsfs}
\usepackage{graphicx}
\usepackage{subfig}
\usepackage{standalone}
\usepackage[outdir=./Imagenes/]{epstopdf}
\usepackage{siunitx}
\usepackage{physics}
\usepackage{color}
\usepackage{float}
\usepackage{hyperref}
\usepackage{multicol}
%\usepackage{milista}
\usepackage{anyfontsize}
\usepackage{anysize}
%\usepackage{enumerate}
\usepackage[shortlabels]{enumitem}
\usepackage{capt-of}
\usepackage{bm}
\usepackage{relsize}
\usepackage{placeins}
\usepackage{empheq}
\usepackage{cancel}
\usepackage{wrapfig}
\usepackage[flushleft]{threeparttable}
\usepackage{makecell}
\usepackage{fancyhdr}
\usepackage{tikz}
\usepackage{bigints}
\usepackage{scalerel}
\usepackage{pgfplots}
\usepackage{pdflscape}
\pgfplotsset{compat=1.16}
\spanishdecimal{.}
\renewcommand{\baselinestretch}{1.5} 
\renewcommand\labelenumii{\theenumi.{\arabic{enumii}})}
\newcommand{\ptilde}[1]{\ensuremath{{#1}^{\prime}}}
\newcommand{\stilde}[1]{\ensuremath{{#1}^{\prime \prime}}}
\newcommand{\ttilde}[1]{\ensuremath{{#1}^{\prime \prime \prime}}}
\newcommand{\ntilde}[2]{\ensuremath{{#1}^{(#2)}}}

\newtheorem{defi}{{\it Definición}}[section]
\newtheorem{teo}{{\it Teorema}}[section]
\newtheorem{ejemplo}{{\it Ejemplo}}[section]
\newtheorem{propiedad}{{\it Propiedad}}[section]
\newtheorem{lema}{{\it Lema}}[section]
\newtheorem{cor}{Corolario}
\newtheorem{ejer}{Ejercicio}[section]

\newlist{milista}{enumerate}{2}
\setlist[milista,1]{label=\arabic*)}
\setlist[milista,2]{label=\arabic{milistai}.\arabic*)}
\newlength{\depthofsumsign}
\setlength{\depthofsumsign}{\depthof{$\sum$}}
\newcommand{\nsum}[1][1.4]{% only for \displaystyle
    \mathop{%
        \raisebox
            {-#1\depthofsumsign+1\depthofsumsign}
            {\scalebox
                {#1}
                {$\displaystyle\sum$}%
            }
    }
}
\def\scaleint#1{\vcenter{\hbox{\scaleto[3ex]{\displaystyle\int}{#1}}}}
\def\bs{\mkern-12mu}


\title{Métodos de solución para las EDP} \vspace{-3ex}
\author{M. en C. Gustavo Contreras Mayén}
\date{ }
\newcommand{\Cancel}[2][black]{{\color{#1}\cancel{\color{black}#2}}}
\begin{document}
\vspace{-4cm}
\maketitle
\fontsize{14}{14}\selectfont
\tableofcontents
\newpage

\section{Métodos de solución para las EDP.}
\subsection{Lista de métodos.}

\textcolor{blue}{\textbf{¿Cómo se resuelve una EDP?}}
\par
Esta es una buena pregunta que debemos de plantearnos. Resulta que hay conjunto amplio de métodos disponibles para resolver las EDP; los métodos \emph{más importantes son los que convierten las EDP en EDO}, ya que simplifican el manejo y su solución.
\par
A continuación se presenta una lista con $10$ técnicas de solución para EDP.

No es una lista definitiva, pero nos servirá de referencia para contemplar otras estrategias de solución.

Técnicas de solución:
\begin{enumerate}
\item \emph{Separación de variables}.
\\
Esta técnica reduce una EDP de $n$ variables, a un sistema de $n$ EDO.
\item \emph{Transformadas integrales}.
\\
Este procedimiento reduce una EDP de $n$ variables independientes a una de $n - 1$ variables; por lo tanto, una EDP en dos variables podría cambiarse a una EDO.
\item \emph{Cambio de coordenadas}.
\\
Este método cambia la EDP original a una EDO o bien a otra EDP (una más fácil) cambiando las coordenadas del problema (rotando el eje o transformaciones similares).
\item \emph{Transformación de la variable dependiente}.
\\
Este método transforma la variable incógnita de una EDP en una nueva incógnita que es más fácil de encontrar.
\item \emph{Métodos numéricos}. 
\\
Estos métodos cambian una EDP a un sistema de ecuaciones en diferencias que puede resolverse mediante un algoritmo con técnicas iterativas en una computadora; en muchos casos, esta es la única técnica que funcionará. 
\par
Además de los métodos que reemplazan las EDP por ecuaciones en diferencias, existen otros métodos que intentan aproximar soluciones mediante curvas polinomiales (aproximaciones spline).
\item \emph{Métodos de perturbación}.
\\
Este método convierte un problema no lineal en una secuencia de problemas lineales que se aproxima al no lineal.
\item \emph{Técnica impulso-respuesta}.
\\
Este procedimiento descompone las condiciones iniciales y de frontera del problema en impulsos simples y encuentra la respuesta a cada impulso. La respuesta general se encuentra luego agregando estas respuestas simples.
\item \emph{Ecuaciones integrales}.
\\
Esta técnica cambia una EDP a una ecuación integral (una ecuación donde la incógnita está dentro de la integral). Luego, la ecuación integral se resuelve mediante varias técnicas.
\item \emph{Métodos de cálculo de variaciones}.
\\
Estos métodos encuentran la solución a las EDP reformulando la ecuación como un problema de minimización. Resulta que el mínimo de cierta expresión (muy probablemente la expresión representará la energía total) también es la solución a la EDP.
\item \emph{Expansión de funciones propias (eigenfunciones)}.
\\
Este método intenta encontrar la solución de una EDP como una suma infinita de funciones propias. Estas funciones propias se encuentran resolviendo lo que se conoce como un problema de valores propios correspondiente al problema original.
\end{enumerate}

Es por ello que el título de este Tema 2 es Primeras técnicas de solución, ya que abordaremos tres estrategias para resolver EDP.
\par
En el caso que nos encontremos una EDP muy específica en donde no sea posible resolverla mediante alguna de las primeras técnicas, tendremos que ocupar una estrategia particular.
\end{document}