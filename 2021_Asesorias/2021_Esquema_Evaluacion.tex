\documentclass[12pt]{article}
\usepackage[left=0.25cm,top=1cm,right=0.25cm,bottom=1cm]{geometry}
\textwidth = 20cm
\hoffset = -1cm
\usepackage[utf8]{inputenc}
\usepackage[spanish,es-tabla]{babel}
\usepackage[autostyle,spanish=mexican]{csquotes}
\usepackage[tbtags]{amsmath}
\usepackage{nccmath}
\usepackage{amsthm}
\usepackage{amssymb}
\usepackage{graphicx}
\usepackage{standalone}
\usepackage[outdir=./]{epstopdf}
\usepackage{siunitx}
\usepackage{physics}
\usepackage{color}
\usepackage{float}
\usepackage{multicol}
%\usepackage{milista}
\usepackage{enumitem}
\usepackage{anyfontsize}
\usepackage{anysize}
\usepackage{enumitem}
\usepackage{capt-of}
\usepackage{bm}
\usepackage{relsize}
\usepackage{placeins}
\usepackage{empheq}
\usepackage{cancel}
\usepackage{wrapfig}
\spanishdecimal{.}
\renewcommand{\baselinestretch}{1.5} 
\renewcommand\labelenumii{\theenumi.{\arabic{enumii}}}
\newcommand{\ptilde}[1]{\ensuremath{{#1}^{\prime}}}
\newcommand{\stilde}[1]{\ensuremath{{#1}^{\prime \prime}}}
\newcommand{\ttilde}[1]{\ensuremath{{#1}^{\prime \prime \prime}}}
\newcommand{\ntilde}[2]{\ensuremath{{#1}^{(#2)}}}


\title{Esquema de evaluación para el curso} \vspace{-3ex}
\author{M. en C. Gustavo Contreras Mayén}
\date{\today}
\newcommand{\Cancel}[2][black]{{\color{#1}\cancel{\color{black}#2}}}
\begin{document}
\vspace{-4cm}
\maketitle
\fontsize{14}{14}\selectfont


\section{Esquema de evaluación.}
\subsection{Temas a revisar.}

Cuando se comentó el plan de asesorías para ir reactivando el ritmo de trabajo, se mencionó el incluir una serie de ejercicios para revisión.
\par
Los ejercicios servirán de apoyo para resolver ejercicios nuevos en cada tema.
\par
Los temas para los que tendríamos ejercicios son:
\begin{enumerate}
\item Tema 1 - La física y la geometría.
\item Primeras técnicas de solución.
\item Funciones especiales.
\item Transformadas integrales.
\end{enumerate}

Para el Tema 1 que se revisó al inicio del semestre y que se repasó en las asesorías, se tendrán $5$ nuevos ejercicios a resolver.
\par
La fecha límite de entrega de todos los ejercicios se mencionará más adelante.
\par
Para el tema de Técnicas de solución, ya revisamos la técnica de separación de variables y estamos completando la solución en series de potencias que estamos atendiendo, los ejercicios a resolver serán $5$.
\par
Para la parte de funciones especiales y transformadas integrales, tendrán para cada uno de ellos, $5$ ejercicios a resolver. Teniendo un total de $20$ ejercicios a resolver.
\par
La fecha de entrega, criterios de calificación y escala de puntaje se revisará en un momento.


\section{Material de consulta.}
\subsection{Archivos y videos.}

Para los dos siguientes temas proponemos lo siguiente:
\par
En las semanas de vacaciones administrativas, del 5 al 23 de julio, se estarán publicando materiales de consulta y videos en la plataforma Moodle y en el canal de YouTube, para que se puedan consultar de manera previa, y así adelantar la revisión.
\par
En el regreso de las vacaciones, durante las siguientes tres semanas continuaremos con las sesiones de videoconferencia los días lunes, miércoles y viernes con el siguiente cronograma:
\begin{enumerate}
\item Funciones especiales: 26, 28, 30 de julio y 2, 4, 6 de agosto.
\item Transformadas integrales: 9, 11, 13 de agosto.
\end{enumerate}

Dejando la última semana del 16 al 20 de agosto, para evaluar la última parte y promediar sus calificaciones de los ejercicios.
\par
Si se consultan de manera previa los materiales, se contaría con una ventaja para adelantar la solución de los ejercicios.
\par
En las semanas señaladas, se revisarían los temas indicados, es decir, trabajaríamos las sesiones con materiales adicionales, que complementarían lo que estaría en las plataformas y se entregaría la lista con ejercicios para estos temas.

\section{Entrega de ejercicios}
\subsection{Fecha límite de entrega}

Para los temas de la física y la geometría, técnicas de solución y funciones especiales, \textbf{los ejercicios a resolver se deberán de entregar a más tardar el viernes 13 de agosto.}
\par
Para el tema de transformadas integrales, \textbf{los ejercicios a resolver se deberán de entregar a más tardar el lunes 16 de agosto.}

\section{Puntaje y escala de calificación.}
\subsection{Puntaje de los ejercicios.}

A cada uno de los ejercicios se le otorgará $1$ punto, siempre y cuando esté bien resuelto.
\par
En caso de tener una respuesta parcial, el ejercicio tendrá un puntaje proporcional que sumará a la calificación.
\par
Es claro que mientras más ejercicios resueltos entreguen, la calificación de cada ejercicio bien resuelto, aportará puntaje para el promedio.

\subsection{Escala de calificación.}

El puntaje mínimo para obtener una calificación aprobatoria es de $10$ puntos.
\par
Si entregan $10$ ejercicios y están bien resueltos, cumplirían el mínimo de puntaje.  Pero si en algún ejercicio no se logra el punto completo, ya no se alcanzaría el puntaje mínimo para acreditar el curso.

\begin{table}[H]
\Large
\centering
\begin{tabular}{l | l}
Puntaje & Calificación \\ \hline
$< 10$ & No acredita \\ 
$10$ puntos (mínimo) a $14$ & $8$ Ocho \\ 
De $15$  a $18$ puntos & $9$ Nueve \\ 
De $19$  a $20$ puntos & $10$ Diez \\
\end{tabular}
\end{table}

En caso de que el ejercicio no resulte con un $1$ punto completo, la suma del puntaje de los ejercicios entregados \textcolor{red}{se mantendrá incluyendo los decimales}.
\par

Como ejemplo: Entregué $16$ ejercicios y el puntaje que obtuve fue de $14.7$.  El puntaje no \enquote{sube} a $15$, por lo que la calificación obtenida será de $8$ ocho.

\section{Continuamos con el trabajo.}
\subsection{Decisión personal.}

Esperamos que esta información te sea lo más completa posible para que cuentes con los elementos que te permitan tomar la decisión que consideres más pertinente.

De acuerdo con el comunicado del Consejo Técnico de la Facultad, el día 31 de julio es el último día para realizar la solicitud de baja del curso.
\par
En el supuesto de que no se realice la baja en la fecha señalada, la situación es de que sigues inscrito en el curso, por tanto se deberían de atender las actividades para la evaluación, pero en el caso de que no se entreguen los ejercicios, la calificación que se asentará en el acta del curso, será \textcolor{blue}{No Presentó -  NP}.  Que no afecta en el promedio, pero si en el conteo de veces que se ha inscrito el curso (para el caso de una inscripción ordinaria)
\par
Se continuará con la sesión del próximo viernes 2 de julio a las 3 pm, para revisar un ejercicio y dar las indicaciones sobre los materiales de consulta y ejercicios a resolver del Tema 1 y del tema de Técnicas de solución.
\end{document}