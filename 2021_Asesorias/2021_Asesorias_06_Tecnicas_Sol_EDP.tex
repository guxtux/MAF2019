\documentclass[12pt]{beamer}
\usepackage{../Estilos/BeamerMAF}
%Sección para el tema de beamer, con el theme, usercolortheme y sección de footers
\usetheme{CambridgeUS}
\usecolortheme{beaver}
%\useoutertheme{default}
\setbeamercovered{invisible}
% or whatever (possibly just delete it)
\setbeamertemplate{section in toc}[sections numbered]
\setbeamertemplate{subsection in toc}[subsections numbered]
\setbeamertemplate{subsection in toc}{\leavevmode\leftskip=3.2em\rlap{\hskip-2em\inserttocsectionnumber.\inserttocsubsectionnumber}\inserttocsubsection\par}
\setbeamercolor{section in toc}{fg=blue}
\setbeamercolor{subsection in toc}{fg=blue}
\setbeamercolor{frametitle}{fg=blue}
\setbeamertemplate{caption}[numbered]

\setbeamertemplate{footline}
\beamertemplatenavigationsymbolsempty
\setbeamertemplate{headline}{}


\makeatletter
\setbeamercolor{section in foot}{bg=gray!30, fg=black!90!orange}
\setbeamercolor{subsection in foot}{bg=blue!30!yellow, fg=red}
\setbeamercolor{date in foot}{bg=black, fg=white}
\setbeamertemplate{footline}
{
  \leavevmode%
  \hbox{%
  \begin{beamercolorbox}[wd=.333333\paperwidth,ht=2.25ex,dp=1ex,center]{section in foot}%
    \usebeamerfont{section in foot} \insertsection
  \end{beamercolorbox}%
  \begin{beamercolorbox}[wd=.333333\paperwidth,ht=2.25ex,dp=1ex,center]{subsection in foot}%
    \usebeamerfont{subsection in foot}  \insertsubsection
  \end{beamercolorbox}%
  \begin{beamercolorbox}[wd=.333333\paperwidth,ht=2.25ex,dp=1ex,right]{date in head/foot}%
    \usebeamerfont{date in head/foot} \insertshortdate{} \hspace*{2em}
    \insertframenumber{} / \inserttotalframenumber \hspace*{2ex} 
  \end{beamercolorbox}}%
  \vskip0pt%
}
\makeatother\newlength{\depthofsumsign}
\setlength{\depthofsumsign}{\depthof{$\sum$}}
\newcommand{\nsum}[1][1.4]{% only for \displaystyle
    \mathop{%
        \raisebox
            {-#1\depthofsumsign+1\depthofsumsign}
            {\scalebox
                {#1}
                {$\displaystyle\sum$}%
            }
    }
}
\def\scaleint#1{\vcenter{\hbox{\scaleto[3ex]{\displaystyle\int}{#1}}}}
\def\bs{\mkern-12mu}





\date{}
\title{Métodos de solución para las EDP}
\author{M. en C. Gustavo Contreras Mayén}

\begin{document}
\maketitle
\fontsize{14}{14}\selectfont
\spanishdecimal{.}

\section*{Contenido}
\frame{\tableofcontents[currentsection, hideallsubsections]}

\section{Métodos de solución para las EDP}
\frame{\tableofcontents[currentsection, hideothersubsections]}
\subsection{Lista de métodos}

\begin{frame}
\frametitle{Técnicas de solución}
\textcolor{blue}{\textbf{¿Cómo se resuelve una EDP?}}
\\
\bigskip
\pause
Esta es una buena pregunta que debemos de plantearnos. Resulta que hay conjunto amplio de métodos disponibles para resolver las EDP; los métodos \emph{más importantes son los que convierten las EDP en EDO}, ya que simplifican el manejo y su solución.
\end{frame}
\begin{frame}
\frametitle{Lista de 10 técnicas de solución}
A continuación se presenta una lista con $10$ técnicas de solución para EDP.
\\
\bigskip
No es una lista definitiva, pero nos servirá de referencia para contemplar otras estrategias de solución.
\end{frame}
\begin{frame}
\frametitle{Técnicas de solución}
\setbeamercolor{item projected}{bg=blue!70!black,fg=yellow}
\setbeamertemplate{enumerate items}[circle]
\begin{enumerate}
\item \emph{Separación de variables}.
\\
\bigskip
Esta técnica reduce una EDP de $n$ variables, a un sistema de $n$ EDO.
\seti
\end{enumerate}
\end{frame}
\begin{frame}
\frametitle{Técnicas de solución}
\setbeamercolor{item projected}{bg=blue!70!black,fg=yellow}
\setbeamertemplate{enumerate items}[circle]
\begin{enumerate}
\conti
\item \emph{Transformadas integrales}. 
\\
\bigskip
Este procedimiento reduce una EDP de $n$ variables independientes a una de $n - 1$ variables; por lo tanto, una EDP en dos variables podría cambiarse a una EDO.
\seti
\end{enumerate}
\end{frame}
\begin{frame}
\frametitle{Técnicas de solución}
\setbeamercolor{item projected}{bg=blue!70!black,fg=yellow}
\setbeamertemplate{enumerate items}[circle]
\begin{enumerate}
\conti
\item \emph{Cambio de coordenadas}.
\\
\bigskip
Este método cambia la EDP original a una EDO o bien a otra EDP (una más fácil) cambiando las coordenadas del problema (rotando el eje o transformaciones similares).
\seti
\end{enumerate}
\end{frame}
\begin{frame}
\frametitle{Técnicas de solución}
\setbeamercolor{item projected}{bg=blue!70!black,fg=yellow}
\setbeamertemplate{enumerate items}[circle]
\begin{enumerate}
\conti
\item \emph{Transformación de la variable dependiente}.
\\
\bigskip
Este método transforma la variable incógnita de una EDP en una nueva incógnita que es más fácil de encontrar.
\seti
\end{enumerate}
\end{frame}
\begin{frame}
\frametitle{Técnicas de solución}
\setbeamercolor{item projected}{bg=blue!70!black,fg=yellow}
\setbeamertemplate{enumerate items}[circle]
\begin{enumerate}
\conti
\item \emph{Métodos numéricos}. 
\\
\bigskip
Estos métodos cambian una EDP a un sistema de ecuaciones en diferencias que puede resolverse mediante un algoritmo con técnicas iterativas en una computadora; en muchos casos, esta es la única técnica que funcionará. 
\seti
\end{enumerate}
\end{frame}
\begin{frame}
\frametitle{Métodos numéricos}
Además de los métodos que reemplazan las EDP por ecuaciones en diferencias, existen otros métodos que intentan aproximar soluciones mediante curvas polinomiales (aproximaciones spline).
\end{frame}
\begin{frame}
\frametitle{Técnicas de solución}
\setbeamercolor{item projected}{bg=blue!70!black,fg=yellow}
\setbeamertemplate{enumerate items}[circle]
\begin{enumerate}
\conti
\item \emph{Métodos de perturbación}.
\\
\bigskip
Este método convierte un problema no lineal en una secuencia de problemas lineales que se aproxima al no lineal.
\seti
\end{enumerate}
\end{frame}
\begin{frame}
\frametitle{Técnicas de solución}
\setbeamercolor{item projected}{bg=blue!70!black,fg=yellow}
\setbeamertemplate{enumerate items}[circle]
\begin{enumerate}
\conti
\item \emph{Técnica impulso-respuesta}.
\\
\bigskip
Este procedimiento descompone las condiciones iniciales y de frontera del problema en impulsos simples y encuentra la respuesta a cada impulso. La respuesta general se encuentra luego agregando estas respuestas simples.
\seti
\end{enumerate}
\end{frame}
\begin{frame}
\frametitle{Técnicas de solución}
\setbeamercolor{item projected}{bg=blue!70!black,fg=yellow}
\setbeamertemplate{enumerate items}[circle]
\begin{enumerate}
\conti
\item \emph{Ecuaciones integrales}.
\\
\bigskip
Esta técnica cambia una EDP a una ecuación integral (una ecuación donde la incógnita está dentro de la integral). Luego, la ecuación integral se resuelve mediante varias técnicas.
\seti
\end{enumerate}
\end{frame}
\begin{frame}
\frametitle{Técnicas de solución}
\setbeamercolor{item projected}{bg=blue!70!black,fg=yellow}
\setbeamertemplate{enumerate items}[circle]
\begin{enumerate}
\conti
\item \emph{Métodos de cálculo de variaciones}.
\\
\bigskip
Estos métodos encuentran la solución a las EDP reformulando la ecuación como un problema de minimización. Resulta que el mínimo de cierta expresión (muy probablemente la expresión representará la energía total) también es la solución a la EDP.
\seti
\end{enumerate}
\end{frame}
\begin{frame}
\frametitle{Técnicas de solución}
\setbeamercolor{item projected}{bg=blue!70!black,fg=yellow}
\setbeamertemplate{enumerate items}[circle]
\begin{enumerate}
\conti
\item \emph{Expansión de funciones propias (eigenfunciones)}.
\\
\bigskip
Este método intenta encontrar la solución de una EDP como una suma infinita de funciones propias. Estas funciones propias se encuentran resolviendo lo que se conoce como un problema de valores propios correspondiente al problema original.
\end{enumerate}
\end{frame}
\begin{frame}
\frametitle{Primeras técnicas}
Es por ello que el título de este Tema 2 es Primeras técnicas de solución, ya que abordaremos tres estrategias para resolver EDP.
\\
\bigskip
\pause
En el caso que nos encontremos una EDP muy específica en donde no sea posible resolverla mediante alguna de las primeras técnicas, tendremos que ocupar una estrategia particular.
\end{frame}


\end{document}