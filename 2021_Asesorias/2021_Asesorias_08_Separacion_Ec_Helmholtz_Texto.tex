\documentclass[12pt]{article}
\usepackage[left=0.25cm,top=1cm,right=0.25cm,bottom=1cm]{geometry}
\textwidth = 20cm
\hoffset = -1cm
\usepackage[utf8]{inputenc}
\usepackage[spanish,es-tabla]{babel}
\usepackage[autostyle,spanish=mexican]{csquotes}
\usepackage[tbtags]{amsmath}
\usepackage{nccmath}
\usepackage{amsthm}
\usepackage{amssymb}
\usepackage{graphicx}
\usepackage{standalone}
\usepackage[outdir=./]{epstopdf}
\usepackage{siunitx}
\usepackage{physics}
\usepackage{color}
\usepackage{float}
\usepackage{multicol}
%\usepackage{milista}
\usepackage{enumitem}
\usepackage{anyfontsize}
\usepackage{anysize}
\usepackage{enumitem}
\usepackage{capt-of}
\usepackage{bm}
\usepackage{relsize}
\usepackage{placeins}
\usepackage{empheq}
\usepackage{cancel}
\usepackage{wrapfig}
\spanishdecimal{.}
\renewcommand{\baselinestretch}{1.5} 
\renewcommand\labelenumii{\theenumi.{\arabic{enumii}}}
\newcommand{\ptilde}[1]{\ensuremath{{#1}^{\prime}}}
\newcommand{\stilde}[1]{\ensuremath{{#1}^{\prime \prime}}}
\newcommand{\ttilde}[1]{\ensuremath{{#1}^{\prime \prime \prime}}}
\newcommand{\ntilde}[2]{\ensuremath{{#1}^{(#2)}}}


\title{Separación de variables \\ Ecuación de Helmholtz} \vspace{-3ex}
\author{M. en C. Gustavo Contreras Mayén}
\date{ }
\newcommand{\Cancel}[2][black]{{\color{#1}\cancel{\color{black}#2}}}
\begin{document}
\vspace{-4cm}
\maketitle
\fontsize{14}{14}\selectfont
\tableofcontents
\newpage

\section{Enunciado.}

\subsection{El ejercicio a resolver.}

Demostrar que la ecuación de Helmholtz:
\begin{align*}
\laplacian{\psi} + k^{2} \, \psi = 0
\end{align*}

\emph{es separable} en coordenadas cilíndricas circulares,  si $k^{2}$ se generaliza como:

\begin{align*}
k^{2} + f(\rho) + \left( \dfrac{1}{\rho^{2}} \right) \, g(\varphi) + h(z)
\end{align*}

La ecuación que debemos de demostrar que es separable es:
\begin{align*}
\laplacian{\psi} + \left[ k^{2} + f(\rho) + \left( \dfrac{1}{\rho^{2}} \right) \, g(\varphi) + h(z) \right] \, \psi = 0
\end{align*}

Debemos de utilizar el operador laplaciano en el sistema de coordenadas cilíndricas circulares.
\par
Recordemos que tenemos una expresión que nos determina el operador diferencial laplaciano en términos de un sistema de coordenadas generalizado.
\par
Será necesario contar con las reglas de transformación así como de los factores de escala de ese sistema, para obtener el operador, todo esto lo recuperamos del Tema 1.
\par
Tenemos entonces que la expresión resulta ser:
\begin{align*}
&{} \dfrac{1}{\rho} \, \pdv{\rho} \left( \rho \, \pdv{\psi}{\rho} \right) + \dfrac{1}{\rho^{2}} \, \pdv[2]{\psi}{\varphi} + \pdv[2]{\psi}{z} + \\[1em]
&+ \left[ k^{2} + f(\rho) + \left( \dfrac{1}{\rho^{2}} \right) \, g(\varphi) + h(z) \right] \, \psi = 0
\end{align*}

\section{Resolviendo el problema.}

\subsection{Propuesta de solución.}

Para aplicar el método de separación de variables, proponemos la siguiente solución:
\begin{align*}
\psi (\rho, \varphi, z) = R(\rho) \, \Phi (\varphi) \, Z(z)
\end{align*}

donde cada función con letra mayúscula, depende de una sola variable.


\subsection{Obteniendo las derivadas.}

Como ya propusimos una solución, ahora hay que calcular las derivadas parciales y sustituirlas en la ecuación de Helmholtz.
\par
Haremos en este ejercicio el procedimiento de diferenciación.
\par
Comenzamos con las derivadas parciales con respecto a $\rho$:
\begin{align*}
\rho \, \pdv{\psi}{\rho} =  \rho \, \pdv{\rho} R(\rho) \Phi (\varphi) Z(z) =  \rho \left( \ptilde{R} \, \Phi Z \, \right)
\end{align*}

Notemos que:
\begin{enumerate}
\item Cada función con mayúsculas depende de una sola variable.
\item El primado indica que tenemos una derivada ordinaria de la función con respecto a su variable.
\end{enumerate}

Continuamos derivando con respecto a $\rho$, la siguiente derivada parcial es:
\begin{align*}
\dfrac{1}{\rho} \pdv{\rho} \left( \rho \, \pdv{\psi}{\rho}  \right) =  \stilde{R} \, \Phi \, Z + \dfrac{1}{\rho} \, \ptilde{R} \, \Phi \, Z
\end{align*}

Ya concluimos las diferenciaciones con respecto a $\rho$, por lo que podemos continuar con las otras dos variables.
\par
Para las otras dos derivadas, tendremos entonces:
\begin{align*}
\dfrac{1}{\rho^{2}} \, \pdv[2]{\psi}{\varphi} &=  \dfrac{1}{\rho^{2}} \, \pdv[2]{\varphi} \, R(\rho) \Phi (\varphi) Z(z) =  \dfrac{1}{\rho^{2}} \, R \, \stilde{\Phi} \, Z \\[1em] 
\pdv[2]{\psi}{z} &=  \pdv[2]{z} \, R(\rho) \Phi (\varphi) Z(z) =  R \, \Phi \, \stilde{Z}
\end{align*}

Al incluir las derivadas que hemos obtenido, la ecuación resulta ser:
\begin{align*}
&{} \stilde{R} \, \Phi \, Z + \dfrac{1}{\rho} \, \ptilde{R} \, \Phi \, Z + \dfrac{1}{\rho^{2}} R \, \stilde{\Phi} \, Z + R \, \Phi \, \stilde{Z} + \\[1em]
&+ \left[ k^{2} + f(\rho) + \left( \dfrac{1}{\rho^{2}} \right) \, g(\varphi) + h(z) \right] \, R \, \Phi \, Z = 0
\end{align*}

\subsection{Siguiente paso: dividir entre la solución.}

El siguiente paso que debemos de realizar es: dividir toda la expresión entre la solución propuesta, es decir, entre: $R \, \Phi \, Z$, por lo que tenemos:

\begin{align*}
&{} \dfrac{1}{R} \left( \stilde{R} + \dfrac{1}{\rho} \, \ptilde{R} \right) + \dfrac{1}{\rho^{2}} \, \dfrac{\stilde{\Phi}}{\Phi} + \dfrac{\stilde{Z}}{Z} + \\[1em]
&+ k^{2} + f(\rho) + \left( \dfrac{1}{\rho^{2}} \right) \, g(\varphi) + h(z) = 0    
\end{align*}

\subsection{Constantes de separación.}

Procedemos a reacomodar los términos por cada una de las variables, buscando que haya una dependencia en una sola de ellas en la expresión.
\par
Veamos lo qué pasa cuando pasamos los términos que involucran a la variable $z$ al lado derecho de la expresión. Entonces vemos que:
\begin{align*}
&{} \dfrac{1}{R} \left( \stilde{R} + \dfrac{1}{\rho} \, \ptilde{R} \right) + \dfrac{1}{\rho^{2}} \, \dfrac{\stilde{\Phi}}{\Phi} + k^{2} + \\[1em]
&+ f(\rho) + \left( \dfrac{1}{\rho^{2}} \right) \, g(\varphi) = - \dfrac{\stilde{Z}}{Z} - h(z)
\end{align*}

La expresión del lado izquierdo depende solo de $\rho$ y de $\varphi$, mientras que la del lado derecho depende solo de $z$.
\par
Para que la igualdad anterior se mantenga, la única manera posible es que ambos lados de la expresión sean iguales a una constante, en este caso, la \emph{primera constante de separación}:
\begin{align*}
- \dfrac{\stilde{Z}}{Z} - h(z) = \alpha^{2}
\end{align*}

Reacomodamos los términos por cada variable, al ocupar la primera constante de separación $\alpha^{2}$, la ecuación queda expresada como:
\begin{align*}
&{} \dfrac{1}{R} \left( \stilde{R} + \dfrac{1}{\rho} \, \ptilde{R} \right) +  f(\rho) + \left( \dfrac{1}{\rho^{2}} \right) \, g(\varphi) + \\[1em]
&+ \dfrac{1}{\rho^{2}} \, \dfrac{\stilde{\Phi}}{\Phi} + k^{2} - \alpha^{2} = 0
\end{align*}

La expresión anterior ya depende solo de las variables $\rho$ y $\varphi$, que son independientes entre sí.
\par
Por lo que podemos repetir la separación de la ecuación, dejando en cada lado de la igualdad una variable.

Tendremos ahora que:

\begin{align*}
\dfrac{1}{R} \left( \stilde{R} + \dfrac{1}{\rho} \, \ptilde{R} \right) + f(\rho) + k^{2} - \alpha^{2} = - \dfrac{1}{\rho^{2}} \, \dfrac{\stilde{\Phi}}{\Phi} - \left( \dfrac{1}{\rho^{2}} \right) \, g(\varphi)
\end{align*}

Aunque hay un factor $1/\rho^{2}$ en el lado derecho de la igualdad, para tener una dependencia solo de $\varphi$ hay que cancelar ese término,  por lo que multiplicamos toda la expresión por $\rho^{2}$.

Multiplicando por el factor mencionado:
\begin{align*}
\dfrac{\rho}{R} \left( \stilde{R} + \dfrac{1}{\rho} \, \ptilde{R} \right) + \rho^{2} \bigg[ f(\rho) + k^{2} - \alpha^{2} \bigg] = - \dfrac{\stilde{\Phi}}{\Phi} - g(\varphi)
\end{align*}

Que ahora si ya tenemos de cada lado de la igualdad, la dependencia de una sola variable.
\par
Como vimos anteriormente, para que esto sea válido, la única manera es que las funciones sean iguales a una constante:  la \emph{segunda constante de separación}.
\begin{align*}
- \dfrac{\stilde{\Phi}}{\Phi} - g(\varphi) = \beta^{2}
\end{align*}

Hemos demostrado que la ecuación de Helmholtz en coordenadas cilíndricas con la $k^{2}$ generalizada:

\begin{align*}
\dfrac{\rho}{R} \left( \stilde{R} + \dfrac{1}{\rho} \, \ptilde{R} \right) + \rho^{2} \bigg[ f(\rho) + k^{2} - \alpha^{2} \bigg] - \beta^{2} = 0
\end{align*}

es una \textcolor{blue}{ecuación separable}.

\subsection{Un extra.}

La ecuación diferencial de Helmholtz se puede resolver mediante la separación de variables en 11 sistemas coordenados.
\par
Donde 10 sistemas (con excepción del sistema coordenado paraboloide confocal) son casos particulares del sistema coordenado elipsoidal confocal.

Los sistemas en donde es separable la ecuación de Helmholtz son:
\begin{enumerate}
\item Cartesiano.
\item Elipsoidal confocal.
\item Paraboloide confocal.
\item Cónico.
\item Cilíndrico.
\item Cilíndrico elíptico.
\item Esferoidal oblato.
\item Esferoidal prolato.
\item Cilíndrico parabólico.
\item Parabólico.
\item Esférico.
\end{enumerate}


Si $k = 0$, recuperamos la ecuación de Laplace, que es separable en otros dos sistemas coordenados:
\begin{enumerate}
\item Biesférico.
\item Toroidal.
\end{enumerate}

¿Aceptas el reto de elegir al azar dos sistemas que hemos mencionado y concluir que se cumple este hecho?

\subsection{Ejercicio a resolver.}

Con la finalidad de repasar lo que hemos trabajado en este ejercicio, te pedimos que demuestres que la ecuación de Helmholtz \emph{es separable} en un sistema de coordenadas esférico
\begin{align*}
\laplacian{\psi} + k^{2} \, \psi = 0
\end{align*}
si $k^{2}$ se generaliza como
\begin{align*}
k^{2} + f(r) + \dfrac{1}{r^{2}} \, g(\theta) + \dfrac{1}{r^{2} \, \sin \theta} \, h(\varphi)
\end{align*}

Buscando que haya un repaso completo de lo que hemos visto, te pedimos que ocupes en este sistema coordenado esférico:
\begin{enumerate}
\item Las reglas de transformación de $(x, y, z) \to (r, \theta, \varphi)$
\item Calcules los factores de escala.
\item Obtengas el operador laplaciano.
\item Ocupes una propuesta de solución
\begin{align*}
\psi(r, \theta, \varphi) = R(r)\, T(\theta) \, F(\varphi)
\end{align*}
\item Obtengas las correspondientes constantes de separación.
\item Concluyas que la ecuación admite la separación de variables.
\end{enumerate}
\end{document}