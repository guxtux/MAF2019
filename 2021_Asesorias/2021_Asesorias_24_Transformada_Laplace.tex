\documentclass[hidelinks,12pt]{article}
\usepackage[left=0.25cm,top=1cm,right=0.25cm,bottom=1cm]{geometry}
%\usepackage[landscape]{geometry}
\textwidth = 20cm
\hoffset = -1cm
\usepackage[utf8]{inputenc}
\usepackage[spanish,es-tabla]{babel}
\usepackage[autostyle,spanish=mexican]{csquotes}
\usepackage[tbtags]{amsmath}
\usepackage{nccmath}
\usepackage{amsthm}
\usepackage{amssymb}
\usepackage{mathrsfs}
\usepackage{graphicx}
\usepackage{subfig}
\usepackage{standalone}
\usepackage[outdir=./Imagenes/]{epstopdf}
\usepackage{siunitx}
\usepackage{physics}
\usepackage{color}
\usepackage{float}
\usepackage{hyperref}
\usepackage{multicol}
%\usepackage{milista}
\usepackage{anyfontsize}
\usepackage{anysize}
%\usepackage{enumerate}
\usepackage[shortlabels]{enumitem}
\usepackage{capt-of}
\usepackage{bm}
\usepackage{relsize}
\usepackage{placeins}
\usepackage{empheq}
\usepackage{cancel}
\usepackage{wrapfig}
\usepackage[flushleft]{threeparttable}
\usepackage{makecell}
\usepackage{fancyhdr}
\usepackage{tikz}
\usepackage{bigints}
\usepackage{scalerel}
\usepackage{pgfplots}
\usepackage{pdflscape}
\pgfplotsset{compat=1.16}
\spanishdecimal{.}
\renewcommand{\baselinestretch}{1.5} 
\renewcommand\labelenumii{\theenumi.{\arabic{enumii}})}
\newcommand{\ptilde}[1]{\ensuremath{{#1}^{\prime}}}
\newcommand{\stilde}[1]{\ensuremath{{#1}^{\prime \prime}}}
\newcommand{\ttilde}[1]{\ensuremath{{#1}^{\prime \prime \prime}}}
\newcommand{\ntilde}[2]{\ensuremath{{#1}^{(#2)}}}

\newtheorem{defi}{{\it Definición}}[section]
\newtheorem{teo}{{\it Teorema}}[section]
\newtheorem{ejemplo}{{\it Ejemplo}}[section]
\newtheorem{propiedad}{{\it Propiedad}}[section]
\newtheorem{lema}{{\it Lema}}[section]
\newtheorem{cor}{Corolario}
\newtheorem{ejer}{Ejercicio}[section]

\newlist{milista}{enumerate}{2}
\setlist[milista,1]{label=\arabic*)}
\setlist[milista,2]{label=\arabic{milistai}.\arabic*)}
\newlength{\depthofsumsign}
\setlength{\depthofsumsign}{\depthof{$\sum$}}
\newcommand{\nsum}[1][1.4]{% only for \displaystyle
    \mathop{%
        \raisebox
            {-#1\depthofsumsign+1\depthofsumsign}
            {\scalebox
                {#1}
                {$\displaystyle\sum$}%
            }
    }
}
\def\scaleint#1{\vcenter{\hbox{\scaleto[3ex]{\displaystyle\int}{#1}}}}
\def\bs{\mkern-12mu}


\title{Transformada de Laplace} \vspace{-3ex}
\author{M. en C. Gustavo Contreras Mayén}
\date{ }
\newcommand{\Cancel}[2][black]{{\color{#1}\cancel{\color{black}#2}}}
\begin{document}
\vspace{-4cm}
\maketitle
\fontsize{14}{14}\selectfont
\tableofcontents
\newpage

\section{Introducción.}

En el material de trabajo sobre la transformada de Fourier, se revisó que si la integral:
\begin{align*}
\int_{-\infty}^{\infty} \abs{f(t)} \dd{t}
\end{align*}
no converge, la transformada de Fourier $F(\xi)$ no existe para todo valor real de $\xi$. Por ejemplo: si $f(t) = \sin \omega \, t$ que es un real, la $F(\xi)$ no existe. Pero tales situaciones surgen ocasionalmente en la práctica. Para manejar esta situación, consideramos una nueva función $f_{1} (t)$ conectada a $f (t)$ definida por:
\begin{align*}
f_{1}(t) =  \exp(- \gamma \, t) \, f(t) \, H(t)
\end{align*}
donde $\gamma$ es una constante arbitraria real positiva y $H(t)$ es la función de paso unitario de Heaviside\footnote{Para la definición de la función de Heaviside, revisa las notas de la Transformada de Fourier.}. Claramente se tiene que $f_{1}(t) \in A_{1}(R) \, \forall \, t \in R$ y por lo tanto, la transformada de Fourier de $f_{1}(t)$ existe, ya que:
\begin{align*}
\int_{-\infty}^{+\infty} f_{1} \dd{t} = \int_{0}^{\infty} \exp(-\gamma \, t) \, f(t) \, \dd{t}
\end{align*}
es convergente. De hecho, en este caso, del teorema de la integral de Fourier:
\begin{align*}
f_{1} (t) = \dfrac{1}{2 \, \pi} \int_{-\infty}^{\infty} \exp(-i \, \xi \, t) \dd{\xi} \, \int_{-\infty}^{+\infty} f_{1}(u) \, \exp(i \, \xi \, u) \dd{u}
\end{align*}
implica que:
\begin{align*}
f (t) = \dfrac{1}{2 \, \pi} \int_{-\infty}^{\infty} \exp(-i \, \xi \, t) \dd{\xi} \cdot \exp{\gamma \, t} \int_{0}^{+\infty} f(u) \, \exp(- (\gamma - i \, \xi) \, u) \dd{u}
\end{align*}

Si escribimos $p = - i \, \xi$, la relación anterior puede expresarse como:
\begin{align}
\begin{aligned}[b]
f (t) &= \dfrac{1}{2 \, \pi \, i} \int_{\gamma -i \, \infty}^{\gamma + i \, \infty} \exp(p \, t) \bigg[ \int_{0}^{+\infty} f(u) \, \exp(- p \, u) \dd{u} \bigg] \dd{p} = \\[0.5em]
&= \dfrac{1}{2 \, \pi \, i} \int_{\gamma -i \, \infty}^{\gamma + i \, \infty} \exp(p \, t) \, \overline{f} (p) \dd{p} 
\end{aligned}
\label{eq:ecuacion_03_01}
\end{align}
donde:
\begin{align}
\overline{f} (p) = \int_{0}^{+\infty} f(u) \, \exp(- p \, u) \dd{u}, \hspace{1cm} \Re{p} = \gamma > 0
\label{eq:ecuacion_03_02}
\end{align}

Las ecs. (\ref{eq:ecuacion_03_01}) y (\ref{eq:ecuacion_03_02}) constituyen una transformada con $K(p, u) = \exp(- p \, u)$ como núcleo o kernel.

\section{Definiciones.}

Definimos la transformada de Laplace de una función continua en tramos de variable real $t$, definida en un semieje $t \geq 0$ como:
\begin{align}
L \big[f(t); t \to p\big] = L \big[f(t)\big] = F(p) = \overline{f}(p) = \int_{0}^{\infty} f(t) \, \exp(-p \, t) \dd{t}
\label{eq:ecuacion_03_03}
\end{align}
y la transformada inversa de Laplace se define como:
\begin{align}
f(t) = L^{-1} \big[\overline{f}(t); p \to t\big] = L^{-1} \big[F(p)\big] = \dfrac{1}{2 \pi \, i} \int_{\gamma-i \infty}^{\gamma+\infty} \exp(p \, t) \, \overline{f} (p) \, \dd{p}
\label{eq:ecuacion_03_04}
\end{align}
donde $\gamma = \Re{p} > 0$.
\\[1em]
\noindent \textbf{Nota 1: } Algunos autores usan la variable $s$ en lugar de $p$.
\\
\textbf{Nota 2: } La función $f(t)$ debe de ser de orden exponencial\footnote{Este concepto se define en el siguiente numeral.} para la existencia de la transformada de Laplace.
\\
\textbf{Nota 3: } En términos de notación de operadores, la ec. (\ref{eq:ecuacion_03_03}) se expresa como:
\begin{align}
L \big[ f(t); t \to p] = \overline{f} (p) \big]
\label{eq:ecuacion_03_05}
\end{align}
y la relación en la ec. (\ref{eq:ecuacion_03_04}) se expresa como:
\begin{align}
L^{-1} \big[\overline{f}(p); p \to t \big] = f (t)
\label{eq:ecuacion_03_06}
\end{align}
Por lo que:
\begin{align*}
L^{-1} \big[L(f(t))]\big] = f(t) &= I \big[f(t)\big] \\[0.5em]
\Rightarrow \hspace{0.4cm} L^{-1} \, L \equiv I
\end{align*}
También se tiene:
\begin{align*}
L \big[L^{-1} (\overline{f}(p))]\big] = \overline{f}(p) &= I \big[\overline{f}(p)\big] \\[0.5em]
\Rightarrow \hspace{0.4cm} L \, L^{-1} \equiv I
\end{align*}

Esto significa que $L^{-1} \, L \equiv L \, L^{-1} \equiv I$, demostrando entonces que esos operadores $L$ y $L^{-1}$ son conmutativos.

\subsection{Condiciones suficientes para la existencia de la transformada de Laplace.}

\noindent \textbf{Teorema: } Si $f(t)$ es una función de algún orden exponencial para un valor de $t$ grande, y es continua en tramos sobre el intervalo $0 \leq t \leq \infty$, entonces la transformada de Laplace de $f(t)$ existe.
\\[0.5em]
\textbf{Definición: } Se dice que la función $f(t)$ es de \textbf{orden exponencial} conforme $t \to + \infty$ si existen constantes no negativas $M$, $\sigma$ y $t_{0}$ tales que
\begin{align}
\abs{f(t)} \leq M \; e^{\sigma \, t} \hspace{1cm} \mbox{para } t \geq t_{0}
\label{eq:023}
\end{align}

Así, una función es de orden exponencial siempre que su incremento (conforme $t \to + \infty$) no sea más rápido que un múltiplo constante de alguna función exponencial con un exponente lineal. Los valores particulares de $M$, $\sigma$ y $t_{0}$ no son tan importantes; lo importante es que algunos de esos valores existan de tal manera que la condición en (\ref{eq:023}) se satisfaga.
\par
La condición en (\ref{eq:023}) simplemente dice que $f(t) / e^{\sigma t}$ se encuentra entre $-M$ y $M$, y es por tanto acotada en su valor para $t$ suficientemente grande. En particular, esto se cumple (con $\sigma = 0$) si $f(t)$ en sí misma está acotada. Por tanto, toda función acotada -tal como $\cos k \, t$ o $\sin k \, t$ - es de orden exponencial.
\par
\noindent \textbf{Demostración: } Sea $f(t)$ de orden exponencial $\sigma$ tal que:
\begin{align*}
\abs{f(t)} < M \, e^{\sigma t} \hspace{1cm} \mbox{para} t \geq t_{0}
\end{align*}

Entonces se tenemos:
\begin{align*}
L \big[f(t); t \to p\big] &= \int_{0}^{\infty} \exp(-p \, t) \, f(t) \dd{t} = \\[0.5em]
&= \int_{0}^{t_{0}} \exp(-p \, t) \, f(t) \dd{t} + \int_{t_{0}}^{\infty} \exp(-p \, t) \, f(t) \dd{t} = \\[0.5em]
&= I_{1} + I_{2}
\end{align*}

Ya que $f(t)$ es una función continua en tramos en cada intervalo finito $0 \leq t \leq t_{0}$, la integral $I_{1}$ existe y es convergente. También se tiene:
\begin{align*}
\abs{I_{2}} &= \abs{\int_{t_{0}}^{\infty} e^{- p t} \, f(t) \dd{t}} \leq \\[0.5em]
&\leq \int_{t_{0}}^{\infty} e^{- p t} \, \abs{f(t)} \dd{t} < \\[0.5em]
&< M \, \int_{t_{0}}^{\infty} \exp(-(p - \sigma) \, t) \dd{t} = \dfrac{M \, \exp(-(p - \sigma) \, t_{0})}{(p - \sigma)}, \hspace{1cm} \mbox{si  } p > \sigma
\end{align*}

Por lo que $\abs{I_{2}}$ es finito para todo $t_{0} > 0$ y $p > \sigma$, y de aquí resulta que $I_{2}$ es convergente. Por tanto $L\abs{f(t)}$ existe para toda $p > \sigma$.
Aunque las condiciones establecidas en el teorema anterior son suficientes para la existencia de la transformada de Laplace, no lo son. Esto significa que incluso si una función no satisface las condiciones anteriores, la transformada de Laplace de esa función puede existir o no. 
\par
Consideremos el siguiente ejemplo con $f(t) = 1 / \sqrt{t}$:
\par
De la función se tiene que $f(t) \to \infty$ mientras que $t \to 0$, por lo que $f(t)$ no es una función continua en tramos para cada intervalo finito para $t \geq 0$.
\par
Ahora veamos:
\begin{align*}
L \big[\dfrac{1}{\sqrt{t}}\big] &= \int_{0}^{\infty} e^{-p t} \, \dfrac{1}{\sqrt{t}} \dd{t} = \\[0.5em]
&= \dfrac{2}{\sqrt{p}} \int_{0}^{\infty} e^{x^{2}} \dd{x} = \\[0.5em]
&= \sqrt{\dfrac{\pi}{p}} \hspace{1cm} p > 0
\end{align*}

Esto prueba la existencia de la transformada de Laplace de $f(t)$.

\section{Propiedades.}

\subsection{Propiedad de linealidad.}

Si $L \big[f_{1}(t); t \to p\big]$ y $L \big[f_{2}(t); t \to p\big]$ existen ambas y con las constantes $c_{1}$ y $c_{2}$, se tiene entonces:
\begin{align*}
L \big[c_{1} \, f_{1}(t) + c_{2} \, f_{2}(t) ; t \to p\big] = c_{1} \, L \big[f_{1}(t); t \to p\big] + c_{2} \, L c\big[f_{2}(t); t \to p\big]
\end{align*}

\subsection{Primer teorema de desplazamiento.}

\noindent \textbf{Teorema: } Si la transformada de Laplace de $f(t)$ es $\overline{f}(p)$, entonces la transformada de Laplace de $\exp(a \, t)$ es $\overline{f}(p - a)$.
\\[0.5em]
\textbf{Demostración: } Como tenemos:
\begin{align*}
L \big[f(t); t \to p\big] &= \int_{0}^{\infty} f(t) \, \exp{-p \, t} \dd{t} = \\[0.5em]+
&= \overline{f} (p)
\end{align*}

Entonces:
\begin{align}
\begin{aligned}[b]
L \big[\exp (a \, t) \, f(t); t \to p\big] &= \int_{0}^{\infty} \exp(a \, t) \, f(t) \, \exp(-p \, t) \dd{t} = \\[0.5em]
&= \int_{0}^{\infty} f(t) \, \exp(-(p - a) \, t) \dd{t} = \\[0.5em]
&= \overline{f} (p - a)
\end{aligned}
\label{eq:ecuacion_03_07}
\end{align}

\subsection{Segundo teorema de desplazamiento.}

\noindent \textbf{Teorema: } Si la transformada de Laplace de $f(t)$ es $\overline{f}(p)$, entonces la transformada de Laplace de $f(t - a) \, H(t - a)$ es $\exp(-a \, p) \, \overline{f}(p)$.
\\[0.5em]
\textbf{Demostración: } 
Como sabemos que:
\begin{align*}
L \big[f(t); t \to p\big] &= \int_{0}^{\infty} f(t) \, \exp(-p \, t) \dd{t} = \\[0.5em]
&= \overline{f} (p)
\end{align*}

Entonces:
\begin{align}
\begin{aligned}
L \big[f(t - a) \, H (t - a); t \to p\big] &= \int_{0}^{\infty} f(t - a) \, H(t - a) \, \exp(-p \, t) \dd{t} = \\[0.5em]
&= \int_{a}^{\infty} f(t - a) \, \exp(-p \, x) \cdot \exp(-p \, a) \dd{x} = \\[0.5em]
&= \exp(-p \, a) \, \overline{f} (p) 
\end{aligned}
\label{eq:ecuacion_03_08}
\end{align}

\subsection{Propiedad de cambio de escala.}

\noindent \textbf{Teorema: } Si:
\begin{align*}
L \big[f(t); t \to p\big] &= \overline{f} (p)
\end{align*}
Entonces:
\begin{align*}
L \big[f(a \, t); t \to p\big] &= \dfrac{1}{a} \, \overline{f} \left(\dfrac{p}{a}\right)
\end{align*}
\\[0.5em]
\textbf{Demostración: } La transformada de Laplace de $f(t)$ es:
\begin{align*}
L \big[f(t); t \to p\big] &= \int_{0}^{\infty} f(t) \, \exp(-p \, t) \dd{t} = \\[0.5em]
&= \overline{f} (p)
\end{align*}
Entonces:
\begin{align}
\begin{aligned}
L \big[f(a \, t); t \to p\big] &= \int_{0}^{\infty} f(a \, t) \, \exp(-p \, t) \dd{t} = \\[0.5em]
&= \dfrac{1}{a} \int_{0}^{\infty} f(x) \, \exp(- (p/a) \, t) \dd{x} = \\[0.5em]
&= \dfrac{1}{a} \, \overline{f} \left(\dfrac{p}{a}\right)
\end{aligned}
\label{eq:ecuacion_03_09}
\end{align}

\subsection*{Ejemplos.}

Calculemos la transformada de Laplace de algunas funciones sencillas a partir de la definición.

\begin{ejemplo}
\begin{align*}
L \big[H(t); t \to p\big] &= \int_{0}^{\infty} \exp(- p \, t) \, H(t) \dd{t} = \\[0.5em]
&= \int_{0}^{\infty} \exp(- p \, t) \dd{t} = \\[0.5em]
&= \dfrac{1}{p}
\end{align*}
De aquí que:
\begin{align*}
L \big[H(t - a); t \to p\big] &= \int_{0}^{\infty} \exp(- p \, t) \, H(t - a) \dd{t} = \\[0.5em]
&= \int_{a}^{\infty} \exp(- p \, t) \dd{t} = \\[0.5em]
&= \dfrac{\exp(-a \, p)}{p}
\end{align*}
\end{ejemplo}

\begin{ejemplo}
\begin{align*}
L \big[t^{\nu}; t \to p\big] &= \int_{0}^{\infty} t^{\nu} \, \exp(- p \, t) \dd{t} = \\[0.5em]
&= \int_{0}^{\infty} u^{\nu} \, \exp(-u) \dd{u} \hspace{1cm} \mbox{cuando  } u = p \, t \\[0.5em]
&= p^{-\nu - 1} \, \Gamma (\nu + 1) \hspace{1cm} \mbox{la cual existe cuando  } \Re{\nu} > - 1
\end{align*}
\end{ejemplo}

\begin{ejemplo}
\begin{align*}
L \big[e^{a t}; t \to p\big] &= \int_{0}^{\infty} e^{-p t} \, e^{a \, t} \dd{t} = \\[0.5em]
&= \dfrac{1}{p - a} \hspace{1cm} p > a
\end{align*}
\end{ejemplo}

\begin{ejemplo}
\begin{align*}
L \big[\sin (a t); t \to p\big] &= \int_{0}^{\infty} e^{-p t} \, \sin (a \, t) \dd{t} = \\[0.5em]
&= \int_{0}^{\infty} e^{-p t} \,\left[ \dfrac{e^{a i t} - e^{- a i t}}{2 \, i} \right] \dd{t} = \\[0.5em]
&= \left[ \int_{0}^{\infty}  \dfrac{e^{-(p - a i) t} - e^{- (p + a i) t}}{2 \, i} \right] \dd{t} = \\[0.5em]
&= \dfrac{1}{2 \, i} \left[ \dfrac{1}{p {-} a \, i} {-} \dfrac{1}{p {+} a \, i} \right] = \hspace{0.25cm} \mbox{(por la propiedad de linealidad)} \\[0.5em]
&= \dfrac{a}{p^{2} + a^{2}}, \hspace{1cm} p > 0
\end{align*}
\end{ejemplo}

\begin{ejemplo}
\begin{align*}
L \big[\cos (a t); t \to p\big] &= \int_{0}^{\infty} e^{-p t} \,\dfrac{e^{a i t} + e^{- a i t}}{2} \dd{t} = \\[0.5em]
&= \dfrac{1}{2} \left[ \int_{0}^{\infty} e^{-(p - a i) t} \dd{t} \right] + \dfrac{1}{2} \left[ \int_{0}^{\infty} e^{-(p + a i) t} \dd{t} \right] = \\[0.5em]
&\mbox{(por la propiedad de linealidad)} \\[0.5em]
&= \dfrac{1}{2} \left[ \dfrac{1}{p - a \, i} + \dfrac{1}{p + a \, i}\right] = \\[0.5em]
&= \dfrac{p}{p^{2} + a^{2}}, \hspace{1cm} p > 0
\end{align*}
\end{ejemplo}

Evalúa la transformada de Laplace de la función que se indica:
\begin{ejemplo}
\begin{align*}
L\big[f(t)\big] \hspace{0.5cm} \mbox{donde} \hspace{0.5em} f(t) = \begin{cases}
t/a & 0 < t < a \\
1 & t > a
\end{cases}
\end{align*}

Solución:
\begin{align*}
L\big[f(t)\big] = \int_{0}^{\infty} e^{-p t} \, f(t) \dd{t}  &= \int_{0}^{a} e^{-p t} \cdot \dfrac{t}{a} \dd{t} + \int_{0}^{a} e^{-p t} \dd{t} = \\[0.5em]
&=\dfrac{1 - e^{- a p}}{a \, p^{2}}
\end{align*}
\end{ejemplo}

\begin{ejemplo}
Evalúa $L \bigg[\dfrac{1}{\sqrt{\pi \, t}}\bigg]$

Solución:
\begin{align*}
L \bigg[\dfrac{1}{\sqrt{\pi \, t}}\bigg] &= \int_{0}^{\infty} e^{- p t} \, \dfrac{1}{\sqrt{\pi \, t}} \dd{t} = \\[0.5em]
&= \dfrac{\sqrt{p}}{\sqrt{\pi}} \int_{0}^{\infty} e^{-x} \, x^{1/2 - 1} \dfrac{\dd{x}}{p} = \\[0.5em]
&\mbox{(reconocemos que la integral es la función Gamma)} \\[0.5em]
&= \dfrac{1}{\sqrt{\pi} \, p} \Gamma \left( \dfrac{1}{2} \right) = \\[0.5em]
&= \dfrac{\sqrt{\pi}}{\sqrt{\pi \, p}} = \dfrac{1}{\sqrt{\pi}}
\end{align*}
\end{ejemplo}

\begin{ejemplo}
Evaluar $L \big[t^{\nu} \, e^{-a t} \big]$, si $\Re{\nu + 1} > 0$

Solución: Se sabe que:
\begin{align*}
L \big[t^{\nu}; t \to p\big] = \dfrac{\Gamma (\nu + 1)}{p^{\nu+1}}
\end{align*}

Entonces, ocupando el teorema del desplazamiento, obtenemos:
\begin{align*}
L \big[t^{\nu} \, e^{-a t}; t \to p\big] = \dfrac{\Gamma (\nu + 1)}{(p + a)^{\nu+1}}
\end{align*}
\end{ejemplo}

\begin{ejemplo}
Evalúa $L \big[\sin (t - a) \, H(t- a); t \to p\big]$. \\
Para luego evaluar: $L \big[e^{(t-a) k} \, \sin (t - a) \, H(t- a); t \to p\big]$

Solución: Sabemos que:
\begin{align*}
L \big[\sin t; t \to p\big] = \dfrac{1}{p^{2} + 1}
\end{align*}

Por lo que ocupando el segundo teorema del desplazamiento, se obtiene:
\begin{align*}
L \big[\sin (t - a) \, H(t- a)\big] = \dfrac{e^{-p a}}{(1 + p^{2})}
\end{align*}

por lo que:
\begin{align*}
L \big[e^{(t-a) k} \, \sin (t - a) \, H(t- a); t \to p\big] &= \dfrac{e^{-(p - k)}}{\big[1 + (p - k^{2})\big]}
\end{align*}

después de haber utilizado el primer teorema del desplazamiento.
\end{ejemplo}

\subsection{La transformada de Laplace de la derivada de una función.}

\noindent \textbf{Teorema: } Sea $f(t)$ una función continua de $t \geq 0$ y es de orden exponencial para un valor de $t$ grande y si $\ptilde{f}(t)$ es una función continua a tramos para $t \geq 0$, entonces la transformada de Laplace de la derivada $\ptilde{f}(t)$ existe cuando $p > \sigma$ y está dada por:
\begin{align*}
L \big[\ptilde{f}(t)\big] = p \, L \big[f(t)\big] - f(0)
\end{align*}

\noindent \textbf{Demostración:} Por definición de la transformada de Laplace:
\begin{align}
\begin{aligned}[b]
L \big[\ptilde{f}(t)\big] &= \int_{0}^{\infty} e^{-p t} \, \ptilde{f} (t) \dd{t} = \\[0.5em]
&= \big[e^{-p t} \cdot f(t)\big] \eval_{0}^{\infty} - \int_{0}^{\infty} (-p) \, e^{-p t} \, \ptilde{f} (t) \dd{t} = \\[0.5em]
&= \lim_{t \to \infty} e^{-p t} \, f(t) - f(0) + p \, L \big[f(t)\big]
\end{aligned}
\label{eq:ecuacion_03_10}
\end{align}
Como $f(t)$ es de orden exponencial $\sigma$ para $t \to \infty$, se tiene
\begin{align*}
\abs{f(t)} \leq M \, e^{-\sigma t}, \hspace{0.5em} t \geq 0
\end{align*}
Por lo que
\begin{align*}
\abs{f(t) \, e^{-p t}} = e^{-p t} \, \abs{f(t)} \leq M \, e^{-p t} \cdot e^{\sigma t} = M \, e^{-(p - \sigma) t}
\end{align*}
Así pues:
\begin{align*}
\lim_{t \to \infty} e^{-p t} \, f(t) = 0 \hspace{0.5cm} \mbox{mientras  } p - \sigma > 0
\end{align*}
Entonces, de la ec. (\ref{eq:ecuacion_03_10}), llegamos a:
\begin{align}
L \big[\ptilde{f}(t)\big] = p \, L \big[f(t)\big] - f(0)
\label{eq:ecuacion_03_11}
\end{align}

\noindent \textbf{Corolario 1: } Si $\stilde{f}(t)$ existe para $t \geq 0$ y es una función continua en tramos, entonces siguiendo el mismo procedimiento anterior, es posible extender el resultado del teorema como:
\begin{align}
\begin{aligned}[b]
L \big[\stilde{f}(t)\big] &= p \, L \big[\ptilde{f}(t)\big] - \ptilde{f}(0) = \\[0.5em]
&= p \, \big[p \left\{ L (f(t)) \right\} - f(0) \big] - \ptilde{f} (0) = \\[0.5em]
&= p^{2} \, L \big[f(t)\big] - p \, f(0) - \ptilde{f}(0)
\end{aligned}
\label{eq:ecuacion_03_12}
\end{align}

\noindent \textbf{Corolario 2:} En general, si $\ntilde{f}{n}(t)$ existe para $t \geq 0$ y es una función continua en tramos de $t$, entonces:
\begin{align}
L \big[\ntilde{f}{n}(t)\big] = p^{n} \, L \big[f(t)\big] - p^{n-1} \, f(0) - p^{n-2} \, \ptilde{f}(0) - \ldots - \ntilde{f}{n-1} (0)
\label{eq:ecuacion_03_13}
\end{align}

\subsection{Transformada de Laplace de la integral de una función.}

Si la transformada de Laplace de una función $f(t)$ es $\overline{f}(p$). Entonces la transformada de la integral:
\begin{align}
\int_{0}^{t} f(\tau) \dd{\tau} = \dfrac{\overline{f}(p)}{p}
\label{eq:ecuacion_03_16}
\end{align}

\subsection{La Transformada de Laplace de una función periódica.}

Sea $f(t)$ una función con período $\tau$, tal que 
\begin{align*}
f(t + n \, \tau) = f(t) \hspace{1cm} n = 1, 2, 3, \ldots
\end{align*}

Si $f(t)$ es una función continua en tramos para $t > 0$, entonces:
\begin{align}
L \big[f(t)\big] = \dfrac{1}{1 - e^{-p \tau}} \int_{0}^{\tau} e^{-p t} \, f(t) \dd{t}
\end{align}

\subsection{Convolución de fos funciones.}

Sean $f (t)$ y $g (t)$ dos funciones continuas por partes y son de algún orden exponencial para $t$ grande y para todo $t \geq 0$. Entonces la \emph{convolución}\footnote{Recuerda la observación sobre la traducción del término en inglés \emph{convolution} que se discutió en el material de trabajo: Transformada de Fourier.} de estas funciones se denota por $f * g (t)$ y se define por:
\begin{align}
f * g(t) = \int_{0}^{t} f(u) \, g(t - u) \dd{u}
\label{03_55}
\end{align}

Esta relación también se le conoce como \emph{faltung} de $f(t)$ y $g(t)$. Por definición:
\begin{align}
\begin{aligned}
f * g(t) &= \int_{0}^{t} f(u) \, g(t - u) \dd{u} = \\[0.5em]
&= \int_{0}^{t} f(t - \tau) \, g(\tau) \dd{\tau} = \\[0.5em]
&= g * f(t)
\end{aligned}
\label{eq:ecuacion_03_56}
\end{align}

Por lo tanto, la convolución de dos funciones satisface la ley conmutativa.

Satisface también:
\begin{enumerate}
\item La ley distributiva
\begin{align}
f * \big[g + h\big](t) = f * g(t) + f * h(t)
\label{eq:ecuacion_03_57}
\end{align}
\item La ley asociativa
\begin{align}
\big[f * (g * h)\big](t) = \big[(f * g) * h\big](t)
\label{eq:ecuacion_03_58}
\end{align}
\end{enumerate}

Ahora podemos enfocarnos a evaluar la transformada de Laplace de la convolución de dos funciones $f(t)$ y $g(t)$:
\begin{align}
\begin{aligned}[b]
L \big[(f * g)(t); t \to p\big] &= L \left[ \int_{0}^{t} f(\tau) \, g(t - \tau) \dd{\tau}; t \to p \right] = \\[0.5em]
&= \int_{0}^{\infty} e^{-p t} \, \left[ \int_{0}^{t} f(\tau) \, g(t - \tau) \dd{\tau}\right] \, \dd{t} = \\[0.5em]
&= \int_{0}^{\infty} f(\tau) \, \left[ \int_{\tau}^{\infty} e^{-p t} \, g(t - \tau) \dd{t}\right] \, \dd{\tau} = \\[0.5em]
&= \int_{0}^{\infty} f(\tau) \, e^{- p t} \, \left[ \int_{0}^{\infty} e^{-p \eta} \, g(\eta) \dd{\eta} \right] \, \dd{\tau} = \\[0.5em]
&= \overline{f}(p) \, \overline{g}(p)
\end{aligned}
\label{eq:ecuacion_03_59}
\end{align}

En particular se tiene:
\begin{align*}
L \big[f * f(t); t \to p\big] = \big[\overline{f}(p)\big]^{2}
\end{align*}

\section{La transformada inversa de Laplace.}

\subsection{Introducción.}

Si $f (t)$ pertenece a una clase $A$ (lo que significa que es una función continua por partes sobre $0 \leq t < \infty$ y es de algún orden exponencial), su transformada de Laplace $\overline{f} (p)$ existe. Esto se denota simbólicamente por:
\begin{align}
L \big[f(t); t \to p\big] = \int_{0}^{\infty} e^{-p t} \, f(t) \dd{t} \equiv \overline{f} (p)
\label{ec:ecuacion_04_01}
\end{align}
Entonces, a su vez $f (t)$, la función de objeto es la transformada de Laplace inversa de la función de imagen $\overline{f} (p)$ y está dada simbólicamente por:
\begin{align}
L^{-1} \big[\overline{f}(t); p \to t\big] \equiv f (t)
\label{ec:ecuacion_04_02}
\end{align}
Así por la ec. (\ref{ec:ecuacion_04_01}), se puede evaluar $\overline{f} (p)$ para una $f (t)$ dada, ya que $\overline{f} (p)$ existe. Ahora consideraremos el problema inverso: derivar información de la $\overline{f} (p)$ prescrita que nos permite obtener la función original $f (t)$ a través de alguna fórmula, llamada \emph{fórmula de inversión de Laplace}.
\par
Antes de responder a este problema básico planteado anteriormente, en primer lugar presentaremos el enfoque heurístico para determinar la transformada inversa de Laplace de alguna $\overline{f}(p)$. También, según sea necesario, definimos una nueva función, llamada \textbf{función nula}, en esta conexión junto con un teorema vinculado, conocido como \emph{teorema de Lerch}.
\par
\noindent \textbf{Definición.} \textbf{Función nula.} Una función $N (t)$ que satisface la condición
\begin{align*}
\int_{0}^{\tau} N(\tau) \dd{\tau} = 0 \hspace{0.5cm} \forall \, t > 0
\end{align*}
se denomina función nula.
\par
\noindent \textbf{Teorema de Lerch.}

\noindent Si $L \big[f_{1} (t); t \to p\big], \Re {p} > c_{1}$ y $L \big[f_{2} (t); t \to p\big], \Re {p} > c_{2}$ existen ambas, y si
\begin{align*}
L \big[f_{1} (t); t \to p\big] = L \big[f_{2} (t); t \to p\big]
\end{align*}
para $Re{p} > c = \max{c_{1},c_{2}}$, entonces:
\begin{align*}
f_{2} (t) - f_{1} (t) = N_{t}
\end{align*}
Además, si $f_{1}(t)$ y $f_{2}(t)$ son continuas en toda una línea real, entonces $f_{1}(t) = f_{2}(t), \hspace{0.3cm} t > 0$.
\section{Propiedades.}

\subsection{Linealidad de la transformada inversa.}

Si:
\begin{align*}
\overline{f}_{1} (p) = L \big[f_{1} (t); t \to p\big] \hspace{0.5cm} \mbox{y} \hspace{0.5cm} \overline{f}_{2} (p) = L \big[f_{2} (t); t \to p\big]
\end{align*}
y $c{1}$ y $c_{2}$ dos constantes arbitrarias, entonces se tiene que:
\begin{align*}
L^{-1} \, \big[ c_{1} \, \overline{f}_{1} + c_{2} \, \overline{f}_{2} (p); p \to t\big] &= c_{1} \, L^{-1} \, \big[ \overline{f}_{1}; p \to t\big] + c_{2} \, L^{-1} \, \big[ \overline{f}_{2}; p \to t\big] = \\[0.5em]
&= c_{1} \, f_{1}(t) + c_{2} \, f_{2}(t)
\end{align*}

\subsection{Cálculo de la transformada inversa de Laplace de algunas funciones.}

Antes de encontrar la transformada inversa de Laplace de funciones elementales que son un poco complejas por naturaleza, a continuación exponemos algunos resultados obtenidos previamente. Estos resultados pueden considerarse fórmulas para otras aplicaciones.

\begin{enumerate}[label=\alph*)]
\item $ L \big[H(t); t \to p\big] = \dfrac{1}{p} \hspace{0.5cm} \Rightarrow \hspace{0.5em} L^{-1} \bigg[ \dfrac{1}{p}; p \to t \bigg] = H(t)$
\item $ L \big[t^{\nu}; t \to p\big] = \dfrac{\Gamma(\nu + 1)}{p^{\nu + 1}} \hspace{0.3cm} \Rightarrow \hspace{0.3cm} L^{-1} \bigg[ \dfrac{\Gamma(\nu + 1)}{p^{\nu + 1}}; p \to t \bigg] = t^{\nu}, \hspace{0.1cm} \Re{\nu} > - 1$
\item $ L \big[e^{a t}; t \to p\big] = \dfrac{1}{p - a} \hspace{0.3cm} \Rightarrow \hspace{0.3cm} L^{-1} \bigg[ \dfrac{1}{p - a}; p \to t \bigg] = e^{a t}, \hspace{0.2cm} \Re{p} > a$
\item $ L \big[\sin (a \, t); t \to p\big] = \dfrac{a}{p^{2} + a^{2}} \hspace{0.2cm} \Rightarrow \hspace{0.2cm} L^{-1} \bigg[ \dfrac{a}{p^{2} + a^{2}}; p \to t \bigg] {=} \sin (a \, t), \hspace{0.1cm} p > 0$
\item $ L \big[\cos (a \, t); t \to p\big] = \dfrac{p}{p^{2} + a^{2}} \hspace{0.2cm} \Rightarrow \hspace{0.2cm} L^{-1} \bigg[ \dfrac{p}{p^{2} + a^{2}}; p \to t \bigg] {=} \cos (a \, t), \hspace{0.1cm} p > 0$
\end{enumerate}
Ahora dirigiremos nuestra atención al discutir algunas reglas de manipulación de la inversión de Laplace de algunas combinaciones de funciones elementales de p a través de los siguientes ejemplos.
\begin{ejemplo}
Evaluar
\begin{align*}
L^{-1} \bigg[ \dfrac{1}{(p + 1)(p^2) + 1)}; p \to t \bigg]
\end{align*}

Solución:
\\[0.5em]
Resolviendo mediante fracciones parciales siguiendo el método tradicional, se tiene que:
\begin{align*}
\dfrac{1}{(p^2) + 1)(p + 1)} \equiv \dfrac{1}{2} \, \dfrac{1}{p + 1} - \dfrac{1}{2} \, \dfrac{p}{p^{2} + 1} + \dfrac{1}{2} \, \dfrac{1}{p^{2} + 1}
\end{align*}

Por lo tanto:
\begin{align*}
L^{-1} \bigg[ \dfrac{1}{(p + 1)(p^2) + 1)}\bigg] &= \dfrac{1}{2} \, L^{-1} \bigg[ \dfrac{1}{p + 1}\bigg] {-} \dfrac{1}{2} \, L^{-1} \bigg[\dfrac{p}{p^{2} + 1} \bigg] {+} \dfrac{1}{2} L^{-1} \, \bigg[\dfrac{1}{p^{2} + 1} \bigg] = \\[0.5em]
&= \dfrac{1}{2} \, \big[ e^{-t} - \cos t + \sin t \big]
\end{align*}
\end{ejemplo}

\begin{ejemplo} Evaluar:
\begin{align*}
L^{-1} \left[ \dfrac{6 \, p^{2} + 22 \, p + 18}{p^{3} + 6 \, p^{2} + 11 \, p + 6} \right]
\end{align*}

\noindent Solución:
\\[0.5em]
Tenemos entonces:
\begin{align*}
L^{-1} \left[ \dfrac{6 \, p^{2} + 22 \, p + 18}{p^{3} + 6 \, p^{2} + 11 \, p + 6} \right] &= L^{-1} \left[ \dfrac{6 \, p^{2} + 22 \, p + 18}{(p + 1)(p + 2)(p + 3)} \right] = \\[0.5em]
&= L^{-1} \left[ \dfrac{1}{p + 1} + \dfrac{2}{p + 2} + \dfrac{3}{p + 3} \right] = \\[0.5em]
&= e^{t} + 2 \, e^{2 t} + 3 \, e^{- 3 t}
\end{align*}
\end{ejemplo}

\section{Método de expansión en fracciones parciales del cociente de dos polinomios.}

Sean $f (p)$ y $g (p)$ dos polinomios en $p$ tales que el grado de $f (p)$ es menor que el de $g (p)$. Entonces $f (p) / g (p)$ se puede expresar como:
\begin{align*}
\dfrac{f(p)}{g(p)} = \sum_{r=1}^{n} \dfrac{A_{r}}{p - a_{r}} \hspace{0.5cm} \mbox{donde } p - a{r} \mbox{ son los factores de } g(p)
\end{align*}
con
\begin{align*}
A_{r} = \lim_{p \to a_{r}} \dfrac{(p - a_{r}) \, f(p)}{g(p)} = \dfrac{f(a_{r})}{\ptilde{g}(a_{r})} \hspace{0.5cm} \mbox{para  } r = 1, 2, \ldots, n
\end{align*}

\begin{ejemplo} Evalúa:
\begin{align*}
L^{-1} \left[ \dfrac{p + 5}{(p + 1)(p^{2} + 1)} \right]
\end{align*}

\noindent Solución:
\\[0.5em]
Al expresar en términos de fracciones parciales, se tiene:
\begin{align*}
\left[ \dfrac{p + 5}{(p + 1)(p^{2} + 1)} \right] &\equiv \dfrac{A_{1}}{p + 1} + \dfrac{A_{2}}{p + 1} + \dfrac{A_{3}}{p - i} = \\[0.5em]
&= \dfrac{2}{p -1} + \dfrac{\dfrac{i - 5}{2(1 + i)}}{p + i} + \dfrac{\dfrac{i + 5}{2(i - 1)}}{p - i} = \\[0.5em]
&= \dfrac{2}{p + 1} + \dfrac{-2 \, p}{p^{2 + 1}} + \dfrac{3}{p^2 + 1}
\end{align*}

Por lo tanto:
\begin{align*}
L^{-1} \left[ \dfrac{p + 5}{(p + 1)(p^{2} + 1)} \right] = 2 \, e^{-t} - 2 \, \cos t +  3 \, \sin t
\end{align*}
\end{ejemplo}

\begin{ejemplo}
Evalúa:
\begin{align*}
L^{-1} \left[ \dfrac{4 \, p + 5}{(p - 1)^{2} \, (p + 2)} \right]
\end{align*}
\\[0.5em]
Por el método de fracciones parciales, se tiene:
\begin{align*}
\dfrac{4 \, p + 5}{(p - 1)^{2} \, (p + 2)} \equiv \dfrac{A}{p - 1} + \dfrac{B}{(p - 1)^{2}} + \dfrac{C}{p + 2}
\end{align*}

Entonces:
\begin{align*}
4 \, p + 5 \equiv A \, (p - 1)(p + 2) + B \, (p + 2) + C \, (p - 1)^{2}
\end{align*}

De esta identidad tenemos:
\begin{align*}
9 &= 3 \, B \hspace{0.4cm} \mbox{haciendo que } p = 1 \hspace{0.3cm} \Rightarrow \hspace{0.3cm} B = 3 \\[0.5em]
- 3 &= 9 \, C \hspace{0.4cm} \mbox{haciendo que } p = -2 \hspace{0.3cm} \Rightarrow \hspace{0.3cm} C = -\dfrac{1}{3} \\[0.5em]
5 &= -2 \, A {+}  2 \, B {+} C \hspace{0.4cm} \mbox{igualando el término independiente de p} \hspace{0.3cm} \Rightarrow \hspace{0.3cm} A = \dfrac{1}{3}
\end{align*}

Por lo tanto:
\begin{align*}
L^{-1} \left[ \dfrac{4 \, p + 5}{(p - 1)^{2} \, (p + 2)} \right] &= \dfrac{1}{3} \, e^{t} +  3 \, L^{-1} \left[\dfrac{1}{(p - 1)^{2}} \right] - \dfrac{1}{3} \, e^{-2 t} = \\[0.5em]
&= \dfrac{1}{3} \, e^{t} - \dfrac{1}{3} \, e^{- 2 t} + 3 \, t \, e^{-t}
\end{align*}
\end{ejemplo}

\section{Aplicaciones de las transformadas de Laplace.}

\subsection{Solución de EDO con coeficientes constantes.}

Supongamos que deseamos resolver una ecuación diferencial ordinaria (EDO) lineal de orden $n$:
\begin{align}
\dv[n]{y}{t} + c_{1} , \dv[n-1]{y}{t} + \ldots + c_{n-1} \, \dv{y}{t} + c_{n} \, y = 0
\label{eq:ecuacion_04_54}
\end{align}

donde $c_{i} = 1, 2, \ldots, n$ son constantes dadas, sujetas a las condiciones iniciales:
\begin{align*}
y(0) = k_{1}, \, \ptilde{y}(0) = k_{2} , \ldots, \ntilde{y}{n=1}(0)= k_{n}
\end{align*}

Tomando la transformada de Laplace en ambos lados de la EDO, utilizando los resultados y propiedades revisados previamente en estas notas, así como las condiciones iniciales dadas en la EDO, se tiene que:
\begin{align*}
\big[\overline{y}(p) \, &p^{n} - p^{n-1} \, k_{1} - \ldots - p \, k_{n-1} - k_{n} \big] + \\[0.5em] &+ \overline{f}(p) + k_{1} \, \big[\overline{y}(p) \, p^{n-1} - p^{n-2} \, k_{1} - \ldots - k_{n-1} - k_{n} \big] + \\[0.5em]
&+ \ldots + c_{n-1} \big[p \, \overline{y}(p) - k_{1} \big] + c_{n} \, \overline{y}(p) = \overline{f}(p)
\end{align*}

Por lo tanto:
\begin{align*}
\overline{y}(p) \big[&p^{n} + c_{1} \, p^{n-1} \, + \ldots + c_{n} \big] = \overline{f}(p) + \\[0.5em]
&+ k_{1} \, \big[p^{n-1} + c_{1} \, p^{n-2} + \ldots + c_{n-1} \, p\big] + \\[0.5em]
&+ k_{2} \, \big[p^{n-2} + c_{1} \, p^{n-3} + \ldots \big] +  \ldots + \\[0.5em]
&+ k_{n-1} \big[p + c_{1} \big] + k_{n}
\end{align*}

Lo que implica:
\begin{align}
\begin{aligned}[b]
\overline{y}(p) &= \dfrac{\big[\overline{f}(p) + \overline{g}(p) \big]}{\big[p^{n}+ c_{1} \, p^{n-1} + \ldots + c_{n}\big]} \\[0.5em]
\overline{y}(p) &= \dfrac{\overline{f}(p)}{h(p)} + \dfrac{\overline{g}(p)}{h(p)}
\end{aligned}
\label{eq:ecuacion_04_55}
\end{align}

donde $\overline{g}(p)$ es un polinomio de grado $n-1$ en $p$ y $h(p)$ es también un polinomio de grado $n$ en $p$.
\par
Por lo tanto, la solución de la EDO en las condiciones iniciales dadas se puede obtener después de tomar la transformada inversa de Laplace de la última ecuación (\ref{eq:ecuacion_04_55}). Dado que el lado derecho es una función conocida, tenemos:
\begin{align}
y(t) = L^{-1} \left[ \dfrac{\overline{f}(p)}{h(p)} \right] + L^{-1} \left[ \dfrac{\overline{g}(p)}{h(p)} \right]
\label{eq:ecuacion_04_56}
\end{align}

Si la ec. (\ref{eq:ecuacion_04_54}) es homogénea, entonces $f (t) \equiv 0$ y, por tanto, la La solución de la EDO homogénea con condiciones iniciales dadas es:
\begin{align}
y(t) = L^{-1} \left[ \dfrac{\overline{g}(p)}{h(p)} \right]
\label{eq:ecuacion_04_57}
\end{align}

\begin{ejemplo}
Resuelve usando la transformada de Laplace el problema de valores iniciales:
\begin{align*}
\ttilde{y}(t) +  2 \, \stilde{y} (t) - \ptilde{y}(t) - 2 \, y(t) = 0
\end{align*}

con las condiciones iniciales: $y(0) = \ptilde{y}(0) = 0$ y $\stilde{y} = 6$
\noindent Solución:
\\[0.5em]
Aplicando la transformada de Laplace en ambos lados de la EDO con las condiciones iniciales dadas, se tiene que:
\begin{align*}
\big[p^{3} \, \overline{y}(p) &- p^{2} \, y(0) - p \, \ptilde{y}(0) - \stilde{y}(0) \big] + 2 \, \big[p^{2} \, \overline{y}(p) - p \, y(0) - \ptilde{y}(0) \big] + \\[0.5em]
&- \big[p \, \overline{y}(p) - y(0) \big] - 2 \, \overline{y}(p) = 0 \\[1em]
&\Rightarrow \overline{y}(p) = \dfrac{6}{\big[p^{3} - 2 \, p^{2} - p -2 \big]} \\[0.5em]
&\Rightarrow \overline{y}(p) = \dfrac{6}{(p - 1)(p + 1)(p +2)} \equiv \dfrac{1}{p - 1} - \dfrac{3}{p + 1} + \dfrac{2}{p + 2}  
\end{align*}

Entonces al aplicar la transformada inversa de Laplace en ambos lados de la expresión, se obtiene:
\begin{align*}
y(t) = e^{t} - 3 \, e^{-t} + 2 \, e^{- 2 t}
\end{align*}
\end{ejemplo}

\begin{ejemplo}
Resuelve el siguiente problema de valores iniciales, definido por:
\begin{align*}
\left[ \dv[2]{t} + n^{2}\right] \, x(t) = a \, \sin (n \, t +  \alpha) \hspace{0.5cm} x(0) = \ptilde{x}(0) = 0
\end{align*}

\noindent Solución:
\\[0.5em]
Aplicando la transformada de Laplace en ambos lados de la EDO con las condiciones iniciales dadas, se tiene que:
\begin{align*}
&(p^{2} + n^{2}) \, \overline{x}(p) = a \, \cos \alpha \, \dfrac{n}{p^{2} + n^{2}} +  a \, \sin \alpha \, \dfrac{p}{p^{2} + n^{2}} \\[0.5em]
&\Rightarrow \overline{x}(p) = a \, n \, \cos \alpha \, \dfrac{1}{(p^{2} + n^{2})^{2}} + a \, n \, \sin \alpha \, \dfrac{p}{(p^{2} + n^{2})^{2}}
\end{align*}

Aplicando ahora la transformada inversa, se llega a:
\begin{align*}
x(t) &= a \, n \, \cos \alpha \, \dfrac{1}{2 \, n^{3}} \, \big[\sin n \, t - n \, t \, \cos n \, t \big] + a \, \sin \alpha \, \dfrac{t}{2 \, n} \, \sin n \, t = \\[0.5em]
&= \dfrac{a \big[\sin n \, t \, \cos \alpha - n \, t \, \cos (n \, t +  \alpha)]}{2 \, n^{2}}
\end{align*}
\end{ejemplo}

\begin{ejemplo}
Se aplica un voltaje $E \, e^{- a t}$ al tiempo $t = 0$ a un circuito de inductancia $L$ y resistencia $R$ conectadas en serie, donde $a$, $E$, $L$, y $R$ son constantes. Determina la corriente en cualquier tiempo, es decir: $i(t)$.
\noindent Solución:
\\[0.5em]
En un circuito eléctrico en donde se tiene un voltaje $E(t)$, una resistencia $R$ e inductancia $L$, la corriente $i(t)$ en el circuito está dada por:
\begin{align*}
L \, \dv{i}{t} + R \, i(t) =  E(t)
\end{align*}

En este caso: $E(t) = E \, e^{-a t}$ y para el $t = 0$, $i(0) = 0$. Al aplicar la transformada de Laplace a la ecuación de la corriente, con las condiciones dadas, se tiene que:
\begin{align*}
&L \big[ p \, \overline{i}(p) - 0 \big] + R \, \overline{i} (p) = E \, \dfrac{1}{p + a} \\[0.5em]
&\Rightarrow \overline{i}(p) = \dfrac{E}{(L \, p + R)(p + a)} \equiv \dfrac{E}{L \, \left(a - \dfrac{R}{L} \right)} \, \left[ \dfrac{1}{p + \dfrac{R}{L}}  - \dfrac{1}{p + a}\right] \\[0.5em]
&= \dfrac{E}{R - a \, L} \left[ \dfrac{1}{p + a} - \dfrac{1}{p + \dfrac{R}{L}}\right]
\end{align*}

Aplicando la transformada inversa de Laplace:
\begin{align*}
i(t) = \dfrac{E}{R - a \, L} \, \big[ e^{-a t} - e^{-R t /L}\big]
\end{align*}
\end{ejemplo}

\subsection{Solución de EDO simultáneas con coeficientes constantes.}

La EDO que involucra más de una variable dependiente pero con una sola variable independiente da lugar a ecuaciones simultáneas. El procedimiento para resolver tales ecuaciones simultáneas es casi el mismo que se discutió previamente. Aquí también, tenemos que tomar la transformada de Laplace de las ecuaciones simultáneas para reducirlas al número correspondiente de ecuaciones algebraicas que luego pueden resolverse para las variables dependientes transformadas de Laplace. Finalmente invirtiendo estas relaciones podemos recuperar las variables dependientes formando las soluciones requeridas.

\begin{ejemplo}
Resolvamos el siguiente problema de valores iniciales definido por las EDO simultáneas:
\begin{align*}
\dv{x}{t} - y &= e^{t} \hspace{1.5cm} x(0) =  1 \\[0.5em]
\dv{y}{t} + x &= \sin t \hspace{1.5cm} y(0) =  0
\end{align*}

Solución:
\\[0.5em]
Aplicando la transformada de Fourier de las dos EDO con las condiciones iniciales indicadas, se tiene que:
\begin{align*}
p \, \overline{x} (p) - 1 - \overline{y} (p) = \dfrac{1}{p -1} \\[0.5em]
p \, \overline{y} (p) + \overline{x} (p) = \dfrac{1}{p^{2} -1}
\end{align*}

Resolviendo éstas dos ecuaciones para $\overline{x}(p)$ y $\overline{y}(p)$, llegamos a:
\begin{align*}
\overline{x}(p) = \dfrac{p}{p^{2} + 1} + \dfrac{p}{(p - 1)(p^{2} + 1)} + \dfrac{1}{(p^{2} + 1)^{2}}
\end{align*}

Usando la transformada inversa de Laplace, se tiene:
\begin{align*}
x(t) = \dfrac{1}{2} \big[ \cos t + 2 \, \sin t +  e^{t} - t \, \cos t \big]
\end{align*}

De la misma manera se llega a:
\begin{align*}
\overline{y} (p) = - 1 - \dfrac{1}{p - 1} + p \, \overline{x} (p)
\end{align*}

Que al aplicar la transformada inversa de Laplace y simplificando la expresión, tenemos que:
\begin{align*}
y(t)= \dfrac{1}{2} \big[ t \, \sin t - e^{t} + \cos t - \sin t \big]
\end{align*}
\end{ejemplo}

\end{document}