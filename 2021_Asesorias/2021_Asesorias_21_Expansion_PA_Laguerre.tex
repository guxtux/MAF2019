\documentclass[12pt]{beamer}
\usepackage{../Estilos/BeamerMAF}
\input{../Preambulos/preambulo_Beamer_Cambridge_beaver}
\newcommand{\Cancel}[2][black]{{\color{#1}\cancel{\color{black}#2}}}
\date{}
\title{Uso y propiedades de los Polinomios asociados de Laguerre}
\author{M. en C. Gustavo Contreras Mayén}
\begin{document}
\maketitle
\fontsize{14}{14}\selectfont
\spanishdecimal{.}

\section*{Contenido}
\frame[allowframebreaks]{\tableofcontents[currentsection, hideallsubsections]}


\section{Ejercicio a resolver}
\frame{\tableofcontents[currentsection, hideothersubsections]}
\subsection{Enunciado del problema}

\begin{frame}
\frametitle{El ejercicio a resolver}
Mediante el teorema del desarrollo, demuestra que la expansión de la función $f(x) = \exp(- a \, x)$ en la base $L_{n}^{k} (x)$ dejando $k$ fijo mientras que $n$ cambia de $0$ a $\infty$, es:
\pause
\begin{align*}
\exp(-a \, x) = \dfrac{1}{(1 + a)^{1 + k}} \, &\nsum_{n=0}^{\infty} \left( \dfrac{a}{1 + a} \right)^{n} \, L_{n}^{k} (x) \\[1em]
&\mbox{con   } 0 \leq x < \infty
\end{align*}
\end{frame}

\section{Teorema del desarrollo}
\frame{\tableofcontents[currentsection, hideothersubsections]}
\subsection{Definición}

\begin{frame}
\frametitle{Definición del teorema}
El teorema del desarrollo nos permite expresar una función cualquiera en términos de una serie infinita incluyendo una función especial:
\pause
\begin{align*}
f(x) = \nsum_{n=0}^{\infty} c_{n} \, \mathbf{F}_{n}(x)
\end{align*}
donde $\mathbf{F}_{n}(x)$ es una función especial que depende de un parámetro $n$.
\end{frame}
\begin{frame}
\frametitle{La base completa de los $L_{n}^{k}(x)$}
Los polinomios asociados de Laguerre forman una base completa:
\begin{align*}
\left\{ \scaleto{L_{n}^{k}(x)}{3ex} \right\}
\end{align*}
\pause
por lo que de acuerdo con el teorema del desarrollo, podemos expresar una función $f(x)$ como una serie infinita del producto de unos coeficientes por los polinomios asociados de Laguerre.
\end{frame}
\begin{frame}
\frametitle{Problema a resolver}
Entonces tendremos que:
\pause
\begin{align*}
\exp(-a \, x) = \nsum_{n=0}^{\infty} c_{n} \, \scaleto{L_{n}^{k}(x)}{3ex}
\end{align*}
Debiendo obtener los respectivos coeficientes $c_{n}$.
\end{frame}

\section{Solución al ejercicio}
\frame{\tableofcontents[currentsection, hideothersubsections]}
\subsection{Manejo algebraico}

\begin{frame}
\frametitle{Paso previo a la solución}
Se requiere \enquote{adecuar} la expresión que tenemos para un siguiente paso, en donde se ocuparán algunas de las propiedades de los polinomios asociados de Laguerre.
\pause
\\
\bigskip
Dependiendo de la función especial que esté involucrada, el manejo es diferente, pero la idea es la misma.
\end{frame}
\begin{frame}
\frametitle{Multiplicando por un $1$}
Multiplicamos la función por:
\pause
\begin{align*}
\exp(-x) \, x^{k} \, L_{m}^{k}(x)
\end{align*}
para luego integrar con respecto a $x$ de $0$ a $\infty$:
\pause
\begin{align*}
\scaleint{5ex}_{\bs 0}^{\infty} \exp(- a \, x) \, \bigg[ \exp(-x) \, x^{k} \, L_{m}^{k}(x) \bigg] \dd{x}
= \\[1em]
\scaleint{5ex}_{\bs 0}^{\infty} \, \nsum_{n=0}^{\infty} c_{n} \, L_{n}^{k}(x) \, \bigg[ \exp(-x) \, x^{k} \, L_{m}^{k}(x) \bigg] \dd{x}
\end{align*}    
\end{frame}
\begin{frame}
\frametitle{Acomodamos términos}
Haciendo álgebra:
\pause
\begin{align*}
\scaleint{5ex}_{\bs 0}^{\infty} \exp( -[ a + 1] \, x) \, x^{k} \, L_{m}^{k}(x) \dd{x}
= \\[1em]
\nsum_{n=0}^{\infty} c_{n} \scaleint{5ex}_{\bs 0}^{\infty} \exp(-x) \, x^{k} \, L_{n}^{k}(x) \, L_{m}^{k}(x) \dd{x}    
\end{align*}
\pause
Usaremos la propiedad de ortogonalidad de los $L_{m}^{k}(x)$.
\end{frame}

\subsection{Usando propiedades}

\begin{frame}
\frametitle{La propiedad de ortogonalidad}
La propiedad de ortogonalidad de los polinomios asociados de Laguerre es:
\pause
\begin{align*}
\scaleint{5ex}_{\bs 0}^{\infty} \exp(-x) \, x^{k} \, L_{n}^{k}(x) \, L_{m}^{k}(x) \dd{x} = \dfrac{(n + k)!}{n!} \, \delta_{mn}
\end{align*}
\end{frame}
\begin{frame}
\frametitle{Lado derecho de la igualdad}
Se tendrá entonces que el lado derecho de la igualdad es:
\begin{eqnarray*}
= \nsum_{n=0}^{\infty} c_{n} \, \dfrac{(n + k)!}{n!} \, \delta_{mn} = \pause c_{n} \, \dfrac{(n + k)!}{n!}
\end{eqnarray*}
\end{frame}
\begin{frame}
\frametitle{Regresando a la igualdad}
Entonces el lado izquierdo de la igualdad es:
\pause
\begin{align*}
\scaleint{5ex}_{\bs 0}^{\infty} \exp( -[ a + 1] \, x) \, x^{k} \, L_{m}^{k}(x) \dd{x} = c_{n} \, \dfrac{(n + k)!}{n!}
\end{align*}
\pause
De donde podemos despejar a los coeficientes $c_{n}$.
\end{frame}
\begin{frame}
\frametitle{Expresión para los coeficientes}
Al separar los $c_{n}$, llegamos a:
\pause
\begin{align*}
c_{n} = \dfrac{n!}{(n+ k)!} \, \scaleint{5ex}_{\bs 0}^{\infty} \exp( -[ a + 1] \, x) \, x^{k} \, L_{m}^{k}(x) \dd{x}
\end{align*}
\pause
Para resolver la integral, tendremos que ocupar otra de las propiedades de los $L_{m}^{k}(x)$.
\end{frame}
\begin{frame}
\frametitle{Uso de la fórmula de Rodrigues}
La expresión que define a la fórmula de Rodrigues para los polinomios asociados de Laguerre es:
\begin{align*}
\scaleto{L_{m}^{k}(x)}{3ex} = \dfrac{e^{x} \, x^{-k}}{n!} \, \dv[n]{x} \bigg( e^{-x} \, x^{n+k} \bigg)
\end{align*}
\end{frame}
\begin{frame}
\frametitle{Con la fórmula de Rodrigues}
La expresión de los coeficientes pasa a ser:
\begin{align*}
c_{n} = &\dfrac{n!}{(n+ k)!} \, \scaleint{5ex}_{\bs 0}^{\infty} \exp( -[ a + 1] \, x) \, x^{k} \, \times \\[1em]  
&\times \bigg[ \dfrac{e^{x} \, x^{-k}}{n!} \, \dv[n]{x} \bigg( e^{-x} \, x^{n+k} \bigg) \bigg] \dd{x}
\end{align*}
\pause
Organizamos y reducimos los términos de la expresión.
\end{frame}
\begin{frame}
\frametitle{Expresión simplificada}
Se tiene que:
\pause
\begin{align*}
c_{n} = &\dfrac{ \Cancel[red]{n}!}{\Cancel[red]{n!} (n+ k)!} \, \scaleint{5ex}_{\bs 0}^{\infty} \exp( -[ a \, \Cancel[red]{+ 1} \Cancel[red]{- 1}] \, x) \, x^{\Cancel[red]{k}-\Cancel[red]{k}} \, \times \\[1em]  
&\times \bigg[ \dv[n]{x} \bigg( e^{-x} \, x^{n+k} \bigg) \bigg] \dd{x}
\end{align*}
\end{frame}
\begin{frame}
\frametitle{Expresión simplificada}
La expresión para los coeficientes es:
\begin{align*}
c_{n} = &\dfrac{ 1}{(n+ k)!} \, \scaleint{5ex}_{\bs 0}^{\infty} \exp( - a \, x) \, \dv[n]{x} \bigg( e^{-x} \, x^{n+k} \bigg) \dd{x}
\end{align*}
Tenemos que el integrando involucra la derivada de orden $n$.
\end{frame}
\begin{frame}
\frametitle{Resolviendo la integral}
La integral se resuelve por partes, donde:
\begin{align*}
u = \exp(-a \, x) \hspace{1cm} \dd{u} = - a \, \exp(-a \, x) \dd{u} 
\end{align*}
\pause
\begin{align*}
\dd{v} = \dv[n]{x} \bigg[ e^{-x} \, x^{n+k} \bigg] \hspace{1cm} v = \dv[n-1]{x} \bigg[ e^{-x} \, x^{n+k+1} \bigg]
\end{align*}
\pause
Usaremos de nuevo la fórmula de Rodrigues para expresar el valor de $v$.
\end{frame}
\begin{frame}
\frametitle{Un resultado oportuno}
El valor de $v$ será entonces:
\pause
\begin{align*}
v = (n - 1)! \, e^{-x} \, x^{k+1} \, \scaleto{L_{n-1}^{k+1}(x)}{3ex}
\end{align*}
\pause
que habrá que ocupar en el cálculo de los exponentes $c_{m}$:
\end{frame}
\begin{frame}
\frametitle{Resultado de la integración por partes}
Se tiene que:
\begin{align*}
c_{n} &= \dfrac{ 1}{(n+ k)!} \, \bigg[ \exp(-[a + 1] \, x) \, x^{k+1} \, (n - 1)! L_{n-1}^{k+1}(x) \bigg] \bigg\vert_{0}^{\infty} + \\[1em]
&+ a \, \scaleint{5ex}_{\bs 0}^{\infty} \exp( - a \, x) \, \dv[n-1]{x} \bigg( e^{-x} \, x^{n+k} \bigg) \dd{x}
\end{align*}
\pause
Estudiemos el término entre corchetes.
\end{frame}
\begin{frame}
\frametitle{Término entre corchetes}
\begin{align*}
\bigg[ \exp(-[a + 1] \, x) \, x^{k+1} \, (n - 1)! L_{n-1}^{k+1}(x) \bigg] \bigg\vert_{0}^{\infty}
\end{align*}
\pause
Cuando $x \to \infty$, todo el corchete de anula, debido a que el término $\exp(-[a + 1] \, x) \to 0$, \pause el corchete se cancela.
\\
\bigskip
\pause
Ahora que cuando $x \to 0$, todo el corchete también se anula, debido a que el término $x^{k+1} = 0$, el corchete también se cancela.
\end{frame}
\begin{frame}
\frametitle{Continuando con el cálculo de los $c_{n}$}
Entonces se tiene que los coeficientes son:
\begin{align*}
c_{n} = a \, \scaleint{5ex}_{\bs 0}^{\infty} \exp( - a \, x) \, \dv[n-1]{x} \bigg( e^{-x} \, x^{n+k} \bigg) \dd{x}
\end{align*}
\pause
Por lo que nuevamente integramos por partes,
donde:
\pause
\begin{align*}
u = \exp(-a \, x) \hspace{1cm} \dd{u} = - a \, \exp(-a \, x) \dd{u} 
\end{align*}
\pause
\begin{align*}
\dd{v} = \dv[n-1]{x} \bigg[ e^{-x} \, x^{n+k} \bigg] \hspace{1cm} v = \dv[n-2]{x} \bigg[ e^{-x} \, x^{n+k} \bigg]
\end{align*}
\end{frame}
\begin{frame}
\frametitle{Ocupando de nuevo la fórmula de Rodrigues}
De la expresión de $v$, con la fórmula de Rodrigues, tenemos que:
\begin{align*}
v = (n - 2)! \, e^{-x} \, x^{k+2} \, \scaleto{L_{n-2}^{k+2}(x)}{3ex}
\end{align*}
\pause
Para ocuparlo en la expresión de los $c_{n}$.
\end{frame}
\begin{frame}
\frametitle{Integrando por partes de nuevo}
Resultar ser que:
\begin{align*}
c_{n} &= \dfrac{a}{(n+ k)!} \, \bigg[ \exp(-[a + 1] \, x) \, x^{k+2} \, (n - 2)! L_{n-2}^{k+2}(x) \bigg] \bigg\vert_{0}^{\infty} + \\[1em]
&+ a \, \scaleint{5ex}_{\bs 0}^{\infty} \exp( - a \, x) \, \dv[n-2]{x} \bigg( e^{-x} \, x^{n+k} \bigg) \dd{x}
\end{align*}
\pause
Otra vez estudiemos el término entre corchetes.
\end{frame}
\begin{frame}
\frametitle{Término entre corchetes}
\begin{align*}
\bigg[ \exp(-[a + 1] \, x) \, x^{k+2} \, (n - 2)! \scaleto{L_{n-2}^{k+2}(x)}{3ex} \bigg] \bigg\vert_{0}^{\infty}
\end{align*}
\pause
Cuando $x \to \infty$, el término entre corchetes se anula, debido a que el término $\exp(-[a + 1] \, x) \to 0$
\\
\bigskip
\pause
Ahora que cuando $x \to 0$, el término $x^{k+2} \to  0$, entonces el corchete también se anula.
\end{frame}    
\begin{frame}
\frametitle{Resultado de la segunda integración}
Entonces se tiene que los coeficientes ahora son:
\begin{align*}
c_{n} = \dfrac{a}{(n + k)!} \, \scaleint{5ex}_{\bs 0}^{\infty} \exp( - a \, x) \, \dv[n-2]{x} \bigg( e^{-x} \, x^{n+k} \bigg) \dd{x}
\end{align*}
\pause
Seguimos resolviendo esta integral por partes $n - 2$ veces, para así obtener:
\end{frame}
\begin{frame}
\frametitle{Resultado luego de integrar $m-2$ veces}
Llegamos al resultado:
\pause
\begin{eqnarray*}
c_{n} &=& \dfrac{a^{n}}{(n + k)!} \, \scaleint{5ex}_{\bs 0}^{\infty} \exp( - a \, x) \, \exp(-x) \, x^{n+k} \dd{x} = \\[1em] \pause
&=& \dfrac{a^{n}}{(n + k)!} \, \scaleint{5ex}_{\bs 0}^{\infty} \exp(-[a + 1] \, x) \, x^{n+k} \dd{x}
\end{eqnarray*}
\pause
Para resolver esta integral haremos un cambio de variable.
\end{frame}
\begin{frame}
\frametitle{Cambio de variable}
Hacemos el cambio de variable: $x = t / (a + 1)$, por lo que:
\pause
\begin{eqnarray*}
\dd{x} &=& \dfrac{\dd{t}}{(a + 1)}
\\[1em] \pause
\exp([-[a + 1] x]) &=& \exp(-t) \\[1em] \pause
x^{n+k} &=& \dfrac{t^{n+k}}{(a + 1)^{n+k}}
\end{eqnarray*}
\end{frame}
\begin{frame}
\frametitle{Cambio de variable}
Así la integral se transforma en:
\pause
\begin{align*}
c_{n} = \dfrac{a^{n}}{(n + k)! (a + 1)^{n+k+1}} \, \scaleint{5ex}_{\bs 0}^{\infty} e^{-t} \, t^{n+k} \dd{t}
\end{align*}
\pause
del integrando reconocemos a la función Gamma, es decir:
\pause
\begin{align*}
\scaleint{5ex}_{\bs 0}^{\infty} e^{-t} \, t^{n+k} \dd{t} = (n + k)!
\end{align*}
\end{frame}
\begin{frame}
\frametitle{Simplificando la expresión}
Tendremos entonces que:
\pause
\begin{eqnarray*}
c_{n} &=& \dfrac{a^{n}}{(n + k)! (a + 1)^{n+k+1}} \, \dfrac{\Cancel[blue]{(n + k)!}}{\Cancel[blue]{(n + k)!}} = \\[1em] \pause
&=& \dfrac{1}{(a + 1)^{1+k}} \, \left( \dfrac{a}{a + 1} \right)^{n}
\end{eqnarray*}
\end{frame}
\subsection*{Conclusión}
\begin{frame}
\frametitle{Resultado final}
Al haber definido los coeficientes $c_{n}$, concluimos que mediante el teorema del desarrollo, en la base $\left\{ \scaleto{L_{n}^{k}(x)}{3ex}\right\}$  con $k$ fijo, variando $n$ desde $0$ hasta infinito, la función: 
\begin{align*}
\exp(-a \, x) = \nsum_{n=0}^{\infty} c_{n} \, \scaleto{L_{n}^{k} (x)}{3ex}
\end{align*}
\end{frame}
\begin{frame}
\frametitle{Resultado final}
Se expresa como:
\pause
\begin{align*}
\exp(-a \, x) = \dfrac{1}{(1+a)^{1+k}} \nsum_{n=0}^{\infty} \left( \dfrac{a}{1 + a} \right)^{n} \, \scaleto{L_{n}^{k} (x)}{3ex} \qed
\end{align*}    
\end{frame}
\end{document}