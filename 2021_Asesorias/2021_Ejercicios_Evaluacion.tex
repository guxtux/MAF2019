\documentclass[12pt]{article}
\usepackage[left=0.25cm,top=1cm,right=0.25cm,bottom=1cm]{geometry}
\textwidth = 20cm
\hoffset = -1cm
\usepackage[utf8]{inputenc}
\usepackage[spanish,es-tabla]{babel}
\usepackage[autostyle,spanish=mexican]{csquotes}
\usepackage[tbtags]{amsmath}
\usepackage{nccmath}
\usepackage{amsthm}
\usepackage{amssymb}
\usepackage{graphicx}
\usepackage{standalone}
\usepackage[outdir=./]{epstopdf}
\usepackage{siunitx}
\usepackage{physics}
\usepackage{color}
\usepackage{float}
\usepackage{multicol}
%\usepackage{milista}
\usepackage{enumitem}
\usepackage{anyfontsize}
\usepackage{anysize}
\usepackage{enumitem}
\usepackage{capt-of}
\usepackage{bm}
\usepackage{relsize}
\usepackage{placeins}
\usepackage{empheq}
\usepackage{cancel}
\usepackage{wrapfig}
\spanishdecimal{.}
\renewcommand{\baselinestretch}{1.5} 
\renewcommand\labelenumii{\theenumi.{\arabic{enumii}}}
\newcommand{\ptilde}[1]{\ensuremath{{#1}^{\prime}}}
\newcommand{\stilde}[1]{\ensuremath{{#1}^{\prime \prime}}}
\newcommand{\ttilde}[1]{\ensuremath{{#1}^{\prime \prime \prime}}}
\newcommand{\ntilde}[2]{\ensuremath{{#1}^{(#2)}}}


\title{Ejercicios para evaluación} \vspace{-3ex}
\author{M. en C. Gustavo Contreras Mayén}
\date{ }
\newcommand{\Cancel}[2][black]{{\color{#1}\cancel{\color{black}#2}}}

\begin{document}
\vspace{-4cm}
\maketitle
\fontsize{14}{14}\selectfont

<<<<<<< Updated upstream
=======
\textbf{Indicaciones: } La solución a cada ejercicio deberá de hacerse lo más completa posible, si se requiere el uso de factores de escala, habrá que obtenerlos (aunque se tenga el material en donde se mencionen, así podrán corroborar sus resultados), en el caso de los operadores diferenciales, ocupen las expresiones para un sistema coordenado generalizado, precisamente que requiere del uso de los factores de escala.
\par
Los ejercicios que incluyen $4$ incisos, el puntaje para cada inciso es de $0.25$. Al tener los $4$ incisos bien resueltos, se le otorgará $1$ punto al problema. En caso de no tener los $4$ incisos, se sumará el puntaje parcial de los que estén bien resueltos.

>>>>>>> Stashed changes
\section{La física y la geometría.}

\begin{enumerate}
%Ref. Arfken 
\item Demuestra que las componentes de velocidad y aceleración en un sistema coordenado esférico son las siguientes:
<<<<<<< Updated upstream
\begin{align*}
v_{r} &= \dot{r} \\[0.5em]
v_{\theta} &= r \, \dot{\theta} \\[0.5em]
v_{\varphi} &= r \, \sin \theta \, \dot{\varphi} \\[0.5em]
a_{r} &= \ddot{r} - r \, \dot{\theta}^{2} - r \, \sin^{2} \theta \, \dot{\varphi}^{2} \\[0.5em]
a_{\theta} &= r \, \ddot{\theta} + 2 \, \dot{r} \, \dot{\theta} - r \, \sin \theta \, \cos \theta \, \dot{\varphi}^{2} \\[0.5em]
a_{\varphi} &= r \, \sin \theta \, \ddot{\varphi} + 2 \, \dot{r} \, \sin \theta \, \dot{\varphi} + 2 \, r \, \cos \theta \, \dot{\theta} \, \dot{\varphi}
\end{align*}
Considera que:
\begin{align*}
\vb{r}(t) &= \vu{r}(t) \, r(t) = \\[0.5em]
&= \bigg[ \vu{x} \sin \theta (t) \cos \varphi (t) + \vu{y} \sin \theta (t) \sin \varphi (t) + \vu{z} \cos \theta (t) \bigg] \, r(t)
\end{align*}
%Boas (2005) - Chap. 10 - Problema 18
\item Evalúa las siguientes expresiones en un sistema de coordenadas cilíndrico:
\begin{align*}
\curl{\ln r \, \vu{e}_{z}} \hspace{1cm} \grad{\ln r} \hspace{1cm} \div{(r \, \vu{e}_{r} + z \, \vu{e}_{z})}
\end{align*}
=======
\begin{gather}
\scalebox{1.15}{$
\begin{align*}
v_{r} &= \dot{r} \\
v_{\theta} &= r \, \dot{\theta} \\
v_{\varphi} &= r \, \sin \theta \, \dot{\varphi} \\
a_{r} &= \ddot{r} - r \, \dot{\theta}^{2} - r \, \sin^{2} \theta \, \dot{\varphi}^{2} \\
a_{\theta} &= r \, \ddot{\theta} + 2 \, \dot{r} \, \dot{\theta} - r \, \sin \theta \, \cos \theta \, \dot{\varphi}^{2} \\
a_{\varphi} &= r \, \sin \theta \, \ddot{\varphi} + 2 \, \dot{r} \, \sin \theta \, \dot{\varphi} + 2 \, r \, \cos \theta \, \dot{\theta} \, \dot{\varphi}
\end{align*} $}
\end{gather}
Considera que:
\begin{gather}
\scalebox{1.05}{$    
\begin{align*}
\vb{r}(t) &= \vu{r}(t) \, r(t) = \\[0.5em]
&= \bigg[ \vu{x} \sin \theta (t) \cos \varphi (t) + \vu{y} \sin \theta (t) \sin \varphi (t) + \vu{z} \cos \theta (t) \bigg] \, r(t)
\end{align*} $}
\end{gather}
%Boas (2005) - Chap. 10 - Problema 18
\item Evalúa las siguientes expresiones en un sistema de coordenadas cilíndrico:
\begin{gather}
\scalebox{1.05}{$
\begin{align*}
\hspace{-4cm}
\curl{\ln r \, \vu{e}_{z}}, \hspace{0.5cm} \grad{\ln r}, \hspace{0.5cm} \div{(r \, \vu{e}_{r} + z \, \vu{e}_{z})}
\end{align*} $}
\end{gather}
>>>>>>> Stashed changes
%Ref. Kirkwood
\item Para una esfera de radio $r$, calcula el volumen en un sistema de coordenadas oblatas. Considera que $a = 1$.
\item La inductancia magnética $\vu{B}$ es el rotacional del potencial magnético $\vu{A}$. Supongamos que en un sistema coordenado bipolar, $\vu{A} = - c \, \eta \, \vu{e}_{z}$. Calcula $\vu{B}$. Este problema describe el caso de dos alambres que conducen el mismo valor de corriente en direcciones paralelas y opuestas al eje $z$.
%Ref. Boas (2005) Chap. 11 - 13 Problems
\item A partir de la definición de la función Beta $B(m, n)$, demuestra que:
\begin{align*}
B(m, n) \, B(m + n, k) = B(n, k) \, B(n + k, m)
\end{align*}
\end{enumerate}

<<<<<<< Updated upstream
\section{Técnicas de solución.}

\begin{enumerate}
\item Describe el tipo de ecuación y las regiones en las que es de tipo hiperbólica, parabólica y/o elíptica. Recuerda que debes de justificar el por qué de tu respuesta, es decir, deberás de realizar las operaciones y cálculos necesarios para tu respuesta.

=======
\newpage
\section{Técnicas de solución.}

\begin{enumerate}
\item Describe el tipo de ecuación y las regiones en las que es de tipo hiperbólica, parabólica y/o elíptica. Recuerda que debes de justificar el por qué de tu respuesta (calcular el discriminante).
>>>>>>> Stashed changes
\begin{enumerate}
\Large
\item $u_{xx} - u_{xy} - 2 \, u_{yy} = 0$
\item $u_{xx} + 2 \, u_{xy} + u_{yy} = 0$
\item $2 \, u_{xx} + 4 \, u_{xy} + 3 \, u_{yy} - 5 \, u = 0$
\item $e^{x y} \, u_{xx} + (\sinh x) \, u_{yy} + u = 0$
\end{enumerate}
\item Para la siguiente ecuación diferencial parcial, ¿a qué ecuaciones diferenciales ordinarias llegamos cuando se ocupa el método de separación de variables?
\begin{align*}
\mbox{\Large $ \displaystyle
\pdv{u}{t} = \dfrac{k}{r} \, \pdv{r} \left( r \, \pdv{u}{r} \right)$}
\end{align*}
%Ref. Pinchover 5.7 - 1
\item Con el método de separación de variables resuelve la siguiente EDP:
\begin{align*}
\mbox{\Large $u_{t} = 17 \, u_{xx} \hspace{1.5cm} 0 < x < \pi, \hspace{0.5cm} t > 0 $}
\end{align*}
con las condiciones de frontera:
\begin{align*}
u(0,t) = u(\pi, t) = 0 \hspace{1cm} t \geq 0
\end{align*}
y las condiciones iniciales:
\begin{align*}
u(x, 0) = \begin{cases}
0 & 0 < x \leq \dfrac{\pi}{2} \\
2 & \dfrac{\pi}{2} < x \leq \pi
\end{cases}
\end{align*}

\item Clasifica los puntos singulares de las siguientes EDO, es decir, identifica si son puntos singulares regulares o irregulares.
\begin{enumerate}
\Large
\item $x^{3} \, \stilde{y} + 4 \, x^{2} \, \ptilde{y} + 3 \, y = 0$
\item $x \, \stilde{y} - (x + 3)^{-2} \, y = 0$
\item $(x^{2} - 9)^{2} \, \stilde{y} + (x + 3) \, \ptilde{y} + 2 \, y = 0$
\item $\stilde{y} - \dfrac{1}{x} \, \ptilde{y} + \dfrac{1}{(x - 1)^{3}} \, y = 0$
\end{enumerate}
%Ref. Arfken - 9.5.5
\item Con el método de Frobenius resuelve la ecuación de Legendre:
\begin{align*}
\mbox{ \Large $
(1 - x^{2}) \, \stilde{y} - 2 \, x \, \ptilde{y} + n (n + 1) \, y = 0$}
\end{align*}
<<<<<<< Updated upstream
A modo de guía para este ejercicio, comprueba que:
=======
A modo de guía para este ejercicio, los siguientes elementos te indicarán si vas por el camino correcto para la solución.
>>>>>>> Stashed changes
\begin{itemize}
\item La ecuación de índices es: $r (r - 1) = 0$.
\item Que con la raíz $r_{1} = 0$, la relación de recurrencia es:
\begin{align*}
a_{n+2} = \dfrac{n (n + 1) - n (n + 1)}{(n + 1)(n + 2)} \, a_{n}
\end{align*}
\item La solución con esta raíz $r_{1}$, nos genera exponentes pares de $x$ (haciendo que $a_{1} = 0$), es:
\begin{align*}
y_{\mbox{\small{par}}} = a_{0} \bigg[ 1 - \dfrac{n (n + 1)}{2!} \, x^{2} + \dfrac{n (n - 2)(n + 1)(n + 3)}{4!} \, x^{4} + \ldots \bigg]
\end{align*}
\item Que con la raíz $r_{2} = 1$, la relación de recurrencia es:
\begin{align*}
a_{n+2} = \dfrac{(n + 1)(n + 2) - n (n + 1)}{(n + 2)(n + 3)} \, a_{n}
\end{align*}
\item La solución con esta raíz $r_{2}$, nos genera exponentes impares de $x$ (haciendo que $a_{1} = 1$), es:
\begin{align*}
y_{\mbox{\small{impar}}} &= a_{1} \bigg[ x - \dfrac{n (n - 1)(n + 2)}{3!} \, x^{3} + \\[1em]
&+ \dfrac{n (n - 1)(n - 3)(n + 2)(n + 4)}{5!} \, x^{5} + \ldots \bigg]
\end{align*}
\end{itemize}

\end{enumerate}


\end{document}