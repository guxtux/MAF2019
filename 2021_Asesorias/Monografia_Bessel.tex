\documentclass[hidelinks,12pt]{article}
\usepackage[left=0.25cm,top=1cm,right=0.25cm,bottom=1cm]{geometry}
%\usepackage[landscape]{geometry}
\textwidth = 20cm
\hoffset = -1cm
\usepackage[utf8]{inputenc}
\usepackage[spanish,es-tabla]{babel}
\usepackage[autostyle,spanish=mexican]{csquotes}
\usepackage[tbtags]{amsmath}
\usepackage{nccmath}
\usepackage{amsthm}
\usepackage{amssymb}
\usepackage{mathrsfs}
\usepackage{graphicx}
\usepackage{subfig}
\usepackage{standalone}
\usepackage[outdir=./Imagenes/]{epstopdf}
\usepackage{siunitx}
\usepackage{physics}
\usepackage{color}
\usepackage{float}
\usepackage{hyperref}
\usepackage{multicol}
%\usepackage{milista}
\usepackage{anyfontsize}
\usepackage{anysize}
%\usepackage{enumerate}
\usepackage[shortlabels]{enumitem}
\usepackage{capt-of}
\usepackage{bm}
\usepackage{relsize}
\usepackage{placeins}
\usepackage{empheq}
\usepackage{cancel}
\usepackage{wrapfig}
\usepackage[flushleft]{threeparttable}
\usepackage{makecell}
\usepackage{fancyhdr}
\usepackage{tikz}
\usepackage{bigints}
\usepackage{scalerel}
\usepackage{pgfplots}
\usepackage{pdflscape}
\pgfplotsset{compat=1.16}
\spanishdecimal{.}
\renewcommand{\baselinestretch}{1.5} 
\renewcommand\labelenumii{\theenumi.{\arabic{enumii}})}
\newcommand{\ptilde}[1]{\ensuremath{{#1}^{\prime}}}
\newcommand{\stilde}[1]{\ensuremath{{#1}^{\prime \prime}}}
\newcommand{\ttilde}[1]{\ensuremath{{#1}^{\prime \prime \prime}}}
\newcommand{\ntilde}[2]{\ensuremath{{#1}^{(#2)}}}

\newtheorem{defi}{{\it Definición}}[section]
\newtheorem{teo}{{\it Teorema}}[section]
\newtheorem{ejemplo}{{\it Ejemplo}}[section]
\newtheorem{propiedad}{{\it Propiedad}}[section]
\newtheorem{lema}{{\it Lema}}[section]
\newtheorem{cor}{Corolario}
\newtheorem{ejer}{Ejercicio}[section]

\newlist{milista}{enumerate}{2}
\setlist[milista,1]{label=\arabic*)}
\setlist[milista,2]{label=\arabic{milistai}.\arabic*)}
\newlength{\depthofsumsign}
\setlength{\depthofsumsign}{\depthof{$\sum$}}
\newcommand{\nsum}[1][1.4]{% only for \displaystyle
    \mathop{%
        \raisebox
            {-#1\depthofsumsign+1\depthofsumsign}
            {\scalebox
                {#1}
                {$\displaystyle\sum$}%
            }
    }
}
\def\scaleint#1{\vcenter{\hbox{\scaleto[3ex]{\displaystyle\int}{#1}}}}
\def\bs{\mkern-12mu}


%\usepackage{showframe}
\title{Monografía funciones Bessel \\ \large {Matemáticas Avanzadas de la Física} \vspace{-3ex}}
\author{M. en C. Gustavo Contreras Mayén}
\date{ }
\begin{document}
\maketitle
\fontsize{14}{14}\selectfont
\renewcommand\arraystretch{2}
\vspace*{-3cm}
\begin{table}[H]
    \centering
\begin{tabular}{| p{5cm} | p{12cm} |} \hline
\multicolumn{2}{|c|}{\textbf{Polinomios ordinarios de Legendre}} \\ \hline
Problema(s) de la Física & \makecell[l]{Ecuación de Laplace en coordenadas cilíndricas \\ Cilindros dentro de cilindros \\ Membranas rígidas} \\ \hline
Geometría & Sistema coordenado cilíndrico \\ \hline
Ecuación diferencial & \(\displaystyle
x^{2} \, \stilde{y} (x) +  x \, \ptilde{y} (x) +  (x^{2} - \nu^{2}) \, y(x) = 0
\) \\ \hline
Soluciones & \makecell[l]{ Con $\nu$ real y positivo: \\ \( J_{v} (x) = \sum_{n=0}^{\infty} \dfrac{(-1)^{n}}{n! \, \Gamma (\nu + n + 1)} \left( \dfrac{x}{2} \right)^{\nu+2n} \) \\ \( J_{-v} (x) = \sum_{n=0}^{\infty} \dfrac{(-1)^{n}}{n! \, \Gamma (-\nu + n + 1)} \left( \dfrac{x}{2} \right)^{-\nu+2n} \) \\ Con $\nu$ entero: \\
La primera solución es $J_{\nu}(x)$ y la segunda solución es: \\
\( Y_{n} (x) = \dfrac{\cos \nu \, \pi \, J_{\nu} (x) - J_{-\nu} (x)}{\sin \nu \, \pi}
\)} \\ \hline
Solución general & \makecell[l]{ Con $\nu$ real y positivo: \\ \( \displaystyle y(x) = c_{1} \, J_{\nu}(x) + c_{2} \, J_{-\nu} (x) \) \\ Con $\nu$ entero: \\ \( \displaystyle y(x) = C_{1} \, J_{\nu}(x) + C_{2} \, Y_{\nu} (x) \) } \\ \hline
Función generatriz & \(\displaystyle \exp(\dfrac{x \, (t - 1)}{2 \, t}) = \sum_{-\infty}^{\infty} t^{n} \, J_{n} (x) \) \\ \hline
Ortogonalidad & \makecell[l]{ \( \displaystyle \scaleint{5ex}_{\bs 0}^{a} x \, J_{\nu} (\lambda_{i} \, x) \, &J_{\nu} (\lambda_{j} \, x) \dd{x} = \\[1em]
&= \begin{cases}
0 & i \neq j \\
\dfrac{a^{2}}{2} \, J_{\nu+1}^{2} (\lambda_{i} \, a) & i = j, \hspace{0.5cm} i = 1, 2, \ldots
\end{cases} \) } \\ \hline
Relaciones de recurrencia & \makecell[l]{ \( 
 \)} \\ \hline
Paridad & \( \displaystyle P_{\ell} (-u) = (-1)^{\ell} \, P_{\ell} (u) \) \\ \hline
Expresión integral & \\ \hline
\end{tabular}
\end{table}
\end{document}