\documentclass[12pt]{article}
\usepackage[left=0.25cm,top=1cm,right=0.25cm,bottom=1cm]{geometry}
\textwidth = 20cm
\hoffset = -1cm
\usepackage[utf8]{inputenc}
\usepackage[spanish,es-tabla]{babel}
\usepackage[autostyle,spanish=mexican]{csquotes}
\usepackage[tbtags]{amsmath}
\usepackage{nccmath}
\usepackage{amsthm}
\usepackage{amssymb}
\usepackage{graphicx}
\usepackage{standalone}
\usepackage[outdir=./]{epstopdf}
\usepackage{siunitx}
\usepackage{physics}
\usepackage{color}
\usepackage{float}
\usepackage{multicol}
%\usepackage{milista}
\usepackage{enumitem}
\usepackage{anyfontsize}
\usepackage{anysize}
\usepackage{enumitem}
\usepackage{capt-of}
\usepackage{bm}
\usepackage{relsize}
\usepackage{placeins}
\usepackage{empheq}
\usepackage{cancel}
\usepackage{wrapfig}
\spanishdecimal{.}
\renewcommand{\baselinestretch}{1.5} 
\renewcommand\labelenumii{\theenumi.{\arabic{enumii}}}
\newcommand{\ptilde}[1]{\ensuremath{{#1}^{\prime}}}
\newcommand{\stilde}[1]{\ensuremath{{#1}^{\prime \prime}}}
\newcommand{\ttilde}[1]{\ensuremath{{#1}^{\prime \prime \prime}}}
\newcommand{\ntilde}[2]{\ensuremath{{#1}^{(#2)}}}


\title{Sobre el curso de MAF} \vspace{-3ex}
\author{M. en C. Gustavo Contreras Mayén}
\date{\today}
\newcommand{\Cancel}[2][black]{{\color{#1}\cancel{\color{black}#2}}}
\begin{document}
\vspace{-4cm}
\maketitle
\fontsize{14}{14}\selectfont

\section{Situación actual.}

El pasado 24 de junio recibimos un correo electrónico por parte del Consejo Técnico de la Facultad de Ciencias, tanto los estudiantes como docentes para notificarnos que entre otros asuntos, el particular sobre la manera en la que concluirá el actual semestre 2021-2.
\par
Cuando inició el paro de actividades, las autoridades pidieron respetuosamente no entrar en conflicto con el movimiento y suspensión de clases, durante estas semanas les enviamos mensajes para comentarles que aún no se contaba con el aviso de reanudación de las actividades, ya que le correspondería a las mismas autoridades indicar que se regresarían a las clases a distancia. El mensaje del pasado 24 de junio deja en claro que no hay una postura por parte de las autoridades de la Facultad para anunciar si el paro concluyó o no, dejando tanto a los alumnos como docentes en una situación de incertidumbre.
\par
En el mismo mensaje de correo electrónico, las autoridades de la Facultad señalan las fechas en las que debe de evaluarse el semestre, tanto para el sistema interno de la Facultad, como para el sistema central de la Administración Escolar.

\section{Esquema de evaluación.}

En vista de las circunstancias, y ante la petición de evaluar el semestre, les presentamos un esquema de evaluación que consideramos nos permitiría contar con elementos para establecer una calificación en el acta del curso de Matemáticas Avanzadas de la Física.
\par
Cuando se comentó el plan de asesorías para ir reactivando el ritmo de trabajo, se mencionó el incluir una serie de ejercicios para revisión, éstos ejercicios serán los que se tomaremos en cuenta para la evaluación. Los temas para los que tendríamos ejercicios son:
\begin{enumerate}
\item Tema 1 - La física y la geometría.
\item Primeras técnicas de solución.
\item Funciones especiales.
\item Transformadas integrales.
\end{enumerate}

Para el Tema 1 que se revisó al inicio del semestre y que se repasó en las asesorías, contendría una serie de nuevos ejercicios a resolver (entre 5 - 7).
\par
Para el tema de Técnicas de solución, ya revisamos la técnica de separación de variables y estamos completando la solución en series de potencias que estamos atendiendo, los ejercicios a resolver serían también entre 5 - 7.

\subsection{Material de consulta.}

Para los dos siguientes temas proponemos lo siguiente: En las semanas de vacaciones administrativas, del 5 al 23 de julio, se estarán publicando materiales de consulta y videos en la plataforma Moodle y en el canal de YouTube, para que se puedan consultar de manera previa, y así adelantar la revisión.
\par
En las siguientes tres semanas, tendríamos sesiones de videoconferencia los días lunes, miércoles y viernes con el siguiente cronograma:
\begin{enumerate}
\item Funciones especiales: 26, 28, 30 de julio y 2, 4, 6 de agosto.
\item Transformadas integrales: 9, 11, 13 de agosto.
\end{enumerate}

Si se consultan de manera previa los materiales, se contaría con una ventaja para adelantar la solución de los ejercicios. En las semanas señaladas, se revisarían los temas indicados, es decir, trabajaríamos las sesiones con materiales adicionales, que complementarían lo que estaría en las plataformas y se entregaría una lista con ejercicios para estos temas, que serían entre 5 - 7.

\subsection{Entrega de ejercicios.}

Para los temas de la física y la geometría, técnicas de solución y funciones especiales, \textbf{los ejercicios a resolver se deberán de entregar a más tardar el viernes 13 de agosto.}
\par
Para el tema de transformadas integrales, \textbf{los ejercicios a resolver se deberán de entregar a más tardar el lunes 16 de agosto.}
\par
Es claro que mientras más ejercicios resueltos entreguen, la calificación de cada ejercicio bien resuelto, aportará puntaje para el promedio. De esta manera tendremos la oportunidad de revisar y evaluar los ejercicios para promediar la calificación y darles aviso de la misma, para así asentarla en el acta correspondiente del curso y realizar la respectiva firma electrónica.
\par
Es cierto que este plan en corto tiempo nos deja en una situación que no es la mejor para el curso, es decir, en condiciones normales de tiempo, el alcance para cada tema sería distinto. Es por ello que al plantear esta estrategia para la evaluación, ustedes cuenten con la información necesaria para tomar una decisión: de permanecer inscritos en el curso y completar las actividades, o en su caso, inclinarse por solicitar la baja del curso, recordando que no les afectaría en su historial académico, así como en el número de veces que pueden llevar la asignatura, si están como alumnos regulares. 
\\[0.5em]
Esperamos su participación en la videoconferencia del miércoles 30 de junio a las 3 pm para escuchar sus opiniones, comentarios y sugerencias.
\\[1.5em]
Reciban un cordial saludo.


\end{document}
