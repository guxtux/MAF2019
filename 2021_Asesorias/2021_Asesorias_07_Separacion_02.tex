\documentclass[12pt]{beamer}
\usepackage{../Estilos/BeamerMAF}
\input{../Preambulos/preambulo_Beamer_Cambridge_beaver}
\date{}
\title{Ecuación de Helmholtz \\ \Large{Separación de variables}}
\author{M. en C. Gustavo Contreras Mayén}

\begin{document}
\maketitle
\fontsize{14}{14}\selectfont
\spanishdecimal{.}

\section*{Contenido}
\frame{\tableofcontents[currentsection, hideallsubsections]}

\section{Enunciado}
\frame{\tableofcontents[currentsection, hideothersubsections]}
\subsection{El ejercicio a resolver}

\begin{frame}
\frametitle{El problema}
Demostrar que la ecuación de Helmholtz:
\begin{align*}
\laplacian{\psi} + k^{2} \, \psi = 0
\end{align*}
\emph{es separable} en coordenadas cilíndricas circulares, \pause si $k^{2}$ se generaliza como:
\pause
\begin{align*}
k^{2} + f(\rho) + \left( \dfrac{1}{\rho^{2}} \right) \, g(\varphi) + h(z)
\end{align*}
\end{frame}
\begin{frame}
\frametitle{Reexpresando la ecuación}
La ecuación que debemos de demostrar que es separable es:
\begin{align*}
\laplacian{\psi} + \left[ k^{2} + f(\rho) + \left( \dfrac{1}{\rho^{2}} \right) \, g(\varphi) + h(z) \right] \, \psi = 0
\end{align*}
\pause
Debemos de utilizar el operador laplaciano en el sistema de coordenadas cilíndricas circulares.
\end{frame}
\begin{frame}
\frametitle{Ocupando lo que ya conocemos}
Recordemos que tenemos una expresión que nos determina el operador diferencial laplaciano en términos de un sistema de coordenadas generalizado.
\\
\bigskip
\pause
Será necesario contar con las reglas de transformación así como de los factores de escala de ese sistema, para obtener el operador, todo esto lo recuperamos del Tema 1.
\end{frame}
\begin{frame}
\frametitle{La ecuación en el sistema cilíndrico}
Tenemos entonces que la expresión resulta ser:
\pause
\begin{align*}
&{} \dfrac{1}{\rho} \, \pdv{\rho} \left( \rho \, \pdv{\psi}{\rho} \right) + \dfrac{1}{\rho^{2}} \, \pdv[2]{\psi}{\varphi} + \pdv[2]{\psi}{z} + \\[1em]
&+ \left[ k^{2} + f(\rho) + \left( \dfrac{1}{\rho^{2}} \right) \, g(\varphi) + h(z) \right] \, \psi = 0
\end{align*}
\end{frame}

\section{Resolviendo el problema}
\frame{\tableofcontents[currentsection, hideothersubsections]}
\subsection{Propuesta de solución}

\begin{frame}
\frametitle{Proponemos una solución}
Para aplicar el método de separación de variables, proponemos la siguiente solución:
\begin{align*}
\psi (\rho, \varphi, z) = R(\rho) \, \Phi (\varphi) \, Z(z)
\end{align*}
\pause
donde cada función con letra mayúscula, depende de una sola variable.
\end{frame}

\subsection{Obteniendo las derivadas}

\begin{frame}
\frametitle{Obteniendo las derivadas}
Como ya propusimos una solución, ahora hay que calcular las derivadas parciales y sustituirlas en la ecuación de Helmholtz.
\\
\bigskip
\pause
Haremos en este ejercicio el procedimiento de diferenciación.
\end{frame}
\begin{frame}
\frametitle{Calculando las derivadas}
Comenzamos con las derivadas parciales con respecto a $\rho$:
\pause
\begin{eqnarray*}
\rho \, \pdv{\psi}{\rho} = \pause \rho \, \pdv{\rho} R(\rho) \Phi (\varphi) Z(z) = \pause \rho \left( \ptilde{R} \, \Phi Z \, \right)
\end{eqnarray*}
\pause
Notemos que:
\setbeamercolor{item projected}{bg=blue!70!black,fg=yellow}
\setbeamertemplate{enumerate items}[circle]
\begin{enumerate}[<+->]
\item Cada función con mayúsculas depende de una sola variable.
\item El primado indica que tenemos una derivada ordinaria de la función con respecto a su variable.
\end{enumerate}
\end{frame}
\begin{frame}
\frametitle{Continuamos derivando con respecto a $\rho$}
La siguiente derivada parcial es:
\begin{align*}
\dfrac{1}{\rho} \pdv{\rho} \left( \rho \, \pdv{\psi}{\rho}  \right) = \pause \stilde{R} \, \Phi \, Z + \dfrac{1}{\rho} \, \ptilde{R} \, \Phi \, Z
\end{align*}
\\
\bigskip
\pause
Ya concluimos las diferenciaciones con respecto a $\rho$, por lo que podemos continuar con las otras dos variables.
\end{frame}
\begin{frame}
\frametitle{Las otras dos derivadas}
Tendremos entonces:
\begin{eqnarray*}
\dfrac{1}{\rho^{2}} \, \pdv[2]{\psi}{\varphi} &=& \pause \dfrac{1}{\rho^{2}} \, \pdv[2]{\varphi} \, R(\rho) \Phi (\varphi) Z(z) = \pause \dfrac{1}{\rho^{2}} \, R \, \stilde{\Phi} \, Z \\[1em] \pause
\pdv[2]{\psi}{z} &=& \pause \pdv[2]{z} \, R(\rho) \Phi (\varphi) Z(z) = \pause R \, \Phi \, \stilde{Z}
\end{eqnarray*}
\end{frame}
\begin{frame}
\frametitle{Juntado los elementos en la ecuación}
Al incluir las derivadas que hemos obtenido, la ecuación resulta ser:
\begin{align*}
&{} \stilde{R} \, \Phi \, Z + \dfrac{1}{\rho} \, \ptilde{R} \, \Phi \, Z + \dfrac{1}{\rho^{2}} R \, \stilde{\Phi} \, Z + R \, \Phi \, \stilde{Z} + \\[1em]
&+ \left[ k^{2} + f(\rho) + \left( \dfrac{1}{\rho^{2}} \right) \, g(\varphi) + h(z) \right] \, R \, \Phi \, Z = 0
\end{align*}
\end{frame}

\subsection{Siguiente paso: dividir entre la solución}

\begin{frame}
\frametitle{El siguiente paso}
El paso que debemos de realizar es: dividir toda la expresión entre la solución propuesta, es decir, entre: $R \, \Phi \, Z$, por lo que tenemos:
\pause
\begin{align*}
&{} \dfrac{1}{R} \left( \stilde{R} + \dfrac{1}{\rho} \, \ptilde{R} \right) + \dfrac{1}{\rho^{2}} \, \dfrac{\stilde{\Phi}}{\Phi} + \dfrac{\stilde{Z}}{Z} + \\[1em]
&+ k^{2} + f(\rho) + \left( \dfrac{1}{\rho^{2}} \right) \, g(\varphi) + h(z) = 0    
\end{align*}
\end{frame}

\subsection{Constantes de separación}

\begin{frame}
\frametitle{Reacomodando los términos}
Procedemos a reacomodar los términos por cada una de las variables, buscando que haya una dependencia en una sola de ellas en la expresión.
\\
\bigskip
\pause
Veamos lo qué pasa cuando pasamos los términos que involucran a la variable $z$ al lado derecho de la expresión.
\end{frame}
\begin{frame}
\frametitle{La expresión separada}
Entonces vemos que:
\pause
\begin{align*}
&{} \dfrac{1}{R} \left( \stilde{R} + \dfrac{1}{\rho} \, \ptilde{R} \right) + \dfrac{1}{\rho^{2}} \, \dfrac{\stilde{\Phi}}{\Phi} + k^{2} + \\[1em]
&+ f(\rho) + \left( \dfrac{1}{\rho^{2}} \right) \, g(\varphi) = - \dfrac{\stilde{Z}}{Z} - h(z)
\end{align*}
\pause
La expresión del lado izquierdo depende solo de $\rho$ y de $\varphi$, mientras que la del lado derecho depende solo de $z$.
\end{frame}
\begin{frame}
\frametitle{La constante de separación}
Para que la igualdad anterior se mantenga, la única manera posible es que ambos lados de la expresión sean iguales a una constante, en este caso, la \emph{primera constante de separación}:
\pause
\begin{align*}
- \dfrac{\stilde{Z}}{Z} - h(z) = \alpha^{2}
\end{align*}
\end{frame}
\begin{frame}
\frametitle{La ecuación resultante}
Reacomodamos los términos por cada variable, al ocupar la primera constante de separación $\alpha^{2}$, la ecuación queda expresada como:
\pause
\begin{align*}
&{} \dfrac{1}{R} \left( \stilde{R} + \dfrac{1}{\rho} \, \ptilde{R} \right) +  f(\rho) + \left( \dfrac{1}{\rho^{2}} \right) \, g(\varphi) + \\[1em]
&+ \dfrac{1}{\rho^{2}} \, \dfrac{\stilde{\Phi}}{\Phi} + k^{2} - \alpha^{2} = 0
\end{align*}
\end{frame}
\begin{frame}
\frametitle{Siguiente separación}
La expresión anterior ya depende solo de las variables $\rho$ y $\varphi$, que son independientes entre sí.
\\
\bigskip
\pause
Por lo que podemos repetir la separación de la ecuación, dejando en cada lado de la igualdad una variable.
\end{frame}
\begin{frame}
\frametitle{Siguiente separación}
Tendremos ahora que:
\pause
\begin{align*}
\dfrac{1}{R} \left( \stilde{R} + \dfrac{1}{\rho} \, \ptilde{R} \right) + f(\rho) + k^{2} - \alpha^{2} = - \dfrac{1}{\rho^{2}} \, \dfrac{\stilde{\Phi}}{\Phi} - \left( \dfrac{1}{\rho^{2}} \right) \, g(\varphi)
\end{align*}
\\
\bigskip
\pause
Aunque hay un factor $1/\rho^{2}$ en el lado derecho de la igualdad, para tener una dependencia solo de $\varphi$ hay que cancelar ese término, \pause por lo que multiplicamos toda la expresión por $\rho^{2}$.
\end{frame}
\begin{frame}
\frametitle{Ecuación resultante}
Multiplicando por el factor mencionado:
\pause
\begin{align*}
\dfrac{\rho}{R} \left( \stilde{R} + \dfrac{1}{\rho} \, \ptilde{R} \right) + \rho^{2} \bigg[ f(\rho) + k^{2} - \alpha^{2} \bigg] = - \dfrac{\stilde{\Phi}}{\Phi} - g(\varphi)
\end{align*}
\pause
Que ahora si ya tenemos de cada lado de la igualdad, la dependencia de una sola variable.
\end{frame}
\begin{frame}
\frametitle{Segunda constante de separación}
Como vimos anteriormente, para que esto sea válido, la única manera es que las funciones sean iguales a una constante: \pause la \emph{segunda constante de separación}.
\pause
\begin{align*}
- \dfrac{\stilde{\Phi}}{\Phi} - g(\varphi) = \beta^{2}
\end{align*}
\end{frame}
\begin{frame}
\frametitle{Conclusión}
Hemos demostrado que la ecuación de Helmholtz en coordenadas cilíndricas con la $k^{2}$ generalizada:
\pause
\begin{align*}
\dfrac{\rho}{R} \left( \stilde{R} + \dfrac{1}{\rho} \, \ptilde{R} \right) + \rho^{2} \bigg[ f(\rho) + k^{2} - \alpha^{2} \bigg] - \beta^{2} = 0
\end{align*}
\pause
es una \textcolor{blue}{ecuación separable}.
\end{frame}

\subsection{Un extra}

\begin{frame}
\frametitle{Particularidad de la Ec. de Helmholtz}
La ecuación diferencial de Helmholtz se puede resolver mediante la separación de variables en 11 sistemas coordenados.
\\
\bigskip
\pause
Donde 10 sistemas (con excepción del sistema coordenado paraboloide confocal) son casos particulares del sistema coordenado elipsoidal confocal.
\end{frame}
\begin{frame}
\frametitle{Sistemas donde es separable la ecuación}
\setbeamercolor{item projected}{bg=red!80!black,fg=white}
\setbeamertemplate{enumerate items}[circle]
\begin{enumerate}[<+->]
\item Cartesiano.
\item Elipsoidal confocal.
\item Paraboloide confocal.
\item Cónico.
\item Cilíndrico.
\item Cilíndrico elíptico.
\seti
\end{enumerate}
\end{frame}
\begin{frame}
\frametitle{Sistemas donde es separable la ecuación}
\setbeamercolor{item projected}{bg=red!80!black,fg=white}
\setbeamertemplate{enumerate items}[circle]
\begin{enumerate}[<+->]
\conti    
\item Esferoidal oblato.
\item Esferoidal prolato.
\item Cilíndrico parabólico.
\item Parabólico.
\item Esférico.
\end{enumerate}
\end{frame}
\begin{frame}
\frametitle{Sistemas donde es separable la ecuación}
Si $k = 0$, recuperamos la ecuación de Laplace, que es separable en otros dos sistemas coordenados:
\setbeamercolor{item projected}{bg=red!80!black,fg=white}
\setbeamertemplate{enumerate items}[circle]
\begin{enumerate}[<+->]
\item Biesférico.
\item Toroidal.
\end{enumerate}
\end{frame}
\begin{frame}
\frametitle{Propuesta}
¿Aceptas el reto de elegir al azar dos sistemas que hemos mencionado y concluir que se cumple este hecho?
\end{frame}
\begin{frame}
\frametitle{Ejercicio de repaso}
Con la finalidad de repasar lo que hemos trabajado en este ejercicio, te pedimos que demuestres que la ecuación de Helmholtz \emph{es separable} en un sistema de coordenadas esférico
\begin{align*}
\laplacian{\psi} + k^{2} \, \psi = 0
\end{align*}
si $k^{2}$ se generaliza como
\begin{align*}
k^{2} + f(r) + \dfrac{1}{r^{2}} \, g(\theta) + \dfrac{1}{r^{2} \, \sin \theta} \, h(\varphi)
\end{align*}
\end{frame}
\begin{frame}
\frametitle{Puntos a resolver}
Buscando que haya un repaso completo de lo que hemos visto, te pedimos que ocupes en este sistema coordenado esférico:
\setbeamercolor{item projected}{bg=blue!70!black,fg=yellow}
\setbeamertemplate{enumerate items}[circle]
\begin{enumerate}[<+->]
\item Las reglas de transformación de $(x, y, z) \to (r, \theta, \varphi)$
\item Calcules los factores de escala.
\item Obtengas el operador laplaciano.
\seti
\end{enumerate}
\end{frame}
\begin{frame}
\frametitle{Puntos a resolver}
\setbeamercolor{item projected}{bg=blue!70!black,fg=yellow}
\setbeamertemplate{enumerate items}[circle]
\begin{enumerate}[<+->]
\conti    
\item Ocupes una propuesta de solución
\begin{align*}
\psi(r, \theta, \varphi) = R(r)\, T(\theta) \, F(\varphi)
\end{align*}
\item Obtengas las correspondientes constantes de separación.
\item Concluyas que la ecuación admite la separación de variables.
\end{enumerate}
\end{frame}
\end{document}