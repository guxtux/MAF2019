\documentclass[12pt]{beamer}
\usepackage{../Estilos/BeamerMAF}
\input{../Preambulos/preambulo_Beamer_Cambridge_beaver}
\date{}
\title{Método de Frobenius}
\author{M. en C. Gustavo Contreras Mayén}

\begin{document}
\maketitle
\fontsize{14}{14}\selectfont

\section*{Contenido}
\frame{\tableofcontents[currentsection, hideallsubsections]}
\section{Series de potencias}
\frame{\tableofcontents[currentsection, hideothersubsections]}

\subsection{Introducción}
%Ref. Bruzzone - Introducción al método de Frobenius
\begin{frame}
\frametitle{El método de Frobenius}
El método propone la búsqueda de soluciones en series de potencias para ecuaciones diferenciales lineales de segundo orden.
\\
\bigskip
\pause
Este procedimiento requiere el deducir:
\setbeamercolor{item projected}{bg=blue!70!black,fg=yellow}
\setbeamertemplate{enumerate items}[circle]
\begin{enumerate}[<+->]
\item Una ecuación de índices y sus raíces.
\item Determinar las relaciones de recurrencia.
\item Calcular los coeficientes de las series buscadas, a partir de las raíces y las relaciones de recurrencia.
\end{enumerate}
\end{frame}
\subsection{Soluciones analíticas}
\begin{frame}
\frametitle{Soluciones analíticas}
Una clase muy extensa de ecuaciones diferenciales poseen soluciones que se expresan en series de potencias, las cuales son válidas en un dominio determinado.
\\
\bigskip
Las funciones que gozan de esta particularidad se les llama \emph{analíticas}.
\end{frame}
\begin{frame}
\frametitle{Soluciones analíticas}
 Las ecuaciones diferenciales más familiares como la ecuación de un oscilador armónico
 \begin{align*}
\ddot{x} + \omega^{2} \, x = 0
 \end{align*}
\pause
 admite soluciones del tipo
\begin{align*}
x(s) = A_{1} \, \sin( \omega \, s) + A_{2} \, \cos (\omega \, s)
\end{align*}
siendo claro que $\sin( \omega \, s)$ y $\cos( \omega \, s)$ son funciones analíticas.
\end{frame}
\begin{frame}
\frametitle{Soluciones analíticas}
De igual manera para la ecuación de un oscilador amortiguado, como en un gran número de ecuaciones de la física matemática, nos encontraremos que forman parte de este tipo de ecuaciones.
\end{frame}

% \subsection{Definición}

% \begin{frame}
% \frametitle{Definición de serie de potencias}
% Una expresión de la forma:
% \begin{align}
% a_{0} + a_{1} \, (x - x_{0}) + \ldots + a_{n} \, x^{n} = \sum_{n=0}^{\infty} a_{n} \, (x - x_{0})^{n}
% \label{eq:ecuacion_01}    
% \end{align}
% se llama \textit{serie de potencias}.
% \end{frame}
% \begin{frame}
% \frametitle{Límite de la serie}
% La serie puede estar definida por el límite
% \begin{align*}
% \lim_{N \to \infty} \sum_{n=0}^{N} a_{n} \, (x - x_{0})
% \end{align*}
% para aquellos valores de $x$ en que exista el límite.
% \\
% \bigskip
% \pause
% En ese caso, se le conoce a la serie como una \textcolor{blue}{serie convergente}.
% \end{frame}
% \begin{frame}
% \frametitle{Criterio de convergencia}
% Para determinar los valores de $x$ que cumplen la condición de convergencia, se utiliza el criterio del cociente:
% \pause
% \begin{align*}
% \lim_{n \to \infty} \dfrac{a_{n+1}}{a_{n}} = \rho \hspace{1.5cm} \begin{cases}
% \mbox{Converge si } & \rho < 1 \\
% \mbox{Diverge si } & \rho > 1
% \end{cases}
% \end{align*}
% \pause
% El criterio no clasifica si $\rho = 1$.
% \end{frame}
% \begin{frame}
% \frametitle{Criterio de convergencia}
% Más general es considerar el valor absoluto de dicho cociente, si está acotado por cierto numero $\sigma$ cuando $n \to \infty$, la serie converge cuando $\sigma < 1$.
% \end{frame}
% \begin{frame}
% \frametitle{Criterio de convergencia}
% Por lo tanto, tendríamos que
% \begin{align*}
% \rho = \lim_{n \to \infty} \abs{\dfrac{a_{n+1}}{a_{n}}} \, \abs{x - x_{0}} = L \, \abs{x - x_{0}}
% \end{align*}
% en donde
% \begin{align*}
% L = \lim_{n \to \infty} \abs{\dfrac{a_{n+1}}{a_{n}}}
% \end{align*}
% \end{frame}
% \begin{frame}
% \frametitle{Criterio de convergencia}
% Si este límite existe, se deduce por la ec. (\ref{eq:ecuacion_01}):
% \pause
% \begin{align}
% \begin{aligned}        
% \mbox{converge si } &\abs{x - x_{0}} < \dfrac{1}{L} \\[0.5em]
% \mbox{diverge si } &\abs{x - x_{0}} > \dfrac{1}{L}
% \end{aligned}
% \label{eq:ecuacion_02}    
% \end{align}
% \end{frame}
% \begin{frame}
% \frametitle{Intervalo de convergencia}
% De esta manera tendremos un intervalo de convergencia cuando $L$ existe:
% \begin{align*}
% \left( x_{0} - \dfrac{1}{L}, x_{0} + \dfrac{1}{L} \right)
% \end{align*}
% \pause
% Este intervalo es simétrico respecto de $x_{0}$, de manera tal que \emph{la serie es convergente dentro} de este intervalo y \emph{divergente fuera} del mismo.
% \end{frame}

\section{Puntos singulares}
\frame{\tableofcontents[currentsection, hideothersubsections]}
\subsection{Definiciones}

%Ref. Arfken
\begin{frame}
\frametitle{Definiendo puntos}
Se presenta el concepto de un \emph{punto singular o singularidad} (tal como se aplica a una ecuación diferencial).
\\
\bigskip
\pause
El interés en este concepto radica en su utilidad para:
\setbeamercolor{item projected}{bg=blue!70!black,fg=yellow}
\setbeamertemplate{enumerate items}[circle]
\begin{enumerate}[<+->]
\item Clasificar las EDO.
\item Revisar la viabilidad de una solución en series, ésta viabilidad es parte del teorema de Fuchs.
\end{enumerate}
\end{frame}
\begin{frame}
\frametitle{Punto ordinario}
Usando la notación $\displaystyle \dv[2]{y}{x} = \stilde{y}$, tenemos que una EDO2 es del tipo:
\begin{align}
\stilde{y} = f(x, y, \stilde{y})
\label{eq:ecuacion_09_74}
\end{align}
\pause
Ahora bien, si en la ec. (\ref{eq:ecuacion_09_74}): las funciones $y$ e $y^{\prime}$ pueden tener todos los valores finitos a $x = x_{0}$ e $y^{\prime \prime}$ permanece finita, el punto $x = x_{0}$ es un \emph{punto ordinario}.
\end{frame}
\begin{frame}
\frametitle{Punto singular}
Por otra parte, si $y^{\prime \prime}$ se vuelve infinita para cualquier selección finita de $y$ e  $y^{\prime \prime}$, el punto $x = x_{0}$ se denomina \emph{punto singular}.
\end{frame}
\begin{frame}
\frametitle{Reescribiendo la EDO2}
Si escribimos esta EDO2H (en $y$) como
\begin{align}
y^{\prime \prime} + P(x) \: y^{\prime} + Q(x) \: y = 0
\label{eq:ecuacion_09_75}
\end{align}
\pause
Una ecuación diferencial de segundo orden, lineal y homogénea, con coeficientes variables $P(x)$ y $Q(x)$.
\end{frame}
\begin{frame}
\frametitle{Punto singular}
Ahora bien, si las funciones $P(x)$ y $Q(x)$ permanecen finitas con $x = x_{0}$, el punto $x = x_{0}$ es un \emph{\textcolor{red}{punto ordinario}}.
\\
\bigskip
\pause
Al contrario, si $P(x)$ y/o $Q(x)$ divergen mientras $x \to x_{0}$, el punto $x_{0}$ es un \emph{\textcolor{green!50!black}{punto singular}}.
\end{frame}
\begin{frame}
\frametitle{Tipos de puntos singulares}
Usando la ecuación (\ref{eq:ecuacion_09_75}) podemos distinguir entre dos tipos de puntos singulares:
\pause
\setbeamercolor{item projected}{bg=blue!70!black,fg=yellow}
\setbeamertemplate{enumerate items}[circle]
\begin{enumerate}
\item Si $P(x)$ y/o $Q(x)$ divergen a medida que $x \to x_{0}$, pero
\begin{align*}
(x - x_{0}) \: P(x) \hspace{0.5cm} \mbox{y} \hspace{0.5cm} (x - x_{0})^{2} \: Q(x)
\end{align*}
permanecen finitas a medida que $x \to x_{0}$, \pause entonces el punto $x = x_{0}$ se llama \textbf{punto singular regular o punto singular no esencial}.
\seti
\end{enumerate}
\end{frame}
\begin{frame}
\frametitle{Tipos de puntos singulares}
\setbeamercolor{item projected}{bg=blue!70!black,fg=yellow}
\setbeamertemplate{enumerate items}[circle]
\begin{enumerate}
\conti 
\item Si $P(x)$ diverge más rápidamente que $\dfrac{1}{(x - x_{0})}$, de tal modo que:
\begin{align*}
(x - x_{0}) \: P(x) \to \infty \hspace{0.4cm} \mbox{a medida que} \hspace{0.4cm} x \to x_{0}
\end{align*}
\pause 
o cuando $Q(x)$ diverge más rápidamente que $\dfrac{1}{(x - x_{0})^{2}}$, de modo que:
\begin{align*}
(x - x_{0})^{2} \: Q(x) \to \infty \hspace{0.4cm} \mbox{a medida que} \hspace{0.4cm} x \to x_{0}
\end{align*}
\pause
entonces el punto $x = x_{0}$ se llama \textbf{singularidad esencial o singularidad irregular}.
\end{enumerate}
\end{frame}
\begin{frame}
\frametitle{Validez de las definiciones}
Estas definiciones son válidas para todos los valores finitos de $x_{0}$. 
\\
\bigskip
El análisis de los puntos al infinito $(x \to \infty)$ es similar al tratamiento que se hace para las funciones en variable compleja.
\end{frame}
% \begin{frame}
% \frametitle{Análisis puntos al infinito}
% Hacemos el cambio de variable $x = 1/z$, sustituyendo en la ED y entonces hacemos que $z \to 0$. 
% \\
% \bigskip
% Haciendo el cambio de variable en las derivadas:
% \begin{align}
% \dv{y(x)}{x} = \dv{y(z^{-1})}{z} \: \dv{z}{x} = - \dfrac{1}{x^{2}} \dv{y(z^{-1})}{z} = -z^{2} \: \dv{y(z^{-1})}{z}
% \label{eq:ecuacion_09_76}
% \end{align}
% \end{frame}
% \begin{frame}
% \frametitle{Análisis puntos al infinito}
% Entonces:
% \begin{align}
% \begin{aligned}
% \dv[2]{y(x)}{x} &= \dv{z} \left[ \dv{y(x)}{x} \right] \dv{z}{x} = \\
% &= (-z^{2}) \left[ -2 \: z \dv{y(z^{-1})}{z} - z^{2} \: \dv[2]{y(z^{-1})}{z} \right] = \\
% &= 2 \: z^{3} \: \dv{y(z^{-1})}{z} + z^{4} \: \dv[2]{y(z^{-1})}{z}
% \end{aligned}
% \label{eq:ecuacion_09_77}
% \end{align}
% \end{frame}
% \begin{frame}
% \frametitle{Análisis puntos al infinito}
% Usando estos resultados, podemos transformar la ecuación (\ref{eq:ecuacion_09_75}) en
% \begin{align}
% z^{4} \: \dv[2]{y}{z} + [ 2 \: z^{3} - z^{2} \: P(z^{-1})] \: \dv{y}{z} + Q(z^{-1}) \: y = 0
% \label{eq:ecuacion_09_78}
% \end{align}
% \end{frame}
% \begin{frame}
% \frametitle{Análisis puntos al infinito}
% El comportamiento en $x = \infty, (z = 0)$ entonces dependerá del comportamiento de los nuevos coeficientes
% \begin{align*}
% \dfrac{2 \: z - P(z^{-1})}{z^{2}} \hspace{1cm} \text{ y } \hspace{1cm} \dfrac{Q(z^{-1})}{z^{4}}
% \end{align*}
% a medida que $z \to 0$.
% \end{frame}
% \begin{frame}
% \frametitle{Análisis puntos al infinito}
% Si estas dos expresiones se mantienen finitas, el punto $x = \infty$ es un punto ordinario.
% \\
% \bigskip
% Si las expresiones divergen con mayor rapidez que $1/z$ y $1/z^{2}$, respectivamente, el punto $x = \infty$ es un punto regular singular, de otra manera, el punto es irregular singular (una singularidad esencial).
% \end{frame}

%Ref. Hassani 2009 Chap. 26
\section{Método de Frobenius}
\frame{\tableofcontents[currentsection, hideothersubsections]}
\subsection{El método}

\begin{frame}
\frametitle{Descripción del método}
El supuesto básico del método de Frobenius es que la solución de la ED se puede \emph{representar mediante una serie de potencias}.
\end{frame}
\begin{frame}
\frametitle{Descripción del método}
Esta no es una suposición restrictiva porque todas las funciones encontradas en aplicaciones físicas pueden escribirse como series de potencias siempre que estemos interesados en sus valores que se encuentran en su intervalo de convergencia.
\\
\bigskip
Este intervalo puede ser muy pequeño o puede cubrir toda la línea real.
\end{frame}
\begin{frame}
\frametitle{La EDO2H general}
Una ecuación diferencial ordinaria de segundo orden, lineal y  homogénea, se puede escribir como:
\begin{align}
p_{2} (x) \, \dv[2]{y}{x} + p_{1} (x) \, \dv{y}{x} + p_{0}(x) \, y = 0
\label{eq:ecuacion_26_07}    
\end{align}
\end{frame}
\begin{frame}
\frametitle{Características de las $p_{i}(x)$}
Para casi todas las aplicaciones que se encuentran en física, consideramos que $p_{0}, p_{1}, p_{2}$ son polinomios.
\\
\bigskip
\pause
Es posible que la EDO no se presente en la forma que se muestra a partir de, digamos, el método de separación de variables, pero se puede \enquote{llevar} a esa forma.
\end{frame}
\begin{frame}
\frametitle{Características de las $p_{i}(x)$}
La forma más complicada de los coeficientes de las derivadas en una ED son típicamente funciones racionales (razones de dos polinomios).
\\
\bigskip
\pause
Por lo tanto, multiplicar la EDO por el producto de los tres denominadores nos devolverá la EDO en la forma dada en la ec. (\ref{eq:ecuacion_26_07}).
\end{frame}
\begin{frame}
\frametitle{Identificando puntos singulares}
El primer paso que se debe de realizar, es identificar la presencia de singularidades esenciales, ya que el método de Frobenius servirá siempre y cuando tengamos a lo más, puntos singulares regulares.
\end{frame}
\begin{frame}
\frametitle{Identificando puntos singulares}Veremos más adelante que será común en algunos problemas, que la EDO2H tendrá singularidades irregulares, por lo que el método en sí, no sería el pertinente para resolver la EDO.
\\
\bigskip
\pause
Antes de descartar el método, podemos revisar la posibilidad de \emph{remover las singularidades} en la EDO.
\end{frame}
\begin{frame}
\frametitle{El método de Frobenius}
El siguiente paso en el método de Frobenius es \emph{asumir que existe una solución en serie de potencias infinita para $y$}.
\\
\bigskip
Es común elegir que el punto de expansión sea $x = 0$.
\end{frame}
\begin{frame}
\frametitle{El método de Frobenius}
Si $p_{2} (0) \neq 0$, solo es necesario considerar las potencias no negativas de $x$.
\\
\bigskip
\pause
Si $p_{2} (0) = 0$, la EDO pierde su carácter de \enquote{segundo orden}, y las soluciones no se revisarían en estas notas.
\end{frame}
\begin{frame}
\frametitle{Dos opciones}
Se tienen dos opciones:
\setbeamercolor{item projected}{bg=blue!70!black,fg=yellow}
\setbeamertemplate{enumerate items}[circle]
\begin{enumerate}[<+->]
\item Elegir un punto de expansión diferente a $x_{0} \neq 0$, tal que $p_{2} (x_{0}) \neq 0$.
\item Permitir las potencias no positivas de $x$ en la expansión de $y$.
\end{enumerate}
\end{frame}
\begin{frame}
\frametitle{Segunda opción}
Rara vez se utiliza la primera opción. Resulta que la forma más económica, pero general, de incorporar la segunda opción, es escribir la solución como se muestra a continuación:
\end{frame}
\begin{frame}
\frametitle{Solución supuesta}
La solución que suponemos es del tipo:
\begin{eqnarray}
\begin{aligned}
y &= x^{r} \, \sum_{n=0}^{\infty} a_{n} \, x^{n} = \\[0.5em] \pause
&= \sum_{n=0}^{\infty} a_{n} \, x^{n+r} = \\[0.5em] \pause
&= a_{0} \, x^{r} + a_{1} \, x^{r+1} + a_{2} \, x^{r+2} + \ldots
\end{aligned}
\label{eq:ecuacion_26_08}    
\end{eqnarray}
donde $r$ es un número real (no necesariamente un entero positivo) que quedará determinado por la ED.
\end{frame}
\begin{frame}
\frametitle{El valor de $a_{0}$}
Es habitual elegir $a_{0} = 1$ porque cualquier múltiplo constante de una solución también es una solución.
\\
\bigskip
\pause
Si $a_{0} \neq 1$, entonces se multiplica la serie por $1/a_{0}$ y así obtener el valor.
\end{frame}
\begin{frame}
\frametitle{Característica de la serie}
Ya que una serie de potencias es uniformemente convergente (con su radio de convergencia), entonces podemos diferenciar término a término la serie.
\end{frame}
\begin{frame}
\frametitle{Diferenciando la solución}
Por lo que al diferenciar la solución en una primera ocasión, tenemos:
\begin{eqnarray}
\begin{aligned}
\dv{y}{x} &= \sum_{n=0}^{\infty} a_{n} \, (n + r) \, x^{n+r-21} = \\[0.5em] \pause
&= r \, a_{0} \, x^{r-1} + (r + 1) \, a_{1} \, x^{r} + \ldots
\end{aligned}
\label{eq:ecuacion_26_09a}
\end{eqnarray}
\end{frame}
\begin{frame}
\frametitle{Diferenciando nuevamente la solución}
Por lo que al diferenciar por segunda vez, tenemos:
\begin{eqnarray}
\begin{aligned}
\dv[2]{y}{x} &= \sum_{n=0}^{\infty} a_{n} \, (n + r) \, (n + r - 1) \, x^{n+r-2} = \\[0.5em] \pause
&= r \, (r - 1) \, a_{0} \, x^{r-2} + (r + 1) \, r \, a_{1} \, x^{r}-1 + \ldots
\end{aligned}
\label{eq:ecuacion_26_09b}
\end{eqnarray}
\end{frame}
\begin{frame}
\frametitle{Recuperando la EDO}
Ahora sustituimos las ecuaciones (\ref{eq:ecuacion_26_08}), (\ref{eq:ecuacion_26_09a}) y (\ref{eq:ecuacion_26_09b}) en la EDO (\ref{eq:ecuacion_26_07}).
\pause
\begin{align*}
&p_{2}(x) \bigg[ \sum_{n=0}^{\infty} a_{n} \, (n + r) \, (n + r - 2) \, x^{n+r-1} \bigg] + \\[1em]
&+ p_{1} (x) \big[ \sum_{n=0}^{\infty} a_{n} \, (n + r) \, x^{n+r-1} \bigg] + p_{0}(x) \bigg[ \sum_{n=0}^{\infty} a_{n} \, x^{n+r} \bigg] = 0
\end{align*}
\end{frame}
\begin{frame}
\frametitle{Resolviendo la expresión}
Multiplicamos los polinomios en la serie, agrupamos todas las potencias distintas de $x$ y establecemos el coeficiente de cada término igual a cero.
\end{frame}
\begin{frame}
\frametitle{Ecuación de índices}
Así obtenemos un conjunto de ecuaciones cuya solución determina el valor de $r$ y de los coeficientes $a_{n}$.
\\
\bigskip
\pause
La ecuación que surge de la \emph{potencia más baja de x} involucra solo a $r$, se llama \textbf{ecuación de índices}.
\end{frame}
\begin{frame}
\frametitle{Ecuación de índices}
Esta suele ser una ecuación cuadrática en $r$ que se puede resolver para obtener el(los) posible(s) valor(es) de $r$, cada uno de los cuales conduce generalmente a una solución diferente.
\end{frame}
\begin{frame}
\frametitle{Relaciones de recurrencia}
Las otras ecuaciones que provienen de potencias superiores de $x$ permiten establecer \emph{relaciones de recurrencia}, es decir, ecuaciones que dan $a_{n}$ en términos de $a_{n-1}$ y $a_{n-2}$.
\\
\bigskip
Al iterar esta relación, se pueden obtener todos los $a_{n}$ en términos de solo dos coeficientes.
\end{frame}
\subsection{Ejercicio}
%Ref. Zill ED pág. 279
\begin{frame}
\frametitle{Ejercicio práctico}
Resuelve la siguiente EDO2H mediante el método de Frobenius:
\begin{align}
3 \, x \, y^{\prime \prime} + y^{\prime} - y = 0
\label{eq:ecuacion_04}    
\end{align}
\pause
Obtén la(s) solución(es) en términos de una serie de potencias.
\end{frame}

\subsection*{Tipos de puntos}

\begin{frame}
\frametitle{Determinando el tipo de puntos}
Cuando $p_{2}(x)$, $p_{1}(x)$ y $p_{0}(x)$ son polinomios \emph{sin factores comunes}, un punto $x = x_{0}$ es:
\setbeamercolor{item projected}{bg=blue!70!black,fg=yellow}
\setbeamertemplate{enumerate items}[circle]
\begin{enumerate}[<+->]
\item Punto ordinario, si $p_{2}(x_{0}) \neq 0$, o bien.
\item Punto singular,  si $p_{2}(x_{0}) = 0$.
\end{enumerate}
\end{frame}
\begin{frame}
\frametitle{Punto singular}
Tenemos entonces que $x = 0$ es punto singular, por lo que es posible utilizar el método de Frobenius.
\\
\bigskip
\pause
Recordemos que para que el método funciones, a lo más, debemos de tener puntos singulares.
\end{frame}

\subsection*{Solución propuesta}

\begin{frame}
\frametitle{Solución propuesta}
Se propone una solución en serie de potencias del tipo:
\begin{align*}
y(x) = \sum_{n=0}^{\infty} a_{n} \, x^{n+r} \hspace{1.5cm} a_{0} \neq 0
\end{align*}
\pause
Al diferenciar con respecto a $x$ en dos ocasiones, tenemos:
\begin{eqnarray*}
\ptilde{y} &=& \sum_{n=0}^{\infty} (n + r) \, a_{n} \, x^{n+r-1} \\[0.5em] \pause
\stilde{y} &=& \sum_{n=0}^{\infty} (n + r) \, (n + r - 1) \, a_{n} \, x^{n+r-2}
\end{eqnarray*}
\end{frame}

\subsection*{Recuperando la EDO}

\begin{frame}
\frametitle{Expresando la EDO}
Al incorporar las expresiones de las derivadas en la EDO, se tiene que:
\pause
\begin{align*}
&3 \, x \, \stilde{y} + \ptilde{y} - y = 3 \, \sum_{n=0}^{\infty} (n + r) \, (n + r - 1) \, a_{n} \, x^{n+r-1} + \\[0.5em]
&+ \sum_{n=0}^{\infty} (n + r) \, a_{n} \, x^{n+r-1} - \sum_{n=0}^{\infty} a_{n} \, x^{n+r} = 0
\end{align*}
\pause
Recomendamos escribir los términos de izquierda a derecha con las potencias más bajas a las más altas.
\end{frame}

\subsection*{Simplificación la expresión}

\begin{frame}
\frametitle{Simplificando la expresión}
Como tenemos términos que se multiplican por el mismo término $x^{n+r-1}$, factorizamos los correspondientes términos:
\pause
\begin{align*}
\sum_{n=0}^{\infty} &\bigg[ 3 \, a_{n} \, (n + r) \, (n + r - 1) + a_{n} \, (n + r) \bigg] \, x^{n+r-1} + \\[1em]
&- \sum_{n=0}^{\infty} a_{n} \, x^{n+r} = 0
\end{align*}
\end{frame}
\begin{frame}
\frametitle{Simplificando la expresión}
El factor común dentro de los corchetes es $a_{n} \, (n + r)$, así que:
\pause
\begin{align*}
\sum_{n=0}^{\infty} &\bigg[ a_{n} \, (n + r) [3 \, (n + r - 1) + 1 ] \bigg] \, x^{n+r-1} + \\[1em]
&- \sum_{n=0}^{\infty} a_{n} \, x^{n+r} = 0
\end{align*}
\end{frame}
\begin{frame}
\frametitle{Simplificando la expresión}
Entonces se obtiene:
\pause
\begin{align*}
\sum_{n=0}^{\infty} &\bigg[ a_{n} \, (n + r) (3 \, n +  3 \, r - 2 ) \bigg] \, x^{n+r-1} + \\[1em]
&- \sum_{n=0}^{\infty} a_{n} \, x^{n+r} = 0
\end{align*}
\pause
Notemos que las sumas inician en el mismo índice $n = 0$.
\end{frame}
\begin{frame}
\frametitle{Coeficiente de la potencia más baja}
Será necesario obtener el coeficiente de la potencia más baja, para ello hacemos que $n = 0$ en la primera suma, por lo tanto:
\begin{align*}
&a_{0} \, r (3 \, r {-}  2) \, x^{r-1} + \sum_{n=1}^{\infty} \bigg[ a_{n} \, (n {+} r) (3 \, n {+}  3 \, r {-} 2 ) \bigg] \, x^{n+r-1} + \\[1em]
&- \sum_{n=0}^{\infty} a_{n} \, x^{n+r} = 0
\end{align*}
\end{frame}
\begin{frame}
\frametitle{Agrupando de nuevo las sumas}
Para factorizar los términos de las sumas, necesitamos que:
\setbeamercolor{item projected}{bg=blue!70!black,fg=yellow}
\setbeamertemplate{enumerate items}[circle]
\begin{enumerate}[<+->]
\item El exponente del término $x$ sea el mismo.
\item Los índices de las sumas deben de iniciar en el mismo valor.
\end{enumerate}
\pause
Encontramos que los exponentes de $x$ son distintos y los índices de la sumas, en la primera comienza con $n = 1$, mientras que la segunda con $n = 0$.
\end{frame}
\begin{frame}
\frametitle{Agrupando de nuevo las sumas}
Si buscamos realizar la agrupación, se requiere que las sumas  tengan el mismo exponente en el término de $x$ y que comiencen con el mismo índice.
\end{frame}
\begin{frame}
\frametitle{Uso de una propiedad de las sumas infinitas}
De la teoría de las sumas infinitas, se tiene una propiedad que nos será de mucha utilidad para dejar los índices en el mismo valor inicial:
\pause
\begin{align*}
\sum_{n = k}^{\infty} f(n) = \sum_{n=0}^{\infty} f(n + k)
\end{align*}
Que usaremos en la expresión de nuestras sumas.
\end{frame}
\begin{frame}
\frametitle{Sumas con el mismo índice}
Al ocupar la propiedad anterior en la primera suma infinita, llegamos a:
\begin{align*}
&a_{0} \, r (3 \, r {-}  2) \, x^{r-1} + \\[1em]
&+ \sum_{n=0}^{\infty} \bigg[ a_{n+1} \, (n + 1 {+} r) (3 \, (n + 1) {+}  3 \, r {-} 2 ) \bigg] \, x^{n+r} + \\[1em]
&- \sum_{n=0}^{\infty} a_{n} \, x^{n+r} = 0
\end{align*}
\end{frame}
\begin{frame}
\frametitle{Sumas simplificadas}
Simplificando los términos
\begin{align*}
&a_{0} \, r (3 \, r {-}  2) \, x^{r-1} + \\[1em]
&+ \sum_{n=0}^{\infty} \bigg[ a_{n+1} \, (n + 1 {+} r) (3 \, n {+}  3 \, r {+} 1 ) \bigg] \, x^{n+r} + \\[1em]
&- \sum_{n=0}^{\infty} a_{n} \, x^{n+r} = 0
\end{align*}
\end{frame}
\begin{frame}
\frametitle{Agrupando los términos de las sumas}
Dado que las sumas tienen el mismo factor $x^{n+r}$ y comienzan con el mismo índice, $n = 0$, podemos factorizar los términos:
\pause
\begin{align*}
&a_{0} \, r (3 \, r {-}  2) \, x^{r-1} + \\[1em]
&+ \sum_{n=0}^{\infty} \bigg[ a_{n+1} \, (n + 1 {+} r) (3 \, n {+}  3 \, r {+} 1 ) - a_{n} \bigg] \, x^{n+r} = 0
\end{align*}    
\end{frame}

\subsection*{Coeficientes de la expresión}

\begin{frame}
\frametitle{Los coeficientes de la expresión}
Para que la expresión anterior se anule, los coeficientes deben de anularse, recordemos que $a_{0} \neq 0$, entonces:
\pause
\begin{eqnarray}
&{}& a_{0} \, r \, (3 \, r - 2) = 0 \label{eq:ecuacion_indices}\\[1em] \pause
&{}& a_{n+1} \, (n + 1 {+} r) (3 \, n {+}  3 \, r {+} 1 ) - a_{n} = 0 \label{eq:ecuacion_recurrencia}
\end{eqnarray}
\end{frame}

\subsection*{Ecuación de índices}

\begin{frame}
\frametitle{Ecuación de índices}
De la ecuación\footnote{En algunos textos se le conoce como \emph{ecuación indicial}.} (\ref{eq:ecuacion_indices}), como $a_{0} \neq 0$, obtenemos la \textbf{ecuación de índices}:
\pause
\begin{align*}
r \, (3 \, r - 2) = 0
\end{align*}
\pause
Que tiene por raíces:
\pause
\begin{align*}
r_{1} = \dfrac{2}{3}, \hspace{1.5cm} r_{2} = 0
\end{align*}
\end{frame}

\subsection*{Relación de recurrencia}

\begin{frame}
\frametitle{Relación de recurrencia}
De la ec. (\ref{eq:ecuacion_recurrencia}), tenemos el segundo resultado importante: la \textbf{relación de recurrencia}:
\begin{align*}
a_{n+1} (n + r + 1)(3 \, n + 3 \, r + 1) - a_{n} = 0
\end{align*}
\pause
Entonces:
\begin{align}
a_{n+1} = \dfrac{a_{n}}{(n + r + 1)(3 \, n + 3 \, r + 1)}
\label{eq:ecuacion_07}
\end{align}
\end{frame}
\begin{frame}
\frametitle{La relación de recurrencia}
Lo que nos dice la relación de recurrencia, es que podemos conocer el coeficiente $a_{n+1}$ a partir del coeficiente $a_{n}$, ocupando el valor de las raíces, y el término $n$.
\\
\bigskip
\pause
Ocuparemos el valor de las raíces en la relación de recurrencia para determinar el valor de los coeficientes, para así obtener una solución a la EDO con $r_{1}$ y la otra solución con $r_{2}$.
\end{frame}

\subsection*{Obteniendo los coeficientes}

\begin{frame}
\frametitle{Usando la primera raíz $r_{1}$}
Ocupamos la primera raíz $r_{1} = 2/3$ en la relación de recurrencia (\ref{eq:ecuacion_07}) y con $a_{0} \neq 0$:
\begin{align}
a_{n+1} = \dfrac{a_{n}}{(3 \, n + 5)(n + 1)} \hspace{1.5cm} n = 0, 1, 2, \ldots
\label{eq:ecuacion_08}    
\end{align}
\end{frame}
\begin{frame}
\frametitle{Primeros coeficientes con $r_{1}$}
Entonces:
\begin{eqnarray*}
a_{1} &=& \dfrac{a_{0}}{(3 \cdot 0 + 5)(0 + 1)}  = \dfrac{a_{0}}{5 \cdot 1} \\[0.5em] \pause
a_{2} &=& \dfrac{a_{1}}{8 \cdot 2} = \dfrac{a_{0}}{2! \, 5 \cdot 8} \\[0.5em] \pause
a_{3} &=& \dfrac{a_{2}}{11 \cdot 3} = \dfrac{a_{0}}{3! \, 5 \cdot 8 \cdot 11} \\
\vdots \\[0.5em] \pause
a_{n} &=& \dfrac{a_{0}}{n! \, 5 \cdot 8 \cdot 11 \ldots (3\, n + 2)} \hspace{1cm} n = 1, 2, 3, \ldots
\end{eqnarray*}
\end{frame}

\subsection*{Primera solución de la EDO}

\begin{frame}
\frametitle{Primera solución de la EDO}
Hemos obtenido la primera solución de la EDO: $y_{1}$ ocupando la raíz $r_{1}$:
\begin{align}
y_{1}(x) = a_{0} \, x^{2/3} \left[ 1 + \sum_{n=1}^{\infty} \dfrac{a_{0}}{n! \, 5 \cdot 8 \cdot 11 \ldots (3\, n + 2)} \, x^{n} \right]
\label{eq:ecuacion_10}    
\end{align}
\end{frame}

\subsection*{Obteniendo los coeficientes}

\begin{frame}
\frametitle{Desarrollo con la segunda raíz $r_{2}$}
La segunda raíz de la ecuación de índices: $r_{2} = 0$ nos genera una regla de recurrencia distinta a la anterior:
\begin{align}
a_{n+1} = \dfrac{a_{n}}{(n+1)(3 \, n +1)} \hspace{1.5cm} n = 0, 1, 2, \ldots
\label{eq:ecuacion_09}    
\end{align}
\end{frame}
\begin{frame}
\frametitle{Primeros coeficientes con $r_{2}$}
Los coeficientes que se obtienen son:
\begin{eqnarray*}
a_{1} &=& \dfrac{a_{0}}{(0+1)(3 \cdot 0 +1)} = \dfrac{a_{0}}{1 \cdot 1} \\[0.5em] \pause
a_{2} &=& \dfrac{a_{1}}{2 \cdot 4} = \dfrac{a_{0}}{2! \, 1 \cdot 4}  \\[0.5em] \pause
a_{3} &=& \dfrac{a_{2}}{3 \cdot 7} = \dfrac{a_{0}}{3! \, 4 \cdot 7}  \\[0.5em]
\vdots \\ \pause
a_{n} &=& \dfrac{a_{0}}{n! \, 1 \cdot 4 \cdot 7 \ldots (3 \, n - 2)} \hspace{1cm} n = 1, 2, 3, \ldots
\end{eqnarray*}
\end{frame}

\subsection*{Segunda solución de la EDO}

\begin{frame}
\frametitle{Segunda solución a la EDO}
La segunda solución de la EDO: $y_{2}(x)$ que se obtiene es:
\begin{align}
y_{2}(x) = a_{0} \, x^{0} \left[ 1 + \sum_{n=1}^{\infty} \dfrac{1}{n! \, 1 \cdot 4 \cdot 7 \ldots (3\, n - 2)} \, x^{n} \right]
\label{eq:ecuacion_11}
\end{align}    
\end{frame}

\subsection*{Propiedades de las soluciones}

\begin{frame}
\frametitle{Convergencia de las soluciones}
Se puede demostrar que las soluciones (\ref{eq:ecuacion_10}) y (\ref{eq:ecuacion_11}) convergen ambas para todos los valores finitos de $x$.
\end{frame}
\begin{frame}
\frametitle{Independencia de las series}
También es posible ver que las soluciones no son múltiplo una  de la otra, por lo que $y_{1}(x)$ y $y_{2}(x)$ son linealmente independientes con respecto a $x$.
\end{frame}
\begin{frame}
\frametitle{Principio de superposición}
Por el principio de superposición, expresamos la solución:
\begin{align*}
y(x) &= C_{1} \, y_{1} (x) + C_{2} \, y_{2} = \\[0.5em]
&= C_{1} \, \left[ x^{2/3} + \sum_{n=1}^{\infty} \dfrac{a_{0}}{n! \, 5 \cdot 8 \cdot 11 \ldots (3\, n + 2)} \, x^{n} \right] + \\[0.5em]
&+ C_{2} \, \left[ 1 + \sum_{n=1}^{\infty} \dfrac{1}{n! \, 1 \cdot 4 \cdot 7 \ldots (3\, n - 2)} \, x^{n} \right]
\end{align*}
\end{frame}

% \subsection{Casos de las raíces}

% \begin{frame}
% \frametitle{La ecuación de índices}
% Al ocupar el método de Frobenius se pueden presentar tres casos, que corresponden a la naturaleza de las raíces de la ecuación de índices.
% \end{frame}
% \begin{frame}
% \frametitle{La ecuación de índices}
% Haremos la suposición ie $r_{1}$ y $r_{2}$ son las soluciones \emph{reales} de la ecuación de índices, que cuando son distintas, $r_{1}$ representa la raíz mayor.
% \end{frame}
% \subsection*{Caso 1}
% \begin{frame}
% \frametitle{Caso 1}
% \textbf{Las raíces no difieren un entero}. Si $r_{1}$ y $r_{2}$ son distintas, pero no difieren  en un entero, entonces existen dos soluciones linealmente independientes de la ED, cuya forma es:
% \begin{subequations}
% \begin{align}
% y_{1} &= \sum_{n=0}^{\infty} a_{n} \, x^{n+r_{1}} \hspace{0.5cm} a_{0} \neq 0 \label{eq:ecuacion_14a} \\[0.5em]
% y_{2} &= \sum_{n=0}^{\infty} b_{n} \, x^{n+r_{2}} \hspace{0.5cm} b_{0} \neq 0 \label{eq:ecuacion_14b}
% \end{align}
% \end{subequations}
% \end{frame}
% \subsection*{Caso 2}
% \begin{frame}
% \frametitle{Caso 2}
% \textbf{Las raíces difieren en un entero positivo.} Si $r_{1} - r_{2} = N$, donde $N$ es un entero positivo, entonces existe dos soluciones linealmente independientes de la ED, de la forma:
% \begin{subequations}
% \begin{align}
% y_{1} &= \sum_{n=0}^{\infty} a_{n} \, x^{n+r_{1}} \hspace{0.5cm} a_{0} \neq 0 \label{eq:ecuacion_20a} \\[0.5em]
% y_{2} &= C \, y_{1} (x) \ln x + \sum_{n=0}^{\infty} b_{n} \, x^{n+r_{2}} \hspace{0.5cm} b_{0} \neq 0 \label{eq:ecuacion_20b}
% \end{align}
% \end{subequations}
% \end{frame}
% \subsection*{Caso 3}
% \begin{frame}
% \frametitle{Caso 3}
% \textbf{Las raíces son iguales.} Si $r_{1} = r_{2}$, siempre existen dos soluciones linealmente independientes de la ED, de la forma:
% \begin{subequations}
% \begin{align}
% y_{1} &= \sum_{n=0}^{\infty} a_{n} \, x^{n+r_{1}} \hspace{0.5cm} a_{0} \neq 0 \label{eq:ecuacion_21a} \\[0.5em]
% y_{2} &= y_{1} (x) \ln x + \sum_{n=0}^{\infty} b_{n} \, x^{n+r_{1}} \hspace{0.5cm} b_{0} \neq 0 \label{eq:ecuacion_21b}
% \end{align}
% \end{subequations}
% \end{frame}

% \section{Ejercicios}
% \frame{\tableofcontents[currentsection, hideothersubsections]}
% \subsection{Problemas a resolver}

% %Ref. Zill. Ejercicios 6.3
% \begin{frame}
% \frametitle{Ejercicios a cuenta}
% Determina los puntos singulares de las siguientes EDO, clasifica cada punto singular en regular o irregular.
% \begin{enumerate}
% \item $x^{3} \, y^{\prime \prime} + 4 \, x^{2} \, y^{\prime} + 3 \, y = 0$
% \item $x \, y^{\prime \prime} - (x + 3)^{-2} \, y = 0$
% \item $(x^{2} - 9)^{2} \, y^{\prime \prime} + (x + 3) \, y^{\prime} + 2 \, y = 0$
% \item $y^{\prime \prime} - \dfrac{1}{x} \, y^{\prime} + \dfrac{1}{(x - 1)^{3}} \, y = 0$
% \end{enumerate}
% \end{frame}
% \begin{frame}
% \frametitle{Ejercicios a cuenta}
% Resuelve las siguientes ED con el método de Frobenius alrededor de $x_{0} = 0$:
% \begin{enumerate}
% \item $2 \, x \, y^{\prime \prime} - y^{\prime} + 2 \, y = 0$
% \item $2 \, x \, y^{\prime \prime} + 5 \, y^{\prime} + x \, y = 0$
% \item $x (x - 1) \, y^{\prime \prime} + 3 \, y^{\prime} - 2 \, y = 0$
% \item $y^{\prime \prime} - \dfrac{3}{x} \, y^{\prime} - 2 \, y = 0$
% \end{enumerate}
% \end{frame}
% \subsection{El teorema de Fuchs}
% \begin{frame}
% \frametitle{El teorema de Fuchs}
% El teorema nos dice que se obtendrá al menos una solución en serie de potencias al aplicar el método de Frobenius si el punto de expansión es un punto singular ordinario o regular.
%\end{frame}
\end{document}