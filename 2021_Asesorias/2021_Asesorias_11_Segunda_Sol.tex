\documentclass[hidelinks,12pt]{article}
\usepackage[left=0.25cm,top=1cm,right=0.25cm,bottom=1cm]{geometry}
%\usepackage[landscape]{geometry}
\textwidth = 20cm
\hoffset = -1cm
\usepackage[utf8]{inputenc}
\usepackage[spanish,es-tabla]{babel}
\usepackage[autostyle,spanish=mexican]{csquotes}
\usepackage[tbtags]{amsmath}
\usepackage{nccmath}
\usepackage{amsthm}
\usepackage{amssymb}
\usepackage{mathrsfs}
\usepackage{graphicx}
\usepackage{subfig}
\usepackage{standalone}
\usepackage[outdir=./Imagenes/]{epstopdf}
\usepackage{siunitx}
\usepackage{physics}
\usepackage{color}
\usepackage{float}
\usepackage{hyperref}
\usepackage{multicol}
%\usepackage{milista}
\usepackage{anyfontsize}
\usepackage{anysize}
%\usepackage{enumerate}
\usepackage[shortlabels]{enumitem}
\usepackage{capt-of}
\usepackage{bm}
\usepackage{relsize}
\usepackage{placeins}
\usepackage{empheq}
\usepackage{cancel}
\usepackage{wrapfig}
\usepackage[flushleft]{threeparttable}
\usepackage{makecell}
\usepackage{fancyhdr}
\usepackage{tikz}
\usepackage{bigints}
\usepackage{scalerel}
\usepackage{pgfplots}
\usepackage{pdflscape}
\pgfplotsset{compat=1.16}
\spanishdecimal{.}
\renewcommand{\baselinestretch}{1.5} 
\renewcommand\labelenumii{\theenumi.{\arabic{enumii}})}
\newcommand{\ptilde}[1]{\ensuremath{{#1}^{\prime}}}
\newcommand{\stilde}[1]{\ensuremath{{#1}^{\prime \prime}}}
\newcommand{\ttilde}[1]{\ensuremath{{#1}^{\prime \prime \prime}}}
\newcommand{\ntilde}[2]{\ensuremath{{#1}^{(#2)}}}

\newtheorem{defi}{{\it Definición}}[section]
\newtheorem{teo}{{\it Teorema}}[section]
\newtheorem{ejemplo}{{\it Ejemplo}}[section]
\newtheorem{propiedad}{{\it Propiedad}}[section]
\newtheorem{lema}{{\it Lema}}[section]
\newtheorem{cor}{Corolario}
\newtheorem{ejer}{Ejercicio}[section]

\newlist{milista}{enumerate}{2}
\setlist[milista,1]{label=\arabic*)}
\setlist[milista,2]{label=\arabic{milistai}.\arabic*)}
\newlength{\depthofsumsign}
\setlength{\depthofsumsign}{\depthof{$\sum$}}
\newcommand{\nsum}[1][1.4]{% only for \displaystyle
    \mathop{%
        \raisebox
            {-#1\depthofsumsign+1\depthofsumsign}
            {\scalebox
                {#1}
                {$\displaystyle\sum$}%
            }
    }
}
\def\scaleint#1{\vcenter{\hbox{\scaleto[3ex]{\displaystyle\int}{#1}}}}
\def\bs{\mkern-12mu}


\title{Segunda solución linealmente independiente} \vspace{-3ex}
\author{M. en C. Gustavo Contreras Mayén}
\date{ }
\newcommand{\Cancel}[2][black]{{\color{#1}\cancel{\color{black}#2}}}
\begin{document}
\vspace{-4cm}
\maketitle
\fontsize{14}{14}\selectfont
\tableofcontents
\newpage

%Ref. Makarenko. 17. Integración de las ED mediante series
\section{Solución mediante series.}

Este método resulta muy común al aplicarlo a las ecuaciones diferenciales lineales, como se ha visto en el tema, seguiremos manejando ecuaciones de segundo orden.
\par
Sea la EDO2H lineal:
\begin{align}
\stilde{y} + p(x) \, \ptilde{y} + q(x) \, y = 0
\label{eq:ecuacion_17_01}
\end{align}

Haremos la suposición que los coeficientes $p(x)$ y $q(x)$ son funciones analíticas, es decir, se expresan en forma de series de potencias, dispuestas según las potencias enteras positivas de $x$, de modo que la ec. (\ref{eq:ecuacion_17_01}) se puede escribir de la forma:
\begin{align}
\stilde{y} + \big( a_{0} + a_{1} \, x + a_{2} \, x^{2} + \ldots \big) \, \ptilde{y} + (b_{0} + b_{1} \, x + b_{2} + x^{2} + \ldots) \, y = 0
\label{eq:ecuacion_17_02}
\end{align}

\section{Solución en una serie de potencias generalizada.}

Definimos la serie de la forma:
\begin{align}
x^{\rho} \sum_{n=0}^{\infty} c_{n} \, x^{n}, \hspace{1.5cm} c_{0} \neq 0
\label{eq:ecuacion_17_14}
\end{align}
donde $\rho$ es un número dado y la serie de potencias
\begin{align*}
\sum_{n=0}^{\infty} c_{n} \, x^{n}
\end{align*}
converge para $\abs{x} < R$, se le denomina \emph{serie de potencias generalizada.}
\par
Si $\rho$ es un número entero no negativo, la serie de potencias generalizada (\ref{eq:ecuacion_17_14}) se convierte en una serie de potencias ordinaria.
\begin{teo}
Si $x = 0$ es un punto singular regular de la ec. (\ref{eq:ecuacion_17_01}), cuyos coeficientes $p(x)$ y $q(x)$ admiten desarrollos:
\begin{align}
\begin{aligned}
p(x) &= \dfrac{\displaystyle \sum_{n=0}^{\infty} a_{n} \, x^{n}}{x} \\[1em]
q(x) &= \dfrac{\displaystyle \sum_{n=0}^{\infty} b_{n} \, x^{n}}{x^{2}}
\end{aligned}
\label{eq:ecuacion_17_15}
\end{align}
donde las series que figuran en los numeradores son convergentes en $\abs{x} < R$, y los coeficientes $a_{0}$, $b_{0}$, $c_{0}$ no son simultáneamente iguales a cero, entonces la ec. (\ref{eq:ecuacion_17_01}) posee al menos una solución en forma de serie de potencias generalizada:
\begin{align}
y(x) = x^{\rho} \sum_{n=0}^{\infty} C_{n} \, x^{n}, \hspace{1.5cm} C_{n} \neq 0
\label{eq:ecuacion_17_16}
\end{align}
que es convergente al menos en el mismo intervalo $\abs{x} < R$.
\end{teo}

Para hallar el exponente $\rho$ y los coeficientes $C_{n}$, es necesario expresar la serie (\ref{eq:ecuacion_17_16}) en la ec. (\ref{eq:ecuacion_17_01}), simplificar para $x^{\rho}$ e igualar a cero los coeficientes en distintas potencias de $x$ (método de coeficientes indeterminados).

\subsection{Obteniendo el exponente \texorpdfstring{$\rho$}{r}.}

Al considerar la solución propuesta:
\begin{align*}
y = \sum_{n=0}^{\infty} c_{n} \, x^{n+\rho}
\end{align*}
se requiere conocer la primera y  segunda derivada de $y$ con respecto a $x$:
\begin{align*}
\ptilde{y} &= \sum_{n=0}^{\infty} (n + \rho) \, c_{n} \, x^{n+\rho-1} \\[1em]
\stilde{y} &= \sum_{n=0}^{\infty} (n + \rho)(n + \rho - 1) \, c_{n} \, x^{n+\rho-2}
\end{align*}

que se sustituyen en la ec. (\ref{eq:ecuacion_17_01}) en conjunto con las definiciones de $p(x)$ y $q(x)$ dadas en la ec. (\ref{eq:ecuacion_17_15}):
\begin{align*}
\sum_{n=0}^{\infty} (n &+ \rho)(n + \rho + 1) \, c_{n} \, x^{n+\rho-2} + \left[ \dfrac{\displaystyle \sum_{n=0}^{\infty} a_{n} \, x^{n}}{x} \right] \, \left[ \sum_{n=0}^{\infty} (n + \rho) \, c_{n} \, x^{n+\rho-1} \right] + \\[1em]
&+ \left[ \dfrac{\displaystyle \sum_{n=0}^{\infty} b_{n} \, x^{n}}{x^{2}} \right] \, \sum_{n=0}^{\infty} c_{n} \, x^{n+\rho} = 0
\end{align*}

Simplificando la expresión:
\begin{align*}
\sum_{n=0}^{\infty} (n &+ \rho)(n + \rho + 1) \, c_{n} \, x^{n+\rho-2} + \sum_{n=0}^{\infty} a_{n} \, x^{n-1} \, \sum_{n=0}^{\infty} (n + \rho) \, c_{n} \, x^{n+\rho-1} + \\[1em]
&+ \sum_{n=0}^{\infty} b_{n} \, x^{n-2} \, \sum_{n=0}^{\infty} c_{n} \, x^{n+\rho} = 0
\end{align*}

Abreviando el producto de las sumas:
\begin{align*}
\sum_{n=0}^{\infty} \big[ (n &+ \rho)(n + \rho + 1) \, c_{n} \big] \, x^{n+\rho-2} + \sum_{n=0}^{\infty} \big[ (n + \rho) \, a_{n} \, c_{n} \big] \, x^{2n+\rho-2} + \\[1em]
&+ \sum_{n=0}^{\infty} \big[ b_{n} \, c_{n} \big] \, x^{2n+\rho-2} = 0
\end{align*}

Factorizando para el exponente mayor de $x$:
\begin{align*}
\sum_{n=0}^{\infty} \big[ (n &+ \rho)(n + \rho + 1) \, c_{n} \big] \, x^{n+\rho-2} + \\[1em]
&+ \sum_{n=0}^{\infty} \big[ (n + \rho) \, a_{n} \, c_{n} + b_{n} \, c_{n} \big] \, x^{2n+\rho-2} = 0
\end{align*}

Haciendo $n = 0$ obtenemos:
\begin{align*}
\big[ \rho (\rho - 1) c_{0} + \rho \, a_{0} \, c_{0} + b_{0} \, c_{0} \big] \, x^{\rho-2} = 0
\end{align*}

En este producto los coeficientes deben de anularse y como $c_{0} \neq 0$, se tiene que:
\begin{align}
\setlength{\fboxsep}{3\fboxsep}\boxed{
\rho (\rho - 1) + \rho \, a_{0} + b_{0} = 0 }
\label{eq:ecuacion_17_17}
\end{align}
a esta expresión se le denomina \emph{ecuación determinativa}.
\par
Donde:
\begin{align}
a_{0} = \lim_{x \to 0} x \, p(x) \hspace{1cm} \mbox{y} \hspace{1cm} b_{0} = \lim_{x \to 0} x^{2} \, q(x)
\label{eq:ecuacion_17_18}
\end{align}

\section{Tres casos.}

Supongamos que $\rho_{1} > \rho_{2}$ son las raíces de la ec. (\ref{eq:ecuacion_17_17}), se distinguen tres casos en la solución de la ec. (\ref{eq:ecuacion_17_01}):

\subsection{Caso 1.}\label{caso1}

Si la diferencia $\rho_{1} - \rho_{2}$ no es un número entero o cero, se pueden construir dos soluciones de la forma:
\begin{align*}
y_{1}(x) &= x^{\rho_{1}} \sum_{n=0}^{\infty} c_{n} \, x^{n} \hspace{1.5cm} c_{0} \neq 0 \\ 
y_{2}(x) &= x^{\rho_{2}} \sum_{n=0}^{\infty} a_{n} \, x^{n} \hspace{1.5cm} a_{0} \neq 0
\end{align*}

\subsection{Caso 2.}

Si la diferencia $\rho_{1} - \rho_{2}$ es un número entero positivo, por lo general, solo se puede construir una serie como solución de la ec. (\ref{eq:ecuacion_17_01}):
\begin{align}
y_{1}(x) = x^{\rho_{1}} \sum_{n=0}^{\infty} c_{n} \, x^{n}
\label{eq:ecuacion_17_19}
\end{align}

\subsection{Caso 3.}

Si la ec. (\ref{eq:ecuacion_17_17}) posee una raíz múltiple $\rho_{1} = \rho_{2}$, también se construye solo una serie como solución a la ec. \ref{eq:ecuacion_17_01}.

Queda claro que en el caso \ref{caso1}, las soluciones $y_{1}(x)$ e $y_{2}(x)$ son linealmente independientes. En el segundo y tercer caso, se ha obtenido una única solución como se indica en la ec. (\ref{eq:ecuacion_17_19}).
\par
Se puede demostrar que si la diferencia $\rho_{1} - \rho_{2}$ es un número entero positivo o cero, además de la solución (\ref{eq:ecuacion_17_19}), habrá una solución de la forma:
\begin{align}
y_{2}(x) = A \, y_{1}(x) \, \ln x + x^{\rho_{2}} \sum_{n=0}^{\infty} A_{n} \, x^{n}
\label{eq:ecuacion_17_20}
\end{align}

Vemos que $y_{2}(x)$ contiene un término complementario de la forma:
\begin{align*}
A \, y_{1} (x) \, \ln x
\end{align*}
\noindent
donde $y_{1}(x)$ es de la forma (\ref{eq:ecuacion_17_19}). Como punto importante hay que considerar que la constante $A$ en la ec. (\ref{eq:ecuacion_17_20}) sea nula, por lo que la segunda solución $y_{2}(x)$ resulta una expresión de la forma de una serie de potencias generalizada.

\end{document}