\documentclass[hidelinks,12pt]{article}
\usepackage[left=0.25cm,top=1cm,right=0.25cm,bottom=1cm]{geometry}
%\usepackage[landscape]{geometry}
\textwidth = 20cm
\hoffset = -1cm
\usepackage[utf8]{inputenc}
\usepackage[spanish,es-tabla]{babel}
\usepackage[autostyle,spanish=mexican]{csquotes}
\usepackage[tbtags]{amsmath}
\usepackage{nccmath}
\usepackage{amsthm}
\usepackage{amssymb}
\usepackage{mathrsfs}
\usepackage{graphicx}
\usepackage{subfig}
\usepackage{standalone}
\usepackage[outdir=./Imagenes/]{epstopdf}
\usepackage{siunitx}
\usepackage{physics}
\usepackage{color}
\usepackage{float}
\usepackage{hyperref}
\usepackage{multicol}
%\usepackage{milista}
\usepackage{anyfontsize}
\usepackage{anysize}
%\usepackage{enumerate}
\usepackage[shortlabels]{enumitem}
\usepackage{capt-of}
\usepackage{bm}
\usepackage{relsize}
\usepackage{placeins}
\usepackage{empheq}
\usepackage{cancel}
\usepackage{wrapfig}
\usepackage[flushleft]{threeparttable}
\usepackage{makecell}
\usepackage{fancyhdr}
\usepackage{tikz}
\usepackage{bigints}
\usepackage{scalerel}
\usepackage{pgfplots}
\usepackage{pdflscape}
\pgfplotsset{compat=1.16}
\spanishdecimal{.}
\renewcommand{\baselinestretch}{1.5} 
\renewcommand\labelenumii{\theenumi.{\arabic{enumii}})}
\newcommand{\ptilde}[1]{\ensuremath{{#1}^{\prime}}}
\newcommand{\stilde}[1]{\ensuremath{{#1}^{\prime \prime}}}
\newcommand{\ttilde}[1]{\ensuremath{{#1}^{\prime \prime \prime}}}
\newcommand{\ntilde}[2]{\ensuremath{{#1}^{(#2)}}}

\newtheorem{defi}{{\it Definición}}[section]
\newtheorem{teo}{{\it Teorema}}[section]
\newtheorem{ejemplo}{{\it Ejemplo}}[section]
\newtheorem{propiedad}{{\it Propiedad}}[section]
\newtheorem{lema}{{\it Lema}}[section]
\newtheorem{cor}{Corolario}
\newtheorem{ejer}{Ejercicio}[section]

\newlist{milista}{enumerate}{2}
\setlist[milista,1]{label=\arabic*)}
\setlist[milista,2]{label=\arabic{milistai}.\arabic*)}
\newlength{\depthofsumsign}
\setlength{\depthofsumsign}{\depthof{$\sum$}}
\newcommand{\nsum}[1][1.4]{% only for \displaystyle
    \mathop{%
        \raisebox
            {-#1\depthofsumsign+1\depthofsumsign}
            {\scalebox
                {#1}
                {$\displaystyle\sum$}%
            }
    }
}
\def\scaleint#1{\vcenter{\hbox{\scaleto[3ex]{\displaystyle\int}{#1}}}}
\def\bs{\mkern-12mu}


\title{Segunda solución linealmente independiente} \vspace{-3ex}
\author{M. en C. Gustavo Contreras Mayén}
\date{ }
\newcommand{\Cancel}[2][black]{{\color{#1}\cancel{\color{black}#2}}}
\begin{document}
\vspace{-4cm}
\maketitle
\fontsize{14}{14}\selectfont
\tableofcontents
\newpage

%Ref. Makarenko. 17. Integración de las ED mediante series
\section{Solución mediante series.}

Este método resulta muy común al aplicarlo a las ecuaciones diferenciales lineales, como se ha visto en el tema, seguiremos manejando ecuaciones de segundo orden.
\par
Sea la EDO2H lineal:
\begin{align}
\stilde{y} + p(x) \, \ptilde{y} + q(x) \, y = 0
\label{eq:ecuacion_17_01}
\end{align}

Haremos la suposición que los coeficientes $p(x)$ y $q(x)$ son funciones analíticas, es decir, se expresan en forma de series de potencias, dispuestas según las potencias enteras positivas de $x$, de modo que la ec. (\ref{eq:ecuacion_17_01}) se puede escribir de la forma:
\begin{align}
\stilde{y} + \big( a_{0} + a_{1} \, x + a_{2} \, x^{2} + \ldots \big) \, \ptilde{y} + (b_{0} + b_{1} \, x + b_{2} + x^{2} + \ldots) \, y = 0
\label{eq:ecuacion_17_02}
\end{align}

\section{Solución en una serie de potencias generalizada.}

Definimos la serie de la forma:
\begin{align}
x^{\rho} \sum_{n=0}^{\infty} c_{n} \, x^{n}, \hspace{1.5cm} c_{0} \neq 0
\label{eq:ecuacion_17_14}
\end{align}
donde $\rho$ es un número dado y la serie de potencias
\begin{align*}
\sum_{n=0}^{\infty} c_{n} \, x^{n}
\end{align*}
converge para $\abs{x} < R$, se le denomina \emph{serie de potencias generalizada.}
\par
Si $\rho$ es un número entero no negativo, la serie de potencias generalizada (\ref{eq:ecuacion_17_14}) se convierte en una serie de potencias ordinaria.
\begin{teo}
Si $x = 0$ es un punto singular regular de la ec. (\ref{eq:ecuacion_17_01}), cuyos coeficientes $p(x)$ y $q(x)$ admiten desarrollos:
\begin{align}
\begin{aligned}
p(x) &= \dfrac{\displaystyle \sum_{n=0}^{\infty} a_{n} \, x^{n}}{x} \\[1em]
q(x) &= \dfrac{\displaystyle \sum_{n=0}^{\infty} b_{n} \, x^{n}}{x^{2}}
\end{aligned}
\label{eq:ecuacion_17_15}
\end{align}
donde las series que figuran en los numeradores son convergentes en $\abs{x} < R$, y los coeficientes $a_{0}$, $b_{0}$, $c_{0}$ no son simultáneamente iguales a cero, entonces la ec. (\ref{eq:ecuacion_17_01}) posee al menos una solución en forma de serie de potencias generalizada:
\begin{align}
y(x) = x^{\rho} \sum_{n=0}^{\infty} C_{n} \, x^{n}, \hspace{1.5cm} C_{n} \neq 0
\label{eq:ecuacion_17_16}
\end{align}
que es convergente al menos en el mismo intervalo $\abs{x} < R$.
\end{teo}

Para hallar el exponente $\rho$ y los coeficientes $C_{n}$, es necesario expresar la serie (\ref{eq:ecuacion_17_16}) en la ec. (\ref{eq:ecuacion_17_01}), simplificar para $x^{\rho}$ e igualar a cero los coeficientes en distintas potencias de $x$ (método de coeficientes indeterminados).

\subsection{Obteniendo el exponente \texorpdfstring{$\rho$}{r}.}

Al considerar la solución propuesta:
\begin{align*}
y = \sum_{n=0}^{\infty} c_{n} \, x^{n+\rho}
\end{align*}
se requiere conocer la primera y  segunda derivada de $y$ con respecto a $x$:
\begin{align*}
\ptilde{y} &= \sum_{n=0}^{\infty} (n + \rho) \, c_{n} \, x^{n+\rho-1} \\[1em]
\stilde{y} &= \sum_{n=0}^{\infty} (n + \rho)(n + \rho - 1) \, c_{n} \, x^{n+\rho-2}
\end{align*}

que se sustituyen en la ec. (\ref{eq:ecuacion_17_01}) en conjunto con las definiciones de $p(x)$ y $q(x)$ dadas en la ec. (\ref{eq:ecuacion_17_15}):
\begin{align*}
\sum_{n=0}^{\infty} (n &+ \rho)(n + \rho + 1) \, c_{n} \, x^{n+\rho-2} + \left[ \dfrac{\displaystyle \sum_{n=0}^{\infty} a_{n} \, x^{n}}{x} \right] \, \left[ \sum_{n=0}^{\infty} (n + \rho) \, c_{n} \, x^{n+\rho-1} \right] + \\[1em]
&+ \left[ \dfrac{\displaystyle \sum_{n=0}^{\infty} b_{n} \, x^{n}}{x^{2}} \right] \, \sum_{n=0}^{\infty} c_{n} \, x^{n+\rho} = 0
\end{align*}

Simplificando la expresión:
\begin{align*}
\sum_{n=0}^{\infty} (n &+ \rho)(n + \rho + 1) \, c_{n} \, x^{n+\rho-2} + \sum_{n=0}^{\infty} a_{n} \, x^{n-1} \, \sum_{n=0}^{\infty} (n + \rho) \, c_{n} \, x^{n+\rho-1} + \\[1em]
&+ \sum_{n=0}^{\infty} b_{n} \, x^{n-2} \, \sum_{n=0}^{\infty} c_{n} \, x^{n+\rho} = 0
\end{align*}

Abreviando el producto de las sumas:
\begin{align*}
\sum_{n=0}^{\infty} \big[ (n &+ \rho)(n + \rho + 1) \, c_{n} \big] \, x^{n+\rho-2} + \sum_{n=0}^{\infty} \big[ (n + \rho) \, a_{n} \, c_{n} \big] \, x^{2n+\rho-2} + \\[1em]
&+ \sum_{n=0}^{\infty} \big[ b_{n} \, c_{n} \big] \, x^{2n+\rho-2} = 0
\end{align*}

Factorizando para el exponente mayor de $x$:
\begin{align*}
\sum_{n=0}^{\infty} \big[ (n &+ \rho)(n + \rho + 1) \, c_{n} \big] \, x^{n+\rho-2} + \\[1em]
&+ \sum_{n=0}^{\infty} \big[ (n + \rho) \, a_{n} \, c_{n} + b_{n} \, c_{n} \big] \, x^{2n+\rho-2} = 0
\end{align*}

Haciendo $n = 0$ obtenemos:
\begin{align*}
\big[ \rho (\rho - 1) c_{0} + \rho \, a_{0} \, c_{0} + b_{0} \, c_{0} \big] \, x^{\rho-2} = 0
\end{align*}

En este producto los coeficientes deben de anularse y como $c_{0} \neq 0$, se tiene que:
\begin{align}
\setlength{\fboxsep}{3\fboxsep}\boxed{
\rho (\rho - 1) + \rho \, a_{0} + b_{0} = 0 }
\label{eq:ecuacion_17_17}
\end{align}
a esta expresión se le denomina \emph{ecuación determinativa}.
\par
Donde:
\begin{align}
a_{0} = \lim_{x \to 0} x \, p(x) \hspace{1cm} \mbox{y} \hspace{1cm} b_{0} = \lim_{x \to 0} x^{2} \, q(x)
\label{eq:ecuacion_17_18}
\end{align}

\section{Tres casos.}

Supongamos que $\rho_{1} > \rho_{2}$ son las raíces de la ec. (\ref{eq:ecuacion_17_17}), se distinguen tres casos en la solución de la ec. (\ref{eq:ecuacion_17_01}):

\subsection{Caso 1.}\label{caso1}

Si la diferencia $\rho_{1} - \rho_{2}$ no es un número entero o cero, se pueden construir dos soluciones de la forma:
\begin{align*}
y_{1}(x) &= x^{\rho_{1}} \sum_{n=0}^{\infty} c_{n} \, x^{n} \hspace{1.5cm} c_{0} \neq 0 \\ 
y_{2}(x) &= x^{\rho_{2}} \sum_{n=0}^{\infty} a_{n} \, x^{n} \hspace{1.5cm} a_{0} \neq 0
\end{align*}

\subsection{Caso 2.} \label{caso2}

Si la diferencia $\rho_{1} - \rho_{2}$ es un número entero positivo, por lo general, solo se puede construir una serie como solución de la ec. (\ref{eq:ecuacion_17_01}):
\begin{align}
y_{1}(x) = x^{\rho_{1}} \sum_{n=0}^{\infty} c_{n} \, x^{n}
\label{eq:ecuacion_17_19}
\end{align}

Se puede demostrar que si la diferencia $\rho_{1} - \rho_{2}$ es un número entero positivo o cero, además de la solución (\ref{eq:ecuacion_17_19}), habrá una solución de la forma:
\begin{align}
y_{2}(x) = A \, y_{1}(x) \, \ln x + x^{\rho_{2}} \sum_{n=0}^{\infty} A_{n} \, x^{n}
\label{eq:ecuacion_17_20}
\end{align}

Vemos que $y_{2}(x)$ contiene un término complementario de la forma:
\begin{align*}
A \, y_{1} (x) \, \ln x
\end{align*}
\noindent
donde $y_{1}(x)$ es de la forma (\ref{eq:ecuacion_17_19}). Como punto importante hay que considerar que la constante $A$ en la ec. (\ref{eq:ecuacion_17_20}) sea nula, por lo que la segunda solución $y_{2}(x)$ resulta una expresión de la forma de una serie de potencias generalizada.


\subsection{Caso 3.}

Si la ec. (\ref{eq:ecuacion_17_17}) posee una raíz múltiple $\rho_{1} = \rho_{2}$, también se construye solo una serie como solución a la ec. \ref{eq:ecuacion_17_01}.
\begin{align*}
y_{1}(x) = x^{\rho_{1}} \sum_{n=0}^{\infty} c_{n} \, x^{n}
\end{align*}

También es posible demostrar que en este caso, se puede obtener una segunda solución de la forma:
\begin{align}
y_{2}(x) = y_{1}(x) \, \ln x + x^{\rho_{2}} \sum_{n=1}^{\infty} b_{n} \, x^{n+\rho_{1}}
\label{eq:ecuacion_22_Zill}
\end{align}

Comparando con la ec. (\ref{eq:ecuacion_17_20}) se tiene que $A = 1$.
\\[0.5em]
\noindent
Queda claro que en el caso \ref{caso1}, las soluciones $y_{1}(x)$ e $y_{2}(x)$ son linealmente independientes. En el segundo y tercer caso, se ha obtenido una única solución tomando la raíz $\rho_{1}$ de la ecuación de índices, como se indica en la ec. (\ref{eq:ecuacion_17_19}).
\par

%Ref. Zill (2015) Ejemplo 4. Cap. 6
\section{Ejemplo: \texorpdfstring{$\rho_{1} - \rho_{2} = N$}{r1-r2=N} entero positivo.}\label{ejercicio}

Resuelve mediante el método de Frobenius la siguiente ecuación diferencial de segundo orden lineal:
\begin{align*}
x \, \stilde{y} + y = 0
\end{align*}

Antes de hacer la sustitución y desarrollo en serie de potencias, estudiemos la ecuación determinativa:
\begin{align*}
\rho (\rho - 1) + \rho \, a_{0} + b_{0} = 0
\end{align*}

En donde:
\begin{align*}
a_{0} = \lim_{x \to 0} x \, p(x) \hspace{1cm} \mbox{y} \hspace{1cm} b_{0} = \lim_{x \to 0} x^{2} \, q(x)
\end{align*}

En este caso, se tiene que:
\begin{align*}
a_{0} &= \lim_{x \to 0} x \, p(x) = 0 \\[1em]
b_{0} &= \lim_{x \to 0} x^{2} \, q(x) = \lim_{x \to 0} \dfrac{x^{2}}{x} = \lim_{x \to 0} x = 0
\end{align*}

Por lo que $a_{0} = 0$ y $b_{0} = 0$, entonces la ecuación determinativa resulta:

\begin{align*}
\rho (\rho - 1) + \rho \, 0 + 0 &= 0 \\[1em]
\rho (\rho - 1) &= 0 
\end{align*}

donde la raíces son:
\begin{align*}
\rho_{1} = 1 \hspace{1.5cm} \rho_{2} = 0
\end{align*}

de tal manera que la diferencia $\rho_{1} - \rho_{2} = 1$, un entero positivo, por lo estamos en el caso (\ref{caso2}), es decir, tendremos solo una solución a la EDO2H.
\par
Haciendo el desarrollo como ya lo hemos manejado previamente, ocupando una solución inicial del tipo:
\begin{align*}
y(x) = \sum_{n=0}^{\infty} a_{n} \, x^{n+r}
\end{align*}

se llega a una solución de la forma\footnote{Como ejercicio moral realiza todo el procedimiento.}:
\begin{align}
\setlength{\fboxsep}{3\fboxsep}\boxed{
y(x) = \sum_{n=0}^{\infty} \dfrac{(-1)^{n} \, a_{0}}{(n + 1)! \, n!} \, x^{n+1}}
\label{eq:ecuacion_sol}
\end{align}

Al utilizar la segunda raíz $\rho_{2} = 0$, de la relación de recurrencia, se obtienen los mismos coeficientes de la solución anterior (\ref{eq:ecuacion_sol}), por lo que el método de Frobenius solo devuelve una solución.

\section{Forma de la segunda solución.}

Cuando la diferencia $\rho_{1} - \rho_{2}$ es un entero positivo (caso \ref{caso2}) se podría o no encontrar dos soluciones de la forma:
\begin{align*}
y(x) = \sum_{n=0}^{\infty} a_{n} \, x^{n+\rho}
\end{align*}

Esto es algo que no se sabe con anticipación, pero se determina luego de haber encontrado las raíces de la ecuación de índices o mediante la ecuación determinativa, y también haber examinado con cuidado la relación de recurrencia que definen los coeficientes $a_{n}$.
\par
Podríamos tener la oportunidad de encontrar dos soluciones que impliquen solo potencias de $x$, es decir:
\begin{align*}
y_{1} (x) &= \sum_{n=0}^{\infty} a_{n} \, x^{n+\rho_{1}} \\[1em]
y_{2} (x) &= \sum_{n=0}^{\infty} b_{n} \, x^{n+\rho_{2}}
\end{align*}

\subsection{Segunda solución.}

Una forma de obtener la segunda solución linealmente independiente con el término logarítmico es usar el hecho de que:
\begin{align}
y_{2} (x) = y_{1}(x) \bigintss \dfrac{\exp(\displaystyle -\int P(x') \dd{x'})}{\big[ y_{1}(x) \big]^{2}} \dd{x}
\label{eq:ecuacion_23_Zill}
\end{align}

también es solución de la EDO2H:
\begin{align*}
\stilde{y} + p(x) \, \ptilde{y} + q(x) \, y = 0
\end{align*}

siempre y cuando $y_{1}(x)$ sea una solución conocida.

\subsection{Completando el ejercicio.}

Del ejercicio (numeral \ref{ejercicio})
\begin{align*}
x \, \stilde{y} + y = 0
\end{align*}

encuentra la solución general.
\par
De la solución obtenida $y_{1}(x)$:
\begin{align*}
y_{1}(x) = x - \dfrac{1}{2} x^{2} + \dfrac{1}{12} x^{3} - \dfrac{1}{144} x^{4}  + \ldots
\end{align*}

se puede construir una segunda solución $y_{2}(x)$ ocupando la ec. (\ref{eq:ecuacion_23_Zill}), para ello habrá que elevar al cuadrado una serie, luego una división y la integración del cociente a mano. Es decir:
\begin{align*}
y_{2}(x) &= y_{1}(x) \bigintss \dfrac{\exp(\displaystyle -\int 0 \dd{x'})}{\big[ y_{1}(x) \big]^{2}} \dd{x} \\[1em]
&= y_{1} (x) \bigintss \dfrac{\dd{x}}{\bigg[ x - \dfrac{1}{2} x^{2} + \dfrac{1}{12} x^{3} - \dfrac{1}{144} x^{4}  + \ldots \bigg]^{2}} 
\end{align*}
\begin{enumerate}[label=\alph*)]
\item Elevando al cuadrado el denominador tenemos que:
\begin{align*}
= y_{1} (x) \bigintss \dfrac{\dd{x}}{\bigg[ x^{2} - x^{3} + \dfrac{5}{12} x^{4} - \dfrac{7}{72} x^{5}  + \ldots \bigg]} 
\end{align*}
\item Haciendo la división:
\begin{align*}
= y_{1} (x) \bigintss \bigg[ \dfrac{1}{x^{2}} + \dfrac{1}{x} + \dfrac{7}{12} + \dfrac{19}{72} x  + \ldots \bigg] \, \dd{x}
\end{align*}
\item Después de integrar término a término:
\begin{align*}
= y_{1} (x) \, \ln x + y_{1} \bigg[ - \dfrac{1}{x} + \dfrac{7}{12} x + \dfrac{19}{144} x^{2}  + \ldots \bigg]
\end{align*}
\item Multiplicando $y_{1}(x)$ con los términos del corchete, llegamos a la segunda solución:
\begin{align*}
y_{2} (x) = y_{1} (x) \, \ln x + \bigg[ - 1 + \dfrac{1}{2} x + \dfrac{1}{2} x^{2}  + \ldots \bigg]
\end{align*}
\end{enumerate}

Por lo que en el intervalo $(0, \infty)$, la solución general es:
\begin{align*}
y(x) = C_{1} \, y_{1} (x) + C_{2} \, y_{2}(x) \qed
\end{align*}

%Ref. Makarenko
\section{Ejemplo: Ecuación de Bessel.}

Consideremos la ecuación diferencial de Bessel\footnote{Nuevamente se ocupa una EDO2H lineal con un nombre en particular, la obtención de esta ecuación así como el estudio y análisis se trabaja en el tema de \emph{Funciones Especiales}, ya que se considera desde la geometría particular de un problema de la física, que conduce a una ecuación diferencial que deberá de resolverse mediante alguna de las técnicas que hemos visto, para luego revisar las soluciones linealmente independientes y con ello, generar una base que nos permita expresar la solución al problema con esa base, además de otras particulares características de las soluciones.}

\begin{align}
x^{2} \, \stilde{y} + x \, \ptilde{y} + (x^{2} - p^{2}) \, y = 0
\label{eq:ecuacion_17_E2_01}
\end{align}

donde $p$ es una constante dada, tal que $p > 0$.
\par
Para resolver esta ecuación, la escribimos de la forma:
\begin{align}
\stilde{y} + \dfrac{1}{x} \, \ptilde{y} + \dfrac{(x^{2} - p^{2})}{x^{2}} \, y = 0
\label{eq:ecuacion_17_E2_02}    
\end{align}

de donde reconocemos que:
\begin{align*}
p(x) &= \dfrac{1}{x} \\[1em]
q(x) &= \dfrac{(x^{2} - p^{2})}{x^{2}}
\end{align*}

de modo que:
\begin{align*}
a_{0} &= \lim_{x \to 0} x \, p(x) =  1 \\[1em]
b_{0} &= \lim_{x \to 0} x^{2} \, q(x) =  -p^{2}
\end{align*}

La ecuación determinativa para $\rho$ es:
\begin{align}
\rho (\rho - 1) + 1 \cdot \rho - p^{2} = 0 \hspace{1cm} \Rightarrow \hspace{1cm} \rho^{2} - p^{2} = 0
\label{eq:ecuacion_17_E2_03}
\end{align}

Las raíces de la ecuación (\ref{eq:ecuacion_17_E2_03}) son:
\begin{align}
\rho_{1} &= p \\[0.5em]
\rho_{2} &= -p
\label{eq:ecuacion_17_E2_04}
\end{align}

Notamos que $\rho_{1} - \rho_{2} = 0$. Buscamos la primera solución particular de la ecuación de Bessel (\ref{eq:ecuacion_17_E2_01}) en forma de una serie de potencias generalizada:
\begin{align}
y(x) = x^{p} \sum_{n=0}^{\infty} c_{n} \, x^{n}
\label{eq:ecuacion_17_E2_05}
\end{align}

El siguiente paso es reemplazar $y$, $\ptilde{y}$ e $\stilde{y}$ en la ec. (\ref{eq:ecuacion_17_E2_01}), en donde obtenemos lo siguiente:
\begin{align*}
x^{2} \sum_{n=0}^{\infty} &(n + p)(n + p - 1) \, c_{n} \, x^{n+p-2} + x \sum_{n=0}^{\infty} (n + p) \, c_{n} \, x^{n+p-1} + \\[1em]
&+ (x^{2} - p^{2}) \sum_{n=0}^{\infty} c_{n} \, x^{n+p} = 0
\end{align*}

Luego de realizar las correspondientes operaciones para transformar y simplificar para $x^{p}$, la expresión es:
\begin{align}
\sum_{n=0}^{\infty} \bigg[ (n + p)^{2} - p^{2} \bigg] c_{n} \, x^{n} + \sum_{n=0}^{\infty} c_{n} \, x^{n+2} = 0
\label{eq:ecuacion_17_E2_06}
\end{align}

De esta expresión, igualamos a cero los coeficientes en distintas potencias de $x$, teniendo que\footnote{Una manera práctica para la revisión de los coeficientes es hacer la tabla que se muestra, ya que genera una mejor legibilidad y seguimiento para obtener la relación de recurrencia.}:
\begin{equation}
\begin{array}{l | l |}
x^{0} & (p^{2} - p^{2}) \, c_{0} = 0 \\
x^{1} & \big[(1 + p)^{2} - p^{2} \big] \, c_{1} = 0 \\
x^{2} & \big[(2 + p)^{2} - p^{2} \big] \, c_{2} + c_{0} = 0 \\
x^{3} & \big[(3 + p)^{2} - p^{2} \big] \, c_{3} + c_{1} = 0 \\
x^{4} & \big[(4 + p)^{2} - p^{2} \big] \, c_{4} + c_{2} = 0 \\
\cdot & \ldots \hspace{1cm} \ldots \hspace{1cm} \ldots \\
x^{n} & \big[(n + p)^{2} - p^{2} \big] \, c_{n} + c_{n-2} = 0 \\
\cdot & \ldots \hspace{1cm} \ldots \hspace{1cm} \ldots \\
\end{array}
\label{eq:ecuacion_17_E2_07}
\end{equation}

La primera de las relaciones (\ref{eq:ecuacion_17_E2_07}) se cumple para cualquier valor del coeficiente $c_{0}$.
\par
De la segunda relación de la misma tabla (\ref{eq:ecuacion_17_E2_07}), se tiene que $c_{1} = 0$. De la tercera relación:
\begin{align*}
c_{2} = - \dfrac{c_{0}}{(2 + p)^{2} - p^{2}} = - \dfrac{c_{0}}{2^{2} (1 + p)}
\end{align*}

De la cuarta relación: $c_{3} = 0$. De la quinta:
\begin{align*}
c_{4} = - \dfrac{c_{2}}{(4 + p)^{2} - p^{2}} = - \dfrac{c_{0}}{2^{4} (1 + p)(2 + p) \cdot 1 \cdot 2}
\end{align*}

Es evidente que todos los coeficientes de subíndice impar son iguales a cero, es decir:
\begin{align}
c_{2n+1} = 0, \hspace{1.5cm} k = 0,1, 2, 3, \ldots
\label{eq:ecuacion_17_E2_08}
\end{align}

Los coeficientes de subíndice par son de la forma:
\begin{align}
c_{2n} = \dfrac{(-1)^{n} \, c_{0}}{2^{2n} \, (p + 1)(p + 2) \cdots (p + n) \cdot n!} \hspace{1.5cm} k = 1, 2, 3, \ldots
\label{eq:ecuacion_17_E2_09}
\end{align}

Para simplificar los siguientes cálculos, ocuparemos la función Gamma\footnote{Revisa el material de trabajo que ya se presentó sobre la función Gamma $\Gamma(x)$.}
\begin{align}
c_{0} = \dfrac{1}{2^{p} \, \Gamma(n) \, (p + 1)}
\label{eq:ecuacion_17_E2_10}
\end{align}
\newpage
Recordemos un par de propiedades de la función Gamma que nos serán de utilidad: Si $n$ es un número entero positivo, se tiene que:
\begin{enumerate}
\item $\Gamma (\nu + n + 1) = (\nu + 1)(\nu + 2) \ldots (\nu + n) \, \Gamma(\nu + 1)$
\item $\Gamma (n + 1) = n!$
\end{enumerate}

Ocupando la definición del coeficiente (\ref{eq:ecuacion_17_E2_10}) y las propiedades de la función Gamma, vamos a reescribir el coeficiente $c_{2n}$:
\begin{align*}
c_{2n} &= \dfrac{(-1)^{n}}{ 2^{2n} \, (p + 1)(p + 2) \cdots (p + n) \cdot n! 2^{p} \, \Gamma(p + 1)} = \\[1em]
&= \dfrac{(-1)^{n}}{ 2^{2n+p} \cdot n! \, \Gamma(p + n + 1)}
\end{align*}

La solución particular de la ecuación de Bessel, que se identificará como $J_{p}$ toma la forma:
\begin{align}
J_{p}(x) = \sum_{n=0}^{\infty} \dfrac{(-1)^{n}}{n! \, \Gamma(p + n + 1)} \left( \dfrac{x}{2} \right)^{2n+p}
\label{eq:ecuacion_E2_12}
\end{align}

Esta función se llama \emph{función de Bessel de primera clase de orden $p$}.
\par
La segunda solución particular de la ecuación de Bessel (\ref{eq:ecuacion_17_E2_01}), se buscará de la forma:
\begin{align}
y(x) = x^{-p} \sum_{n=0}^{\infty} a_{n} \, x^{n}
\label{eq:ecuacion_17_E2_13}
\end{align}

donde $-p$ es la segunda raíz de la ecuación determinativa. Queda claro que esta solución puede obtener de la solución que se indica en la ec. (\ref{eq:ecuacion_E2_12}), sustituyendo $p$ por $-p$, ya que en la ecuación (\ref{eq:ecuacion_17_01}) $p$ está elevado a una potencia par y no varía al sustituir $p$ por $-p$.
\par
Entonces tenemos que:
\begin{align}
J_{-p}(x) = \sum_{n=0}^{\infty} \dfrac{(-1)^{n}}{n! \, \Gamma(n + 1 - p)} \left( \dfrac{x}{2} \right)^{2n-p}
\label{eq:ecuacion_E2_14}
\end{align}    

A esta función se le llama \emph{función de Bessel de primera clase de orden $-p$}. Este tipo de soluciones las conoceremos más adelante como \textbf{funciones especiales}, en su estudio y análisis encontraremos un conjunto de propiedades y características importantes. La relevancia de estas funciones proviene del hecho de que la mayoría se obtienen a partir del planteamiento de un problema de la física.

\subsection{Cuando \texorpdfstring{$p$}{p} no es entero.}

Si $p$ no es un número entero, las soluciones $J_{p}(x)$ y $J_{-p}(x)$ son linealmente independientes, ya que sus desarrollos en series comienzan con potencias distintas de $x$, por lo que la combinación lineal:
\begin{align*}
\alpha_{1} \, J_{p}(x) + \alpha_{2} \, J_{-p}(x)
\end{align*}
puede ser igual a cero idénticamente solo cuando
\begin{align*}
\alpha_{1} = \alpha_{2} = 0
\end{align*}

\subsection{Cuando \texorpdfstring{$p$}{p} es entero.}

Si $p$ es un número entero, las funciones $J_{p}(x)$ y $J_{-p}(x)$ son linealmente dependientes, ya que\footnote{A esta propiedad de las funciones de Bessel se les conoce como paridad, ya que al introducir el signo $(-1)^{n}$, se determina si es positivo o negativo a partir del orden de la función de Bessel.}:
\begin{align}
J_{-n}(x) = (-1)^{n} \, J_{n} (x) \hspace{1.5cm} \mbox{$n$ es entero}
\label{eq:ecuacion_E2_15}
\end{align}

Por lo que, cuando $p$ es entero, en lugar de $J_{-p}(x)$ hay que buscar otra solución que sea linealmente independiente con $J_{p}(x)$. Para esto, introducimos una nueva función:
\begin{align}
Y_{p}(x) = \dfrac{J_{p}(x) \, \cos p \, \pi - J_{-p}(x)}{\sin p \, \pi}
\label{eq:ecuacion_17_E2_16}
\end{align}

suponiendo primero que $p$ no es entero.
\par
Es evidente que la función $Y_{p}(x)$ definida de este modo, es solución de la ec. (\ref{eq:ecuacion_17_E2_01}), ya que representa una combinación lineal de las soluciones particulares $J_{p}(x)$ y $J_{-p}(x)$.
\par
Tomando el límite en la ec. ()\ref{eq:ecuacion_17_E2_16}), cuando $p$ tiende a un \emph{número entero}, se obtiene la solución particular $Y_{p}(x)$ linealmente independiente con $J_{p}(x)$ y definida ya para valores enteros de $p$.
\par
La función $Y_{p}(x)$ definida anteriormente se llama \emph{función de Bessel de segunda clase de orden $p$}, también conocida como \emph{función de Neumann}\footnote{En algunos textos le llaman función de Weber.}. De esta manera se ha construido el \emph{sistema fundamental de soluciones} de la ecuación de Bessel (\ref{eq:ecuacion_17_E2_01}).
\par
La solución general a la ec. (\ref{eq:ecuacion_17_E2_01}) se puede expresar de la forma:
\begin{align}
y(x) = A \, J_{p}(x) + B \, Y_{p}(x)
\label{eq:ecuacion_17_E2_17}
\end{align}

donde $A$ y $B$ son constantes arbitrarias.
\par
No obstante, cuando $p$ no es entero, la solución general de la ecuación de Bessel se puede tomar de la forma:
\begin{align}
y(x) = \alpha_{1} \, J_{p}(x) + \alpha_{2} \, J_{-p}(x)
\label{eq:ecuacion_17_E2_18}
\end{align}

donde $A$ y $B$ son constantes arbitrarias.    
\end{document}  