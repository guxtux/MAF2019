\documentclass[hidelinks,12pt]{article}
\usepackage[left=0.25cm,top=1cm,right=0.25cm,bottom=1cm]{geometry}
%\usepackage[landscape]{geometry}
\textwidth = 20cm
\hoffset = -1cm
\usepackage[utf8]{inputenc}
\usepackage[spanish,es-tabla]{babel}
\usepackage[autostyle,spanish=mexican]{csquotes}
\usepackage[tbtags]{amsmath}
\usepackage{nccmath}
\usepackage{amsthm}
\usepackage{amssymb}
\usepackage{mathrsfs}
\usepackage{graphicx}
\usepackage{subfig}
\usepackage{standalone}
\usepackage[outdir=./Imagenes/]{epstopdf}
\usepackage{siunitx}
\usepackage{physics}
\usepackage{color}
\usepackage{float}
\usepackage{hyperref}
\usepackage{multicol}
%\usepackage{milista}
\usepackage{anyfontsize}
\usepackage{anysize}
%\usepackage{enumerate}
\usepackage[shortlabels]{enumitem}
\usepackage{capt-of}
\usepackage{bm}
\usepackage{relsize}
\usepackage{placeins}
\usepackage{empheq}
\usepackage{cancel}
\usepackage{wrapfig}
\usepackage[flushleft]{threeparttable}
\usepackage{makecell}
\usepackage{fancyhdr}
\usepackage{tikz}
\usepackage{bigints}
\usepackage{scalerel}
\usepackage{pgfplots}
\usepackage{pdflscape}
\pgfplotsset{compat=1.16}
\spanishdecimal{.}
\renewcommand{\baselinestretch}{1.5} 
\renewcommand\labelenumii{\theenumi.{\arabic{enumii}})}
\newcommand{\ptilde}[1]{\ensuremath{{#1}^{\prime}}}
\newcommand{\stilde}[1]{\ensuremath{{#1}^{\prime \prime}}}
\newcommand{\ttilde}[1]{\ensuremath{{#1}^{\prime \prime \prime}}}
\newcommand{\ntilde}[2]{\ensuremath{{#1}^{(#2)}}}

\newtheorem{defi}{{\it Definición}}[section]
\newtheorem{teo}{{\it Teorema}}[section]
\newtheorem{ejemplo}{{\it Ejemplo}}[section]
\newtheorem{propiedad}{{\it Propiedad}}[section]
\newtheorem{lema}{{\it Lema}}[section]
\newtheorem{cor}{Corolario}
\newtheorem{ejer}{Ejercicio}[section]

\newlist{milista}{enumerate}{2}
\setlist[milista,1]{label=\arabic*)}
\setlist[milista,2]{label=\arabic{milistai}.\arabic*)}
\newlength{\depthofsumsign}
\setlength{\depthofsumsign}{\depthof{$\sum$}}
\newcommand{\nsum}[1][1.4]{% only for \displaystyle
    \mathop{%
        \raisebox
            {-#1\depthofsumsign+1\depthofsumsign}
            {\scalebox
                {#1}
                {$\displaystyle\sum$}%
            }
    }
}
\def\scaleint#1{\vcenter{\hbox{\scaleto[3ex]{\displaystyle\int}{#1}}}}
\def\bs{\mkern-12mu}


\usepackage{apacite}
\title{Examen - Tarea 2 \\[0.3em]  \large{Matemáticas Avanzadas de la Física}\vspace{-3ex}}
\author{M. en C. Gustavo Contreras Mayén}
\date{ }
\begin{document}
\vspace{-4cm}
\maketitle
\fontsize{14}{14}\selectfont

\textbf{Indicaciones: } Deberás de resolver cada ejercicio de la manera más completa, ordenada y clara posible, anotando cada paso así como las operaciones involucradas. El puntaje de cada ejercicio es de \textbf{1 punto}.
\par
Este examen requiere que presentes todo el desarrollo de acuerdo como se trabajó con separación de variables, con el método de Frobenius: identificar puntos singulares en una EDO, se requiere que presentes la ecuación de índices, así como la relación de recurrencia y las dos soluciones linealmente independientes.

\begin{enumerate}
\item Determina el tipo de las siguientes EDP:
\begin{enumerate}
\item $u_{xx} + 4 \, u_{xy} + u_{yy} +u_{x} + u_{y} + 2 \, u - x^{2} \, y = 0$
\item $y^{m+1} \, u_{xx} + u_{yy} - u_{x} = 0$ donde $m$ es un entero no negativo.
\item $x \, u_{xx} + y \, u_{yy} - u = 0$
\end{enumerate}
\item Demuestra que la ecuación de Helmholtz:
\begin{align*}
\laplacian{\psi} + k^{2} \, \psi = 0
\end{align*}
es separable en el sistema coordenado cilíndrico parabólico $(\mu, \nu, z)$.
%Ref. Pinchover 5.3
\item Usando el \textbf{método de separación de variables (MSP)}:\begin{enumerate}
\item Determina una solución al problema de una cuerda vibrando con los extremos fijos:
\begin{table}[H]
\centering
\begin{tabular}{ l l}
$u_{tt} - c^{2} \, u_{xx} = 0$ & $0 < x < L, t > 0$, \\
$u(0,t) = u(L, t) = 0$ & $t \geq 0,$ \\
$u(x, 0) = f(x)$ & $0 \leq x \leq L$ \\
$u_{t}(x, 0) = g(x)$ & $0 \leq x \leq L$
\end{tabular}
\end{table}
\item Demuestra que la solución obtenida representa la superposición de una onda desplazándose hacia adelante y hacia atrás.
\end{enumerate}
%Ref. Pinchover 5.5b
\item Resuelve con el MSP la ecuación de calor $u_{t} = 12 \, u_{xx}$ en $0 < x < \pi, t > 0$ sujeto a las siguientes condiciones de frontera e iniciales:
\begin{table}[H]
\centering
\begin{tabular}{l l}
$u_{x} (0, t) = u_{x} (\pi, t) = 0$ & $t \geq 0$, \\
$u(x, 0) = 1 + \sin^{3} x$ & $0 \leq x \leq \pi$
\end{tabular}
\end{table}
%Ref. Pinchover 5.6
\item Con el MSP: 
\begin{enumerate}
\item Calcula una solución al siguiente problema periódico con la ecuación de calor:
\begin{table}[H]
\centering
\begin{tabular}{ l l}
$u_{t} - k \, u_{xx} = 0$ & $0 < x < 2 \, \pi,  t > 0$, \\
$u(0, t) = u(2 \, \pi, t) = 0, \quad u_{x} (0, t) = u_{x}(2 \, \pi, t)$ & $t \geq 0$, \\
$u(x, 0) = f(x)$ & $0 \leq x \leq 2 \, \pi$
\end{tabular}
\end{table}
donde $f$ es una función periódica \emph{suave} (bien portada, continua, derivable). Este sistema describe la evolución del calor en un alambre circular aislado de longitud $2 \, \pi$.
\item Encuentra el $\displaystyle{\lim_{t \to \infty}} \, u(x, t)$ para todo $0 < x < 2 \, \pi$, ¿qué interpretación física se tiene de este resultado?
\end{enumerate}
\item Demuestra que las siguientes ecuaciones diferenciales tienen singularidades en los puntos que se indican:
\begin{table}[H]
\centering
\begin{tabular}{l c c}
 & S. regular & S. irregular \\
Hipergeométrica & & \\
$x(x-1) \stilde{y} + [(1 + a + b) x - c] \ptilde{y} + a b y = 0$ & $0, 1$ & $--$ \\
Chevyshev & & \\
$(1 - x^{2}) \stilde{y} - x \ptilde{y} + n^{2} y = 0$ & $- 1, 1$ & $--$ \\
Laguerre & & \\
$x \stilde{y} + (1 - x) \ptilde{y} + a y = 0$ & $0$ & $\infty$
\end{tabular}
\end{table}
%Ref. Arfken 9.5.8
\item Resuelve la ecuación diferencial de Laguerre mediante una solución en series:
\begin{align*}
x \, \stilde{L}_{n} (x) + (1 - x) \, \ptilde{L}_{n} (x) + n \, L_{n} (x) = 0
\end{align*}
Elige el parámetro $n$ tal que se trunque la serie y la solución se exprese como un polinomio.
\end{enumerate}
Identifica el(los) punto(s) singular(es) y en caso de que sea conducente, con el método de Frobenius resuelve las siguientes ecuaciones diferenciales.
\begin{enumerate}[resume]
%Ref. Duffy A.2.1
\item $(2 x^{2} + x^{3}) \stilde{y} + (x + 3 x^{2}) \ptilde{y} - (1 - 4 x) y = 0$
%Ref. Duffy Pag. 477 8)
%\item $2 x^{2} \stilde{y} - 3 (x + x^{2}) \ptilde{y} + (2 + 3 x) y = 0$
%Ref. Duffy Pag. 480 4
\item $x^{2} \stilde{y} - x (1 + x) \ptilde{y} + y = 0$
%REf. Duffy Pag. 480 8)
\item $x^{2} \stilde{y} + x (2 x - 1) \ptilde{y} + x(x - 1) y = 0$
\end{enumerate}

\end{document}