\documentclass[12pt]{article}
\usepackage[letterpaper]{geometry}
%\textwidth = 345.0pt
%\hoffset = -3cm
\usepackage[utf8]{inputenc}
\usepackage[spanish,es-tabla]{babel}
\usepackage[autostyle,spanish=mexican]{csquotes}
\usepackage{amsmath}
\usepackage{standalone}
\usepackage{tikz}   
\usepackage{tikz-3dplot}
\usepackage{nccmath}
\usepackage{amsthm}
\usepackage{amssymb}
\usepackage{graphicx}
\usepackage{comment}
\usepackage{siunitx}
\usepackage{physics}
\usepackage{color}
\usepackage{float}
\usepackage{multicol}
%\usepackage{milista}
\usepackage{enumitem}
\usepackage{anyfontsize}
\usepackage{anysize}
\marginsize{1cm}{1cm}{2cm}{2cm}
\usepackage{enumitem}
\usepackage{capt-of}
\usepackage{bm}
\usepackage{relsize}
\newlist{milista}{enumerate}{2}
\setlist[milista,1]{label=\arabic*)}
\setlist[milista,2]{label=\arabic{milistai}.\arabic*)}
\spanishdecimal{.}
\renewcommand{\baselinestretch}{1.5}
\author{ }
\usepackage{etoolbox}
\AtBeginEnvironment{tikzpicture}{\shorthandoff{>}\shorthandoff{<}}{}{}
\title{Problemas para  la Tarea Examen del Tema 3 \\ \large{Matemáticas Avanzadas de la Física}\vspace{-8ex}}
\date{ }
\begin{document}
\vspace{-4cm}
\renewcommand\labelenumii{\theenumi.{\arabic{enumii}}}
\maketitle
\fontsize{14}{14}\selectfont
\begin{milista}
\item Determina las condiciones de ortogonalidad y completez de las siguientes bases con funciones:
\begin{milista}
\item $\left\{ \varphi_{n} (\phi) \right\} = \left\{ \sqrt{\dfrac{2}{\pi}} \, \cos n \,\phi, \dfrac{1}{\sqrt{\pi}} \right\} \hspace{1.5cm} 0 \leq \phi \leq \pi$
\item $\left\{ \varphi_{n} (\phi) \right\} = \left\{ \dfrac{1}{\sqrt{\pi}} \, \sin n \, \pi,  \dfrac{1}{\sqrt{\pi}} \, \cos n \, \pi , \dfrac{1}{\sqrt{2 \, \pi}} \right\} \hspace{1.5cm} c \leq \phi \leq c + 2 \, \pi$
\item $\left\{ \varphi_{n} (\phi) \right\} = \left\{ \dfrac{1}{\sqrt{2 \, \pi}} \, \exp(2 \, i \, n \, \phi)  \right\} \hspace{1.5cm} 0 \leq \phi \leq \pi$
\end{milista}
\item Demuestra que la EDO2H de Laguerre
\begin{align*}
x \, y^{\prime \prime}(x) + (1- x) \, y^{\prime} + n \, y (x) = 0
\end{align*}
puede expresarse en forma de una ecuación autoadjunta al multiplicarla por $e^{-x}$ y que $w(x) = e^{-x}$ es la función de peso.
\item Demuestra que la EDO2H de Hermite
\begin{align*}
H_{n}^{\prime \prime} (x) + 2 \, x \, H_{n}^{\prime} (x) + 2 \, n \, H_{n} (x) = 0
\end{align*}
puede expresarse en forma de una ecuación autoadjunta al multiplicarla por $e^{-x^{2}}$ y que entonces $w(x) = e^{-x^{2}}$ es la apropiada función de peso.
\item Cuentas con los siguientes elementos:
\begin{milista}
\item Un conjunto de funciones $\left\{ u_{n} (x) \right\} = \left\{ x^{n} \right\}, \mbox{ con } n = 1, 2, \ldots$
\item El intervalo $(0, \infty)$
\item Una función de peso $w(x) = x \, e^{-x}$
\end{milista}
Con el método de Gram-Schmidt construye las primeras \textbf{cinco funciones ortonormales} del conjunto $u_{n}(x)$, con ese intervalo dado y función de peso dada.
\item Construye un conjunto ortogonal en el intervalo $0 \leq x < \infty$, usando $\left\{ u_{n}(x) \right\} = \left\{ \exp(-n \, x) \right\}, \mbox{ con } n = 1, 2, 3$. Considera la función de peso $w(x) = 1$. Esas funciones son soluciones de
\begin{align*}
u_{n}^{\prime \prime} - n^{2} \, u_{n} = 0
\end{align*}
que no es de la forma Sturm-Liouville (autoadjunta). ¿Por qué la teoría de Sturm-Liuoville no garantiza la ortogonalidad de esas funciones?
\item Considera un estado el cual está dado en términos de tres vectores ortonormales $\ket{\phi_{1}}, \ket{\phi_{2}}, \ket{\phi_{3}}$ de la siguiente manera
\begin{align*}
\ket{\psi} = \dfrac{1}{\sqrt{15}} \, \ket{\phi_{1}} + \dfrac{1}{\sqrt{3}} \, \ket{\phi_{2}} + \dfrac{1}{\sqrt{5}} \, \ket{\phi_{5}}
\end{align*}
donde los $\ket{\phi_{n}}$ son los estados propios de un operador 
\begin{align*}
\hat{B} \, \ket{\phi_{n}} = (3 \, n^{2} - 1 \, \ket{\phi_{n}} \hspace{1cm} n = 1, 2, 3
\end{align*}
\begin{milista}
\item Encuentra la norma del estado $\ket{\psi}$.
\item Encuentra el valor esperado de $\hat{B}$ para el estado $\ket{\psi}$.
\item Encuentra el valor esperado de $\hat{B}^{2}$ para el estado $\ket{\psi}$.
\end{milista}
\item En las siguientes expresiones, $\hat{A}$ es un operador, especifica la naturaleza de cada expresión, es decir, ya sea operador, bra, o ket; luego encuentra su hermitiano conjugado
\begin{milista}
\item $\bra{\phi} \, \hat{A} \, \ket{\psi} \bra{\psi}$
\item $\hat{A} \, \ket{\psi} \bra{\phi}$
\item $\bra{\phi} \, \hat{A} \, \ket{\psi} \, \ket{\psi} \bra{\phi} \, \hat{A}$
\item $\bra{\psi} \, \hat{A} \, \ket{\phi} \ket{\phi} + i \, \hat{A} \, \ket{\psi}$
\item $\left( \ket{\phi} \bra{\phi} \, \hat{A} \right) - i \left( \hat{A} \, \ket{\psi} \, \bra{\psi} \right)$
\end{milista}
\end{milista}
\end{document}