\documentclass[hidelinks,12pt]{article}
\usepackage[left=0.25cm,top=1cm,right=0.25cm,bottom=1cm]{geometry}
%\usepackage[landscape]{geometry}
\textwidth = 20cm
\hoffset = -1cm
\usepackage[utf8]{inputenc}
\usepackage[spanish,es-tabla]{babel}
\usepackage[autostyle,spanish=mexican]{csquotes}
\usepackage[tbtags]{amsmath}
\usepackage{nccmath}
\usepackage{amsthm}
\usepackage{amssymb}
\usepackage{mathrsfs}
\usepackage{graphicx}
\usepackage{subfig}
\usepackage{standalone}
\usepackage[outdir=./Imagenes/]{epstopdf}
\usepackage{siunitx}
\usepackage{physics}
\usepackage{color}
\usepackage{float}
\usepackage{hyperref}
\usepackage{multicol}
%\usepackage{milista}
\usepackage{anyfontsize}
\usepackage{anysize}
%\usepackage{enumerate}
\usepackage[shortlabels]{enumitem}
\usepackage{capt-of}
\usepackage{bm}
\usepackage{relsize}
\usepackage{placeins}
\usepackage{empheq}
\usepackage{cancel}
\usepackage{wrapfig}
\usepackage[flushleft]{threeparttable}
\usepackage{makecell}
\usepackage{fancyhdr}
\usepackage{tikz}
\usepackage{bigints}
\usepackage{scalerel}
\usepackage{pgfplots}
\usepackage{pdflscape}
\pgfplotsset{compat=1.16}
\spanishdecimal{.}
\renewcommand{\baselinestretch}{1.5} 
\renewcommand\labelenumii{\theenumi.{\arabic{enumii}})}
\newcommand{\ptilde}[1]{\ensuremath{{#1}^{\prime}}}
\newcommand{\stilde}[1]{\ensuremath{{#1}^{\prime \prime}}}
\newcommand{\ttilde}[1]{\ensuremath{{#1}^{\prime \prime \prime}}}
\newcommand{\ntilde}[2]{\ensuremath{{#1}^{(#2)}}}

\newtheorem{defi}{{\it Definición}}[section]
\newtheorem{teo}{{\it Teorema}}[section]
\newtheorem{ejemplo}{{\it Ejemplo}}[section]
\newtheorem{propiedad}{{\it Propiedad}}[section]
\newtheorem{lema}{{\it Lema}}[section]
\newtheorem{cor}{Corolario}
\newtheorem{ejer}{Ejercicio}[section]

\newlist{milista}{enumerate}{2}
\setlist[milista,1]{label=\arabic*)}
\setlist[milista,2]{label=\arabic{milistai}.\arabic*)}
\newlength{\depthofsumsign}
\setlength{\depthofsumsign}{\depthof{$\sum$}}
\newcommand{\nsum}[1][1.4]{% only for \displaystyle
    \mathop{%
        \raisebox
            {-#1\depthofsumsign+1\depthofsumsign}
            {\scalebox
                {#1}
                {$\displaystyle\sum$}%
            }
    }
}
\def\scaleint#1{\vcenter{\hbox{\scaleto[3ex]{\displaystyle\int}{#1}}}}
\def\bs{\mkern-12mu}


\geometry{top=1.25cm, bottom=1.5cm, left=1.25cm, right=0.8cm}
%\usepackage{showframe}
\title{Examen Tarea Tema 3 \\ \large {Matemáticas Avanzadas de la Física}  \vspace{-3ex}}
\author{M. en C. Gustavo Contreras Mayén}
\date{ }
\begin{document}∏
\vspace{-4cm}
\maketitle
\fontsize{14}{14}\selectfont
\textbf{Indicaciones: } Deberás de resolver cada ejercicio de la manera más completa, ordenada y clara posible, anotando cada paso así como las operaciones involucradas. El puntaje de cada ejercicio es de \textbf{2.5 puntos}.
\par
La fecha límite para enviar este examen-tarea es el \textbf{día 30 de noviembre de 2020 a las 12 del día}, ya sea mediante la plataforma Moodle, subiendo un archivo pdf que no exceda de 10 MB, o compartiendo el archivo en una carpeta en Drive de Gmail, recuerda que puedes digitalizar las hojas que hayas ocupado o tomar una fotografía clara y que abarque toda la hoja.
\section{Tema 3. Bases ortogonales y completez.}
\begin{enumerate}
% \item Con el método de Gram-Schmidt construye un conjunto de polinomios ortogonales $P_{n}^{*}(x)$ (con la función de peso unitaria) en el rango $[0, 1]$ del conjunto $[1, x]$. Normaliza tal que $P_{n}^{*}(1) = 1$.
% \par
% Considera que el ${}^{"*"}$ es una notación para indicar lo \enquote{modificado} del intervalo, es decir: $[0, 1]$, en vez de $[-1, 1]$, \textbf{NO} se debe de interpretar como el conjugado complejo.
\item Dadas las expresiones para las condiciones de:
\par
Ortonormalidad:
\begin{align*}
\int_{a}^{b} \varphi_{n}^{*} (x) \, \varphi_{m} (x) \omega(x) \dd{x} = \delta_{nm}
\end{align*}
y Completez:
\begin{align*}
\sum_{n} \varphi_{n}^{*} (x) \, \varphi_{n} (\ptilde{x})\omega(x) = \delta(x - \ptilde{x})
\end{align*}
Con $n = 1, 2, 3, \ldots$ y la función de peso unitaria: determina las condiciones de ortonormalidad y completez de la base:
\begin{align*}
\left\{ \varphi_{n}(x) \right\} = \left\{ \sqrt{\dfrac{2}{L}} \, \sin  \left( \dfrac{n \, \pi \, x}{L} \right) \right\}
\end{align*}
\item Usando el método de ortogonalización de Gram-Schmidt genera los primeros tres polinomios de Chebychev de tipo II, a partir de:
\begin{align*}
u_{n}(x) = x^{n}, \hspace{0.5cm} n = 0, 1, 2, \ldots, \hspace{0.5cm} -1 \leq x \leq 1, \hspace{0.5cm} \omega(x) = (1 - x^{2})^{+1/2}
\end{align*}
Considera la normalización para este conjunto de polinomios:
\begin{align*}
\int_{-1}^{1} U_{m}(x) \, U_{n} (x) \, \omega(x) \dd{x} = \delta_{mn} \, \dfrac{\pi}{2}
\end{align*}
Considera el resultado de la siguiente integral:
\begin{align*}
\int_{-1}^{1} (1 - x^{2})^{1/2} \, x^{2n} \dd{x} = \begin{cases}
\dfrac{\pi}{2} \cp \dfrac{1 \cdot 3 \cdot 5 \cdots (2 \, n - 1)}{4 \cdot 6 \cdot 8 \cdots (2 \, n + 2)} & n = 1, 2, 3, \ldots \\[0.5em]
= \dfrac{\pi}{2} & n = 0
\end{cases} 
\end{align*}
El resultado que debes de obtener es: $U_{0}(x) = 1$, $U_{1}(x) = 2 \, x$ y $U_{2}(x) = 4 \, x^{2} - 1$,
\item Demuestra que se puede escribir:
\begin{align*}
\delta(x - \xi) = \dfrac{2}{L} \sum_{n=1}^{\infty} \sin \left( \dfrac{n \, \pi \, \xi}{L} \right) \, \sin \left( \dfrac{n \, \pi \, x}{L} \right), \hspace{1cm} 0 < \xi < L
\end{align*}
\item ¿Los siguientes operadores son Hermitíanos o no? Justifica tu respuesta:
\begin{enumerate}
\item $(\hat{A} + \hat{A}^{\dagger})$
\item $i \, (\hat{A} + \hat{A}^{\dagger})$
\item $i \, (\hat{A} - \hat{A}^{\dagger})$
\end{enumerate}
\item \textbf{Bonus: 1 punto} Considera los dos siguientes kets:
\begin{align*}
\ket{\psi} = \mqty(5 \, i \\ 2 \\ -i) \hspace{1.5cm} \ket{\phi} = \mqty(3  \\ 8 \, i \\ - 9 \, i)
\end{align*}
\begin{enumerate}
\item Calcula $\ket{\psi}^{*}$ y $\bra{\psi}$
\item ¿El ket $\ket{\psi}$ está normalizado? En caso de que no, normaliza el ket.
\item ¿Son $\ket{\psi}$ y $\ket{\phi}$ ortogonales?
\end{enumerate}
\end{enumerate}

\end{document}