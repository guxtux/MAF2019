\documentclass[hidelinks,12pt]{article}
\usepackage[left=0.25cm,top=1cm,right=0.25cm,bottom=1cm]{geometry}
%\usepackage[landscape]{geometry}
\textwidth = 20cm
\hoffset = -1cm
\usepackage[utf8]{inputenc}
\usepackage[spanish,es-tabla]{babel}
\usepackage[autostyle,spanish=mexican]{csquotes}
\usepackage[tbtags]{amsmath}
\usepackage{nccmath}
\usepackage{amsthm}
\usepackage{amssymb}
\usepackage{mathrsfs}
\usepackage{graphicx}
\usepackage{subfig}
\usepackage{standalone}
\usepackage[outdir=./Imagenes/]{epstopdf}
\usepackage{siunitx}
\usepackage{physics}
\usepackage{color}
\usepackage{float}
\usepackage{hyperref}
\usepackage{multicol}
%\usepackage{milista}
\usepackage{anyfontsize}
\usepackage{anysize}
%\usepackage{enumerate}
\usepackage[shortlabels]{enumitem}
\usepackage{capt-of}
\usepackage{bm}
\usepackage{relsize}
\usepackage{placeins}
\usepackage{empheq}
\usepackage{cancel}
\usepackage{wrapfig}
\usepackage[flushleft]{threeparttable}
\usepackage{makecell}
\usepackage{fancyhdr}
\usepackage{tikz}
\usepackage{bigints}
\usepackage{scalerel}
\usepackage{pgfplots}
\usepackage{pdflscape}
\pgfplotsset{compat=1.16}
\spanishdecimal{.}
\renewcommand{\baselinestretch}{1.5} 
\renewcommand\labelenumii{\theenumi.{\arabic{enumii}})}
\newcommand{\ptilde}[1]{\ensuremath{{#1}^{\prime}}}
\newcommand{\stilde}[1]{\ensuremath{{#1}^{\prime \prime}}}
\newcommand{\ttilde}[1]{\ensuremath{{#1}^{\prime \prime \prime}}}
\newcommand{\ntilde}[2]{\ensuremath{{#1}^{(#2)}}}

\newtheorem{defi}{{\it Definición}}[section]
\newtheorem{teo}{{\it Teorema}}[section]
\newtheorem{ejemplo}{{\it Ejemplo}}[section]
\newtheorem{propiedad}{{\it Propiedad}}[section]
\newtheorem{lema}{{\it Lema}}[section]
\newtheorem{cor}{Corolario}
\newtheorem{ejer}{Ejercicio}[section]

\newlist{milista}{enumerate}{2}
\setlist[milista,1]{label=\arabic*)}
\setlist[milista,2]{label=\arabic{milistai}.\arabic*)}
\newlength{\depthofsumsign}
\setlength{\depthofsumsign}{\depthof{$\sum$}}
\newcommand{\nsum}[1][1.4]{% only for \displaystyle
    \mathop{%
        \raisebox
            {-#1\depthofsumsign+1\depthofsumsign}
            {\scalebox
                {#1}
                {$\displaystyle\sum$}%
            }
    }
}
\def\scaleint#1{\vcenter{\hbox{\scaleto[3ex]{\displaystyle\int}{#1}}}}
\def\bs{\mkern-12mu}


\usepackage{apacite}
\title{Examen - Tarea 3 \\[0.3em]  \large{Matemáticas Avanzadas de la Física}\vspace{-3ex}}
\author{M. en C. Gustavo Contreras Mayén}
\date{ }
\begin{document}
\vspace{-4cm}
\maketitle
\fontsize{14}{14}\selectfont

\textbf{Indicaciones: } Deberás de resolver cada ejercicio de la manera más completa, ordenada y clara posible, anotando cada paso así como las operaciones involucradas. El puntaje de cada ejercicio es de \textbf{1 punto}, excepto en donde se indica.

\begin{enumerate}
%Ref. Haberman (2014) Exercises 3.2
\item Considera la siguiente ED:
\begin{align*}
\rho \, \pdv[2]{u}{t} = T_{0} \, \pdv[2]{u}{x} + \alpha \, u + \beta \, \pdv{u}{t}
\end{align*}
\begin{enumerate}
\item (\textbf{0.5 puntos.)} Intrepreta físicamente el fenómeno que modela la ED.
\item (\textbf{0.5 puntos.)} ¿Qué signos deben de tener $\alpha$ y $\beta$ para que tenga un sentido físico?
\item (\textbf{0.5 puntos.)} Permitiendo que $\rho, \alpha$ y $\beta$ sean funciones de $x$. Demuestra que la técnica de separación de variables funciona solo si $\beta = c \, \rho$, donde $c$ es una constante.
\item (\textbf{0.5 puntos.)} Si $\beta = c \, \rho$, demuestra que la ecuación espacial es de tipo Sturm-Liouville. Resuelve la ecuación temporal.
\end{enumerate}
%Ref. Notas SL Problem 4
\item Calcula el adjunto del operador y su dominio para:
\begin{align*}
L \, u = \sderivada{u} +  4 \, \pderivada{u} - 3 \, u
\end{align*}
donde $u$ satisface las siguientes condiciones:
\begin{align*}
\pderivada{u}(0) + 4 \, u(0) &= 0 \\[0.5em]
\pderivada{u}(1) + 4 \, u(1) &= 0
\end{align*}
%Ref. Datos Arfken Tabla 10.3
\item Con el método de ortogonalización de Gram-Schmidt construye los primeros tres polinomios asociados de Legendre, ocupando para ello la siguiente información:
\begin{enumerate}[label=\alph*)]
\item $u_{n}(x) = x^{n}$
\item $n = 0, 1, 2, \ldots$
\item $0 \leq x \leq 1$
\item $\omega (x) = 1$
\item Normalización:
\begin{align*}
\scaleint{6ex}_{\bs 0}^{1} \big[ P_{n}^{*} (x) \big]^{2} \dd{x} = \dfrac{1}{2 \, n + 1}
\end{align*}
\end{enumerate} 
%Ref. Zettili Example 2.6 (a)
\item ¿Los siguientes operadores son Hermitianos? Justifica tu respuesta.
\begin{enumerate}
\item $\big( \hat{A} + \hat{A}^{\dagger} \big)$
\item $i \, \big( \hat{A} + \hat{A}^{\dagger} \big)$
\item $i \, \big( \hat{A} - \hat{A}^{\dagger} \big)$
\end{enumerate}
%Ref. Zettili Example 2.6 (c)
\item Se define el valor esperado $\expval{\hat{A}}$ de un operador $\hat{A}$ con respecto al estado $\ket{\psi}$ como:
\begin{align*}
\expval{\hat{A}} = \dfrac{\ev{\hat{A}}{\psi}}{\braket{\psi}}    
\end{align*}
Demuestra que el valor esperado de un operador Hermitiano es real y que el anti-operador Hermitiano (con signo negativo) es un valor imaginario.
\item En el álgebra de los conmutadores, donde $\hat{A}, \hat{B}, \hat{C}, \hat{D}$ son operadores, se tiene que:
\begin{align*}
\big[ \hat{A}, \hat{B} \, \hat{C} \big] &= \big[ \hat{A}, \hat{B} \big] \, \hat{C} + \hat{B} \, \big[ \hat{A}, \hat{C} \big] \\[0.5em]
\big[ \hat{A} \, \hat{B} , \hat{C} \big] &= \hat{A} \, \big[\hat{B}, \hat{C} \big] + \big[ \hat{A}, \hat{C} \big] \, \hat{B}
\end{align*}
Evalúa el conmutador: $\big[ \hat{A}, \big[ \hat{B}, \hat{C} ] \, \hat{D}]$
%Ref. Ghatak 2004 - Problema 11.3
\item Demuestra que:
\begin{align*}
\big( \ket{P} \, \bra{Q} \big)^{*} = \ket{Q} \, \bra{P}
\end{align*}
%Ref. Tamvakis
\item Considera las siguientes funciones de onda unidimensionales que están normalizadas: $\psi_{0}(x)$ y $\psi_{1}(x)$, que cuentan con las propiedades:
\begin{align*}
\psi_{0}(-x) &= \psi_{0}(x) = \psi_{0}^{*} (x) \\
\psi_{1}(x) &= N \, \dv{\psi_{0}}{x}
\end{align*}
Considera también la combinación lineal:
\begin{align*}
\psi(x) = c_{1} \, \psi_{0}(x) + c_{2} \, \psi_{1} (x)
\end{align*}
con $\abs{c_{1}}^{2} + \abs{c_{2}}^{2} = 1$. Las constantes $N, c_{1}, c_{2}$, las consideramos conocidas.
\begin{enumerate}
\item Demuestra que $\psi_{0}(x)$ y $\psi_{1}(x)$ son ortogonales.
\item Demuestra que $\psi(x)$ está normalizada.
\end{enumerate}
\end{enumerate}




\end{document}