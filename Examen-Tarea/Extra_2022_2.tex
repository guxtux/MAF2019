\documentclass[hidelinks,12pt]{article}
\usepackage[left=0.25cm,top=1cm,right=0.25cm,bottom=1cm]{geometry}
%\usepackage[landscape]{geometry}
\textwidth = 20cm
\hoffset = -1cm
\usepackage[utf8]{inputenc}
\usepackage[spanish,es-tabla]{babel}
\usepackage[autostyle,spanish=mexican]{csquotes}
\usepackage[tbtags]{amsmath}
\usepackage{nccmath}
\usepackage{amsthm}
\usepackage{amssymb}
\usepackage{mathrsfs}
\usepackage{graphicx}
\usepackage{subfig}
\usepackage{standalone}
\usepackage[outdir=./Imagenes/]{epstopdf}
\usepackage{siunitx}
\usepackage{physics}
\usepackage{color}
\usepackage{float}
\usepackage{hyperref}
\usepackage{multicol}
%\usepackage{milista}
\usepackage{anyfontsize}
\usepackage{anysize}
%\usepackage{enumerate}
\usepackage[shortlabels]{enumitem}
\usepackage{capt-of}
\usepackage{bm}
\usepackage{relsize}
\usepackage{placeins}
\usepackage{empheq}
\usepackage{cancel}
\usepackage{wrapfig}
\usepackage[flushleft]{threeparttable}
\usepackage{makecell}
\usepackage{fancyhdr}
\usepackage{tikz}
\usepackage{bigints}
\usepackage{scalerel}
\usepackage{pgfplots}
\usepackage{pdflscape}
\pgfplotsset{compat=1.16}
\spanishdecimal{.}
\renewcommand{\baselinestretch}{1.5} 
\renewcommand\labelenumii{\theenumi.{\arabic{enumii}})}
\newcommand{\ptilde}[1]{\ensuremath{{#1}^{\prime}}}
\newcommand{\stilde}[1]{\ensuremath{{#1}^{\prime \prime}}}
\newcommand{\ttilde}[1]{\ensuremath{{#1}^{\prime \prime \prime}}}
\newcommand{\ntilde}[2]{\ensuremath{{#1}^{(#2)}}}

\newtheorem{defi}{{\it Definición}}[section]
\newtheorem{teo}{{\it Teorema}}[section]
\newtheorem{ejemplo}{{\it Ejemplo}}[section]
\newtheorem{propiedad}{{\it Propiedad}}[section]
\newtheorem{lema}{{\it Lema}}[section]
\newtheorem{cor}{Corolario}
\newtheorem{ejer}{Ejercicio}[section]

\newlist{milista}{enumerate}{2}
\setlist[milista,1]{label=\arabic*)}
\setlist[milista,2]{label=\arabic{milistai}.\arabic*)}
\newlength{\depthofsumsign}
\setlength{\depthofsumsign}{\depthof{$\sum$}}
\newcommand{\nsum}[1][1.4]{% only for \displaystyle
    \mathop{%
        \raisebox
            {-#1\depthofsumsign+1\depthofsumsign}
            {\scalebox
                {#1}
                {$\displaystyle\sum$}%
            }
    }
}
\def\scaleint#1{\vcenter{\hbox{\scaleto[3ex]{\displaystyle\int}{#1}}}}
\def\bs{\mkern-12mu}


\title{Evaluación del Examen Extraordinario 2022-2 \\[0.3em]  \large{Matemáticas Avanzadas de la Física}\vspace{-3ex}}
\author{M. en C. Gustavo Contreras Mayén}
\date{\today}
\begin{document}
\vspace{-4cm}
\maketitle
\fontsize{14}{14}\selectfont

El examen resuelto no cuenta con el orden solicitado, la claridad en las soluciones, el manejo en las respuestas no es suficiente, ni el esperado para un examen extraordinario de MAF. No se apoya en la herramienta de la física matemática, en cada ejercicio la solución no cuenta con un debido planteamiento, se generaliza demasiado y no hay una solución viable.

\begin{enumerate}
\item Problema 1: \textbf{No presenta la solución.} En el mensaje de correo se menciona que eligió el problema 1, pero en los siete archivos pdf, no está la solución al problema.
\item Problema 3: \textbf{Solución no correcta.}
\begin{enumerate}[label=\roman*)]
\item Inicia escribiendo la ecuación de Schrödinger con el operador gradiente cuando debe de ser el operador Laplaciano.
\item Debería de haber escrito la ecuación del problema en el sistema coordenado cilíndrico, utilizando la correspondiente expresión del Laplaciano en este sistema coordenado.
\item Propone una función de onda a modo de solución para utilizar la técnica de separación de variables, pero al no tener la ecuación en el sistema coordenado, no lleva el procedimiento para obtener las tres ecuaciones diferenciales ordinarias.
\item Su planteamiento del Laplaciano de la función radial no es correcto.
\item Lo siguiente no corresponde a la solución del problema.
\end{enumerate}
\item Problema 4: \textbf{Solución no correcta.}
\begin{enumerate}[label=\roman*)]
\item La solución propuesta no considera la geometría esférica que se indica en el enunciado.
\item No hay un desarrollo de la ecuación de Laplace en coordenadas esféricas.
\item No hay un desarrollo para obtener la solución completa del potencial electrostático en coordenadas esféricas con simetría azimutal.
\item Las expresiones que indica como solución, no corresponden a las soluciones correctas del problema.
\end{enumerate}
\item Problema 6: \textbf{Solución no correcta.} La solución que presenta está mal referenciada, la indica como problema 7. Considerando que el orden de los archivos en donde está la solución es: página 5 del archivo 1.pdf, archivo 2.pdf, archivo 3.pdf y archivo 4.pdf.
\begin{enumerate}[label=\roman*)]
\item  En la ecuación 1 indica la condición inicial $f(0, t)
 = h_{0} (t)$, no menciona de dónde la obtiene o por qué aparece.
\item En el paso siguiente donde señala que \enquote{inserta la derivada de $(2)$ en $(1)$}, no utiliza la correspondiente expresión para obtener la derivada de orden $3$ de la transformada de Fourier de la función $u (x, t)$.
\item Todo el desarrollo posterior no lleva a la solución del problema. 
\end{enumerate}
\item Problema 7: \textbf{Solución no correcta.} La solución que presenta está mal referenciada, la indica como problema 6. Páginas 1 y 2 del archivo 1.pdf.
\begin{enumerate}[label=\roman*)]
\item En la solución que propone, señala una equivalencia de los polinomios de Laguerre (menciona que \enquote{Laguerre se puede expresar como un polinomio ¿?})
\item Menciona que (la transformada de) Laplace es lineal, \enquote{podemos hacer la transformada de $X^{n}$ }, cuando debería de haber considerado: a) aplicar la transformada de Laplace en la ED de Laguerre, b) ocupar la expresión que evalúa la derivada de la transformada de Laguerre, c) organizar la expresión y ocupar las condiciones iniciales que se mencionan en el enunciado.
\item El desarrollo posterior no lleva a la solución del problema.
\end{enumerate}
\end{enumerate}

\end{document}