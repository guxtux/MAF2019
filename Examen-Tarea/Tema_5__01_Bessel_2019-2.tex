\documentclass[12pt]{article}
\usepackage[letterpaper]{geometry}
%\textwidth = 345.0pt
%\hoffset = -3cm
\usepackage[utf8]{inputenc}
\usepackage[spanish,es-tabla]{babel}
\usepackage[autostyle,spanish=mexican]{csquotes}
\usepackage{amsmath}
\usepackage{standalone}
\usepackage{tikz}   
\usepackage{tikz-3dplot}
\usepackage{nccmath}
\usepackage{amsthm}
\usepackage{amssymb}
\usepackage{graphicx}
\usepackage{comment}
\usepackage{siunitx}
\usepackage{physics}
\usepackage{color}
\usepackage{float}
\usepackage{multicol}
%\usepackage{milista}
\usepackage{enumitem}
\usepackage{anyfontsize}
\usepackage{anysize}
\marginsize{1cm}{1cm}{2cm}{2cm}
\usepackage{enumitem}
\usepackage{capt-of}
\usepackage{bm}
\usepackage{relsize}
\newlist{milista}{enumerate}{2}
\setlist[milista,1]{label=\arabic*)}
\setlist[milista,2]{label=\arabic{milistai}.\arabic*)}
\spanishdecimal{.}
\renewcommand{\baselinestretch}{1.5}
\author{ }
\usepackage{etoolbox}
\AtBeginEnvironment{tikzpicture}{\shorthandoff{>}\shorthandoff{<}}{}{}
\author{}
\title{Primera parte de la Tarea - Examen del Tema 5  \\ \large{Funciones de Bessel \\ Matemáticas Avanzadas de la Física}} \vspace{-1.5\baselineskip}
\date{ }
\begin{document}
\vspace{-4cm}
%\renewcommand\theenumii{\arabic{theenumii.enumii}}
\renewcommand\labelenumii{\theenumi.{\arabic{enumii})}}
\maketitle
\fontsize{14}{14}\selectfont
\begin{enumerate}
%Referencia Hassani
\item Un cilindro largo conductor de calor de radio $a$ se compone de dos mitades (con secciones transversales semicirculares) con un espacio infinitesimal entre ellas. Las mitades superior e inferior del cilindro están en contacto con baños térmicos $+T_{0}$ y $-T_{0}$, respectivamente. Encuentra la temperatura tanto dentro como fuera del cilindro.
\item Un cilindro largo conductor de calor de radio $a$ se compone de dos mitades (con secciones transversales semicirculares) con un espacio infinitesimal entre ellas. Las mitades superior e inferior del cilindro están en contacto con baños térmicos $+T_{1}$ y $-T_{1}$, respectivamente. El cilindro está dentro de otro cilindro de radio $b$ más grande ( $a < b$ y coaxial con él) que se mantiene a la temperatura $T_{2}$. Encuentra la temperatura dentro del cilindro interno, entre los dos cilindros y fuera del cilindro externo.
%Referencia Andrews
\item Una onda con distorsión modulada de fase puede representarse por
\begin{align*}
s(t) = R \, \cos [ \omega_{0} \, t +  \epsilon (t)]
\end{align*}
donde $R$ es la amplitud de la onda y $\epsilon(t)$ representa el \enquote{término de distorsión}. A menudo es suficiente aproximar $\epsilon (t)$ por el primer término de su serie de Fourier, es decir,
\begin{align*}
\epsilon (t) \cong a \, \sin \omega_{m} \, t
\end{align*}
donde $a$ es el pico de la fase y $\omega_{m}$ es la frecuencia, ambas de la distorsión. Por lo que la onda original es
\begin{align*}
s(t) \cong R \, \cos (\omega_{0} \, t + a \, \sin \omega_{m} \, t)
\end{align*}
Demuestra que $s(t)$ puede descomponerse en sus componentes armónicos de acuerdo a la siguiente expresión:
\begin{align*}
s(t) \cong R \, J_{0}(a) \, \cos \omega_{0} \, t + R \sum_{n=1}^{\infty} J_{n} (a) [\cos (\omega_{0} \, t + n \, \omega_{m} \, t) + (-1)^{n} \cos (\omega_{0} \, t - n \, \omega_{m} \, t) ]
\end{align*}
%Referencia Arfken
\item Una partícula (de masa $m$) está contenida en un cilindro circular rígido de radio $R$ y altura $H$. La partícula se describe mediante una función de onda que satisface la ecuación de onda de Schrödinger
\begin{align*}
- \dfrac{\hbar^{2}}{2 m} \laplacian \psi(\rho, \varphi, z) = E \, \psi(\rho, \varphi, z)
\end{align*}
y la condición de que la función de onda se anula en la superficie del cilindro. Demuestra que la energía mínima permitida (energía de punto cero) es:
\begin{align*}
E_{\text{min}} = \dfrac{\hbar^{2}}{2 m} \left[ \left( \dfrac{z_{pq}}{R} \right)^{2} + \left( \dfrac{n \pi}{H} \right)^{2} \right]
\end{align*}
donde $z_{pq}$ es el $q-$ésimo cero de $J_{p}$ y el índice $p$ está determinado por la dependencia azimutal.
% \item La amplitud de una onda difractada a través de una apertura circular está dada por
% \begin{align*}
% U = k \int_{0}^{a} \int_{0}^{2 \pi} \exp(i \, b \, r \, \sin \theta) \, r \dd{\theta} \dd{r}
% \end{align*}
% donde $k$ es una constante física, $a$ es el radio de la abertura, $\theta$ es el ángulo azimutal en el plano de la abertura y $b$ es una constante inversamente proporcional a la longitud de onda de la onda incidente. Demuestra que la intensidad de la luz en el patrón de difracción está dado por
% \begin{align*}
% I = \abs{U}^{2} = \dfrac{4 \, \pi \, k^{2} \, a^{2}}{b^{2}} \left[ J_{1} (a \, b) \right]^{2}
% \end{align*}
\item En el problema de la cadena oscilante discutido en clase, para el segundo modo $(n=2)$, existe un \emph{nodo} (punto fijo) ubicado en el intervalo $0 < x < L$. Demuestra que este nodo siempre se presenta en el punto $x \cong 0.190 \, L$.
\end{enumerate}
\end{document}