\documentclass[hidelinks,12pt]{article}
\usepackage[left=0.25cm,top=1cm,right=0.25cm,bottom=1cm]{geometry}
%\usepackage[landscape]{geometry}
\textwidth = 20cm
\hoffset = -1cm
\usepackage[utf8]{inputenc}
\usepackage[spanish,es-tabla]{babel}
\usepackage[autostyle,spanish=mexican]{csquotes}
\usepackage[tbtags]{amsmath}
\usepackage{nccmath}
\usepackage{amsthm}
\usepackage{amssymb}
\usepackage{mathrsfs}
\usepackage{graphicx}
\usepackage{subfig}
\usepackage{standalone}
\usepackage[outdir=./Imagenes/]{epstopdf}
\usepackage{siunitx}
\usepackage{physics}
\usepackage{color}
\usepackage{float}
\usepackage{hyperref}
\usepackage{multicol}
%\usepackage{milista}
\usepackage{anyfontsize}
\usepackage{anysize}
%\usepackage{enumerate}
\usepackage[shortlabels]{enumitem}
\usepackage{capt-of}
\usepackage{bm}
\usepackage{relsize}
\usepackage{placeins}
\usepackage{empheq}
\usepackage{cancel}
\usepackage{wrapfig}
\usepackage[flushleft]{threeparttable}
\usepackage{makecell}
\usepackage{fancyhdr}
\usepackage{tikz}
\usepackage{bigints}
\usepackage{scalerel}
\usepackage{pgfplots}
\usepackage{pdflscape}
\pgfplotsset{compat=1.16}
\spanishdecimal{.}
\renewcommand{\baselinestretch}{1.5} 
\renewcommand\labelenumii{\theenumi.{\arabic{enumii}})}
\newcommand{\ptilde}[1]{\ensuremath{{#1}^{\prime}}}
\newcommand{\stilde}[1]{\ensuremath{{#1}^{\prime \prime}}}
\newcommand{\ttilde}[1]{\ensuremath{{#1}^{\prime \prime \prime}}}
\newcommand{\ntilde}[2]{\ensuremath{{#1}^{(#2)}}}

\newtheorem{defi}{{\it Definición}}[section]
\newtheorem{teo}{{\it Teorema}}[section]
\newtheorem{ejemplo}{{\it Ejemplo}}[section]
\newtheorem{propiedad}{{\it Propiedad}}[section]
\newtheorem{lema}{{\it Lema}}[section]
\newtheorem{cor}{Corolario}
\newtheorem{ejer}{Ejercicio}[section]

\newlist{milista}{enumerate}{2}
\setlist[milista,1]{label=\arabic*)}
\setlist[milista,2]{label=\arabic{milistai}.\arabic*)}
\newlength{\depthofsumsign}
\setlength{\depthofsumsign}{\depthof{$\sum$}}
\newcommand{\nsum}[1][1.4]{% only for \displaystyle
    \mathop{%
        \raisebox
            {-#1\depthofsumsign+1\depthofsumsign}
            {\scalebox
                {#1}
                {$\displaystyle\sum$}%
            }
    }
}
\def\scaleint#1{\vcenter{\hbox{\scaleto[3ex]{\displaystyle\int}{#1}}}}
\def\bs{\mkern-12mu}


\usepackage{apacite}
\title{Examen - Tarea 1 \\[0.3em]  \large{Matemáticas Avanzadas de la Física}\vspace{-3ex}}
\author{M. en C. Gustavo Contreras Mayén}
\date{ }
\begin{document}
\vspace{-4cm}
\maketitle
\fontsize{14}{14}\selectfont

\textbf{Indicaciones: } Deberás de resolver cada ejercicio de la manera más completa, ordenada y clara posible, anotando cada paso así como las operaciones involucradas. El puntaje de cada ejercicio es de \textbf{1 punto}.

\begin{enumerate}
%Ref. Arfken 3.5.6
\item Para una partícula moviéndose en una órbita circular:
\begin{align*}
\vb{r} = \vu{e}_{x} \, r \, \cos \omega \, t + \vu{e}_{y} \, r \, \sin \omega \, t
\end{align*}
\begin{enumerate}
\item Evalúa $\vb{r} \cp \vb{\dot{r}}$ donde $\vb{\dot{r}} = \dv*{\vb{r}}{t} = \vb{v}$
\item Demuestra que $\vb{\ddot{r}} + \omega^{2} \, \vb{r} = 0$ donde $\vb{\ddot{r}} = \dv*{\vb{v}}{t}$
\end{enumerate}
%Ref. Arfken 2.4.11 6a ed.
\item Para el flujo de un fluido incompresible las ecuaciones de Navier-Stokes nos conducen a la expresión:
\begin{align*}
- \curl{( \mathbf{v} \cp (\curl{\vb{v}}))} =  \dfrac{\eta}{\rho_{0}} \, \laplacian{(\curl{\vb{v}})}
\end{align*}
Donde $\eta$ es la viscosidad y $\rho_{0}$ la densidad del fluido. Para un flujo axial dentro de un cilindro, consideremos que la velocidad $\mathbf{v}$ está dada por:
\begin{align*}
\mathbf{v} =  \vu{z} \, v (\rho)
\end{align*}
Considera que: $\curl{( \mathbf{v} \cp (\curl{\vb{v}}))} = 0$ 
para este valor de $\mathbf{v}$.
\begin{enumerate}
\item Demuestra que:
\begin{align*}
    \laplacian{( \curl{\vb{v}} )} = 0
\end{align*}
nos lleva a la ecuación diferencial:
\begin{align*}
    \dfrac{1}{\rho} \, \dv{\rho} \left( \rho \, \dv[2]{v}{\rho} \right) -  \dfrac{1}{\rho^{2}} \, \dv{v}{\rho} = 0
\end{align*}
\item Y que la siguiente expresión es solución de la misma ecuación diferencial:
\begin{align*}
    v = v_{0} + a_{2} \, \rho^{2}
\end{align*}
\end{enumerate}
%Ref. Arfken 2.4.15
\item El cálculo del efecto de retracción magnetohidrodinámica (efecto \emph{pinch}) implica la evaluación de $(\vb{B \cdot \nabla}) \, \vb{B}$. Si la inducción magnética $\vb{B}$ se considera que es: $\vb{B} = \vu{\varphi} \, B_{\varphi}(\rho)$, demuestra que:
\begin{align*}
\vb{(B \cdot \nabla}) \, \vb{B} = - \vu{\rho} \, \dfrac{B_{\varphi}^{2}}{\rho}
\end{align*}
\item La presencia de un campo eléctrico externo $E_{0}$, a lo largo del eje $z$ positivo, agrega un término de energía potencial $-e \, E_{0} \, z$ a la ecuación de onda de Schrödinger. Para el hidrógeno se tiene:
\begin{align*}
-\dfrac{\hbar}{2 \, M} \laplacian{\psi} - \dfrac{e^{2}}{r} \, \psi - e \, E_{0} \, z \, \psi = E \, \psi
\end{align*}
Esta es la ecuación de \emph{efecto Stark}. Expresa esta ecuación en el sistema de coordenadas parabólicas.
\item Utilizando las coordenadas esferoidales prolatas, calcula el volumen de una elipsoide prolata, es decir:
\begin{enumerate}
\item Expresa la integral que representa el volumen.
\item Evalúa la misma.
\end{enumerate}
%Ref. 2.5.13 (requiere el resultado del ejercicio 2.5.12)
\item Demuestra que:
\begin{align*}
- i \left( x \, \pdv{y} - y \, \pdv{x} \right) = - i \, \pdv{\varphi}
\end{align*}
Esta expresión en mecánica cuántica corresponde a la componente $z$ del operador momento angular orbital.
Sugerencias: 
\begin{enumerate}[label=\alph*)]
\item Calcula $\pdv*{x}$, $\pdv*{y}$, $\pdv*{y}$ en coordenadas esféricas.
\item Para el inciso anterior intenta igualar  $\nabla_{xyz}$ con $\nabla_{r \theta \varphi}$.
\end{enumerate}
%Ref. 2.5.14
\item \label{operador_momento_angular} Con el operador momento angular orbital de la mecánica cuántica definido como $\vb{L} = - i (\vb{r} \cp \nabla )$, demuestra que:
\begin{enumerate}
\item $L_{x} + i \, L_{y} = e^{i \varphi} \, \left( \displaystyle \pdv{\theta} + i \, \cot \theta \, \pdv{\varphi} \right)$
\item $L_{x} - i \, L_{y} = -e^{i \varphi} \, \left( \displaystyle \pdv{\theta} - i \, \cot \theta \, \pdv{\varphi} \right)$
\end{enumerate}
%Ref. 3.10.31 7th Ed.
\item Comprueba que en coordenadas esféricas: $\vb{L} \cp \vb{L} = i \, \vb{L}$, donde el operador momento angular orbital se definió en el problema (\ref{operador_momento_angular}).
usando la notación del conmutador: $[A, B] = A \, B -  B \, A$ y el símbolo de $\epsilon$ permutación (símbolo de Levi-Civita) $\epsilon_{ijk}$, se tiene que el conmutador de $L_{i}$ con $L_{j}$ es:
\begin{align*}
[L_{i}, L_{j}] = i \, \epsilon_{ijk} \, L_{k}
\end{align*}
Con $i, j, k$ son $x, y, z$ en ese orden. Escritas en forma de componentes, la relación anterior sería:
\begin{align*}
L_{y} \, L_{z} - L_{z} \, L_{y} &= i \, L_{x} \\
L_{z} \, L_{x} - L_{x} \, L_{z} &= - L_{y} \\
L_{x} \, L_{y} - L_{y} \, L_{x} &= i \, L_{z}
\end{align*}
\end{enumerate}
Los siguientes ejercicios (\ref{ejercicio_09}) y (\ref{ejercicio_10}) deberás de resolverlos ocupando las funciones Gamma y Beta.
\begin{enumerate}[resume]
%Ref. II-19
\item \label{ejercicio_09} Evalúa la integral:
\begin{align*}
\scaleint{5ex}_{\bs 0}^{1} \dfrac{x^{5}}{\sqrt[3]{1 - x^{4}}} \dd{x}
\end{align*}
%Ref Farrell II-7
\item \label{ejercicio_10} Demuestra que:
\begin{align*}
\scaleint{5ex}_{\bs 0}^{1} x^{m} \, \bigg[ \ln \left( \dfrac{1}{x} \right) \bigg]^{n} \dd{x} = \dfrac{\Gamma(n + 1)}{(m + 1)^{n+1}} \hspace{1.5cm} m > -1, n > -1
\end{align*}
\end{enumerate}

\end{document}