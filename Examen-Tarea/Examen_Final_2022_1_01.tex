\documentclass[12pt]{article}
\usepackage[left=0.25cm,top=1cm,right=0.25cm,bottom=1cm]{geometry}
\textwidth = 20cm
\hoffset = -1cm
\usepackage[utf8]{inputenc}
\usepackage[spanish,es-tabla]{babel}
\usepackage[autostyle,spanish=mexican]{csquotes}
\usepackage[tbtags]{amsmath}
\usepackage{nccmath}
\usepackage{amsthm}
\usepackage{amssymb}
\usepackage{graphicx}
\usepackage{standalone}
\usepackage[outdir=./]{epstopdf}
\usepackage{siunitx}
\usepackage{physics}
\usepackage{color}
\usepackage{float}
\usepackage{multicol}
%\usepackage{milista}
\usepackage{enumitem}
\usepackage{anyfontsize}
\usepackage{anysize}
\usepackage{enumitem}
\usepackage{capt-of}
\usepackage{bm}
\usepackage{relsize}
\usepackage{placeins}
\usepackage{empheq}
\usepackage{cancel}
\usepackage{wrapfig}
\spanishdecimal{.}
\renewcommand{\baselinestretch}{1.5} 
\renewcommand\labelenumii{\theenumi.{\arabic{enumii}}}
\newcommand{\ptilde}[1]{\ensuremath{{#1}^{\prime}}}
\newcommand{\stilde}[1]{\ensuremath{{#1}^{\prime \prime}}}
\newcommand{\ttilde}[1]{\ensuremath{{#1}^{\prime \prime \prime}}}
\newcommand{\ntilde}[2]{\ensuremath{{#1}^{(#2)}}}


\usepackage{apacite}
\title{Examen Final \\[0.3em]  \large{Matemáticas Avanzadas de la Física}\vspace{-3ex}}
\author{M. en C. Gustavo Contreras Mayén}
\date{ }
\begin{document}
\vspace{-4cm}
\maketitle
\fontsize{14}{14}\selectfont

\textbf{Importante: } Este examen cubre el contenido del curso, por lo que se te pide gentilmente que resuelvas de la manera más clara, completa y ordenada cada uno de los ejercicios. No se aceptan comprobaciones o cálculos obtenidos con Mathematica, Matlab, etc.
\par
Considera lo siguiente:
\begin{enumerate}
\item El examen deberá de devolverse completo, es decir, los seis ejercicios tendrán que estar resueltos, en caso contrario, se considerará como examen no entregado.
\item La calificación obtenida en caso de que sea aprobatoria, es la que se asentará en el acta del curso. En caso contrario, se puede presentar una segunda ronda.
\item Para presentar la segunda ronda de final se debe de entregar el examen resuelto de la primera ronda, es decir, no se puede optar por presentar la segunda ronda sin haber hecho la primera.
\item En caso de que el examen final no se entregue, se asentará 5 (cinco) en el acta.
\item \textbf{El examen resuelto deberá de ser enviado por correo a la cuenta del profesor, la fecha límite es para el sábado 29 de enero a las 5 pm.}
\end{enumerate}

\newpage

\begin{enumerate}
%Ref. Arfken (2006) 2.5.18
\item Demuestra que las tres siguientes formas (coordenadas esféricas) de $\laplacian{\psi} (r)$ son equivalentes:
\begin{multicols}{3}
\begin{enumerate}[label=\alph*)]
\item $\displaystyle \dfrac{1}{r^{2}} \, \dv{r} \bigg[ r^{2} \, \dv{\psi (r)}{r} \bigg]$
\item $\displaystyle  \dfrac{1}{r} \, \dv[2]{r} \big[ r \, \psi (r) \big]$
\item $\displaystyle  \dv[2]{\psi (r)}{r} + \dfrac{2}{r} \, \dv{\psi (r)}{r}$
\end{enumerate}
\end{multicols}
%Ref. 9.5.9
\item Resuelve
\begin{align*}
(1 - x^{2}) \sderivada{U}_{n} (x) - 3 \, x \, \pderivada{U}_{n} (x) + n (n + 2) \, U_{n} (x) = 0
\end{align*}
seleccionando la raíz de la ecuación de índices para obtener una serie de potencias \emph{impar} de $x$, ya que la serie diverge para $x = 1$, escoge $n$ para convertir la misma en un polinomio. Toma en cuenta que:
\begin{align*}
k (k - 1) = 0
\end{align*}
y que para $k = 1$:
\begin{align*}
a_{j+2} = \dfrac{(j + 1)(j + 3) - n (n + 2)}{(j + 2)(j + 3)} \, a_{j}
\end{align*}
%Ref. Riley 2006 - 17.7
\item Considera el conjunto de funciones, $\left\{ f (x) \right\}$, de variable real $x$, definida en el intervalo $-\infty < x < \infty$, que $\to 0$ al menos tan rápidamente como $x^{-1}$ cuando $x \to \pm \infty$. Con la función de peso unitaria, determina si cada uno de los siguientes operadores lineales es autoadjunto (Hermitiano) cuando actúa sobre $\left\{ f (x) \right\}$:
\begin{multicols}{2}
\begin{enumerate}[label=\alph*)]
\item $\displaystyle \dv{x} + x$
\item $\displaystyle - i \, \dv{x} + x^{2}$
\item $\displaystyle i \, x \, \dv{x}$
\item $\displaystyle i \, \dv[3]{x}$
\end{enumerate}
\end{multicols}
%Ref. Arfken 12.2.2
\item Con los polinomios de Legendre: mediante la diferenciación de la función generatriz $g (t, x)$ con respecto a $t$, multiplicando por $2 t$, para luego agregar $g (t,x )$, demuestra que:
\begin{align*}
\dfrac{1 - t^{2}}{(1 - 2 \, t \, x + t^{2})^{3/2}} = \nsum_{n=0}^{\infty} (2 \, n + 1) \, P_{n} (x) \, t^{n}
\end{align*}
Este resultado es de utilidad en el cálculo de la carga inducida en una esfera de metal conectada a tierra por medio de una carga de punto $q$.
%Ref. Arfken (2006) 11.1.25
\item La amplitud $U(\rho, \varphi, t)$ de una membrana circular oscilante de radio $a$ satisface la ecuación de onda:
\begin{align*}
\laplacian{U} - \dfrac{1}{v^{2}} \, \dv[2]{U}{t} = 0
\end{align*}
Donde $v$ es la velocidad de fase de la onda determinada por las constantes elásticas y cualesquiera que sea el amortiguamiento impuesto. Demuestra que una solución es:
\begin{align*}
U (\rho, \varphi, t) = J_{m} (k \, \rho) \big( a_{1} \, e^{i m \varphi} + a_{2} \, e^{- i m \varphi} \big)(b_{1} \, e^{i \omega t} + b_{2} \, e^{-i \omega t})
\end{align*}
%Ref. Arfken 15.8.5
\item Usando fracciones parciales, demuestra que para $a^{2} \neq b^{2}$:
\begin{enumerate}[label=\alph*)]
\item $L^{-1} \bigg[ \dfrac{1}{(p^{2} + a^{2})(p^{2} + b^{2})} \bigg] = - \dfrac{1}{a^{2} - b^{2}} \, \bigg[ \dfrac{\sin a t}{a} - \dfrac{\sin b t}{b} \bigg]$
\item $L^{-1} \bigg[ \dfrac{p^{2}}{(p^{2} + a^{2})(p^{2} + b^{2})} \bigg] = \dfrac{1}{a^{2} - b^{2}} \, \big[ a \, \sin a t -  b \, \sin b t \big]$
\end{enumerate} 
\end{enumerate}

\end{document}