\documentclass[hidelinks,12pt]{article}
\usepackage[left=0.25cm,top=1cm,right=0.25cm,bottom=1cm]{geometry}
%\usepackage[landscape]{geometry}
\textwidth = 20cm
\hoffset = -1cm
\usepackage[utf8]{inputenc}
\usepackage[spanish,es-tabla, es-lcroman]{babel}
\usepackage[autostyle,spanish=mexican]{csquotes}
\usepackage[tbtags]{amsmath}
\usepackage{nccmath}
\usepackage{amsthm}
\usepackage{amssymb}
\usepackage{mathrsfs}
\usepackage{graphicx}
\usepackage{subfig}
\usepackage{caption}
%\usepackage{subcaption}
\usepackage{standalone}
\graphicspath{{Imagenes/}{../Imagenes/}}
\usepackage[outdir=./Imagenes/]{epstopdf}
\usepackage{siunitx}
\usepackage{physics}
\AtBeginDocument{\RenewCommandCopy\qty\SI}
\ExplSyntaxOn
\msg_redirect_name:nnn { siunitx } { physics-pkg } { none }
\ExplSyntaxOff
\usepackage{color}
\usepackage{float}
\usepackage{hyperref}
\usepackage{multicol}
\usepackage{multirow}
%\usepackage{milista}
\usepackage{anyfontsize}
\usepackage{anysize}
%\usepackage{enumerate}
\usepackage[shortlabels]{enumitem}
\usepackage{capt-of}
\usepackage{bm}
\usepackage{mdframed}
\usepackage{relsize}
\usepackage{placeins}
\usepackage{empheq}
\usepackage{cancel}
\usepackage{pdfpages}
\usepackage{wrapfig}
\usepackage[flushleft]{threeparttable}
\usepackage{makecell}
\usepackage{fancyhdr}
\usepackage{tikz}
\usepackage{bigints}
\usepackage{tcolorbox}
\tcbuselibrary{breakable}
\usepackage{scalerel}
\usepackage{pgfplots}
\usepackage{pdflscape}
\usepackage{enumitem}

\pgfplotsset{compat=1.16}
\spanishdecimal{.}
\renewcommand{\baselinestretch}{1.5}
\def\scaleint#1{\vcenter{\hbox{\scaleto[3ex]{\displaystyle\int}{#1}}}}
\def\scaleoint#1{\vcenter{\hbox{\scaleto[3ex]{\displaystyle\oint}{#1}}}}
\def\scaleiint#1{\vcenter{\hbox{\scaleto[3ex]{\displaystyle\iint}{#1}}}}
\def\scaleiiint#1{\vcenter{\hbox{\scaleto[3ex]{\displaystyle\iiint}{#1}}}}
\def\bs{\mkern-12mu}

\newcommand{\Cancel}[2][black]{{\color{#1}\cancel{\color{black}#2}}}

% \newcommand{\qed}{\tag*{$\blacksquare$}}
\renewcommand{\qed}{\hfill\blacksquare}

\newcommand{\pderivada}[1]{\ensuremath{{#1}^{\prime}}}
\newcommand{\sderivada}[1]{\ensuremath{{#1}^{\prime \prime}}}
\newcommand{\tderivada}[1]{\ensuremath{{#1}^{\prime \prime \prime}}}
\newcommand{\nderivada}[2]{\ensuremath{{#1}^{(#2)}}}

\newlength{\depthofsumsign}
\setlength{\depthofsumsign}{\depthof{$\sum$}}
\newcommand{\nsum}[1][1.4]{% only for \displaystyle
    \mathop{%
        \raisebox
            {-#1\depthofsumsign+1\depthofsumsign}
            {\scalebox
                {#1}
                {$\displaystyle\sum$}%
            }
    }
}

\newlist{milista}{enumerate}{2}
\setlist[milista,1]{label=\roman*)}
\setlist[milista,2]{label=\roman{milista}.\roman*)}

\title{Tarea Tema 3 \\[0.3em] \large{Matemáticas Avanzadas de la Física}\vspace{-3ex}}
\author{M. en C. Gustavo Contreras Mayén}
\date{ }

\begin{document}
\vspace{-4cm}
\maketitle

\fontsize{14}{14}\selectfont

\textbf{Indicaciones: } Se te pide gentilmente que resuelvas de manera detallada, clara y ordenada los siguientes ejercicios, el puntaje que otorga cada enunciado es de \textbf{1 punto}. En caso de que requieras apoyarte en alguna propiedad, si fue vista en clase, solo indícalo, pero si esa propiedad aunque esté relacionada al ejercicio y no se haya mencionado en clase, habrá que demostrarla debidamente.

\begin{enumerate}
\item Demuestra que la ecuación diferencial de Hermite:
\begin{align*}
\sderivada{y} - 2 \, x \, \pderivada{y} + 2 \, \alpha \, y = 0
\end{align*}
se puede escribir de forma autoadjunta multiplicando por $\exp (-x^{2})$, usando la función de peso $\sigma (x) = \exp (-x^{2})$.
\item Cuentas con los siguientes elementos:
\begin{milista}
\item Un conjunto de funciones $\left\{ u_{n} (x) \right\} = \left\{ x^{n} \right\}, \mbox{ con } n = 1, 2, \ldots$
\item El intervalo $(0, \infty)$
\item Una función de peso $w(x) = x \, e^{-x}$
\end{milista}
Con el método de Gram-Schmidt construye las primeras \textbf{tres funciones ortonormales} del conjunto $u_{n}(x)$, con ese intervalo dado y con la función de peso dada.
% Ref. Ghatak (2004) - Quantum mechanics. 1.8. Problema 1.1
\item Para $f (x) = \abs{x}$, demuestra que:
\begin{align*}
\sderivada{f} (x) = 2 \, \delta (x)
\end{align*}
%Ref. Arfken 10.4.4
\item En lugar de la expansión de una función $F(x)$ dada por:
\begin{align*}
F (x) = \nsum_{n=0}^{\infty} a_{n} \, \varphi_{n} (x)
\end{align*}
con los coeficientes:
\begin{align*}
a_{n} = \scaleint{6ex}_{\bs a}^{b} F(x) \, \varphi_{n} (x) \, \omega (x) \dd{x}
\end{align*}
Considera la aproximación por una serie \textbf{finita}:
\begin{align*}
F (x) \approx \nsum_{n=0}^{m} c_{n} \, \varphi_{n} (x)
\end{align*}
Demuestra que el cuadrado del error medio:
\begin{align*}
\scaleint{6ex}_{\bs a}^{b} \bigg[ F(x) - \nsum_{n=0}^{m} c_{n} \, \varphi_{n} (x) \bigg]^{2} \, \omega (x) \dd{x}
\end{align*}
se minimiza cuando $c_{n} = a_{n}$.
\par
\noindent
\textbf{Nota: } Los valores de los coeficientes son independientes del número de términos en la serie finita. Esta independencia es una consecuencia de la ortogonalidad y no sería válida para un ajuste por mínimos cuadrados utilizando potencias de $x$.
%Ref. Arfken 10.4.7
% \item Recupera la desigualdad de Schwarz de la siguiente identidad:
% \begin{align*}
% &\bigg[ \scaleint{6ex}_{\bs a}^{b} f(x) \, g(x) \dd{x} \bigg
% ]^{2} {=} \scaleint{6ex}_{\bs a}^{b} \big[ f(x) \big
% ]^{2} \dd{x} \, \scaleint{6ex}_{\bs a}^{b} \big[ g(x) \big
% ]^{2} \dd{x} + \\[0.5em]
% &- \dfrac{1}{2} \, \scaleint{6ex}_{\bs a}^{b} \, \scaleint{6ex}_{\bs a}^{b} \bigg[ f(x) \, g(y) - f(y) \, g(x) \bigg
% ]^{2} \dd{x} \dd{y}
% \end{align*}
% \textbf{Nota:} Cuida el signo de la expresión, recuerda que al cortar el renglón, se deja el signo $+$, en el siguiente renglón se tiene el signo $-$, por lo que el segundo término está restando el producto del primer término.
\item De la lectura del artículo de V. Balakrishnan \enquote{\emph{All about the Dirac delta function}}, responde por qué el autor considera que la forma de $\delta (x - x_{0})$ se asemeja más al kernel de un operador integral.
\end{enumerate}


\end{document}