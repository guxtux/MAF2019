\documentclass[12pt]{article}
\usepackage[letterpaper]{geometry}
%\textwidth = 345.0pt
%\hoffset = -3cm
\usepackage[utf8]{inputenc}
\usepackage[spanish,es-tabla]{babel}
\usepackage[autostyle,spanish=mexican]{csquotes}
\usepackage{amsmath}
\usepackage{standalone}
\usepackage{tikz}   
\usepackage{tikz-3dplot}
\usepackage{nccmath}
\usepackage{amsthm}
\usepackage{amssymb}
\usepackage{graphicx}
\usepackage{comment}
\usepackage{siunitx}
\usepackage{physics}
\usepackage{color}
\usepackage{float}
\usepackage{multicol}
%\usepackage{milista}
\usepackage{enumitem}
\usepackage{anyfontsize}
\usepackage{anysize}
\marginsize{1cm}{1cm}{2cm}{2cm}
\usepackage{enumitem}
\usepackage{capt-of}
\usepackage{bm}
\usepackage{relsize}
\newlist{milista}{enumerate}{2}
\setlist[milista,1]{label=\arabic*)}
\setlist[milista,2]{label=\arabic{milistai}.\arabic*)}
\spanishdecimal{.}
\renewcommand{\baselinestretch}{1.5}
\author{ }
\usepackage{etoolbox}
\AtBeginEnvironment{tikzpicture}{\shorthandoff{>}\shorthandoff{<}}{}{}
\author{}
\title{Tarea Examen del Tema 6  \\ \large{Funciones de Legendre y Armónicos esféricos\\ Matemáticas Avanzadas de la Física}} \vspace{-1.5\baselineskip}
\date{ }
\begin{document}
\vspace{-4cm}
%\renewcommand\theenumii{\arabic{theenumii.enumii}}
\renewcommand\labelenumii{\theenumi.{\arabic{enumii})}}
\maketitle
\fontsize{14}{14}\selectfont
\begin{enumerate}
%Referencia Hassani
\item Una esfera conductora de calor de radio $a$ está compuesta por dos hemisferios con un espacio infinitesimal aislante entre ellos, revisa la figura (\ref{fig:figura2}). Las mitades superior e inferior de la esfera están en contacto con baños térmicos de temperaturas $+ T_{1}$ y $-T_{1}$, respectivamente. La esfera está dentro de otra esfera conductora de calor de radio $b$ con una temperatura $T_{2}$. Encuentra la temperatura en los puntos:
\begin{enumerate}
\item Dentro de la esfera interior,
\item En la región entre las dos esferas y
\item Por fuera de la esfera exterior.
\end{enumerate} 
\begin{figure}[!ht]
    \centering
    \includestandalone[scale=0.7]{Figuras/esfera1}
    %\includestandalone{esfera1}
    \caption{La esfera interior se encuentra a diferente temperatura.}
    \label{fig:figura2}
\end{figure}
%Referencia Jackson 4.1
\item Calcula los momentos multipolares $q_{\ell m}$ de una distribución de carga como se muestra en la figura (\ref{fig:figura_multipolo_01}).
\begin{figure}[!ht]
    \centering
    \includestandalone[scale=0.7]{Figuras/multipolo_01}
    \caption{Distribución de cargas para el ejercicio.}
    \label{fig:figura_multipolo_01}
\end{figure}
Calcula los resultados para los momentos no nulos válidos para todos los $\ell$.
\par
Considera que:
\begin{enumerate}
\item La densidad de carga en coordenadas esféricas es
\begin{align*}
\rho = \dfrac{q}{2 \, \pi \, a^{2}} \, \delta(r -a) [\delta (\cos \theta - 1) + \delta (\cos \theta + 1)] - \dfrac{2 \, q}{4 \, \pi \, r^{2}} \delta(r)
\end{align*}
\item Utiliza este valor de densidad de carga en la definición de momento multipolar y evalúa la integral
\begin{align*}
q_{\ell m} &= \int Y_{\ell  m}^{*} (\theta^{\prime}, \phi^{\prime}) \, r^{\prime \, \ell} \rho (\vb{r}^{\prime}) \dd{\vb{x}^{\prime}} \\
\Rightarrow q_{\ell m} &= \int Y_{\ell  m}^{*} (\theta^{\prime}, \phi^{\prime}) \, r^{\prime \, \ell} \, \left[\dfrac{q}{2 \, \pi \, a^{2}} \ \, \delta(r^{\prime} - a) [\delta (\cos \theta^{\prime} - 1) + \delta (\cos \theta^{\prime} + 1)]  \right. + \\
&- \left. \dfrac{2 \, q}{4 \, \pi \, r^{\prime \, 2}} \, \delta(r^{\prime}) \right] \dd{\vb{x}^{\prime}} \\
\Rightarrow q_{\ell m} &= \dfrac{q}{2 \, \pi} \int_{0}^{2 \pi} \int_{0}^{\pi} Y_{\ell  m}^{*} (\theta^{\prime}, \phi^{\prime}) \, a^{\ell} \, [\delta (\cos \theta^{\prime} - 1) + \delta (\cos \theta^{\prime} + 1)] \sin \theta^{\prime} \dd{\theta^{\prime}} \dd{\phi^{\prime}} + \\
&- \dfrac{2 \, q}{4 \, \pi} \int_{0}^{2 \pi} \int_{0}^{\pi} \int_{0}^{\infty} Y_{\ell  m}^{*} (\theta^{\prime}, \phi^{\prime}) \, r^{\prime \, \ell} \, \delta(r^{\prime}) \, \sin \theta^{\prime} \dd{r^{\prime}} \dd{\theta^{\prime}} \dd{\phi^{\prime}}
\end{align*}
Expande los armónicos esféricos y resuelve la integral para la coordenada azimutal.
\end{enumerate}
%Referencia. Arfken. Cap. 12 Legendre functions. Problem 12.3.11
\item La amplitud de una onda dispersada está dada por
\begin{align*}
f(\theta) = \dfrac{1}{k} \sum_{\ell = 0}^{\infty} (2 \, \ell + 1) \, \exp(i \, \delta_{\ell}) \, \sin \delta_{\ell} \, P_{\ell} (\cos \theta)
\end{align*}
Donde $\theta$ es el ángulo de dispersión, $\ell$ es el valor propio del momento angular, $\hbar \, k$ es el momento incidente, y $\delta_{\ell}$ es el desplazamiento de fase producido por el potencial central que está haciendo la dispersión. La sección transversal total es 
\begin{align*}
\sigma_{\text{tot}} = \int \abs{f(\theta)}^{2} \dd{\Omega}
\end{align*}
Demuestra que
\begin{align*}
\sigma_{\text{tot}} = \dfrac{4 \, \pi}{k^{2}} \sum_{\ell=0}^{\infty} (2 \, \ell + 1) \, \sin^{2} \delta_{\ell}
\end{align*}
%Referencia. Arfken. Cap. 12 Legendre functions. Problem 12.6.7
\item Los operadores de momento angular en mecánica cuántica $L_{x} \pm i \, L_{y}$, están dados por las expresiones:
\begin{align*}
L_{x} + i \, L_{y} &= e^{i \, \varphi} \left( \pdv{\theta} + i \, \cot \theta \, \pdv{\varphi} \right) \\
L_{x} - i \, L_{y} &= -e^{-i \, \varphi} \left( \pdv{\theta} - i \, \cot \theta \, \pdv{\varphi} \right)
\end{align*}
Demuestra que:
\begin{enumerate}
\itemsep1em
\item $(L_{x} + i \, L_{y}) \, Y_{L}^{M} (\theta, \varphi) = \sqrt{(L - M)(L + M + 1)} \, Y_{L}^{M+1} (\theta, \varphi)$
\item $(L_{x} - i \, L_{y}) \, Y_{L}^{M} (\theta, \varphi) = \sqrt{(L + M)(L - M + 1)} \, Y_{L}^{M-1} (\theta, \varphi)$
\end{enumerate}
\end{enumerate}
\end{document}