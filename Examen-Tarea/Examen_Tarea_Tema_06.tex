\documentclass[12pt]{article}
\usepackage[left=0.25cm,top=1cm,right=0.25cm,bottom=1cm]{geometry}
\textwidth = 20cm
\hoffset = -1cm
\usepackage[utf8]{inputenc}
\usepackage[spanish,es-tabla]{babel}
\usepackage[autostyle,spanish=mexican]{csquotes}
\usepackage[tbtags]{amsmath}
\usepackage{nccmath}
\usepackage{amsthm}
\usepackage{amssymb}
\usepackage{graphicx}
\usepackage{standalone}
\usepackage[outdir=./]{epstopdf}
\usepackage{siunitx}
\usepackage{physics}
\usepackage{color}
\usepackage{float}
\usepackage{multicol}
%\usepackage{milista}
\usepackage{enumitem}
\usepackage{anyfontsize}
\usepackage{anysize}
\usepackage{enumitem}
\usepackage{capt-of}
\usepackage{bm}
\usepackage{relsize}
\usepackage{placeins}
\usepackage{empheq}
\usepackage{cancel}
\usepackage{wrapfig}
\spanishdecimal{.}
\renewcommand{\baselinestretch}{1.5} 
\renewcommand\labelenumii{\theenumi.{\arabic{enumii}}}
\newcommand{\ptilde}[1]{\ensuremath{{#1}^{\prime}}}
\newcommand{\stilde}[1]{\ensuremath{{#1}^{\prime \prime}}}
\newcommand{\ttilde}[1]{\ensuremath{{#1}^{\prime \prime \prime}}}
\newcommand{\ntilde}[2]{\ensuremath{{#1}^{(#2)}}}


\usepackage{apacite}
\title{Examen - Tarea 6 \\[0.3em]  \large{Matemáticas Avanzadas de la Física}\vspace{-3ex}}
\author{M. en C. Gustavo Contreras Mayén}
\date{ }
\begin{document}
\vspace{-4cm}
\maketitle
\fontsize{14}{14}\selectfont

\textbf{Indicaciones: } Deberás de resolver cada ejercicio de la manera más completa, ordenada y clara posible, anotando cada paso así como las operaciones involucradas. El puntaje de cada ejercicio es de \textbf{1 punto}.

\begin{enumerate}
%Ref. Patra (2018) Ejercicios Cap. (1) (1)
\item Calcula la transformada de Fourier de la siguiente función:
\begin{align*}
f(x) = \dfrac{\sin a x}{x}, \hspace{1cm} a > 0
\end{align*}
% (10)
\item Demuestra que:
\begin{enumerate}
\item $F \bigg[ e^{-x^{2}}; x \to \xi \bigg] = \dfrac{1}{\sqrt{2}} \, \exp\bigg( - \dfrac{\xi^{2}}{2} \bigg)$
\item $F_{s} \bigg[ \dfrac{e^{- a x}}{x}; x \to \xi \bigg] = \sqrt{\dfrac{2}{\pi}} \tan^{-1} \dfrac{\xi}{a}$
\end{enumerate}
%Ref. Example 1.11
\item Sea la función:
\begin{align*}
f(x) = \begin{cases}
1, & 0 < x < a \\
0, & x > a
\end{cases}
\end{align*}
Utiliza una de las identidades de Parseval para evaluar:
\begin{align*}
\scaleint{6ex}_{\bs 0}^{\infty} \dfrac{\sin^{2} a x}{x^{2}} \dd{x}
\end{align*}
%Ref. Exercises (29)
\item Usando la TF resuelve las siguientes EDO:
\begin{enumerate}
\item $\sderivada{y}(x) - y (x) +  2 \, f(x) = 0$ donde $f(x) = 0$ cuando $x < -a$ y cuando $x > a$ y $f(x)$ y sus derivadas se anulan cuando $x \to \pm \infty$.
\item $2 \, \sderivada{y} (x) +  x \, \pderivada{y} (x) + y(x) = 0$
\end{enumerate}
%Ref. Patra (2018) Laplace Exercises (2)
\item Evalúa $L \big[ f(t); t \to p \big]$, donde:
\begin{enumerate}
\item $f (t) = \begin{cases}
\sin t, & 0 < t < \pi \\
0, & t > \pi
\end{cases}$
\item $f (t) = \sin \sqrt{t}$
\end{enumerate}
\item Evalúa $L \big[ (\sin a t - a t \cos a t ); t \to p \big]$ y calcula el valor del límite:
\begin{align*}
\lim_{t \to 0} \, L \big[ (\sin a t - a t \cos a t ); t \to p \big]
\end{align*}
\item Usando la transformada de Laplace evalúa la siguiente integral:
\begin{align*}
I = \scaleint{6ex}_{\bs 0}^{\infty} t \, \exp (- 2 t) \dd{t}
\end{align*}
para demostrar que $I = \dfrac{3}{25}$
%Ref. Spiegel - Schaum's Chap. 2 22 Pag. 57
\item Evalúa cada una de las siguientes expresiones utilizando el teorema de convolución:
\begin{enumerate}
\item $\mathcal{L}^{-1} \left\{ \dfrac{p}{(p^{2} + a^{2})^{2}} \right\}$
\item $\mathcal{L}^{-1} \left\{ \dfrac{1}{p^{2} (p + 1)^{2}} \right\}$
\end{enumerate}
\end{enumerate}
\end{document}