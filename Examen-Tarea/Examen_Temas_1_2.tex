\documentclass[hidelinks,12pt]{article}
\usepackage[left=0.25cm,top=1cm,right=0.25cm,bottom=1cm]{geometry}
%\usepackage[landscape]{geometry}
\textwidth = 20cm
\hoffset = -1cm
\usepackage[utf8]{inputenc}
\usepackage[spanish,es-tabla, es-lcroman]{babel}
\usepackage[autostyle,spanish=mexican]{csquotes}
\usepackage[tbtags]{amsmath}
\usepackage{nccmath}
\usepackage{amsthm}
\usepackage{amssymb}
\usepackage{mathrsfs}
\usepackage{graphicx}
\usepackage{subfig}
\usepackage{caption}
%\usepackage{subcaption}
\usepackage{standalone}
\graphicspath{{Imagenes/}{../Imagenes/}}
\usepackage[outdir=./Imagenes/]{epstopdf}
\usepackage{siunitx}
\usepackage{physics}
\AtBeginDocument{\RenewCommandCopy\qty\SI}
\ExplSyntaxOn
\msg_redirect_name:nnn { siunitx } { physics-pkg } { none }
\ExplSyntaxOff
\usepackage{color}
\usepackage{float}
\usepackage{hyperref}
\usepackage{multicol}
\usepackage{multirow}
%\usepackage{milista}
\usepackage{anyfontsize}
\usepackage{anysize}
%\usepackage{enumerate}
\usepackage[shortlabels]{enumitem}
\usepackage{capt-of}
\usepackage{bm}
\usepackage{mdframed}
\usepackage{relsize}
\usepackage{placeins}
\usepackage{empheq}
\usepackage{cancel}
\usepackage{pdfpages}
\usepackage{wrapfig}
\usepackage[flushleft]{threeparttable}
\usepackage{makecell}
\usepackage{fancyhdr}
\usepackage{tikz}
\usepackage{bigints}
\usepackage{tcolorbox}
\tcbuselibrary{breakable}
\usepackage{scalerel}
\usepackage{pgfplots}
\usepackage{pdflscape}
\usepackage{enumitem}
\pgfplotsset{compat=1.16}
\spanishdecimal{.}
\renewcommand{\baselinestretch}{1.5}
\def\scaleint#1{\vcenter{\hbox{\scaleto[3ex]{\displaystyle\int}{#1}}}}
\def\scaleoint#1{\vcenter{\hbox{\scaleto[3ex]{\displaystyle\oint}{#1}}}}
\def\scaleiint#1{\vcenter{\hbox{\scaleto[3ex]{\displaystyle\iint}{#1}}}}
\def\scaleiiint#1{\vcenter{\hbox{\scaleto[3ex]{\displaystyle\iiint}{#1}}}}
\def\bs{\mkern-12mu}

\newcommand{\Cancel}[2][black]{{\color{#1}\cancel{\color{black}#2}}}

% \newcommand{\qed}{\tag*{$\blacksquare$}}
\renewcommand{\qed}{\hfill\blacksquare}

\newcommand{\pderivada}[1]{\ensuremath{{#1}^{\prime}}}
\newcommand{\sderivada}[1]{\ensuremath{{#1}^{\prime \prime}}}
\newcommand{\tderivada}[1]{\ensuremath{{#1}^{\prime \prime \prime}}}
\newcommand{\nderivada}[2]{\ensuremath{{#1}^{(#2)}}}

\title{Examen Parcial 1 (Temas 1 y 2)\\[0.3em]  \large{Matemáticas Avanzadas de la Física}\vspace{-3ex}}
\author{M. en C. Gustavo Contreras Mayén}
\date{ }

\begin{document}
\vspace{-4cm}
\maketitle

\fontsize{14}{14}\selectfont

\textbf{Indicaciones: } Se te pide gentilmente que resuelvas de manera detallada, clara y ordenada los siguientes ejercicios, el puntaje que otorga cada enunciado es de \textbf{1 punto}. En caso de que requieras apoyarte en alguna propiedad, si fue vista en clase, solo indícalo, pero si esa propiedad aunque esté relacionada al ejercicio y no se haya mencionado en clase, habrá que demostrarla debidamente.

\begin{enumerate}
%Ref. Boas (2005) Section 8. Problems 9. pag. 525
\item Con el sistema de coordenadas bipolar $(u, v)$:
\begin{align*}
x &= \dfrac{a \, \sinh u}{\cosh u + \cos v} \\[0.5em]
y &= \dfrac{a \, \sin v}{\cosh u + \cos v}
\end{align*}
\begin{enumerate}[label=\alph*)]
\item Describe las superficies coordenadas que se generan.
\item \label{inciso_1_b} Determina el elemento $\dd{\vb{s}}$ y de los vectores unitarios $\vu{e}$, calcula las componentes de la velocidad y la aceleración.
\end{enumerate}
% \item En el sistema coordenado bipolar, calcula los operadores diferenciales: gradiente, divergencia, rotacional y laplaciano.
%Ref Arfken (2006) 8.1.13
\item Para $s$ entero no negativo, demuestra que:
\begin{align*}
(- 2 s - 1)!! = \dfrac{(-1)^{s}}{(2 s - 1)!!} = \dfrac{(-1)^{s} \, 2^{s} \, s!}{(2 s)!}
\end{align*}
%Ref Arfken (2006) 8.1.17
% \item \begin{enumerate}[label=\alph*)]
\item Demuestra que:
\begin{align*}
\Gamma \bigg( \dfrac{1}{2} - n \bigg) \, \Gamma \bigg( \dfrac{1}{2} + n \bigg) = (-1)^{n} \, \pi
\end{align*}
donde $n$ es un entero.
% \item Expresa $\Gamma \bigg( \dfrac{1}{2} + n \bigg)$ y $\Gamma \bigg( \dfrac{1}{2} - n \bigg)$ por separado en términos de $\pi^{\frac{1}{2}}$ y del doble factorial.
% \end{enumerate}
%Ref. Pinchover 5.6
\item Con el método de separación de variables (MSP): 
\begin{enumerate}[label=\roman*)]
\item Calcula una solución al siguiente problema periódico con la ecuación de calor:
\begin{table}[H]
\centering
\large
\begin{tabular}{l l}
$u_{t} - k \, u_{xx} = 0$ & $0 < x < 2 \, \pi,  t > 0$, \\
$u(0, t) = u(2 \, \pi, t) = 0, \quad u_{x} (0, t) = u_{x}(2 \, \pi, t)$ & $t \geq 0$, \\
$u(x, 0) = f(x)$ & $0 \leq x \leq 2 \, \pi$
\end{tabular}
\end{table}
donde $f$ es una función periódica \emph{suave} (bien portada, continua, derivable). Este sistema describe la evolución del calor en un alambre circular aislado de longitud $2 \, \pi$.
\item Encuentra el $\displaystyle{\lim_{t \to \infty}} \, u(x, t)$ para todo $0 < x < 2 \, \pi$, ¿qué interpretación física se tiene de este resultado?
\end{enumerate}
%Ref. Arfken 9.5.7
\item Resuelve la ecuación diferencial de Laguerre mediante una solución en series:
\begin{align*}
x \, \sderivada{L}_{n} (x) + (1 - x) \, \pderivada{L}_{n} (x) + n \, L_{n} (x) = 0
\end{align*}
Elige el parámetro $n$ tal que se trunque la serie y la solución se exprese como un polinomio.
%Ref. Arfken 9.6.3
\item Usando el determinante Wronskiano, demuestra que el conjunto de funciones:
\begin{align*}
\left\{ 1, \dfrac{x^{n}}{n!} \hspace{0.2cm} (n = 1, 2, \ldots, N) \right\}
\end{align*}
es linealmente independiente.
\item Resuelve las siguientes EDO2H, verificando inicialmente si presenta puntos singulares regulares, en caso de que se tenga a lo más, este tipo de puntos. Resuelve con el método de Frobenius cada ecuación, necesariamente con dos soluciones linealmente independientes.
\begin{enumerate}[label=\roman*)]
\item $x^{2} \sderivada{y} - x^{2} \pderivada{y} + (x^{2} - 2) y = 0$
\item $x^{2} \sderivada{y} - 3 \pderivada{y} + 4(x + 1) y = 0$
\end{enumerate}
\end{enumerate}

\end{document}