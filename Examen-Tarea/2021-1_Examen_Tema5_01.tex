\documentclass[hidelinks,12pt]{article}
\usepackage[left=0.25cm,top=1cm,right=0.25cm,bottom=1cm]{geometry}
%\usepackage[landscape]{geometry}
\textwidth = 20cm
\hoffset = -1cm
\usepackage[utf8]{inputenc}
\usepackage[spanish,es-tabla]{babel}
\usepackage[autostyle,spanish=mexican]{csquotes}
\usepackage[tbtags]{amsmath}
\usepackage{nccmath}
\usepackage{amsthm}
\usepackage{amssymb}
\usepackage{mathrsfs}
\usepackage{graphicx}
\usepackage{subfig}
\usepackage{standalone}
\usepackage[outdir=./Imagenes/]{epstopdf}
\usepackage{siunitx}
\usepackage{physics}
\usepackage{color}
\usepackage{float}
\usepackage{hyperref}
\usepackage{multicol}
%\usepackage{milista}
\usepackage{anyfontsize}
\usepackage{anysize}
%\usepackage{enumerate}
\usepackage[shortlabels]{enumitem}
\usepackage{capt-of}
\usepackage{bm}
\usepackage{relsize}
\usepackage{placeins}
\usepackage{empheq}
\usepackage{cancel}
\usepackage{wrapfig}
\usepackage[flushleft]{threeparttable}
\usepackage{makecell}
\usepackage{fancyhdr}
\usepackage{tikz}
\usepackage{bigints}
\usepackage{scalerel}
\usepackage{pgfplots}
\usepackage{pdflscape}
\pgfplotsset{compat=1.16}
\spanishdecimal{.}
\renewcommand{\baselinestretch}{1.5} 
\renewcommand\labelenumii{\theenumi.{\arabic{enumii}})}
\newcommand{\ptilde}[1]{\ensuremath{{#1}^{\prime}}}
\newcommand{\stilde}[1]{\ensuremath{{#1}^{\prime \prime}}}
\newcommand{\ttilde}[1]{\ensuremath{{#1}^{\prime \prime \prime}}}
\newcommand{\ntilde}[2]{\ensuremath{{#1}^{(#2)}}}

\newtheorem{defi}{{\it Definición}}[section]
\newtheorem{teo}{{\it Teorema}}[section]
\newtheorem{ejemplo}{{\it Ejemplo}}[section]
\newtheorem{propiedad}{{\it Propiedad}}[section]
\newtheorem{lema}{{\it Lema}}[section]
\newtheorem{cor}{Corolario}
\newtheorem{ejer}{Ejercicio}[section]

\newlist{milista}{enumerate}{2}
\setlist[milista,1]{label=\arabic*)}
\setlist[milista,2]{label=\arabic{milistai}.\arabic*)}
\newlength{\depthofsumsign}
\setlength{\depthofsumsign}{\depthof{$\sum$}}
\newcommand{\nsum}[1][1.4]{% only for \displaystyle
    \mathop{%
        \raisebox
            {-#1\depthofsumsign+1\depthofsumsign}
            {\scalebox
                {#1}
                {$\displaystyle\sum$}%
            }
    }
}
\def\scaleint#1{\vcenter{\hbox{\scaleto[3ex]{\displaystyle\int}{#1}}}}
\def\bs{\mkern-12mu}


\geometry{top=1.25cm, bottom=1.5cm, left=1.25cm, right=0.8cm}
%\usepackage{showframe}
\title{Examen Tarea Tema 5 - Primera parte \\ \large {Matemáticas Avanzadas de la Física}  \vspace{-3ex}}
\author{M. en C. Gustavo Contreras Mayén}
\date{ }
\begin{document}
\vspace{-4cm}
\maketitle
\fontsize{14}{14}\selectfont
\section{Funciones de Laguerre.}

\begin{enumerate}
\item \textbf{(1 punto.) } De acuerdo con la ecuación
\begin{align*}
\psi_{n l m} (r , \theta, \varphi) &= \left[ \left( \dfrac{2 \, Z}{n \, a_{0}} \right)^{3} \, \dfrac{(n - l -1)!}{2 \, n \, (n + l)!} \right]^{1/2} \, \exp \left( - \dfrac{\alpha \, r}{2} \right) * \\[1em]
&* (\alpha \, r)^{L} \, L_{n - l +1}^{2 l +1} \, (\alpha \, r) \, Y_{l}^{m} \, (\theta, \varphi)
\end{align*}
la parte normalizada de la función de onda para el átomo de hidrógeno es
\begin{align*}
R_{n l} (r) = \left[ \alpha^{3} \dfrac{(n -l -1)!}{2 \, n \, (n + l)!} \right]^{1/2} \, \exp \left( \dfrac{-\alpha \, r}{2} \right) \, (\alpha \, r)^{l} \, L_{n - l +1}^{2 l +1} \, (\alpha \, r) 
\end{align*}
en donde 
\begin{align*}
\alpha = \dfrac{2 \, Z}{n \, a_{0}} = \dfrac{2 \, Z \, m \, e^{2}}{4 \, \pi \, \epsilon_{0} \, \hbar^{2}}
\end{align*}
La cantidad $\expval{r}$ es el desplazamiento promedio del electrón con respecto al núcleo, mientras que $\expval{r^{-1}}$ es el promedio del movimiento recíproco.
\par
Demuestra que al evaluar las siguientes integrales se obtienen los valores indicados:
\begin{enumerate}
\item $\displaystyle \expval{r} = \int_{0}^{\infty} r \, R_{n l} (\alpha \, r) \, R_{n l} (\alpha \, r) \, r^{2} \dd{r} = \dfrac{a_{0}}{2} [3 \, n^{2} - l (l + 1)]$
\item $\displaystyle \expval{r^{-1}} = \int_{0}^{\infty} r^{-1} \, R_{n l} (\alpha \, r) \, R_{n l} (\alpha \, r) \, r^{2} \dd{r} = \dfrac{1}{n^{2} \, a_{0}}$
\\[0.5em]
Tip: Ocupa las propiedades de ortonormalización de los polinomios asociados de Laguerre.
\end{enumerate}
\item \textbf{(1 punto.) }Mediante el teorema del desarrollo, demuestra que la expansión de la siguiente función $f(x) = \exp(- a \, x)$ en la base $L_{n}^{k} (x)$ dejando $k$ fijo mientras que $n$ cambia de $0$ a $\infty$, en términos de los polinomios asociados de Laguerre es:
\begin{align*}
\exp(-a \, x) = \dfrac{1}{(1 + a)^{1 + k}} \, \sum_{n=0}^{\infty} \left( \dfrac{a}{1 + a} \right)^{n} \, L_{n}^{k} (x) \hspace{1.5cm} 0 \leq x < \infty
\end{align*}
\\[0.5em]
Tip: Deberás de realizar la expansión a partir de la función generatriz y determinar los coeficientes en el desarrollo supuesto.
\end{enumerate}

\section{Funciones de Bessel.}

\begin{enumerate}
% \item \textbf{(1 punto.) } Un cilindro largo conductor de calor de radio $a$ se compone de dos mitades (con secciones transversales semicirculares) con un espacio infinitesimal entre ellas. Las mitades superior e inferior del cilindro están en contacto con baños térmicos $+T_{0}$ y $-T_{0}$, respectivamente. Encuentra la temperatura tanto dentro como fuera del cilindro.
\item \textbf{(1 punto.) } Un cilindro largo conductor de calor de radio $a$ se compone de dos mitades (con secciones transversales semicirculares) con un espacio infinitesimal entre ellas. Las mitades superior e inferior del cilindro están en contacto con baños térmicos $+T_{1}$ y $-T_{1}$, respectivamente. El cilindro está dentro de otro cilindro de radio $b$ más grande ( $a < b$ y coaxial con él) que se mantiene a la temperatura $T_{2}$. Encuentra la temperatura dentro del cilindro interno, entre los dos cilindros y fuera del cilindro externo.
%Referencia Andrews - Chapter 6, Problem 26
\item \textbf{(1 punto.) } Una onda con distorsión modulada de fase puede representarse por
\begin{align*}
s(t) = R \, \cos [ \omega_{0} \, t +  \epsilon (t)]
\end{align*}
donde $R$ es la amplitud de la onda y $\epsilon(t)$ representa el \enquote{término de distorsión}. A menudo es suficiente aproximar $\epsilon (t)$ por el primer término de su serie de Fourier, es decir,
\begin{align*}
\epsilon (t) \cong a \, \sin \omega_{m} \, t
\end{align*}
donde $a$ es el pico de la fase y $\omega_{m}$ es la frecuencia, ambas de la distorsión. Por lo que la onda original es
\begin{align*}
s(t) \cong R \, \cos (\omega_{0} \, t + a \, \sin \omega_{m} \, t)
\end{align*}
Demuestra que $s(t)$ puede descomponerse en sus componentes armónicos de acuerdo a la siguiente expresión:
\begin{align*}
s(t) \cong R \, J_{0}(a) \, \cos \omega_{0} \, t + R \sum_{n=1}^{\infty} J_{n} (a) [\cos (\omega_{0} \, t + n \, \omega_{m} \, t) + (-1)^{n} \cos (\omega_{0} \, t - n \, \omega_{m} \, t) ]
\end{align*}
Recuerda que si ocupas una expresión que te sirva para la solución, deberás de demostrar cada expresión que utilices.
\end{enumerate}

\section{Funciones de Hermite.}

%Referencia Arfken
\begin{enumerate}
\item \textbf{(1 punto.) } La transición de probabilidad entre dos estados $\psi_{m}$ y $\psi_{n}$ del oscilador armónico cuántico, depende de que el valor de la siguiente integral sea:
\begin{align*}
\int_{-\infty}^{\infty} x \; e^{-x^{2}} \; H_{n} (x) \, H_{m}(x) \dd{x} = \sqrt{\pi} \; 2^{n-1} \; n! \; \delta_{m,n-1} + \sqrt{\pi} \; 2^{n} \; (n+1)! \; \delta_{m,n+1}
\end{align*}
Obtén explícitamente este valor. El resultado demuestra que cada transición puede ocurrir solamente entre estados con niveles de energía adyacentes, $m = n \pm 1$.
\item \textbf{(1 punto.) } Demuestra que
\begin{align*}
\int_{-\infty}^{+\infty} x^{2} \, \exp \left( -x^{2} \right) \, H_{n} (x) \, H_{n} (x) \dd{x} = \pi^{1/2} \, 2^{n} \, n! \, \left( n + \dfrac{1}{2} \right)
\end{align*}
La integral se presenta en el cálculo del desplazamiento medio cuadrático del oscilador cuántico.
\end{enumerate}
\end{document}