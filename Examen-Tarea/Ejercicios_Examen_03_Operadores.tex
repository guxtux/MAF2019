\documentclass[12pt]{article}
\usepackage[left=0.25cm,top=1cm,right=0.25cm,bottom=1cm]{geometry}
\textwidth = 20cm
\hoffset = -1cm
\usepackage[utf8]{inputenc}
\usepackage[spanish,es-tabla]{babel}
\usepackage[autostyle,spanish=mexican]{csquotes}
\usepackage[tbtags]{amsmath}
\usepackage{nccmath}
\usepackage{amsthm}
\usepackage{amssymb}
\usepackage{graphicx}
\usepackage{standalone}
\usepackage[outdir=./]{epstopdf}
\usepackage{siunitx}
\usepackage{physics}
\usepackage{color}
\usepackage{float}
\usepackage{multicol}
%\usepackage{milista}
\usepackage{enumitem}
\usepackage{anyfontsize}
\usepackage{anysize}
\usepackage{enumitem}
\usepackage{capt-of}
\usepackage{bm}
\usepackage{relsize}
\usepackage{placeins}
\usepackage{empheq}
\usepackage{cancel}
\usepackage{wrapfig}
\spanishdecimal{.}
\renewcommand{\baselinestretch}{1.5} 
\renewcommand\labelenumii{\theenumi.{\arabic{enumii}}}
\newcommand{\ptilde}[1]{\ensuremath{{#1}^{\prime}}}
\newcommand{\stilde}[1]{\ensuremath{{#1}^{\prime \prime}}}
\newcommand{\ttilde}[1]{\ensuremath{{#1}^{\prime \prime \prime}}}
\newcommand{\ntilde}[2]{\ensuremath{{#1}^{(#2)}}}


%\usepackage{showframe}
\title{Solución a ejercicios operadores \\ \large {Tema 3 - Bases completas y ortogonales} \vspace{-3ex}}
\author{M. en C. Gustavo Contreras Mayén}
\date{ }
\begin{document}
\vspace{-4cm}
\maketitle
\fontsize{14}{14}\selectfont

%Ref. Zettili (2009) Exercise 2.15

\begin{enumerate}
\item Demuestra la siguiente relación:
\begin{align*}
e^{\hat{A}} \, e^{\hat{B}} = e^{\hat{A} + \hat{B}} \, e^{\comm{\hat{A}}{\hat{B}}/2} 
\end{align*}
Para demostrar este resultado, debemos de revisar primero lo siguiente:
\begin{align*}
\mbox{Si } \comm{\hat{A}}{\comm{\hat{A}}{\hat{B}}} = 0 \hspace{0.3cm} \Rightarrow \hspace{0.3cm} e^{\hat{A}} \, \hat{B} \, e^{-\hat{A}} = \hat{B} + \comm{\hat{A}}{\hat{B}}
\end{align*}

Definimos:
\begin{align*}
\hat{B}(t) = e^{t \hat{A}} \, \hat{B} \, e^{-t \hat{A}}
\end{align*}
Obtenemos la primera y segunda derivada de $\hat{B}(t)$ con respecto a $t$:
\begin{align*}
\dv{\hat{B}(t)}{t} &= \hat{A} \, e^{t \hat{A}} \, \hat{B} \, e^{-t \hat{A}} - e^{t \hat{A}} \, \hat{B} \, e^{-t \hat{A}} = \comm{\hat{A}}{\hat{B}(t)} \\[0.5em]
\dv[2]{\hat{B}(t)}{t} &= \hat{A}^{2} \, e^{t \hat{A}} \, \hat{B} \, e^{-t \hat{A}} - 2 \, \hat{A} \, e^{t \hat{A}} \, \hat{B} \, e^{-t \hat{A}} + e^{t \hat{A}} \, \hat{B} \, e^{-t \hat{A}} \, \hat{A}^{2} = \\[0.5em]
&= \comm{\hat{A}}{\comm{\hat{A}}{\hat{B}(t)}}
\end{align*}

Como $\comm{\hat{A}}{\comm{\hat{A}}{\hat{B}}} = 0$, se debe de cumplir que al reemplazar \hfill \break $\hat{A} \to \, t \, \hat{A}, \forall \, t$, se tiene entonces que:
\begin{align*}
\comm{\hat{A}}{\comm{\hat{A}}{\hat{B}}} = 0 \hspace{0.5cm} \mbox{y} \hspace{0.5cm} \dv[2]{\hat{B}(t)}{t} = 0
\end{align*}


\end{enumerate}

\end{document}