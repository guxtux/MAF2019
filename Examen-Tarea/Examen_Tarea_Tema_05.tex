\documentclass[hidelinks,12pt]{article}
\usepackage[left=0.25cm,top=1cm,right=0.25cm,bottom=1cm]{geometry}
%\usepackage[landscape]{geometry}
\textwidth = 20cm
\hoffset = -1cm
\usepackage[utf8]{inputenc}
\usepackage[spanish,es-tabla]{babel}
\usepackage[autostyle,spanish=mexican]{csquotes}
\usepackage[tbtags]{amsmath}
\usepackage{nccmath}
\usepackage{amsthm}
\usepackage{amssymb}
\usepackage{mathrsfs}
\usepackage{graphicx}
\usepackage{subfig}
\usepackage{standalone}
\usepackage[outdir=./Imagenes/]{epstopdf}
\usepackage{siunitx}
\usepackage{physics}
\usepackage{color}
\usepackage{float}
\usepackage{hyperref}
\usepackage{multicol}
%\usepackage{milista}
\usepackage{anyfontsize}
\usepackage{anysize}
%\usepackage{enumerate}
\usepackage[shortlabels]{enumitem}
\usepackage{capt-of}
\usepackage{bm}
\usepackage{relsize}
\usepackage{placeins}
\usepackage{empheq}
\usepackage{cancel}
\usepackage{wrapfig}
\usepackage[flushleft]{threeparttable}
\usepackage{makecell}
\usepackage{fancyhdr}
\usepackage{tikz}
\usepackage{bigints}
\usepackage{scalerel}
\usepackage{pgfplots}
\usepackage{pdflscape}
\pgfplotsset{compat=1.16}
\spanishdecimal{.}
\renewcommand{\baselinestretch}{1.5} 
\renewcommand\labelenumii{\theenumi.{\arabic{enumii}})}
\newcommand{\ptilde}[1]{\ensuremath{{#1}^{\prime}}}
\newcommand{\stilde}[1]{\ensuremath{{#1}^{\prime \prime}}}
\newcommand{\ttilde}[1]{\ensuremath{{#1}^{\prime \prime \prime}}}
\newcommand{\ntilde}[2]{\ensuremath{{#1}^{(#2)}}}

\newtheorem{defi}{{\it Definición}}[section]
\newtheorem{teo}{{\it Teorema}}[section]
\newtheorem{ejemplo}{{\it Ejemplo}}[section]
\newtheorem{propiedad}{{\it Propiedad}}[section]
\newtheorem{lema}{{\it Lema}}[section]
\newtheorem{cor}{Corolario}
\newtheorem{ejer}{Ejercicio}[section]

\newlist{milista}{enumerate}{2}
\setlist[milista,1]{label=\arabic*)}
\setlist[milista,2]{label=\arabic{milistai}.\arabic*)}
\newlength{\depthofsumsign}
\setlength{\depthofsumsign}{\depthof{$\sum$}}
\newcommand{\nsum}[1][1.4]{% only for \displaystyle
    \mathop{%
        \raisebox
            {-#1\depthofsumsign+1\depthofsumsign}
            {\scalebox
                {#1}
                {$\displaystyle\sum$}%
            }
    }
}
\def\scaleint#1{\vcenter{\hbox{\scaleto[3ex]{\displaystyle\int}{#1}}}}
\def\bs{\mkern-12mu}


\usepackage{apacite}
\title{Examen - Tarea 5 \\[0.3em]  \large{Matemáticas Avanzadas de la Física}\vspace{-3ex}}
\author{M. en C. Gustavo Contreras Mayén}
\date{ }
\begin{document}
\vspace{-4cm}
\maketitle
\fontsize{14}{14}\selectfont

\textbf{Indicaciones: } Deberás de resolver cada ejercicio de la manera más completa, ordenada y clara posible, anotando cada paso así como las operaciones involucradas. El puntaje de cada ejercicio es de \textbf{2 puntos}, excepto en donde se indica.

\begin{enumerate}
%Ref. Arfken (2006) 13.2.9
\item \textbf{Laguerre. } De acuerdo con la ecuación:
\begin{align*}
\psi_{n l m} (r , \theta, \varphi) &= \left[ \left( \dfrac{2 \, Z}{n \, a_{0}} \right)^{3} \, \dfrac{(n - l -1)!}{2 \, n \, (n + l)!} \right]^{1/2} \, \exp \left( - \dfrac{\alpha \, r}{2} \right) \times \\[1em]
&\times (\alpha \, r)^{L} \, L_{n - l +1}^{2 l +1} \, (\alpha \, r) \, Y_{l}^{m} \, (\theta, \varphi)
\end{align*}
la parte normalizada de la función de onda para el átomo de hidrógeno es:
\begin{align*}
R_{n l} (r) = \left[ \alpha^{3} \dfrac{(n -l -1)!}{2 \, n \, (n + l)!} \right]^{1/2} \, \exp \left( \dfrac{-\alpha \, r}{2} \right) \, (\alpha \, r)^{l} \, L_{n - l +1}^{2 l +1} \, (\alpha \, r) 
\end{align*}
en donde: 
\begin{align*}
\alpha = \dfrac{2 \, Z}{n \, a_{0}} = \dfrac{2 \, Z \, m \, e^{2}}{4 \, \pi \, \epsilon_{0} \, \hbar^{2}}
\end{align*}
La cantidad $\expval{r}$ es el desplazamiento promedio del electrón con respecto al núcleo, mientras que $\expval{r^{-1}}$ es el promedio del movimiento recíproco.
\par
Demuestra que al evaluar las siguientes integrales se obtienen los valores indicados:
\begin{enumerate}
\item $\displaystyle \expval{r} = \int_{0}^{\infty} r \, R_{n l} (\alpha \, r) \, R_{n l} (\alpha \, r) \, r^{2} \dd{r} = \dfrac{a_{0}}{2} [3 \, n^{2} - l (l + 1)]$
\item $\displaystyle \expval{r^{-1}} = \int_{0}^{\infty} r^{-1} \, R_{n l} (\alpha \, r) \, R_{n l} (\alpha \, r) \, r^{2} \dd{r} = \dfrac{1}{n^{2} \, a_{0}}$
\\[0.5em]
Tip: Ocupa las propiedades de ortonormalización de los polinomios asociados de Laguerre.
\end{enumerate}
%Ref. Arfken (2006) 13.1.14
\item \textbf{Hermite. }Con
\begin{align*}
\psi_{n} (x) = \exp \bigg( - \dfrac{x^{2}}{2} \bigg) \, \dfrac{H_{n}(x)}{\big( 2^{n} \, n! \, \pi^{\frac{1}{2}})^{\frac{1}{2}}}
\end{align*}
verifica que:
\begin{align*}
\hat{a}_{n} \, \psi (x) &= \dfrac{1}{\sqrt{2}} \bigg( x + \dv{x} \bigg) \, \psi (x) = n^{\frac{1}{2}} \, \psi_{n-1} (x) \\[0.5em]
\hat{a}_{n}^{\dagger} \, \psi (x) &= \dfrac{1}{\sqrt{2}} \bigg( x - \dv{x} \bigg) \, \psi (x) = \big( n + 1 \big)^{\frac{1}{2}} \, \psi_{n+1} (x)
\end{align*}
El procedimiento del operador mecánico cuántico usual establece estas propiedades de ascenso y descenso \emph{antes} de conocer la forma de $\psi_{n}(x)$.
%Ref. Arfken (2006) 11.1.12
\item \textbf{Bessel. (Un punto) } Un análisis de los patrones de difracción de antena para un sistema con una abertura circular, involucra la ecuación:
\begin{align*}
g(u) = \scaleint{6ex}_{\bs 0}^{1} f(r) \, J_{0}(u \, r) \, r \dd{r}
\end{align*}
Si $f(r) = 1 - r^{2}$. Demuestra que:
\begin{align*}
g(u) = \dfrac{2}{u^{2}} \, J_{2} (u)
\end{align*}
%Ref. Arfken (2006) 11.1.15
\item \textbf{Bessel. (Un punto) } Una partícula (de masa $m$) se encuentra contenida en un cilindro circular recto (\emph{caja de píldoras}) de radio $R$ y altura $H$. La partícula se describe por medio de una función de onda que satisface la ecuación de Schrödinger:
\begin{align*}
- \dfrac{\hbar^{2}}{2 \, m} \laplacian \psi (\rho, \varphi, z) = E \, \psi (\rho, \varphi, z)
\end{align*}
y la condición de que la función de onda se anula en la superficie de la caja de píldoras. Demuestra que la energía permisible más baja (punto cero) es:
\begin{align*}
E = \dfrac{\hbar}{2 \, m} \bigg[ \bigg( \dfrac{z_{pq}}{R} \bigg)^{2} + \bigg( \dfrac{n \, \pi}{H} \bigg)^{2} \bigg] 
\end{align*}
en donde $z_{pq}$ es el q-ésimo cero de $J_{p}$ y en donde el índice $p$ está fijado por la dependencia azimutal. Entonces:
\begin{align*}
E_{\min} = \dfrac{\hbar}{2 \, m} \bigg[ \bigg( \dfrac{2.405}{R} \bigg)^{2} + \bigg( \dfrac{\pi}{H} \bigg)^{2} \bigg] 
\end{align*}
%Ref. Arfken (2006) 13.3.20
\item \textbf{Chebyshev. } Existen varias ecuaciones que relacionan los dos tipos de polinomios de Chebyshev. Demuestra que:
\begin{enumerate}
\item $T_{n}(x) = U_{n}(x) - x \, U_{n-1}(x)$
\item $(1 - x^{2}) \, U_{n}(x) = x \, T_{n+1}(x) - T_{n+2}(x)$
\end{enumerate}
%Ref. Arfken (2006) 13.4.3 (b)-(c)
\item \textbf{Hipergeométrica. } Demuestra que los siguiente polinomios transformados a funciones hipergeométricas con argumento $x^{2}$, son:
\begin{enumerate}
\item $x^{-1} \, T_{2n+1} (x) = (-1)^{n} \, (2 \, n + 1 ) \, \, {}_{2}F_{1} \bigg(-n, n+1; \dfrac{3}{2}; x^{2} \bigg)$
\item $U_{2n} (x) = (-1)^{n} \, \, \, {}_{2}F_{1} \bigg(-n, n + 1; \dfrac{1}{2}; x^{2} \bigg)$
\end{enumerate}
\end{enumerate}

\end{document}