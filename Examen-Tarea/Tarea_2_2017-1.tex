\documentclass[12pt]{article}
\usepackage[letterpaper]{geometry}
%\textwidth = 345.0pt
%\hoffset = -3cm
\usepackage[utf8]{inputenc}
\usepackage[spanish,es-tabla]{babel}
\usepackage[autostyle,spanish=mexican]{csquotes}
\usepackage{amsmath}
\usepackage{nccmath}
\usepackage{amsthm}
\usepackage{amssymb}
\usepackage{graphicx}
\usepackage{comment}
\usepackage{siunitx}
\usepackage{physics}
\usepackage{color}
\usepackage{float}
\usepackage{multicol}
%\usepackage{milista}
\usepackage{enumitem}
\usepackage{anyfontsize}
\usepackage{anysize}
\marginsize{1cm}{1cm}{1cm}{1cm}
\usepackage{enumitem}
\usepackage{capt-of}
\usepackage{bm}
\usepackage{relsize}
\newlist{milista}{enumerate}{2}
\setlist[milista,1]{label=\arabic*)}
\setlist[milista,2]{label=\arabic{milistai}.\arabic*)}
\spanishdecimal{.}
\renewcommand{\baselinestretch}{1.5}
\author{ }
\title{Problemas para  la Tarea Examen del Tema 2 \\ \large{Matemáticas Avanzadas de la Física}\vspace{-8ex}}
\date{ }
\begin{document}
\vspace{-4cm}
\renewcommand\labelenumii{\theenumi.{\arabic{enumii}}}
\maketitle
\fontsize{14}{14}\selectfont
\begin{milista}
\item Demostrar que la ecuación de Helmholtz
\begin{align*}
\laplacian \psi + k^{2} \: \psi = 0
\end{align*}
es separable en coordenadas cilíndricas circulares si $k^{2}$ se generaliza como
\begin{align*}
k^{2} + f(\rho) + \left( \dfrac{1}{\rho^{2}} \right) \: g(\varphi) + h(z)
\end{align*}
\item Demostrar que
\begin{align*}
\laplacian \psi (r, \theta, \varphi) + \left[ k^{2} + f(r) + \dfrac{1}{r^{2}} \: g(\theta) + \dfrac{1}{r^{2} \sin^{2}} \: h(\varphi) \right] \: \psi (r, \theta, \varphi) = 0
\end{align*}
es separable en coordenadas esféricas. Las funciones $f, g, h$ son funciones sólo de las variables indicadas, $k^{2}$ es constante.
%\item Para una esfera sólida homogénea con constante de difusión términa $K$, la ecuación de conducción de calor (sin fuentes) es
%\[ \dfrac{\partial T(r,t)}{\partial t} =  K \nabla^{2} T(r,t) \]
%Mediante la técnica de separación de variables, suponemos que tiene una solución de la forma
%\[ T =R(r) T(t) \]
%Demuestra que la ecuación radial toma la forma estándar
%\[ r^{2} \dfrac{d^{2} R}{d r^{2}} + 2r \dfrac{d R}{d r} + \left[ \alpha^{2} r^{2} - n(n+1) \right] R = 0, \hspace{1cm} n = \mbox{ entero} \]
\item Utiliza el método de Frobenius para obtener la solución general de cada una de las siguientes ecuaciones diferenciales, para un entorno de $x = 0$:
\begin{milista}
\begin{fleqn}
\setlength\itemsep{1em}
\item  $ \displaystyle 2 \, x \, \dv[2]{y}{x} + (1 - x^{2}) \dv{y}{x} - y = 0 $
\item $ \displaystyle x^{2} \, \dv[2]{y}{x} + x \,\dv{y}{x} + (x^{2} - 1) \, y = 0 $
\end{fleqn}
\end{milista}
\item Una solución a la ecuación diferencial de Laguerre
\begin{align*}
x \, y^{\prime \prime} + (1-x) y^\prime + n \, y = 0
\end{align*}
para $n=0$ es $y_{1}(x)=1$. Desarrolla una segunda solución linealmente independiente.
\item A partir del estudio en mecánica cuántica del efecto Stark (en coordenadas parabólicas), nos conduce a la ecuación diferencial
\begin{align*}
\dv{\xi} \left( \xi\, \dv{u}{\xi} \right) + \left( \dfrac{1}{2} \, E \, \xi + \alpha - \dfrac{m^{2}}{4 \, \xi} - \dfrac{1}{4} \, F \, \xi^{2} \right) \, u = 0
\end{align*}
donde $\alpha$ es la constante de separación, $E$ es la energía total del sistema, $F$ es una constante, $Fz$ es la energía potencial que se agrega al introducir un campo eléctrico.

Usando la raíz más grande de la ecuación de índices, desarrolla una solución en series de potencias, alrededor de $\xi=0$. Evalúa los primeros tres coeficientes en términos de $a_{0}$, a lo que tienes que llegar es a la expresión $u (\xi)$:
\begin{align*}
& \text{Ecuación indicial } \to \hspace{1.5cm} k^{2} - \dfrac{m^{2}}{4} = 0 \\
u(\xi) &=  a_{0} \, \xi^{m/2} \left\lbrace 1 - \dfrac{\alpha}{m+1} \, \xi + \left[ \dfrac{\alpha^{2}}{2(m+1)(m+2)} - \dfrac{E}{4(m+2)} \right] \, \xi^{2} + \ldots \right\rbrace
\end{align*}
Checa que la perturbación $E$ no se presenta hasta que el término $a_{3}$ se incluye.
\item En el estudio del ion molecular del hidrógeno $(H2^{+})$ en mecánica cuántica, para el caso especial en donde no hay dependencia en la coordenada azimutal, se llega a la ecuación
\begin{align*}
\dv{\eta} \left[ (1 - \eta^{2} ) \, \dv{u}{\eta} \right] + \alpha \, u + \beta \, \eta^{2} \, u = 0
\end{align*}
Desarrolla una solución en series de potencias para $u(\eta)$. Evalúa los primeros tres coeficientes no nulos en términos de $a_{0}$, tendrás que obtener la expresión para $u_{k=1}$:
\begin{align*}
& \text{Ecuación indicial } \to \hspace{1.5cm} k(k-1) = 0 \\
u_{k=1} &=  a_{0} \, \eta \left\lbrace 1 - \dfrac{2- \alpha}{6} \, \eta^{2} + \left[ \dfrac{(2-\alpha)(12-\alpha)}{120} - \dfrac{\beta}{20} \right] \, \eta^{4} + \ldots \right\rbrace
\end{align*}
% \item Inicia con 
% \[ J_{0} (x) = 1 - \dfrac{x^{2}}{4} +  \dfrac{x^{4}}{64} -  \dfrac{x^{6}}{2304} + \ldots \]
% Obtén la segunda solución linealmente independiente
% \[ y_{2}(x) = J_{0}(x) ln(x) + \dfrac{x^{2}}{4} -  \dfrac{3 x^{4}}{128} +  \dfrac{11 x^{6}}{13284}+ \ldots \]
% de la ecuación de Bessel de orden cero.
\item Demuestra que se puede escribir
\begin{align*}
\delta (x - \xi) = \dfrac{2}{L} \sum_{n=1}^{\infty} \sin \left( \dfrac{n \, \pi \, \xi}{L} \right) \, \sin \left( \dfrac{n \, \pi \, x}{L} \right) \hspace{1.5cm} 0 < \xi < L
\end{align*}
\item Una representación importante de la delta de Dirac se construye considerando el límite $n \to \infty$ de la secuencia
\begin{align*}
\delta_{n} = \begin{cases}
c_{n} \, (1 - x^{2})^{n} & \mbox{ para } 0 \leq \abs{x} \leq 1, \hspace{0.5cm} n = 1, 2, 3, \ldots \\
0 & \mbox{ para } \abs{x} > 1
\end{cases}
\end{align*}
\begin{milista}
\item Determina los coeficientes $c_{n}$ tales que $ \displaystyle \int_{-1}^{1} \delta_{n} (x) \, \dd{x} = 1 $
\item Demuestra que $\displaystyle \lim_{n \to \infty} \int_{-1}^{1} f(x) \, \delta_{n} (x) \, \dd{x} = f(0)$
\end{milista}
Para los problemas \ref{p1-Greiner}, \ref{p2-Greiner} y \ref{p3-Greiner}, te pedimos que consultes la referencia  \textit{pág. 45 de Walter Greiner. Classical Electrodynamics. Theoretical Physics, Springer, 1991.} En donde se explica parte de la solución, tendrás que detallar TODO el proceso, sin omitir pasos y explicando lo más posible cada uno de ellos. Estos ejercicios tienen el objetivo de guiar el uso del teorema de Green para la solución de problemas en electrodinámica.

\item \label{p1-Greiner} Construye la siguiente ecuación:
\begin{align*}
\int_{V} [ \varphi(\vb{r^{\prime}}) \, \Delta^{\prime} \, \psi (\vb{r^{\prime}}) &-  \psi (\vb{r^{\prime}}) \, \Delta^{\prime} \, \varphi (\vb{r^{\prime}})] \dd V^{\prime} = \\
&\oint_{S} \left[ \varphi (\vb{r^\prime}) \, \pdv{\psi (\vb{r^{\prime}})}{n^{\prime}} - \psi (\vb{r^{\prime}}) \pdv{\varphi (\vb{r^{\prime}})}{n^{\prime}} \right] \, \dd a^{\prime}
\end{align*}
donde $\laplacian{} = \Delta$. La expresión anterior es la representación integral del potencial, sinedo una representación más general de los teoremas de Green y de la ED de Poisson. La ecuación de partida es
\begin{align*}
\phi (\vb{r}) = \int_{V} \dfrac{\rho (\vb{r^{\prime}})}{\abs{\vb{r} - \vb{r^{\prime}}}} \, \dd{V^{\prime}}
\end{align*}
\item \label{p2-Greiner} Resuelve el problema del potencial para un punto con carga cerca de una esfera aterrizada ($\Phi = 0$ en la superficie)
\item \label{p3-Greiner} Resuelve el problema de dos semiesferas conductoras a diferentes potenciales: la semiesfera superior a un potencial $+V$ y la semiesfera inferior a un potencial $-V$.
%\vfill
% \bibliographystyle{unsrt}
% \bibliography{Tarea_Referencia}
\end{milista}

\end{document}