\documentclass[12pt]{article}
\usepackage[letterpaper]{geometry}
%\textwidth = 345.0pt
%\hoffset = -3cm
\usepackage[utf8]{inputenc}
\usepackage[spanish,es-tabla]{babel}
\usepackage[autostyle,spanish=mexican]{csquotes}
\usepackage{amsmath}
\usepackage{nccmath}
\usepackage{amsthm}
\usepackage{amssymb}
\usepackage{graphicx}
\usepackage{comment}
\usepackage{siunitx}
\usepackage{physics}
\usepackage{color}
\usepackage{float}
\usepackage{multicol}
%\usepackage{milista}
\usepackage{enumitem}
\usepackage{anyfontsize}
\usepackage{anysize}
\marginsize{1cm}{1cm}{1cm}{1cm}
\usepackage{enumitem}
\usepackage{capt-of}
\usepackage{bm}
\usepackage{relsize}
\newlist{milista}{enumerate}{2}
\setlist[milista,1]{label=\arabic*)}
\setlist[milista,2]{label=\arabic{milistai}.\arabic*)}
\spanishdecimal{.}
\renewcommand{\baselinestretch}{1.5}
\author{ }
\author{}
\title{Segunda parte de la Tarea - Examen del Tema 5  \\ \large{Funciones de Bessel \\ Matemáticas Avanzadas de la Física}} \vspace{-1.5\baselineskip}
\date{ }
\begin{document}
\vspace{-4cm}
%\renewcommand\theenumii{\arabic{theenumii.enumii}}
\renewcommand\labelenumii{\theenumi.{\arabic{enumii})}}
\maketitle
\fontsize{14}{14}\selectfont
\begin{enumerate}
%Referencia Arfken
\item La transición de probabilidad entre dos estados $\psi_{m}$ y $\psi_{n}$ del oscilador armónico cuántico, depende de que el valor de la siguiente integral sea:
\begin{align*}
\int_{-\infty}^{\infty} x \; e^{-x^{2}} \; H_{n} (x) \, H_{m}(x) \dd{x} = \sqrt{\pi} \; 2^{n-1} \; n! \; \delta_{m,n-1} + \sqrt{\pi} \; 2^{n} \; (n+1)! \; \delta_{m,n+1}
\end{align*}
Obtén explícitamente este valor. El resultado demuestra que cada transición puede ocurrir solamente entre estados con niveles de energía adyacentes, $m = n \pm 1$.
\item Demuestra que
\begin{align*}
\int_{-\infty}^{+\infty} x^{2} \, \exp \left( -x^{2} \right) \, H_{n} (x) \, H_{n} (x) \dd{x} = \pi^{1/2} \, 2^{n} \, n! \, \left( n + \dfrac{1}{2} \right)
\end{align*}
La integral se presenta en el cálculo del desplazamiento medio cuadrático del oscilador cuántico.
%Referencia: Sepúlveda
\item Demuestra que la expansión de $\exp(-a \, x)$ en la base $\left\{ L_{n}^{k} (x) \right\}$ en $ x \in (0, \infty)$ es
\begin{align*}
\exp(-a \, x) = \dfrac{1}{(1 + a)^{1+k}} \sum_{n=0}^{\infty} \left( \dfrac{a}{1 + a} \right)^{n} \, L_{n}^{k} (x)
\end{align*}
\item El estado normal del átomo de hidrógeno es el de más baja energía, $(n = 1)$. Le corresponde una función de onda:
\begin{align*}
\psi_{100} = \dfrac{1}{\sqrt{\pi \, a_{0}^{3}}} \, e^{-r/a_{0}}\end{align*}
donde $a_{0} = 2 \, \alpha = 4 \, \hbar^{2} \, \pi \varepsilon_{0}/ m \,q^{2}$ es el radio de Bohr. La densidad volumétrica de probabilidad de localización del electrón es $\dv*{P}{V} = \psi_{100}^{*} \, \psi_{100}$, de modo que 
\begin{align*}
\dv{P}{r} = 4 \, \pi r^{2} \, \abs{\psi_{100}}^{2}
\end{align*}
Demuestra que la distancia más probable a la que se encuentra el electrón del núcleo es $r = a_{0}$, coincidente con el radio de la primera órbita de Bohr.
\end{enumerate}
\end{document}