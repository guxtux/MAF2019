\documentclass[12pt]{article}
%\usepackage[left=0.25cm,top=1cm,right=0.25cm,bottom=1cm]{geometry}
\usepackage{geometry}
\textwidth = 20cm
\hoffset = -1cm
\usepackage[utf8]{inputenc}
\usepackage[spanish,es-tabla]{babel}
\usepackage{amsmath}
\usepackage{nccmath}
\usepackage{amsthm}
\usepackage{amssymb}
\usepackage{graphicx}
\usepackage{color}
\usepackage{float}
\usepackage{multicol}
\usepackage{enumerate}
\usepackage{anyfontsize}
\usepackage{anysize}
\usepackage{enumitem}
\usepackage{capt-of}
\usepackage{bm}
\usepackage{relsize}
\usepackage{physics}
\usepackage{empheq}
\usepackage{mathtools}
\usepackage{bigints}
\usepackage{scalerel}

\spanishdecimal{.}
\setlist[enumerate]{itemsep=0mm}

\renewcommand{\baselinestretch}{1.2}

\let\oldbibliography\thebibliography
\renewcommand{\thebibliography}[1]{\oldbibliography{#1}
\setlength{\itemsep}{0pt}}
%\marginsize{1.5cm}{1.5cm}{0cm}{2cm}
%\renewcommand\theenumii{\arabic{theenumii.enumii}}
\renewcommand\labelenumii{\theenumi.{\arabic{enumii}}}

\newcommand{\pderivada}[1]{\ensuremath{{#1}^{\prime}}}
\newcommand{\sderivada}[1]{\ensuremath{{#1}^{\prime \prime}}}
\newcommand{\tderivada}[1]{\ensuremath{{#1}^{\prime \prime \prime}}}
\newcommand{\nderivada}[2]{\ensuremath{{#1}^{(#2)}}}

\def\scaleint#1{\vcenter{\hbox{\scaleto[3ex]{\displaystyle\int}{#1}}}}
\def\scaleoint#1{\vcenter{\hbox{\scaleto[3ex]{\displaystyle\oint}{#1}}}}
\def\scaleiiint#1{\vcenter{\hbox{\scaleto[3ex]{\displaystyle\iiint}{#1}}}}
\def\bs{\mkern-12mu}

\title{Examen Parcial 1 - Primera parte \\ \large {Matemáticas Avanzadas de la Física}  \vspace{-3ex}}
\author{M. en C. Gustavo Contreras Mayén}
\date{ }

\begin{document}
\vspace{-4cm}
\maketitle
\fontsize{14}{14}\selectfont

\textbf{Indicaciones: } Deberás de resolver cada ejercicio de la manera más completa, ordenada y clara posible, anotando cada paso así como las operaciones involucradas. El puntaje de cada ejercicio es de \textbf{1 punto}.
\par
Los ejercicios que se presentan corresponden a una primera parte del examen parcial 1, que considera los temas 1 y 2, para darles oportunidad de ir avanzando en la resolución de los mismo, es por eso que se presenta en un primer documento.
\par
Al terminar el tema 2, dejaremos algunos ejercicios tanto de la tarea 2, como del examen parcial 2, así que esperando ya hayan completado estos ejercicios, no se les cargará el trabajo para resolver esos ejercicios. La recomendación es que comiencen a resolverlos a la brevedad posible.

\section{Tema 1. La geometría y física.}

\begin{enumerate}
% \item Resuelve los vectores unitarios cilíndricos circulares en sus componentes cartesianos:
% \begin{align*}
% \vb{r}_{0} &= \vu{i} \, \sin \theta \cos \varphi + \vu{j} \, \sin \theta \sin \varphi + \vu{k} \, \cos \theta \\[0.5em]
% \bm{\theta}_{0} &= \vu{i} \, \cos \theta \cos \varphi + \vu{j} \, \cos \theta \sin \varphi - \vu{k} \, \sin \theta \\[0.5em]
% \bm{\varphi}_{0} &= - \vu{i} \, \sin \varphi + \vu{j} \, \cos \varphi
% \end{align*}
% \item Un cuerpo rígido gira alrededor de un eje fijo con una velocidad angular constante $\omega$. Toma $\omega$ para situarse a lo largo del eje $z$. Expresa el vector de posición $r$ en coordenadas cilíndricas circulares:
% \begin{enumerate}
% \item Calcula $\vb{r} = \bm{\omega} \cp \vb{r}$
% \item Calcula $\curl{\vb{r}}$
% \end{enumerate}
\item Las ecuaciones de Navier-Stokes para el flujo de un fluido incompresible:
\begin{align*}
- \curl{( \mathbf{v} \cp (\curl{\vb{v}}))} =  \dfrac{\eta}{\rho_{0}} \, \laplacian{(\curl{\vb{v}})}
\end{align*}
Donde $\eta$ es la viscosidad y $\rho_{0}$ la densidad del fluido. Para un flujo axial dentro de un cilindro, consideremos que la velocidad $\mathbf{v}$ está dada por:
\begin{align*}
\mathbf{v} =  \vu{z} \, v (\rho)
\end{align*}
Considera que $\curl{( \mathbf{v} \cp (\curl{\vb{v}}))} = 0$ para este valor de $\mathbf{v}$.
\\
Demuestra que:
\begin{align*}
\laplacian{( \curl{\vb{v}} )} = 0
\end{align*}
nos lleva a la ecuación diferencial:
\begin{align*}
\dfrac{1}{\rho} \, \dv{\rho} \left( \rho \, \dv[2]{v}{\rho} \right) -  \dfrac{1}{\rho^{2}} \, \dv{v}{\rho} = 0
\end{align*}
y que la siguiente expresión es solución de la misma ecuación diferencial:
\begin{align*}
v = v_{0} + a_{2} \, \rho^{2}
\end{align*}
\item Escribe los operadores diferenciales: $\grad{\varphi}$, $\div{\vb{A}}$, $\curl{\vb{A}}$, $\laplacian{\varphi}$ en coordenadas esferoidales prolatas, calculando los respectivos factores de escala.
\item Los siguientes ejercicios deberás de resolverlos ocupando las funciones Gamma y Beta.
\begin{enumerate}
\item Demuestra que:
\begin{align*}
\mathlarger{\Gamma} \left( \dfrac{1}{2} - n \right) \, \left( \dfrac{1}{2} + n \right) = (-1)^{n} \, \pi
\end{align*}
donde $n$ es un entero.
%Ref. II-19
\item \label{ejercicio_09} Evalúa la integral:
\begin{align*}
\scaleint{5ex}_{\bs 0}^{1} \dfrac{x^{5}}{\sqrt[3]{1 - x^{4}}} \dd{x}
\end{align*}
%Ref Farrell II-7
\item \label{ejercicio_10} Demuestra que:
\begin{align*}
\scaleint{5ex}_{\bs 0}^{1} x^{m} \, \bigg[ \ln \left( \dfrac{1}{x} \right) \bigg]^{n} \dd{x} = \dfrac{\Gamma(n + 1)}{(m + 1)^{n+1}} \hspace{1.5cm} m > -1, n > -1
\end{align*}
\end{enumerate}
\end{enumerate}

\section{Tema 2. Primeras técnicas de solución.}

\begin{enumerate}
% \item Demostrar que la ecuación de Helmholtz
% \begin{align*}
% \laplacian \psi + \left[ k^{2} + f(\rho) + \left( \dfrac{1}{\rho^{2}} \right) \: g(\varphi) + h(z) \right] \: \psi = 0
% \end{align*}
% es separable en coordenadas cilíndricas circulares.
% \item Demostrar que
% \begin{align*}
% \laplacian \psi (r, \theta, \varphi) + \left[ k^{2} + f(r) + \dfrac{1}{r^{2}} \: g(\theta) + \dfrac{1}{r^{2} \sin^{2}} \: h(\varphi) \right] \: \psi (r, \theta, \varphi) = 0
% \end{align*}
% es separable en coordenadas esféricas. Las funciones $f, g, h$ son funciones sólo de las variables indicadas, $k^{2}$ es constante.
\item Usando el \textbf{método de separación de variables (MSP)}:
\begin{enumerate}
\item Determina una solución al problema de una cuerda vibrando con los extremos fijos:
\begin{align*}
u_{tt} - c^{2} \, u_{xx} = 0 \hspace{1.5cm} \text{en } 0 < x < L, \quad t > 0
\end{align*}
sujeta a las siguientes condiciones de frontera e iniciales:
\begin{align*}
    u (0, t) &= u (L, t) = 0 \hspace{1.5cm} t \geq 0 \\
    u (x, 0) &= f (x) \hspace{1.5cm} 0 \leq x \leq L \\
    u_{t} (x, 0) &= g (x) \hspace{1.5cm} 0 \leq x \leq L    
\end{align*}
Demuestra que la solución obtenida representa la superposición de una onda desplazándose hacia adelante y hacia atrás.
%Ref. Pinchover 5.5b
\item Resuelve con el MSP la ecuación de calor:
\begin{align*}
    u_{t} = 12 \, u_{xx} \hspace{1.5cm} \text{en } 0 < x < \pi, t > 0
\end{align*}
sujeto a las siguientes condiciones de frontera e iniciales:
\begin{align*}
    u_{x} (0, t) &= u_{x} (\pi, t) = 0 \hspace{1.5cm} t \geq 0, \\
    u (x, 0) &= 1 + \sin^{3} x \hspace{1.5cm} 0 \leq x \leq \pi
\end{align*}
\end{enumerate}
\item Para el caso especial en donde no hay dependencia en la coordenada azimutal, del estudio del ion molecular del hidrógeno $(H2^{+})$ en mecánica cuántica, se llega a la ecuación
\begin{align*}
\dv{\eta} \left[ (1 - \eta^{2} ) \dv{u}{\eta} \right] + \alpha \, u + \beta \, \eta^{2} \, u = 0
\end{align*}
Desarrolla una solución en series de potencias para $u(\eta)$. Evalúa los primeros tres coeficientes no nulos en términos de $a_{0}$. Demuestra que la ecuación de índices es $k \, (k - 1) = 0$
\\
La expresión a la que debes de llegar es:
\begin{align*}
u_{k=1} &=  a_{0} \: \eta \left\lbrace 1 - \dfrac{2- \alpha}{6} \, \eta^{2} + \left[ \dfrac{(2-\alpha)(12-\alpha)}{120} - \dfrac{\beta}{20} \right] \eta^{4} + \ldots \right\rbrace
\end{align*}
\item Demuestra que las siguientes ecuaciones diferenciales tienen singularidades en los puntos que se indican:
\begin{table}[H]
\centering
\begin{tabular}{l c c}
 & Sing. regular & Sing. irregular \\
Hipergeométrica & & \\
$x (x-1) \sderivada{y} + [(1 + a + b) x - c] \pderivada{y} + a b y = 0$ & $0, 1$ & $--$ \\
Chevyshev & & \\
$(1 - x^{2}) \sderivada{y} - x \pderivada{y} + n^{2} y = 0$ & $- 1, 1$ & $--$ \\
Laguerre & & \\
$x \sderivada{y} + (1 - x) \pderivada{y} + a y = 0$ & $0$ & $\infty$
\end{tabular}
\end{table}
% \item Una solución para la ecuación de Chebychev
% \begin{align*}
% (1 -x^{2}) \, \sderivada{y} - x \, \pderivada{y} + n^{2} \, y = 0
% \end{align*}
% para $n = 1$ es $y_{1}(x) = x$. Ocupando la doble integral del Wronskiano calcula la segunda solución $y_{2}(x)$.
% \item Desarrolla los tres primeros términos no nulos en la serie de potencias en torno a $x = 0$, para la solución general en la siguiente ecuación, con $x > 0$:
% \begin{align*}
% x^{2} \, \sderivada{y} + x \, \pderivada{y} + x^{2} \, y = 0
% \end{align*}
% Recuerda que debes de obtener la primera solución mediante el método de Frobenius y para la segunda solución, considera de la ecuación de índices, si las raíces $r_{1}$ y $r_{2}$ son tales que:
% \begin{itemize}
% \item Si $r_{1} = r_{2}$ entonces la segunda solución linealmente independiente es de la forma:
% \begin{align*}
% y_{2} (x) = y_{1} \, \ln(x) + \sum_{n=1}^{\infty} b_{n} \, x^{n+r_{1}}
% \end{align*} 
% \item Si $r_{1} - r_{2}$ es un entero positivo, entonces la segunda solución linealmente independiente es de la forma:
% \begin{align*}
% y_{2} (x) = C \, y_{1} \, \ln(x) + \sum_{n=0}^{\infty} b_{n} \, x^{n+r_{2}} \hspace{1cm} b_{0} \neq 0
% \end{align*}
% donde $C$ es una constante que podría anularse.
% \end{itemize}
\end{enumerate}
\end{document}