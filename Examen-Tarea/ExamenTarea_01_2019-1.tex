\documentclass[12pt]{article}
\usepackage[utf8]{inputenc}
\usepackage[spanish,es-lcroman, es-tabla]{babel}
\usepackage[autostyle,spanish=mexican]{csquotes}
\usepackage{amsmath}
\usepackage{amssymb}
\usepackage{nccmath}
\numberwithin{equation}{section}
\usepackage{amsthm}
\usepackage{graphicx}
\usepackage{epstopdf}
\DeclareGraphicsExtensions{.pdf,.png,.jpg,.eps}
\usepackage{color}
\usepackage{float}
\usepackage{multicol}
\usepackage{enumerate}
\usepackage[shortlabels]{enumitem}
\usepackage{anyfontsize}
\usepackage{anysize}
\usepackage{array}
\usepackage{multirow}
\usepackage{enumitem}
\usepackage{cancel}
\usepackage{tikz}
\usepackage{circuitikz}
\usepackage{tikz-3dplot}
\usetikzlibrary{babel}
\usetikzlibrary{shapes}
\usepackage{bm}
\usepackage{mathtools}
\usepackage{esvect}
\usepackage{hyperref}
\usepackage{relsize}
\usepackage{siunitx}
\usepackage{physics}
%\usepackage{biblatex}
\usepackage{standalone}
\usepackage{mathrsfs}
\usepackage{bigints}
\usepackage{bookmark}
\spanishdecimal{.}

\setlist[enumerate]{itemsep=0mm}

\renewcommand{\baselinestretch}{1.5}

\let\oldbibliography\thebibliography

\renewcommand{\thebibliography}[1]{\oldbibliography{#1}

\setlength{\itemsep}{0pt}}
%\marginsize{1.5cm}{1.5cm}{2cm}{2cm}


\newtheorem{defi}{{\it Definición}}[section]
\newtheorem{teo}{{\it Teorema}}[section]
\newtheorem{ejemplo}{{\it Ejemplo}}[section]
\newtheorem{propiedad}{{\it Propiedad}}[section]
\newtheorem{lema}{{\it Lema}}[section]

\usepackage{standalone}
\usepackage{enumerate}
\usepackage[left=1.5cm,top=1.5cm,right=1.5cm,bottom=1.5cm]{geometry}
\title{Tarea Examen 1 - (Temas 1 y 2) \\ \large{Matemáticas Avanzadas de la Física}}
\date{ }
\begin{document}
\vspace{-4cm}
%\renewcommand\theenumii{\arabic{theenumii.enumii}}
\renewcommand\labelenumii{\theenumi.{\arabic{enumii}}}
\maketitle
\fontsize{14}{14}\selectfont
%Fecha de entrega: \underline{\textbf{Miércoles 12 de septiembre de 2018.}}
\begin{enumerate}
\item Partiendo del Lagrangiano de una partícula libre en coordenadas esféricas:
\[ L = \dfrac{m}{2} \left( \dot{r}^{2} + r^{2} \, \dot{\theta}^{2} + r^{2} \, \sin \theta^{2} \, \dot{\varphi}^{2} \right)\]
Calcula los símbolos de Christoffel.
\item Las ecuaciones de Navier-Stokes para el flujo de un fluido incompresible
\[ - \bm{\nabla} \times ( \mathbf{v} \times (\bm{\nabla} \times \mathbf{v} )) =  \dfrac{\eta}{\rho_{0}} \bm{\nabla}^{2} (\bm{\nabla} \times \mathbf{v}) \]
Donde $\eta$ es la viscosidad y $\rho_{0}$ la densidad del fluido. Para un flujo axial dentro de un cilindro, consideremos que la velocidad $\mathbf{v}$ está dada por
\[ \mathbf{v} =  \mathbf{k} \, v (\rho) \]
Considera que
\[ \bm{\nabla} \times (\mathbf{v} \times (\bm{\nabla} \times \mathbf{v})) = 0 \]
para este valor de $\mathbf{v}$.
\\
Demuestra que
\[ \bm{\nabla}^{2} ( \bm{\nabla} \times \mathbf{v}) = 0  \]
nos lleva a la ecuación diferencial
\[ \dfrac{1}{\rho} \dfrac{d}{d \rho} \left( \rho \dfrac{d^{2} v}{d \rho^{2}} \right) -  \dfrac{1}{\rho^{2}} \dfrac{d v}{d \rho} = 0 \]
y que la siguiente expresión es solución de la misma ecuación diferencial
\[ v = v_{0} + a_{2} \rho^{2} \]
\item El cálculo de efecto \enquote{pinch} en magnetohidrodinámica, involucra la evaluación de $(\mathbf{B} \cdot \bm{\nabla}) \mathbf{B}$. Si la inducción magnética $\mathbf{B}$ se toma como $\mathbf{B} = \bm{\varphi}_{0} \, B_{\varphi} (\rho)$, demostrar que
\[ (\mathbf{B} \cdot \bm{\nabla}) \mathbf{B} = - \dfrac{\bm{\rho_{0}} \: B_{\varphi}^{2}}{\rho}	 \]
\item Un cierto campo de fuerza está dado por
\[ \mathbf{F} = \mathbf{r}_{0} \dfrac{2 \, P \, \cos \theta}{r^{3}} + \bm{\theta}_{0} \dfrac{P}{r^{3}} \sin \theta, \hspace{1.5cm} r \geq P/2 \]
en coordenadas esféricas.
\begin{enumerate}
\item Revisa $\bm{\nabla} \times \mathbf{F}$ para checar si existe un potencial.
\item Calcular $\mathlarger{\oint} \mathbf{F} \cdot d \bm{\lambda}$ para un círculo unitario en el plano $\theta = \pi/2$.\
¿Qué nos dice esto sobre la fuerza?¿Es conservativa o no conservativa?
\item Si considera que $\mathbf{F}$ se puede describir por $\mathbf{F} = - \bm{\nabla}\psi$, encuentra $\psi$. De otra manera argumenta que no es posible que un potencial exista.
\end{enumerate}
\item Usando las ecuaciones de Maxwell, demostrar que
\[ \dfrac{d}{dt} ( P_{\text{mec}} + P_{\text{campo}} )_{\alpha} = \sum_{\beta} \int_{V} \dfrac{\partial}{\partial x_{\beta}} T_{\alpha \beta} \; d^{3} x \]
con
\[ T_{\alpha \beta} =  \dfrac{1}{4 \pi} \left[ E_{\alpha} \, E_{\beta} + B_{\alpha} \, B_{\beta} - \dfrac{1}{2} (\mathbf{E} \cdot \mathbf{E} + \mathbf{B} \cdot \mathbf{B}) \delta_{\alpha \beta} \right]  \]
donde
\[ P_{\text{campo}} = \dfrac{1}{4 \pi c} \int_{V} (\mathbf{E} \times \mathbf{B}) \; d^{3} x \]
Nota: el ejercicio está hecho en el libro \emph{Classical Electrodynamics}, de John Jackson, en las secciones 6.8 y 6.9. En esencia la actividad consiste en el manejo de las ecuaciones de Maxwell (divergencias, rotacionales, etc.) y usar la notación de índices que se revisó en clase.
\item Demostrar que la ecuación de Helmholtz
\[ \nabla^{2} \psi + k^{2} \: \psi = 0 \]
es separable en coordenadas cilíndricas circulares si $k^{2}$ se generaliza como
\[ k^{2} + f(\rho) + \left( \dfrac{1}{\rho^{2}} \right) g(\varphi) + h(z) \]
\item Demostrar que
\[ \nabla^{2} \psi (r, \theta, \varphi) + \left[ k^{2} + f(r) + \dfrac{1}{r^{2}} \, g(\theta) + \dfrac{1}{r^{2} \, \sin^{2}} \: h(\varphi) \right] \psi (r, \theta, \varphi) = 0 \]
es separable en coordenadas esféricas. Las funciones $f,g,h$ son funciones sólo de las variables indicadas, $k^{2}$ es constante.
% \item Para una esfera sólida homogénea con constante de difusión términa $K$, la ecuación de conducción de calor (sin fuentes) es
% \[ \dfrac{\partial T(r,t)}{\partial t} =  K \nabla^{2} T(r,t) \]
% Mediante la técnica de separación de variables, suponemos que tiene una solución de la forma
% \[ T =R(r) T(t) \]
% Demuestra que la ecuación radial toma la forma estándar
% \[ r^{2} \dfrac{d^{2} R}{d r^{2}} + 2r \dfrac{d R}{d r} + \left[ \alpha^{2} r^{2} - n(n+1) \right] R = 0, \hspace{1cm} n = \mbox{ entero} \]
\item Utiliza el método de Frobenius para obtener la solución general de cada una de las siguientes ecuaciones diferenciales, para un entorno de $x = 0$:
\begin{enumerate}[label=\alph*)]
\begin{fleqn}
\setlength\itemsep{1em}
\item  $ 2 \, x \, \dfrac{d^{2} y}{d x^{2}} + (1 - x^{2}) \, \dfrac{d y}{d x} - y = 0 $
\item $ x^{2} \, \dfrac{d^{2} y}{d x^{2}} + x \, \dfrac{d y}{d x} + (x^{2} - 1) \, y = 0 $
\end{fleqn}
\end{enumerate}
\item Una solución a la ecuación diferencial de Laguerre
\[ x \: y^{\prime \prime} + (1 - x) \, y^{\prime} + n \: y = 0\]
para $n = 0$ es $y_{1}(x) = 1$. Desarrolla una segunda solución linealmente independiente.
\item A partir del estudio en mecánica cuántica del efecto Stark (en coordenadas parabólicas), nos conduce a la ecuación difencial
\[ \dfrac{d}{d \xi} \left( \xi \: \dfrac{d u}{d \xi} \right) + \left( \dfrac{1}{2} \, E \, \xi + \alpha - \dfrac{m^{2}}{4 \, \xi} - \dfrac{1}{4} \, F \,\xi^{2} \right) u = 0 \]
donde
\begin{enumerate}[label=\roman*)]
\item $\alpha$ es la constante de separación.
\item $E$ es la energía total del sistema.
\item $F$ es una constante.
\item $Fz$ es la energía potencial que se agrega al introducir un campo eléctrico.
\end{enumerate}
Usando la raíz más grande de la ecuación de índices, desarrolla una solución en series de potencias, alrededor de $\xi = 0$. Evalúa los primeros tres coeficientes en términos de $a_{0}$
\[  \begin{split}
& \text{Ecuación indicial } \hspace{1.5cm} k^{2} - \dfrac{m^{2}}{4} = 0 \\
u(\xi) &=  a_{0} \, \xi^{m/2} \left\lbrace 1 - \dfrac{\alpha}{m+1} \xi + \left[ \dfrac{\alpha^{2}}{2(m+1)(m+2)} - \dfrac{E}{4(m+2)} \right] \xi^{2} + \ldots \right\rbrace
\end{split} \]
Checa que la perturbación $F$ no se presenta hasta que el término $a_{3}$ se incluye.
\item Para el caso especial en donde no hay dependencia en la coordenada azimutal, del estudio del ion molecular del hidrógeno $(H2^{+})$ en mecánica cuántica, se llega a la ecuación
\[ \dfrac{d}{d \eta} \left[ (1 - \eta^{2} ) \dfrac{d u}{d \eta} \right] + \alpha \, u + \beta \, \eta^{2} \, u = 0 \]
Desarrolla una solución en series de potencias para $u(\eta)$. Evalúa los primeros tres coeficientes no nulos en términos de $a_{0}$
\[  \begin{split}
& \text{Ecuación indicial } \hspace{1.5cm} k(k-1) = 0 \\
u_{k=1} &=  a_{0} \: \eta \left\lbrace 1 - \dfrac{2- \alpha}{6} \, \eta^{2} + \left[ \dfrac{(2-\alpha)(12-\alpha)}{120} - \dfrac{\beta}{20} \right] \eta^{4} + \ldots \right\rbrace
\end{split} \] 

\item Inicia con 
\[ J_{0} (x) = 1 - \dfrac{x^{2}}{4} +  \dfrac{x^{4}}{64} -  \dfrac{x^{6}}{2304} + \ldots \]
Obtén la segunda solución linealmente independiente
\[ y_{2}(x) = J_{0}(x) ln(x) + \dfrac{x^{2}}{4} -  \dfrac{3 x^{4}}{128} +  \dfrac{11 x^{6}}{13284}+ \ldots \]
de la ecuación de Bessel de orden cero.
\end{enumerate}

\end{document}