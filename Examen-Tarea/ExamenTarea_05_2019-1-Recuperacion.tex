\documentclass[12pt]{article}
\usepackage[utf8]{inputenc}
\usepackage[spanish,es-lcroman, es-tabla]{babel}
\usepackage[autostyle,spanish=mexican]{csquotes}
\usepackage{amsmath}
\usepackage{amssymb}
\usepackage{nccmath}
\numberwithin{equation}{section}
\usepackage{amsthm}
\usepackage{graphicx}
\usepackage{epstopdf}
\DeclareGraphicsExtensions{.pdf,.png,.jpg,.eps}
\usepackage{color}
\usepackage{float}
\usepackage{multicol}
\usepackage{enumerate}
\usepackage[shortlabels]{enumitem}
\usepackage{anyfontsize}
\usepackage{anysize}
\usepackage{array}
\usepackage{multirow}
\usepackage{enumitem}
\usepackage{cancel}
\usepackage{tikz}
\usepackage{circuitikz}
\usepackage{tikz-3dplot}
\usetikzlibrary{babel}
\usetikzlibrary{shapes}
\usepackage{bm}
\usepackage{mathtools}
\usepackage{esvect}
\usepackage{hyperref}
\usepackage{relsize}
\usepackage{siunitx}
\usepackage{physics}
%\usepackage{biblatex}
\usepackage{standalone}
\usepackage{mathrsfs}
\usepackage{bigints}
\usepackage{bookmark}
\spanishdecimal{.}

\setlist[enumerate]{itemsep=0mm}

\renewcommand{\baselinestretch}{1.5}

\let\oldbibliography\thebibliography

\renewcommand{\thebibliography}[1]{\oldbibliography{#1}

\setlength{\itemsep}{0pt}}
%\marginsize{1.5cm}{1.5cm}{2cm}{2cm}


\newtheorem{defi}{{\it Definición}}[section]
\newtheorem{teo}{{\it Teorema}}[section]
\newtheorem{ejemplo}{{\it Ejemplo}}[section]
\newtheorem{propiedad}{{\it Propiedad}}[section]
\newtheorem{lema}{{\it Lema}}[section]

\usepackage{standalone}
\usepackage{enumerate}
\usepackage[left=1.5cm,top=1.5cm,right=1.5cm,bottom=1.5cm]{geometry}
\title{Tarea Recuperación \\ \large{Matemáticas Avanzadas de la Física}}
\date{ }
\begin{document}
\maketitle
\vspace{-2cm}
Fecha de entrega: \underline{\textbf{Jueves 29 de noviembre de 2018.}}
\par
\textbf{Consideraciones importantes: } Este examen tiene el objetivo de recuperar puntaje en los exámenes parciales del curso, toma en cuenta de que en caso de aceptar y entregar las soluciones a los problemas, deberás de entregar \underline{todos los problemas resueltos}, si en tu caso tienes un examen parcial no aprobado, tienes la opción de presentar sólo la reposición de ese examen.
\par
Si entregas todo el examen de recuperación, podrás agregar hasta $1.2$ puntos más a los cuatro exámenes parciales (suponiendo que la solución es correcta para cada problema, por lo que es una buena ventaja el que entregues todo el examen.
\par
Entrega el examen de manera ordenada y lo más claro y completo posible.
\begin{enumerate}
\item Tema 1. Geometría y Física.
\par
Coordenadas cilíndricas elípticas.
\par
Se construyen partiendo de una familia de elipses confocales
\begin{align*}
\dfrac{x^{2}}{A^{2}} +  \dfrac{y^{2}}{A^{2} - a^{2}} =  1 \hspace{1cm} A \geq a
\end{align*}
La familia de curvas ortogonales a las elipses en cada punto es una familia de hipérbolas confocales de la forma
\begin{align*}
\dfrac{x^{2}}{C^{2}} - \dfrac{y^{2}}{a^{2} - C^{2}} = 1 \hspace{1cm} C \leq a
\end{align*}
Las \textbf{coordenadas cilíndricas elípticas} $(\xi, \eta, z)$ se obtienen haciendo $A = a \, \cosh \xi$ y $C = a \, \sen \eta$.
\par
Las superficies coordenadas son:
\begin{enumerate}[label=\alph*)]
\item Cilíndricos elípticos $(\xi = \mbox{ constante})$.
\item Cilíndros hiperbóloicos $(\eta = \mbox{ constante})$.
\item Planos perpendiculares al eje $z$.
\end{enumerate}
Las correspondientes reglas de transformación son:
\begin{align*}
x &= a \, \cosh \xi \, \cos \eta \\
y &= a \, \sinh \xi \, \sin \eta \\
z &= z
\end{align*}
con $\xi: 0 \to \infty, \eta: 0 \to 2 \, \pi, z: -\infty \to \infty$
\begin{enumerate}[label=\roman*.]
\item Escribe los símbolos de Christoffel de este sistema coordenado.
\item Calcula $h_{\xi}, h_{\eta}, h_{z}$
\item Escribe los operadores $\grad{\varphi}, \div{\vb{A}}, \curl \vb{A}, \laplacian{\varphi}$
\end{enumerate}
\item Tema 2. Primeras técnicas de solución.
\par
Sea $u(r, \theta, \phi)$ un potencial en coordenadas esféricas y el potencial en una superfice esférica de radio unitario, tal que $u(1, \phi, \theta) = U / \sqrt{2}$ tal que para $0 \leq \theta \leq \dfrac{\pi}{4}$ y $u(1, \phi, \theta) = U \sin \theta$ para $\dfrac{\pi}{4} \leq \theta \leq \pi$. Encuentra el potencial dentro de la superficie esférica.
\item Tema 3. Completes.
\par
Determina los valores propios y las funciones propias de las siguientes ecuaciones diferenciales. Verifica explícitamente la ortogonalidad. ¿Cuál es el conjunto ortonormalizado?
\begin{enumerate}[label=\roman*.]
\item $x^{2} \, \ddot{y} + a \, x \, \dot{y} + \lambda \, y = 0, \hspace{3.5cm} y(1) = y(e) = 0$
\item $x \, \ddot{y} + \dot{y} + \dfrac{\lambda}{x} \, y = 0 \hspace{4.3cm} y(1) = y(2) = 0$
\item $(3 + x)^{2} \, \ddot{y} + 2 \, (3 + x) \, \dot{y} + \lambda \, y = 0 \hspace{1.5cm} y(-2) = y(1) = 0$
\end{enumerate}
\item Tema 4. Funciones Gamma y Beta.
\par
\begin{enumerate}
\item Demostrar que para un valor de $n$ entero positivo
\begin{align*}
\Gamma(n + \dfrac{1}{2}) = \dfrac{1 \cdot 3 \cdot 5 \ldots (2 \, n - 1)}{2^{n}} \sqrt{\pi} = \dfrac{(2 \, n)!}{4^{n} \, n!} \, \sqrt{\pi}
\end{align*}
\item Demuestra que $B(n, n) = \dfrac{B \left( n, \dfrac{1}{2}\right)}{2^{2n -1}}$
\end{enumerate}
\item Tema 5. Separación de variables en coordenadas esféricas.
\par
Una partícula atómica está confinada dentro de un cascarón esférico de radio $R$. La partícula es descrita por una función de onda que satisface la ecuación de Schrödinger
\begin{align*}
- \dfrac{\hbar^{2}}{2 \, m} \, \laplacian{\psi} =  E \, \psi
\end{align*}
con la condición de que $\psi$ sea nula sobre las paredes.
\par
Demuestra que:
\begin{enumerate}[label=\roman*.]
\item Los niveles de energía permitidos tienen la forma:
\begin{align*}
E_{l n} = \dfrac{\alpha_{l n}^{2} \, \hbar^{2}}{2 \, m \, R^{2}}
\end{align*}
donde $\alpha_{l n}$ son los ceros de la función de Bessel esférica $j_{l}(x)$, es decir, los ceros de $J_{l + 1/2}(x)$.
\item La función de onda es
\begin{align*}
\psi = \sum_{l=0}^{\infty} \sum_{m=-l}^{l} A_{l m} \, j_{l}(\alpha_{l m } \, r / R) \, Y_{l m}(\theta, \varphi)
\end{align*}
\end{enumerate}
\item Tema 6. Funciones especiales.
\par
Demuestra que
\begin{enumerate}[label=\roman*.]
\item $\displaystyle x^{2 r} = \dfrac{(2 \, r)!}{2^{2 r}} \, \sum_{n=0}^{\infty} \dfrac{H_{2 n}(x)}{(2 \, n)! \, (r - n)!}$
\item La expansión de $\exp(-a \, x)$ en la base $L_{n}^{k} (x)$ en $x:(0, \infty)$ es
\begin{align*}
e^{-a x} = \dfrac{1}{(1 + a)^{1 + k}} \, \sum_{n=0}^{\infty} \left( \dfrac{a}{1 + a} \right)^{n} \, L_{n}^{k}(x)
\end{align*}
\end{enumerate}
\item Tema 7. Transformadas integrales.
\par
\begin{enumerate}
\item Encuentra la transformada de Fourier de un pulso triangular
\begin{align*}
f(x) = \begin{cases}
h \, (1- a \, \abs{x}) & \abs{x} < \dfrac{1}{a} \\
0 & \abs{x} > \dfrac{1}{a}
\end{cases}
\end{align*}
\item Desarrolla la función $F(s) =  s^{-1/2} \, \exp(-1/s)$ en potencias de $s^{-1}$ para mostrar que
\begin{align*}
\mathscr{L} \left\{ \dfrac{1}{\sqrt{s}} \, e^{-1/s} \right\} = \dfrac{1}{\sqrt{\pi \, t}} \, \cos 2 \, \sqrt{t}
\end{align*}
\end{enumerate}
\end{enumerate}
\end{document}