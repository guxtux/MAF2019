\documentclass[hidelinks,12pt]{article}
\usepackage[left=0.25cm,top=1cm,right=0.25cm,bottom=1cm]{geometry}
%\usepackage[landscape]{geometry}
\textwidth = 20cm
\hoffset = -1cm
\usepackage[utf8]{inputenc}
\usepackage[spanish,es-tabla]{babel}
\usepackage[autostyle,spanish=mexican]{csquotes}
\usepackage[tbtags]{amsmath}
\usepackage{nccmath}
\usepackage{amsthm}
\usepackage{amssymb}
\usepackage{mathrsfs}
\usepackage{graphicx}
\usepackage{subfig}
\usepackage{standalone}
\usepackage[outdir=./Imagenes/]{epstopdf}
\usepackage{siunitx}
\usepackage{physics}
\usepackage{color}
\usepackage{float}
\usepackage{hyperref}
\usepackage{multicol}
%\usepackage{milista}
\usepackage{anyfontsize}
\usepackage{anysize}
%\usepackage{enumerate}
\usepackage[shortlabels]{enumitem}
\usepackage{capt-of}
\usepackage{bm}
\usepackage{relsize}
\usepackage{placeins}
\usepackage{empheq}
\usepackage{cancel}
\usepackage{wrapfig}
\usepackage[flushleft]{threeparttable}
\usepackage{makecell}
\usepackage{fancyhdr}
\usepackage{tikz}
\usepackage{bigints}
\usepackage{scalerel}
\usepackage{pgfplots}
\usepackage{pdflscape}
\pgfplotsset{compat=1.16}
\spanishdecimal{.}
\renewcommand{\baselinestretch}{1.5} 
\renewcommand\labelenumii{\theenumi.{\arabic{enumii}})}
\newcommand{\ptilde}[1]{\ensuremath{{#1}^{\prime}}}
\newcommand{\stilde}[1]{\ensuremath{{#1}^{\prime \prime}}}
\newcommand{\ttilde}[1]{\ensuremath{{#1}^{\prime \prime \prime}}}
\newcommand{\ntilde}[2]{\ensuremath{{#1}^{(#2)}}}

\newtheorem{defi}{{\it Definición}}[section]
\newtheorem{teo}{{\it Teorema}}[section]
\newtheorem{ejemplo}{{\it Ejemplo}}[section]
\newtheorem{propiedad}{{\it Propiedad}}[section]
\newtheorem{lema}{{\it Lema}}[section]
\newtheorem{cor}{Corolario}
\newtheorem{ejer}{Ejercicio}[section]

\newlist{milista}{enumerate}{2}
\setlist[milista,1]{label=\arabic*)}
\setlist[milista,2]{label=\arabic{milistai}.\arabic*)}
\newlength{\depthofsumsign}
\setlength{\depthofsumsign}{\depthof{$\sum$}}
\newcommand{\nsum}[1][1.4]{% only for \displaystyle
    \mathop{%
        \raisebox
            {-#1\depthofsumsign+1\depthofsumsign}
            {\scalebox
                {#1}
                {$\displaystyle\sum$}%
            }
    }
}
\def\scaleint#1{\vcenter{\hbox{\scaleto[3ex]{\displaystyle\int}{#1}}}}
\def\bs{\mkern-12mu}


\geometry{top=1.25cm, bottom=1.5cm, left=1.25cm, right=0.8cm}
%\usepackage{showframe}
\title{Examen Tarea Temas 1 y 2 \\ \large {Matemáticas Avanzadas de la Física}  \vspace{-3ex}}
\author{M. en C. Gustavo Contreras Mayén}
\date{ }
\begin{document}
\vspace{-4cm}
\maketitle
\fontsize{14}{14}\selectfont
\textbf{Indicaciones: } Deberás de resolver cada ejercicio de la manera más completa, ordenada y clara posible, anotando cada paso así como las operaciones involucradas. El puntaje de cada ejercicio es de \textbf{1 punto}.
\par
La fecha límite para enviar este examen-tarea es el día 16 de noviembre de 2020 a las 12 del día, ya sea mediante la plataforma Moodle, subiendo un archivo pdf que no exceda de 10 MB, o compartiendo el archivo en una carpeta en Drive de Gmail, recuerda que puedes digitalizar las hojas que hayas ocupado o tomar una fotografía clara y que abarque toda la hoja.
\section{Tema 1. La geometría y la física.}
\begin{enumerate}
\item Resuelve los vectores unitarios cilíndricos circulares en sus componentes cartesianos:
\begin{align*}
\vb{r}_{0} &= \vu{i} \, \sin \theta \cos \varphi + \vu{j} \, \sin \theta \sin \varphi + \vu{k} \, \cos \theta \\[0.5em]
\bm{\theta}_{0} &= \vu{i} \, \cos \theta \cos \varphi + \vu{j} \, \cos \theta \sin \varphi - \vu{k} \, \sin \theta \\[0.5em]
\bm{\varphi}_{0} &= - \vu{i} \, \sin \varphi + \vu{j} \, \cos \varphi
\end{align*}
\item Un cuerpo rígido gira alrededor de un eje fijo con una velocidad angular constante $\omega$. Toma $\omega$ para situarse a lo largo del eje $z$. Expresa el vector de posición $r$ en coordenadas cilíndricas circulares:
\begin{enumerate}
\item Calcula $\vb{r} = \bm{\omega} \cp \vb{r}$
\item Calcula $\curl{\vb{r}}$
\end{enumerate}
\item Las ecuaciones de Navier-Stokes para el flujo de un fluido incompresible
\begin{align*}
- \curl{( \mathbf{v} \cp (\curl{\vb{v}}))} =  \dfrac{\eta}{\rho_{0}} \, \laplacian{(\curl{\vb{v}})}
\end{align*}
Donde $\eta$ es la viscosidad y $\rho_{0}$ la densidad del fluido. Para un flujo axial dentro de un cilindro, consideremos que la velocidad $\mathbf{v}$ está dada por
\begin{align*}
\mathbf{v} =  \vu{z} \, v (\rho)
\end{align*}
Considera que $\curl{( \mathbf{v} \cp (\curl{\vb{v}}))} = 0$ 
para este valor de $\mathbf{v}$.
\\
Demuestra que
\begin{align*}
\laplacian{( \curl{\vb{v}} )} = 0
\end{align*}
nos lleva a la ecuación diferencial
\begin{align*}
\dfrac{1}{\rho} \, \dv{\rho} \left( \rho \, \dv[2]{v}{\rho} \right) -  \dfrac{1}{\rho^{2}} \, \dv{v}{\rho} = 0
\end{align*}
y que la siguiente expresión es solución de la misma ecuación diferencial
\begin{align*}
v = v_{0} + a_{2} \, \rho^{2}
\end{align*}
\item Escribe los operadores diferenciales: $\grad{\varphi}$, $\div{\vb{A}}$, $\curl{\vb{A}}$, $\laplacian{\varphi}$ en coordenadas esferoidales prolatas, calculando los respectivos factores de escala.
\item Demuestra que:
\begin{align*}
\mathlarger{\Gamma} \left( \dfrac{1}{2} - n \right) \, \left( \dfrac{1}{2} + n \right) = (-1)^{n} \, \pi
\end{align*}
donde $n$ es un entero.
\end{enumerate}
\section{Tema 2. Primeras técnicas de solución.}
\begin{enumerate}
\item Demostrar que la ecuación de Helmholtz
\begin{align*}
\laplacian \psi + \left[ k^{2} + f(\rho) + \left( \dfrac{1}{\rho^{2}} \right) \: g(\varphi) + h(z) \right] \: \psi = 0
\end{align*}
es separable en coordenadas cilíndricas circulares.
\item Demostrar que
\begin{align*}
\laplacian \psi (r, \theta, \varphi) + \left[ k^{2} + f(r) + \dfrac{1}{r^{2}} \: g(\theta) + \dfrac{1}{r^{2} \sin^{2}} \: h(\varphi) \right] \: \psi (r, \theta, \varphi) = 0
\end{align*}
es separable en coordenadas esféricas. Las funciones $f, g, h$ son funciones sólo de las variables indicadas, $k^{2}$ es constante.
\item Para el caso especial en donde no hay dependencia en la coordenada azimutal, del estudio del ion molecular del hidrógeno $(H2^{+})$ en mecánica cuántica, se llega a la ecuación
\begin{align*}
\dv{\eta} \left[ (1 - \eta^{2} ) \dv{u}{\eta} \right] + \alpha \, u + \beta \, \eta^{2} \, u = 0
\end{align*}
Desarrolla una solución en series de potencias para $u(\eta)$. Evalúa los primeros tres coeficientes no nulos en términos de $a_{0}$. Demuestra que la ecuación de índices es $k \, (k - 1) = 0$
\\
La expresión a la que debes de llegar es:
\begin{align*}
u_{k=1} &=  a_{0} \: \eta \left\lbrace 1 - \dfrac{2- \alpha}{6} \, \eta^{2} + \left[ \dfrac{(2-\alpha)(12-\alpha)}{120} - \dfrac{\beta}{20} \right] \eta^{4} + \ldots \right\rbrace
\end{align*}
\item Una solución para la ecuación de Chebychev
\begin{align*}
(1 -x^{2}) \, \stilde{y} - x \, \ptilde{y} + n^{2} \, y = 0
\end{align*}
para $n = 1$ es $y_{1}(x) = x$. Ocupando la doble integral del Wronskiano calcula la segunda solución $y_{2}(x)$.
\item Desarrolla los tres primeros términos no nulos en la serie de potencias en torno a $x = 0$, para la solución general en la siguiente ecuación, con $x > 0$:
\begin{align*}
x^{2} \, \stilde{y} + x \, \ptilde{y} + x^{2} \, y = 0
\end{align*}
Recuerda que debes de obtener la primera solución mediante el método de Frobenius y para la segunda solución, considera de la ecuación de índices, si las raíces $r_{1}$ y $r_{2}$ son tales que:
\begin{itemize}
\item Si $r_{1} = r_{2}$ entonces la segunda solución linealmente independiente es de la forma:
\begin{align*}
y_{2} (x) = y_{1} \, \ln(x) + \sum_{n=1}^{\infty} b_{n} \, x^{n+r_{1}}
\end{align*} 
\item Si $r_{1} - r_{2}$ es un entero positivo, entonces la segunda solución linealmente independiente es de la forma:
\begin{align*}
y_{2} (x) = C \, y_{1} \, \ln(x) + \sum_{n=0}^{\infty} b_{n} \, x^{n+r_{2}} \hspace{1cm} b_{0} \neq 0
\end{align*}
donde $C$ es una constante que podría anularse.
\end{itemize}
\end{enumerate}

\end{document}