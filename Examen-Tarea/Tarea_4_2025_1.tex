\documentclass[hidelinks,12pt]{article}
\usepackage[left=0.25cm,top=1cm,right=0.25cm,bottom=1cm]{geometry}
%\usepackage[landscape]{geometry}
\textwidth = 20cm
\hoffset = -1cm
\usepackage[utf8]{inputenc}
\usepackage[spanish,es-tabla, es-lcroman]{babel}
\usepackage[autostyle,spanish=mexican]{csquotes}
\usepackage[tbtags]{amsmath}
\usepackage{nccmath}
\usepackage{amsthm}
\usepackage{amssymb}
\usepackage{mathrsfs}
\usepackage{graphicx}
\usepackage{subfig}
\usepackage{caption}
%\usepackage{subcaption}
\usepackage{standalone}
\graphicspath{{Imagenes/}{../Imagenes/}}
\usepackage[outdir=./Imagenes/]{epstopdf}
\usepackage{siunitx}
\usepackage{physics}
\AtBeginDocument{\RenewCommandCopy\qty\SI}
\ExplSyntaxOn
\msg_redirect_name:nnn { siunitx } { physics-pkg } { none }
\ExplSyntaxOff
\usepackage{color}
\usepackage{float}
\usepackage{hyperref}
\usepackage{multicol}
\usepackage{multirow}
%\usepackage{milista}
\usepackage{anyfontsize}
\usepackage{anysize}
%\usepackage{enumerate}
\usepackage[shortlabels]{enumitem}
\usepackage{capt-of}
\usepackage{bm}
\usepackage{mdframed}
\usepackage{relsize}
\usepackage{placeins}
\usepackage{empheq}
\usepackage{cancel}
\usepackage{pdfpages}
\usepackage{wrapfig}
\usepackage[flushleft]{threeparttable}
\usepackage{makecell}
\usepackage{fancyhdr}
\usepackage{tikz}
\usepackage{bigints}
\usepackage{tcolorbox}
\tcbuselibrary{breakable}
\usepackage{scalerel}
\usepackage{pgfplots}
\usepackage{pdflscape}
\usepackage{enumitem}

\pgfplotsset{compat=1.16}
\spanishdecimal{.}
\renewcommand{\baselinestretch}{1.5}
\def\scaleint#1{\vcenter{\hbox{\scaleto[3ex]{\displaystyle\int}{#1}}}}
\def\scaleoint#1{\vcenter{\hbox{\scaleto[3ex]{\displaystyle\oint}{#1}}}}
\def\scaleiint#1{\vcenter{\hbox{\scaleto[3ex]{\displaystyle\iint}{#1}}}}
\def\scaleiiint#1{\vcenter{\hbox{\scaleto[3ex]{\displaystyle\iiint}{#1}}}}
\def\bs{\mkern-12mu}

\newcommand{\Cancel}[2][black]{{\color{#1}\cancel{\color{black}#2}}}

% \newcommand{\qed}{\tag*{$\blacksquare$}}
\renewcommand{\qed}{\hfill\blacksquare}

\newcommand{\pderivada}[1]{\ensuremath{{#1}^{\prime}}}
\newcommand{\sderivada}[1]{\ensuremath{{#1}^{\prime \prime}}}
\newcommand{\tderivada}[1]{\ensuremath{{#1}^{\prime \prime \prime}}}
\newcommand{\nderivada}[2]{\ensuremath{{#1}^{(#2)}}}

\newlength{\depthofsumsign}
\setlength{\depthofsumsign}{\depthof{$\sum$}}
\newcommand{\nsum}[1][1.4]{% only for \displaystyle
    \mathop{%
        \raisebox
            {-#1\depthofsumsign+1\depthofsumsign}
            {\scalebox
                {#1}
                {$\displaystyle\sum$}%
            }
    }
}

\newlist{milista}{enumerate}{2}
\setlist[milista,1]{label=\roman*)}
\setlist[milista,2]{label=\roman{milista}.\roman*)}

\title{Tarea Tema 4 \\[0.3em] \large{Matemáticas Avanzadas de la Física}\vspace{-3ex}}
\author{M. en C. Gustavo Contreras Mayén}
\date{ }

\begin{document}
\vspace{-4cm}
\maketitle

\fontsize{14}{14}\selectfont

\textbf{Indicaciones: } Se te pide gentilmente que resuelvas de manera detallada, clara y ordenada los siguientes ejercicios, el puntaje que otorga cada enunciado es de \textbf{1 punto}. En caso de que requieras apoyarte en alguna propiedad, si fue vista en clase, solo indícalo, pero si esa propiedad aunque esté relacionada al ejercicio y no se haya mencionado en clase, habrá que demostrarla debidamente.

\begin{enumerate}
%Referencia Hassani - Chapter 12
\item \textbf{Bessel. }Un cilindro largo conductor de calor de radio $a$ se compone de dos mitades (con secciones transversales semicirculares) con un espacio infinitesimal entre ellas. Las mitades superior e inferior del cilindro están en contacto con baños térmicos $+T_{0}$ y $-T_{0}$, respectivamente. Encuentra la temperatura tanto dentro como fuera del cilindro.
%Referencia Hassani - Chapter 12
\item Un cilindro largo conductor de calor de radio $a$ se compone de dos mitades (con secciones transversales semicirculares) con un espacio infinitesimal entre ellas. Las mitades superior e inferior del cilindro están en contacto con baños térmicos $+T_{1}$ y $-T_{1}$, respectivamente.
\par 
El cilindro está dentro de otro cilindro de radio $b$ más grande ( $a < b$ y coaxial con él) que se mantiene a la temperatura $T_{2}$. Encuentra la temperatura dentro del cilindro interno, entre los dos cilindros y fuera del cilindro externo.

\newpage

% %Ref. Arfken(2006) 13.1.6 (b)
\item \textbf{Hermite. } Demuestra que la expansión de la función:
\begin{align*}
f (x) =x^{2 r + 1}
\end{align*}
en una serie de polinomios de Hermite de orden impar es:
\begin{align*}
x^{2 r +1} &= \dfrac{(2 r + 1)!}{2^{2 r +1}} \nsum_{n=0}^{r} \dfrac{H_{2n+1}(x)}{(2 \, n + 1)! \, (r - n)!} \\[1em]
&{} \hspace{2cm} r = 0, 1, 2, \ldots
\end{align*}
Nota: Considera usar la fórmula de Rodrigues para luego integrar por partes.
%Ref. Arfken(2006) 13.1.10
\item Demuestra que:
\begin{align*}
\scaleint{6ex}_{\bs - \infty}^{\infty} x^{2} \, e^{-x^{2}} \, H_{n}(x) \, H_{n}(x) \dd{x} = \pi^{\frac{1}{2}} \, 2^{n} \, n! \, \bigg( n + \dfrac{1}{2} \bigg)
\end{align*}
Esta integral se presenta en el cálculo del desplazamiento medio cuadrado del oscilador armónico cuántico.
%Ref. Arfken (2013) 18.4.1
\item \textbf{Chebyshev. } Evaluando la función generatriz para valores especiales de $x$, verifica las siguientes expresiones:
\begin{align*}
T_{n} (1) = 1, \hspace{0.75cm} T_{n} (-1) = (-1)^{n}, \hspace{0.75cm} T_{2n} (0) = (-1)^{n}, \hspace{0.75cm} T_{2n+1} (0) = 0
\end{align*} 
\item Existen varias ecuaciones que relacionan los dos tipos de polinomios de Chebyshev. Demuestra que:
\begin{enumerate}[label=\alph*)]
\item $T_{n}(x) = U_{n}(x) - x \, U_{n-1}(x)$
\item $(1 - x^{2}) \, U_{n}(x) = x \, T_{n+1}(x) - T_{n+2}(x)$
\end{enumerate}
\end{enumerate}

\end{document}