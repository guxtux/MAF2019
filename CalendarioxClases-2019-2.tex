\documentclass[12pt]{article}
\usepackage[left=0.3cm,top=2cm,right=0.3cm,bottom=2cm]{geometry}
\usepackage[utf8]{inputenc}
\usepackage[spanish,es-tabla]{babel}
\usepackage{amsmath}
\usepackage{amsthm}
\usepackage{graphicx}
\usepackage{color}
\usepackage{float}
\usepackage{longtable}
\usepackage{multicol}
\usepackage{enumerate}
\usepackage{anyfontsize}
\usepackage{anysize}
\usepackage{natbib}
\usepackage{enumitem}
\usepackage{capt-of}
\usepackage{multirow}
\usepackage{array}
\newcolumntype{L}[1]{>{\raggedright\let\newline\\\arraybackslash\hspace{0pt}}m{#1}}
\newcolumntype{C}[1]{>{\centering\let\newline\\\arraybackslash\hspace{0pt}}m{#1}}
\spanishdecimal{.}
\setlist[enumerate]{itemsep=0mm}
\renewcommand{\baselinestretch}{1.2}
\setlength{\itemsep}{0pt}
\marginsize{1.5cm}{1.5cm}{1cm}{2cm}
\author{M. en C. Gustavo Contreras Mayén. \texttt{curso.fisica.comp@gmail.com}\\
M. en C. Abraham Lima Buendía. \texttt{abraham3081@ciencias.unam.mx}}
\title{Calendario de clases\\ {\large Matemáticas Avanzadas de la Física - Semestre 2019-1}}
\date{ }
\begin{document}
\renewcommand\labelenumii{\theenumi.{\arabic{enumii}}}
\maketitle
\fontsize{14}{14}\selectfont
\begin{longtable}{| C{3cm} | C{2cm} | L{11cm} |}
%\begin{tabular}{| C{3cm} | C{2cm} | L{11cm} |}
\hline
Mes & Día & Actividad \\ \hline
\multirow{4}{*}{Enero} & 28 & \begin{enumerate}
    \item Presentación del curso de MAF.
    \item Resumen de cómo se trabajará y a dónde queremos llevarlos al final del curso.
    \end{enumerate} \\ \cline{2-3} 
    & 29 & Tema 1. La física y la geometría - Clase 1 \\ \cline{2-3}
    & 30 & Tema 1. La física y la geometría - Clase 2 \\ \cline{2-3}
    & 31 & Tema 1. La física y la geometría - Clase 3 \\ \cline{2-3} \hline
\multirow{15}{*}{Febrero} & 1 & Tema 1. La física y la geometría - Clase 4 \\ \cline{2-3}
    & \textcolor{red}{4} & \textcolor{red}{Día Feriado - Aniversario de la Constitución} \\ \cline{2-3}
    & 5 & Tema 1. La física y la geometría - Clase 5 \\ \cline{2-3}
    & 6 & Tema 1. La física y la geometría - Clase 1 Ejercicios \\ \cline{2-3}
    & 7 & Tema 1. La física y la geometría - Clase 2 Ejercicios \\ \cline{2-3}
    & 8 & Tema 1. La física y la geometría - Clase 3 Ejercicios \\ \cline{2-3}
    & 11 & Tema 2. Primeras técnicas de solución - Clase 1 \\ \cline{2-3}
    & 12 & Tema 2. Primeras técnicas de solución - Clase 2 \\ \cline{2-3}
    & 13 & Tema 2. Primeras técnicas de solución - Clase 3 \\ \cline{2-3}
    & 14 & Tema 2. Primeras técnicas de solución - Clase 4 \\ \cline{2-3}
    & 15 & Tema 2. Primeras técnicas de solución - Clase 5 \\ \cline{2-3}
    & 18 & Tema 2. Primeras técnicas de solución - Clase 6 \\ \cline{2-3}
    & 19 & Tema 2. Primeras técnicas de solución - Clase 7 \\ \cline{2-3}
    & 20 & Tema 2. Primeras técnicas de solución - Clase 8 \\ \cline{2-3}
    & 21 & Tema 2. Primeras técnicas de solución - Clase 9 \\ \cline{2-3}
    & 22 & Tema 2. Primeras técnicas de solución - Clase 10 \\ \cline{2-3}
    & 25 & Tema 2. Primeras técnicas de solución - Ejercicios - Clase 1 \\ \cline{2-3}
    & 26 & Tema 2. Primeras técnicas de solución - Ejercicios - Clase 2 \\ \cline{2-3}
    & 27 & Tema 3. Primeras técnicas de solución - Clase 9 \\ \cline{2-3}
    & 28 & Tema 3. Completes y ortogonalidad - Clase 1 \\ \cline{2-3} \hline
    \multirow{16}{*}{Marzo} & 1 & Tema 3. Completes y ortogonalidad - Clase 2 \\ \cline{2-3}
    & 4 & Tema 3. Completes y ortogonalidad - Clase 3 \\ \cline{2-3}
    & 5 & Tema 3. Completes y ortogonalidad - Clase 4 \\ \cline{2-3}
    & 6 & Tema 3. Completes y ortogonalidad - Clase 5 \\ \cline{2-3}
    & 7 & Tema 3. Completes y ortogonalidad - Clase 6 \\ \cline{2-3}
    & 8 & Tema 3. Completes y ortogonalidad - Clase 7 \\ \cline{2-3}
    & 11 & Tema 3. Completes y ortogonalidad - Clase 8 \\ \cline{2-3}
    & 12 & Tema 4. Función Beta y Gamma - Clase 1 \\ \cline{2-3}
    & 13 & Tema 4. Función Beta y Gamma - Clase 2 \\ \cline{2-3}
    & 14 & Tema 4. Función Beta y Gamma - Clase 3 \\ \cline{2-3}
    & 15 & Tema 4. Función Beta y Gamma - Clase 4 \\ \cline{2-3}
    & \textcolor{red}{18} & \textcolor{red}{Natalicio Benito Juárez} \\ \cline{2-3}
    & 19 & Tema 4. Función Beta y Gamma - Clase 5 \\ \cline{2-3}
    & 20 & Tema 3 y Tema 4. Ejercicios - Clase 1 \\ \cline{2-3}
    & 21 & Tema 3 y Tema 4. Ejercicios - Clase 2 \\ \cline{2-3}
    & 22 & Tema 3 y Tema 4. Ejercicios - Clase 3 \\ \cline{2-3}
    & 25 & Tema 5. Separación de variables en coordenadas esféricas - Clase 1 \\ \cline{2-3}
    & 26 & Tema 5. Separación de variables en coordenadas esféricas - Clase 2 \\ \cline{2-3}
    & 27 & Tema 5. Separación de variables en coordenadas esféricas - Clase 3 \\ \cline{2-3}
    & 28 & Tema 5. Separación de variables en coordenadas esféricas - Clase 4 \\ \cline{2-3}
    & 29 & Tema 5. Separación de variables en coordenadas esféricas - Clase 5 \\ \cline{2-3} \hline
\multirow{16}{*}{Abril} & 1 & Tema 5. Separación de variables en coordenadas esféricas - Clase 6 \\ \cline{2-3}
    & 2 & Tema 5. Separación de variables en coordenadas esféricas - Clase 7 \\ \cline{2-3}
    & 3 & Tema 5. Separación de variables en coordenadas esféricas - Clase 8 \\ \cline{2-3}
    & 4 & Tema 5. Separación de variables en coordenadas esféricas - Clase 9 \\ \cline{2-3}
    & 5 & Tema 5. Separación de variables en coordenadas esféricas - Clase 10 \\ \cline{2-3}
%\end{tabular}
\end{longtable}
\end{document}