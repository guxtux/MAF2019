\documentclass[12pt]{article}
\usepackage[left=0.25cm,top=1cm,right=0.25cm,bottom=1cm]{geometry}
\textwidth = 20cm
\hoffset = -1cm
\usepackage[utf8]{inputenc}
\usepackage[spanish,es-tabla]{babel}
\usepackage[autostyle,spanish=mexican]{csquotes}
\usepackage[tbtags]{amsmath}
\usepackage{nccmath}
\usepackage{amsthm}
\usepackage{amssymb}
\usepackage{graphicx}
\usepackage{standalone}
\usepackage[outdir=./]{epstopdf}
\usepackage{siunitx}
\usepackage{physics}
\usepackage{color}
\usepackage{float}
\usepackage{multicol}
%\usepackage{milista}
\usepackage{enumitem}
\usepackage{anyfontsize}
\usepackage{anysize}
\usepackage{enumitem}
\usepackage{capt-of}
\usepackage{bm}
\usepackage{relsize}
\usepackage{placeins}
\usepackage{empheq}
\usepackage{cancel}
\usepackage{wrapfig}
\spanishdecimal{.}
\renewcommand{\baselinestretch}{1.5} 
\renewcommand\labelenumii{\theenumi.{\arabic{enumii}}}
\newcommand{\ptilde}[1]{\ensuremath{{#1}^{\prime}}}
\newcommand{\stilde}[1]{\ensuremath{{#1}^{\prime \prime}}}
\newcommand{\ttilde}[1]{\ensuremath{{#1}^{\prime \prime \prime}}}
\newcommand{\ntilde}[2]{\ensuremath{{#1}^{(#2)}}}


\title{Estudiando el sistema coordenado cónico - 2a. parte \\[0.3em]  \large{Matemáticas Avanzadas de la Física}\vspace{-3ex}}
\author{M. en C. Gustavo Contreras Mayén}
\date{ }

\begin{document}
\vspace{-4cm}
\maketitle
\fontsize{14}{14}\selectfont
\tableofcontents
\newpage

La construcción de cada uno de los operadores diferenciales se discute en el material de trabajo, por lo que se tomarán las expresiones a las que se llega. No está de más que repasen ese material.

\section{Operadores diferenciales.}

\subsection{Gradiente.}

En un sistema coordenado generalizado $(q_{1}, q_{2}, q_{3})$, escribimos el gradiente de una función escalar $\psi (q_{1}, q_{2}, q_{3})$ como:
\begin{align*}
\grad{\phi} &= \dfrac{1}{h_{1}} \, \vu{q}_{1} \, \pdv{\psi}{q_{1}} + \dfrac{1}{h_{2}} \, \vu{q}_{2} \, \pdv{\psi}{q_{2}} + \dfrac{1}{h_{3}} \, \vu{q}_{3} \, \pdv{\psi}{q_{3}} = \\[0.5em] 
&= \nsum_{i} \vu{q}_{i} \, \dfrac{1}{h_{i}} \, \pdv{\psi}{q_{i}}
\end{align*}

\subsection{Divergencia.}

Definimos la cantidad:
\begin{align*}
h = \dfrac{1}{h_{1} \, h_{2} \, h_{3}}
\end{align*}
que nos servirá para abreviar la expresión.

El operador divergencia de un vector $\vb{B}$ se expresa como:
\begin{align*}
\div{\vb{B}} &= h \, \bigg[ \pdv{q_{1}} \bigg( B_{1} \, h_{2} \, h_{3} \bigg) + \\[0.5em]
&+ \pdv{q_{2}} \bigg( B_{2} \, h_{3} \, h_{1} \bigg) + \pdv{q_{3}} \bigg( B_{3} \, h_{1} \, h_{2} \bigg) \bigg]
\end{align*}

\subsection{Rotacional.}

El operador rotacional de un vector $\vb{B}$ es:
\begin{align*}
\curl{\vb{B}} = h \, \mqty|
\vu{q}_{1} \, h_{1} & \vu{q}_{2} \, h_{2} & \vu{q}_{3} \, h_{3} \\[1em]
\displaystyle \pdv{q_{1}} & \displaystyle \pdv{q_{2}} & \displaystyle \pdv{q_{3}} \\[1em]
h_{1} \, B_{1} & h_{2} \, B_{2} & h_{3} \, B_{3}
|
\end{align*}

\subsection{Laplaciano.}

El operador Laplaciano de una función $\psi (q_{1}, q_{2}, q_{3})$ en un sistema coordenado generalizado, se define por la expresión:
\begin{align*}
\laplacian{\psi} &= h \, \bigg[ \pdv{q_{1}} \bigg( \dfrac{h_{2} h_{3}}{h_{1}} \, \pdv{\psi}{q_{1}} \bigg) + \\[0.5em]
&+ \pdv{q_{2}} \bigg( \dfrac{h_{3} h_{1}}{h_{2}} \, \pdv{\psi}{q_{2}} \bigg) + \pdv{q_{3}} \bigg( \dfrac{h_{1} h_{2}}{h_{3}} \, \pdv{\psi}{q_{3}} \bigg) \bigg]
\end{align*}

\section{Aplicación: La ecuación de Helmholtz.}

\subsection{Construyendo la ecuación.}

Sabemos que la ecuación de Helmholtz se expresa por:
\begin{align*}
\laplacian{E} + k^{2} \, E = 0
\end{align*}
donde $k$ es el número de onda, $E$ es la función escalar que es solución a la ecuación. Expresemos esta ecuación en el sistema coordenado cónico.
\par
Ya conocemos la expresión para el Laplaciano en coordenadas generalizadas:
\begin{align*}
\laplacian{E} &= h \,  \bigg[ \pdv{q_{1}} \bigg( \dfrac{h_{2} h_{3}}{h_{1}} \, \pdv{E}{q_{1}} \bigg) + \\[0.5em]
&+ \pdv{q_{2}} \bigg( \dfrac{h_{3} h_{1}}{h_{2}} \, \pdv{E}{q_{2}} \bigg) + \pdv{q_{3}} \bigg( \dfrac{h_{1} h_{2}}{h_{3}} \, \pdv{E}{q_{3}} \bigg) \bigg]
\end{align*}

Como ya conocemos los factores de escala para este sistema coordenado cónico\footnote{Ver notas de la clase anterior.}:
\begin{align*}
h_{r} &= 1 \\[0.5em]
h_{\theta} &=  r \, \sqrt{\dfrac{(\theta^{2} - \lambda^{2})}{(\theta^{2} - b^{2})(c^{2} - \theta^{2})}} \\[0.5em]
h_{\lambda} &= r \, \sqrt{\dfrac{(\theta^{2} - \lambda^{2})}{(\lambda^{2} - b^{2})(\lambda^{2} - c^{2})}}
\end{align*}
Nos resta más que sustituir aquéllos y hacer un pequeño manejo.
\par
Con la finalidad de simplificar las operaciones, hagamos que:
\begin{align*}
f(\theta) &= \sqrt{(\theta^{2} - b^{2})(c^{2} - \theta^{2})} \\[0.5em] 
f(\lambda) &= \sqrt{(\lambda^{2} - b^{2})(\lambda^{2} - c^{2})}
\end{align*}

Calculemos los cocientes que involucran los factores de escala:
\begin{align*}
\dfrac{1}{h_{1} \, h_{2} \, h_{3}} &= \dfrac{1}{\dfrac{r^{2} (\theta^{2} - \lambda^{2})}{f(\theta) \, f(\lambda)}} = \\[0.5em] 
&= \dfrac{f(\theta) \, f(\lambda)}{r^{2} (\theta^{2} - \lambda^{2})} \\[0.5em]
\dfrac{h_{2} \, h_{3}}{h_{1}} &= \dfrac{r^{2} (\theta^{2} - \lambda^{2})}{f(\theta) \, f(\lambda)} \\[0.5em]
\dfrac{h_{3} \, h_{1}}{h_{2}} &=  \dfrac{\dfrac{r \sqrt{\theta^{2} - \lambda^{2}}}{f(\lambda)}}{\dfrac{r \sqrt{\theta^{2} - \lambda^{2}}}{f(\theta)}} =  \dfrac{f (\theta)}{f (\lambda)}
\dfrac{h_{1} \, h_{2}}{h_{3}} &=  \dfrac{\dfrac{r \sqrt{\theta^{2} - \lambda^{2}}}{f(\theta)}}{\dfrac{r \sqrt{\theta^{2} - \lambda^{2}}}{f(\lambda)}} =  \dfrac{f (\lambda)}{f (\theta)}
\end{align*}

El operador Laplaciano en el sistema coordenado cónico se escribe como:
\begin{align*}
\laplacian{E} &= \dfrac{f(\theta) \, f(\lambda)}{r^{2} (\theta^{2} - \lambda^{2})} \bigg[ \pdv{r} \left( \dfrac{r^{2} (\theta^{2} - \lambda^{2})}{f(\theta) \, f(\lambda)} \, \pdv{E}{r} \right) + \\[0.5em]
&+ \pdv{\theta} \left( \dfrac{f(\theta)}{f(\lambda)} \, \pdv{E}{\theta} \right) + \pdv{\lambda} \left( \dfrac{f(\lambda)}{f(\theta)} \, \pdv{E}{\lambda} \right) \bigg]
\end{align*}
Que habrá que simplificar para entonces expresar la ecuación de Helmholtz.
\par
En el sistema coordenado cónico la ecuación de Helmholtz tiene la forma:
\begin{align*}
&\dfrac{1}{r^{2}} \pdv{r} \left( r^{2} \, \pdv{E}{r} \right) + \dfrac{f(\theta)}{r^{2} (\theta^{2} - \lambda^{2})} \, \pdv{\theta} \left( f(\theta) \, \pdv{E}{\theta} \right) + \\[0.5em]
&+ \dfrac{f(\lambda)}{r^{2} (\theta^{2} - \lambda^{2})} \, \pdv{\lambda} \left( f(\lambda) \, \pdv{E}{\lambda} \right) + k^{2} \, E = 0 \qed
\end{align*}
\end{document}