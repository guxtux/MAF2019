\documentclass[hidelinks,12pt]{article}
\usepackage[left=0.25cm,top=1cm,right=0.25cm,bottom=1cm]{geometry}
%\usepackage[landscape]{geometry}
\textwidth = 20cm
\hoffset = -1cm
\usepackage[utf8]{inputenc}
\usepackage[spanish,es-tabla]{babel}
\usepackage[autostyle,spanish=mexican]{csquotes}
\usepackage[tbtags]{amsmath}
\usepackage{nccmath}
\usepackage{amsthm}
\usepackage{amssymb}
\usepackage{mathrsfs}
\usepackage{graphicx}
\usepackage{subfig}
\usepackage{standalone}
\usepackage[outdir=./Imagenes/]{epstopdf}
\usepackage{siunitx}
\usepackage{physics}
\usepackage{color}
\usepackage{float}
\usepackage{hyperref}
\usepackage{multicol}
%\usepackage{milista}
\usepackage{anyfontsize}
\usepackage{anysize}
%\usepackage{enumerate}
\usepackage[shortlabels]{enumitem}
\usepackage{capt-of}
\usepackage{bm}
\usepackage{relsize}
\usepackage{placeins}
\usepackage{empheq}
\usepackage{cancel}
\usepackage{wrapfig}
\usepackage[flushleft]{threeparttable}
\usepackage{makecell}
\usepackage{fancyhdr}
\usepackage{tikz}
\usepackage{bigints}
\usepackage{scalerel}
\usepackage{pgfplots}
\usepackage{pdflscape}
\pgfplotsset{compat=1.16}
\spanishdecimal{.}
\renewcommand{\baselinestretch}{1.5} 
\renewcommand\labelenumii{\theenumi.{\arabic{enumii}})}
\newcommand{\ptilde}[1]{\ensuremath{{#1}^{\prime}}}
\newcommand{\stilde}[1]{\ensuremath{{#1}^{\prime \prime}}}
\newcommand{\ttilde}[1]{\ensuremath{{#1}^{\prime \prime \prime}}}
\newcommand{\ntilde}[2]{\ensuremath{{#1}^{(#2)}}}

\newtheorem{defi}{{\it Definición}}[section]
\newtheorem{teo}{{\it Teorema}}[section]
\newtheorem{ejemplo}{{\it Ejemplo}}[section]
\newtheorem{propiedad}{{\it Propiedad}}[section]
\newtheorem{lema}{{\it Lema}}[section]
\newtheorem{cor}{Corolario}
\newtheorem{ejer}{Ejercicio}[section]

\newlist{milista}{enumerate}{2}
\setlist[milista,1]{label=\arabic*)}
\setlist[milista,2]{label=\arabic{milistai}.\arabic*)}
\newlength{\depthofsumsign}
\setlength{\depthofsumsign}{\depthof{$\sum$}}
\newcommand{\nsum}[1][1.4]{% only for \displaystyle
    \mathop{%
        \raisebox
            {-#1\depthofsumsign+1\depthofsumsign}
            {\scalebox
                {#1}
                {$\displaystyle\sum$}%
            }
    }
}
\def\scaleint#1{\vcenter{\hbox{\scaleto[3ex]{\displaystyle\int}{#1}}}}
\def\bs{\mkern-12mu}



\title{Lista de ejercicios del Examen Parcial (Tema 1) \\[0.3em]  \large{Matemáticas Avanzadas de la Física}\vspace{-3ex}}
\author{M. en C. Gustavo Contreras Mayén}
\date{ }
\begin{document}
\vspace{-4cm}
\maketitle

\fontsize{14}{14}\selectfont

\begin{enumerate}
\item Con el sistema de coordenadas cilíndricas elípticas:
\begin{enumerate}[label=\alph*)]
\item Describe las superficies coordenadas que se generan.
\item \label{inciso_1_b} Determina el elemento $\dd{\vb{s}}$ y de los vectores unitarios $\vu{e}$, calcula las componentes de la velocidad y la aceleración.
\end{enumerate}
%Ref. Boas (2005) Section 9. Problemas 3
\item Usando coordenadas cilíndricas elípticas, escribe las ecuaciones de Lagrange para el movimiento de una partícula sobre la que actúa una fuerza $\vb{F} = - \grad{V}$, donde $V$ es la energía potencial. Divide cada ecuación de Lagrange por el factor de escala correspondiente para que los componentes de $\vb{F}$ (es decir, de $- \grad{V}$) aparezcan en las ecuaciones. Por tanto, escribe las ecuaciones como las ecuaciones componentes de $\vb{F} = m \, \vb{a}$, y así encuentra las componentes de la aceleración $\vb{a}$. Compara los resultados con el Problema 1 \ref{inciso_1_b}.
 %Ref. Arfken (1981) 2.6.6
\item En coordenadas cilíndricas circulares rectas, una función vectorial particular está definida por:
\begin{align*}
\vb{V} (\rho, \varphi) = \rho_{0} \, V_{\rho} (\rho, \varphi) + \varphi_{0} \, V_{\varphi} (\rho, \varphi)
\end{align*}
Demuestra que $\curl{\vb{V}}$ tiene solamente componente $z$. Observa que este resultado es consistente para cualquier vector confinado a la superficie $q_{3} =$ constante, mientras que cada uno de los productos $h_{1} \, V_{1}$ y $h_{2} \, V_{2}$ sean independientes de $q_{3}$.
%Ref. Arfken (1981) 2.11.1
\item Escribe la ecuación de Laplace en coordenadas esferoidales oblatas. Resuelve la ecuación diferencial dependiente de $\varphi$.
%Ref. Arfken (2006) 2.5.18
\item Demuestra que las tres siguientes formas (coordenadas esféricas) de $\laplacian{\psi} (r)$ son equivalentes:
\begin{multicols}{3}
\begin{enumerate}[label=\alph*)]
\item $\displaystyle \dfrac{1}{r^{2}} \, \dv{r} \bigg[ r^{2} \, \dv{\psi (r)}{r} \bigg]$
\item $\displaystyle  \dfrac{1}{r} \, \dv[2]{r} \big[ r \, \psi (r) \big]$
\item $\displaystyle  \dv[2]{\psi (r)}{r} + \dfrac{2}{r} \, \dv{\psi (r)}{r}$
\end{enumerate}
\end{multicols}
%Ref. Arfken (2006) 2.5.20
\item Un cierto campo de fuerza está dado por:
\begin{align*}
\vb{F} = \vu{r}\, \dfrac{2 P \cos \theta}{r^{3}} + \vu{\theta} \, \dfrac{P}{r^{3}} \sin \theta, \hspace{1.5cm} r \geq \dfrac{P}{2}
\end{align*}
en coordenadas esféricas polares.
\begin{enumerate}[label=\roman*)]
\item Revisa $\curl{\vb{F}}$ para determinar si existe un potencial.
\item Calcular $\scaleoint{6ex} \vb{F} \cdot \dd{\bm{\lambda}}$ para un círculo unitario en el plano $\theta = \pi/2$. ¿Qué indica de que la fuerza sea conservativa o no conservativa?
\item Si consideras que $\vb{F}$ se puede describir por $\vb{F} = - \grad{\psi}$, encuentra $\psi$. De otra manera establece que no es existe un potencial aceptable.
\end{enumerate}
%Ref. Kusse (2006) Chap.3 Exercises 11
\item Con el sistema de coordenadas toroidal $(\xi, \eta, \varphi)$:
\begin{enumerate}[label=\alph*)]
\item Describe las superficies constantes para $\xi$ y $\eta$.
\item Selecciona un punto y describe los vectores base. ¿Este es un sistema derecho?
\item Obtén de manera explícita los factores de escala.
\item Determina los vectores de posición y de desplazamiento.
\item Desarrolla las expresiones para el gradiente $\grad{\Phi}$, la divergencia \hfill \break $\divergence{\vb{A}}$, y el rotacional $\curl{\vb{A}}$.
\item Escribe la ecuación de Helmholtz en este sistema toroidal.
\end{enumerate}
%Ref. Andrews (1998) Special Functions of Mathematics for Engineers. 
%Exercises 2.2 Problem 37
\item En los siguientes incisos, ocupando las identidades de Euler:
\begin{align*}
\cos x = \dfrac{e^{i x} + e^{-i x}}{2} \hspace{1.5cm} \sin x = \dfrac{e^{i x} - e^{-i x}}{2 i}
\end{align*}
así como las propiedades de la función Gamma, suponiendo que $b, x > 0$ y $-\dfrac{\pi}{2} < a < \dfrac{\pi}{2}$. Demuestra que:
\begin{enumerate}[label=\alph*)]
\item $\Gamma (x) \, \cos a x = b^{x} \, \scaleint{6ex}_{\bs 0}^{\infty} t^{x-1} \, \exp(- b \, t \, \cos a) \, \cos (b \, t \, \sin a) \dd{t}$
\item $\Gamma (x) \, \sin a x = b^{x} \, \scaleint{6ex}_{\bs 0}^{\infty} t^{x-1} \, \exp(- b \, t \, \cos a) \, \sin (b \, t \, \sin a) \dd{t}$
\end{enumerate}
%Exercises 2.4 Problem 8
\item Demuestra que:
\begin{align*}
B(x, y) \, B(x + y, z) \, B(x + y + z, w) = \dfrac{\Gamma(x) \, \Gamma (y) \, \Gamma (z) \, \Gamma (w)}{\Gamma (x + y + z + w)}
\end{align*}
%Problems 11, 12 
\item Usando las propiedades de las funciones Gamma y Beta, evalúa las siguientes integrales:
\begin{enumerate}[label=\alph*)]
\item $\scaleint{6ex}_{\bs 0}^{1} \sqrt{x (1 - x)} \dd{x}$
\item $\scaleint{6ex}_{\bs 0}^{1} x^{4} \, (1 - x^{2})^{-\frac{1}{2}} \dd{x}$
\end{enumerate}
\end{enumerate}

\end{document}