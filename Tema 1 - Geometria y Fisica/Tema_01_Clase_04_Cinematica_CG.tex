\documentclass[12pt]{beamer}
\usepackage{../Estilos/BeamerMAF}
\usetheme{Warsaw}
\usecolortheme{seahorse}
%\useoutertheme{default}
\setbeamercovered{invisible}
% or whatever (possibly just delete it)
\setbeamertemplate{section in toc}[sections numbered]
\setbeamertemplate{subsection in toc}[subsections numbered]
\setbeamertemplate{subsection in toc}{\leavevmode\leftskip=3.2em\rlap{\hskip-2em\inserttocsectionnumber.\inserttocsubsectionnumber}\inserttocsubsection\par}
\setbeamercolor{section in toc}{fg=blue}
\setbeamercolor{subsection in toc}{fg=blue}
\setbeamercolor{frametitle}{fg=blue}
\setbeamertemplate{caption}[numbered]

\setbeamertemplate{footline}
\beamertemplatenavigationsymbolsempty
\setbeamertemplate{headline}{}


\makeatletter
\setbeamercolor{section in foot}{bg=gray!30, fg=black!90!orange}
\setbeamercolor{subsection in foot}{bg=blue!30}
\setbeamercolor{date in foot}{bg=black}
\setbeamertemplate{footline}
{
  \leavevmode%
  \hbox{%
  \begin{beamercolorbox}[wd=.333333\paperwidth,ht=2.25ex,dp=1ex,center]{section in foot}%
    \usebeamerfont{section in foot} \insertsection
  \end{beamercolorbox}%
  \begin{beamercolorbox}[wd=.333333\paperwidth,ht=2.25ex,dp=1ex,center]{subsection in foot}%
    \usebeamerfont{subsection in foot}  \insertsubsection
  \end{beamercolorbox}%
  \begin{beamercolorbox}[wd=.333333\paperwidth,ht=2.25ex,dp=1ex,right]{date in head/foot}%
    \usebeamerfont{date in head/foot} \insertshortdate{} \hspace*{2em}
    \insertframenumber{} / \inserttotalframenumber \hspace*{2ex} 
  \end{beamercolorbox}}%
  \vskip0pt%
}
\makeatother

\makeatletter
\patchcmd{\beamer@sectionintoc}{\vskip1.5em}{\vskip0.8em}{}{}
\makeatother

\newlength{\depthofsumsign}
\setlength{\depthofsumsign}{\depthof{$\sum$}}
\newcommand{\nsum}[1][1.4]{% only for \displaystyle
    \mathop{%
        \raisebox
            {-#1\depthofsumsign+1\depthofsumsign}
            {\scalebox
                {#1}
                {$\displaystyle\sum$}%
            }
    }
}
\def\scaleint#1{\vcenter{\hbox{\scaleto[3ex]{\displaystyle\int}{#1}}}}
\def\scaleoint#1{\vcenter{\hbox{\scaleto[3ex]{\displaystyle\oint}{#1}}}}
\def\bs{\mkern-12mu}

\makeatletter
\setbeamertemplate{footline}
{
  \leavevmode%
  \hbox{%
  \begin{beamercolorbox}[wd=.333333\paperwidth,ht=2.25ex,dp=1ex,center]{section in foot}%
    \usebeamerfont{section in foot} \insertsection
  \end{beamercolorbox}%
  \begin{beamercolorbox}[wd=.333333\paperwidth,ht=2.25ex,dp=1ex,center]{subsection in foot}%
    \usebeamerfont{subsection in foot}  \insertsubsection
  \end{beamercolorbox}%
  \begin{beamercolorbox}[wd=.333333\paperwidth,ht=2.25ex,dp=1ex,right]{date in head/foot}%
    \usebeamerfont{date in head/foot} \insertshortdate{} \hspace*{2em}
    \insertframenumber{} / \inserttotalframenumber \hspace*{2ex} 
  \end{beamercolorbox}}%
  \vskip0pt%
}
\makeatother
\date{5 de octubre de 2021}
\title{Cinemática de una partícula}
\subtitle{La física y la geometría}
\begin{document}
\maketitle
\fontsize{14}{14}\selectfont
\spanishdecimal{.}

\section{Introducción}
\frame{\tableofcontents[currentsection, hideothersubsections]}
\subsection{El movimiento de una partícula}

\begin{frame}
\frametitle{Avances en el Tema 1}
Luego de estudiar el sistema de coordenadas generalizado, podemos expresar las superficies coordenadas, calcular el tensor métrico y por ende, los factores de escala, así como los operadores diferenciales.
\end{frame}
\begin{frame}
\frametitle{Expresar una ecuación}
También podemos reescribir una ecuación de la Física Matemática en términos de un nuevo sistema.
\\
\bigskip
\pause
Hemos planteado que la EDP obtenida, será resuelta en el próximo Tema 2 del curso.
\end{frame}
\begin{frame}
\frametitle{¿Qué hay del movimiento de una partícula?}
El estudio de fenómenos en esos sistemas nuevo, no se limita a describir las características del mismo.
\\
\bigskip
\pause
También se considera el análisis del movimiento de una partícula dentro de ese sistema, es decir, la cinemática propia en el sistema coordenado.
\end{frame}
\begin{frame}
\frametitle{Descripción del movimiento}
Sabemos que para describir el movimiento de una partícula, se requieren al menos tres cantidades físicas:
\pause
\setbeamercolor{item projected}{bg=blue!70!black,fg=yellow}
\setbeamertemplate{enumerate items}[circle]
\begin{enumerate}[<+->]
\item Posición.
\item Velocidad.
\item Aceleración.
\end{enumerate}
\pause
Con el estudio previo de sistemas coordenados, la posición de una partícula es algo que ya debemos de manejar sin contratiempos.
\end{frame}
\begin{frame}
\frametitle{La velocidad y aceleración generalizadas}
Ahora abordaremos la descripción de la velocidad y aceleración generalizada, de tal manera que la información recabada en el estudio del sistema coordenado (tensor métrico) nos sea de apoyo.
\\
\bigskip
\pause
De tal manera que podremos ocupar las expresiones generalizadas en cualquier otro sistema coordenado.
\end{frame}

\section{Sistema coord. generalizado}
\frame{\tableofcontents[currentsection, hideothersubsections]}
\subsection{Definición}

\begin{frame}
\frametitle{Sistema coordenado generalizado}
Definimos a las coordenadas generalizadas como un conjunto cualquiera de parámetros numéricos $\left\{ q_{l} \right\}$ que determinan de manera unívoca la posición de una partícula con un número finito de grados de libertad.
\end{frame}
\begin{frame}
\frametitle{Coordenadas independientes}
La cantidad mínima de coordenadas generalizadas que definen el estado del sistema se conoce como coordenadas independientes, esta condición queda expresada como:
\pause
\begin{align*}
\pdv{q_{l}}{q_{m}} = \delta_{l m}
\end{align*}
\end{frame}
\begin{frame}
\frametitle{Movimiento en tres direcciones}
Considerando el movimiento de una partícula con tres grados de libertad, tendremos la misma cantidad de coordenadas generalizadas $q_{1}, q_{2}, q_{3}$, partiendo de esto, la posición de la partícula quedará descrita como:
\pause
\begin{align}
\va{r} = x \, (q_{1}, q_{2}, q_{3}) \, \vu{i} + y \, (q_{1}, q_{2}, q_{3}) \, \vu{j} + z \, (q_{1}, q_{2}, q_{3}) \, \vu{k}
\label{eq:ecuacion_01_01}
\end{align}
\end{frame}
\begin{frame}
\frametitle{Vectores base}
Cada uno de los vectores base $\va{b}_{l}$ de las coordenadas generalizadas es construido mediante la variación infinitesimal de la posición de la partícula en una determinada coordenada generalizada $q_{{l}}$, es decir:
\pause
\begin{align*}
\va{b}_{l} = \pdv{\va{r}}{q_{l}}
\end{align*}
\end{frame}
\begin{frame}
\frametitle{Vectores base}
Trabajamos sobre está última expresión:
\pause
\begin{align}
\va{b}_{l} = \pdv{\va{r}}{q_{l}} = \pdv{x}{q_{l}} \, \vu{i} + \pdv{y}{q_{l}} \, \vu{j} + \pdv{z}{q_{l}} \, \vu{k}
\label{eq:ecuacion_01_02}
\end{align}
\end{frame}
\begin{frame}
\frametitle{Vectores base no normalizados}
Obsérvese que el conjunto de vectores $\left\{ \va{b}_{l} \right\}$ , no es necesariamente ortogonal, ni tampoco son vectores normalizados.
\end{frame}
\begin{frame}
\frametitle{Normalizando la base}
Al dividir entre la propia magnitud, que nos devuelve el concepto ya visto de factor de escala:
\pause
\begin{align*}
h_{i} = \abs{\pdv{\va{r}}{q_{i}}}
\end{align*}
\pause
Por lo que el vector base unitario queda expresado como:
\pause
\begin{align*}
\vu{e}_{i} = \dfrac{\va{b}_{i}}{h_{i}}
\end{align*}
\end{frame}
\begin{frame}
\frametitle{Condición necesaria}
Recordemos que una condición necesaria que debe de cumplirse en el caso de vectores base no ortogonales, es que no deben de ser coplanares, es decir:
\begin{align*}
\va{b}_{1} \cdot \va{b}_{2} \cp \va{b}_{3} = \pdv{(x, y, z)}{(q_{1}, q_{2}, q_{3})} \neq 0
\end{align*}
\end{frame}
\begin{frame}
\frametitle{Con el Jacobiano distinto de cero}
Donde el Jacobiano es:
\begin{align*}
\pdv{(x, y, z)}{(q_{1}, q_{2}, q_{3})} = \mqty|
\displaystyle \pdv{x}{q_{1}} & \displaystyle \pdv{y}{q_{1}} & \displaystyle \pdv{z}{q_{1}} \\[0.5em]
\displaystyle \pdv{x}{q_{2}} & \displaystyle \pdv{y}{q_{2}} & \displaystyle \pdv{z}{q_{2}} \\[0.5em]
\displaystyle \pdv{x}{q_{3}} & \displaystyle \pdv{y}{q_{3}} & \displaystyle \pdv{z}{q_{3}} |
\end{align*}
\end{frame}
\begin{frame}[t]
\frametitle{Independencia coordenadas generalizadas}
Las coordenadas generalizadas $(q_{1}, q_{2}, q_{3})$ son coordenadas independientes, por lo que:
\pause
\begin{align*}
\pdv{q_{1}}{q_{1}} = \pdv{q_{2}}{q_{2}} = \pdv{q_{3}}{q_{3}} = 1
\end{align*}
\pause
y además:
\begin{align*}
\pdv{q_{1}}{q_{2}} = \pdv{q_{1}}{q_{3}} = \pdv{q_{2}}{q_{1}} = \pdv{q_{2}}{q_{3}} = \pdv{q_{3}}{q_{1}} = \pdv{q_{3}}{q_{2}} = 0
\end{align*}
\pause
\begin{tikzpicture}[overlay]
\draw [fill, color=blue, text=white, opacity=0.95](1, 0) rectangle (10, 5) node[pos=0.5] {$\displaystyle \pdv{q_{i}}{q_{j}} = \begin{cases}
    0 & i \neq j \\
    1 & i = j
    \end{cases}$
    };
\draw [fill, color=white, text=black](1, 4) rectangle (10, 5) node[pos=0.5] {Es decir:};
\end{tikzpicture}
\end{frame}
\begin{frame}
\frametitle{Base recíproca}
Ahora construimos la base recíproca de vectores $\left\{ b_{l}^{*} \right\}$.
\pause
Por la independencia lineal de las coordenadas generalizadas se tiene la condición:
\pause
\begin{align*}
\pdv{q_{l}}{q_{m}} = \delta_{l m}
\end{align*}
\end{frame}
\begin{frame}
\frametitle{Exploremos está expresión}
\begin{equation}
\begin{aligned}
\pdv{q_{l}}{q_{m}} &= \pause \pdv{q_{l}}{x} \, \pdv{x}{q_{m}} + \pdv{q_{l}}{y} \, \pdv{y}{q_{m}} + \pdv{q_{l}}{z} \, \pdv{z}{q_{m}} \\[0.5em] \pause
\Rightarrow \pdv{q_{l}}{q_{m}} &= \left( \vu{i} \, \pdv{q_{l}}{x} + \vu{j} \, \pdv{q_{l}}{y} + \vu{k} \, \pdv{q_{l}}{z} \right) \cdot \\[0.5em]
&\cdot \left( \vu{i} \, \pdv{x}{q_{m}} + \vu{j} \, \pdv{y}{q_{m}} + \vu{k} \, \pdv{z}{q_{m}} \right) = \pause
\va{b}_{l}^{*} \cdot \va{b}_{m} \\[0.5em] \pause
&\Leftrightarrow \va{b}_{l}^{*} = \left( \vu{i} \, \pdv{q_{l}}{x} + \vu{j} \, \pdv{q_{l}}{y} + \vu{k} \, \pdv{q_{l}}{z} \right) \pause = \grad{q_{l}}
\end{aligned}
\label{eq:ecuacion_01_05}
\end{equation}
\end{frame}
\begin{frame}
\frametitle{Base recíproca}
Observa que la condición:
\pause
\begin{align*}
\va{b}_{l}^{*} \cdot \va{b}_{m} = \delta_{l, m}
\end{align*}
está definida en ambas bases, partiendo de este hecho, surge la necesidad de definir un espacio covariante y uno contravariante.
\end{frame}
\begin{frame}
\frametitle{Base recíproca}
Al mismo tiempo se verifica el caso particular de un sistema ortogonal:
\begin{align*}
\va{b}_{l}^{*} = \dfrac{\va{b}_{l}}{h_{l}} 
\end{align*}
usando estos elementos escribimos la velocidad y la aceleración de una partícula.
\end{frame}

\section{Velocidad de una partícula}
\frame{\tableofcontents[currentsection, hideothersubsections]}
\subsection{Velocidad y vectores base}

\begin{frame}
\frametitle{Velocidad}
Construimos el vector velocidad, tanto en la base covariante como en la contravariante:
\pause
\begin{align}
\begin{aligned}
\va{v} &= (\va{v} \cdot \va{b}_{l}^{*}) \, \va{b}_{l} = v_{l}^{*} \, b_{l} \\[0.5em]
\va{v} &= (\va{v} \cdot \va{b}_{l}) \, \va{b}_{l}^{*} = v_{l} \, b_{l}^{*}
\end{aligned}
\label{eq:ecuacion_01_06}
\end{align}
\pause
donde:
\begin{align*}
(\va{v} \cdot \va{b}_{l}^{*}) &= v_{l}^{*} \\
(\va{v} \cdot \va{b}_{l}) &= v_{l}
\end{align*}
\end{frame}
\begin{frame}
\frametitle{Componentes de la velocidad}
Podemos construir un procedimiento directo, para obtener expresiones sencillas que permitan realizar cálculos:
\pause
\begin{equation}
\begin{aligned}
v_{l} &= \vb{v} \cdot \va{b}_{l} {=} \pause (\vu{i} \, \dot{x} {+} \vu{j} \, \dot{y} {+} \vu{k} \, \dot{z}) \cdot \left( \vu{i} \, \pdv{x}{q_{l}} + \vu{j} \, \pdv{y}{q_{l}} + \vu{k} \, \pdv{z}{q_{l}}\right) {=} \\[0.5em] \pause
&= \dot{x} \, \pdv{x}{q_{l}} + \dot{y} \, \pdv{y}{q_{l}} + \dot{z} \, \pdv{z}{q_{l}}
\end{aligned}
\label{eq:ecuacion_01_07}
\end{equation}
\end{frame}
\begin{frame}
\frametitle{Componentes de la velocidad}
Ahora veamos que cada una de las componentes de la velocidad está definida por:
\begin{equation}
\begin{aligned}[b]
\dot{x} &= \pdv{x}{q_{m}} \, \dot{q}_{m} \hspace{1cm} \dot{y} = \pdv{y}{q_{m}} \, \dot{q}_{m} \hspace{1cm} \dot{z} = \pdv{z}{q_{m}} \, \dot{q}_{m} \\[0.5em] \pause
\Rightarrow \pdv{\dot{x}}{\dot{q}_{m}} &= \pdv{x}{q_{m}}  \hspace{1cm} \pdv{\dot{y}}{\dot{q}_{m}} = \pdv{y}{q_{m}}  \hspace{1cm} \pdv{\dot{z}}{\dot{q}_{m}} = \pdv{z}{q_{m}} 
\end{aligned}
\label{eq:ecuacion_01_08}
\end{equation}
\end{frame}
\begin{frame}
\frametitle{Las componentes de la velocidad}
Sustituyendo la ec. (\ref{eq:ecuacion_01_08}) en la ec. (\ref{eq:ecuacion_01_07}) se obtiene una expresión para el cálculo de $v_{l}$:
\pause
\begin{equation}
\begin{aligned}[b]
v_{l} &= \dot{x} \, \pdv{\dot{x}}{\dot{q}_{l}} + \dot{y} \, \pdv{\dot{y}}{\dot{q}_{l}} + \dot{z} \, \pdv{\dot{z}}{\dot{q}_{l}} = \\[0.5em] \pause
&= \dfrac{1}{2} \, \pdv{\dot{q}_{l}} \left( \dot{x}^{2} + \dot{y}^{2} + \dot{z}^{2} \right) = \\[0.5em] \pause
&= \pdv{\dot{q}_{l}} \, \left( \dfrac{v^{2}}{2} \right)
\end{aligned}
\label{eq:ecuacion_01_09}
\end{equation}
\end{frame}
\begin{frame}
\frametitle{La expresión para la velocidad}
\begin{align*}
v_{l} = \pdv{\dot{q}_{l}} \, \left( \dfrac{v^{2}}{2} \right)
\end{align*}
La ecuación anterior (\ref{eq:ecuacion_01_09}) nos indica que las componente covariantes de la velocidad son iguales a las derivadas parciales, con respecto a las $\dot{q}_{l}$ de la mitad del cuadrado de la velocidad.
\end{frame}
\begin{frame}
\frametitle{Procedimiento para obtener la velocidad}
Antes de derivar, debemos de expresar $\frac{1}{2} v^{2}$ en función de las coordenadas generalizadas $q_{l}$ y de sus derivadas con respecto a $\dot{q}_{l}$.
\end{frame}
\begin{frame}
\frametitle{Velocidad en coordenadas esféricas}
\textbf{Ejemplo: } Consideremos en el sistema de coordenadas esféricas, tal que:
\pause
\begin{align*}
\dfrac{1}{2} \, v^{2} = \dfrac{1}{2} \left( \dot{r}^{2} + r^{2} \, \dot{\theta}^{2} + r^{2} \, \sin \theta \, \dot{\phi}^{2} \right)
\end{align*}
\pause
Ocuparemos la expresión de la ec. (\ref{eq:ecuacion_01_09}).
\end{frame}
\begin{frame}
\frametitle{Componentes covariantes de la velocidad}
Entonces tendremos que las componentes covariantes son:
\begin{equation*}
\begin{aligned}
v_{r} &= \pdv{\dot{r}} \left( \dfrac{1}{2} \, v^{2} \right) = \pause \dot{r} \\[0.5em] \pause
v_{\theta} &= \pdv{\dot{\theta}} \left( \dfrac{1}{2} \, v^{2} \right) = \pause r^{2} \, \dot{\theta} \\[0.5em] \pause
v_{\varphi} &= \pdv{\dot{\phi}} \left( \dfrac{1}{2} \, v^{2} \right) = \pause r^{2} \, \sin^{2} \theta \, \dot{\phi}
\end{aligned}
\end{equation*}
\end{frame}
\begin{frame}
\frametitle{Usando el tensor métrico}
Podemos escribir la $v^{2}$ en términos del tensor métrico $g_{lm}$, así que:
\pause
\begin{align}
v^{2} = g_{lm} \, v_{l}^{*} \, v_{m}^{*}
\label{eq:ecuacion_01_10}
\end{align}
Donde debemos de expresar las componentes contravariantes de la velocidad: $v_{l}^{*}$.
\end{frame}
\begin{frame}
\frametitle{Componentes contravariantes de la velocidad}
\begin{equation}
\begin{aligned}[b]
v_{l}^{*} &= \va{v} \cdot \va{b}_{l}^{*} = \\[0.5em] \pause
&= (\vu{i} \, \dot{x} {+} \vu{j} \, \dot{y} {+} \vu{k} \, \dot{z}) \cdot \left( \vu{i} \, \pdv{q_{l}}{x} + \vu{j} \, \pdv{q_{l}}{y} + \vu{k} \, \pdv{q_{l}}{z} \right) = \\[0.5em] \pause
&= \dot{x} \, \pdv{q_{l}}{x} + \dot{y} \, \pdv{q_{l}}{y} + \dot{z} \, \pdv{q_{l}}{z}
\end{aligned}
\label{eq:ecuacion_01_11}
\end{equation}
a estas expresiones se les conoce como \emph{velocidades generalizadas}
\end{frame}
\begin{frame}
\frametitle{Ocupando las velocidades generalizadas}
Al sustituir la ec. (\ref{eq:ecuacion_01_11}) en la ec. (\ref{eq:ecuacion_01_10}) se obtiene:
\pause
\begin{equation}
\begin{aligned}[b]
v^{2} &= g_{lm} \, \dot{q}_{l} \, \dot{q}_{m} \\[0.5em] \pause
\Rightarrow v_{l} &= \pdv{\dot{q}_{l}} \, \dfrac{g_{lm} \, \dot{q}_{l} \, \dot{q}_{m}}{2}
\end{aligned}
\label{eq:ecuacion_01_12}
\end{equation} 
\pause
el lado derecho de la ec. (\ref{eq:ecuacion_01_12}) es referido en textos de geometría diferencial como \emph{energía de la variedad}.
\end{frame}
\begin{frame}
\frametitle{Expresión muy útil}
En el caso particular de un sistema ortogonal, la ec. (\ref{eq:ecuacion_01_12}) se reduce a:
\pause
\begin{align*}
v_{l} = h_{i}^{2} \, \dot{q}_{l}
\end{align*}
Que es muy útil para exámenes!!
\end{frame}

\section{Aceleración de una partícula}
\frame{\tableofcontents[currentsection, hideothersubsections]}
\subsection{Construyendo las componentes}

\begin{frame}
\frametitle{Ecuaciones de movimiento}
Las ecuaciones de movimiento de una partícula requieren conocer las aceleraciones de ésta, así que conectamos la ec. (\ref{eq:ecuacion_01_12}) con las ecuaciones de movimiento de una partícula.
\end{frame}
\begin{frame}
\frametitle{Componentes covariantes de la aceleración}
La aceleración es:
\pause
\begin{equation}
\begin{aligned}[b]
\va{a} &= \left( \va{a} \cdot \va{b}_{l} \right) \, \va{b}_{l}^{*} = \\[0.5em] \pause
&= \left( \vu{i} \, \ddot{x} {+} \vu{j} \, \ddot{y} {+} \vu{k} \, \ddot{z} \right) \cdot \left( \vu{i} \, \pdv{x}{q_{j}} \!+ \vu{j} \, \pdv{y}{q_{j}} \!+ \vu{k} \, \pdv{z}{q_{j}} \right) \, \va{b}_{l}^{*} = \\[0.5em] \pause
&= \left( \ddot{x} \, \pdv{x}{q_{j}} \!+ \ddot{y} \, \pdv{y}{q_{j}} \!+ \ddot{z} \, \pdv{z}{q_{j}} \right) \, \va{b}_{l}^{*} = \\[0.5em] \pause
&= a_{l} \, \va{b}_{l}^{*}
\end{aligned}
\label{eq:ecuacion_01_13}
\end{equation}
\end{frame}
\begin{frame}
\frametitle{Las componentes covariantes de la aceleración}
Reescribimos la ec.(\ref{eq:ecuacion_01_13}) de la forma, mostrado para la coordenada $x$:
\pause
\begin{eqnarray*}
\begin{aligned}
\ddot{x} \, \pdv{x}{q_{j}} &= \pause \dv{t} \dot{x} \, \pdv{x}{q_{l}} \!- \dot{x} \, \dv{t} \, \pdv{x}{q_{l}} = \\[0.5em] \pause
&= \dv{t} \left( \dot{x} \, \pdv{x}{q_{l}} \right) \!- \dot{x} \, \pdv{\dot{x}}{q_{l}} 
\end{aligned}
\end{eqnarray*}
\pause
Como se cumple que:
\begin{align*}
\pdv{x}{q_{l}} = \pdv{\dot{x}}{\dot{q}_{l}}
\end{align*}
\end{frame}
\begin{frame}
\frametitle{Componentes covariantes de la aceleración}
Llegamos a:
\pause
\begin{eqnarray*}
\begin{aligned}
\ddot{x} \, \pdv{x}{q_{j}} &= \pause \dv{t} \pdv{\dot{q}_{l}} \left( \dfrac{1}{2} \, \dot{x}^{2} \right) \!- \pdv{q_{l}} \left( \dfrac{1}{2} \, \dot{x}^{2} \right)
\end{aligned}
\end{eqnarray*}
\pause
Haciendo de manera análoga el desarrollo para:
\begin{align*}
\ddot{y} \, \pdv{y}{q_{j}} \hspace{1.5cm} \ddot{z} \, \pdv{z}{q_{j}}
\end{align*}
\end{frame}
\begin{frame}
\frametitle{Componentes covariantes de la aceleración}
Encontramos que las componentes son:
\pause
\begin{eqnarray}
\begin{aligned}[b]
a_{l} &= \pause \dv{t} \pdv{\dot{q}_{l}} \left( \dfrac{\dot{x}^{2} {+} \dot{y}^{2} {+} \dot{z}^{2}}{2} \right) \!- \pdv{q_{l}} \left( \dfrac{\dot{x}^{2} {+} \dot{y}^{2} {+} \dot{z}^{2}}{2} \right) = \\[0.5em] \pause
&= \dv{t} \, \pdv{\dot{q}_{l}} \left( \dfrac{v^{2}}{2} \right) - \pdv{q_{l}} \left( \dfrac{v^{2}}{2} \right) = \\[0.5em] \pause
&= \bigg[ \dv{t} \, \pdv{\dot{q}_{l}} - \pdv{q_{l}} \bigg] \left( \dfrac{v^{2}}{2} \right)
\end{aligned}
\label{eq:ecuacion_01_14}
\end{eqnarray}
\end{frame}
\begin{frame}
\frametitle{\textbf{Ejemplo: } Aceleración en coordenadas cilíndricas}
Para determinar las componentes de la aceleración en coordenadas cilíndricas, tenemos que:
\pause
\begin{align*}
v^{2} = \dot{\rho}^{2} + \rho^{2} \, \dot{\phi}^{2} + \dot{z}^{2}
\end{align*}
\pause
El siguiente paso es calcular las derivadas parciales y ordinarias.
\end{frame}
\begin{frame}
\frametitle{Calculando las componentes}
Así tenemos que:
\begin{eqnarray*}
\begin{aligned}
a_{\rho} &= \va{a} \cdot \vu{e}_{\rho} = \\[0.5em] \pause
&= \bigg[ \dv{t} \, \pdv{\dot{q}_{l}} - \pdv{q_{l}} \bigg] \left( \dfrac{\dot{\rho}^{2} + \rho^{2} \, \dot{\phi}^{2} + \dot{z}^{2}}{2} \right) = \\[0.5em] \pause
&= \ddot{\rho} - \rho \, \dot{\phi}^{2}
\end{aligned}
\end{eqnarray*}
\end{frame}
\begin{frame}
\frametitle{Calculando las otras componentes}
\begin{eqnarray*}
\begin{aligned}
a_{\phi} &= \va{a} \cdot (\rho \, \vu{e}_{\phi}) = \\[0.5em] \pause
&= \dv{t} (\rho^{2} \, \phi) = \\[0.5em] \pause
&= \rho^{2} \, \ddot{\phi} + 2 \, \rho \, \dot{\rho} \, \dot{\phi}
\end{aligned}
\end{eqnarray*}
\end{frame}
\begin{frame}
\frametitle{Calculando las otras componentes}
\begin{eqnarray*}
\begin{aligned}
a_{z} &= \va{a} \cdot \vu{k} = \\[0.5em] \pause
&= \dv{t} \dot{z} = \\[0.5em] \pause
&= \ddot{z}
\end{aligned}
\end{eqnarray*}
\end{frame}

\section{Ejercicio: Coordenadas Parabólicas}
\frame{\tableofcontents[currentsection, hideothersubsections]}
\subsection{Obtener la velocidad y aceleración}

\begin{frame}
\frametitle{Reglas de transformación}
Para finalizar veamos un ejemplo del movimiento de una partícula en un sistema de coordenadas parabólico, las reglas de transformación son:
\pause
\begin{align}
\begin{aligned}
x &= \varepsilon \, \eta \, \cos \varphi \\[0.3em]
y &= \varepsilon \, \eta \, \sin \varphi \\[0.3em]
z &= \dfrac{\varepsilon^{2} - \eta^{2}}{2}
\end{aligned}
\label{eq:ecuacion_01_16}
\end{align}
\end{frame}
\begin{frame}
\frametitle{Base vectorial}
Construimos la base vectorial:
\pause
\begin{align}
\begin{aligned}
\va{b}_{\varepsilon} &= \pdv{\va{r}}{\varepsilon} = \vu{i} \, \eta \, \cos \varphi + \vu{j} \, \eta \, \sin \varphi + \vu{k} \, \varepsilon \\[0.5em]
\va{b}_{\eta} &= \pdv{\va{r}}{\eta} = \vu{i} \, \varepsilon \, \cos \varphi + \vu{j} \, \varepsilon \, \sin \varphi - \vu{k} \, \eta \\[0.5em]
\va{b}_{\varphi} &= \pdv{\va{r}}{\varphi} = - \vu{i} \, \varepsilon \, \eta \, \sin \varphi + \vu{j} \, \varepsilon \, \eta \, \cos \varphi
\end{aligned}
\label{eq:ecuacion_01_17}
\end{align}
\end{frame}
\begin{frame}
\frametitle{Factores de escala}
Calculamos los factores de escala:
\pause
obtenemos los factores de escala:
\begin{align}
\begin{aligned}
h_{\varepsilon} &= \abs{\va{b}_{\varepsilon}} = \sqrt{\varepsilon^{2} + \eta^{2}} \\[0.5em]
h_{\eta} &= \abs{\va{b}_{\eta}} = \sqrt{\varepsilon^{2} + \eta^{2}} \\[0.5em]
h_{\varphi} &= \abs{\va{b}_{\varphi}} = \varepsilon \, \eta
\end{aligned}
\label{eq:ecuacion_01_18}
\end{align}
\end{frame}
\begin{frame}
\frametitle{Vectores no coplanares de la base}
Verificamos la condición:
\pause
\begin{align}
\va{b}_{\varepsilon} \cdot \va{b}_{\eta} \cp \va{b}_{\varphi} = \left( \varepsilon^{2} + \eta^{2} \right) \, \varepsilon \, \eta
\label{eq:ecuacion_01_19}
\end{align}
Vemos que el Jacobiano es distinto de cero, por lo tanto, los vectores de la base son no coplanares.
\end{frame}
\begin{frame}
\frametitle{Vectores ortogonales}
Revisamos la condición de ortogonalidad entre los vectores de la base:
\pause
\begin{align}
\va{b}_{\varepsilon} \cdot \va{b}_{\eta} = \va{b}_{\varepsilon} \cdot \va{b}_{\varphi} = \va{b}_{\eta} \cdot \va{b}_{\varphi} = 0
\label{eq:ecuacion_01_20}
\end{align}
\end{frame}
\begin{frame}
\frametitle{Base recíproca}
Tomando la definición:
\pause
\begin{align*}
\va{b}_{l}^{*} = \dfrac{\va{b}_{l}}{h_{l}}
\end{align*}
\pause
Se construye la base recíproca:
\pause
\begin{align}
\begin{aligned}
\va{b}_{\varepsilon}^{*} &= \dfrac{\vu{i} \, \eta \, \cos \varphi + \vu{j} \, \eta \, \sin \varphi + \vu{k} \, \varepsilon}{\sqrt{\varepsilon^{2} + \eta^{2}}} \\[0.5em]
\va{b}_{\eta}^{*} &= \dfrac{\vu{i} \, \varepsilon \, \cos \varphi + \vu{j} \, \varepsilon \, \sin \varphi - \vu{k} \, \eta}{\sqrt{\varepsilon^{2} + \eta^{2}}} \\[0.5em]
\va{b}_{\varphi}^{*} &= - \vu{i} \, \sin \varphi + \vu{j} \, \cos \varphi
\end{aligned}
\label{eq:ecuacion_01_21}
\end{align}
\end{frame}
\begin{frame}
\frametitle{\textbf{Ejercicio de la sesión: se entrega hoy}}
Verifica que la base recíproca se obtiene por:
\begin{align*}
\va{b}_{\varepsilon}^{*} &= \dfrac{\vb{b}_{\eta} \times \vb{b}_{\varphi}}{\vb{b}_{\varepsilon} \cdot \vb{b}_{\eta} \times \vb{b}_{\varphi}} \\[0.35em]
\va{b}_{\eta}^{*} &= \dfrac{\vb{b}_{\varphi} \times \vb{b}_{\varepsilon}}{\vb{b}_{\varepsilon} \cdot \vb{b}_{\eta} \times \vb{b}_{\varphi}} \\[0.35em]
\va{b}_{\phi}^{*} &= \dfrac{\vb{b}_{\varepsilon} \times \vb{b}_{\eta}}{\vb{b}_{\varepsilon} \cdot \vb{b}_{\eta} \times \vb{b}_{\varphi}}
\end{align*}
Recuerda anotar tu nombre en el chat de Zoom y envía un  mensaje directo al Profesor.
\end{frame}
\begin{frame}
\frametitle{El tensor métrico}
Sabemos que el tensor métrico para este sistema coordenado es:
\begin{align}
\vb{g} = \mqty(
\varepsilon^{2} + \eta^{2} & 0 &  0 \\
0 & \varepsilon^{2} + \eta^{2} &  0 \\
0 & 0 & \varepsilon^{2} \, \eta^{2} )
\label{eq:ecuacion_01_22}
\end{align}
\end{frame}
\begin{frame}
\frametitle{Energía de la variedad}
Con la ec. (\ref{eq:ecuacion_01_22}) y ec. (\ref{eq:ecuacion_01_12}) escribimos la energía de la variedad:
\pause
\begin{align}
\dfrac{v^{2}}{2} = \dfrac{1}{2} \bigg[ \big( \varepsilon^{2} + \eta^{2} \big) \, \big( \dot{\varepsilon}^{2} + \dot{\eta}^{2} \big) + \varepsilon^{2} \, \eta^{2} \, \dot{\varphi}^{2} \bigg]
\label{eq:ecuacion_01_23}
\end{align}  
\end{frame}
\begin{frame}
\frametitle{Componentes de la aceleración}
Para obtener las componentes de la aceleración, ocupamos la expresión:
\pause
\begin{align*}
\bigg[ \dv{t} \, \pdv{\dot{q}_{l}} - \pdv{q_{l}} \bigg] \left( \dfrac{v^{2}}{2} \right)
\end{align*}
\pause
Para abreviar se mostrará el resultado obtenido, pero no es mala idea que realicen las correspondientes derivadas parciales y ordinarias.
\end{frame}
\begin{frame}
\frametitle{Componentes de la aceleración}
Para $a_{\varepsilon}$:
\pause
\begin{align*}
a_{\varepsilon} &= \bigg[ \dv{t} \pdv{\dot{\varepsilon}} - \pdv{\varepsilon} \bigg] \left( \dfrac{v^{2}}{2} \right) = \\[0.5em] 
&= \textcolor{blue}{\ddot{\varepsilon} {+} \dfrac{\varepsilon \dot{\varepsilon}^{2}}{\varepsilon^{2} {+} \eta^{2}} {-} \dfrac{\varepsilon \dot{\eta}^{2}}{\varepsilon^{2} {+} \eta^{2}} {-} \dfrac{\varepsilon \eta^{2} \dot{\varphi}^{2}}{\varepsilon^{2} {+} \eta^{2}} {+} \dfrac{2 \eta \dot{\eta} \dot{\varepsilon}}{\varepsilon^{2} {+} \eta^{2}}}
\end{align*}
\end{frame}
\begin{frame}
\frametitle{Componentes de la aceleración}
Para $a_{\eta}$:
\pause
\begin{align*}
a_{\eta} &= \bigg[ \dv{t} \pdv{\dot{\eta}} - \pdv{\eta} \bigg] \left( \dfrac{v^{2}}{2} \right) = \\[0.5em]
&= \textcolor{blue}{\ddot{\eta} {+} \dfrac{\eta \dot{\eta}^{2}}{\varepsilon^{2} {+} \eta^{2}} {-} \dfrac{\eta \dot{\varepsilon}^{2}}{\varepsilon^{2} {+} \eta^{2}} {-} \dfrac{\varepsilon^{2} \eta \dot{\varphi}^{2}}{\varepsilon^{2} {+} \eta^{2}} {+} \dfrac{2 \varepsilon \dot{\varepsilon} \dot{\eta}}{\varepsilon^{2} {+} \eta^{2}}}
\end{align*}
\end{frame}
\begin{frame}
\frametitle{Componentes para la aceleración}
Para $a_{\varphi}$:
\pause
\begin{align*}
a_{\varphi} &= \bigg[ \dv{t} \pdv{\dot{\varphi}} - \pdv{\varphi} \bigg] \left( \dfrac{v^{2}}{2} \right) = \\[0.5em]
&=  \textcolor{blue}{\ddot{\varphi} + \dfrac{2 \dot{\varepsilon} \dot{\varphi}}{\varepsilon} + \dfrac{2 \dot{\eta} \dot{\varphi}}{\eta}}
\end{align*}
\end{frame}
\begin{frame}
\frametitle{\textbf{Ejercicio de la sesión: se entrega mañana}}
Desarrolla explícitamente la expresión:
\begin{align*}
\bigg[ \dv{t} \pdv{\dot{\varepsilon}} - \pdv{\varepsilon} \bigg] \left( \dfrac{1}{2} \bigg[ \big( \varepsilon^{2} + \eta^{2} \big) \, \big( \dot{\varepsilon}^{2} + \dot{\eta}^{2} \big) + \varepsilon^{2} \, \eta^{2} \, \dot{\varphi}^{2} \bigg] \right)
\end{align*}
\end{frame}

\subsection*{Caso Especial}

\begin{frame}
\frametitle{Caso particular}
Para el caso particular del movimiento de una partícula libre la ec. (\ref{eq:ecuacion_01_14}) define las trayectorias geodésicas de la partícula:
\begin{align*}
\left( \dv{t} \, \pdv{\dot{q}_{l}} - \pdv{q_{l}} \right) &\dfrac{v^{2}}{2} = 0 \\[0.5em] 
\left( \dv{t} \, \pdv{\dot{q}_{l}} - \pdv{q_{l}} \right) &\dfrac{g_{lm} \, \dot{q}_{l} \, \dot{q}_{m}}{2} = 0
\end{align*}
\end{frame}
\begin{frame}
\frametitle{El Lagrangiano}  
El operador:
\begin{align*}
\left( \dv{t} \, \pdv{\dot{q}_{l}} - \pdv{q_{l}} \right)
\end{align*}
es conocido en la literatura como \emph{operador Lagrangiano de primer orden}.
\end{frame}
\begin{frame}
\frametitle{Las trayectorias geodésicas}
La forma general de una geodésica es:
\pause
\begin{align}
\ddot{q}_{l} + \mathlarger{\Gamma}_{mk}^{l} \, \dot{q}_{m} \, \dot{q}_{k} = 0
\label{eq:ecuacion_01_25}
\end{align}
\end{frame}
\begin{frame}
\frametitle{Las geodésicas}
Identificando en la ecs. para $a_{\varepsilon}, a_{\eta}, a_{\varphi}$ los términos en color azul, y con la ec. (\ref{eq:ecuacion_01_25}) se pueden leer los términos $\mathlarger{\Gamma}_{mk}^{l}$ que son conocidos en la literatura como los \emph{símbolos de Christoffel}.
\end{frame}

\end{document}

