\documentclass[hidelinks,12pt]{article}
\usepackage[left=0.25cm,top=1cm,right=0.25cm,bottom=1cm]{geometry}
%\usepackage[landscape]{geometry}
\textwidth = 20cm
\hoffset = -1cm
\usepackage[utf8]{inputenc}
\usepackage[spanish,es-tabla]{babel}
\usepackage[autostyle,spanish=mexican]{csquotes}
\usepackage[tbtags]{amsmath}
\usepackage{nccmath}
\usepackage{amsthm}
\usepackage{amssymb}
\usepackage{mathrsfs}
\usepackage{graphicx}
\usepackage{subfig}
\usepackage{standalone}
\usepackage[outdir=./Imagenes/]{epstopdf}
\usepackage{siunitx}
\usepackage{physics}
\usepackage{color}
\usepackage{float}
\usepackage{hyperref}
\usepackage{multicol}
%\usepackage{milista}
\usepackage{anyfontsize}
\usepackage{anysize}
%\usepackage{enumerate}
\usepackage[shortlabels]{enumitem}
\usepackage{capt-of}
\usepackage{bm}
\usepackage{relsize}
\usepackage{placeins}
\usepackage{empheq}
\usepackage{cancel}
\usepackage{wrapfig}
\usepackage[flushleft]{threeparttable}
\usepackage{makecell}
\usepackage{fancyhdr}
\usepackage{tikz}
\usepackage{bigints}
\usepackage{scalerel}
\usepackage{pgfplots}
\usepackage{pdflscape}
\pgfplotsset{compat=1.16}
\spanishdecimal{.}
\renewcommand{\baselinestretch}{1.5} 
\renewcommand\labelenumii{\theenumi.{\arabic{enumii}})}
\newcommand{\ptilde}[1]{\ensuremath{{#1}^{\prime}}}
\newcommand{\stilde}[1]{\ensuremath{{#1}^{\prime \prime}}}
\newcommand{\ttilde}[1]{\ensuremath{{#1}^{\prime \prime \prime}}}
\newcommand{\ntilde}[2]{\ensuremath{{#1}^{(#2)}}}

\newtheorem{defi}{{\it Definición}}[section]
\newtheorem{teo}{{\it Teorema}}[section]
\newtheorem{ejemplo}{{\it Ejemplo}}[section]
\newtheorem{propiedad}{{\it Propiedad}}[section]
\newtheorem{lema}{{\it Lema}}[section]
\newtheorem{cor}{Corolario}
\newtheorem{ejer}{Ejercicio}[section]

\newlist{milista}{enumerate}{2}
\setlist[milista,1]{label=\arabic*)}
\setlist[milista,2]{label=\arabic{milistai}.\arabic*)}
\newlength{\depthofsumsign}
\setlength{\depthofsumsign}{\depthof{$\sum$}}
\newcommand{\nsum}[1][1.4]{% only for \displaystyle
    \mathop{%
        \raisebox
            {-#1\depthofsumsign+1\depthofsumsign}
            {\scalebox
                {#1}
                {$\displaystyle\sum$}%
            }
    }
}
\def\scaleint#1{\vcenter{\hbox{\scaleto[3ex]{\displaystyle\int}{#1}}}}
\def\bs{\mkern-12mu}


\title{Coordenadas parabólicas\\ \large{Matemáticas Avanzadas de la Física}\vspace{-3ex}}
\author{M. en C. Abraham Lima Buendía}
\date{ }
\begin{document}
\vspace{-4cm}
\maketitle
\fontsize{14}{14}\selectfont
En este material haremos el desarrollo paso a paso para estudiar un sistema coordenado, de tal manera que podrás abordar un sistema coordenado distinto.
\section{Sistema de coordenadas parabólico.}
Considera las reglas de transformación:
\begin{align}
\begin{aligned}
x &= \sigma \, \tau \, \cos \varphi \\[0.5em]
y &= \sigma \, \tau \, \sin \varphi \\[0.5em]
z &= \dfrac{1}{2} (\tau^2 - \sigma^{2})
\end{aligned}
\label{eq:ecuacion_01}
\end{align}
\textbf{Importante: } Considera que la notación de las coordenadas puede cambiar de acuerdo con el autor o texto de referencia, por ejemplo, en \cite{Boas} se indica que el sistema parabólico utiliza las coordenadas $(u, v, \phi)$, mientras que en \cite{Moon}, se ocupan las coordenadas $(\mu, \nu, z)$. La notación es diferente, pero el estudio a realizar, es el mismo.
\section{Superficies coordenadas.}
Para analizar las superficies, debemos de hacer que $\sigma = \mbox{ cte.}$, $\tau = \mbox{ cte.}$ y $\varphi = \mbox{ cte.}$.
\subsection{Cuando $\varphi$ es constante.}
Si consideramos
\begin{align*}
\dfrac{y}{x} = \tan \varphi \hspace{0.3cm} \Rightarrow \hspace{0.3cm} \varphi = \mbox{ cte.}
\end{align*}
Por lo que tenemos la ecuación de las familias de rectas:
\begin{align*}
y = x \, \tan \varphi
\end{align*}
\subsection{Cuando $\sigma$ es constante.}
Sumando el cuadrado de las coordenadas $x$ e $y$:
\begin{align*}
x^{2} + y^{2} = \sigma^{2} \, \tau^{2} \hspace{0.3cm} \Rightarrow \hspace{0.3cm} 2 \, z = \dfrac{(x^{2} + y^{2})}{\sigma^{2}} - \sigma^{2}
\end{align*}
La ecuación resultante representa una familia de paraboloides confocales cuya apertura es en la dirección de $z > 0$.
\subsection{Cuando $\tau$ es constante.}
Nuevamente sumamos el cuadrado de las coordenadas $x$ e $y$ para obtener:
\begin{align*}
2 \, z = - \left( \dfrac{x^{2} + y^{2}}{\tau^{2}}\right) + \tau^{2}
\end{align*}
Lo que nos dice que tenemos una familia de paraboloides con apertura hacia $z < 0$.
\section{Base vectorial.}
Para construir la base vectorial del sistema parabólico, consideremos que:
\begin{align*}
\va{r} = \sigma \, \tau \cos \varphi \, \vu{i} + \sigma \, \tau \sin \varphi \, \vu{j} + \dfrac{1}{2} (\tau^{2} - \sigma^{2}) \, \vu{k}
\end{align*}
Entonces los vectores unitarios son:
\begin{align*}
\va{e}_{\sigma} &= \tau \, \cos \varphi \, \vu{i} + \tau \, \sin \varphi \, \vu{j} + \sigma \, \vu{k} \\
\va{e}_{\tau} &= \sigma \, \cos \varphi \, \vu{i} + \sigma \, \sin \varphi \, \vu{j} + \tau \, \vu{k} \\
\va{e}_{\varphi} &= - \sigma \, \tau \, \sin \varphi \, \vu{i} + \sigma \, \tau \, \cos \varphi \, \vu{j}
\end{align*}
Por lo que
\begin{align*}
\vu{e}_{\sigma} &= \dfrac{\tau \, \cos \varphi \, \vu{i} + \tau \, \sin \varphi \, \vu{j} - \sigma \, \vu{k}}{\sqrt{\tau^{2} + \sigma^{2}}} \\
\vu{e}_{\tau} &= \dfrac{\sigma \, \cos \varphi \, \vu{i} + \sigma \, \sin \varphi \, \vu{j} + \tau \, \vu{k}}{\sqrt{\tau^{2} + \sigma^{2}}} \\
\vu{e}_{\varphi} &= - \sin \varphi \, \vu{i} + \cos \varphi \, \vu{j}
\end{align*}
También podemos expresar los vectores unitarios del sistema cartesiano en términos de la base del sistema coordenado parabólico:
\begin{align*}
\vu{i} &= \dfrac{\sigma \, \cos \varphi}{\sigma^{2} + \tau^{2}} \, \ket{\tau} + \dfrac{\tau \, \cos \varphi}{\sigma^{2} + \tau^{2}} \, \ket{\sigma} + \dfrac{\sin \varphi}{\sigma \, \tau} \, \ket{\varphi} \\[0.5em]
\vu{j} &= \dfrac{\sigma \, \sin \varphi}{\sigma^{2} + \tau^{2}} \, \ket{\tau} + \dfrac{\tau \, \sin \varphi}{\sigma^{2} + \tau^{2}} \, \ket{\sigma} + \dfrac{\cos \varphi}{ \sigma \, \tau} \, \ket{\varphi} \\[0.5em]
\vu{k} &= \dfrac{\tau}{\sigma^{2} + \tau^{2}} \, \ket{\tau} + \dfrac{\sigma}{\sigma^{2} + \tau^{2}} \, \ket{\sigma}
\end{align*}
\section{Matriz métrica y factores de escala.}
Los elementos no nulos de la matriz métrica son:
\begin{align*}
g_{\sigma \sigma} &= \tau^{2} + \sigma^{2} \\[0.5em]
g_{\tau \tau} &= \tau^{2} + \sigma^{2} \\[0.5em]
g_{\varphi \varphi} &= \tau^{2} \, \sigma^{2}
\end{align*}
Mientras que los elementos nulos son:
\begin{align*}
g_{\tau \sigma} = g_{\sigma \tau} = g_{\tau \varphi} = g_{\varphi \tau} = g_{\sigma \varphi} = g_{\varphi \sigma} = 0
\end{align*}
Los factores de escala son:
\begin{align*}
h_{\sigma} &= \sqrt{\tau^{2} + \sigma^{2}} \\[0.5em]
h_{\tau} &= \sqrt{\tau^{2} + \sigma^{2}} \\[0.5em]
h_{\varphi} &= \sigma \, \tau
\end{align*}
\section{Velocidad y energía.}
La velocidad de una partícula en este sistema de coordenadas parabólicas, viene dada por:
\begin{align*}
v = \dot{q} \, \vu{e}_{a} = \left( \dot{\sigma} \, \va{e}_{\varphi} + \dot{\tau} \, \va{e}_{\tau} + \dot{\varphi} \, \vu{e}_{\varphi} \right)
\end{align*}
Una vez conocida la velocidad, ya podemos calcular la energía de la partícula:
\begin{align*}
E &= \dfrac{1}{2} m \, v^{2} = \\[0.5em]
&= \dfrac{m}{2} \left[ (\sigma^{2} + \tau^{2}) (\dot{\sigma}^{2} + \dot{\tau}^{2}) + \sigma^{2} \, \tau^{2} \, \dot{\varphi}^{2} \right]
\end{align*}
\section{Operadores diferenciales.}
Por último se requiere presentar la expresión de los operadores diferenciales $\grad{\psi}$, $\div{\vb{V}}$, $\curl{\vb{V}}$ y $\laplacian{\psi}$ en el sistema coordenado parabólico:
\begin{align*}
\grad{\psi} &= \dfrac{1}{\sqrt{\sigma^{2} + \tau^{2}}} \left( \vu{e}_{\sigma} \, \pdv{\psi} + \vu{e}_{\tau} \, \pdv{\psi}{\tau} \right) + \dfrac{\vu{e}_{\varphi}}{\sigma \tau} \, \pdv{\psi}{\varphi} \\[1em]
\div{\vb{V}} &= \dfrac{1}{(\sigma^{2} + \tau^{2}) \sigma \tau} \left[ \pdv{\sigma} V_{\sigma} \, \left(\sqrt{\sigma^{2} + \tau^{2}} \, \sigma \, \tau \right)  +  \pdv{\tau} V_{\tau} \, \left( \sqrt{\sigma^{2} + \tau^{2}} \, \sigma \, \tau\right) + \right. \\[0.5em]
&+ \left. \pdv{\varphi} V_{\varphi} \, (\sigma^{2} + \tau^{2})  \right] \\[1em]
\curl{\vb{V}} &= \dfrac{1}{\sigma \tau (\sigma^{2} + \tau^{2})} \, \mdet{
h_{\sigma} \, \va{e}_{\sigma} & h_{\tau} \, \va{e}_{\tau} & h_{\varphi} \, \va{e}_{\varphi} \\
\displaystyle \pdv{\sigma} & \displaystyle \pdv{\tau} & \displaystyle \pdv{\varphi} \\
h_{\sigma} \, V_{\sigma} & h_{\tau} \, V_{\tau} & h_{\varphi} \, V_{\varphi}
} \\[1em]
\laplacian{\psi} &= \dfrac{1}{\sigma \tau (\sigma^{2} + \tau^{2})} \, \left[ \pdv{\sigma} \sigma \, \tau \pdv{\psi}{\sigma} + \pdv{\tau} \sigma \, \tau \pdv{\psi}{\tau} + \dfrac{(\sigma^{2} + \tau^{2})}{\sigma \tau} \, \pdv[2]{\psi}{\varphi} \right]
\end{align*}


\vfill
\begin{thebibliography}{X}
\bibitem{Boas}{\textsc{Boas, Mary L.}, \textit{Mathematical methods in the physical sciences}, 3th. ed., John Wiley, USA, 2006.}
\bibitem{Moon}{\textsc{Moon, P.} y \textsc{Spencer, D. E.}, \textit{Field theory handbook}, 2nd. ed., Springer-Verlag, Berlín, 1971.}
\end{thebibliography}
\end{document}