\documentclass[12pt]{article}
%\usepackage[left=0.25cm,top=1cm,right=0.25cm,bottom=1cm]{geometry}
\usepackage{geometry}
\textwidth = 20cm
\hoffset = -1cm
\usepackage[utf8]{inputenc}
\usepackage[spanish,es-tabla]{babel}
\usepackage{amsmath}
\usepackage{nccmath}
\usepackage{amsthm}
\usepackage{amssymb}
\usepackage{graphicx}
\usepackage{color}
\usepackage{float}
\usepackage{multicol}
\usepackage{enumerate}
\usepackage{anyfontsize}
\usepackage{anysize}
\usepackage{enumitem}
\usepackage{capt-of}
\usepackage{bm}
\usepackage{relsize}
\spanishdecimal{.}
\setlist[enumerate]{itemsep=0mm}
\renewcommand{\baselinestretch}{1.2}
\let\oldbibliography\thebibliography
\renewcommand{\thebibliography}[1]{\oldbibliography{#1}
\setlength{\itemsep}{0pt}}
%\marginsize{1.5cm}{1.5cm}{0cm}{2cm}
\title{Tarea 1 - Temas 1 y 2 \\ \large{Matemáticas Avanzadas de la Física}}
%\subtitle{Fecha de entrega: 8 de marzo de 2016.}
\date{ }
\begin{document}
\vspace{-4cm}
%\renewcommand\theenumii{\arabic{theenumii.enumii}}
\renewcommand\labelenumii{\theenumi.{\arabic{enumii}}}
\maketitle
\fontsize{14}{14}\selectfont
Para las respuestas de la tarea, te pedimos sea lo más claro posible, presentando de manera organizada tu solución. La calificación de esta tarea está en función del número de ejercicios que entregues, cuentas con el suficiente tiempo para resolverla y hacer un buen esfuerzo.
\begin{enumerate}
\item Una carga $q$ se está moviendo a velocidad constante $v$ sobre el eje positivo $x$. Otras dos cargas $-q$ y $2q$ se mueven a velocidad constante $v$ y $2v$ a lo largo del eje positivo $y$ y el eje negativo $z$, respectivamente. Suponemos que en $t=0$, la carga $q$ está en el origen, $-q$ está en $(0,a,0)$ y $2q$ está en $(0,0,a)$.
\begin{enumerate}
\item Calcula las componentes cartesianas del campo magnético en el punto $(x,y,z)$ para $t>0$.
\item Calcula las componentes cilíndricas del campo magnético en el punto $(\rho, \varphi, z)$ para $t>0$.
\item Calcula las componentes esféricas del campo magnético en el punto $(r, \theta, \varphi)$ para $t>0$.
\end{enumerate}
\item Para el siguiente sistema de coordenadas
\begin{eqnarray*}
x &=& f \sqrt{(\xi^{2} -1) (1 - \eta^{2}} \; \cos \varphi \nonumber \\
y &=& f \sqrt{(\xi^{2} -1) (1 - \eta^{2}} \; \sin \varphi \nonumber \\
z &=& f \; \xi \; \eta \nonumber
\end{eqnarray*}
\begin{enumerate}
\item Estudiar las curvas $\varphi = \mbox{constante}, \xi = \mbox{constante, y} \eta = \mbox{constante}$.
\item Determine los nuevos vectores base en términos de los vectores canónicos $\mathbf{\widehat{i}}$, $\mathbf{\widehat{j}}$, $\mathbf{\widehat{k}}$.
\item Calcule los factores de escala y muestre que el sistema es ortogonal, encuentre la matriz métrica de este sistema.
\item Encuentre la expresión para la velocidad y la energía cinética.
\item Escriba los operadores gradiente, divergencia, rotacional y laplaciano para este sistema de coordenadas.
\end{enumerate}
\item Dado el vector $\mathbf{A} = (x^{2} - y^{2}) \mathbf{i} +  2xy \mathbf{j}$
\begin{enumerate}
\item Calcula $\bm{\nabla} \times \mathbf{A}$.
\item Evalúa $\mathlarger{\iint} (\bm{\nabla} \times \mathbf{A}) \cdot d \sigma $ en un rectángulo en el plano $x-y$ cercado por las líneas $x=0$, $x=a$, $y=0$, $y=b$.
\item Evalúa $\mathlarger{\oint} \mathbf{A} \cdot d \mathbf{r}$ alrededor de la frontera del rectángulo y verifica el teorema de Stokes para este ejemplo.
\end{enumerate}
\item Demuestra que el operador momento angular en coordenadas esféricas, está dado por:
\[ \mathbf{L} = - i (r \times \bm{\nabla}) = i \left( \bm{\theta}_{0} \dfrac{1}{\sin \theta} \; \dfrac{\partial}{\partial \varphi} - \bm{\varphi}_{0} \dfrac{\partial}{\partial \theta} \right)  \]
\item Partiendo del problema anterior, encuentra los operadores $L_{x}$, $L_{y}$, $L_{z}$, los operadores $L_{+}$, $L_{-}$ y el operador $L^{2}$.
\item Partiendo del Lagrangiano de una partícula libre en coordenadas esféricas:
\[ L = \dfrac{m}{2} \left( \dot{r}^{2} + r^{2} \dot{\theta}^{2} + r^{2} \sin \theta^{2} \dot{\varphi}^{2} \right)\]
Calcula los símbolos de Christoffel.
\item Las ecuaciones de Navier-Stokes para el flujo de un fluido incompresible
\[ - \bm{\nabla} \times ( \mathbf{v} \times (\bm{\nabla} \times \mathbf{v} )) =  \dfrac{\eta}{\rho_{0}} \bm{\nabla}^{2} (\bm{\nabla} \times \mathbf{v}) \]
Donde $\eta$ es la viscosidad y $\rho_{0}$ la densidad del fluido. Para un flujo axial dentro de un cilindro, consideremos que la velocidad $\mathbf{v}$ está dada por
\[ \mathbf{v} =  \mathbf{k} v (\rho) \]
Considera que
\[ \bm{\nabla} \times (\mathbf{v} \times (\bm{\nabla} \times \mathbf{v})) = 0 \]
para este valor de $\mathbf{v}$.
\\
Demuestra que
\[ \bm{\nabla}^{2} ( \bm{\nabla} \times \mathbf{v}) = 0  \]
nos lleva a la ecuación diferencial
\[ \dfrac{1}{\rho} \dfrac{d}{d \rho} \left( \rho \dfrac{d^{2} v}{d \rho^{2}} \right) -  \dfrac{1}{\rho} \dfrac{d v}{d \rho} = 0 \]
y que la siguiente expresión es solución de la misma ecuación diferencial
\[ v = v_{0} + a_{2} \rho^{2} \]
\item El cálculo de efecto ''pinch'' en magnetohidrodinámica, involucra la evaluación de $(\mathbf{B} \cdot \bm{\nabla}) \mathbf{B}$. Si la inducción magnética $\mathbf{B}$ se toma como $\mathbf{B} = \bm{\varphi}_{0} B_{\varphi} (\rho)$, demostrar que
\[ (\mathbf{B} \cdot \bm{\nabla}) \mathbf{B} = - \dfrac{\bm{\rho_{0}} B_{\varphi}^{2}}{\rho}	 \]
\item Un cierto campo de fuerza está dado por
\[ \mathbf{F} = \mathbf{r}_{0} \dfrac{2 P \cos \theta}{r^{3}} + \bm{\theta}_{0} \dfrac{P}{r^{3}} \sin \theta, \hspace{1.5cm} r \geq P/2 \]
en coordenadas polares.
\begin{enumerate}
\item Revisa $\bm{\nabla} \times \mathbf{F}$ para checar si existe un potencial.
\item Calcular $\mathlarger{\oint} \mathbf{F} \cdot d \bm{\lambda}$ para un círculo unitario en el plano $\theta = \pi/2$.\
¿Qué nos dice esto sobre la fuerza?¿Es conservativa o no conservativa?
\item Si considera que $\mathbf{F}$ se puede describir por $\mathbf{F} = - \bm{\nabla}\psi$, encuentra $\psi$. De otra manera argumenta que no es posible que un potencial exista.
\end{enumerate}
%\newpage
\item Mostrar que la ecuación de Helmholtz
\[ \nabla^{2} \psi + k^{2} \psi = 0 \]
Es separable en coordenadas cilíndricas, si $k^{2}$ se generaliza como $k^{2} + f(\rho) + (1/\rho^{2}) g(\varphi) +  h(z)$. 
\item Demuestra que
\[ \nabla^{2} \psi(r,\theta,\varphi) + \left[ k^{2} + f(\rho) + \dfrac{1}{\rho^{2}} g(\theta) + \dfrac{1}{r^{2}\sin^{2} \theta} h(\varphi) \right] \psi (r,\theta,\varphi) = 0 \]
es separable (en coordenadas esféricas). Las funciones $f,g,h$ son funciones sólo de las variables que se indican, $k^{2}$ es una constante.
\item Para una esfera sólida homogénea con constante de difusión términa $K$, la ecuación de conducción de calor (sin fuentes) es
\[ \dfrac{\partial T(r,t)}{\partial t} =  K \nabla^{2} T(r,t) \]
Mediante la técnica de separación de variables, suponemos que tiene una solución de la forma
\[ T =R(r) T(t) \]
Demuestra que la ecuación radial toma la forma estándar
\[ r^{2} \dfrac{d^{2} R}{d r^{2}} + 2r \dfrac{d R}{d r} + \left[ \alpha^{2} r^{2} - n(n+1) \right] R = 0, \hspace{1cm} n = \mbox{ entero} \]
\item Resuelve los siguientes problemas de tipo Sturm-Liouville, y demuestra que los valores y funciones propias, son los que se indican:
\begin{enumerate}[label=(\alph*)]
\setlength\itemsep{1em}
\begin{fleqn}
\item  $y'' + \lambda y = 0$ con $y(0) = y'(L) = 0$ \\ en donde $\lambda_{n} = \dfrac{(2n-1)^{2} \pi^{2}}{4L^{2}}$ y $y_{n}(x) = \dfrac{(2n-1)^{2} \pi x}{2L}$ para $n \geq 1$.
\item $ y''+ \lambda y = 0 $ con $y'(0) = h y(L) + y'(L) = 0 \hspace{0.5cm} (h > 0)$ \\
en donde $\lambda_{n} =  \frac{\beta^{2}}{L^{2}}$ y $y_{n} = \cos \frac{\beta_{n} x}{L}$ para $n \geq 1$, donde $\beta_{n}$ es la $n$-ésima raíz positiva de $\tan x = \frac{hL}{x}$. Para estimar los valores de $\beta_{n}$ para $n$ grande, grafica $y = \tan x$ y $ y=\frac{hL}{x}$.
\end{fleqn}
\end{enumerate}
\item Reducir cada ecuación a una ecuación de valores propios y a otra ecuación con condiciones iniciales, y luego calcular las soluciones particulares:
\begin{enumerate}[label=(\alph*)]
\item \begin{fleqn}
\[ \dfrac{\partial^{2} u}{\partial t^{2}} - \dfrac{\partial^{2} u}{\partial x^{2}} - u = 0 \hspace{1cm} \text{para } 0 < x < 1, t>0 \]
\[ \dfrac{\partial^{2} u}{\partial t} (x,0) = 0\]
\[ u(0,t) = u(1,t) = 0\]
\item \[ \dfrac{\partial^{2} u}{\partial t^{2}} + 2 \dfrac{\partial u}{\partial t} - 4 \dfrac{\partial^{2} u}{\partial x^{2}} +  u = 0 \hspace{1cm} \text{para } 0 < x < 1, t>0 \]
\[ u(x,0) = 0\]
\[ \dfrac{\partial u}{\partial x} (0,t) = u(1,t) = 0\] 
\end{fleqn}
\end{enumerate}
\item Encuentra dos soluciones linealmente independientes en términos de la serie de Frobenius (para $x > 0$) en cada una de las ecuaciones diferenciales:
\begin{enumerate}[label=(\alph*)]
\begin{fleqn}
\item  $4x y'' + 2y' + y = 0 $
\item $ 2x y'' + 3y' - y = 0 $
\end{fleqn}
\end{enumerate}
\item Utiliza el método de Frobenius para obtener la solucion general de cada una de las siguientes ecuaciones diferenciales, para un entorno de $x = 0$:
\begin{enumerate}[label=(\alph*)]
\begin{fleqn}
\setlength\itemsep{1em}
\item  $ 2 x \dfrac{d^{2} y}{d x^{2}} + (1 - x^{2}) \dfrac{d y}{d x} - y = 0 $
\item $ x^{2} \dfrac{d^{2} y}{d x^{2}} + x \dfrac{d y}{d x} + (x^{2} - 1) y = 0 $
\end{fleqn}
\end{enumerate}
\item Una solución a la ecuación diferencial de Laguerre
\[ xy'' + (1-x) y' + ny = 0\]
para $n=0$ es $y_{1}(x)=1$. Desarrolla una segunda solución linealmente independiente.
\item A partir del estudio en mecánica cuántica del efecto Stark (en coordenadas parabólicas), nos conduce a la ecuación difencial
\[ \dfrac{d}{d \xi} \left( \xi \dfrac{d u}{d \xi} \right) + \left( \dfrac{1}{2} E \xi + \alpha - \dfrac{m^{2}}{4 \xi} - \dfrac{1}{4} F \xi^{2} \right) u = 0 \]
donde
\begin{enumerate}[label=(\roman*)]
\item $\alpha$ es la constante de separación.
\item $E$ es la energía total del sistema.
\item $F$ es una constante.
\item $Fz$ es la energía potencial que se agrega al introducir un campo eléctrico.
\end{enumerate}
Usando la raíz más grande de la ecuación indicial, desarrolla una solución en series de potencias, alrededor de $\xi=0$. Evalúa los primeros tres coeficientes en términos de $a_{0}$
\[  \begin{split}
& \text{Ecuación indicial } \hspace{1.5cm} k^{2} - \dfrac{m^{2}}{4} = 0 \\
u(\xi) &=  a_{0} \xi^{m/2} \left\lbrace 1 - \dfrac{\alpha}{m+1} \xi + \left[ \dfrac{\alpha^{2}}{2(m+1)(m+2)} - \dfrac{E}{4(m+2)} \right] \xi^{2} + \ldots \right\rbrace
\end{split} \]
Checa que la perturbación $E$ no se presenta hasta que el término $a_{3}$ se incluye.
\item Para el caso especial en donde no hay dependencia en la coordenada azimutal, del estudio del ion molecular del hidrógeno $(H2^{+})$ en mecánica cuántica, se llega a la ecuación
\[ \dfrac{d}{d \eta} \left[ (1 - \eta^{2} ) \dfrac{d u}{d \eta} \right] + \alpha u + \beta \eta^{2} u = 0 \]
Desarrolla una solución en series de potencias para $u(\eta)$. Evalúa los primeros tres coeficientes no nulos en términos de $a_{0}$
\[  \begin{split}
& \text{Ecuación indicial } \hspace{1.5cm} k(k-1) = 0 \\
u_{k=1} &=  a_{0} \eta \left\lbrace 1 - \dfrac{2- \alpha}{6} \eta^{2} + \left[ \dfrac{(2-\alpha)(12-\alpha)}{120} - \dfrac{\beta}{20} \right] \eta^{4} + \ldots \right\rbrace
\end{split} \]
\item Un vector de potencial magnético está dado por
\[ \mathbf{A} = \dfrac{\mu_{0}}{4 \pi} \dfrac{\mathbf{m} \times \mathbf{r}}{r^{3}} \]
Demuestra que esto nos conduce a un campo de inducción magnética $\mathbf{B}$ de un dipolo magnético, de momento $\mathbf{m}$.
\end{enumerate}
\end{document}