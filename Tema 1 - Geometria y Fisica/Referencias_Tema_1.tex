\documentclass[hidelinks,12pt]{article}
\usepackage[left=0.25cm,top=1cm,right=0.25cm,bottom=1cm]{geometry}
%\usepackage[landscape]{geometry}
\textwidth = 20cm
\hoffset = -1cm
\usepackage[utf8]{inputenc}
\usepackage[spanish,es-tabla]{babel}
\usepackage[autostyle,spanish=mexican]{csquotes}
\usepackage[tbtags]{amsmath}
\usepackage{nccmath}
\usepackage{amsthm}
\usepackage{amssymb}
\usepackage{mathrsfs}
\usepackage{graphicx}
\usepackage{subfig}
\usepackage{standalone}
\usepackage[outdir=./Imagenes/]{epstopdf}
\usepackage{siunitx}
\usepackage{physics}
\usepackage{color}
\usepackage{float}
\usepackage{hyperref}
\usepackage{multicol}
%\usepackage{milista}
\usepackage{anyfontsize}
\usepackage{anysize}
%\usepackage{enumerate}
\usepackage[shortlabels]{enumitem}
\usepackage{capt-of}
\usepackage{bm}
\usepackage{relsize}
\usepackage{placeins}
\usepackage{empheq}
\usepackage{cancel}
\usepackage{wrapfig}
\usepackage[flushleft]{threeparttable}
\usepackage{makecell}
\usepackage{fancyhdr}
\usepackage{tikz}
\usepackage{bigints}
\usepackage{scalerel}
\usepackage{pgfplots}
\usepackage{pdflscape}
\pgfplotsset{compat=1.16}
\spanishdecimal{.}
\renewcommand{\baselinestretch}{1.5} 
\renewcommand\labelenumii{\theenumi.{\arabic{enumii}})}
\newcommand{\ptilde}[1]{\ensuremath{{#1}^{\prime}}}
\newcommand{\stilde}[1]{\ensuremath{{#1}^{\prime \prime}}}
\newcommand{\ttilde}[1]{\ensuremath{{#1}^{\prime \prime \prime}}}
\newcommand{\ntilde}[2]{\ensuremath{{#1}^{(#2)}}}

\newtheorem{defi}{{\it Definición}}[section]
\newtheorem{teo}{{\it Teorema}}[section]
\newtheorem{ejemplo}{{\it Ejemplo}}[section]
\newtheorem{propiedad}{{\it Propiedad}}[section]
\newtheorem{lema}{{\it Lema}}[section]
\newtheorem{cor}{Corolario}
\newtheorem{ejer}{Ejercicio}[section]

\newlist{milista}{enumerate}{2}
\setlist[milista,1]{label=\arabic*)}
\setlist[milista,2]{label=\arabic{milistai}.\arabic*)}
\newlength{\depthofsumsign}
\setlength{\depthofsumsign}{\depthof{$\sum$}}
\newcommand{\nsum}[1][1.4]{% only for \displaystyle
    \mathop{%
        \raisebox
            {-#1\depthofsumsign+1\depthofsumsign}
            {\scalebox
                {#1}
                {$\displaystyle\sum$}%
            }
    }
}
\def\scaleint#1{\vcenter{\hbox{\scaleto[3ex]{\displaystyle\int}{#1}}}}
\def\bs{\mkern-12mu}


\usepackage{apacite}
\renewcommand{\refname}{Bibliografía.}
% \usepackage{biblatex}
% \addbibresource{Referencias_MAF.bib}
\title{Materiales para el Tema 1\\ \large{Matemáticas Avanzadas de la Física}\vspace{-3ex}}
\author{M. en C. Gustavo Contreras Mayén}
\date{ }
\begin{document}
\vspace{-4cm}
\maketitle
\fontsize{14}{14}\selectfont
En los materiales que se comparten para revisión del Tema 1 - La física y la geometría, se hace una anotación sobre los capítulos que podrán leer para extender las presentaciones y los materiales adicionales.
\par
Para una descripción desde la perspectiva de la geometría diferencial, en \cite{Bar2010} los capítulos: 1 \textsc{Euclidian geometry}, 2 \textsc{Curve theory} , 3 \textsc{Classical surface theory} y 4 \textsc{The inner geometry of surfaces}, se revisan de una manera más completa los conceptos de métrica, transporte paralelo, las cartas, entre otros. La referencia quizá demande un poco más la geometría clásica que llevamos en la carrera, pero el manejo y nivel de trabajo es aceptable.
\par
Otra forma de abordar el tema de sistemas coordenados curvilíneos vista desde los \emph{tensores}, se revisa en \cite{Nguyen2017}, en los capítulos: 2 \textsc{Tensor analysis} y 3 \textsc{Elementary differential geometry}, se presenta la construcción, pero es necesario realizar previamente la lectura del capítulo 1, ya que de esa forma, el manejo de vectores y tensores será mucho más ameno, y a su vez, les servirá más adelante.
\par
Un texto clásico en física matemática es el de \cite{Morse1953} Volumen 1, cuyo capítulo 1 es de utilidad para nuestro tema, también hace una revisión de los sistemas coordenados, tensores, operadores diferenciales; un aspecto interesante de este libro es que incluye figuras de tipo estereograma, es decir, se logra una visualización en 3D a partir de una imagen 2D, por lo que el resultado es hasta divertido, considerando que el libro se publicó en 1953, pero hoy en día sigue siendo un texto muy recurrido.
En el artículo de \cite{LeyKoo1994} al abordar el \emph{Desarrollo armónico del potencial de Coulomb en coordenadas toroidales}, se podrá revisar la construcción del sistema toroidal a partir de las reglas de transformación, así también obtener los factores de escala y presentar las ecuaciones Poisson y Laplace en este sistema toroidal. Como primera parte de revisión, consulten hasta la página $807$ del pdf, ya que posteriormente en el Tema 2 del curso: \emph{Primeras técnicas de solución}, podremos regresar a este artículo y repasar la manera en que resuelven la ecuación diferencial obtenida.
\par
\textbf{Nota: } Los materiales son de revisión, entendemos que la extensión de los capítulos es mayor que las presentaciones que hemos dejado en el apartado del Tema 1, pero dejamos los archivos para que en el tiempo que tu lo consideres, los leas y complementes las presentaciones y en su caso, te apoyen para resolver los ejercicios.
\bibliographystyle{apacite}
\bibliography{../Referencias_MAF}
\end{document}