\documentclass[hidelinks,12pt]{article}
\usepackage[left=0.25cm,top=1cm,right=0.25cm,bottom=1cm]{geometry}
%\usepackage[landscape]{geometry}
\textwidth = 20cm
\hoffset = -1cm
\usepackage[utf8]{inputenc}
\usepackage[spanish,es-tabla]{babel}
\usepackage[autostyle,spanish=mexican]{csquotes}
\usepackage[tbtags]{amsmath}
\usepackage{nccmath}
\usepackage{amsthm}
\usepackage{amssymb}
\usepackage{mathrsfs}
\usepackage{graphicx}
\usepackage{subfig}
\usepackage{standalone}
\usepackage[outdir=./Imagenes/]{epstopdf}
\usepackage{siunitx}
\usepackage{physics}
\usepackage{color}
\usepackage{float}
\usepackage{hyperref}
\usepackage{multicol}
%\usepackage{milista}
\usepackage{anyfontsize}
\usepackage{anysize}
%\usepackage{enumerate}
\usepackage[shortlabels]{enumitem}
\usepackage{capt-of}
\usepackage{bm}
\usepackage{relsize}
\usepackage{placeins}
\usepackage{empheq}
\usepackage{cancel}
\usepackage{wrapfig}
\usepackage[flushleft]{threeparttable}
\usepackage{makecell}
\usepackage{fancyhdr}
\usepackage{tikz}
\usepackage{bigints}
\usepackage{scalerel}
\usepackage{pgfplots}
\usepackage{pdflscape}
\pgfplotsset{compat=1.16}
\spanishdecimal{.}
\renewcommand{\baselinestretch}{1.5} 
\renewcommand\labelenumii{\theenumi.{\arabic{enumii}})}
\newcommand{\ptilde}[1]{\ensuremath{{#1}^{\prime}}}
\newcommand{\stilde}[1]{\ensuremath{{#1}^{\prime \prime}}}
\newcommand{\ttilde}[1]{\ensuremath{{#1}^{\prime \prime \prime}}}
\newcommand{\ntilde}[2]{\ensuremath{{#1}^{(#2)}}}

\newtheorem{defi}{{\it Definición}}[section]
\newtheorem{teo}{{\it Teorema}}[section]
\newtheorem{ejemplo}{{\it Ejemplo}}[section]
\newtheorem{propiedad}{{\it Propiedad}}[section]
\newtheorem{lema}{{\it Lema}}[section]
\newtheorem{cor}{Corolario}
\newtheorem{ejer}{Ejercicio}[section]

\newlist{milista}{enumerate}{2}
\setlist[milista,1]{label=\arabic*)}
\setlist[milista,2]{label=\arabic{milistai}.\arabic*)}
\newlength{\depthofsumsign}
\setlength{\depthofsumsign}{\depthof{$\sum$}}
\newcommand{\nsum}[1][1.4]{% only for \displaystyle
    \mathop{%
        \raisebox
            {-#1\depthofsumsign+1\depthofsumsign}
            {\scalebox
                {#1}
                {$\displaystyle\sum$}%
            }
    }
}
\def\scaleint#1{\vcenter{\hbox{\scaleto[3ex]{\displaystyle\int}{#1}}}}
\def\bs{\mkern-12mu}


\setlength{\tabcolsep}{12pt}
\title{Repaso de vectores \\ \large{Matemáticas Avanzadas de la Física}\vspace{-3ex}}
\author{M. en C. Gustavo Contreras Mayén.}
\date{ }
\begin{document}
\vspace{-4cm}
\maketitle
\fontsize{14}{14}\selectfont
%Ref. Heinbockel (1996) Introduction to tensor calculus and continuum mechanics.
\section{Introducción.}
Un campo escalar describe una correspondencia uno a uno entre un solo número escalar y un punto. Un campo vectorial de $n$ dimensiones se describe mediante una correspondencia biunívoca entre $n$ números y un punto. Generalicemos estos conceptos asignando $n$ números al cuadrado a un solo punto o $n$ números al cubo a un solo punto. Cuando estos números obedecen a ciertas leyes de transformación, se convierten en ejemplos de campos tensoriales. En general, los campos escalares se denominan campos tensoriales de rango u orden cero, mientras que los campos vectoriales se denominan campos tensoriales de rango u orden uno.
\par
La notación indicial o de índices está estrechamente asociada con el cálculo tensorial. Resulta que los tensores tienen ciertas propiedades que son independientes del sistema de coordenadas utilizado para describir el tensor. Debido a estas útiles propiedades, podemos usar tensores para representar varias leyes fundamentales que ocurren en física, ingeniería, ciencia y matemáticas. Estas representaciones son extremadamente útiles ya que son independientes de los sistemas de coordenadas considerados.

\section{Notación de índices.}

Dos vectores $\va{A}$ y $\va{B}$ se pueden presentar en una forma de sus componentes:
\begin{align*}
\va{A} = A_{1} \, \vu{e}_{1} + A_{2} \, \vu{e}_{2} + A_{3} \, \vu{e}_{3} \hspace{1cm} \mbox{y} \hspace{1cm} \va{B} = B_{1} \, \vu{e}_{1} + B_{2} \, \vu{e}_{2} + B_{3} \, \vu{e}_{3}
\end{align*}
donde $\vu{e}_{1}, \vu{e}_{2}$ y $\vu{e}_{3}$ son los vectores base ortogonales.
\par
A menudo, cuando no surge ninguna confusión, los vectores $\va{A}$ y $\vu{B}$ se expresan en aras de la brevedad como números triples. Por ejemplo, podemos escribir:
\begin{align*}
\va{A} = A_{1} + A_{2} + A_{3}  \hspace{1cm} \mbox{y} \hspace{1cm} \va{B} = B_{1} + B_{2} + B_{3}
\end{align*}
donde se sobreentiende que solamente se están indicando las componentes de los vectores $\va{A}$ y $\vu{B}$. Los vectores unitarios se representarían como:
\begin{align*}
\vu{e}_{1} = (1, 0, 0), \hspace{1cm} \vu{e}_{2} = (0, 1, 0), \hspace{1cm} \vu{e}_{3} = (0, 0, 1)
\end{align*}
Una notación aún más corta, que representa los vectores $\va{A}$ y $\va{B}$, es la de índices o notación indicial. En la notación de índices, las cantidades
\begin{align*}
A_{i}, \hspace{0.5cm} i = 1, 2, 3 \hspace{1cm} \mbox{y} \hspace{1cm} B_{p}, \hspace{0.5cm} p = 1, 2, 3
\end{align*}
representan las componentes de los vectores $\va{A}$ y $\va{B}$.
\par
Esta notación enfoca la atención solo en los componentes de los vectores y emplea un \emph{subíndice ficticio} cuyo valor en los enteros está especificado.
\par
El símbolo $A_{i}$ se refiere a todos los componentes del vector $\va{A}$ de manera simultánea. El subíndice ficticio (o mudo) $i$ puede tener cualquiera de los valores enteros $1, 2$ o $3$. Para $i = 1$ enfocamos la atención en el componente $A_{1}$ del vector $\va{A}$. Con $i = 2$ nos enfocamos en el segundo componente $A_{2}$ del vector $\va{A}$ y de manera similar cuando $i = 3$ podemos centrar la atención en el tercer componente de $\va{A}$. El subíndice $i$ es un subíndice mudo y puede ser reemplazado por otra letra, digamos $p$, siempre que se especifiquen los valores enteros que puede tener este subíndice mudo.
\par
También es conveniente en este momento mencionar que los vectores de dimensiones superiores pueden definirse como $n$-tuplas ordenadas. Por ejemplo, el vector:
\begin{align*}
\va{X} = (X_{1}, X_{2}, \ldots, X_{N})
\end{align*}
con componentes $X_{i}, i =1, 2, \ldots, N$ es llamado un vector $N$-dimensional. Otra notación que se utilizar para representar a este vector es:
\begin{align*}
\va{X} = X_{1} \, \vu{e}_{1} +  X_{2} \, \vu{e}_{2} +  \ldots +  X_{N} \, \vu{e}_{N}
\end{align*}
donde los
\begin{align*}
\vu{e}_{1}, \vu{e}_{2}, \ldots, \vu{N}
\end{align*}
son los vectores unitarios base linealmente independientes.
\par
Tengamos en cuenta que muchas de las operaciones que ocurren en el uso de la notación de índices se aplican no solo a los vectores tridimensionales, sino también a los vectores $N$-dimensionales.
\par
Más adelante será necesario definir cantidades que se pueden representar mediante una letra con subíndices o superíndices adjuntos. Estas cantidades se denominan sistemas. Cuando estas cantidades obedecen a ciertas leyes de transformación, se las denomina sistemas tensoriales. Por ejemplo, cantidades como
\begin{align*}
A_{ij}^{k} \hspace{0.75cm} e^{ijk} \hspace{0.75cm} \delta_{ij} \hspace{0.75cm} \delta_{i}^{j} \hspace{0.75cm} A^{i} \hspace{0.75cm} B_{j} \hspace{0.75cm} a_{ij}
\end{align*}
Los subíndices o superíndices se denominan índices o sufijos. Cuando se presentan tales cantidades, los índices deben ajustarse a las siguientes reglas:
\begin{enumerate}
\item Son letras minúsculas latinas o griegas.
\item Las letras al final del alfabeto (u, v, w, x, y, z) nunca se emplean como índices.
\end{enumerate}
El número de subíndices y superíndices determina el orden del sistema. \emph{Un sistema con un índice es un sistema de primer orden}. Un sistema con dos índices se denomina sistema de segundo orden. En general, un sistema con $N$ índices se denomina sistema de orden $N$. Un sistema sin índices se denomina sistema de orden escalar o cero.
\par
El tipo de sistema depende del número de subíndices o superíndices que aparecen en una expresión. Por ejemplo: $A_{jk}^{i}$ y $B_{st}^{m}$ (todos los índices van de $1$ a $N$) son del mismo tipo porque tienen el mismo número de subíndices y superíndices. Por el contrario, los sistemas $A_{jk}^{i}$ y $C_{p}^{mn}$ no son del mismo tipo porque un sistema tiene solo tiene un superíndice y el otro sistema tiene dos superíndices.
\par
Para ciertos sistemas, el número de subíndices y superíndices es importante. En otros sistemas no tiene importancia.
\par
En el uso de superíndices no se deben confundir las \enquote{potencias} de una cantidad con los superíndices. Por ejemplo, si reemplazamos las variables independientes $(x, y, z)$ por los símbolos $(x^{1}, x^{2}, x^{3})$, entonces estamos dejando $y = x^{2}$ donde $x^{2}$ es una variable y no $x$ elevado a una potencia. De manera similar, la sustitución $z = x^{3}$ es el reemplazo de $z$ por la variable $x^{3}$ y esto no debe confundirse con $x$ elevado al cubo. Para escribir una cantidad en superíndice a una potencia, usaremos paréntesis, por ejemplo, $(x^{2})^{3}$ es la variable $x^{2}$ al cubo. Una de las razones para introducir las variables de superíndice \emph{es que se pueden hacer muchas ecuaciones de matemáticas y física para que adquieran una forma concisa y compacta}.
\par
Existe una convención de rango asociada con los índices. Esta convención establece que siempre que haya una expresión en la que los índices se presenten sin repetición, debe entenderse que cada uno de los subíndices o superíndices puede tomar cualquiera de los valores enteros $1,2, \ldots, N$ donde $N$ es un valor entero especificado.
\par
Por ejemplo: la \emph{delta de Kronecker}, con símbolo $\delta_{ij}$, definida como
\begin{align*}
\delta_{ij} = \begin{cases}
1 & \mbox{si } i = j \\[0.5em]
0 & \mbox{si } i \neq j
\end{cases}
\end{align*}
con $i, j$ índices que recorren los valores $1, 2, 3$, la delta de Kronecker representa nueve cantidades:
\begin{table}[H]
\centering
\begin{tabular}{c c c}
$\delta_{11} = 1$ & $\delta_{12} = 0$ & $\delta_{13} = 0$ \\
$\delta_{21} = 0$ & $\delta_{22} = 1$ & $\delta_{23} = 0$ \\
$\delta_{31} = 0$ & $\delta_{32} = 0$ & $\delta_{33} = 1$
\end{tabular}
\end{table}
El símbolo $\delta_{ij}$ se refiere a todos los componentes del sistema de manera simultánea. Otro ejemplo, considera la ecuación:
\begin{align}
\vu{e}_{m} \cdot \vu{e}_{n} = \delta_{mn} \hspace{1cm} m, n = 1, 2, 3
\label{eq:ecuacion_01_01}
\end{align}
los subíndices $m, n$ aparecen sin repetición en el lado izquierdo de la ecuación y, por lo tanto, también deben aparecer en el lado derecho de la ecuación. Estos índices se denominan índices \enquote{libres} y pueden tomar cualquiera de los valores $1, 2$ o $3$ según lo especificado por el rango. Dado que hay tres opciones para el valor de $m$ y tres opciones para un valor de $n$, encontramos que la ecuación (\ref{eq:ecuacion_01_01}) representa nueve ecuaciones simultáneamente. Estas nueve ecuaciones son:
\begin{table}[H]
\centering
\begin{tabular}{c c c}
$\vu{e}_{1} \cdot \vu{e}_{1} = 1$ & $\vu{e}_{1} \cdot \vu{e}_{2} = 0$ & $\vu{e}_{1} \cdot \vu{e}_{3} = 0$ \\
$\vu{e}_{2} \cdot \vu{e}_{1} = 0$ & $\vu{e}_{2} \cdot \vu{e}_{2} = 1$ & $\vu{e}_{2} \cdot \vu{e}_{3} = 0$ \\
$\vu{e}_{3} \cdot \vu{e}_{1} = 0$ & $\vu{e}_{3} \cdot \vu{e}_{2} = 0$ & $\vu{e}_{3} \cdot \vu{e}_{3} = 1$
\end{tabular}
\end{table}
\section{Sistemas simétricos y simétricos oblicuos.}
\end{document}