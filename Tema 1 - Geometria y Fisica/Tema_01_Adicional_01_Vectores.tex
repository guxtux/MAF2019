\documentclass[hidelinks,12pt]{article}
\usepackage[left=0.25cm,top=1cm,right=0.25cm,bottom=1cm]{geometry}
%\usepackage[landscape]{geometry}
\textwidth = 20cm
\hoffset = -1cm
\usepackage[utf8]{inputenc}
\usepackage[spanish,es-tabla]{babel}
\usepackage[autostyle,spanish=mexican]{csquotes}
\usepackage[tbtags]{amsmath}
\usepackage{nccmath}
\usepackage{amsthm}
\usepackage{amssymb}
\usepackage{mathrsfs}
\usepackage{graphicx}
\usepackage{subfig}
\usepackage{standalone}
\usepackage[outdir=./Imagenes/]{epstopdf}
\usepackage{siunitx}
\usepackage{physics}
\usepackage{color}
\usepackage{float}
\usepackage{hyperref}
\usepackage{multicol}
%\usepackage{milista}
\usepackage{anyfontsize}
\usepackage{anysize}
%\usepackage{enumerate}
\usepackage[shortlabels]{enumitem}
\usepackage{capt-of}
\usepackage{bm}
\usepackage{relsize}
\usepackage{placeins}
\usepackage{empheq}
\usepackage{cancel}
\usepackage{wrapfig}
\usepackage[flushleft]{threeparttable}
\usepackage{makecell}
\usepackage{fancyhdr}
\usepackage{tikz}
\usepackage{bigints}
\usepackage{scalerel}
\usepackage{pgfplots}
\usepackage{pdflscape}
\pgfplotsset{compat=1.16}
\spanishdecimal{.}
\renewcommand{\baselinestretch}{1.5} 
\renewcommand\labelenumii{\theenumi.{\arabic{enumii}})}
\newcommand{\ptilde}[1]{\ensuremath{{#1}^{\prime}}}
\newcommand{\stilde}[1]{\ensuremath{{#1}^{\prime \prime}}}
\newcommand{\ttilde}[1]{\ensuremath{{#1}^{\prime \prime \prime}}}
\newcommand{\ntilde}[2]{\ensuremath{{#1}^{(#2)}}}

\newtheorem{defi}{{\it Definición}}[section]
\newtheorem{teo}{{\it Teorema}}[section]
\newtheorem{ejemplo}{{\it Ejemplo}}[section]
\newtheorem{propiedad}{{\it Propiedad}}[section]
\newtheorem{lema}{{\it Lema}}[section]
\newtheorem{cor}{Corolario}
\newtheorem{ejer}{Ejercicio}[section]

\newlist{milista}{enumerate}{2}
\setlist[milista,1]{label=\arabic*)}
\setlist[milista,2]{label=\arabic{milistai}.\arabic*)}
\newlength{\depthofsumsign}
\setlength{\depthofsumsign}{\depthof{$\sum$}}
\newcommand{\nsum}[1][1.4]{% only for \displaystyle
    \mathop{%
        \raisebox
            {-#1\depthofsumsign+1\depthofsumsign}
            {\scalebox
                {#1}
                {$\displaystyle\sum$}%
            }
    }
}
\def\scaleint#1{\vcenter{\hbox{\scaleto[3ex]{\displaystyle\int}{#1}}}}
\def\bs{\mkern-12mu}


\usetikzlibrary{babel}
\setlength{\tabcolsep}{12pt}
\title{Notación de índices \\ \large{Material de consulta previo}\vspace{-3ex}}
\author{M. en C. Gustavo Contreras Mayén}
\date{ }
\begin{document}
\vspace{-4cm}
\maketitle
\fontsize{14}{14}\selectfont
\tableofcontents
\newpage


%Ref. Heinbockel (1996) Introduction to tensor calculus and continuum mechanics.

\section{Introducción.}

Un campo escalar describe una correspondencia uno a uno entre un solo número escalar y un punto. Un campo vectorial de $n$ dimensiones se describe mediante una correspondencia biunívoca entre $n$ números y un punto. Generalicemos estos conceptos asignando $n$ números al cuadrado a un solo punto o $n$ números al cubo a un solo punto. Cuando estos números obedecen a ciertas leyes de transformación, se convierten en ejemplos de campos tensoriales. En general, los campos escalares se denominan campos tensoriales de rango u orden cero, mientras que los campos vectoriales se denominan campos tensoriales de rango u orden uno.
\par
La notación indicial o de índices está estrechamente asociada con el cálculo tensorial. Resulta que los tensores tienen ciertas propiedades que son independientes del sistema de coordenadas utilizado para describir el tensor. Debido a estas útiles propiedades, podemos usar tensores para representar varias leyes fundamentales que ocurren en física, ingeniería, ciencia y matemáticas. Estas representaciones son extremadamente útiles ya que son independientes de los sistemas de coordenadas considerados.

\section{Notación de índices.}

Dos vectores $\va{A}$ y $\va{B}$ se pueden presentar en una forma de sus componentes:
\begin{align*}
\va{A} = A_{1} \, \vu{e}_{1} + A_{2} \, \vu{e}_{2} + A_{3} \, \vu{e}_{3} \hspace{1cm} \mbox{y} \hspace{1cm} \va{B} = B_{1} \, \vu{e}_{1} + B_{2} \, \vu{e}_{2} + B_{3} \, \vu{e}_{3}
\end{align*}
donde $\vu{e}_{1}, \vu{e}_{2}$ y $\vu{e}_{3}$ son los vectores base ortogonales.
\par
A menudo, cuando no no hay confusión, los vectores $\va{A}$ y $\vu{B}$ se expresan en aras de la brevedad como tripletes. Por ejemplo, podemos escribir:
\begin{align*}
\va{A} = A_{1} + A_{2} + A_{3}  \hspace{1cm} \mbox{y} \hspace{1cm} \va{B} = B_{1} + B_{2} + B_{3}
\end{align*}
donde se sobreentiende que solamente se están indicando las componentes de los vectores $\va{A}$ y $\va{B}$. Los vectores unitarios se representarían como:
\begin{align*}
\vu{e}_{1} = (1, 0, 0), \hspace{1cm} \vu{e}_{2} = (0, 1, 0), \hspace{1cm} \vu{e}_{3} = (0, 0, 1)
\end{align*}
Una notación aún más corta, que representa los vectores $\va{A}$ y $\va{B}$, es la de índices o notación indicial. En la notación de índices, las cantidades:
\begin{align*}
A_{i}, \hspace{0.5cm} i = 1, 2, 3 \hspace{1cm} \mbox{y} \hspace{1cm} B_{p}, \hspace{0.5cm} p = 1, 2, 3
\end{align*}
representan las componentes de los vectores $\va{A}$ y $\va{B}$.
\par
Esta notación enfoca la atención solo en los componentes de los vectores y emplea un \emph{subíndice ficticio} cuyo valor en los enteros está especificado.
\par
El símbolo $A_{i}$ se refiere a todos los componentes del vector $\va{A}$ de manera simultánea. El subíndice ficticio (o mudo) $i$ puede tener cualquiera de los valores enteros $1, 2$ o $3$. Para $i = 1$ enfocamos la atención en el componente $A_{1}$ del vector $\va{A}$. Con $i = 2$ nos enfocamos en el segundo componente $A_{2}$ del vector $\va{A}$ y de manera similar cuando $i = 3$ podemos centrar la atención en el tercer componente de $\va{A}$. El subíndice $i$ es un subíndice mudo y puede ser reemplazado por otra letra, digamos $p$, siempre que se especifiquen los valores enteros que puede tener este subíndice mudo.
\par
También es conveniente en este momento mencionar que los vectores de dimensiones superiores pueden definirse como $n$-tuplas ordenadas. Por ejemplo, el vector:
\begin{align*}
\va{X} = (X_{1}, X_{2}, \ldots, X_{N})
\end{align*}
con componentes $X_{i}, \, i =1, 2, \ldots, N$ es llamado un vector $N$-dimensional. Otra notación que se utilizar para representar a este vector es:
\begin{align*}
\va{X} = X_{1} \, \vu{e}_{1} +  X_{2} \, \vu{e}_{2} +  \ldots +  X_{N} \, \vu{e}_{N}
\end{align*}
donde los
\begin{align*}
\vu{e}_{1}, \vu{e}_{2}, \ldots, \vu{e}_{N}
\end{align*}
son los vectores unitarios base linealmente independientes.
\par
Tengamos en cuenta que muchas de las operaciones que ocurren en el uso de la notación de índices se aplican no solo a los vectores tridimensionales, sino también a los vectores $N$-dimensionales.
\par
Más adelante será necesario definir cantidades que se pueden representar mediante una letra con subíndices o superíndices adjuntos. Estas cantidades se denominan sistemas. Cuando estas cantidades obedecen a ciertas leyes de transformación, se las denomina sistemas tensoriales. Por ejemplo, cantidades como:
\begin{align*}
A_{ij}^{k} \hspace{0.75cm} e^{ijk} \hspace{0.75cm} \delta_{ij} \hspace{0.75cm} \delta_{i}^{j} \hspace{0.75cm} A^{i} \hspace{0.75cm} B_{j} \hspace{0.75cm} a_{ij}
\end{align*}
Los subíndices o superíndices se denominan índices o sufijos. Cuando se presentan tales cantidades, los índices deben ajustarse a las siguientes reglas:
\begin{enumerate}
\item Son letras minúsculas latinas o griegas.
\item Las letras al final del alfabeto (u, v, w, x, y, z) nunca se emplean como índices.
\end{enumerate}
El número de subíndices y superíndices determina el orden del sistema. \emph{Un sistema con un índice es un sistema de primer orden}. Un sistema con dos índices se denomina sistema de segundo orden. En general, un sistema con $N$ índices se denomina sistema de orden $N$. Un sistema sin índices se denomina sistema de orden escalar o de orden cero.
\par
El tipo de sistema depende del número de subíndices o superíndices que aparecen en una expresión. Por ejemplo: $A_{jk}^{i}$ y $B_{st}^{m}$ (todos los índices van de $1$ a $N$) son del mismo tipo porque tienen el mismo número de subíndices y superíndices. Por el contrario, los sistemas $A_{jk}^{i}$ y $C_{p}^{mn}$ no son del mismo tipo porque un sistema tiene solo tiene un superíndice y el otro sistema tiene dos superíndices.
\par
Para ciertos sistemas, el número de subíndices y superíndices es importante. En otros sistemas no tiene importancia.
\par
En el uso de superíndices no se deben confundir los \enquote{exponentes} de una cantidad con los superíndices. Por ejemplo, si reemplazamos las variables independientes $(x, y, z)$ por los símbolos $(x^{1}, x^{2}, x^{3})$, entonces estamos dejando $y = x^{2}$ donde $x^{2}$ es una variable y no $x$ elevado a la segunda potencia. De manera similar, la sustitución $z = x^{3}$ es el reemplazo de $z$ por la variable $x^{3}$ y esto no debe confundirse con $x$ elevado al cubo. Para escribir una cantidad en superíndice a una potencia, usaremos paréntesis, por ejemplo, $(x^{2})^{3}$ es la variable $x^{2}$ al cubo. Una de las razones para introducir las variables de superíndice \emph{es que se pueden hacer muchas ecuaciones de matemáticas y física para que adquieran una forma concisa y compacta}.
\par
Existe una convención de rango asociada con los índices. Esta convención establece que siempre que haya una expresión en la que los índices se presenten sin repetición, debe entenderse que cada uno de los subíndices o superíndices puede tomar cualquiera de los valores enteros $1, 2, \ldots, N$ donde $N$ es un valor entero especificado.
\par
Por ejemplo: la \emph{delta de Kronecker}, con símbolo $\delta_{ij}$, definida como
\begin{align*}
\delta_{ij} = \begin{cases}
1 & \mbox{si } i = j \\[0.5em]
0 & \mbox{si } i \neq j
\end{cases}
\end{align*}
con $i, j$ índices que recorren los valores $1, 2, 3$, la delta de Kronecker representa nueve cantidades:
\begin{table}[H]
\large
\centering
\begin{tabular}{c c c}
$\delta_{11} = 1$ & $\delta_{12} = 0$ & $\delta_{13} = 0$ \\
$\delta_{21} = 0$ & $\delta_{22} = 1$ & $\delta_{23} = 0$ \\
$\delta_{31} = 0$ & $\delta_{32} = 0$ & $\delta_{33} = 1$
\end{tabular}
\end{table}
El símbolo $\delta_{ij}$ se refiere a todos los componentes del sistema de manera simultánea. Otro ejemplo, considera la ecuación:
\begin{align}
\vu{e}_{m} \cdot \vu{e}_{n} = \delta_{mn} \hspace{1cm} m, n = 1, 2, 3
\label{eq:ecuacion_01_01}
\end{align}
los subíndices $m, n$ aparecen sin repetición en el lado izquierdo de la ecuación y, por lo tanto, también deben aparecer en el lado derecho de la ecuación. Estos índices se denominan índices \enquote{libres} y pueden tomar cualquiera de los valores $1, 2$ o $3$ según lo especificado por el rango. Dado que hay tres opciones para el valor de $m$ y tres opciones para un valor de $n$, encontramos que la ecuación (\ref{eq:ecuacion_01_01}) representa nueve ecuaciones simultáneamente. Estas nueve ecuaciones son:
\begin{table}[H]
\large
\centering
\begin{tabular}{c c c}
$\vu{e}_{1} \cdot \vu{e}_{1} = 1$ & $\vu{e}_{1} \cdot \vu{e}_{2} = 0$ & $\vu{e}_{1} \cdot \vu{e}_{3} = 0$ \\
$\vu{e}_{2} \cdot \vu{e}_{1} = 0$ & $\vu{e}_{2} \cdot \vu{e}_{2} = 1$ & $\vu{e}_{2} \cdot \vu{e}_{3} = 0$ \\
$\vu{e}_{3} \cdot \vu{e}_{1} = 0$ & $\vu{e}_{3} \cdot \vu{e}_{2} = 0$ & $\vu{e}_{3} \cdot \vu{e}_{3} = 1$
\end{tabular}
\end{table}

\section{Sistemas simétricos y asimétricos.}

Se dice que un sistema definido por subíndices y superíndices que abarcan un conjunto de valores es simétrico en dos de sus índices si los componentes no cambian cuando se intercambian los índices. Por ejemplo, el sistema de tercer orden $T_{ijk}$ es simétrico en los índices $i$ y $k$ si
\begin{align*}
T_{ijk} = T_{kji} \hspace{1cm} \forall \hspace{0.2cm} i, j, k
\end{align*}
Se dice que un sistema definido por subíndices y superíndices es asimétrico en dos de sus índices, si los componentes cambian de signo cuando se intercambian los índices. Por ejemplo, el sistema de cuarto orden $T_{ijkl}$ es asimétrico en los índices $i$ y $l$ si
\begin{align*}
T_{ijkl} = -T_{ljki} \hspace{1cm} \forall \hspace{0.2cm} i, j, k, l
\end{align*}
Como otro ejemplo, considere el sistema de tercer orden $a_{prs}$, con $p, r, s = 1, 2, 3$ que es completamente asimétrico en todos sus índices. Entonces tendríamos:
\begin{align*}
a_{prs} = - a_{psr} = a_{spr} = -a_{srp} = a_{rsp} = -a_{rps}
\end{align*}
Sería un buen ejercicio demostrar que este sistema completamente asimétrico tiene $27$ elementos, $21$ de los cuales son cero. Los $6$ elementos distintos de cero están todos relacionados entre sí a través de las ecuaciones anteriores cuando $(p, r, s) = (1, 2, 3)$. Esto se expresa diciendo que el sistema anterior tiene solo un componente independiente.

\section{Convención de la suma.}

La convención de suma establece que siempre que surge una expresión \emph{donde hay un índice que aparece dos veces en el mismo lado de cualquier ecuación, o término dentro de una ecuación, se entiende que representa una suma en estos índices repetidos}.
\par
La suma está por encima de los valores enteros especificados por el rango. Un índice repetido se llama índice de suma, mientras que un índice no repetido se llama índice libre. La convención de suma requiere que nunca se debe permitir que un índice de suma aparezca más de dos veces en una expresión dada. Debido a esta regla, a veces es necesario reemplazar un símbolo de suma ficticio (o mudo) por otro símbolo ficticio para evitar que aparezcan tres o más índices en el mismo lado de la ecuación.
\par
La notación de índice es una notación muy poderosa y se puede utilizar para representar de forma concisa muchas ecuaciones complejas. Esta notación se emplea para definir los componentes tensoriales y las operaciones asociadas con los tensores.
\par
\noindent
\textbf{Ejemplo: }  Las siguientes dos ecuaciones
\begin{align*}
y_{1} & = a_{11} \, x_{1} + a_{12} \, x_{2} \\[0.5em]
y_{2} & = a_{21} \, x_{1} + a_{22} \, x_{2}
\end{align*}
se puede expresar como una sola ecuación al introducir un índice mudo, digamos $k$, tal que las dos ecuaciones anteriores se pueden escribir como:
\begin{align*}
y_{k} = a_{k1} \, x_{1} + a_{k2} \, x_{2} \hspace{1cm} k = 1, 2
\end{align*}
La convención de rango establece que $k$ es libre de tener cualquiera de los valores $1$ o $2$ ($k$ es un índice libre). Esta ecuación ahora se puede escribir en la forma:
\begin{align*}
y_{k} = \nsum_{i=1}^{2} a_{ki} \, x_{i} = a_{k1} \, x_{1} + a_{k2} \, x_{2}
\end{align*}
donde $i$ es el índice mudo de la suma. Cuando se retira la sigma que representa la suma y se ocupa la convención de la suma, se tiene que:
\begin{align*}
y_{k} = a_{ki} \, x_{i} , \hspace{1cm} i, k = 1, 2
\end{align*}
Dado que el subíndice $i$ se repite, la convención de suma requiere que se realice una suma dejando que el subíndice de suma adopte los valores especificados por el rango y luego sumando los resultados. El índice $k$ que aparece solo una vez a la izquierda y solo una vez a la derecha de la ecuación se llama índice libre. Cabe señalar que tanto $k$ como $i$ son subíndices ficticios y pueden reemplazarse por otras letras. Por ejemplo, podemos escribir:
\begin{align*}
y_{n} = a_{nm} \, x_{m} \hspace{1cm} n, m = 1, 2
\end{align*}
donde $m$ es el índice de suma y $n$ el índice libre. Sumando sobre $m$ se obtiene:
\begin{align*}
y_{n} = a_{n1} \, x_{1} + a_{n2} \, x_{2}
\end{align*}
y dejando que el índice libre $n$ tome los valores de $1$ y $2$, recuperamos las dos ecuaciones originales.
\par
\noindent
\textbf{Ejemplo: } Para $y_{i} = a_{ij} \, x_{j}$ con $i, j = 1, 2, 3$ y $x_{i} = b_{ij} \, z_{j}$ con $i, j = 1, 2, 3$, resolver para la variable $y$ en términos de la variable $z$.
\par
En una representación matricial, las dos ecuaciones anteriores se expresan como:
\begin{align*}
\mqty(y_{1} \\ y_{2} \\ y_{3}) = \mqty(
a_{11} & a_{12} & a_{13} \\
a_{21} & a_{22} & a_{23} \\
a_{31} & a_{32} & a_{33})
\, \mqty(
x_{1} \\ x_{2} \\ x_{3}
) \hspace{1cm} \mbox{y} \hspace{1cm}
\mqty(x_{1} \\ x_{2} \\ x_{3}) = \mqty(
b_{11} & b_{12} & b_{13} \\
b_{21} & b_{22} & b_{23} \\
b_{31} & b_{32} & b_{33})
\, \mqty(z_{1} \\ z_{2} \\ z_{3})
\end{align*}
Para resolver $y$ en términos de la variable $z$, tendremos que:
\begin{align*}
\mqty(y_{1} \\ y_{2} \\ y_{3}) = \mqty(
a_{11} & a_{12} & a_{13} \\
a_{21} & a_{22} & a_{23} \\
a_{31} & a_{32} & a_{33})
\, \mqty(
b_{11} & b_{12} & b_{13} \\
b_{21} & b_{22} & b_{23} \\
b_{31} & b_{32} & b_{33})
\, \mqty(z_{1} \\ z_{2} \\ z_{3})
\end{align*}
La notación de índices emplea índices que son índices mudos, por lo que podemos escribir:
\begin{align*}
y_{n} = a_{nm} \, x_{m}, \hspace{0.5cm} n, m = 1, 2, 3 \hspace{1cm} \mbox{y} \hspace{1cm} x_{m} = b_{mj} \, z_{j}, \hspace{0.5cm} m, j = 1, 2, 3
\end{align*}
Aquí hemos cambiado deliberadamente los índices de modo que cuando sustituimos $x_{m}$, de una ecuación a la otra, un índice de suma no se repite más de dos veces. Sustituyendo encontramos la forma en índices de la ecuación matricial anterior como:
\begin{align*}
y_{n} = a_{nm} \, b_{mj} \, z_{j} \hspace{0.5cm} m, n, j = 1, 2, 3
\end{align*}
donde $n$ es el índice libre y $m$, $j$ son los índices mudos para la suma.
\par
\noindent
\textbf{Ejemplo: } El producto punto de dos vectores $A_{q}$ con $q = 1, 2, 3$ y $B_{j}$, con $j = 1, 2, 3$, se representa con la notación de índices como:
\begin{align*}
A_{i} \, B_{i} = A \, B \, \cos \theta \hspace{0.75cm} i = 1, 2, 3, \hspace{0.75cm} A = \abs{\va{A}}, \hspace{0.3cm} B = \abs{\va{B}}
\end{align*}
Ya que el índice $i$ está repetido, se entiende que representa el índice de la suma. Al sumar sobre $i$ en el rango indicado, tenemos:
\begin{align*}
A_{1} \, B_{1} + A_{2} \, B_{2} + A_{3} \, B_{3} = A \, B \, \cos \theta
\end{align*}
Veamos que la notación de índices emplea índices ficticios. En ocasiones, estos índices se modifican para ajustarse a las reglas de suma anteriores, sin que se preste atención al cambio. Como en este ejemplo, los índices $q$ y $j$ son índices ficticios y se pueden cambiar a otras letras si se desea. Además, si no se indica el rango de los índices, se asume que el rango está por encima de los valores enteros $1$, $2$ y $3$.
\par
A los sistemas que contienen subíndices y superíndices se pueden aplicar ciertas operaciones algebraicas. Presentamos a continuación de manera informal las operaciones de suma, multiplicación y contracción.

\section{Suma, multiplicación y contracción.}

La operación algebraica de suma o resta se aplica a sistemas del mismo tipo y orden. Es decir, podemos sumar o restar componentes similares en sistemas. Por ejemplo, la suma de $A_{jk}^{i}$ y $B_{jk}^{i}$ es nuevamente un sistema del mismo tipo y se denota por $C_{jk}^{i} = A_{jk}^{i} + B_{jk}^{i}$, donde se suman componentes similares.
\par
El producto de dos sistemas se obtiene multiplicando cada componente del primer sistema por cada componente del segundo sistema. Tal producto se llama \emph{producto externo}. El orden del sistema de producto resultante es la suma de los órdenes de los dos sistemas involucrados en la formación del producto. Por ejemplo, si $A_{j}^{i}$ es un sistema de segundo orden y $B^{mnl}$ es un sistema de tercer orden, con todos los índices que tienen el rango de $1$ a $N$, entonces el sistema de productos es de quinto orden y se denota $C_{j}^{imnl} = A_{j}^{i} \, B^{mnl}$. El sistema obtenido por el producto representa $N^{5}$ términos construidos a partir de todos los productos posibles de los componentes de $A_{j}^{i}$ con los componentes de $B^{mnl}$.
\par
La operación de contracción se produce cuando un índice inferior se iguala a un índice superior y se invoca la convención de suma. Por ejemplo, si tenemos un sistema de quinto orden $C_{j}^{imnl}$ y hacemos que $i = j$ para luego hacer la suma, entonces hemos formado el sistema:
\begin{align*}
C^{mnl} = C_{j}^{jmnl} = C_{1}^{1mnl} + C_{2}^{2mnl} + \ldots + C_{N}^{Nmnl}
\end{align*}
Aquí el símbolo $C^{mnl}$ se usa para representar el sistema de tercer orden que resulta cuando se realiza la contracción. Siempre que se realiza una contracción, el sistema resultante es siempre de orden $2$ veces menor que el sistema original. Bajo ciertas condiciones especiales, está permitido realizar una contracción en dos índices inferiores.

\section{El símbolo de e-permutación y la delta de Kronecker.}

Hay dos símbolos que se utilizan con bastante frecuencia con la notación de índices: el símbolo de e-permutación y la delta de Kronecker.
\par
El símbolo de e-permutación a veces se denomina \emph{tensor alterno}. El símbolo de e-permutación, como su nombre indica, se ocupa de las permutaciones.
\par
Una permutación es un arreglo de cosas. Cuando se cambia el orden de la disposición, se produce una nueva permutación. Una transposición es un intercambio de dos términos consecutivos en un arreglo.
\par
Como ejemplo, cambiemos los dígitos $1 \, 2 \, 3$ a $3 \, 2 \, 1$ haciendo una secuencia de transposiciones.
\par
Comenzando con los dígitos en el orden $1 \, 2 \, 3$ intercambiamos $2$ y $3$ (primera transposición) para obtener $1 \, 3 \, 2$. Luego, intercambiamos los dígitos $1$ y $3$ (segunda transposición) para obtener $3 \, 1 \, 2$. Finalmente, intercambia los dígitos $1$ y $2$ (tercera transposición) para lograr $3 \, 2 \, 1$. Aquí el número total de transposiciones de $1 \, 2 \, 3$ a $3 \, 2 \, 1$ es tres, un número impar. También se pueden escribir otras transposiciones de $1 \, 2 \, 3$ a $3 \, 2 \, 1$. Sin embargo, también se trata de un número impar de transposiciones.
\par
\noindent
\textbf{Ejemplo: } El número total de formas posibles de ordenar los dígitos $1 \, 2 \, 3$ es seis. Tenemos tres opciones para el primer dígito. Habiendo elegido el primer dígito, solo quedan dos opciones para el segundo dígito. Por tanto, el número restante corresponde al último dígito. El producto $(3) (2) (1) = 3! = 6$ es el número de permutaciones de los dígitos $1$, $2$ y $3$. Estas seis permutaciones son:
\begin{table}[H]
\large
\centering
\begin{tabular}{c l}
$1 \, 2 \, 3$ & permutación par   \\
$1 \, 3 \, 2$ & permutación impar \\
$3 \, 1 \, 2$ & permutación par   \\
$3 \, 2 \, 1$ & permutación impar \\
$2 \, 3 \, 1$ & permutación par   \\
$2 \, 1 \, 3$ & permutación impar \\
\end{tabular}
\end{table}
Aquí, una permutación de $1 \, 2 \, 3$ se llama par o impar dependiendo de si hay un número par o impar de transposiciones de los dígitos. En la figura (\ref{fig:figura_01_01}) se ilustra una regla mnemónica para recordar las permutaciones pares e impares de $1 \, 2 \, 3$. Tengamos en cuenta que las permutaciones pares de $1 \, 2 \, 3$ se obtienen seleccionando tres números consecutivos cualesquiera de la secuencia $1 \, 2 \, 3\, 1 \, 2 \, 3$ y las permutaciones impares resultan seleccionando tres números consecutivos cualesquiera de la secuencia $3 \, 2 \, 1 \, 3 \, 2 \, 1$.
\begin{figure}[H]
    \centering
    \begin{tikzpicture}[thick, scale=1.5, font=\large]
        \node (i) at (90:1cm)  {$1$};
        \node (j) at (-30:1cm) {$2$};
        \node (k) at (210:1cm) {$3$};

        \draw [->] (70:1cm)  arc (70:-10:1cm);
        \draw [->] (-50:1cm) arc (-50:-130:1cm);
        \draw [->] (190:1cm) arc (190:110:1cm);

        \node at (0, 0) {$\mathbf{+}$};
        \node at (0, -1.5) {$1 \, 2 \, 3 \, 1 \, 2 \, 3$};
    \end{tikzpicture}
    \hspace{2cm}
    \begin{tikzpicture}[thick, scale=1.5, font=\large]
        \node (i) at (90:1cm)  {$1$};
        \node (j) at (-30:1cm) {$2$};
        \node (k) at (210:1cm) {$3$};

        \draw [<-] (70:1cm)  arc (70:-10:1cm);
        \draw [<-] (-50:1cm) arc (-50:-130:1cm);
        \draw [<-] (190:1cm) arc (190:110:1cm);

        \node at (0, 0) {$\mathbf{-}$};
        \node at (0, -1.5) {$3 \, 2 \, 1 \, 3 \, 2 \, 1$};
    \end{tikzpicture}
    \caption{Permutaciones de $1 \, 2 \, 3$.}
    \label{fig:figura_01_01}
\end{figure}
En general, el número de permutaciones de $n$ cosas tomadas de $m$, una a la vez, está dada por la relación:
\begin{align*}
P (n, m) = n (n - 1)(n - 2) \ldots (n - m +1)
\end{align*}
Al seleccionar un subconjunto de $m$ objetos de una colección de $n$ objetos, donde $m \leq n$, sin tener en cuenta el orden, se llama una combinación de $n$ objetos tomados $m$ a la vez.
\par
Por ejemplo, las combinaciones de $3$ números tomados del conjunto \\ $\left\{ 1, 2, 3, 4 \right\}$ son $(123), (124), (134), (234)$. Tomemos en cuenta que no se considera el pedido de una combinación. Es decir, las permutaciones
\begin{align*}
(123), (132), (231), (213), (312), (321)
\end{align*}
se consideran iguales. En general, el número de combinaciones de $n$ objetos tomados de $m$ uno a la vez, está dada por:
\begin{align*}
C(n, m) = \binom{n}{m} = \dfrac{n!}{m! \, (n - m)!}
\end{align*}
donde $\displaystyle \binom{n}{m}$ son los coeficientes binomiales que se presentan en la expansión:
\begin{align*}
\left( a + b \right)^{n} = \nsum_{n=0}^{m} \, \binom{n}{m} \, a^{n-m} \, b^{m}
\end{align*}
La definición de permutaciones puede utilizarse para definir el símbolo de la e-permutación:
\\[1em]
\noindent\fbox{%
\parbox{0.9\textwidth}{%
\textbf{Definición: } El símbolo de e-permutación (o también llamado tensor \hfill \break alternante) se define por:
\begin{align*}
e^{ijk \ldots l} = e_{ijk \ldots l} {=} \begin{cases}
1  & \mbox{si } ijk \ldots l \mbox{ es una perm. par de } 1\, 2 \, 3, \ldots, n   \\
-1 & \mbox{si } ijk \ldots l \mbox{ es una perm. impar de } 1\, 2 \, 3, \ldots, n \\
0  & \mbox{en todos los otros casos}
\end{cases}
\end{align*}
}%
}
\\[1em]
\noindent
\textbf{Ejemplo: } Calcula $e_{612453}$.
\\[1em]
\noindent
Para determinar si $612453$ es una permutación par o impar de $123456$, escribimos los números dados y debajo de ellos escribimos los números enteros del $1$ al $6$. Luego, los números iguales se conectan mediante una línea y obtenemos la figura (\ref{fig:figura_01_02}).
\begin{figure}[H]
    \centering
    \begin{tikzpicture}[thick, font=\large]
        \node (A1) at (0, 0) {$6$};
        \node (B1) at (1, 0) {$1$};
        \node (C1) at (2, 0) {$2$};
        \node (D1) at (3, 0) {$4$};
        \node (E1) at (4, 0) {$5$};
        \node (F1) at (5, 0) {$3$};

        \node (A2) at (0, -5) {$1$};
        \node (B2) at (1, -5) {$2$};
        \node (C2) at (2.2, -5) {$3$};
        \node (D2) at (3, -5) {$4$};
        \node (E2) at (4, -5) {$5$};
        \node (F2) at (5.5, -5) {$6$};

        \draw (A1) -- (F2);
        \draw (B1) -- (A2);
        \draw (C1) -- (B2);
        \draw (D1) -- (D2);
        \draw (E1) -- (E2);
        \draw (F1) -- (C2);
    \end{tikzpicture}
    \caption{Permutaciones de $123456$}
    \label{fig:figura_01_02}
\end{figure}
En la figura (\ref{fig:figura_01_02}) hay siete intersecciones de las líneas que conectan números iguales. El número de intersecciones es un número impar y muestra que se debe realizar un número impar de transposiciones. Estos resultados implican que $e_{612453} = -1$
\par
Otra definición utilizada con bastante frecuencia en la representación de cantidades matemáticas y de ingeniería es la delta de Kronecker, que ahora definimos en términos de subíndices y superíndices:

\noindent\fbox{%
\parbox{0.9\textwidth}{%
\textbf{Definición: } La delta de Kronecker está definida como:
\begin{align*}
\delta_{ij} = \delta_{i}^{j} = \begin{cases}
1 & \mbox{si } i = j    \\
0 & \mbox{si } i \neq j
\end{cases}
\end{align*}
}%
}
\\[1em]
\noindent
\textbf{Ejemplo: } Algunos ejemplos del símbolo de e-permutación y de la delta de Kronecker son:
\begin{table}[H]
\large
\centering
\begin{tabular}{l c c}
$e_{123} = e^{123} = +1$ & $\delta_{1}^{1} =  1$ & $\delta_{12} = 0$ \\
$e_{213} = e^{213} = -1$ & $\delta_{2}^{1} =  0$ & $\delta_{22} = 1$ \\
$e_{112} = e^{112} = 0$  & $\delta_{3}^{1} =  0$ & $\delta_{32} = 0$
\end{tabular}
\end{table}

\noindent
\textbf{Ejemplo:} Cuando un índice de la delta de Kronecker $\delta_{ij}$ está involucrado en la convención de suma, el efecto es el de reemplazar un índice con un índice diferente. Por ejemplo, denotemos $a_{ij}$ los elementos de una matriz $N \times N$. Aquí se permite que $i$ y $j$ oscilen sobre los valores enteros $1, 2, \ldots , N$. Considera el producto
\begin{align*}
a_{ij} \, \delta_{ik}
\end{align*}
donde el rango de $i, j, k$ es $1, 2, \ldots , N$. El índice $i$ se repite y por lo tanto se entiende que representa una suma en el rango. El índice $i$ se llama índice de suma. Los otros índices $j$ y $k$ son índices libres. Son libres de que se les asigne cualquier valor del rango de los índices. No están involucrados en ninguna suma y sus valores, independientemente de lo que elija asignarles, son fijos. Asignemos un valor de $\underline{j}$ y $\underline{k}$ a los valores de $j$ y $k$, el guión bajo es para recordarle que estos valores para $j$ y $k$ son fijos y no se deben sumar. Cuando realizamos la suma sobre el índice de suma $i$, asignamos valores a $i$ del rango y luego sumamos sobre estos valores. Realizando la suma indicada obtenemos:
\begin{align*}
a_{i \underline{j}} \, \delta_{i \underline{k}} = a_{1 \underline{j}} \, \delta_{1 \underline{k}} + a_{2 \underline{j}} \, \delta_{2 \underline{k}} + \ldots + a_{\underline{k} \, \underline{j}} \, \delta_{\underline{k} \, \underline{k}} + \ldots + a_{N \underline{j}} \, \delta_{N \underline{k}}
\end{align*}
En esta suma, la delta de Kronecker es cero en todos los lugares donde los subíndices son diferentes y es igual a uno cuando los subíndices son iguales. Solo hay un término en esta suma que es distinto de cero. Es ese término donde el índice de suma $i$ era igual al valor fijo $k$ Esto da el resultado:
\begin{align*}
    a_{\underline{k} \, \underline{j}} \, \delta_{k \, k} = a_{\underline{k} \, \underline{j}}
\end{align*}
donde el subrayado es para recordar que las cantidades tienen valores fijos y no se deben sumar. Quitando los guiones bajos que escribimos:
\begin{align*}
    a_{ij} \, \delta_{i k} = a_{k j}
\end{align*}
Aquí hemos sustituido el índice $i$ por $k$, por lo que cuando se usa la delta de Kronecker en un proceso de suma, se lo conoce como operador de sustitución. Esta propiedad de sustitución de la delta de Kronecker se puede utilizar para simplificar una variedad de expresiones que involucran la notación de índice. Algunos ejemplos son:
\begin{align*}
B_{ij} \, \delta_{js} & = B_{is}         \\[0.5em]
\delta_{jk} \, \delta_{km} &= \delta_{jm} \\[0.5em]
e_{ijk} \, \delta_{im} \, \delta_{jn} \, \delta_{kp} &= e_{mnp}
\end{align*}
Algunos textos adoptan la notación de que si los índices son letras mayúsculas, no se debe realizar ninguna suma. Por ejemplo:
\begin{align*}
a_{KJ} \, \delta_{KK} = a_{KJ}
\end{align*}
ya que $\delta_{KK}$ representa un solo término debido a las letras mayúsculas. Otra notación que se utiliza para indicar que no hay suma de los índices es poner paréntesis sobre los índices que no se van a sumar. Por ejemplo,
\begin{align*}
a_{(k) j} \, \delta_{(k) (k)} = a_{kj}
\end{align*}
dado que $\delta_{(k) (k)}$ representa un solo término y los paréntesis indican que no se debe realizar ninguna suma. En cualquier momento podemos emplear la notación de subrayado, la notación de letras mayúsculas o la notación de paréntesis para indicar que no se debe realizar ninguna suma de los índices. Para evitar confusiones por completo, se pueden escribir expresiones entre paréntesis como \enquote{(no sumar sobre $k$)}.
\\[0.5em]
\noindent
\textbf{Ejemplo: } En el símbolo delta de Kronecker $\delta_{j}^{i}$ hacemos $j$ igual a $i$, para luego hacer la suma. \emph{Esta operación se llama contracción}. Se obtiene como resultado $\delta_{i}^{i}$, que se suma en el rango del índice $i$. Utilizando el rango $1, 2, \ldots, N$ tenemos:
\begin{align*}
\delta_{i}^{i} & = \delta_{1}^{1} + \delta_{2}^{2} + \ldots + \delta_{N}^{N} \\[0.5em]
\delta_{i}^{1} & = 1 + 1 + \ldots +  1                                       \\[0.em]
\delta_{i}^{1} & = N
\end{align*}
En tres dimensiones tenemos $\delta_{j}^{i}$ con $i, j = 1, 2, 3$ y
\begin{align*}
\delta_{k}^{k} = \delta_{1}^{1} + \delta_{2}^{2} + \delta_{3}^{3} = 3
\end{align*}
En determinadas circunstancias, la delta de Kronecker se puede escribir solo con subíndices. Por ejemplo, $\delta_{ij}$ con $i, j = 1, 2, 3$. Encontraremos que estas circunstancias nos permiten realizar una contracción en los índices inferiores, de modo que $\delta_{ii} = 3$.
\\[0.5em]
\noindent
\textbf{Ejemplo: } El determinante de una matriz $A = (a_{ij})$ se puede expresar con notación de índices. Ocupando el símbolo de e-permutación, el determinante de una matriz de $N \times N$ se expresa como:
\begin{align*}
    \abs{A} = e_{ij  \ldots k} \, a_{1i} \, a_{2j} \, \ldots \, a_{Nk}
\end{align*}
donde $e_{ij \ldots k}$ es un sistema de orden $N$. En el caso especial de una matriz de $2 \times N$, escribimos:
\begin{align*}
\abs{A} = e_{ij} \, a_{1i} \, a_{2j}
\end{align*}
donde la suma se realiza en el rango $1, 2$, y el símbolo de e-permutación es de orden $2$. En el caso especial de una matriz de $3 \times 3$, se tiene que:
\begin{align*}
\abs{A} = \mqty|
a_{11} & a_{12} & a_{13} \\
a_{21} & a_{22} & a_{23} \\
a_{31} & a_{32} & a_{33}
| = e_{ijk} \, a_{i1} \, a_{j2} \, a_{k3} = e_{ijk} \, a_{1i} \, a_{2j} \, a_{3k}
\end{align*}
donde $i, j, k$ son los índices de suma y la suma está en el rango $1, 2, 3$. Aquí $e_{ijk}$ denota el símbolo de e-permutación de orden $3$. Tengamos en cuenta que al intercambiar las filas de la matriz de $3 \times 3$ podemos obtener resultados más generales. Consideremos $(p,q, r)$ una permutación de los enteros $(1, 2, 3)$ y veamos que el determinante se puede expresar como:
\begin{align*}
\Delta = \mqty|
a_{p1} & a_{p2} & a_{p3} \\
a_{q1} & a_{q2} & a_{q3} \\
a_{r1} & a_{r2} & a_{r3}
| = e_{ijk} \, a_{pi} \, a_{jk} \, a_{rs}
\end{align*}
\begin{table}[H]
\large
\centering
\begin{tabular}{l l}
Si $(p, q, r)$ es una permutación par de $(1, 2, 3)$, entonces   & $\Delta = \abs{A}$  \\
Si $(p, q, r)$ es una permutación impar de $(1, 2, 3)$, entonces & $\Delta = -\abs{A}$ \\
Si $(p, q, r)$ no es una permutación de $(1, 2, 3)$, entonces    & $\Delta = 0$
\end{tabular}
\end{table}
Entonces podemos escribir:
\begin{align*}
e_{ijk} \, a_{pi} \, a_{qj} \, a_{rk} = e_{pqr} \, \abs{A}
\end{align*}

\noindent
\textbf{Ejemplo: } La expresión $e_{ijk} \, B_{ij} \, C_{i}$ no tiene sentido ya que el índice $i$ se repite más de dos veces y la convención de suma no lo permite. Si realmente se deseaba sumar sobre un índice que ocurre más de dos veces, entonces debe usarse un signo de suma. Por ejemplo, la expresión anterior sería:
\begin{align*}
\sum_{i=1}^{n} e_{ijk} \, B_{ij} \, C_{i}
\end{align*}

\noindent
\textbf{Ejemplo: } El producto cruz de los vectores unitarios $\vu{e}_{1}, \vu{e}_{2}, \vu{e}_{3}$ se puede expresar en notación de índices como:
\begin{align*}
\vu{e}_{i} \times \vu{e}_{j} = \begin{cases}
\vu{e}_{k} & \mbox{si } (i, j, k) \mbox{ es una permutación par de } (1, 2, 3)   \\
-\vu{e}_{k} & \mbox{si } (i, j, k) \mbox{ es una permutación impar de } (1, 2, 3) \\
0 & \mbox{para todos los demás casos}
\end{cases}
\end{align*}
Este resultado se puede escribir de la forma:
\begin{align*}
\vu{e}_{i} \times \vu{e}_{j} = e_{kij} \, \vu{e}_{k}
\end{align*}
Este último resultado se puede verificar sumando sobre l índice $k$ y escribir las $9$ combinaciones posibles para $i$ y $j$.
\\[0.5em]
\noindent
\textbf{Ejemplo: } Dados los vectores $A_{p}$, con $p = 1, 2, 3$ y $B_{p}$ con $p = 1, 2, 3$ el producto cruz de los dos vectores  es un vector $C_{p}$ con $p = 1, 2, 3$, con componentes:
\begin{align}
C_{i} = e_{ijk} \, A_{j} \, B_{k} \hspace{1cm} i, j, k = 1, 2, 3
\label{eq:ecuacion_01_02}
\end{align}
Las cantidades $C_{i}$ representan las componentes del vector producto cruz:
\begin{align*}
\va{C} = \va{A} \cp \va{B} = C_{1} \, \vu{e}_{1} + C_{2} \, \vu{e}_{2} + C_{3} \, \vu{e}_{3}
\end{align*}
La ec. (\ref{eq:ecuacion_01_02}) define las componentes de $\va{C}$, se sumará en cada uno de los índices que
se repite. Sumando sobre el índice $k$:
\begin{align}
C_{i} = e_{ij1} \, A_{j} \, B_{1} + e_{ij2} \, A_{j} \, B_{2} + e_{ij3} \, A_{j} \, B_{3}
\label{eq:ecuacion_01_03}
\end{align}
La siguiente suma sobre el índice $j$ el cual se repite en cada término de la ecuación (\ref{eq:ecuacion_01_03}), obteniendo:
\begin{align}
\begin{aligned}
C_{i} & = e_{i11} \, A_{1} \, B_{1} + e_{i21} \, A_{2} \, B_{1} + e_{i31} \, A_{3} \, B_{1} \, + \\[0.5em]
& + e_{i12} \, A_{1} \, B_{2} + e_{i22} \, A_{2} \, B_{2} + e_{i32} \, A_{3} \, B_{2} \, + \\[0.5em]
& + e_{i13} \, A_{1} \, B_{3} + e_{i23} \, A_{2} \, B_{3} + e_{i33} \, A_{3} \, B_{3}
\end{aligned}
\label{eq:ecuacion_01_04}
\end{align}
Ahora nos quedamos con $i$ como un índice libre que puede tener cualquiera de los valores de $1, 2$ o $3$. Dejando que $i = 1$, luego dejando $i = 2$, y finalmente dejando que $i = 3$ produce los componentes del producto cruz:
\begin{align*}
C_{1} & = A_{2} \, B_{3} - A_{3} \, B_{2} \\[0.5em]
C_{2} & = A_{3} \, B_{1} - A_{1} \, B_{3} \\[0.5em]
C_{3} & = A_{1} \, B_{2} - A_{2} \, B_{1}
\end{align*}
El producto cruz también puede expresarse de la forma:
\begin{align*}
\va{A} \cp \va{B} = e_{ijk} \, A_{j} \, B_{k} \, \vu{e}_{i}
\end{align*}
\\[0.5em]
\noindent
\textbf{Ejemplo: } Demuestra que:
\begin{align*}
e_{ijk} = - e_{ikj} = e_{jki} \hspace{1cm} \text{para } i, j, k = 1, 2, 3
\end{align*}
El arreglo $i, j, k$ representa una permutación impar de los índices $i, j, k$, y para cada transposición hay un cambio de signo en el símbolo de la e-permutación. De manera similar, $j, k, i$ es una permutación par de $i, j, k$, y no hay un cambio de signo en el símbolo de e-permutación. Lo anterior es válido independientemente de valores numéricos asignados a los índices $i, j, k$.

\section{La identidad \texorpdfstring{$e-\delta$}{e-d}.}

Una identidad que relaciona el símbolo de e-permutación y la delta de Kronecker, que es útil en la simplificación de las expresiones tensoriales, es la identidad $e-\delta$. Esta identidad se puede expresar de diferentes formas. La forma de subíndice para esta identidad es la siguiente:
\begin{align*}
e_{ijk} \, e_{imn} = \delta_{jm} \, \delta_{kn} - \delta_{jn} \, \delta_{km}, \hspace{1cm} i, j, k, m, n = 1, 2, 3
\end{align*}
donde $i$ es el índice de suma y $j$, $k$, $m$, $n$ son índices libres. En la figura (\ref{fig:figura_01_03}) se muestra una manera esquemática para recordar las posiciones de los subíndices.
\par
Los subíndices de los cuatro delta de Kronecker en el lado derecho de la identidad $e-\delta$ se leen:
\begin{align*}
\mbox{(primero) (segundo) - (externo) (interno)}
\end{align*}
Esto se refiere a las posiciones que siguen al índice de suma. Por lo tanto, $j$, $m$ son los primeros índices después del índice de suma y $k$, $n$ son los segundos índices después del índice de suma. Los índices $j$, $n$ son índices externos cuando se comparan con los índices internos $k$, $m$ ya que los índices se consideran escritos en el lado izquierdo de la identidad.
\begin{figure}[H]
    \centering
    \begin{tikzpicture}[thick, font=\large]
        \node (i) at (0, 0) {$i$};
        \node (j) at (2, 0) {$j$};
        \node (k) at (4, 0) {$k$};
        \node (i) at (6, 0) {$i$};
        \node (m) at (8, 0) {$m$};
        \node (n) at (10, 0) {$n$};

        \draw (2, 0.5) -- (2, 2) -- (10, 2) node [midway, above] {externo} -- (10, 0.5);
        \draw (4, 0.5) -- (4, 1) -- (8, 1) node [midway, above] {interno} -- (8, 0.5);
        \draw (2, -0.5) -- (2, -1) -- (8, -1) node [midway, below] {primero} -- (8, -0.5);
        \draw (4, -0.5) -- (4, -2) -- (10, -2) node [midway, below] {segundo} -- (10, -0.5);
    \end{tikzpicture}
    \caption{Regla mnemónica para la posición de los subíndices.}
    \label{fig:figura_01_03}
\end{figure}

Otra forma de esta identidad ocupa tanto subíndices como superíndices, como se muestra a continuación:
\begin{align}
e^{ijk} \, e_{imn} = \delta_{m}^{j} \, \delta_{n}^{k} - \delta_{n}^{j} \, \delta_{m}^{k}
\label{eq:ecuacion_01_05}
\end{align}
Una forma de probar esta identidad es revisando que la ecuación (\ref{eq:ecuacion_01_05}) tiene los índices libres $j, k, m, n$. Cada uno de estos índices puede tener cualquiera de los valores de $1, 2$ o $3$. Hay $3$ opciones que podemos asignar a cada uno de $j, k, m$ o $n$ y esto da un total de $3^{4} = 81$ ecuaciones posibles representadas por la identidad (\ref{eq:ecuacion_01_05}). Al escribir las $81$ ecuaciones, podemos verificar que la identidad es verdadera para todas las combinaciones posibles que se pueden asignar a los índices libres.

\subsection*{Delta de Kronecker generalizada.}

La delta de Kronecker generalizada se define por el determinante de $n \times n$:
\begin{align*}
\delta_{m n \ldots p}^{i j \ldots k} = \mqty|
\delta_{m}^{i} & \delta_{n}^{i} & \ldots & \delta_{p}^{i} \\
\delta_{m}^{j} & \delta_{n}^{j} & \ldots & \delta_{p}^{j} \\
\vdots & \vdots & \ddots & \vdots \\
\delta_{m}^{k} & \delta_{n}^{k} & \ldots & \delta_{p}^{k} |
\end{align*}

Por ejemplo, en tres dimensiones podemos escribir:
\begin{align*}
\delta_{m n p}^{i j k} = \mqty|
\delta_{m}^{i} & \delta_{n}^{i} & \delta_{p}^{i} \\
\delta_{m}^{j} & \delta_{n}^{j} & \delta_{p}^{j} \\
\delta_{m}^{k} & \delta_{n}^{k} & \delta_{p}^{k}| =
e^{ijk} \, e_{mnp}
\end{align*}

\section{Aplicaciones de la notación de índices.}

La notación de índices, junto con la identidad $e-\delta$, se pueden utilizar para demostrar varias identidades vectoriales.
\par
\noindent
\textbf{Ejemplo.} Utilizando la notación de índices, demostrar que:
\begin{align*}
\va{A} \times \va{B} = - \va{B} \times \va{A}
\end{align*}

Para resolver este ejemplo, hagamos lo siguiente:
\begin{align*}
\va{C} & = \va{A} \times \va{B} = C_{1} \, \vu{e}_{1} + C_{2} \, \vu{e}_{2} + C_{3} \, \vu{e}_{3} = C_{i} \, \vu{e}_{i} \\[0.5em]
\va{D} & = \va{B} \times \va{A} = D_{1} \, \vu{e}_{1} + D_{2} \, \vu{e}_{2} + D_{3} \, \vu{e}_{3} = D_{i} \, \vu{e}_{i}
\end{align*}
Hemos demostrado que los componentes de los productos cruz  se pueden representar en la notación de índices como:
\begin{align*}
    C_{i} = e_{ijk} \, A_{j} \, B_{k} \hspace{1.5cm} \mbox{y} \hspace{1.5cm} D_{i} = e_{ijk} \, B_{j} \, A_{k}
\end{align*}
Entonces lo que debemos de demostrar es que: $D_{i} = - C_{i}$, para todos los valores de $i$.
\par
Consideremos los siguientes manejos: sean $B_{j} = B_{s} \, \delta_{sj}$ y $A_{k} = A_{m} \, \delta_{mk}$, para escribir:
\begin{align}
    D_{i} = e_{ijk} \, B_{j} \, A_{k} = e_{ijk} \, B_{s} \, \delta_{sj} \, A_{m} \, \delta_{mk}
    \label{eq:ecuacion_01_06}
\end{align}
donde todos los índices tienen el rango $1, 2, 3$. En la ecuación (\ref{eq:ecuacion_01_06}) observa que ningún índice de suma aparece más del doble porque si un índice aparece más del doble, la convención de suma no tiene sentido. Al reordenar los términos en la ecuación (\ref{eq:ecuacion_01_06}) tenemos:
\begin{align*}
    D_{i} = e_{ijk} \, \delta_{sj} \delta_{mk} \, B_{s} \, A_{m} = e_{ism} \, B_{s} \, A_{m}
\end{align*}
En esta expresión, los índices $s$ y $m$ son índices de mudos de suma y pueden reemplazarse por cualquier otra letra. Reemplazamos $s$ por $k$ y $m$ por $j$ para obtener:
\begin{align*}
    D_{i} = e_{ikj} \, A_{j} \, B_{k} = - e_{ijk} \, A_{j} \, B_{k} = - C_{i}
\end{align*}
Por lo que hemos encontrado:
\begin{align*}
    \va{D} = - \va{C} \hspace{1cm} \mbox{o} \hspace{1cm} \va{B} \times \va{A} = - \va{A} \times \va{B}
\end{align*}
es decir:
\begin{align*}
    \va{D} = D_{i} \, \vu{e}_{i} = - C_{i} \, \vu{e}_{i} = - \va{C}
\end{align*}
\textbf{Nota 1. } Las expresiones:
\begin{align*}
C_{i} = e_{ijk} \, A_{j} \, B_{k} \hspace{1.3cm} \text{y} \hspace{1.3cm} C_{m} = e_{mnp} \, A_{n} \, B_{p}
\end{align*}
teniendo todos los índices el rango $1, 2, 3$, parecen ser diferentes porque se utilizan letras diferentes como subíndices. Hay que recordar que ciertos índices se suman según la convención de la suma y los demás índices son índices libres y pueden tomar cualquier valor del rango asignado. Por lo tanto, después de la suma, cuando se sustituyen los índices involucrados por valores numéricos, ninguna de las letras ficticias utilizadas para representar los componentes aparece en la respuesta.
\par
\noindent
\textbf{Nota 2. } Un segundo punto importante es que cuando se trabaja con expresiones que involucran la notación de índice, los índices se pueden cambiar directamente. Por ejemplo, en la expresión anterior para $D_{i}$ podríamos haber reemplazado $j$ por $k$ y $k$ por $j$ simultáneamente (de modo que ningún índice se repita más de dos veces) para obtener:
\begin{align*}
D_{i} = e_{ijk} \, B_{j} \, A_{k} = e_{ikj} \, B_{k} \, A_{j} = - e_{ijk} \, A_{j} \, B_{k} = - C_{i}
\end{align*}
\noindent
\textbf{Nota 3. } Hay que tener cuidado al alternar entre la notación vectorial y la notación índice. Observa que se puede representar un vector $\va{A}$ como
\begin{align*}
\va{A} = A_{i} \, \vu{e}_{i}
\end{align*}
o sus componentes representadas por:
\begin{align*}
\va{A} \vdot \vu{e}_{i} = A_{i} \hspace{1.5cm} i = 1, 2, 3
\end{align*}
No hagamos un vector igual a un escalar. Es decir, no cometamos el error de escribir $\va{A} = A_{i}$ ya que esto es un mal uso del signo igual. No es posible que un vector sea igual a un escalar porque son dos cantidades completamente diferentes. Un vector tiene magnitud y dirección, mientras que un escalar solo tiene magnitud.

\section{Ejercicios.}

A manera de ejercicios, demuestra las siguientes identidades:
\begin{enumerate}
\item $\va{A} \vdot (\va{B} \cross \va{C}) = \va{B} \vdot (\va{C} \cross \va{A})$
\item $(\va{A} \cross \va{B}) \cross (\va{C} \cross \va{D}) = \va{C} \, ( \va{D} \vdot \va{A} \cross \va{B}) - \va{D} \, (\va{C} \vdot \va{A} \cross \va{B})$
\end{enumerate}

\end{document}