\documentclass[hidelinks,12pt]{article}
\usepackage[left=0.25cm,top=1cm,right=0.25cm,bottom=1cm]{geometry}
%\usepackage[landscape]{geometry}
\textwidth = 20cm
\hoffset = -1cm
\usepackage[utf8]{inputenc}
\usepackage[spanish,es-tabla]{babel}
\usepackage[autostyle,spanish=mexican]{csquotes}
\usepackage[tbtags]{amsmath}
\usepackage{nccmath}
\usepackage{amsthm}
\usepackage{amssymb}
\usepackage{mathrsfs}
\usepackage{graphicx}
\usepackage{subfig}
\usepackage{standalone}
\usepackage[outdir=./Imagenes/]{epstopdf}
\usepackage{siunitx}
\usepackage{physics}
\usepackage{color}
\usepackage{float}
\usepackage{hyperref}
\usepackage{multicol}
%\usepackage{milista}
\usepackage{anyfontsize}
\usepackage{anysize}
%\usepackage{enumerate}
\usepackage[shortlabels]{enumitem}
\usepackage{capt-of}
\usepackage{bm}
\usepackage{relsize}
\usepackage{placeins}
\usepackage{empheq}
\usepackage{cancel}
\usepackage{wrapfig}
\usepackage[flushleft]{threeparttable}
\usepackage{makecell}
\usepackage{fancyhdr}
\usepackage{tikz}
\usepackage{bigints}
\usepackage{scalerel}
\usepackage{pgfplots}
\usepackage{pdflscape}
\pgfplotsset{compat=1.16}
\spanishdecimal{.}
\renewcommand{\baselinestretch}{1.5} 
\renewcommand\labelenumii{\theenumi.{\arabic{enumii}})}
\newcommand{\ptilde}[1]{\ensuremath{{#1}^{\prime}}}
\newcommand{\stilde}[1]{\ensuremath{{#1}^{\prime \prime}}}
\newcommand{\ttilde}[1]{\ensuremath{{#1}^{\prime \prime \prime}}}
\newcommand{\ntilde}[2]{\ensuremath{{#1}^{(#2)}}}

\newtheorem{defi}{{\it Definición}}[section]
\newtheorem{teo}{{\it Teorema}}[section]
\newtheorem{ejemplo}{{\it Ejemplo}}[section]
\newtheorem{propiedad}{{\it Propiedad}}[section]
\newtheorem{lema}{{\it Lema}}[section]
\newtheorem{cor}{Corolario}
\newtheorem{ejer}{Ejercicio}[section]

\newlist{milista}{enumerate}{2}
\setlist[milista,1]{label=\arabic*)}
\setlist[milista,2]{label=\arabic{milistai}.\arabic*)}
\newlength{\depthofsumsign}
\setlength{\depthofsumsign}{\depthof{$\sum$}}
\newcommand{\nsum}[1][1.4]{% only for \displaystyle
    \mathop{%
        \raisebox
            {-#1\depthofsumsign+1\depthofsumsign}
            {\scalebox
                {#1}
                {$\displaystyle\sum$}%
            }
    }
}
\def\scaleint#1{\vcenter{\hbox{\scaleto[3ex]{\displaystyle\int}{#1}}}}
\def\bs{\mkern-12mu}


%\documentclass[12pt]{article}
\usepackage[utf8]{inputenc}
\usepackage[spanish,es-lcroman, es-tabla]{babel}
\usepackage[autostyle,spanish=mexican]{csquotes}
\usepackage{amsmath}
\usepackage{amssymb}
\usepackage{nccmath}
\numberwithin{equation}{section}
\usepackage{amsthm}
\usepackage{graphicx}
\usepackage{epstopdf}
\DeclareGraphicsExtensions{.pdf,.png,.jpg,.eps}
\usepackage{color}
\usepackage{float}
\usepackage{multicol}
\usepackage{enumerate}
\usepackage[shortlabels]{enumitem}
\usepackage{anyfontsize}
\usepackage{anysize}
\usepackage{array}
\usepackage{multirow}
\usepackage{enumitem}
\usepackage{cancel}
\usepackage{tikz}
\usepackage{circuitikz}
\usepackage{tikz-3dplot}
\usetikzlibrary{babel}
\usetikzlibrary{shapes}
\usepackage{bm}
\usepackage{mathtools}
\usepackage{esvect}
\usepackage{hyperref}
\usepackage{relsize}
\usepackage{siunitx}
\usepackage{physics}
%\usepackage{biblatex}
\usepackage{standalone}
\usepackage{mathrsfs}
\usepackage{bigints}
\usepackage{bookmark}
\spanishdecimal{.}

\setlist[enumerate]{itemsep=0mm}

\renewcommand{\baselinestretch}{1.5}

\let\oldbibliography\thebibliography

\renewcommand{\thebibliography}[1]{\oldbibliography{#1}

\setlength{\itemsep}{0pt}}
%\marginsize{1.5cm}{1.5cm}{2cm}{2cm}


\newtheorem{defi}{{\it Definición}}[section]
\newtheorem{teo}{{\it Teorema}}[section]
\newtheorem{ejemplo}{{\it Ejemplo}}[section]
\newtheorem{propiedad}{{\it Propiedad}}[section]
\newtheorem{lema}{{\it Lema}}[section]

\setlength{\jot}{12pt}
\title{Lista de identidades funciones Gamma y Beta \\ {\large Curso Matemáticas Avanzadas de la Física}\vspace{-1.5\baselineskip}}
\author{}
\date{}
\begin{document}
\maketitle
\fontsize{14}{14}\selectfont
%Referencia: Farrell -Solved Problems in Analysis as applied to Gama
Se presenta a continuación una lista de expresiones tanto para la función $\Gamma (x)$ como para la función $B (x, y)$. Cada una de ellas se puede demostrar a partir de la definición. No son todas, es posible construir nuevas expresiones a partir de las que se presentan.
\\\hspace*{\fill}

\textbf{Función Gamma.}

%\begin{fleqn}
%\setlength{\jot}{12pt}
\begingroup
\allowdisplaybreaks
\begin{align*}
%\setlength{\jot}{15pt}
\Gamma (x) &= \int_{0}^{\infty} t^{x-1} \, e^{-t} \dd{t} , \hspace{1cm} x > 0 \\
\Gamma (x) &= \int_{0}^{\infty} 2 \, u^{2 x - 1} \, e^{-u^{2}} , \hspace{1cm} x > 0 \\
\Gamma (x) &= \int_{0}^{1} \log \left( \dfrac{1}{\nu} \right)^{x - 1} \dd{\nu} \hspace{1.5cm} x > 0 \\
\Gamma (x) &= \dfrac{\Gamma(x + 1)}{x} \hspace{1.5cm} x \neq 0, -1, -2, \ldots \\
\Gamma (x) &= (x - 1) \, \Gamma(x - 1) \hspace{1.5cm} x \neq 0, -1, -2, \ldots \\
\Gamma (-x) &= \dfrac{\Gamma (1- x)}{-x} \hspace{1.5cm} x \neq 0, 1, 2, \ldots \\
\Gamma (n) &= (n - 1)!, \hspace{1.5cm} n = 1, 2, 3, \ldots, \hspace{1cm} \mbox{donde} \hspace{1cm} 0! = 1 \\
\Gamma (\dfrac{1}{2}) &= \sqrt{\pi} \\
\Gamma \left(n + \dfrac{1}{2} \right) &= \dfrac{1 \cdot 3 \cdot 5 \ldots (2 \, n - 1)\sqrt{\pi}}{2^{n}}, \hspace{1.5cm} n = 1, 2, 3, \ldots \\
\Gamma (x) \, \Gamma (1 - x) &= \dfrac{\pi}{\sin x \, \pi}, \hspace{1.5cm} x \neq 0, \pm 1, \pm 2, \pm 3, \ldots \\
n! &= \left( \dfrac{n}{e} \right)^{n} \, \sqrt{2 \, \pi \, n} + h \hspace{1.25cm} n = 1, 2, 3, \ldots, \hspace{1cm} 0 < \dfrac{h}{n!} < \dfrac{1}{12 \, n} \\
\int_{0}^{\infty} t^{a} \, e^{-b \, t^{c}} \dd{t} &= \dfrac{\Gamma \left(\dfrac{a + 1}{c} \right)}{c \, b^{(a+1)/c}}, \hspace{1.5cm} a > -1, \hspace{0.5cm} b > 0, \hspace{0.5cm} c >0
\end{align*}
\endgroup
%\end{fleqn}
\\\hspace*{\fill}

\textbf{Función Beta}
%\begin{fleqn}
\begin{align*}
B (x, y) &= \int_{0}^{1} t^{x - 1} \, (1 - t)^{y-1} \dd{t} \hspace{1.5cm} x > 0, \hspace{0.5cm} y > 0 \\
B (x, y) &= \int_{0}^{\pi/2} 2 \, \sin^{2 x - 1} \theta \, \cos^{2 y - 1} \theta \dd{t} \hspace{1.5cm} x > 0, \hspace{0.5cm} y > 0 \\
B (x, y) &= \int_{0}^{\infty} \dfrac{u^{x-1}}{(u + 1)^{x+y}} \dd{u} \hspace{1.5cm} x > 0, \hspace{0.5cm} y > 0 \\
B (x, y) &= \dfrac{\Gamma (x) \, \Gamma (y)}{\Gamma (x + y)} \\
B (x, y) &= B (y, x) \\
B (x, 2-x) &= \dfrac{\pi}{\sin x \, \pi}, \hspace{1.5cm} 0 < x < 1
\end{align*}
%\end{fleqn}

\end{document}