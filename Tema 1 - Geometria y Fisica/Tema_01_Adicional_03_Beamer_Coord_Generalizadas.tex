\documentclass[12pt]{beamer}
\usepackage{../Estilos/BeamerMAF}
\input{../Preambulos/preambulo_Beamer_Warsaw_seahorse}
\makeatletter
\setbeamertemplate{footline}
{
  \leavevmode%
  \hbox{%
  \begin{beamercolorbox}[wd=.333333\paperwidth,ht=2.25ex,dp=1ex,center]{section in foot}%
    \usebeamerfont{section in foot} \insertsection
  \end{beamercolorbox}%
  \begin{beamercolorbox}[wd=.333333\paperwidth,ht=2.25ex,dp=1ex,center]{subsection in foot}%
    \usebeamerfont{subsection in foot}  \insertsubsection
  \end{beamercolorbox}%
  \begin{beamercolorbox}[wd=.333333\paperwidth,ht=2.25ex,dp=1ex,right]{date in head/foot}%
    \usebeamerfont{date in head/foot} \insertshortdate{} \hspace*{2em}
    \insertframenumber{} / \inserttotalframenumber \hspace*{2ex} 
  \end{beamercolorbox}}%
  \vskip0pt%
}
\makeatother
\makeatletter
\patchcmd{\beamer@sectionintoc}{\vskip1.5em}{\vskip0.8em}{}{}
\makeatother
\date{10 de marzo de 2021}
\title{Sistemas de coordenadas generalizadas}
\subtitle{La física y la geometría}
\begin{document}
\maketitle
\fontsize{14}{14}\selectfont
\spanishdecimal{.}
\section*{Contenido}
\frame{\frametitle{Contenido}\tableofcontents[currentsection, hideallsubsections]}

\section{Sistemas de coordenadas}
\frame[allowframebreaks]{\tableofcontents[currentsection, hideothersubsections]}
%Ref. Moon (1988) - Field Theory Handbook. Section I

\subsection{Métrica de un sistema}

\begin{frame}
\frametitle{Coeficientes métricos}
Un sistema coordenado ortogonal $(u^{1}, u^{2}, u^{3})$ se puede expresar mediante los coeficientes métricos:
\begin{align*}
g_{11}, g_{22}, g_{33}
\end{align*}
\end{frame}
\begin{frame}
\frametitle{Tensor métrico}
El tensor métrico es un tensor de rango $2$ que define los conceptos como distancia, ángulo y volumen en un espacio euclídeo.
\pause
\begin{align*}
g = \mqty(
g_{11} & g_{12} & g_{13} \\
g_{21} & g_{22} & g_{23} \\
g_{31} & g_{32} & g_{33}
)
\end{align*}
\end{frame}
\begin{frame}
\frametitle{Relación con los factores de escala}
La relación que existe entre los coeficientes métricos y los factores de escala es la siguiente:
\pause
\begin{align*}
g = \mqty(
h_{1}^{2} & g_{12} & g_{13} \\
g_{21} & h_{2}^{2} & g_{23} \\
g_{31} & g_{32} & h_{3}^{2}
)
\end{align*}
\end{frame}
\begin{frame}
\frametitle{Diferencial de desplazamiento}
Definimos el diferencial del desplazamiento (una distancia infinitesimal) como:
\pause
\begin{align}
(\dd{s})^{2} = g_{11} \, (\dd{u^{1}})^{2} + g_{22} \, (\dd{u^{2}})^{2} + g_{33} \, (\dd{u^{3}})^{2}
\label{eq:ecuacion_01_01}
\end{align}
Donde \pause
\begin{align}
g_{ii} = \left( \pdv{x^{1}}{u^{i}} \right)^{2} + \left( \pdv{x^{2}}{u^{i}} \right)^{2} + \left( \pdv{x^{3}}{u^{i}} \right)^{2}
\label{eq:ecuacion_01_02}
\end{align}
y las $x^{i}$ son las coordenadas rectangulares.
\end{frame}
\begin{frame}
\frametitle{Ejemplo con coordenadas parabólicas}
Consideremos el sistema de coordenadas parabólicas, tal que: $u^{1} = \mu$, $u^{2} = \nu$, $u^{3} = \psi$.
\\
\bigskip
\pause
Entoces:
\begin{align*}
\begin{cases}
x^{1} = \mu \, \nu \, \cos \psi \\
x^{2} = \mu \, \nu \, \sin \psi \\
x^{3} = \dfrac{1}{2} (\mu^{2} - \nu^{2})
\end{cases}
\end{align*}  
\end{frame}
\begin{frame}
\frametitle{Coeficientes métricos del sistema}
Los coeficientes métricos para el sistema de coordenadas parabólico los obtenemos al usar la ec. (\ref{eq:ecuacion_01_02}):
\pause
\begin{eqnarray*}
g_{11} &=& \left( \pdv{x^{1}}{\mu} \right)^{2} + \left( \pdv{x^{2}}{\mu} \right)^{2} + \left( \pdv{x^{3}}{\mu} \right)^{2} = \\[0.5em] \pause
&=& \big( \nu \, \cos \psi \big)^{2} + \big( \nu \, \sin \psi \big)^{2} + \mu^{2} = \\[0.5em] \pause
&=& \mu^{2} + \nu^{2}
\end{eqnarray*}
\end{frame}
\begin{frame}
\frametitle{Coeficientes métricos del sistema}
Haciendo el cálculo de manera similar para los otros dos coeficientes métricos:
\begin{align*}
g_{22} &= \mu^{2} + \nu^{2} \\[0.5em]
g_{33} &= \mu^{2} \, \nu^{2}
\end{align*}
\pause
Una vez conocidos los coeficientes métricos, podemos obtener de manera fácil el volumen, el gradiente, rotacional, etc.
\end{frame}
\begin{frame}
\frametitle{Desplazamientos infinitesimales sobre los ejes}
La ec. (\ref{eq:ecuacion_01_01}) nos dice que las distancias infinitesimales sobre los ejes coordenados son:
\begin{align*}
\big( g_{11} \big)^{\frac{1}{2}} \dd{u^{1}}, \hspace{0.5cm} \big( g_{22} \big)^{\frac{1}{2}} \dd{u^{2}}, \hspace{0.5cm} \big( g_{33} \big)^{\frac{1}{2}} \dd{u^{3}}
\end{align*}
\end{frame}
\begin{frame}
\frametitle{Elemento de área}
El elemento de área sobre la superficie $u^{1} \, u^{2}$ está dado por:
\begin{eqnarray}
\dd{A} &=& \big[ (g_{11})^{\frac{1}{2}} \dd{u^{1}} \big] \, \big[ (g_{22})^{\frac{1}{2}} \dd{u^{2}} \big] = \nonumber \\[0.5em] \pause
&=& (g_{11} \, g_{22})^{\frac{1}{2}} \dd{u^{1}} \dd{u^{2}}
\label{ec:ecuacion_01_03}
\end{eqnarray}
\end{frame}
\begin{frame}
\frametitle{Elemento de volumen}
De manera similar, el elemento de volumen está dado por:
\begin{eqnarray}
\dd{V} &=& (g_{11} \, g_{22} \, g_{33})^{\frac{1}{2}} \, \dd{u^{1}} \dd{u^{2}} \dd{u^{3}} = \nonumber \\[0.5em] \pause
&=& g^{\frac{1}{2}} \, \dd{u^{1}} \dd{u^{2}} \dd{u^{3}} \label{eq:ecuacion_01_04}
\end{eqnarray}
\end{frame}
\begin{frame}
\frametitle{Ejemplo: Área de un paraboloide de revolución}
¿Cuál es el área de un paraboloide de revolución $(\mu = \mu_{0})$ a partir del vértice a una altura definida por $\nu_{0}$?
\\
\bigskip
\pause
Vamos a usar las coordenadas parabólicas ($\mu, \nu, \psi)$, y con:
\begin{align*}
g_{11} = g_{22} = \mu^{2} + v^{2} \hspace{1cm} g_{33} = \mu^{2} \, \nu^{2}
\end{align*}
\end{frame}
\begin{frame}
\frametitle{Área del paraboloide de revolución}
Entonces al ocupar la ec. (\ref{ec:ecuacion_01_03}), llegamos a: \pause
\begin{align*}
\dd{A} = \mu_{0} \, \nu \, \big( \mu_{0}^{2} + \nu^{2} \big)^{\frac{1}{2}} \dd{\nu} \dd{\psi}
\end{align*}
\\
\bigskip
\pause
¿Cuál es el área total de paraboloide?
\end{frame}
\begin{frame}
\frametitle{Área del paraboloide de revolución}
El área total es:
\begin{eqnarray*}
A &=& \int_{0}^{2 \pi} \int_{0}^{\nu_{0}} \mu_{0} \, \nu \, \big( \mu_{0}^{2} + \nu^{2} \big)^{\frac{1}{2}} \dd{\nu} \dd{\psi} = \\[0.5em] \pause
&=& \dfrac{2 \, \pi \, \mu_{0}}{3} \, \big[ \big( \mu_{0}^{2} + \nu^{2} \big)^{\frac{3}{2}} - \mu_{0}^{2} \big]
\end{eqnarray*}
\end{frame}
%Ref. Spiegel - Análisis vectorial. Cap. 7 Coord. generalizadas
\section{Expresando ecuaciones de la física}
\frame{\tableofcontents[currentsection, hideothersubsections]}
\subsection{Utilidad de las coordenadas generalizadas}

\begin{frame}
\frametitle{Presentando ecuaciones conocidas}
La utilidad de expresar una ecuación de la física matemática en un sistema coordenado distinto al cartesiano, radica en el hecho de que facilita su solución.
\end{frame}
\begin{frame}
\frametitle{Primer paso hecho}
Si el primer paso ha sido manejar el cambio de un sistema coordenado a otro, el siguiente paso que veremos ya en el Tema 2, se ocupará de resolver la ecuación obtenida con alguna técnica de solución.
\end{frame}
\begin{frame}
\frametitle{Ecuación de Laplace}
Veamos el siguiente ejemplo: expresa la ecuación de Laplace:
\begin{align*}
\laplacian{\psi} = 0
\end{align*}
en el sistema de coordenadas cilíndricas parabólicas.
\end{frame}
\begin{frame}
\frametitle{El Laplaciano en coordenadas generalizadas}
El Laplaciando en coordenadas curvilíneas generalizadas se expresa por:
\begin{align*}
\laplacian{\psi} &= \dfrac{1}{h_{1} h_{2} h_{3}} \left[ \pdv{u^{1}} \left( \dfrac{h_{2} h_{3}}{h_{1}} \pdv{\psi}{u^{1}} \right) + \pdv{u^{2}} \left( \dfrac{h_{3} h_{1}}{h_{2}} \pdv{\psi}{u^{2}} \right) + \right. \\[1em]
+& \left. \pdv{u^{3}} \left( \dfrac{h_{1} h_{2}}{h_{3}} \pdv{\psi}{u^{3}} \right) \right]
\end{align*}
\end{frame}
\begin{frame}
\frametitle{Usando la información del sistema parabólico}
Para el sistema de coordenadas cilíndricas parabólicas, sabemos que: \pause
\begin{align*}
u^{1} = u, \hspace{1cm} u^{2} = v, \hspace{1cm} u^{3} = z
\end{align*}
\pause
\begin{align*}
x &= \dfrac{1}{2}  \big( u^{2} - v^{2} \big) \\[0.5em]
y &=  u \, v \\[0.5em]
z &= z
\end{align*}
\end{frame}
\begin{frame}
\frametitle{La matriz métrica}
La matriz métrica de este sistema coordenado cilíndrico parabólico es:
\begin{align*}
g = \mqty(
u^{2} + v^{2} & 0 & 0 \\
0 & u^{2} + v^{2} & 0 \\
0 & 0 & 1
)
\end{align*}
\end{frame}
\begin{frame}
\frametitle{Los factores de escala}
Por lo que los factores de escala son:
\begin{align*}
h_{u} = h_{v} &= \sqrt{u^{2} + v^{2}} \\[0.5em]
h_{z} &= 1
\end{align*}
\end{frame}
\begin{frame}
\frametitle{El Laplaciano}
Al ocupar los factores de escala y sustiyendo en la expresión para el Laplaciano, tenemos que:
\begin{align*}
\laplacian{\psi} &= \dfrac{1}{u^{2} +v^{2}} \left[ \pdv{u} \left( \pdv{\psi}{u} \right) + \pdv{v} \left( \pdv{\psi}{v} \right) + \right. \\[1em]
+& \left. \pdv{z} \left( (u^{2} + v^{2}) \pdv{\psi}{z} \right) \right] =
\end{align*}
\end{frame}
\begin{frame}
\frametitle{El resultado}
\begin{align*}
= \dfrac{1}{u^{2} + v^{2}} \left( \pdv[2]{\psi}{u} + \pdv[2]{\psi}{v} \right) + \pdv[2]{\psi}{z}
\end{align*}
\pause
Entonces tenemos que:
\begin{align*}
\pdv[2]{\psi}{u} + \pdv[2]{\psi}{v} + (u^{2} + v^{2}) \, \pdv[2]{\psi}{z} = 0
\end{align*}
\end{frame}
\begin{frame}
\frametitle{Ecuación de calor }
Expresa la ecuación de calor
\begin{align*}
\pdv{U}{t} = \kappa \, \laplacian{U}
\end{align*}
en un sistema de coordenadas cilíndricas elípticas.
\end{frame}
\begin{frame}
\frametitle{Coordenadas cilíndricas elípticas}
Para este sistema tenemos que:
\begin{align*}
u^{1} = u, \hspace{1cm} u^{2} = v, \hspace{1cm} u^{3} = z
\end{align*}
\pause
\begin{align*}
x &= a \, \cosh u \, \cos v \\[0.5em]
y &=  a \, \sinh u \, \sin v \\[0.5em]
z &= z
\end{align*}
\end{frame}
\begin{frame}
\frametitle{La matriz métrica}
La matriz métrica de este sistema coordenado cilíndrico parabólico es:
\begin{align*}
g = \mqty(
a^{2} \,\sinh^{2} u + \sin^{2} v & 0 & 0 \\
0 & a^{2} \, \sinh^{2} u + \sin^{2} v & 0 \\
0 & 0 & 1
)
\end{align*}
\end{frame}
\begin{frame}
\frametitle{Los factores de escala}
Por lo que los factores de escala son:
\begin{align*}
h_{u} = h_{v} &= a \, \sqrt{\sinh^{2} u + \sin^{2} v} \\[0.5em]
h_{z} &= 1
\end{align*}
\end{frame}
\begin{frame}
\frametitle{Ajustando el Laplaciano}
Ocupamos la expresión para el Laplaciano en coordenadas curvilíneas generalizadas:
\begin{align*}
\laplacian{U} &= \dfrac{1}{a^{2} (\sinh^{2} u + \sin^{2} v)} \left[ \pdv{u} \left( \pdv{U}{u} \right) + \right. \\[1em]
&+ \left. \pdv{v} \left( \pdv{U}{v} \right) + \pdv{z} \big( a^{2} (\sinh^{2} u + \sin^{2} v) \big) \, \pdv{U}{z} \right] =
\end{align*}
\end{frame}
\begin{frame}
\frametitle{El Laplaciano}
\begin{align*}
= \dfrac{1}{a^{2} (\sinh^{2} u + \sin^{2} v)}  \left[ \pdv[2]{U}{u} + \pdv[2]{U}{v} \right] + \pdv[2]{U}{z}
\end{align*}
Por lo que la ecuación de calor en coordenadas cilíndricas parabólicas resulta:
\begin{align*}
\setlength{\fboxsep}{2\fboxsep}\boxed{
\pdv{U}{t} = \kappa \, \dfrac{1}{a^{2} (\sinh^{2} u + \sin^{2} v)}  \left[ \pdv[2]{U}{u} + \pdv[2]{U}{v} \right] + \pdv[2]{U}{z}
}
\end{align*}
\end{frame}

\section{Propiedades sistemas generalizados}
\frame{\tableofcontents[currentsection, hideothersubsections]}

\subsection{Coordenadas generalizadas}

\begin{frame}
\frametitle{Característica importante}
Consideremos el sistema de coordenadas generales $u_{1}, u_{2}, u_{3}$.
\\
\bigskip
Demuestra que
\begin{align*}
\pdv{\vb{r}}{u_{1}}, \pdv{\vb{r}}{u_{2}}, \pdv{\vb{r}}{u_{3}} \hspace{0.5cm} \mbox{y} \hspace{0.5cm} \grad{u_{1}}, \grad{u_{2}}, \grad{u_{3}}  
\end{align*}
son sistemas de vectores recíprocos.
\end{frame}
\begin{frame}
\frametitle{El punto a demostrar}
Lo que se nos pide en el ejercicio es demostrar que:
\begin{align*}
\pdv{\vb{r}}{u_{p}} \cdot \grad{u_{q}} = \begin{cases}
1 & \mbox{si } p = q \\
0 & \mbox{si } p \neq q
\end{cases}
\end{align*}
con $p, q = 1, 2, 3$
\end{frame}
\begin{frame}
\frametitle{Solución al ejercicio}
Tenemos que:
\begin{align*}
\dd{\vb{r}} = \pdv{\vb{r}}{u_{1}} \dd{u_{1}} + \pdv{\vb{r}}{u_{2}} \dd{u_{2}} + \pdv{\vb{r}}{u_{3}} \dd{u_{3}}
\end{align*}
\end{frame}
\begin{frame}
\frametitle{Solución al ejercicio}
Al multiplicar por $\grad{u_{1}} \cdot$, \pause tenemos que:
\begin{eqnarray*}
&{}& \grad{u_{1}} \cdot \dd{\vb{r}} = \pause \dd{u_{1}} = \\[0.5em] \pause
&=& \left( \grad{u_{1}} \cdot \pdv{\vb{r}}{u_{1}} \right) \dd{u_{1}} {+} \left( \grad{u_{1}} \cdot \pdv{\vb{r}}{u_{2}} \right) \dd{u_{2}} {+} \\[0.5em]
&+& \left( \grad{u_{1}} \cdot \pdv{\vb{r}}{u_{3}} \right) \dd{u_{3}}
\end{eqnarray*}
\end{frame}
\begin{frame}
\frametitle{Solución al ejercicio}
Que nos dice: \pause
\begin{align*}
\grad{u_{1}} \cdot \pdv{\vb{r}}{u_{1}} = 1, \hspace{0.5cm} \grad{u_{1}} \cdot \pdv{\vb{r}}{u_{2}} = 0, \hspace{0.5cm} \grad{u_{1}} \cdot \pdv{\vb{r}}{u_{3}}  = 0
\end{align*}
\pause
\\
\bigskip
De manera análoga, las dos relaciones restantes, se demuestran con la multiplicación por
\begin{align*}
\grad{u_{2}} \cdot \hspace{1cm} \grad{u_{3}} \cdot
\end{align*}
\end{frame}
\begin{frame}
\frametitle{Semana 3 del curso}
La semana del 14 al 19 de marzo (el lunes 15 es día feriado) se deberán de atender las actividades de trabajo para el Tema de las funciones Gamma y Beta.
\\
\bigskip
\pause
El lunes 15 de marzo se publicará en Moodle el bloque de la Semana 3 con los materiales de trabajo, lista de ejercicios a cuenta, el espacio para el envío de los ejercicios de la Semana 2.
\end{frame}
\begin{frame}
\frametitle{Primer - Examen Tarea}
Al estar ya en la fase final del Tema 1, se enviará durante la siguiente semana la relación de ejercicios para el examen - tarea 1 del curso.
\\
\bigskip
\pause
Con las indicaciones de solución y la fecha de entrega.
\end{frame}
\end{document}