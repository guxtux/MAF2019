\documentclass[12pt]{beamer}
\usepackage{../Estilos/BeamerMAF}
\usepackage{../Estilos/ColoresLatex}
\input{../Preambulos/preambulo_Beamer_Warsaw_seahorse}
\usepackage{pifont}
\captionsetup{labelformat=empty,labelsep=none}

\AtBeginDocument{\RenewCommandCopy\qty\SI}
\ExplSyntaxOn
\msg_redirect_name:nnn { siunitx } { physics-pkg } { none }
\ExplSyntaxOff


\title{\large{1 - Coordenadas curvilíneas ortogonales}}
\subtitle{Tema 1 - La física y la geometría}

\author{M. en C. Gustavo Contreras Mayén}
\date{}

\begin{document}
\maketitle
\fontsize{14}{14}\selectfont

\section*{Contenido}
\frame{\frametitle{Contenido} \tableofcontents[currentsection, hideallsubsections]}

\section{Coordenadas curvilíneas ortogonales}
\frame{\frametitle{Temas a revisar} \tableofcontents[currentsection, hideothersubsections]}
\subsection{Coord. cartesianas}

%Ref. Sepúlveda (2004) - 1 Coord. curvilíneas ortogonales
\begin{frame}
\frametitle{Coordenadas cartesianas}
Consideremos el espacio euclidiano tridimensional con las coordenadas cartesianas, \pause vamos a definir una de ellas, por ejemplo $x_{0} = \text{constante}$ y las otras dos coordenadas $y$ y $z$ como variables, es decir:
\pause
\begin{align*}
x &= x_{0}, \hspace{0.25cm} y = y, \hspace{0.25cm} z = z
\end{align*}
\end{frame}
\begin{frame}
\frametitle{Dominio de las coordenadas}
Hay que considerar que el dominio de las coordenadas $x$, $y$ y $z$ es todo el espacio tridimensional, es decir:
\pause
\begin{align*}
   -\infty \leq x \leq +\infty, \\[0.5em]
   -\infty \leq y \leq +\infty, \\[0.5em]
   -\infty \leq z \leq +\infty
\end{align*}
\end{frame}
\begin{frame}
\frametitle{Planos constantes}
A continuación se presentan las superficies coordenadas para cada una de las variables, y al final, la intersección de los tres planos.
\end{frame}
% \begin{frame}
% Para el caso de $(x_{0}, y, z)$, se tiene:
% \begin{align*}
% x = x_{0}, \hspace{0.25cm} y = y_{0}, \hspace{0.25cm} z = z_{0}
% \end{align*}
% \pause
% Como se ve en la figura \ref{fig:plano_x}:
% \end{frame}
{ % Start of local scope
\setbeamercolor{background canvas}{bg=black} 
\begin{frame}
\frametitle{Planos constantes $x_{0}$}
\begin{figure}[H]
   \centering
   \includegraphics[width=0.6\linewidth]{Imagenes/superficies_01_plano_x.png}
   \label{fig:plano_x}
\end{figure}
\end{frame}
\begin{frame}\frametitle{Planos constantes $y_{0}$}
\begin{figure}[H]
   \centering
   \includegraphics[width=0.6\linewidth]{Imagenes/superficies_02_plano_y.png}
   \label{fig:plano_y}
\end{figure}
\end{frame}
\begin{frame}
\frametitle{Planos constantes $z_{0}$}
\begin{figure}[H]
   \centering
   \includegraphics[width=0.6\linewidth]{Imagenes/superficies_03_plano_z.png}
   \label{fig:plano_z}
\end{figure}
\end{frame}
\begin{frame}
\frametitle{Intersección de los tres planos constantes}
\begin{figure}[H]
   \centering
   \includegraphics[width=0.6\linewidth]{Imagenes/superficies_04_tres_planos.png}
   \label{fig:tresplanos}
\end{figure}
\end{frame}
}
% \begin{frame}
% \frametitle{Coordenadas cartesianas}
% La intersección de los planos $x = x_{0}$ y $y = y_{0}$ genera una línea recta paralela al eje $z$ que pasa por el punto $(x_{0}, y_{0}, 0)$.
% \end{frame}
\begin{frame}
\frametitle{Coordenadas cartesianas}
La intersección de los tres planos:
\begin{align*}
x = x_{0}, y = y_{0}, z = z_{0}   
\end{align*}
genera un punto de coordenadas $(x_{0}, y_{0}, z_{0})$.
\end{frame}

\subsection{Coord. cilíndricas}

\begin{frame}
\frametitle{Estudiando el sistema coordenado}
Hemos comentado que hay que estudiar el sistema coordenado que se nos proporcione, ya sea por algún problema en particular o el que se nos haya dado.
\end{frame}
\begin{frame}
\frametitle{Las reglas de transformación}
El primer punto que debemos de considerar, son las reglas de transformación entre coordenadas cartesianas y las coordenadas del sistema que se nos proporcione.
\end{frame}
\begin{frame}
\frametitle{Las reglas de transformación}
Estas reglas siempre se nos proporcionarán así como su dominio, \pause para el caso del sistema coordenado cilíndrico $(\rho, \phi, z)$, las reglas de transformación son:
\pause
\begin{minipage}[t]{0.4\linewidth}
\begin{align*}
x &= \rho \, \cos \phi \\[0.5em]
y &= \rho \, \sin \phi \\[0.5em]
z &= z
\end{align*}
\end{minipage}
\begin{minipage}[t]{0.4\linewidth}
\begin{align*}
   -\infty \leq &\rho \leq +\infty \\[0.5em]
   0 \leq &\phi < 2 \, \pi \\[0.5em]
   -\infty \leq &z \leq +\infty
\end{align*}
\end{minipage}
\end{frame}
\begin{frame}
\frametitle{Superficies coordenadas}
El segundo paso que debemos de considerar, es la construcción de las superficies coordenadas.
\\
\bigskip
\pause
Fijamos una coordenada como constante, por ejemplo $\rho = \rho_{0}$, y las otras dos como variables, es decir:
\begin{align*}
\rho &= \rho_{0}, \hspace{0.25cm} \phi = \phi, \hspace{0.25cm} z = z
\end{align*}
\end{frame}
\begin{frame}
\frametitle{Uso de la geometría del espacio}
Tendremos que apoyarnos con el uso de la geometría del espacio, para poder identificar las superficies coordenadas.
\end{frame}
\begin{frame}
\frametitle{Superficie con $r_{0}$ constante}
Tenemos lo siguiente:
\pause
\begin{align*}
x &= \rho \, \cos \phi \hspace{1.5cm} y &= \rho \, \sin \phi \hspace{1.5cm} z &= z
\end{align*}
\pause
Elevamos al cuadrado:
\begin{align*}
x^{2} = \rho^{2} \, \cos^{2} \phi \hspace{1.5cm} y^{2} = \rho^{2} \, \sin^{2} \phi \hspace{1.5cm} z^{2} = z^{2}
\end{align*}
\end{frame}
\begin{frame}
\frametitle{Superficie con $\rho_{0}$ constante}
Sumamos $x^{2} + y^{2}$  y encontramos que:
\pause
\begin{eqnarray*}
\begin{aligned}
x^{2} + y^{2} &= \rho^{2} \, \cos^{2} \phi + \rho^{2} \, \sin^{2} \phi \\[0.5em] \pause
&= \rho^{2} \left( \cos^{2} \phi + \sin^{2} \phi \right) = \\[0.5em] \pause
&= \rho^{2} 
\end{aligned}
\end{eqnarray*}
lo que nos define una circunferencia de radio $\rho$ en el plano $xy$.
\end{frame}
\begin{frame}
\frametitle{Primera superficie coordenada}
Con $\rho_{0}$ constante, tenemos que al variar $\phi$ y $z$, se genera un cilindro circular de radio $\rho_{0}$, que es la primera superficie coordenada del sistema cilíndrico.
\end{frame}
{ % Start of local scope
\setbeamercolor{background canvas}{bg=black} 
\begin{frame}
\frametitle{Primera superficie coordenada}
\begin{figure}
   \centering
   \includegraphics[width=0.95\linewidth]{Imagenes/superficies_cilindricas_01.png}
   \label{fig:superficies_cilindricas_01}
\end{figure}
\end{frame}
}
\begin{frame}
\frametitle{Segunda superficie coordenada}
Fijamos ahora la coordenada $\phi = \phi_{0}$, y las otras dos como variables, es decir:
\begin{align*}
\rho &= \rho, \hspace{0.25cm} \phi = \phi_{0}, \hspace{0.25cm} z = z
\end{align*}
\pause
Lo que nos define un plano meridiano.
\end{frame}
{ % Start of local scope
\setbeamercolor{background canvas}{bg=black} 
\begin{frame}
\frametitle{Segunda superficie coordenada}
\begin{figure}
   \centering
   \includegraphics[width=0.95\linewidth]{Imagenes/superficies_cilindricas_02.png}
   \label{fig:superficies_cilindricas_02}
\end{figure}
\end{frame}
}
\begin{frame}
\frametitle{Tercera superficie coordenada}
Por último, fijamos la coordenada $z = z_{0}$, y las otras dos como variables, es decir:
\begin{align*}
\rho &= \rho, \hspace{0.25cm} \phi = \phi, \hspace{0.25cm} z = z_{0}
\end{align*}
\pause
Lo que nos define un plano horizontal.
\end{frame}
{ % Start of local scope
\setbeamercolor{background canvas}{bg=black} 
\begin{frame}
\frametitle{Tercera superficie coordenada}
\begin{figure}
   \centering
   \includegraphics[width=0.95\linewidth]{Imagenes/superficies_cilindricas_03.png}
   \label{fig:superficies_cilindricas_03}
\end{figure}
\end{frame}
}
\begin{frame}
\frametitle{Intersección de las superficies}
Al intersectar las tres superficies coordenadas, se define un punto de coordenadas $(\rho_{0}, \phi_{0}, z_{0})$.
\end{frame}
{ % Start of local scope
\setbeamercolor{background canvas}{bg=black} 
\begin{frame}
\frametitle{Intersección de las superficies}
\begin{figure}
   \centering
   \includegraphics[width=0.95\linewidth]{Imagenes/superficies_cilindricas_04.png}
   \label{fig:superficies_cilindricas_04}
\end{figure}
\end{frame}
}
\begin{frame}
\frametitle{Primer paso completo}
Una vez que se revisan las reglas de transformación y las superficies coordenadas, podemos decir que hemos completado el primer paso para estudiar el sistema coordenado cilíndrico.
\\
\bigskip
\pause
Veremos otro ejercicio antes de avanzar en la construcción de los sistemas coordenados curvilíneos ortogonales.
\end{frame}

\subsection{Coord. esféricas}

\begin{frame}
\frametitle{Sistema coordenado esférico}
Ahora trabajaremos con el sistema coordenado esférico.
\\
\bigskip
\pause
El primer paso es revisar las reglas de transformación entre coordenadas cartesianas y coordenadas esféricas.
\end{frame}
\begin{frame}
\frametitle{Reglas de transformación}
La conexión entre coordenadas cartesianas y esféricas (reglas de transformación) tiene la forma:
\pause
\begin{minipage}[t]{0.4\linewidth}
\begin{align*}
x &= r \, \sin \theta \cos \varphi \\
y &= r \sin \theta \sin \varphi \\
z &= r \cos \theta
\end{align*}
\end{minipage}
\begin{minipage}[t]{0.4\linewidth}
\begin{align*}
   0 \leq \, &r < +\infty \\[0.5em]
   0 \leq \, &\theta < \pi \\[0.5em]
   0 \leq \, &\varphi < 2 \, \pi
\end{align*}
\end{minipage}
\end{frame}
\begin{frame}
\frametitle{Superficies coordenadas}
Ya sabemos que para obtener las superficies coordenadas, fijamos una coordenada como constante y las otras dos como variables.
\end{frame}
\begin{frame}
\frametitle{Primera superficie coordenada}
Fijamos la coordenada $r = r_{0}$, y las otras dos como variables, es decir:
\begin{align*}
r &= r_{0}, \hspace{0.25cm} \theta = \theta, \hspace{0.25cm} \varphi = \varphi
\end{align*}
\end{frame}
\begin{frame}
\frametitle{Apoyo de la geometría del espacio}
Como en cada una de las expresiones de las coordenadas, tenemos un componente trigonométrico, debemos de manejar la geometría del espacio para identificar las superficies coordenadas.
\end{frame}
\begin{frame}
\frametitle{Apoyo de la geometría del espacio}
Tenemos que:
\begin{align*}
x = r \, \sin \theta \cos \varphi \hspace{1cm} y = r \sin \theta \sin \varphi \hspace{1cm} z = r \cos \theta
\end{align*}
\pause
Elevamos al cuadrado:
\begin{align*}
x^{2} &= r^{2} \, \sin^{2} \theta \cos^{2} \varphi \\[0.25em]
y^{2} &= r^{2} \, \sin^{2} \theta \sin^{2} \varphi \\[0.25em]
z^{2} &= r^{2} \cos^{2} \theta
\end{align*}
\end{frame}
\begin{frame}
\frametitle{Apoyo con la geometría del espacio}
Sumamos las tres expresiones para obtener:
\pause
\begin{eqnarray*}
\begin{aligned}
x^{2} + y^{2} + z^{2} &= r^{2} \sin^{2} \theta \cos^{2} \varphi {+} r^{2} \sin^{2} \theta \sin^{2} \varphi {+} r^{2} \cos^{2} \theta \\[0.5em] \pause
&= r^{2} \left( \sin^{2} \theta \cos^{2} \varphi + \sin^{2} \theta \sin^{2} \varphi + \cos^{2} \theta \right) \\[0.5em] \pause
&= r^{2} \left( \sin^{2} \theta \left( \cos^{2} \varphi + \sin^{2} \varphi \right) + \cos^{2} \theta \right) \\[0.5em] \pause
&= r^{2} \left( \sin^{2} \theta + \cos^{2} \theta \right) = \\[0.5em]
&=  r^{2}
\end{aligned}
\end{eqnarray*}
\end{frame}
\begin{frame}
\frametitle{Primer superficie coordenada}
Tendremos entonces que la primera superficie coordenada esférica, con $r_{0}$ constante, es una esfera de radio $r_{0}$, que se extiende indefinidamente en todas direcciones.
\end{frame}
{ % Start of local scope
\setbeamercolor{background canvas}{bg=black}
\begin{frame}
\frametitle{Primera superficie coordenada}
\begin{figure}
   \centering
   \includegraphics[width=0.95\linewidth]{Imagenes/superficies_esfericas_01.png}
   \label{fig:superficies_esfericas_01}
\end{figure}
\end{frame}
}
\begin{frame}
\frametitle{Segunda superficie coordenada}
Para la segunda superficie coordenada, fijamos la coordenada $\theta = \theta_{0}$, y las otras dos como variables, es decir:
\begin{align*}
r &= r, \hspace{0.25cm} \theta = \theta_{0}, \hspace{0.25cm} \varphi = \varphi
\end{align*}
\end{frame}
\begin{frame}
\frametitle{Segunda superficie coordenada}
Tenemos que:
\begin{align*}
   z = r \cos \theta_{0} \hspace{1cm} \sqrt{x^{2} + y^{2}} = r \, \sin \theta_{0}
\end{align*}
\end{frame}
\begin{frame}
\frametitle{Manejando las expresiones}
Igualamos con $r$ las dos expresiones anteriores:
\pause
\begin{align*}
   r = \dfrac{\sqrt{x^{2} + y^{2}}}{\sin \theta_{0}}
\end{align*}
\end{frame}
\begin{frame}
\frametitle{Manejando las expresiones}
Sustituimos el valor de $r$ en la expresión de $z$:
\pause
\begin{eqnarray*}
\begin{aligned}
z &= \left[ \dfrac{\sqrt{x^{2} + y^{2}}}{\sin \theta_{0}} \right] \, \cos \theta_{0} \\[0.5em] \pause
&= \dfrac{\sqrt{x^{2} + y^{2}} \, \cos \theta_{0}}{\sin \theta_{0}} \\[0.5em] \pause
&= \sqrt{x^{2} + y^{2}} \, \cot \theta_{0}
\end{aligned}
\end{eqnarray*}
\end{frame}
\begin{frame}
\frametitle{Manejando las expresiones}
Elevamos al cuadrado la expresión resultante:
\pause
\begin{align*}
   z^{2} = (x^{2} + y^{2}) \, \cot^{2} \theta_{0}
\end{align*}
Que es la ecuación de un cono circular recto con vértice en el origen y simetría alrededor del eje $z$.
\end{frame}
{
\setbeamercolor{background canvas}{bg=black}
\begin{frame}
\frametitle{Segunda superficie coordenada}
\begin{figure}
   \centering
   \includegraphics[width=0.95\linewidth]{Imagenes/superficies_esfericas_02.png}
   \label{fig:superficies_esfericas_02}
\end{figure}
\end{frame}
}
\begin{frame}
\frametitle{Tercera superficie coordenada}
Ahora nos resta considerar la tercera superficie coordenada, fijamos la coordenada $\varphi = \varphi_{0}$, y las otras dos como variables, es decir:
\begin{align*}
r &= r, \hspace{0.25cm} \theta = \theta, \hspace{0.25cm} \varphi = \varphi_{0}
\end{align*}
\end{frame}
\begin{frame}
\frametitle{Tercera superficie coordenada}
De las expresiones de las coordenadas esféricas, tenemos que:
\pause
\begin{align*}
x = r \, \sin \theta \cos \varphi_{0} \hspace{1cm} y = r \sin \theta \sin \varphi_{0}
\end{align*}
\end{frame}
\begin{frame}
\frametitle{Manejando las expresiones}
Dividimos las dos expresiones anteriores:
\pause
\begin{eqnarray*}
\begin{aligned}
\dfrac{y}{x} &= \dfrac{r \sin \theta \sin \varphi_{0}}{r \sin \theta \cos \varphi} \\[0.5em]
&= \tan \varphi_{0} \\[0.5em] \pause
\varphi_{0} &= \tan^{-1} \left( \dfrac{y}{x} \right)
\end{aligned}
\end{eqnarray*}
Que nos define un plano meridiano que pasa por el eje $z$ y forma un ángulo $\varphi_{0}$ con el eje $x$.
\end{frame}
{ % Start of local scope
\setbeamercolor{background canvas}{bg=black}
\begin{frame}
\frametitle{Tercera superficie coordenada}
\begin{figure}
   \centering
   \includegraphics[width=0.95\linewidth]{Imagenes/superficies_esfericas_03.png}
   \label{fig:superficies_esfericas_03}
\end{figure}
\end{frame}
}
\begin{frame}
\frametitle{Intersección de las superficies}
Al intersectar las tres superficies coordenadas, se define un punto de coordenadas $(r_{0}, \theta_{0}, \varphi_{0})$.
\end{frame}
{ % Start of local scope
\setbeamercolor{background canvas}{bg=black}
\begin{frame}
\frametitle{Intersección de las superficies}
\begin{figure}
   \centering
   \includegraphics[width=0.95\linewidth]{Imagenes/superficies_esfericas_04.png}
   \caption{}
   \label{fig:superficies_esfericas_04}
\end{figure}
\end{frame}
}

\subsection{Coordenadas curvilíneas ortogonales}

\begin{frame}
\frametitle{Generalización}
Una generalización directa permite pensar en tres familias de superficies, en general curvas, que en cada punto del espacio se intersectan en ángulo recto.
\end{frame}
\begin{frame}
\frametitle{Generalización}
Estas superficies pueden describirse mediante las ecuaciones:
\begin{align*}
u_{1} &= f_{1}(x, y, z) \\
u_{2} &= f_{2}(x, y, z) \\
u_{3} &= f_{3}(x, y, z)
\end{align*}
\end{frame}
\begin{frame}
\frametitle{Generalización}
Equivalentemente:
\begin{align*}
x &= x(u_{i}) \\
y &= y(u_{i}) \\
z &= z(u_{i})
\end{align*}
\pause
Estas ecuaciones son a la vez las reglas de transformación entre \textocolor{ao}{coordenadas cartesianas} y las \textocolor{byzantine}{coordenadas curvilíneas ortogonales}.
\end{frame}
\begin{frame}
\frametitle{Generalización}
Las superficies $u_{1} = \mbox{cte.}$ y $u_{2} = \mbox{cte.}$ se intersectan en una curva a lo largo de la cual solo $u_{3}$ varía, esta curva define a la coordenada $u_{3}$, como se ve en la figura \ref{fig:figura_Sistema_Curvilineo_Ortogonal}.
\end{frame}
\begin{frame}
\begin{figure}[h!]
   \centering
   \includestandalone[scale=0.8]{Figuras/Sistema_Curvilineo_Ortogonal}
   \caption{Sistema curvilíneo ortogonal. Los vectores unitarios $(\vu{e}_{1}, \vu{e}_{2}, \vu{e}_{3})$ son tangentes a las superficies.}
   \label{fig:figura_Sistema_Curvilineo_Ortogonal}
\end{figure}
\end{frame}
\begin{frame}
   \frametitle{Generalización}
Análogamente las superficies $u_{1} = \mbox{cte.}$ y $u_{3} = \mbox{cte.}$ generan la curva $u_{2}$; y las superficies $u_{2} = \mbox{cte.}$ y $u_{3} = \mbox{cte.}$ generan la curva $u_{1}$.
\end{frame}
\begin{frame}
\frametitle{Generalización}   
La intersección de las tres superficies genera un punto cuyas coordenadas son $(u_{1}, u_{2}, u_{3})$.
\\
\bigskip
\pause
Dado un punto $(x, y, z)$ es posible asignarle unívocamente un conjunto $(u_{1}, u_{2}, u_{3})$ de coordenadas curvilíneas.
\end{frame}
\begin{frame}
\frametitle{Características del sistema coordenado}
El sistema de coordenadas curvilíneas construido con estas superficies, tiene las siguientes características:
\pause
\setbeamercolor{item projected}{bg=lava,fg=white}
\setbeamertemplate{enumerate items}{%
\usebeamercolor[bg]{item projected}%
\raisebox{1.5pt}{\colorbox{bg}{\color{fg}\footnotesize\insertenumlabel}}%
}
\begin{enumerate}[<+->]
\item Los ejes coordenados son en general curvas que se intersectan en ángulo recto, de modo que los vectores unitarios $\left\{ \vu{e}_{i} \right\}$, que son tangentes a las curvas, generan una base ortonormal tridimensional.
\seti
\end{enumerate}
\end{frame}
\begin{frame}
\frametitle{Característica del sistema coordenado}
En el curso nos enfocaremos en el estudio de este tipo de sistemas: coordenados ortogonales, los sistemas coordenados no ortogonales no los revisaremos.
\end{frame}
\begin{frame}
\frametitle{Características del sistema coordenado}
\setbeamercolor{item projected}{bg=lava,fg=white}
\setbeamertemplate{enumerate items}{%
\usebeamercolor[bg]{item projected}%
\raisebox{1.5pt}{\colorbox{bg}{\color{fg}\footnotesize\insertenumlabel}}%
}
\begin{enumerate}[<+->]
\conti
\item La orientación de la base $\left\{ \vu{e}_{i} \right\}$ puede cambiar de punto a punto, preservándose su ortonormalidad.
\item El significado físico de los diferenciales de las coordenadas \emph{no es necesariamente una longitud}. En coordenadas esféricas, tenemos una longitud y dos ángulos.
\end{enumerate}
\end{frame}

\subsection*{Campo escalar}

\begin{frame}
\frametitle{Campo escalar}
Un \textocolor{blue-violet}{campo escalar} se define dando un valor numérico en cada punto del espacio.
\\
\bigskip
\pause
El valor de una cantidad escalar en un punto definido del espacio es independiente
del sistema de coordenadas que se utilice.
\end{frame}
\begin{frame}
\frametitle{Campo escalar}
Por lo que decimos que si $(x, y, z)$, $(\rho, \phi, z)$, $(r, \theta, \phi)$ denotan el mismo punto del espacio físico, el valor que en ese punto tome, por ejemplo, la presión atmosférica es el mismo.
\end{frame}

\subsection*{Campo vectorial}

\begin{frame}
\frametitle{Campo vectorial}
Los \textocolor{brown(web)}{campos vectoriales} se definen dando en cada punto del espacio el valor de tres cantidades, conocidas como las \emph{componentes vectoriales}.
\end{frame}
\begin{frame}
\frametitle{Campo vectorial}
Aunque los vectores unitarios y los valores de cada componente sean diferentes en cada sistema de coordenadas, es sin embargo cierto que el vector $\vb{A}$ no cambia cuando cambiamos de sistema coordenado.
\end{frame}
\begin{frame}
\frametitle{Campo vectorial}
Por lo que podemos decir que si los vectores unitarios $\vu{e}_{i}$ y $\vu{e}_{i}^{\prime}$, y las componentes $A_{i}$ y $A_{i}^{\prime}$ en dos sistemas coordenados $S$ y $S^{\prime}$, entonces:
\pause
\begin{align*}
\vb{A} = \nsum_{i=1}^{3} A_{i} \, \vu{e}_{i} = \nsum_{i=1}^{3} A_{i}^{\prime} \, \vu{e}_{i}^{\prime}
\end{align*}
\end{frame}
\begin{frame}
\frametitle{Vectores invariantes}
Esto significa que un vector es \textocolor{burgundy}{invariante} bajo transformaciones de coordenadas.
\\
\bigskip
\pause
Los campos escalares son también invariantes bajo transformaciones de coordenadas. En la \emph{teoría de transformación} se revisa estos temas.
\end{frame}

\subsection*{Ecuaciones matemáticas}

\begin{frame}
\frametitle{Ecuaciones matemáticas}
De esta manera es posible escribir ecuaciones cuya forma matemática es la misma en todos los sistemas de coordenadas en el espacio 3D euclidiano.
\end{frame}
\begin{frame}
\frametitle{Ecuaciones matemáticas}
Por ejemplo, la ecuación de onda:
\begin{align*}
\laplacian{\psi} - \dfrac{1}{v^{2}} \pdv[2]{\psi}{t} = 0
\end{align*}
es válida en todos los sistemas coordenados, es decir, es invariante bajo transformación de coordenadas.
\end{frame}

\section{Teoría de transformación}
\frame{\frametitle{Temas a revisar} \tableofcontents[currentsection, hideothersubsections]}
\subsection{Construcción}

\begin{frame}
\frametitle{Construcción}
En el espacio euclidiano es siempre posible construir un sistema coordenado cartesiano que se extienda indefinidamente.
\\
\bigskip
\pause
A partir de él podemos generar múltiples sistemas coordenados, mediante el uso de las superficies $u_{i} = f(x, y, z)$
\end{frame}
\begin{frame}
\frametitle{Construcción}
Dado que de manera recíproca, $(x, y, z)$ son funciones de $u_{i}$, es decir:
\begin{align*}
x &= x(u_{i}) \\
y &= y(u_{i}) \\
z &= z(u_{i})
\end{align*}
\end{frame}
\begin{frame}
\frametitle{Construcción}
Entonces podemos escribir:
\pause
% se usa eqnarray para que funcione la pausa
\begin{eqnarray*}
\dd{x} = \pdv{x}{u_{1}} \dd{u_{1}} + \pdv{x}{u_{2}} \dd{u_{2}} + \pdv{x}{u_{3}} \dd{u_{3}} \\[0.5em]
\pause
\dd{y} = \pdv{y}{u_{1}} \dd{u_{1}} + \pdv{y}{u_{2}} \dd{u_{2}} + \pdv{y}{u_{3}} \dd{u_{3}} \\[0.5em]
\pause
\dd{z} = \pdv{z}{u_{1}} \dd{u_{1}} + \pdv{z}{u_{2}} \dd{u_{2}} + \pdv{z}{u_{3}} \dd{u_{3}} \\
\end{eqnarray*}
\end{frame}
\begin{frame}
\frametitle{Ecuación vectorial}
Estas tres ecuaciones son las componentes de la ecuación vectorial:
\pause
\begin{eqnarray}
\begin{aligned}
\dd{\vb{r}} &= \pdv{\vb{r}}{u_{1}} \dd{u_{1}} + \pdv{\vb{r}}{u_{2}} \dd{u_{2}} + \pdv{\vb{r}}{u_{3}} \dd{u_{3}} = \\[0.5em] \pause
&= \nsum_{i=1}^{3} \pdv{\vb{r}}{u_{i}} \dd{u_{i}}
\end{aligned}
\label{eq:ecuacion_01_01}
\end{eqnarray}
\end{frame}
\begin{frame}
\frametitle{Vector no unitario}
En general, el factor $\pdv*{\vb{r}}{u_{i}}$ en la ec. (\ref{eq:ecuacion_01_01}) es un vector \emph{no unitario} que toma en cuenta la variación de $\vb{r}$ solo en la dirección de $u_{i}$, y es por tanto, tangente a la curva coordenada $u_{i}$.
\end{frame}
\begin{frame}
\frametitle{Base normalizada}
Con el fin de introducir una \emph{base normalizada} $\vu{e}_{i}$, es decir, un conjunto de vectores unitarios, escribimos:
\pause
\begin{align}
\pdv{\vb{r}}{u_{i}} = h_{i} \, \vu{e}_{i}
\label{eq:ecuacion_01_02}
\end{align}
\pause
donde $\abs{\vb{e}_{i}} = 1$ y $h_{i}$ son funciones de $u_{i}$, que llamaremos \textocolor{cobalt}{factores de escala}.
\end{frame}
\begin{frame}
\frametitle{Diferencial de desplazamiento}
Se sigue entonces que el diferencial de desplazamiento es:
\pause
\begin{align*}
\dd{\vb{r}} = \nsum_{i=1}^{3} h_{i} \, \vu{e}_{i} \dd{u_{i}}
\end{align*}
\end{frame}
\begin{frame}
\frametitle{Bases ortonormales}
Como hicimos la aclaración de trabajar con bases \emph{ortonormales} (perpendiculares y unitarias), podemos escribir:
\pause
\begin{align}
\vu{e}_{i} \cdot \vu{e}_{j} = \delta_{ij}
\label{eq:ecuacion_01_03}
\end{align}
donde $\delta_{ij}$ es la delta de Kronecker:
\pause
\begin{align*}
\delta_{ij} = 
\begin{cases}
0 & \mbox{si } i \neq j \\
1 & \mbox{si } i = j
\end{cases}
\end{align*}
\end{frame}
\begin{frame}
\frametitle{Bases ortonormales}
Entonces tendremos que:
\pause
\begin{align*}
\vu{e}_{1} \cdot \vu{e}_{2} = \vu{e}_{2} \cdot \vu{e}_{3} = \vu{e}_{3} \cdot \vu{e}_{1} = 0  
\end{align*}
y además:
\pause
\begin{align*}
\vu{e}_{1} \cdot \vu{e}_{1} = \abs{\vu{e}_{1}}^{2} = \abs{\vu{e}_{2}}^{2} = \abs{\vu{e}_{3}}^{2} = 1
\end{align*}
\end{frame}
\begin{frame}
\frametitle{Definición del factor de escala}
Por la ecuación (\ref{eq:ecuacion_01_02}), los factores de escala se definen por:
\pause
\begin{align}
\abs{\pdv{\vb{r}}{u_{i}}} = h_{i}
\label{eq:ecuacion_01_04}
\end{align}
por lo que es fácil calcular los factores de escala.
\end{frame}
\begin{frame}
\frametitle{Vectores unitarios}
En consecuencia los vectores unitarios en coordenadas curvilíneas (ec. \ref{eq:ecuacion_01_02}) se escriben como:
\pause
\begin{align}
\vu{e}_{i} = \dfrac{1}{h_{i}} \, \pdv{\vb{r}}{u_{i}}
\label{eq:ecuacion_01_05}
\end{align}
\end{frame}

\subsection{Cambio de coordenadas}

\begin{frame}
\frametitle{Cambio a coordenadas esféricas}
Como ejemplo haremos el cambio de coordenadas cartesianas a coordenadas esféricas.
\begin{align*}
(x, y, z) \longrightarrow (\rho, \theta, \varphi)
\end{align*}
\end{frame}
\begin{frame}
\frametitle{Cambio a coordenadas esféricas}
Un punto $P$ puede localizarse mediante las coordenadas cartesianas $(x, y, z)$ y también mediante coordenadas esféricas $(\rho, \theta, \varphi)$, donde:
\pause
\begin{minipage}{0.4\linewidth}
\begin{align*}
-\infty \le x \le \infty \\
-\infty \le y \le \infty \\
-\infty \le z \le \infty
\end{align*}
\end{minipage}
\hspace{0.3cm}
\begin{minipage}{0.4\linewidth}
\begin{align*}
\rho \geq 0 \\
0 \le \theta \le \pi \\
0 \le \varphi \le 2 \, \pi
\end{align*}
\end{minipage}
\end{frame}
\begin{frame}
\frametitle{Identificación de coordenadas}
Donde:
\pause
\begin{itemize}
\item[\ding{212}] $\rho = \abs{\vb{r}}$ es la coordenada radial.
\item[\ding{212}] $\theta$ es la coordenada polar.
\item[\ding{212}] $\varphi$ es la coordenada azimutal.
\end{itemize}
\end{frame}
\begin{frame}
\frametitle{Reglas de transformación}
Las reglas de transformación entre las coordenadas cartesianas y esféricas son:
\pause
De $(x, y, z) \rightarrow (\rho, \theta, \varphi)$:
\begin{align*}
x &= \rho \, \sin \theta \, \cos \varphi \\
y &= \rho \, \sin \theta \, \sin \varphi \\
z &= \rho \, \cos \theta
\end{align*}
\end{frame}
\begin{frame}
\frametitle{Reglas de transformación}
De $(\rho, \theta, \varphi) \rightarrow (x, y, z)$:
\begin{align*}
\rho &= \sqrt{x^{2} + y^{2} + z^{2}} \\[0.75em]
\theta &= \cos^{-1} \left( \dfrac{z}{\sqrt{x^{2} + y^{2} + z^{2}}} \right) \\[0.75em]
\varphi &= \tan^{-1} \left( \dfrac{y}{x} \right)
\end{align*}
\end{frame}
\begin{frame}
\frametitle{Cambio a coordenadas esféricas}
Entonces el vector de posición es:
\pause
\begin{align}
\begin{aligned}
\vb{r} &= \vu{i} \, x + \vu{j} \, y + \vu{k} \, z \\
&= \vu{i} \, \rho \, \sin \theta \, \cos \varphi + \vu{j} \, \rho \, \sin \theta \, \sin \varphi + \vu{k} \, \rho \cos \theta 
\end{aligned}
\label{eq:ecuacion_01_06}
\end{align}
\pause
Por lo que podemos calcular las derivadas $\pdv*{\vb{r}}{u_{i}}$.
\end{frame}
\begin{frame}
\frametitle{Calculando el factor $\pdv*{\vb{r}}{r}$}
Entonces tenemos que:
\pause
\begin{eqnarray*}
\begin{aligned}
\pdv{\vb{r}}{\rho} &= \pdv{\rho} \left[ \vu{i} \, \rho \, \sin \theta \, \cos \varphi \right] + \pdv{\rho} \left[ \vu{j} \, \rho \, \sin \theta \, \sin \varphi \right] + \\
&+ \pdv{\rho} \left[ \vu{k} \, \rho \, \cos \theta \right] = \\[0.5em] \pause
&= \vu{i} \sin \theta \, \cos \varphi + \vu{j} \, \sin \theta \, \sin \varphi + \vu{k} \, \cos \theta
\end{aligned}
\end{eqnarray*}
\end{frame}
\begin{frame}
\frametitle{Calculando la norma}
\begin{eqnarray*}
\begin{aligned}
\abs{\pdv{\vb{r}}{\rho}} &= \left[ \left( \vu{i} \sin \theta \, \cos \varphi \right)^{2} + \left( \vu{j} \, \sin \theta \, \sin \varphi \right)^{2} + \right. \\[0.35em]
&+ \left. \left( \vu{k} \, \cos \theta \right)^{2} \right]^{1/2} = \\[0.35em] \pause
&= \sqrt{\sin^{2} \theta \, \cos^{2} \varphi + \sin^{2} \theta \, \sin^{2} \varphi + \cos^{2} \theta} = \\[0.35em] \pause
&= \sqrt{\sin^{2} \theta \left( \cos^{2} \varphi + \sin^{2} \varphi \right) + \cos^{2} \theta} = \\[0.35em] \pause
&= \sqrt{\sin^{2} \theta + \cos^{2} \theta} = \\[0.35em] \pause
&= 1
\end{aligned}
\end{eqnarray*}
\end{frame}
\begin{frame}
\frametitle{Primer factor de escala}
Entonces por la ec. (\ref{eq:ecuacion_01_04}):
\pause
\begin{align*}
\abs{\pdv{\vb{r}}{u_{i}}} = h_{i}
\end{align*}
Tendremos el primer factor de escala:
\pause
\begin{align*}
h_{1} = h_{\rho} = 1
\end{align*}
\end{frame}
\begin{frame}
\frametitle{Siguiente factor $\pdv*{\vb{r}}{\theta}$}
Para calcular el siguiente factor de escala, ahora consideramos la derivada parcial:
\pause
\begin{eqnarray*}
\begin{aligned}
\pdv{\vb{r}}{\theta} &=\pdv{\theta} \left[ \vu{i} \, \rho \, \sin \theta \, \cos \varphi \right] + \pdv{\theta} \left[ \vu{j} \, \rho \, \sin \theta \, \sin \varphi \right] + \\
&+ \pdv{\theta} \left[ \vu{k} \, \rho \, \cos \theta \right] = \\[0.5em] \pause
&= \vu{i} \, \rho \, \cos \theta \, \cos \varphi + \vu{j} \, \rho \, \cos \theta \, \sin \varphi - \vu{k} \, \rho \, \sin \theta
\end{aligned}
\end{eqnarray*}
\end{frame}
\begin{frame}
\frametitle{Calculando la norma}
\begin{eqnarray*}
\begin{aligned}
\abs{\pdv{\vb{r}}{\theta}} &= \sqrt{\rho^{2} \left[ \cos^{2} \theta \, \cos^{2} \varphi + \cos^{2} \theta \, \sin^{2} \varphi + \sin^{2} \theta \right]} \\[0.5em] \pause
&= \sqrt{\rho^{2} \left[ \cos^{2} \theta \left(\cos^{2} \varphi + \sin^{2} \varphi \right) + \sin^{2} \theta \right]} \\[0.5em] \pause
&= \sqrt{\rho^{2} (\cos^{2} \theta + \sin^{2} \theta)} \\[0.5em] \pause
&= \rho
\end{aligned}
\end{eqnarray*}
\pause
Así: $h_{2} = h_{\theta} = \rho$
\end{frame}
\begin{frame}
\frametitle{Siguiente factor $\pdv*{\vb{r}}{\varphi}$}
Nos faltaría calcular el último factor de escala, que corresponde a la derivada parcial:
\pause
\begin{eqnarray*}
\begin{aligned}
\pdv{\vb{r}}{\varphi} &= \pdv{\varphi} \left[ \vu{i} \, \rho \, \sin \theta \, \cos \varphi \right] + \pdv{\varphi} \left[ \vu{j} \, \rho \, \sin \theta \, \sin \varphi \right] + \\
&+ \pdv{\varphi} \left[ \vu{k} \, \rho \, \cos \theta \right] = \\[0.5em] \pause
&= - \vu{i} \, \rho \, \sin \theta \, \sin \varphi + \vu{j} \, \rho \, \sin \theta \, \cos \varphi
\end{aligned}
\end{eqnarray*}
\end{frame}
\begin{frame}
\frametitle{Obteniendo la norma}
\begin{eqnarray*}
\begin{aligned}
\abs{\pdv{\vb{r}}{\varphi}} &= \sqrt{\rho^{2} \left( \sin^{2} \theta \, \sin^{2} \varphi + \sin^{2} \theta \, \cos^{2} \varphi \right)} \\[0.5em] \pause
&= \sqrt{\rho^{2} \left[ \sin^{2} \theta \left( \sin^{2} \varphi + \cos^{2} \varphi \right) \right]} \\[0.5em] \pause
&= \sqrt{\rho^{2} \, \sin^{2} \theta} \\[0.5em] \pause
&= \rho \, \sin \theta
\end{aligned}
\end{eqnarray*}
\pause
Así: $h_{3} = h_{\varphi} = \rho \, \sin \theta$
\end{frame}
\begin{frame}
\frametitle{Factores de escala}
Entonces los factores de escala $h_{i}$ para el sistema coordenado esférico son:
\pause
\begin{align}
\begin{aligned}
h_{\rho} &= 1 \\[0.5em]
h_{\theta} &= \rho \\[0.5em]
h_{\varphi} &= \rho \, \sin \theta
\end{aligned}
\label{eq:ecuacion_01_07}
\end{align}
\end{frame}
\begin{frame}
\frametitle{Vectores unitarios}
Al reemplazar los factores de escala en la ec. (\ref{eq:ecuacion_01_05}), los vectores unitarios en coordenadas esféricas pueden expresarse en términos de coordenadas cartesianas, como:
\pause
\begin{align}
\begin{aligned}
\vu{e}_{\rho} &= \vu{i} \, \sin \theta \, \cos \varphi + \vu{j} \, \sin \theta \, \sin \varphi + \vu{k} \, \cos \theta \\[0.25em]
\vu{e}_{\theta} &= \vu{i} \, \cos \theta \, \cos \varphi + \vu{j} \, \cos \theta \, \sin \varphi - \vu{k} \, \sin \theta \\[0.25em]
\vu{e}_{\varphi} &= - \vu{i} \, \sin \varphi + \vu{j} \, \cos \varphi
\end{aligned}
\label{eq:ecuacion_01_08}
\end{align}
Donde $(\vu{e}_{1}, \vu{e}_{2}, \vu{e}_{3}) = (\vu{e}_{\rho}, \vu{e}_{\theta}, \vu{e}_{\varphi})$
\end{frame}
\begin{frame}
\frametitle{Ecuaciones inversas}
Las ecs. (\ref{eq:ecuacion_01_08}) pueden invertirse algebraicamente para expresar los vectores unitarios $\vu{i}, \vu{j}, \vu{k}$ en términos de $\vu{e}_{\rho}, \vu{e}_{\theta}, \vu{e}_{\varphi}$, tal que:
\end{frame}
\begin{frame}
\frametitle{Ecuaciones inversas}
Las ecuaciones son:
\pause
\begin{align*}
\vu{i} &= \vu{e}_{\rho} \, \sin \theta \, \cos \varphi + \vu{e}_{\theta} \, \cos \theta \, \cos \varphi - \vu{e}_{\varphi} \, \sin \varphi \\[0.5em]
\vu{j} &= \vu{e}_{\rho} \, \sin \theta \, \sin \varphi + \vu{e}_{\theta} \, \cos \theta \, \sin \varphi + \vu{e}_{\varphi} \, \cos \varphi \\[0.5em]
\vu{k} &= \vu{e}_{\rho} \, \cos \theta - \vu{e}_{\theta} \, \sin \theta
\end{align*}
\end{frame}
\begin{frame}
\frametitle{Representación con matrices}
En forma matricial, tenemos que:
\pause
\begin{align*}
\vu{e} = \vb{A} \, \vu{\epsilon} \hspace{1.5cm} \vu{\epsilon} = \tilde{\vb{A}} \, \vu{e}
\end{align*}
donde:
\end{frame}
\begin{frame}
\frametitle{Representación con matrices}
\begin{align*}
\vu{e} = \mqty(
\vu{e}_{1} \\
\vu{e}_{2} \\
\vu{e}_{3})
= \mqty(
\vu{e}_{\rho} \\
\vu{e}_{\theta} \\
\vu{e}_{\phi}), \hspace{1.5cm}
\vu{\epsilon} = \mqty(
\vu{\epsilon}_{1} \\
\vu{\epsilon}_{2} \\
\vu{\epsilon}_{3}) = \mqty(
\vu{i} \\
\vu{j} \\
\vu{k})
\end{align*}
\end{frame}
\begin{frame}
\frametitle{Representación con matrices}
Mientras que la matriz $\vb{A}$ es:
\pause
\begin{align*}
\vb{A} = \mqty(
\sin \theta \, \cos \phi & \sin \theta \, \sin \phi & \cos \theta \\
\cos \theta \, \cos \phi & \cos \theta \, \sin \phi & - \sin \theta \\
- \sin \phi & \cos \phi & 0
)
\end{align*}
\end{frame}
\begin{frame}
\frametitle{Sobre la matriz $\vb{A}$}
Es cierto que $\vb{A} \, \tilde{\vb{A}} = \vb{I}$, \pause por lo que:
\pause
\begin{itemize}
\item[\ding{212}] La matriz $\vb{A}$ es ortogonal.
\item[\ding{212}] Además $\abs{\vb{A}} = 1$  
\end{itemize}
\end{frame}

% \begin{frame}
% \frametitle{Ejercicio a cuenta}
% Considera la transformación de coordenadas:
% \begin{align*}
% x &= 2 \, u \, v \\[0.5em]
% y &= u^{2} + v^{2} \\[0.5em]
% z &= w
% \end{align*}
% Demuestra que el nuevo sistema de coordenadas \emph{no} es ortogonal.
% \end{frame}

\subsection{Derivadas de vectores unitarios}

\begin{frame}
\frametitle{Derivadas parciales vectores unitarios}
En aplicaciones de análisis vectorial se requiere utilizar las derivadas parciales $\pdv*{\vu{e}_{i}}{u_{j}}$
\\
\bigskip
\pause
Partiendo de:
\begin{align*}
\dd{\vb{r}} = \nsum_{i} h_{i} \, \vu{e}_{i} \dd{u_{i}}
\end{align*}
\end{frame}
\begin{frame}
\frametitle{Derivadas parciales de vectores unitarios}
Entonces podemos escribir:
\pause
\begin{align*}
\pdv{\vb{r}}{u_{j}} = h_{j} \, \vu{e}_{j} \hspace{1cm} \pdv{\vb{r}}{u_{i}} = h_{i} \, \vu{e}_{i}
\end{align*}
\pause
Que al calcular la segunda derivada mixta:
\begin{align*}
\pdv[2]{\vb{r}}{u_{i}}{u_{j}} = \pdv{u_{i}} (h_{j} \, \vu{e}_{j}) \hspace{1cm} \pdv[2]{\vb{r}}{u_{j}}{u_{i}} = \pdv{u_{j}} (h_{i} \, \vu{e}_{i})
\end{align*}
\end{frame}
\begin{frame}
\frametitle{Derivadas parciales de vectores unitarios}
Restando estas ecuaciones y teniendo en cuenta que $\displaystyle \pdv{\vu{e}_{i}}{u_{j}}$ es paralelo al vector $\vu{e}_{j}$ y $\displaystyle \pdv{\vu{e}_{j}}{u_{i}}$ es paralelo a $\vu{e}_{i}$, se obtiene que:
\pause
\begin{align*}
\pdv{\vu{e}_{i}}{u_{j}} = \dfrac{\vu{e}_{j}}{h_{i}} \, \pdv{h_{j}}{u_{i}}
\end{align*}
que es válida para $i \neq j$
\end{frame}

\subsection*{Símbolo de Levi-Civita}

\begin{frame}
\frametitle{Símbolo de Levi-Civita}
Haremos una pausa, para revisar que el símbolo de Levi-Civita se define como:
\pause
\fontsize{12}{12}\selectfont
\begin{align*}
\epsilon_{ijk} = \begin{cases}
+1 & \mbox{si } (i, j, k) \mbox{ es } (1, 2, 3), (2, 3, 1), (3, 1, 2) \\[0.5em]
-1 & \mbox{si } (i, j, k) \mbox{ es } (3, 2, 1), (1, 3, 2), (2, 1, 3) \\[0.5em]
0 & \mbox{de otro modo } i = j, j = k, k = i 
\end{cases}
\end{align*}
\end{frame}
\begin{frame}
\frametitle{Símbolo de Levi-Civita}
Una forma algebraica bastante simple que contiene todas las propiedades del símbolo de Levi-Civita es:
\pause
\begin{align*}
\epsilon_{ijk} = \dfrac{1}{2} (i - j) (j - k) (k - i)
\end{align*}
\end{frame}
\begin{frame}
\frametitle{Regresamos al tema}
Sabemos que:
\pause
\begin{align*}
\vu{e}_{i} = \dfrac{1}{2} \nsum_{jk} \epsilon_{ijk} \, \vu{e}_{j} \cp \vu{e}_{k}
\end{align*}
por lo que al derivar primero, luego ocupar la última ecuación y haciendo álgebra, llegamos a:
\pause
\begin{align*}
\pdv{\vu{e}_{i}}{u_{i}} = \nsum_{jkl} \epsilon_{ijk} \, \epsilon_{ilj} \, \dfrac{\vu{e}_{l}}{h_{k}} \, \pdv{h_{i}}{u_{k}}
\end{align*}
\end{frame}
\begin{frame}
\frametitle{Resultado importante}
Utilizando la siguiente propiedad:
\pause
\begin{align*}
\nsum_{k=1}^{3} \epsilon_{ijk} \, \epsilon_{lmk} = \mdet{
\delta_{il} & \delta_{im} \\
\delta_{jl} & \delta_{jm} }
\end{align*}
Podemos concluir con el siguiente resultado:
\end{frame}
\begin{frame}
\frametitle{Resultado importante}
Conclusión de la derivación parcial de vectores unitarios:
\pause
\begin{align*}
\pdv{\vu{e}_{i}}{u_{i}} = - \nsum_{k \neq i} \dfrac{\vu{e}_{k}}{h_{k}} \, \pdv{h_{i}}{u_k}
\end{align*}
\end{frame}

\section{Ejercicios a cuenta}
\frame{\frametitle{Temas a revisar} \tableofcontents[currentsection, hideothersubsections]}
\subsection{Tipos de ejercicios}

\begin{frame}
\frametitle{Ejercicios de práctica}
Con la finalidad de repasar lo que se vaya viendo en las clases, los ejercicios a cuenta, formarán parte de las tareas.
\end{frame}
\begin{frame}
\frametitle{Ejercicios de práctica}
Tendrán oportunidad de hacer consultas, preguntas, etc. para que cuenten con los elementos necesarios y presentar una solución completa.
\end{frame}
\begin{frame}
\frametitle{Ejercicio a cuenta 1}
Considera la transformación de coordenadas:
\begin{align*}
x &= 2 \, u \, v \\[0.5em]
y &= u^{2} + v^{2} \\[0.5em]
z &= w
\end{align*}
Demuestra que el nuevo sistema de coordenadas \textocolor{red}{es no ortogonal}.
\end{frame}
\begin{frame}
\frametitle{Ejercicio a cuenta 2}
Considera la transformación de coordenadas:
\begin{align*}
x &= 2 \, u \, v \\[0.5em]
y &= u^{2} - v^{2} \\[0.5em]
z &= w
\end{align*}
Demuestra que el nuevo sistema de coordenadas \textocolor{ao}{es ortogonal}.
\end{frame}
\begin{frame}
\frametitle{Ejercicio a cuenta 3}
Escribe en coordenadas esféricas el siguiente vector:
\begin{align*}
\vb{A} = x \, y \, \vu{i} - x \, \vu{j} + 3 \, x \, \vu{k}
\end{align*}
adicionalmente expresa $A_{\rho}, A_{\theta}, A_{\phi}$ en términos de $\rho, \theta, \phi$.
\end{frame}
\begin{frame}
\frametitle{Ejercicio a cuenta 4}
Calcula las componentes rectangulares de la velocidad y aceleración de la partícula cuyo vector de posición está dado por:
\setbeamercolor{item projected}{bg=auburn,fg=white}
\setbeamertemplate{enumerate items}{%
\usebeamercolor[bg]{item projected}%
\raisebox{1.5pt}{\colorbox{bg}{\color{fg}\footnotesize\insertenumlabel}}%
}
\begin{enumerate}[<+->]
\item $\vb{r} = A \, \cos n \, \omega \,  t \, \vu{i} + B \, \sin m \, \omega \, t \, \vu{j}$, con $n, m$ enteros.
\item $\vb{r} = 3 \, t \, \vu{i} - 4 \, t \, \vu{j} + \left( t^{2} + 3 \right) \, \vu{k}$
\end{enumerate}
\end{frame}
\begin{frame}
\frametitle{Ejercicio a cuenta 5}
Calcula las expresiones de las componentes polares de la velocidad y aceleración de las partículas cuyo vector bidimensional de posición está dado por:
\setbeamercolor{item projected}{bg=bulgarianrose,fg=white}
\setbeamertemplate{enumerate items}{%
\usebeamercolor[bg]{item projected}%
\raisebox{1.5pt}{\colorbox{bg}{\color{fg}\footnotesize\insertenumlabel}}%
}
\begin{enumerate}[<+->]
\item $r = \dfrac{5}{2 - \cos \phi}, \hspace{0.5cm} \phi = \omega \, t$
\item $r = \dfrac{a}{t}, \hspace{0.5cm} \phi = b \, t$
\end{enumerate}
\end{frame}
\section{Jacobianos y transformaciones}
\frame{\frametitle{Temas a revisar} \tableofcontents[currentsection, hideothersubsections]}

\subsection{El Jacobiano}

\begin{frame}
\frametitle{El Jacobiano}
En componentes, la ec. (\ref{eq:ecuacion_01_01}) se escribe como:
\pause
\begin{align}
\dd{x_{i}} = \nsum_{j} \pdv{x_{i}}{u_{j}} \dd{u_{j}} = \nsum_{j} J_{ij} \dd{u_{j}}
\label{eq:ecuacion_01_10}
\end{align}
Donde:
\begin{align}
J_{ij} = \pdv{x_{i}}{u_{j}}
\label{eq:ecuacion_01_11}
\end{align}
\end{frame}
\begin{frame}
\frametitle{El Jacobiano}
En forma matricial la ec. (\ref{eq:ecuacion_01_10}) se escribe como:
\pause
\begin{align}
\dd{x} = \vb{J} \dd{u}
\end{align}
\pause
donde $\dd{x}$ y $\dd{u}$ son los vectores columna:
\begin{align*}
\dd{x} = \begin{pmatrix}
\dd{x_{1}} \\
\dd{x_{2}} \\
\dd{x_{3}} \\
\end{pmatrix}
\hspace{1.5cm}
\dd{u} = \begin{pmatrix}
\dd{u_{1}} \\
\dd{u_{2}} \\
\dd{u_{3}} \\
\end{pmatrix}
\end{align*}
\end{frame}
\begin{frame}
\frametitle{El Jacobiano}
La matriz $\vb{J}$ es la matriz de transformación de los diferenciales de coordenadas, cuyos elementos son:
\pause
\begin{align*}
J_{ij} = \pdv{x_{i}}{u_{j}}
\end{align*}
\pause
El determinante $\abs{\vb{J}}$ se le conoce como el \emph{Jacobiano}, que debe de ser \textocolor{falured}{distinto de cero} para garantizar que la transformación sea invertible.
\end{frame}
\begin{frame}
\frametitle{Transformación vectores unitarios}
Con el vector:
\pause
\begin{align*}
\vb{r} = \vu{i} \, x + \vb{j} \, y + \vb{k} \,z
\end{align*}
\pause
La ec. \ref{eq:ecuacion_01_05} toma la forma:
\begin{align}
\vu{e}_{i} = \dfrac{1}{h_{i}} \left( \vu{i} \, \pdv{x}{u_{i}} + \vu{j} \, \pdv{y}{u_{i}} + \vu{k} \, \pdv{z}{u_{i}} \right)
\label{eq:ecuacion_01_12}
\end{align}
que es la regla de transformación de vectores unitarios.
\end{frame}
\begin{frame}
\frametitle{Transformación de vectores unitarios}
Introduciendo la notación $(\vu{\epsilon}_{1}, \vu{\epsilon}_{2}, \vu{\epsilon}_{3}) = (\vu{i}, \vu{j}, \vu{k})$, escribimos:
\pause
\begin{align*}
\vb{r} = \nsum_{j} \vu{\epsilon}_{j} \, x_{j}
\end{align*}
\pause
Tal que la ec. (\ref{eq:ecuacion_01_05}) también se escribe como:
\begin{align}
\vu{e}_{i} = \nsum_{j} \dfrac{\vu{e}_{j}}{h_{i}} \, \pdv{x_{j}}{u_{i}} = \nsum_{j} a_{ji} \, \vu{e}_{j}
\label{eq:ecuacion_01_13}
\end{align}
\end{frame}
\begin{frame}
\frametitle{Transformación de vectores unitarios}
Donde se ha definido la cantidad:
\pause
\begin{align}
a_{ji} = \dfrac{1}{h_{i}} \, \pdv{x_{j}}{u_{i}}
\label{eq:ecuacion_01_14}
\end{align}
\end{frame}
\begin{frame}
\frametitle{Transformación de vectores unitarios}
Introduciendo los vectores columna:
\pause
\begin{align*}
\vu{e} =
\begin{pmatrix}
\vu{e}_{1} \\
\vu{e}_{2} \\
\vu{e}_{3} \\
\end{pmatrix}
\hspace{1.5cm}
\vu{\epsilon} =
\begin{pmatrix}
\vu{\epsilon}_{1} \\
\vu{\epsilon}_{2} \\
\vu{\epsilon}_{3} \\
\end{pmatrix}
\end{align*}
\pause
Considerando los $a_{ij}$ como los elementos de la matriz $\vb{A}$ (los $a_{ji}$ son los elementos de la matriz transpuesta)
\end{frame}
\begin{frame}
\frametitle{Transformación de vectores unitarios}
Entonces podemos escribir la ec. (\ref{eq:ecuacion_01_13}) como:
\pause
\begin{align}
\vu{e} = \vb{A}^{\intercal} \, \epsilon
\end{align}
donde $\vb{A}^{\intercal}$ es la transpuesta de $\vb{A}$.
\end{frame}
\begin{frame}
\frametitle{Representación matricial del jacobiano}
De acuerdo a las ecs. (\ref{eq:ecuacion_01_11}) y (\ref{eq:ecuacion_01_14}) que:
\pause
\begin{align*}
J_{ji} = a_{ji} \, h_{i}
\end{align*}
\pause
Que en términos de matrices:
\begin{align}
\vb{J} = \vb{A} \, \vb{H}
\label{eq:ecuacion_01_16}
\end{align}
\end{frame}
\begin{frame}
\frametitle{Representación matricial del jacobiano}
Donde:
\pause
\begin{align}
\vb{H} = \begin{pmatrix}
h_{1} & 0 & 0 \\
0 & h_{2} & 0 \\
0 & 0 & h_{3} \\
\end{pmatrix}
\label{eq:ecuacion_01_17}
\end{align}
\end{frame}

\subsection*{Invarianza bajo trasformaciones}

\begin{frame}
\frametitle{Invarianza bajo transformaciones}
Un postulado básico de la teoría de transformación asegura la invarianza bajo transformación de coordenadas del elemento de línea $\dd\vb{r}$, y en general de cualquier vector, bajo transformación de coordenadas.
\\
\bigskip
\pause
Por lo que los módulos de los vectores, son también invariantes.
\end{frame}
\begin{frame}
\frametitle{Invarianza bajo transformaciones}
Es cierto que en coordenadas cartesianas:
\pause
\begin{align*}
\dd{l}^{2} = \nsum_{i} \dd{x_{i}} \dd{x_{i}}
\end{align*}
\pause
y en coordenadas curvilíneas:
\begin{align*}
\dd{l}^{2} &= \vb{r} \cdot \vb{r} = \nsum_{jk} h_{j} \, h_{k} \, \vu{e}_{j} \cdot \vu{e}_{k} \dd{u_{j}} \dd{u_{k}} = \\[0.5em]
&= \nsum_{jk} h_{j} \, h_{k} \dd{u_{j}} \dd{u_{k}} \delta_{jk}
\end{align*}
\end{frame}
\begin{frame}
\frametitle{Invarianza bajo transformaciones}
La invarianza de $\dd{l}^{2}$ asegura que su valor es el mismo en el sistema coordenado original y en el nuevo, es decir:
\end{frame}
\begin{frame}
\frametitle{Invarianza bajo transformaciones}
Al igualar el elemento de desplazamiento:
\pause
\begin{align*}
\nsum_{i} \dd{x_{i}} \dd{x_{i}} = \nsum_{jk} h_{j} \, h_{k} \dd{u_{j}} \dd{u_{k}} \delta_{jk}
\end{align*}
\pause
Por la ec. (\ref{eq:ecuacion_01_11}), se tiene:
\pause
\begin{align*}
\dd{x_{i}} = \nsum_{j} J_{ij} \dd{u_{j}}
\end{align*}
\end{frame}
\begin{frame}
\frametitle{Invarianza bajo transformaciones}
Se sigue con:
\pause
\begin{align*}
\nsum_{ijk} J_{ij} \, J_{ik} \, \dd{u_{j}} \dd{u_{k}} = \nsum_{jk} h_{j} \, h_{k} \dd{u_{j}} \dd{u_{k}} \delta_{jk}
\end{align*}
\pause
de donde
\begin{align*}
\nsum_{i} J_{ij} \, J_{ik} = h_{j}^{2} \, \delta_{jk}
\end{align*}
\end{frame}
\begin{frame}
\frametitle{Forma matricial}
Que en forma matricial se escribe como:
\pause
\begin{align*}
\vb{J}^{\intercal} \, \vb{J} = \vb{H}^{2}
\end{align*}
siendo la matriz $\vb{J}^{\intercal}$, la matriz transpuesta de $\vb{J}$.
\end{frame}
\begin{frame}
\frametitle{Forma matricial}
De las expresiones:
\pause
\begin{align*}
\vb{J}^{\intercal} \, \vb{J} = \vb{H}^{2} \hspace{0.5cm} \mbox{y} \hspace{0.5cm} \vb{J} = \vb{A} \, \vb{H}
\end{align*}
se tiene que
\begin{align}
\vb{A}^{\intercal} \, \vb{A} = \vb{I}
\label{eq:ecuacion_01_18}
\end{align}
\pause
De modo que la matriz $\vb{A}$ es \emph{ortogonal}: $\vb{A}^{\intercal} = \vb{A}^{-1}$.
\end{frame}
\begin{frame}
\frametitle{Resultado importante}
De la ec. (\ref{eq:ecuacion_01_18}) se sigue que $\abs{\vb{A}} = \pm 1$.
\\
\bigskip
\pause
Los tipos posibles de transformación son:
\setbeamercolor{item projected}{bg=ao,fg=bananayellow}
\setbeamertemplate{enumerate items}{%
\usebeamercolor[bg]{item projected}%
\raisebox{1.5pt}{\colorbox{bg}{\color{fg}\footnotesize\insertenumlabel}}%
}
\begin{enumerate}[<+->]
\item De un $S$ cartesiano a otro $S^{\prime}$ \emph{rotado, reflejado o invertido}.
\item De un $S$ cartesiano a uno \emph{curvilíneo}.
\end{enumerate}
\end{frame}
\begin{frame}
\frametitle{Matriz de transformación}
En el caso de una rotación, o del paso de coordenadas cartesianas a curvilíneas, puesto que la matriz de transformación ha de contener la identidad, entonces $\abs{\vb{A}} = +1$.
\\
\bigskip
\pause
Para el caso de la reflexión e inversión, se tiene $\abs{\vb{A}} = -1$.
\end{frame}
\begin{frame}
\frametitle{Otro resultado importante}
Veamos que:
\pause
\begin{eqnarray*}
\begin{aligned}
\abs{\vb{J}} &= \pdv{\vb{r}}{u_{1}} \cdot \pdv{\vb{r}}{u_{2}} \cp \pdv{\vb{r}}{u_{3}} = \\[0.5em] \pause
&= \left( h_{1} \, \vu{e}_{1} \right) \vdot \left( h_{2} \, \vu{e}_{2} \cp {h_{3} \, \vu{e}_{3}} \right) = \\[0.5em] \pause
&= h_{1} \, h_{2} \, h_{3} \, \left( \vu{e}_{1} \cdot \vu{e}_{2} \cp \vu{e}_{3} \right)
\end{aligned}
\end{eqnarray*}
es diferente de cero, ya que $\vu{e}_{1}, \vu{e}_{2}, \vu{e}_{3}$ son no coplanares.
\end{frame}
\begin{frame}
\frametitle{Otro resultado importante}
De hecho, puesto que:
\pause
\begin{align*}
\vu{e}_{1} \cdot \vu{e}_{2} \cp \vu{e}_{3} = 1
\end{align*}
\pause
Se sigue que:
\begin{align*}
\abs{\vb{J}} = h_{1} \, h_{2} \, h_{3}
\end{align*}
\end{frame}
\begin{frame}
\frametitle{Listos para el siguiente paso}
En la siguiente presentación abordaremos la construcción de los elementos de línea, superficie y volumen en los sistemas coordenados curvilíneos.
\end{frame}
\end{document}