\documentclass[12pt]{article}
\usepackage[left=0.25cm,top=1cm,right=0.25cm,bottom=1cm]{geometry}
%\usepackage{geometry}
\textwidth = 20cm
\hoffset = -1cm
\usepackage[utf8]{inputenc}
\usepackage[spanish,es-tabla]{babel}
\usepackage[autostyle,spanish=mexican]{csquotes}
\usepackage{amsmath}
\usepackage{nccmath}
\usepackage{amsthm}
\usepackage{amssymb}
\usepackage{graphicx}
\usepackage{physics}
\usepackage{color}
\usepackage{float}
\usepackage{multicol}
%\usepackage{milista}
\usepackage{enumitem}
\usepackage{anyfontsize}
\usepackage{anysize}
\usepackage{enumitem}
\usepackage{capt-of}
\usepackage{bm}
\usepackage{relsize}
\newlist{milista}{enumerate}{2}
\setlist[milista,1]{label=\arabic*)}
\setlist[milista,2]{label=\arabic{milistai}.\arabic*)}
\spanishdecimal{.}
\renewcommand{\baselinestretch}{1.5} 
\title{Problemas para la Tarea Examen del Tema 1\\ \large{Matemáticas Avanzadas de la Física}\vspace{-3ex}}
\date{ }
\begin{document}
\vspace{-4cm}
%\renewcommand\theenumii{\arabic{theenumii.enumii}}
\renewcommand\labelenumii{\theenumi.{\arabic{enumii}}}
\maketitle
\fontsize{14}{14}\selectfont
Para las respuestas de la tarea, te pedimos sea lo más claro posible, presentando de manera organizada tu solución. La calificación de esta tarea está en función del número de ejercicios que entregues, cuentas con el suficiente tiempo para resolverla y hacer un buen esfuerzo. Recuerda que se debe de entregar el total de los problemas, en caso contrario, sólo se  revisarán las soluciones, pero no se tomarán en cuenta.
\begin{milista}
\item Una carga $q$ se está moviendo a velocidad constante $v$ sobre el eje positivo $x$. Otras dos cargas $-q$ y $2 \: q$ se mueven a velocidad constante $v$ y $2 \: v$ a lo largo del eje positivo $y$ y el eje negativo $z$, respectivamente. Suponemos que en $t = 0$, la carga $q$ está en el origen, $-q$ está en $(0, a, 0)$ y $2 \: q$ está en $(0, 0, a)$.
\begin{milista}
\item Calcula las componentes cartesianas del campo magnético en el punto $(x, y, z)$ para $t > 0$.
\item Calcula las componentes cilíndricas del campo magnético en el punto $(\rho, \varphi, z)$ para $t > 0$.
\item Calcula las componentes esféricas del campo magnético en el punto $(r, \theta, \varphi)$ para $t > 0$.
\end{milista}
\item Para el siguiente sistema de coordenadas
\begin{align*}
x &= \sqrt{(\xi^{2} -1) (1 - \eta^{2}}) \; \cos \varphi \\
y &= \sqrt{(\xi^{2} -1) (1 - \eta^{2}}) \; \sin \varphi \\
z &= \; \xi \; \eta
\end{align*}
\begin{milista}
\item Estudiar las curvas $\varphi =$ constante, $\xi =$ constante, y  $\eta =$ constante.
\item Determine los nuevos vectores base en términos de los vectores canónicos $\vu{i}$, $\vu{j}$, $\vu{k}$.
\item Calcule los factores de escala y muestre que el sistema es ortogonal, encuentre la matriz métrica de este sistema.
\item Encuentre la expresión para la velocidad y la energía cinética.
\item Encuentre las aceleraciones.
\item Escriba los operadores gradiente, divergencia, rotacional y laplaciano para este sistema de coordenadas.
\end{milista}
\item Dado el vector $\vb{A} = (x^{2} - y^{2}) \: \vu{i} + 2 \: x \: y \: \vu{j}$
\begin{milista}
\item Calcula $\vb{\nabla} \cp \vb{A}$.
\item Evalúa $\mathlarger{\iint} (\vb{\nabla} \cp \vu{A}) \vdot \dd{\sigma} $ en un rectángulo en el plano $x-y$ cercado por las líneas $x=0$, $x=a$, $y=0$, $y=b$.
\item Evalúa $\mathlarger{\oint} \vb{A} \vdot \dd{\vb{r}}$ alrededor de la frontera del rectángulo y verifica el teorema de Stokes para este ejemplo.
\end{milista}
\item Partiendo del Lagrangiano de una partícula libre en coordenadas esféricas:
\[ L = \dfrac{m}{2} \left( \dot{r}^{2} + r^{2} \: \dot{\theta}^{2} + r^{2} \: \sin \theta^{2} \: \dot{\varphi}^{2} \right)\]
Calcula los símbolos de Christoffel.
\item Las ecuaciones de Navier-Stokes para el flujo de un fluido incompresible
\[ - \vb{\nabla} \cp ( \vb{v} \cp (\vb{\nabla} \cp \vb{v} )) =  \dfrac{\eta}{\rho_{0}} \vb{\nabla}^{2} (\vb{\nabla} \cp \vb{v}) \]
Donde $\eta$ es la viscosidad y $\rho_{0}$ la densidad del fluido. Para un flujo axial dentro de un cilindro, consideremos que la velocidad $\vb{v}$ está dada por
\[ \vb{v} =  \vu{k} \: v (\rho) \]
Considera que
\[ \vb{\nabla} \cp (\vb{v} \cp (\vb{\nabla} \cp \vb{v})) = 0 \]
para este valor de $\vb{v}$.
\\
Demuestra que se cumple la siguiente expresión:
\[ \vb{\nabla}^{2} ( \vb{\nabla} \cp \vb{v}) = 0  \]
nos lleva a la ecuación diferencial
\[ \dfrac{1}{\rho} \: \dv{\rho} \left( \rho \: \dv[2]{v}{\rho} \right) -  \dfrac{1}{\rho^{2}} \: \dv{v}{\rho} = 0 \]
y que la siguiente expresión es solución de la misma ecuación diferencial
\[ v = v_{0} + a_{2} \: \rho^{2} \]
\item El cálculo de efecto \enquote{pinch} en magnetohidrodinámica, involucra la evaluación de $(\vb{B} \vdot \vb{\nabla}) \vb{B}$. Si la inducción magnética $\vb{B}$ se toma como $\vb{B} = \vu*{\varphi} \: B_{\varphi} (\rho)$, demostrar que
\[ (\vb{B} \vdot \vb{\nabla}) \vb{B} = - \dfrac{\vu{\rho} \: B_{\varphi}^{2}}{\rho} \]
\item Un cierto campo de fuerza está dado por
\[ \mathbf{F} = \mathbf{r}_{0} \dfrac{2 P \cos \theta}{r^{3}} + \bm{\theta}_{0} \dfrac{P}{r^{3}} \sin \theta, \hspace{1.5cm} r \geq P/2 \]
en coordenadas polares.
\begin{milista}
\item Revisa $\bm{\nabla} \times \mathbf{F}$ para checar si existe un potencial.
\item Calcular $\mathlarger{\oint} \mathbf{F} \cdot d \bm{\lambda}$ para un círculo unitario en el plano $\theta = \pi/2$.\
¿Qué nos dice esto sobre la fuerza?¿Es conservativa o no conservativa?
\item Si considera que $\mathbf{F}$ se puede describir por $\mathbf{F} = - \bm{\nabla}\psi$, encuentra $\psi$. De otra manera argumenta que no es posible que un potencial exista.
\end{milista}
\item Con $\vb{L} = -i \: \vb{r} \cp \vb{\nabla}$, verifica las siguientes identidades de cada operador:
\begin{milista}
\item $\vb{\nabla} = \vu{r} \: \displaystyle \pdv{r} - i \: \dfrac{\vb{r} \cp \vb{L}}{r^{2}}$
\item $\vb{r} \: \vb{\nabla^{2}} - \vb{\nabla} \left( 1 + r \: \displaystyle \pdv{r} \right) = i \: \vb{\nabla} \cp \vb{L}$
\end{milista}
\item Demostrar que las siguientes tres expresiones del laplaciano: $\nabla^{2} \psi(r)$ en coordenadas esféricas son equivalentes:
\begin{milista}\itemsep4pt
\item $\displaystyle \dfrac{1}{r^{2}} \: \dv{r} \left[ r^{2} \: \dv{\psi(r)}{r} \right]$
\item $\displaystyle \dfrac{1}{r} \: \dv[2]{r} \left[ r \: \psi(r) \right]$
\item $\displaystyle\dv[2]{\psi(r)}{r} + \dfrac{2}{r} \: \dv{\psi(r)}{r}$ 
\end{milista}
%\item Para el vector $\va{A} = - 5 \: \vu{i} + 6 \: \vu{j}$ y con vectores base $\va{e}_{1} = \vu{i} +  2 \: \vu{j}$ y $\va{e}_{2} = - 2 \: \vu{i} - \vu{j}$, calcular:
% \begin{milista}
% \item Los vectores base duales $\va{e}^{\: 1}$ y $\va{e}^{\: 2}$ de los vectores base $\va{e}_{1}$ y $\va{e}_{2}$
% \item Las componentes contravariantes $\va{A}_{1}$ y $\va{A}_{2}$
% \item Las componentes covariantes $\va{A}^{1}$ y $\va{A}^{2}$
% \end{milista}
\item El vector de potencial magnético para una cáscara esférica uniformemente cargada que está rotando es
\begin{align*}
\vb{A} = \begin{cases}
\vu*{\varphi} \: \dfrac{\mu_{0} \: a^{4} \: \sigma \: \omega}{3} \: \dfrac{\sin \theta}{r^{2}} & r > a \\[1em]
\vu*{\varphi} \: \dfrac{\mu_{0} \: a \: \sigma \: \omega}{3} \: r \: \cos \theta & r < a
\end{cases}
\end{align*}
donde $a$ es el radio de la cáscara esférica, $\sigma$ es la densidad superficial de carga y $\omega$ la velocidad angular. Demuestra que la inducción magnética $\vb{B} = \vb{\nabla} \cp \vb{A}$ tiene las siguientes componentes:
\begin{milista}\itemsep4pt
\item $B_{r} (r, \theta) = \dfrac{2 \: \mu_{0} \: a^{4} \: \sigma \: \omega}{3} \: \dfrac{\cos \theta}{r^{3}}, \hspace{1cm} r>a$
\item $B_{\theta} (r, \theta) = \dfrac{\mu_{0} \: a^{4} \: \sigma \: \omega}{3} \: \dfrac{\sin \theta}{r^{3}}, \hspace{1cm} r>a$
\item $\vb{B} = \vu{z} \: \dfrac{2 \: \mu_{0} \: a \: \sigma \: \omega}{3}, \hspace{1cm} r < a$
\end{milista}
\end{milista}
\end{document}