\documentclass[hidelinks,12pt]{article}
% \usepackage{../Estilos/ColoresLatex}
\usepackage[left=0.25cm,top=1cm,right=0.25cm,bottom=1cm]{geometry}
%\usepackage[landscape]{geometry}
\textwidth = 20cm
\hoffset = -1cm
\usepackage[utf8]{inputenc}
\usepackage[spanish,es-tabla, es-lcroman]{babel}
\usepackage[autostyle,spanish=mexican]{csquotes}
\usepackage[tbtags]{amsmath}
\usepackage{nccmath}
\usepackage{amsthm}
\usepackage{amssymb}
\usepackage{mathrsfs}
\usepackage{graphicx}
\usepackage{subfig}
\usepackage{caption}
%\usepackage{subcaption}
\usepackage{standalone}
\graphicspath{{Imagenes/}{../Imagenes/}}
\usepackage[outdir=./Imagenes/]{epstopdf}
\usepackage{siunitx}
\usepackage{physics}
\AtBeginDocument{\RenewCommandCopy\qty\SI}
\ExplSyntaxOn
\msg_redirect_name:nnn { siunitx } { physics-pkg } { none }
\ExplSyntaxOff
\usepackage{color}
\usepackage{float}
\usepackage{hyperref}
\usepackage{multicol}
\usepackage{multirow}
%\usepackage{milista}
\usepackage{anyfontsize}
\usepackage{anysize}
%\usepackage{enumerate}
\usepackage[shortlabels]{enumitem}
\usepackage{capt-of}
\usepackage{bm}
\usepackage{mdframed}
\usepackage{relsize}
\usepackage{placeins}
\usepackage{empheq}
\usepackage{cancel}
\usepackage{pdfpages}
\usepackage{wrapfig}
\usepackage[flushleft]{threeparttable}
\usepackage{makecell}
\usepackage{fancyhdr}
\usepackage{tikz}
\usepackage{bigints}
\usepackage{tcolorbox}
\tcbuselibrary{breakable}
\usepackage{scalerel}
\usepackage{pgfplots}
\usepackage{pdflscape}
\usepackage{enumitem}
\pgfplotsset{compat=1.16}
\spanishdecimal{.}
\renewcommand{\baselinestretch}{1.5}
\def\scaleint#1{\vcenter{\hbox{\scaleto[3ex]{\displaystyle\int}{#1}}}}
\def\scaleoint#1{\vcenter{\hbox{\scaleto[3ex]{\displaystyle\oint}{#1}}}}
\def\scaleiint#1{\vcenter{\hbox{\scaleto[3ex]{\displaystyle\iint}{#1}}}}
\def\scaleiiint#1{\vcenter{\hbox{\scaleto[3ex]{\displaystyle\iiint}{#1}}}}
\def\bs{\mkern-12mu}

\newcommand{\Cancel}[2][black]{{\color{#1}\cancel{\color{black}#2}}}

% \newcommand{\qed}{\tag*{$\blacksquare$}}
\renewcommand{\qed}{\hfill\blacksquare}

\usepackage{titling}


\title{Lista de ejercicios del Tema 1 \\[0.3em]  \large{Matemáticas Avanzadas de la Física}\vspace{-3ex}}
\author{M. en C. Gustavo Contreras Mayén}
\date{ }

\setlength{\droptitle}{-3cm}

\begin{document}

\vspace{-6cm}
\maketitle

\fontsize{14}{14}\selectfont

\noindent
\textbf{Nota importante:} Considera que conforme se vayan presentando los ejercicios en clase, se actualizará el documento para que tengas los enunciados y vayas resolviendo cada ejercicio de manera oportuna.
\\[0.75em]
\noindent
\textbf{Indicaciones: } Se te pide gentilmente que resuelvas de manera detallada, clara y ordenada los siguientes ejercicios, el puntaje que otorga cada enunciado es de \textbf{1 punto}. En caso de que requieras apoyarte en alguna propiedad, si fue vista en clase, solo indícalo, pero si esa propiedad aunque esté relacionada al ejercicio y no se haya mencionado en clase, habrá que demostrarla debidamente.

\begin{enumerate}
\item Considera la transformación de coordenadas:
\begin{align*}
x &= 2 \, u \, v \\[0.5em]
y &= u^{2} + v^{2} \\[0.5em]
z &= w
\end{align*}
Demuestra que el nuevo sistema de coordenadas \textbf{es no ortogonal}.
\item Considera la transformación de coordenadas:
\begin{align*}
x &= 2 \, u \, v \\[0.5em]
y &= u^{2} - v^{2} \\[0.5em]
z &= w
\end{align*}
Demuestra que el nuevo sistema de coordenadas \textbf{es ortogonal}.
\item Escribe en coordenadas esféricas $(r, \theta, \phi)$ el siguiente vector:
\begin{align*}
\vb{A} = x \, y \, \vu{i} - x \, \vu{j} + 3 \, x \, \vu{k}
\end{align*}
adicionalmente expresa $A_{r}, A_{\theta}, A_{\phi}$ en términos de $r, \theta, \phi$.
\end{enumerate}

\end{document}