\documentclass[12pt,landscape]{article}
\usepackage[utf8]{inputenc}
\usepackage[spanish,es-lcroman, es-tabla]{babel}
\usepackage[autostyle,spanish=mexican]{csquotes}
\usepackage{amsmath}
\usepackage{amssymb}
\usepackage{nccmath}
\numberwithin{equation}{section}
\usepackage{amsthm}
\usepackage{graphicx}
\usepackage[outdir=./]{epstopdf}
\DeclareGraphicsExtensions{.pdf,.png,.jpg,.eps}
\usepackage{color}
\usepackage{float}
\usepackage{fancyhdr}
\usepackage{multicol}
\usepackage{enumerate}
\usepackage[shortlabels]{enumitem}
\usepackage{anyfontsize}
\usepackage{anysize}
\usepackage{array}
\usepackage{multirow}
\usepackage{enumitem}
\usepackage{cancel}
\usepackage{nameref}
\usepackage{pdflscape}
\usepackage{makecell}
\usepackage{longtable}
\usepackage{pgfplots}
\usepackage{tikz}
\usepackage{circuitikz}
\usepackage{tikz-3dplot}
\usepackage{caption}
\usepackage{bm}
\usepackage{mathtools}
\usepackage{esvect}
\usepackage{hyperref}
\usepackage{relsize}
\usepackage{siunitx}
\usepackage{physics}
%\usepackage[backend=biber]{biblatex}
\usepackage{standalone}
\usepackage{mathrsfs}
\usepackage{bigints}
\usepackage{bookmark}
\spanishdecimal{.}
\title{Tabla de Sistemas Coordenados Ortogonales \\ {\large Matemáticas Avanzadas de la Física}}
\date{ }
\author{}
\pgfplotsset{compat=1.12}
\begin{document}

\renewcommand\labelenumii{\theenumi.{\arabic{enumii}}}
\maketitle
\fontsize{14}{14}\selectfont
\vspace{-2cm}
{\renewcommand{\arraystretch}{4}%
\begin{longtable}{| l | p{5cm} | l | p{7.3cm} |}
\hline

\makecell{Sistema \\ Coordenado} & \makecell{Coordenadas \\ curvilíneas \\ $(q_{1}, q_{2}, q_{3})$} & \makecell{Transformación cartesiana \\ $(x, y,z)$} & \makecell{Factores \\ de escala} \\ \hline
\endfirsthead

\hline
\makecell{Sistema \\ Coordenado} & \makecell{Coordenadas \\ curvilíneas \\ $(q_{1}, q_{2}, q_{2})$} & \makecell{Transformación cartesiana \\ $(x, y,z)$} & \makecell{Factores \\ de escala} \\ \hline
\endhead

Esféricas polares & 
$\!\begin{aligned}
r &\in [0, \infty) \\
\theta &\in [0, \pi] \\
\phi &\in [0, 2\pi)
\end{aligned}$ & $\!\begin{aligned} 
x &= r \sin \theta \cos \phi \\ 
y &= r \sin \theta \sin \phi \\
z &= r \cos \theta
\end{aligned}$ & 
$\!\begin{aligned} %
h_{1} &= 1 \\
h_{2} &= r \\
h_{3} &= r \sin \theta
\end{aligned}$ \\ \hline

Cilíndricas polares & $\!\begin{aligned}
r &\in[0,\infty) \\
\phi &\in [0,2\pi) \\
z &\in (-\infty,\infty) 
\end{aligned}$ &
$\!\begin{aligned}
x &= r \cos \phi \\
y &= r \sin \phi \\
z &= z
\end{aligned}$ &
$\!\begin{aligned}
h_{1 }&= h_{3} = 1 \\
h_{2} &= r
\end{aligned}$ \\ \hline

Cilíndricas parabólicas & $\!\begin{aligned}
u &\in (-\infty,\infty) \\
v &\in [0,\infty) \\
z &\in(-\infty,\infty)
\end{aligned}$ & $\!\begin{aligned}
 x &= \dfrac{1}{2}(u^{2 } -v^{2})\\
y &= u \: v \\
z &=z
\end{aligned}$ & $\!\begin{aligned}
h_{1 } &= h_{2} = \sqrt{u^{2 } +v^{2}} \\
h_{3 } &= 1
\end{aligned}$ \\ \hline

Parabólicas & $\!\begin{aligned}
u &\in [0, \infty) \\
v &\in [0, \infty) \\
\phi &\in [0, 2 \: \pi)
\end{aligned}$ & $\!\begin{aligned}
x &= u \: v \: \cos \phi \\
y &= u \: v \: \sin \phi \\
z &= \dfrac{1}{2} (u^{2} - v^{2})
\end{aligned}$ & $\!\begin{aligned}
h_{1} &= h_{2} = \sqrt{u^{2 } +v^{2}} \\
h_{3} &= u \: v
\end{aligned}$ \\ \hline

Paraboloide & $\!\begin{aligned}
& (\lambda, \mu, \nu) \\
& \lambda < b^{2} < \mu < a^{2} < \nu
\end{aligned}$ & $\!\begin{aligned}
\dfrac{x^{2}}{q_{i} - a^{2}} + \dfrac{y^{2}}{q_{i} - b^{2}} = 2 \: z + q_{i} \\[1em]
\mbox{donde } (q_{1}, q_{2}, q_{3})=(\lambda, \mu, \nu)
\end{aligned}$ & $\!\begin{aligned}
h_{i} = \dfrac{1}{2} \sqrt{\dfrac{(q_{j} - q_{i})(q_{k} - q_{i})}{(a^{2} - q_{i})(b^{2} - q_{ })}}
\end{aligned}$ \\\hline

Cilíndricas elípticas & $\!\begin{aligned}
u &\in [0,\infty) \\
v &\in [0, 2 \: \pi) \\
z &\in (-\infty, \infty)
\end{aligned}$ & $\!\begin{aligned}
x &= a \: \cosh u \: \cos v \\
y &= a \: \sinh u \: \sin v \\
z &= z
\end{aligned}$ & $\!\begin{aligned}
h_{1 }&= h_{2} = a \: \sqrt{\sinh^{2 }u + \sin^{2 }v} \\
h_{3} &= 1
\end{aligned}$ \\ \hline

Esferoidales prolatas & $\!\begin{aligned}
\xi &\in [0, \infty) \\
\eta &\in [0, \pi] \\
\phi &\in [0, 2 \: \pi)
\end{aligned}$ & $\!\begin{aligned}
x &= a \: \sinh \xi \: \sin \eta \: \cos \phi\\
y &= a \: \sinh \xi \: \sin \eta \: \sin \phi\\
z &= a \: \cosh \xi \: \cos \eta
\end{aligned}$ & $\!\begin{aligned}
h_{1} &= h_{2} = a \: \sqrt{\sinh^{2} \xi + \sin^{2} \eta} \\
h_{3 }&= a \: \sinh \xi \: \sin \eta
\end{aligned}$ \\ \hline

Esferoidales oblatas & $\!\begin{aligned}
\xi &\in [0, \infty) \\
\eta &\in \left[- \dfrac{\pi}{2}, \dfrac{\pi}{2} \right] \\
\phi &\in [0, 2\: \pi)
\end{aligned}$ & $\!\begin{aligned}
x &= a \: \cosh \xi \: \cos \eta \: \cos \phi \\
y &= a \: \cosh \xi \: \cos \eta \: \sin \phi\\
z &= a \: \sinh \xi \: \sin \eta
\end{aligned}$ & $\!\begin{aligned}
h_{1 }&= h_{2} = a \: \sqrt{\sinh^{2} \xi + \sin^{2} \eta} \\
h_{3 }&= a \: \cosh \xi \: \cos \eta
\end{aligned}$ \\ \hline

Elipsoidales & $\!\begin{aligned}
& (\lambda, \mu, \nu) \\
& \lambda < c^{2} < b^{2} < a^{2}, \\
& c^{2} < \mu < b^{2} < a^{2}, \\
& c^{2} < b^{2} < \nu < a^{2},
\end{aligned}$ & $\!\begin{aligned}
\dfrac{x^{2}}{a^{2} - q_{i}} + \dfrac{y^{2}}{b^{2} - q_{i}} + \dfrac{z^{2}}{c^{2} - q_{i}} = 1 \\[1em]
\mbox{donde} (q_{1}, q_{2}, q_{3})= (\lambda, \mu, \nu)
\end{aligned}$ & $\!\begin{aligned}
h_{i} = \dfrac{1}{2} \sqrt{\dfrac{(q_{j} - q_{i})(q_{k} - q_{i})}{(a^{2} - q_{i})(b^{2} - q_{i})(c^{2} - q_{i})}}
\end{aligned}$ \\ \hline

Cilíndricas bipolares & $\!\begin{aligned}
u &\in [0, 2 \: \pi) \\
v &\in (-\infty, \infty) \\
z &\in (-\infty, \infty)
\end{aligned}$ & $\!\begin{aligned}
x &= \dfrac{a \: \sinh v}{\cosh v - \cos u} \\
y &= \dfrac{a \: \sin u}{\cosh v - \cos u} \\
z &= z
\end{aligned}$ & $\!\begin{aligned}
h_{1} &= h_{2} = \dfrac{a}{\cosh v - \cos u} \\
h_{3} &= 1
\end{aligned}$ \\ \hline

Toroidales & $\!\begin{aligned}
u &\in (-\pi, \pi] \\
v &\in [0, \infty) \\
\phi &\in [0, 2 \: \pi)
\end{aligned}$ & $\!\begin{aligned}
x &= \dfrac{a \: \sinh v \cos \phi}{\cosh v - \cos u} \\
y &= \dfrac{a \: \sinh v \sin \phi}{\cosh v - \cos u} \\
z &= \dfrac{a \: \sin u}{\cosh v - \cos u}
\end{aligned}$ & $\!\begin{aligned}
h_{1} &= h_{2} = \dfrac{a}{\cosh v - \cos u} \\
h_{3} &= \dfrac{a \: \sinh v}{\cosh v - \cos u}
\end{aligned}$ \\ \hline

Biesféricas & $\!\begin{aligned}
u &\in (-\pi, \pi] \\
v &\in [0, \infty) \\
\phi &\in [0, 2\: \pi)
\end{aligned}$ & $\!\begin{aligned}
x &= \dfrac{a \: \sin u \: \cos \phi}{\cosh v - \cos u} \\
y &= \dfrac{a \: \sin u \: \sin \phi}{\cosh v - \cos u} \\
z &= \dfrac{a \: \sinh v}{\cosh v - \cos u}
\end{aligned}$ & $\!\begin{aligned}
h_{1 }&= h_{2} = \dfrac{a}{\cosh v - \cos u} \\
h_{3} &= \dfrac{a \: \sin u}{\cosh v - \cos u}
\end{aligned}$ \\ \hline

Cónicas& $\!\begin{aligned}
&(r, \theta, \lambda) \\
&r \hspace{1cm} 0 \leq r < \infty \\
&\theta \hspace{1cm} b^{2} < \theta < c^{2} \\
&\lambda \hspace{1cm} 0 < \lambda < b^{2} \\
&\mbox{donde }
\\
&c^{2} > \theta^{2} > b^{2} > \lambda^{2} > 0
% & \nu^{2} < b^{2} < \mu^{2} < a^{2} \\
% & \lambda \in [0, \infty)
\end{aligned}$ & $\!\begin{aligned}
x &= \dfrac{r \, \theta \, \lambda}{b \, c} \\
y &= \dfrac{r}{b} \, \sqrt{\dfrac{(\theta^{2} - b^{2})(\lambda^{2} - b^{2})}{(b^{2} - c^{2})}} \\
z &= \dfrac{r}{c} \, \sqrt{\dfrac{(\theta^{2} - c^{2})(\lambda^{2} - c^{2})}{(c^{2} - b^{2})}}
\end{aligned}$ & $\!\begin{aligned}
h_{1}^{2} &= 1\\
h_{2}^{2} &= \dfrac{r^{2} (\theta^{2} - \lambda^{2})}{(\theta^{2} - b^{2})(c^{2} - \theta^{2})} \\
h_{3}^{2} &= \dfrac{r^{2} (\theta^{2} - \lambda^{2})}{(\lambda^{2} - b^{2})(\lambda^{2} - c^{2})}
\end{aligned}$ \\ \hline
\end{longtable}}

Tomen en cuenta que tanto la referencia de las variables $q_{i}$ como en algunos sistemas coordenados, las reglas de transformación pueden tener variantes, esto dependerá del texto de física matemática que se consulte.
\par
Es un buen ejercicio obtener los factores de escala de los sistemas mencionados, ya que les brindará la oportunidad de hacer un repaso para la diferenciación parcial y del álgebra necesaria para llegar a la expresión que se señala en la columna de factores de escala.
\end{document}