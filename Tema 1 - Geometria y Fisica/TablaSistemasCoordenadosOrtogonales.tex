\documentclass[12pt,landscape]{article}
\usepackage[utf8]{inputenc}
\usepackage[spanish,es-lcroman, es-tabla]{babel}
\usepackage[autostyle,spanish=mexican]{csquotes}
\usepackage{amsmath}
\usepackage{amssymb}
\usepackage{nccmath}
\numberwithin{equation}{section}
\usepackage{amsthm}
\usepackage{graphicx}
\usepackage[outdir=./]{epstopdf}
\DeclareGraphicsExtensions{.pdf,.png,.jpg,.eps}
\usepackage{color}
\usepackage{float}
\usepackage{fancyhdr}
\usepackage{multicol}
\usepackage{enumerate}
\usepackage[shortlabels]{enumitem}
\usepackage{anyfontsize}
\usepackage{anysize}
\usepackage{array}
\usepackage{multirow}
\usepackage{enumitem}
\usepackage{cancel}
\usepackage{nameref}
\usepackage{pdflscape}
\usepackage{makecell}
\usepackage{pgfplots}
\usepackage{tikz}
\usepackage{circuitikz}
\usepackage{tikz-3dplot}
\usepackage{caption}
\usepackage{bm}
\usepackage{mathtools}
\usepackage{esvect}
\usepackage{hyperref}
\usepackage{relsize}
\usepackage{siunitx}
\usepackage{physics}
%\usepackage[backend=biber]{biblatex}
\usepackage{standalone}
\usepackage{mathrsfs}
\usepackage{bigints}
\usepackage{bookmark}
\spanishdecimal{.}
\title{Tabla de Sistemas Coordenados Ortogonales \\ {\large Matemáticas Avanzadas de la Física}}
\date{ }
\begin{document}

\renewcommand\labelenumii{\theenumi.{\arabic{enumii}}}
\maketitle
\fontsize{14}{14}\selectfont
{\renewcommand{\arraystretch}{4}%
\begin{tabular}{| l | p{5cm} | l | p{5cm} |}
\hline
Sistema & \makecell{Coordenadas \\ curvilíneas \\ $(q_{1}, q_{2}, q_{2})$} & \makecell{Transformación cartesiana \\ $(x, y,z)$} & \makecell{Factores \\ de escala} \\ \hline
Coordenadas esféricas polares & 
$\!\begin{aligned}
r &\in [0, \infty) \\
\theta &\in [0, \pi] \\
\phi &\in [0,2\pi)
\end{aligned}$ & $\!\begin{aligned} 
x &= r \sin \theta \cos \phi \\ 
y &= r \sin \theta \sin \phi \\
z &= r \cos \theta
\end{aligned}$ & 
$\!\begin{aligned} %
h_{1} &= 1 \\
h_{2} &= r \\
h_{3} &= r \sin \theta
\end{aligned}$ \\ \hline
Coordenadas cilíndricas polares & $\!\begin{aligned}
r &\in[0,\infty) \\
\phi &\in [0,2\pi) \\
z &\in (-\infty,\infty) 
\end{aligned}$ &
$\!\begin{aligned}
x &= r \cos \phi \\
y &= r \sin \phi \\
z &= z
\end{aligned}$ &
$\!\begin{aligned}
h_{1 }&= h_{3} = 1 \\
h_{2} &= r
\end{aligned}$ \\ \hline
\end{tabular}}
% | [[Parabolic cylindrical coordinates]]
% <math>(u, v, z)\in(-\infty,\infty)\times[0,\infty)\times(-\infty,\infty)</math>
% | <math>\begin{align}
% x&=\frac{1}{2}(u^2-v^2)\\
% y&=uv\\
% z&=z
% \end{align}</math>
% | <math>\begin{align}
% h_1&=h_2=\sqrt{u^2+v^2} \\
% h_3&=1
% \end{align}</math> 
% |-
% | [[Parabolic_coordinates#Three-dimensional_parabolic_coordinates|Parabolic coordinates]]
% <math>(u, v, \phi)\in[0,\infty)\times[0,\infty)\times[0,2\pi)</math>
% | <math>\begin{align}
% x&=uv\cos\phi\\
% y&=uv\sin\phi\\
% z&=\frac{1}{2}(u^2-v^2)
% \end{align}</math>
% | <math>\begin{align}
% h_1&=h_2=\sqrt{u^2+v^2} \\
% h_3&=uv
% \end{align}</math> 
% |-
% | [[Paraboloidal coordinates]]
% <math>\begin{align}
% & (\lambda, \mu, \nu)\\
% & \lambda < b^2 < \mu < a^2 < \nu
% \end{align}</math>
% | <math>\frac{x^2}{q_i - a^2} + \frac{y^2}{q_i - b^2} = 2 z + q_i</math>
% where 
% <math>(q_1,q_2,q_3)=(\lambda,\mu,\nu)</math>
% | <math>h_i=\frac{1}{2} \sqrt{\frac{(q_j-q_i)(q_k-q_i)}{(a^2-q_i)(b^2-q_i)}}</math> 
% |-
% | [[Elliptic cylindrical coordinates]]
% <math>(u, v, z)\in[0,\infty)\times[0,2\pi)\times(-\infty,\infty)</math>
% | <math>\begin{align}
% x&=a\cosh u \cos v\\
% y&=a\sinh u \sin v\\
% z&=z
% \end{align}</math>
% | <math>\begin{align}
% h_1&=h_2=a\sqrt{\sinh^2u+\sin^2v} \\
% h_3&=1
% \end{align}</math> 
% |-
% | [[Prolate spheroidal coordinates]]
% <math>(\xi, \eta, \phi)\in[0,\infty)\times[0,\pi]\times[0,2\pi)</math>
% | <math>\begin{align}
% x&=a\sinh\xi\sin\eta\cos\phi\\
% y&=a\sinh\xi\sin\eta\sin\phi\\
% z&=a\cosh\xi\cos\eta
% \end{align}</math>
% | <math>\begin{align}
% h_1&=h_2=a\sqrt{\sinh^2\xi+\sin^2\eta} \\
% h_3&=a\sinh\xi\sin\eta
% \end{align}</math> 
% |-
% | [[Oblate spheroidal coordinates]]
% <math>(\xi, \eta, \phi)\in[0,\infty)\times\left[-\frac{\pi}{2},\frac{\pi}{2}\right]\times[0,2\pi)</math>
% | <math>\begin{align}
% x&=a\cosh\xi\cos\eta\cos\phi\\
% y&=a\cosh\xi\cos\eta\sin\phi\\
% z&=a\sinh\xi\sin\eta
% \end{align}</math>
% | <math>\begin{align}
% h_1&=h_2=a\sqrt{\sinh^2\xi+\sin^2\eta} \\
% h_3&=a\cosh\xi\cos\eta
% \end{align}</math> 
% |-
% | [[Ellipsoidal coordinates]]
% <math>\begin{align}
% & (\lambda, \mu, \nu)\\
% & \lambda < c^2 < b^2 < a^2,\\
% & c^2 < \mu < b^2 < a^2,\\
% & c^2 < b^2 < \nu < a^2,
% \end{align}</math>
% | <math>\frac{x^2}{a^2 - q_i} + \frac{y^2}{b^2 - q_i} + \frac{z^2}{c^2 - q_i} = 1</math>
% where 
% <math>(q_1,q_2,q_3)=(\lambda,\mu,\nu)</math>
% | <math>h_i=\frac{1}{2} \sqrt{\frac{(q_j-q_i)(q_k-q_i)}{(a^2-q_i)(b^2-q_i)(c^2-q_i)}}</math> 
% |-
% | [[Bipolar cylindrical coordinates]]
% <math>(u,v,z)\in[0,2\pi)\times(-\infty,\infty)\times(-\infty,\infty)</math>
% | <math>\begin{align}
% x&=\frac{a\sinh v}{\cosh v - \cos u}\\
% y&=\frac{a\sin u}{\cosh v - \cos u}\\
% z&=z
% \end{align}</math>
% | <math>\begin{align}
% h_1&=h_2=\frac{a}{\cosh v - \cos u}\\
% h_3&=1
% \end{align}</math>
% |-
% | [[Toroidal coordinates]]
% <math>(u,v,\phi)\in(-\pi,\pi]\times[0,\infty)\times[0,2\pi)</math>
% | <math>\begin{align}
% x &= \frac{a\sinh v \cos\phi}{\cosh v - \cos u}\\
% y &= \frac{a\sinh v \sin\phi}{\cosh v - \cos u} \\
% z &= \frac{a\sin u}{\cosh v - \cos u}
% \end{align}</math>
% | <math>\begin{align}
% h_1&=h_2=\frac{a}{\cosh v - \cos u}\\
% h_3&=\frac{a\sinh v}{\cosh v - \cos u}
% \end{align}</math>
% |-
% | [[Bispherical coordinates]]
% <math>(u,v,\phi)\in(-\pi,\pi]\times[0,\infty)\times[0,2\pi)</math>
% | <math>\begin{align}
% x &= \frac{a\sin u \cos \phi}{\cosh v - \cos u}\\
% y &= \frac{a\sin u \sin \phi}{\cosh v - \cos u} \\
% z &= \frac{a\sinh v}{\cosh v - \cos u}
% \end{align}</math>
% | <math>\begin{align}
% h_1&=h_2=\frac{a}{\cosh v - \cos u}\\
% h_3&=\frac{a\sin u}{\cosh v - \cos u}
% \end{align}</math>
% |-
% | [[Conical coordinates]]
% <math>\begin{align}
% & (\lambda,\mu,\nu)\\
% & \nu^2 < b^2 < \mu^2 < a^2 \\
% & \lambda \in [0,\infty)
% \end{align}</math>
% | <math>\begin{align}
% x &= \frac{\lambda\mu\nu}{ab}\\
% y &= \frac{\lambda}{a}\sqrt{\frac{(\mu^2-a^2)(\nu^2-a^2)}{a^2-b^2}} \\
% z &= \frac{\lambda}{b}\sqrt{\frac{(\mu^2-b^2)(\nu^2-b^2)}{a^2-b^2}}
% \end{align}</math>
% | <math>\begin{align}
% h_1&=1\\
% h_2^2&=\frac{\lambda^2(\mu^2-\nu^2)}{(\mu^2-a^2)(b^2-\mu^2)}\\
% h_3^2&=\frac{\lambda^2(\mu^2-\nu^2)}{(\nu^2-a^2)(\nu^2-b^2)}
% \end{align}</math>
\end{document}