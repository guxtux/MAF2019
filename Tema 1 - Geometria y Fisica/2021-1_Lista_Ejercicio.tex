\input{../Preambulos/preambulo_presentacion_Warsaw_seahorse}
\title{\large{Construcción de sistemas coordenados especiales}}
\subtitle{Tema 1 - La física y la geometría}
\author{M. en C. Gustavo Contreras Mayén}
\date{\today}
\institute{Facultad de Ciencias - UNAM}
\titlegraphic{\includegraphics[width=1.75cm]{../Imagenes/escudo-facultad-ciencias}\hspace*{4.75cm}~%
   \includegraphics[width=1.75cm]{../Imagenes/escudo-unam}
}
\setbeamertemplate{navigation symbols}{}
\begin{document}
\maketitle
\fontsize{14}{14}\selectfont
\spanishdecimal{.}
\section*{Contenido}
\frame[allowframebreaks]{\tableofcontents[currentsection, hideallsubsections]}
\section{Presentación 1}
\frame{\tableofcontents[currentsection, hideothersubsections]}
\subsection{E1 - Sistema no ortogonal}
\begin{frame}
\frametitle{Ejercicio a cuenta}
Considera la transformación de coordenadas:
\begin{align*}
x &= 2 \, u \, v \\[0.5em]
y &= u^{2} + v^{2} \\[0.5em]
z &= w
\end{align*}
Demuestra que el nuevo sistema de coordenadas \emph{no} es ortogonal.
\end{frame}
\section{Presentación 2}
\frame[allowframebreaks]{\tableofcontents[currentsection, hideothersubsections]}
\subsection{E1 - Vel, acel coord. esféricas}
\begin{frame}
\frametitle{Ejercicio a cuenta}
La velocidad y la aceleración se definen en la forma vectorial como:
\begin{align*}
\vb{v} = \dv{\vb{r}}{t} = \dot{\vb{r}} \hspace{1cm} \vb{a} = \dot{\vb{v}} = \ddot{\vb{r}}
\end{align*}
Calcula para el sistema coordenado esférico:
\setbeamercolor{item projected}{bg=blue!70!black,fg=yellow}
\setbeamertemplate{enumerate items}[circle]
\begin{enumerate}
\item $\dot{\vu{e}}_{r}$, $\dot{\vu{e}}_{\theta}$, $\dot{\vu{e}}_{\varphi}$ 
\item La velocidad $\vb{v}$
\item La aceleración $\vb{a}$
\end{enumerate}
\end{frame}
\subsection{E2 - Productos vectoriales}
\begin{frame}
\frametitle{Ejercicio a cuenta}
Demuestra que para dos vectores $\vb{A}$ y $\vb{B}$:
\setbeamercolor{item projected}{bg=blue!70!black,fg=yellow}
\setbeamertemplate{enumerate items}[circle]
\begin{enumerate}
\item $\vb{A} \cp \vb{B} = \displaystyle \sum_{ijk} \, \vu{e}_{i} \, \epsilon_{ijk} \, A_{j} \, B_{k}$ \\[1em]
\item $\vb{A} \cdot \vb{B} \cp \vb{C} = \displaystyle \sum_{ijk} \, \epsilon_{ijk} \, A_{i} \, B_{j} \, C_{k}$
\end{enumerate}
\end{frame}
\subsection{E3 - Gradiente}
\begin{frame}
\frametitle{Ejercicios a cuenta}
Para ejercitar el avance que llevamos, resuelve:
\setbeamercolor{item projected}{bg=blue!70!black,fg=yellow}
\setbeamertemplate{enumerate items}[circle]
\begin{enumerate}
\item Demuestra que $\grad{\phi \psi} = \phi \, \grad{\psi} + \psi \, \grad{\phi}$
\item Si $f = f(r)$ con $r = \sqrt{x^{2} + y^{2}+ z^{2}}$, demuestra que
\begin{align*}
\nabla{f(r)} = \vu{r} \, \dv{f(r)}{r}
\end{align*}
\end{enumerate}
\end{frame}
\subsection{E4 - Divergencia campo eléctrico}
\begin{frame}
\frametitle{Ejercicio a cuenta}
Demuestra que el campo eléctrico de una carga puntal
\begin{align*}
\vb{E} = \dfrac{q \, \vu{r}}{4 \, \pi \epsilon_{0} \, r^{2}}
\end{align*}
cumple $\div{E} = 0$, para $r \neq 0$.
\end{frame}
\subsection{E5 - Ley de Gauss}
\begin{frame}
\frametitle{Ejercicio a cuenta}
La ley de Gauss para el campo eléctrico tiene la forma:
\begin{align*}
\oint \vb{E} \cdot \dd{\vb{S}} = \dfrac{q}{\epsilon_{0}}
\end{align*}
donde $q = \displaystyle \int \rho \dd{V}$ es la carga encerrada en la superficie y $\rho$ su densidad volumétrica.
\\
\bigskip
Demuestra la ley de Gauss en forma diferencial
\begin{align*}
\div{E} = \dfrac{\rho}{\epsilon_{0}}
\end{align*}
\end{frame}
\subsection{E6 - Rotacional}
\begin{frame}
\frametitle{Ejercicio a cuenta}
Demuestra que:
\begin{align*}
\curl(\phi \, \vb{A}) = \phi \, \curl{\vb{A}} + \grad{\phi} \times \vb{A}
\end{align*}
\end{frame}
\subsection{E7 - Divergencia y rotacional}
\begin{frame}
\frametitle{Ejercicio a cuenta}
El campo electrostático de un dipolo eléctrico $\vb{p} = p_{0} \, \vu{e}_{z}$ es
\begin{align*}
\vb{E} = \dfrac{p_{0} (2 \, \vb{e}_{r} \, \cos \theta + \vu{e}_{\theta} \, \sin \theta)}{r^{3}}
\end{align*}
Demuestra que:
\setbeamercolor{item projected}{bg=blue!70!black,fg=yellow}
\setbeamertemplate{enumerate items}[circle]
\begin{enumerate}
\item $\curl{\vb{E}} = 0$
\item para $r \neq 0$, se tiene $\div{\vb{E}} = 0$
\end{enumerate}
\end{frame}
\subsection{E8 - Laplaciano}
\begin{frame}
\frametitle{Ejercicio a cuenta}
Demuestra que el Laplaciano en coordenadas cilíndricas y esféricas es el que se presenta, para ello tendrás que calcular los respectivos factores de escala.
\end{frame}
\section{Presentación 3}
\frame{\tableofcontents[currentsection, hideothersubsections]}
\subsection{E1 - Coord. esferoidales prolatas}
\begin{frame}
\frametitle{Ejercicio a cuenta}
\textbf{Ejercicio a cuenta: } 
Para el sistema de coordenadas esferoidales prolatas $(u, v, \varphi)$, cuyas reglas de transformación son:
\begin{align*}
x &= a \, \sinh u \, \sin v \, \cos \varphi \\
y &= a \, \sinh u \, \sin v \, \sin \varphi \\
z &= a \, \cosh u \, \cos v
\end{align*}
\end{frame}
\begin{frame}
\frametitle{Ejercicio a cuenta}
\textbf{Ejercicio a cuenta: } 
\setbeamercolor{item projected}{bg=blue!70!black,fg=yellow}
\setbeamertemplate{enumerate items}[circle]
\begin{enumerate}
\item Describe las superficies coordenadas del sistema.
\item Calcula de manera explícita los factores de escala $(h_{u}, h_{v}, h_{\varphi})$.
\end{enumerate}
\end{frame}
\subsection{E2 - Operadores diferenciales}
\begin{frame}
\frametitle{Ejercicio a cuenta}
Ocupando el mismo el sistema de coordenadas esferoidales prolatas $(u, v, \varphi)$ del ejercicio a cuenta anterior:
\setbeamercolor{item projected}{bg=blue!70!black,fg=yellow}
\setbeamertemplate{enumerate items}[circle]
\begin{enumerate}
\item Calcula los operadores diferenciales $\grad{\phi}$, $\div{\vb{B}}$, $\curl{\vb{B}}$ y $\laplacian{\phi}$
\end{enumerate}
\end{frame}
\section{Envío de las soluciones}
\frame{\tableofcontents[currentsection, hideothersubsections]}
\subsection{Solución a los ejercicios}
\begin{frame}
\frametitle{Solución}
La solución para cada uno de los ejercicios se debe de hacer:
\setbeamercolor{item projected}{bg=blue!70!black,fg=yellow}
\setbeamertemplate{enumerate items}[circle]
\begin{enumerate}[<+->]
\item Lo más completa posible.
\item Con escritura legible.
\item En orden y secuencia.
\end{enumerate}
\end{frame}
\subsection{Hojas escaneadas}
\begin{frame}
\frametitle{Hojas escaneadas}
De ser posible se deberán de escanear las hojas de las respuestas en el orden para cada ejercicio.
\\
\bigskip
Con las hojas que hayas ocupado, organiza un archivo pdf por cada ejercicio. Se deberá de preparar un archivo comprimido con los pdf de cada ejercicio.
\end{frame}
\begin{frame}
\frametitle{Envío de fotos}
En caso de que no cuentes con un escáner, se podrán tomar fotos a cada hoja para hacer el envío.
\\
\bigskip
Por lo que para cada ejercicio, deberás de identificar la imagen de tal manera que lleve un orden y secuencia para cada ejercicio.
\end{frame}
\subsection{Envío de soluciones}
\begin{frame}
\frametitle{Envío por Moodle}
Se habilitará un espacio en Moodle a partir del 7 de octubre para cargar el archivo comprimido con la solución de los ejercicios.
\\
\bigskip
Será el único medio para recibir el archivo comprimido.
\end{frame}
\begin{frame}
\frametitle{Fecha de entrega}
La fecha límite es el día: 16 de viernes de 2020 a las 3 pm.
\\
\bigskip
En caso de que tengas el archivo comprimido antes de la fecha de envío, puedes cargar el mismo en la plataforma a partir del 7 de octubre.
\end{frame}
\end{document}