\input{../Preambulos/preambulo_presentacion_Warsaw_seahorse}
\title{\large{Símbolos de Christoffel}}
\subtitle{Tema 1 - La física y la geometría}
\author{}
\date{\today}
\institute{Facultad de Ciencias - UNAM}
\titlegraphic{\includegraphics[width=1.75cm]{../Imagenes/escudo-facultad-ciencias}\hspace*{4.75cm}~%
   \includegraphics[width=1.75cm]{../Imagenes/escudo-unam}
}
\setbeamertemplate{navigation symbols}{}
\begin{document}
\maketitle
\fontsize{14}{14}\selectfont
\spanishdecimal{.}
\section*{Contenido}
\frame[allowframebreaks]{\tableofcontents[currentsection, hideallsubsections]}
\section{Matriz métrica}
\frame[allowframebreaks]{\tableofcontents[currentsection, hideothersubsections]}
\subsection{Elemento de línea}
\begin{frame}
\frametitle{Vector de posición}
Consideremos el diferencial del vector de posición en coordenadas curvilíneas:
\begin{align*}
\dd{\vb{r}} &= \pdv{\vb{r}}{u_{1}} \dd{u_{1}} + \pdv{\vb{r}}{u_{2}} \dd{u_{2}} + \pdv{\vb{r}}{u_{3}} \dd{u_{3}} = \\[1em]
&= \bm{\alpha}_{1} \dd{u_{1}} + \bm{\alpha}_{2} \dd{u_{2}} + \bm{\alpha}_{3} \dd{u_{3}}
\end{align*}
\end{frame}
\begin{frame}
\frametitle{Elemento de línea al cuadrado}
Entonces al considerar el cuadrado del elemento de longitud de arco:
\begin{align*}
\dd{s^{2}} &= \dd{\vb{r}} \cdot \dd{\vb{r}} = \bm{\alpha}_{1} \cdot \bm{\alpha}_{1} \dd{u_{1}^{2}} + \bm{\alpha}_{1} \cdot \bm{\alpha}_{2} \dd{u_{1}} \dd{u_{2}} + \\[0.5em]
&+ \bm{\alpha}_{1} \cdot \bm{\alpha}_{3} \dd{u_{1}} \dd{u_{3}} + \bm{\alpha}_{2} \cdot \bm{\alpha}_{1} \dd{u_{2}} \dd{u_{1}} + \\[0.5em]
&+ \bm{\alpha}_{2} \cdot \bm{\alpha}_{2} \dd{u_{2}^{2}} + \bm{\alpha}_{2} \cdot \bm{\alpha}_{3} \dd{u_{2}} \dd{u_{3}} + \\[0.5em]
&+ \bm{\alpha}_{3} \cdot \bm{\alpha}_{1} \dd{u_{3}} \dd{u_{1}} + \bm{\alpha}_{3} \cdot \bm{\alpha}_{2} \dd{u_{3}} \dd{u_{2}} + \\[0.5em]
&+ \bm{\alpha}_{3} \cdot \bm{\alpha}_{3} \dd{u_{3}^{2}} =
\end{align*}
\end{frame}
\begin{frame}
\frametitle{Elemento de línea al cuadrado}
Entonces
\begin{align*}
\dd{s^{2}} = \sum_{p=1}^{3} \sum_{q=1}^{3} g_{pq} \, \dd{u_{p}} \dd{u_{q}} \hspace{1.25cm} \mbox{con } g_{p q} = \bm{\alpha}_{p} \cdot \bm{\alpha}_{q}
\end{align*}
\pause
Que es llamada la \emph{forma cuadrática fundamental} o \emph{forma métrica.}
\end{frame}
\subsection{Matriz métrica}
\begin{frame}
\frametitle{La matriz métrica}
Las cantidades $g_{pq}$ se denominan coeficientes métricos, que a su vez, constituyen la matriz métrica:
\begin{align*}
g = \mqty(
g_{11} & g_{12} & g_{23} \\
g_{21} & g_{22} & g_{23} \\
g_{31} & g_{32} & g_{33}
)
\end{align*}
\end{frame}
\begin{frame}
\frametitle{Propiedades de la matriz métrica}
\setbeamercolor{item projected}{bg=blue!70!black,fg=yellow}
\setbeamertemplate{enumerate items}[circle]
\begin{enumerate}[<+->]
\item Los términos de la matriz métrica son simétricos: $g_{pq} = g_{qp}$
\item Si $g_{pq} = 0$ y $p \neq q$, entonces el sistema coordenado es ortogonal.
\item El cuadrado de los factores de escala, corresponden los elementos de la diagonal principal:
\begin{align*}
g_{11} = h_{1}^{2} \hspace{1cm} g_{22} = h_{2}^{2} \hspace{1cm} g_{33} = h_{3}^{2}
\end{align*}
\end{enumerate}
\end{frame}
\begin{frame}
\frametitle{Generalización a un espacio}
Una generalización para un espacio N-dimensional con coordenadas
\begin{align*}
(x^{1}, x^{2}, \ldots, x^{N})
\end{align*}
es inmediata. Definimos el elemento de línea $\dd{s}$ en este espacio, mediante la forma cuadrática o métrica:
\begin{align*}
\dd{s^{2}} = \sum_{p=1}^{N} \sum_{q=1}^{N} g_{pq} \, \dd{x^{p}} \dd{x^{q}}
\end{align*}
\end{frame}
\begin{frame}
\frametitle{Tensor métrico}
Los $g_{pq}$ son los componentes del \emph{tensor métrico}, también conocido como \emph{tensor fundamental}. 
\end{frame}
\section{Símbolos de Christoffel}
\frame{\tableofcontents[currentsection, hideothersubsections]}
\subsection{Definición}
\begin{frame}
\frametitle{Los símbolos de Christoffel 1a. clase}
Se definen los \emph{símbolos de Christoffel de primera clase} como
\begin{align*}
[p \, q, r] = \dfrac{1}{2} \left( \pdv{g_{pr}}{x^{q}} + \pdv{g_{qr}}{x^{p}} + \pdv{g_{pq}}{x^{r}} \right)
\end{align*}
\end{frame}
\begin{frame}
\frametitle{Los símbolos de Christoffel de 2a. clase}
Se definen los símbolos de Christoffel de segunda clase como:
\begin{align*}
\mathlarger{\Gamma_{pq}^{s}} = \qty\Bigg{ \mqty{s \\[-0.1em] pq} } = g^{sr} \, [p \, q, r]
\end{align*}
\pause
Es conocido el debate sobre sí éstos símbolos a pesar de su caracter tensorial, corresponden propiamente a esta clasificación.
\end{frame}
\section{Ejemplo}
\frame{\tableofcontents[currentsection, hideothersubsections]}
\subsection{Coordenadas cilíndricas}
\begin{frame}
\frametitle{Ejercicio}
Como ejercicio calculemos los símbolos de Christoffel para un sistema de coordenadas cilíndrico.
\\
\bigskip
\pause
El primer paso es determinar el cuadrado de la longitud de arco, para luego calcular el tensor métrico para este sistema de coordenadas.
\end{frame}
\begin{frame}
\frametitle{Longitud de arco}
Las reglas de transformación para el sistema cilíndrico son:
\begin{align*}
x &= \rho \, \cos \phi \\[0.5em]
y &= \rho \, \sin \phi \\[0.5em]
z &= z
\end{align*}
Entonces el vector de posición es:
\end{frame}
\begin{frame}
\frametitle{Longitud de arco}
El vector de posición es:
\begin{align*}
\vb{r} = \rho \, \cos \phi \, \vu{i} + \rho \, \sin \phi \, \vu{k} + z \, \vu{k}
\end{align*}
\pause
Por lo que:
\begin{align*}
\dd{\vb{r}} = \pdv{\vb{r}}{\rho} \dd{\rho} + \pdv{\vb{r}}{\phi} \dd{\phi} + \pdv{\vb{r}}{z} \dd{z}
\end{align*}
\end{frame}
\begin{frame}
\frametitle{Longitud de arco}
\begin{eqnarray*}
\dd{\vb{r}} &=& (\cos \phi \, \vu{i} + \sin \phi \, \vu{j} ) \dd{\rho} + \\[0.5em]
&+& (- \rho \, \sin \phi \, \vu{i} + \rho \, \cos \phi \, \vu{j} ) \dd{\phi} + \vu{k} \, \dd{z} = \\[0.5em] \pause
&=& (\cos \phi \dd{\rho} - \rho \, \sin \phi \dd{\phi}) \, \vu{i} \\[0.5em]
&+& ( \sin \phi \dd{\rho} + \rho \, \cos \phi \dd{\phi} ) \, \vu{j} + \vu{k} \, \dd{z}
\end{eqnarray*}
\end{frame}
\begin{frame}
\frametitle{Longitud de arco}
Por lo que
\begin{align*}
\dd{s^{2}} = \dd{\vb{r}} \cdot \dd{\vb{r}} &= (\cos \phi \dd{\rho} - \rho \, \sin \phi \dd{\phi})^{2} +  \\[0.5em]
&+ ( \sin \phi \dd{\rho} + \rho \, \cos \phi \dd{\phi} )^{2} + (\dd{z})^{2} = \\[0.5em] \pause
&= (\dd{\rho})^{2} + \rho^{2} \, (\dd{\phi})^{2} + (\dd{z})^{2}
\end{align*}
\end{frame}
\begin{frame}
\frametitle{Matriz métrica}
Si hacemos que 
\begin{align*}
x^{1} &= \rho \\[0.5em]
x^{2} &= \phi \\[0.5em]
x^{3} &= z
\end{align*}
\pause
Tendremos que
\begin{align*}
g_{11} = 1 \hspace{1cm} g_{22} = \rho^{2} \hspace{1cm} g_{33} = 1
\end{align*}
\end{frame}
\begin{frame}
\frametitle{Matriz métrica}
Los demás elementos
\begin{align*}
g_{12} = g_{21} = 0 \\[0.5em]
g_{23} = g_{32} = 0 \\[0.5em]
g_{31} = g_{13} = 0
\end{align*}
\end{frame}
\begin{frame}
\frametitle{Matriz métrica}
\begin{align*}
g = \mqty(
g_{11} & g_{12} & g_{23} \\
g_{21} & g_{22} & g_{23} \\
g_{31} & g_{32} & g_{33}
) = 
\mqty(
1 & 0 & 0 \\
0 & \rho^{2} & 0 \\
0 & 0 & 1
)
\end{align*}
\end{frame}
\begin{frame}
\frametitle{Símbolos de Christoffel}
Como ya definimos que
\begin{align*}
x^{1} = \rho \hspace{1cm} x^{2} = \phi \hspace{1cm} x^{3} = z
\end{align*}
\pause
Además ya calculamos los elementos:
\begin{align*}
g_{11} = 1 \hspace{1cm} g_{22} = \rho^{2} \hspace{1cm} g_{33} = 1
\end{align*}
\end{frame}
\begin{frame}
\frametitle{Resultado necesario}
Vamos a requerir un resultado que se expresa a partir de los $g_{rq}$:
\begin{align*}
g^{p \, q} \, g_{r \, q} = \delta_{\, r}^{\, p}
\end{align*}
\pause
Que en términos del tensor \emph{covariante}, el tensor $g^{pq}$ es un tensor \emph{contravariante} también de segundo rango. Este resultado se puede demostrar pero aquí lo ocuparemos para los símbolos de Christoffel.
\end{frame}
\begin{frame}
\frametitle{Símbolos de Christoffel}
Entonces los símbolos de Christoffel no nulos, se presentan cuando $p = 2$, los $g_{11}$ y $g_{33}$ al ser constantes, al momento de diferenciar se anularán.
\pause
De tal manera que:
\end{frame}
\begin{frame}
\frametitle{Símbolos de Christoffel}
\begin{align*}
\mathlarger{\Gamma}_{2 \, 2}^{1} &= - \dfrac{1}{2 \, g_{11}} \, \pdv{g_{22}}{x^{1}} = \\[0.5em]
&= - \dfrac{1}{2} \, \pdv{\rho} \left( \rho^{2} \right) = \\[0.em]
&= - \rho
\end{align*}
\end{frame}
\begin{frame}
\frametitle{Símbolos de Christoffel}
\begin{align*}
\mathlarger{\Gamma}_{2 \, 1}^{2} = \mathlarger{\Gamma}_{1 \, 2}^{2}&=  \dfrac{1}{2 \, g_{22}} \, \pdv{g_{22}}{x^{1}} = \\[0.5em]
&= - \dfrac{1}{2 \, \rho^{2}} \, \pdv{\rho} \left( \rho^{2} \right) = \\[0.em]
&= \dfrac{1}{\rho}
\end{align*}
\end{frame}
\begin{frame}
\frametitle{Siguente paso}
Con esta breve guía podrás recuperar los símbolos de Christoffel para el sistema de coordenadas parabólico del material adicional.
\end{frame}
\end{document}


