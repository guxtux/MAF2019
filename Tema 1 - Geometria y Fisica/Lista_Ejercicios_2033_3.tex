\documentclass[12pt]{beamer}
\usepackage{../Estilos/BeamerMAF}
\usepackage{../Estilos/ColoresLatex}
\input{../Preambulos/preambulo_Beamer_Warsaw_seahorse}

\title{\large{Evaluación Semanal Tema 1}}
\subtitle{Tema 1 - La física y la geometría}

\author{M. en C. Gustavo Contreras Mayén}

\date{}

\begin{document}
\maketitle
\fontsize{14}{14}\selectfont
\spanishdecimal{.}

\section*{Contenido}
\frame[allowframebreaks]{\tableofcontents[currentsection, hideallsubsections]}

\section{Evaluación Tema 1}
\frame{\tableofcontents[currentsection, hideothersubsections]}
\subsection{Enunciado 1}

\begin{frame}
\frametitle{Ejercicio 1}
Un cierto campo de fuerza está dado por:
\begin{align*}
\vb{F} = \vu{r}\, \dfrac{2 P \cos \theta}{r^{3}} + \vu{\theta} \, \dfrac{P}{r^{3}} \sin \theta, \hspace{1.5cm} r \geq \dfrac{P}{2}
\end{align*}
en coordenadas esféricas polares.
\end{frame}
\begin{frame}
\frametitle{Por resolver}
\setbeamercolor{item projected}{bg=ao,fg=white}
\setbeamertemplate{enumerate items}{%
\usebeamercolor[bg]{item projected}%
\raisebox{1.5pt}{\colorbox{bg}{\color{fg}\footnotesize\insertenumlabel}}%
}
\begin{enumerate}[<+->]
\conti
\item Revisa $\curl{\vb{F}}$ para determinar si existe un potencial.
\item Calcular $\scaleoint{6ex} \vb{F} \cdot \dd{\bm{\lambda}}$ para un círculo unitario en el plano $\theta = \pi/2$. ¿Qué indica de que la fuerza sea conservativa o no conservativa?
\item Si consideras que $\vb{F}$ se puede describir por $\vb{F} = - \grad{\psi}$, encuentra $\psi$. De otra manera establece que no es existe un potencial aceptable.
\end{enumerate}
\end{frame}

\subsection{Enunciado 2}

\begin{frame}
\frametitle{Por resolver}
%Ref. Kusse (2006) Chap.3 Exercises 11
Con el sistema de coordenadas toroidal $(\xi, \eta, \varphi)$:
\setbeamercolor{item projected}{bg=ao,fg=bananayellow}
\setbeamertemplate{enumerate items}{%
\usebeamercolor[bg]{item projected}%
\raisebox{1.5pt}{\colorbox{bg}{\color{fg}\footnotesize\insertenumlabel}}%
}
\begin{enumerate}[<+->]
\item Describe las superficies constantes para $\xi$ y $\eta$.
\item Selecciona un punto y describe los vectores base. ¿Este es un sistema derecho?
\item Obtén de manera explícita los factores de escala.
\seti
\end{enumerate}
\end{frame}
\begin{frame}
\frametitle{Por resolver}
%Ref. Kusse (2006) Chap.3 Exercises 11
Con el sistema de coordenadas toroidal $(\xi, \eta, \varphi)$:
\setbeamercolor{item projected}{bg=ao,fg=bananayellow}
\setbeamertemplate{enumerate items}{%
\usebeamercolor[bg]{item projected}%
\raisebox{1.5pt}{\colorbox{bg}{\color{fg}\footnotesize\insertenumlabel}}%
}
\begin{enumerate}[<+->]
\conti
\item Determina los vectores de posición y de desplazamiento.
\item Desarrolla las expresiones para el gradiente $\grad{\Phi}$, la divergencia \hfill \break $\divergence{\vb{A}}$, y el rotacional $\curl{\vb{A}}$.
\item Escribe la ecuación de Helmholtz en este sistema toroidal.
\end{enumerate}
\end{frame}

\subsection{Enunciado 3}

\begin{frame}
\frametitle{Ejercicio 3}
Usando las propiedades de las funciones Gamma y Beta, evalúa las siguientes integrales:
\setbeamercolor{item projected}{bg=black,fg=white}
\setbeamertemplate{enumerate items}{%
\usebeamercolor[bg]{item projected}%
\raisebox{1.5pt}{\colorbox{bg}{\color{fg}\footnotesize\insertenumlabel}}%
}
\begin{enumerate}[<+->]
\item $\scaleint{6ex}_{\bs 0}^{1} \sqrt{x (1 - x)} \dd{x}$
\item $\scaleint{6ex}_{\bs 0}^{1} x^{4} \, (1 - x^{2})^{-\frac{1}{2}} \dd{x}$
\end{enumerate}
\end{frame}
\end{document}