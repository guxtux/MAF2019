\documentclass[12pt]{beamer}
\usepackage{../Estilos/BeamerMAF}
\input{../Preambulos/preambulo_Beamer_Warsaw_seahorse}
\makeatletter
% \setbeamercolor{section in foot}{bg=gray!30, fg=black!90!orange}
% \setbeamercolor{subsection in foot}{bg=blue!30!yellow, fg=red}
% \setbeamercolor{date in foot}{bg=black, fg=white}
\setbeamertemplate{footline}
{
  \leavevmode%
  \hbox{%
  \begin{beamercolorbox}[wd=.333333\paperwidth,ht=2.25ex,dp=1ex,center]{section in foot}%
    \usebeamerfont{section in foot} \insertsection
  \end{beamercolorbox}%
  \begin{beamercolorbox}[wd=.333333\paperwidth,ht=2.25ex,dp=1ex,center]{subsection in foot}%
    \usebeamerfont{subsection in foot}  \insertsubsection
  \end{beamercolorbox}%
  \begin{beamercolorbox}[wd=.333333\paperwidth,ht=2.25ex,dp=1ex,right]{date in head/foot}%
    \usebeamerfont{date in head/foot} \insertshortdate{} \hspace*{2em}
    \insertframenumber{} / \inserttotalframenumber \hspace*{2ex} 
  \end{beamercolorbox}}%
  \vskip0pt%
}
\makeatother
\date{15 de febrero de 2022}
\title{La física y la geometría}
\subtitle{Objetivos del Tema 1}

\begin{document}
\maketitle
\fontsize{14}{14}\selectfont
\spanishdecimal{.}

\section*{Contenido}
\frame[allowframebreaks]{\frametitle{Contenido} \tableofcontents[currentsection, hideallsubsections]}


\section{Introducción}
\frame[allowframebreaks]{\tableofcontents[currentsection, hideothersubsections]}
\subsection{Sistemas conocidos}

\begin{frame}
\frametitle{Introducción}
Estamos familiarizados con el uso de distintos sistemas coordenados para describir problemas físicos.
\\
\bigskip
\pause
Hemos usado coordenadas cartesianas, cilíndricas y esféricas para problemas con esas simetrías.
\end{frame}
\begin{frame}
\frametitle{Otros sistemas}
Veremos que existen otros sistemas coordenados que poco a poco durante la carrera, nos encontraremos con ellos para problemas en particular.
\\
\bigskip
\pause
El Tema 1 nos brindará una estrategia de trabajo con ellos.
\end{frame}

% \section{Ejemplos}
% \frame{\tableofcontents[currentsection, hideothersubsections]}
% \subsection{Dos ejemplos}

\begin{frame}
\frametitle{Primer ejemplo}
Consideremos el vector campo eléctrico $(\va{E})$ creado por una carga puntual $q$ localizada en el origen de un sistema cartesiano. \pause Usando los vectores de la base canónica en $\mathbb{R}^{3}$, el campo es:
\pause
\begin{align*}
\va{E} = \dfrac{q}{4 \pi \varepsilon_{0}} \, \dfrac{x \, \vu{e}_{x} + y \, \vu{e}_{y} + z \, \vu{e}_{z}}{\left( x^{2} + y^{2} + z^{2} \right)^{3/2}}
\end{align*}
\end{frame}
\begin{frame}
\frametitle{Primer ejemplo}
En coordenadas esféricas $(r, \theta, \phi)$, en donde se aprovecha completamente la simetría del problema para este campo, se simplifica la expresión anterior, \pause así el campo eléctrico resulta ser:
\begin{align*}
\va{\vb{E}} = \dfrac{q}{4 \pi \varepsilon_{0}} \, \dfrac{\vu{e}_{r}}{r^{2}}
\end{align*}
\end{frame}
\begin{frame}
\frametitle{¿Cómo cambiamos de un sistema a otro?}
Será necesario utilizar conceptos como reglas de transformación, factores de escala, vectores unitarios, así como representar en el nuevo sistema coordenado, el conjunto de operadores diferenciales que ya conocemos: gradiente, divergencia y rotacional.
\end{frame}
\begin{frame}
\frametitle{Segundo ejemplo}
Ahora consideramos un problema con ondas estacionarias en un sistema bidimensional con simetría circular: una membrana circular delgada y perfectamente flexible, por ejemplo: una membrana de tambor circular idealizada de radio $r$.
\pause
\begin{figure}[H]
  \centering
  \includegraphics[scale=0.75]{Imagenes/Tambor.png}
\end{figure}
\end{frame}
\begin{frame}
\frametitle{Segundo ejemplo}
La ecuación de onda en coordenadas bidimensionales cilíndricas $(x , y \rightarrow r, \varphi)$ para la amplitud de desplazamiento, $\psi (r, \varphi, t)$ viene dada por:
\pause
\begin{align*}
\laplacian \psi (r, \varphi, t) - \dfrac{1}{v^{2}} \pdv[2]{\psi (r, \varphi, t)}{t} = 0
\end{align*}
\end{frame}
\begin{frame}
\frametitle{Elegir el sistema coordenado}
Para la elección del sistema coordenado debemos de considerar aquel en donde la simetría del problema nos simplique el trabajo.
\\
\bigskip
\pause
En este caso, corresponde elegir el sistema coordenado cilíndrico.
\end{frame}
\begin{frame}
\frametitle{Geometría del sistema cilíndrico}
\begin{figure}[H]
  \centering
  \includegraphics[scale=0.7]{Imagenes/Coordenadas_Cilindricas_01.eps}
\end{figure}
\end{frame}
\begin{frame}
\frametitle{Reexpresar la ecuación de onda}
Una vez elegido el sistema coordenado, debemos de realizar lo necesario para reexpresar la ecuación diferencial del problema en el sistema.
\\
\bigskip
\pause
Habrá que \enquote{adaptar} el operador Laplaciano en este sistema.
\end{frame}
\begin{frame}
\frametitle{Solución a la ED obtenida}
Una vez que se realiza la reexpresión de la ED, el siguiente paso es resolverla.
\\
\bigskip
\pause
Para ello, ocuparemos lo que se revisará en el \emph{\textcolor{blue}{Tema 2 - Primeras técnicas de solución}}.
\end{frame}
\begin{frame}
\frametitle{Solución al problema}
Cuando veamos el tema de \emph{función de Bessel}, tomaremos de nuevo este ejercicio.
\\
\bigskip
\pause
La solución obtenida la podemos ocupar con otras herramientas computacionales para \emph{simular} el comportamiento de la membrana circular.
\end{frame}
\begin{frame}
\frametitle{Simulación computacional}
\begin{figure}[h!]
    \centering
    \includegraphics[scale=0.32]{Imagenes/Modos_Vibracion_Membrana_Circular_01.eps}
\end{figure}
\end{frame}
\begin{frame}
\frametitle{¿Y si tenemos un toroide?}
\begin{figure}
    \centering
    \includegraphics[scale=0.4]{Imagenes/Geometria_Toro.png}
\end{figure}
\end{frame}

\section{Sistemas coordenados}
\frame{\tableofcontents[currentsection, hideothersubsections]}
\subsection{Sistemas comunes}

\begin{frame}
\frametitle{Sistemas coordenados}
\setbeamercolor{item projected}{bg=blue!70!black,fg=yellow}
\setbeamertemplate{enumerate items}[circle]
\begin{enumerate}[<+->]
\item Cartesiano.
\item Esféricas polares.
\item Cilíndricas polares.
\item Cilíndricas parabólicas.
\item Cilíndricas elípticas.
\item Cilíndricas bipolares.
\item Esferoidales prolatas.
\item Esferoidales oblatas.
\seti
\end{enumerate}
\end{frame}
\begin{frame}
\frametitle{Sistemas coordenados}
\setbeamercolor{item projected}{bg=blue!70!black,fg=yellow}
\setbeamertemplate{enumerate items}[circle]
\begin{enumerate}[<+->]
\conti
\item Parabólicas.
\item Elipsoidal.
\item Toroidal.
\item Biesféricas.
\item Cónicas.
\item Elipsoidales confocales.
\item Parabólicas confocales.
\end{enumerate}
\end{frame}
\begin{frame}
\frametitle{Otros sistemas coordenados}
Cuando se realizan ciertas traslaciones en alguna dirección perpendicular en particular de los sistemas mencionados, o cuando se tienen superficies coordenadas que son planos paralelos, \pause se obtiene un conjunto particular de sistemas coordenados.
\end{frame}
\begin{frame}
\frametitle{Sistemas coordenados especiales}
\setbeamercolor{item projected}{bg=blue!70!black,fg=yellow}
\setbeamertemplate{enumerate items}[circle]
\begin{enumerate}[<+->]
\item Cilíndrico tangente.
\item Cilíndrico cardioide.
\item Cilíndrico hiperbólico.
\item Cilíndrico logarítmico.
\item Coordenadas Zeta.
\end{enumerate}
\end{frame}

\subsection{Ejercicio clásico}

\begin{frame}
\frametitle{La ecuación de Helmholtz}
La ecuación de Helmholtz se encuentra muy a menudo en la física:
\pause
\begin{align*}
\laplacian{\psi} + k \, \psi = 0
\end{align*}
\pause
Un buen ejercicio que nos brindará el desarrollo de habilidades trabajando con distintos sistemas coordenados, es expresar esta ecuación diferencial en cada uno de los sistemas coordenados que hemos mencionado.
\end{frame}
\begin{frame}
\frametitle{La ecuación de Helmholtz}
Con lo que veremos en el Tema 1 y al inicio del Tema 2, encontramos que la ecuación es \emph{separable} en 11 de los sistemas coordenados mencionados previamente.
\pause
Se obtendrá un sistema de ecuaciones diferenciales las cuales se podrán resolver \emph{más fácilmente} que en el sistema inicial.
\end{frame}

\subsection*{Ecuaciones lineales.}

\begin{frame}
\frametitle{Otras ecuaciones relevantes en la física}
Enumeremos algunas EDP conocidas:
\setbeamercolor{item projected}{bg=blue!70!black,fg=yellow}
\setbeamertemplate{enumerate items}[circle]
\begin{enumerate}[<+->]
\item La ecuación de calor:
\begin{align}
\text{\Large{$u_{t} - u_{xx} = 0$}}
\label{eq:ecuacion_S11}
\end{align}
\item La ecuación de onda:
\begin{align}
\text{\Large{$u_{tt} - u_{xx} = 0$}}
\label{eq:ecuacion_S12}
\end{align}
\item Ecuación de Laplace:
\begin{align}
\text{\Large{$u_{xx} + u_{yy} = 0$}}
\label{eq:ecuacion_S14}
\end{align}
\seti
\end{enumerate}
\end{frame}
\begin{frame}
\frametitle{Otras ecuaciones relevantes en la física}
\setbeamercolor{item projected}{bg=blue!70!black,fg=yellow}
\setbeamertemplate{enumerate items}[circle]
\begin{enumerate}[<+->]
\conti
\item Ecuación no lineal de calor:
\begin{align}
\pdv{u}{t} = \pdv{x} \left[ f(u) \, \pdv{u}{x} \right]
\label{eq:ecuacion_S27}  
\end{align}
\seti
\end{enumerate}
\end{frame}
\begin{frame}
\frametitle{Otras ecuaciones relevantes en la física}
\setbeamercolor{item projected}{bg=blue!70!black,fg=yellow}
\setbeamertemplate{enumerate items}[circle]
\begin{enumerate}[<+->]
\conti
\item Ecuación Kolmogorov-Petrovskii-Piskunov:
\begin{align}
\pdv{u}{t} = a \, \pdv[2]{w}{x} + f(u), \hspace{1cm} a > 0
\label{eq:ecuacion_S28}
\end{align}
Las ecuaciones de esta forma se encuentran a menudo en varios problemas de transferencia de masa y calor (siendo $f$ la reacción de cambio del volumen en una reacción química), teoría de la combustión, biología y ecología.
\seti
\end{enumerate}
\end{frame}
\begin{frame}
\frametitle{Otras ecuaciones relevantes en la física}
\setbeamercolor{item projected}{bg=blue!70!black,fg=yellow}
\setbeamertemplate{enumerate items}[circle]
\begin{enumerate}[<+->]
\conti
\item Ecuación de Burgers:
\begin{align}
\pdv{w}{t} + u \, \pdv{u}{x} = \pdv[2]{u}{x}
\label{eq:ecuacion_S29}
\end{align}
Se ocupa para describir procesos ondulatorios en la dinámica de gases, hidrodinámica, en acústica y para el flujo del tráfico.
\seti
\end{enumerate}
\end{frame}
\begin{frame}
\frametitle{Otras ecuaciones relevantes en la física}
\setbeamercolor{item projected}{bg=blue!70!black,fg=yellow}
\setbeamertemplate{enumerate items}[circle]
\begin{enumerate}[<+->]
\conti
\item Ecuación de onda no lineal:
\begin{align}
\pdv[2]{u}{t} = \pdv{x} \left[ f(u) \, \pdv{u}{x} \right]
\label{eq:ecuacion_S30}
\end{align}
Esta ecuación se encuentra en dinámica de ondas y gases, con $f(u) > 0$.
\end{enumerate}
\end{frame}

\section{Objetivos}
\frame{\tableofcontents[currentsection, hideothersubsections]}
\subsection{Tema 1}

\begin{frame}
\frametitle{Objetivos}
Al concluir el Tema 1, el alumno:
\setbeamercolor{item projected}{bg=blue!70!black,fg=yellow}
\setbeamertemplate{enumerate items}[circle]
\begin{enumerate}[<+->]
\item Describirá las superficies coordenadas a partir de las reglas de transformación entre el sistema cartesiano y otro sistema de estudio.
\item Determinará los factores de escala del nuevo sistema de estudio, así como los vectores base y la interpretación con los vectores cartesianos.
\seti
\end{enumerate}
\end{frame}
\begin{frame}
\frametitle{Objetivos}
\setbeamercolor{item projected}{bg=blue!70!black,fg=yellow}
\setbeamertemplate{enumerate items}[circle]
\begin{enumerate}[<+->]
\conti
\item Calculará los operadores diferenciales en el nuevo sistema de estudio.
\item Resolverá mediante las funciones Gamma y Beta ejercicios de la física.
\end{enumerate}
\end{frame}

\section{Trabajo asíncrono}
\frame{\tableofcontents[currentsection, hideothersubsections]}
\subsection{Materiales de trabajo}

\begin{frame}
\frametitle{Trabajo a realizar}
Como se revisó en la presentación del curso de MAF, el trabajo asíncrono por parte del alumno, consistirá en leer los materiales del curso en la plataforma Moodle, siendo los siguientes:
\end{frame}
\begin{frame}
\frametitle{Materiales de trabajo}
\setbeamercolor{item projected}{bg=blue!70!black,fg=yellow}
\setbeamertemplate{enumerate items}[circle]
\begin{enumerate}[<+->]
\item 1 - Sistema de coordenadas curvilíneas ortogonales.
\item 2 - Diferenciales y operadores diferenciales.
\item 3 - Funciones Gamma y Beta.
\end{enumerate}
\end{frame}
\begin{frame}
\frametitle{Trabajo por semana}
Recomendación para la revisión de materiales de trabajo por semana:
\setbeamercolor{item projected}{bg=blue!70!black,fg=yellow}
\setbeamertemplate{enumerate items}[circle]
\begin{enumerate}[<+->]
\item Semana 1: Material de trabajo 1.
\item Semana 2: Material de trabajo 2.
\item Semana 3: Material de trabajo 3.
\end{enumerate}
\end{frame}

\subsection{Ejercicios}

\begin{frame}
\frametitle{Listado de ejercicios}
La lista de $10$ ejercicios del Tema 1, se publicarán en la plataforma Moodle el jueves 9 de febrero.
\\
\bigskip
\pause
De esta manera, tendrán el tiempo necesario para resolverlos. \pause Considerando además de que podrán hacer consultas, aclaraciones, atención de dudas.
\end{frame}
\begin{frame}
\frametitle{Entrega de los ejercicios}
La entrega de los ejercicios resueltos será por Moodle, teniendo como fecha límite \textcolor{blue}{el $8$ de marzo a las 6 pm.}
\end{frame}

\subsection{Ejercicios para el Examen 1}

\begin{frame}
\frametitle{Enunciados del Examen Tema 1}
Con la finalidad de contar con el suficiente tiempo para resolver los ejercicios del Tema 1 para el examen intermedio, el día martes 22 de febrero se revisará en clase el listado de ejercicios a resolver.
\end{frame}

\subsection{Materiales complementarios}

\begin{frame}
\frametitle{Materiales complementarios}
Tendrán disponibles materiales complementarios, que como mencionamos en la presentación del curso, les serán de utilidad ya que cuentan con un enfoque elevado, pero lo necesario para una consulta.
\\
\bigskip
Siendo también un atractivo para que extiendan la consulta en las referencias bibliográficas del curso.
\end{frame}
\begin{frame}
\frametitle{Materiales complementarios}
\begin{figure}
  \centering
  \includegraphics<1>[scale=0.3]{Imagenes/Material_Ley.png}
  \includegraphics<2>[scale=0.25]{Imagenes/Material_Morse.png}
  \includegraphics<3>[scale=0.4]{Imagenes/Material_Nguyen.png}
\end{figure}
\end{frame}

\section{Sesiones síncronas}
\frame{\tableofcontents[currentsection, hideothersubsections]}

\subsection{Programa de videoconferencias}

\begin{frame}
\frametitle{Sesiones síncronas}
Hemos programado sesiones los días martes y jueves en un horario de 3:30 a 4:30 pm.
\\
\bigskip
\pause
En estas sesiones se trabajarán algunos ejercicios para complementar los materiales de trabajo, y se dará espacio para dudas, comentarios, etc. tanto de las presentaciones como de los ejercicios.
\end{frame}
\begin{frame}
\frametitle{Calendario de reuniones}
\setbeamercolor{item projected}{bg=blue!70!black,fg=yellow}
\setbeamertemplate{enumerate items}[circle]
\begin{enumerate}[<+->]
\item Martes 15 de febrero.
\item Jueves 17 de febrero.
\item Martes 22 de febrero.
\item Jueves 24 de febrero.
\item Martes 1 de marzo.
\end{enumerate}
\pause
Se les enviará por correo el enlace y la contraseña oportunamente.
\end{frame}
\end{document}