\documentclass[12pt]{article}
\usepackage[utf8]{inputenc}
\usepackage[spanish,es-lcroman, es-tabla]{babel}
\usepackage[autostyle,spanish=mexican]{csquotes}
\usepackage{amsmath}
\usepackage{amssymb}
\usepackage{nccmath}
\numberwithin{equation}{section}
\usepackage{amsthm}
\usepackage{graphicx}
\usepackage{epstopdf}
\DeclareGraphicsExtensions{.pdf,.png,.jpg,.eps}
\usepackage{color}
\usepackage{float}
\usepackage{multicol}
\usepackage{enumerate}
\usepackage[shortlabels]{enumitem}
\usepackage{anyfontsize}
\usepackage{anysize}
\usepackage{array}
\usepackage{multirow}
\usepackage{enumitem}
\usepackage{cancel}
\usepackage{tikz}
\usepackage{circuitikz}
\usepackage{tikz-3dplot}
\usetikzlibrary{babel}
\usepackage{bm}
\usepackage{mathtools}
\usepackage{esvect}
\usepackage{hyperref}
\usepackage{relsize}
\usepackage{siunitx}
\usepackage{physics}
%\usepackage{biblatex}
\usepackage{standalone}
\usepackage{mathrsfs}
\usepackage{bigints}
\usepackage{bookmark}
\spanishdecimal{.}

\setlist[enumerate]{itemsep=0mm}

\renewcommand{\baselinestretch}{1.5}

\let\oldbibliography\thebibliography

\renewcommand{\thebibliography}[1]{\oldbibliography{#1}

\setlength{\itemsep}{0pt}}
%\marginsize{1.5cm}{1.5cm}{2cm}{2cm}


\newtheorem{defi}{{\it Definición}}[section]
\newtheorem{teo}{{\it Teorema}}[section]
\newtheorem{ejemplo}{{\it Ejemplo}}[section]
\newtheorem{propiedad}{{\it Propiedad}}[section]
\newtheorem{lema}{{\it Lema}}[section]

\author{}
\marginsize{1cm}{1cm}{1cm}{1cm} 
\title{Sistemas de Coordenadas Especiales \\ {\large Matemáticas Avanzadas de la Física}}
\date{ }
\begin{document}
\renewcommand\labelenumii{\theenumi.{\arabic{enumii}}}
\maketitle
\fontsize{14}{14}\selectfont
\vspace{-2cm}
A continuación se presenta una lista de sistemas coordenados, clasificados de acuerdo con el hecho de que tengas o no un eje de traslación (perpendicular a la familia de superficies de plano paralelo) o un ejer de simetría rotacional.
\par
\begin{table}[H]
\fontsize{14}{14}\selectfont
\centering
\begin{tabular}{p{5cm} p{6cm} p{5cm}}
Eje de traslación & Eje de rotación & Ninguno \\ \hline
Cartesiano ($3$ ejes) & & Confocal elipsoidal \\[1em]
Circular cilíndrico & Circular cilíndrico & \\
 & Polar esferoidal ($3$ ejes) & \\[1em]
 Elíptico cilíndrico & Esferoidal prolato o alargado & \\
  & Esferoidal oblato & \\[1em]
Parabólico cilíndrico & Parabólico & \\[1em]
Bipolar & Toroidal & \\
 & Biesférico & \\[1em]
  & & Cónico \\
  & & Confocal paraboidal \\
\end{tabular}
\end{table}
El espaciamiento en la tabla indica las relaciones entre los diversos sistemas coordenados. Si se considera el caso en dos dimensiones ($z = 0$) de un sistema con un eje de traslación (columna izquierda) y se le hace girar alrededor de un eje de simentría de reflexión, se genera el sistema coordenado correspondiente indicado en la columna central hacia la derecha.
\par
Por ejemplo: la rotación del plano $z = 0$ del sistema cilíndrico elíptico alrededor del eje principal genera el sistema esferoidal alargado; la rotación alrededor del eje menor resulta en el sistema esferoidal oblato.
\par
También se consideran tres sistemas que no tienen eje de traslación o eje de rotación. En este grupo asimétrico, el sistema elipsoidal confocal se utiliza algunas veces como el sistema más general y del que casi todos los demás sistemas se obtienen del mismo.
\end{document}