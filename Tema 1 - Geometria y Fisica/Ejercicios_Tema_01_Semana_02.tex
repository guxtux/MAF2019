\documentclass[12pt]{article}
\usepackage[left=0.25cm,top=1cm,right=0.25cm,bottom=1cm]{geometry}
\textwidth = 20cm
\hoffset = -1cm
\usepackage[utf8]{inputenc}
\usepackage[spanish,es-tabla]{babel}
\usepackage[autostyle,spanish=mexican]{csquotes}
\usepackage[tbtags]{amsmath}
\usepackage{nccmath}
\usepackage{amsthm}
\usepackage{amssymb}
\usepackage{graphicx}
\usepackage{standalone}
\usepackage[outdir=./]{epstopdf}
\usepackage{siunitx}
\usepackage{physics}
\usepackage{color}
\usepackage{float}
\usepackage{multicol}
%\usepackage{milista}
\usepackage{enumitem}
\usepackage{anyfontsize}
\usepackage{anysize}
\usepackage{enumitem}
\usepackage{capt-of}
\usepackage{bm}
\usepackage{relsize}
\usepackage{placeins}
\usepackage{empheq}
\usepackage{cancel}
\usepackage{wrapfig}
\spanishdecimal{.}
\renewcommand{\baselinestretch}{1.5} 
\renewcommand\labelenumii{\theenumi.{\arabic{enumii}}}
\newcommand{\ptilde}[1]{\ensuremath{{#1}^{\prime}}}
\newcommand{\stilde}[1]{\ensuremath{{#1}^{\prime \prime}}}
\newcommand{\ttilde}[1]{\ensuremath{{#1}^{\prime \prime \prime}}}
\newcommand{\ntilde}[2]{\ensuremath{{#1}^{(#2)}}}


\title{Ejercicios Semana 2 del Tema 1\\ \large{Matemáticas Avanzadas de la Física}\vspace{-3ex}}
\author{M. en C. Gustavo Contreras Mayén}
\date{ }
\begin{document}
\vspace{-4cm}
\maketitle
\fontsize{14}{14}\selectfont
Presentación 2.
\begin{enumerate}
\item Si $f = f(r)$ con $r = \sqrt{x^{2} + y^{2}+ z^{2}}$, demuestra que
\begin{align*}
\nabla{f(r)} = \vu{r} \, \dv{f(r)}{r}
\end{align*}
\item Demuestra que el campo eléctrico de una carga puntal
\begin{align*}
\vb{E} = \dfrac{q \, \vu{r}}{4 \, \pi \epsilon_{0} \, r^{2}}
\end{align*}
cumple $\div{E} = 0$, para $r \neq 0$.
\item La ley de Gauss para el campo eléctrico tiene la forma:
\begin{align*}
\oint \vb{E} \cdot \dd{\vb{S}} = \dfrac{q}{\epsilon_{0}}
\end{align*}
donde $q = \displaystyle \int \rho \dd{V}$ es la carga encerrada en la superficie y $\rho$ su densidad volumétrica.
\\
\bigskip
Demuestra la ley de Gauss en forma diferencial
\begin{align*}
\div{E} = \dfrac{\rho}{\epsilon_{0}}
\end{align*}
\item Demuestra que:
\begin{align*}
\curl(\phi \, \vb{A}) = \phi \, \curl{\vb{A}} + \grad{\phi} \times \vb{A}
\end{align*}
\item El campo electrostático de un dipolo eléctrico $\vb{p} = p_{0} \, \vu{e}_{z}$ es
\begin{align*}
\vb{E} = \dfrac{p_{0} (2 \, \vb{e}_{r} \, \cos \theta + \vu{e}_{\theta} \, \sin \theta)}{r^{3}}
\end{align*}
Demuestra que:
\begin{enumerate}
\item $\curl{\vb{E}} = 0$
\item para $r \neq 0$, se tiene $\div{\vb{E}} = 0$
\end{enumerate}
\item Demuestra que el Laplaciano en coordenadas cilíndricas y esféricas es el que se presenta, para ello tendrás que calcular los respectivos factores de escala.
\end{enumerate}
Presentación 3.
\begin{enumerate}
\item Para el sistema de coordenadas esferoidales prolatas $(\xi, \eta, \phi)$, cuyas reglas de transformación son:
\begin{align*}
x &= a \: \sinh \xi \: \sin \eta \: \cos \phi\\
y &= a \: \sinh \xi \: \sin \eta \: \sin \phi\\
z &= a \: \cosh \xi \: \cos \eta
\end{align*}
\begin{enumerate}
\item Describe las superficies coordenadas del sistema.
\item Calcula de manera explícita los factores de escala $(h_{\xi}, h_{\eta}, h_{\phi})$.
\end{enumerate}
\item Ocupando el mismo el sistema de coordenadas esferoidales prolatas $(\xi, \eta, \phi)$ del ejercicio a cuenta anterior:
\begin{enumerate}
\item Calcula los operadores diferenciales $\grad{\phi}$, $\div{\vb{B}}$, $\curl{\vb{B}}$ y $\laplacian{\phi}$
\end{enumerate}
\end{enumerate}
\end{document}