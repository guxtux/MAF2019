\documentclass[12pt]{article}
\usepackage[left=0.25cm,top=1cm,right=0.25cm,bottom=1cm]{geometry}
\textwidth = 20cm
\hoffset = -1cm
\usepackage[utf8]{inputenc}
\usepackage[spanish,es-tabla]{babel}
\usepackage[autostyle,spanish=mexican]{csquotes}
\usepackage[tbtags]{amsmath}
\usepackage{nccmath}
\usepackage{amsthm}
\usepackage{amssymb}
\usepackage{graphicx}
\usepackage{standalone}
\usepackage[outdir=./]{epstopdf}
\usepackage{siunitx}
\usepackage{physics}
\usepackage{color}
\usepackage{float}
\usepackage{multicol}
%\usepackage{milista}
\usepackage{enumitem}
\usepackage{anyfontsize}
\usepackage{anysize}
\usepackage{enumitem}
\usepackage{capt-of}
\usepackage{bm}
\usepackage{relsize}
\usepackage{placeins}
\usepackage{empheq}
\usepackage{cancel}
\usepackage{wrapfig}
\spanishdecimal{.}
\renewcommand{\baselinestretch}{1.5} 
\renewcommand\labelenumii{\theenumi.{\arabic{enumii}}}
\newcommand{\ptilde}[1]{\ensuremath{{#1}^{\prime}}}
\newcommand{\stilde}[1]{\ensuremath{{#1}^{\prime \prime}}}
\newcommand{\ttilde}[1]{\ensuremath{{#1}^{\prime \prime \prime}}}
\newcommand{\ntilde}[2]{\ensuremath{{#1}^{(#2)}}}


\usetikzlibrary{babel}
\setlength{\tabcolsep}{12pt}
\title{Guía para la entrega de tareas y exámenes\\ \large{Curso Matemáticas Avanzadas de la Física}\vspace{-3ex}}
\author{M. en C. Gustavo Contreras Mayén}
\date{ }
\begin{document}
\vspace{-4cm}
\maketitle
\fontsize{14}{14}\selectfont
\begin{enumerate}
\item El envío de los ejercicios resueltos es individual. Podrán crear grupos de discusión y revisión, pero la redacción y desarrollo del ejercicio debe de ser propio del estudiante.
\item Al reconocer respuestas y/o soluciones idénticas en dos o más entregas, el puntaje obtenido en el ejercicio se dividirá entre el total de copias, esto como primer aviso para evitar las copias. En una segunda ocasión, \emph{se cancelará la tarea completa} de los involucrados.
\item Los ejercicios deberán de entregarse lo más detallados posible, de preferencia escritos con bolígrafo o con un trazo firme de lápiz. Esto nos facilitará la lectura y revisión.
\item Una vez que tengan las hojas necesarias con la solución de la tarea o examen, en su caso, deberán de digitalizar las mismas con un escáner y generar un archivo pdf.
\item En caso de no contar con un escáner, deberán de tomar una foto con la cámara ya sea de un teléfono celular o de una tableta. Con las fotos ordenadas deberán de incluirlas en un documento con un procesador de textos (Word, LibreOffice, Google Docs, etc.) y generar un archivo pdf.
\item Al inicio de su documento ya sea con escáner o con las imágenes en el procesador de textos, debe de incluir una hoja con su nombre completo.
\item El nombre del archivo pdf debe llevar la siguiente estructura: \hfill \break 
\hspace*{2cm} \texttt{PrimerApellido\_SegundoApellido\_Tarea1.pdf}
\\
Ejemplo: \texttt{Cervantes\_Saavedra\_Tarea1.pdf}
\item \label{inciso_tamano} El envío del archivo se realizará en Moodle, tomando en cuenta que el tamaño del archivo que está permitido es de $10$ MB, en caso de que su archivo pdf exceda este tamaño, deberán de subir su archivo a su espacio en Drive (con la cuenta de \texttt{@ciencias} tienen un espacio considerable) y entonces enviar por correo la liga del archivo compartido al equipo académico.
\item Dentro de Moodle se configura con fecha y hora para el envío del archivo, pasada la hora límite, la plataforma deshabilita el botón de envío por lo que no podrán subir su archivo. \textbf{Recomendamos que se anticipen a la hora límite de entrega: las 18:00 pm (6 de la tarde)}.
\item No se evaluarán tareas que hayan sido enviadas por correo, ya sea que se mandaron antes de la hora límite o posterior a la hora límite, el medio de recepción será la plataforma, siendo la única excepción lo que se indica en el inciso (\ref{inciso_tamano}).
\item Cada tarea se revisará y comentará indicando el puntaje obtenido, un ejercicio bien resuelto otorga un punto, en caso de tener un avance, se otorgará una parte proporcional del punto. Recomendamos ampliamente que resuelvan todos los ejercicios.
\item La evaluación y realimentación se les enviará por correo, la calificación se dejará en Moodle para que lleven su avance y tengan conocimiento en todo momento de las evaluaciones. A más tardar pasadas dos semanas luego de la entrega, se les entregarán los comentarios, pudiendo recibirlos antes de este plazo.
\item Contarán con tres días hábiles luego de haber recibido la evaluación y comentarios, para solicitar una aclaración sobre su calificación. Como se pide que los ejercicios sean lo más detallados y legibles posibles, en caso de que se haya omitido algún elemento en la revisión, al tener su tarea o examen a mano, apoyarán de esta manera su aclaración. Pasados los tres días hábiles, ya no se aceptarán revisiones de tareas o exámenes.
\item Recuerden que cuentan con la libertad de solicitar una consulta o asesoría, ya sea por correo electrónico o en su momento, mediante Telegram. Para ello, les pedimos que de manera anticipada lo comenten tanto con Abraham Lima o con Luis Estrada. El horario de atención es de lunes a viernes en el horario de 9 am a 7 pm.
\end{enumerate}

\noindent
Equipo académico:
\begin{table}[H]
\hspace{-0.5cm}
\large
\begin{tabular}{l l}
M. en C. Abraham Lima Buendía. & \texttt{abraham3081@ciencias.unam.mx} \\
M. en C. Luis Odín Estrada Ramos & \texttt{physichess@gmail.com} \\
M. en C. Gustavo Contreras Mayén & \texttt{gux7avo@ciencias.unam.mx}
\end{tabular}
\end{table}

\end{document}