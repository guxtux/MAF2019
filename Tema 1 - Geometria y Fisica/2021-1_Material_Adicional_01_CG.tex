\documentclass[hidelinks,12pt]{article}
\usepackage[left=0.25cm,top=1cm,right=0.25cm,bottom=1cm]{geometry}
%\usepackage[landscape]{geometry}
\textwidth = 20cm
\hoffset = -1cm
\usepackage[utf8]{inputenc}
\usepackage[spanish,es-tabla]{babel}
\usepackage[autostyle,spanish=mexican]{csquotes}
\usepackage[tbtags]{amsmath}
\usepackage{nccmath}
\usepackage{amsthm}
\usepackage{amssymb}
\usepackage{mathrsfs}
\usepackage{graphicx}
\usepackage{subfig}
\usepackage{standalone}
\usepackage[outdir=./Imagenes/]{epstopdf}
\usepackage{siunitx}
\usepackage{physics}
\usepackage{color}
\usepackage{float}
\usepackage{hyperref}
\usepackage{multicol}
%\usepackage{milista}
\usepackage{anyfontsize}
\usepackage{anysize}
%\usepackage{enumerate}
\usepackage[shortlabels]{enumitem}
\usepackage{capt-of}
\usepackage{bm}
\usepackage{relsize}
\usepackage{placeins}
\usepackage{empheq}
\usepackage{cancel}
\usepackage{wrapfig}
\usepackage[flushleft]{threeparttable}
\usepackage{makecell}
\usepackage{fancyhdr}
\usepackage{tikz}
\usepackage{bigints}
\usepackage{scalerel}
\usepackage{pgfplots}
\usepackage{pdflscape}
\pgfplotsset{compat=1.16}
\spanishdecimal{.}
\renewcommand{\baselinestretch}{1.5} 
\renewcommand\labelenumii{\theenumi.{\arabic{enumii}})}
\newcommand{\ptilde}[1]{\ensuremath{{#1}^{\prime}}}
\newcommand{\stilde}[1]{\ensuremath{{#1}^{\prime \prime}}}
\newcommand{\ttilde}[1]{\ensuremath{{#1}^{\prime \prime \prime}}}
\newcommand{\ntilde}[2]{\ensuremath{{#1}^{(#2)}}}

\newtheorem{defi}{{\it Definición}}[section]
\newtheorem{teo}{{\it Teorema}}[section]
\newtheorem{ejemplo}{{\it Ejemplo}}[section]
\newtheorem{propiedad}{{\it Propiedad}}[section]
\newtheorem{lema}{{\it Lema}}[section]
\newtheorem{cor}{Corolario}
\newtheorem{ejer}{Ejercicio}[section]

\newlist{milista}{enumerate}{2}
\setlist[milista,1]{label=\arabic*)}
\setlist[milista,2]{label=\arabic{milistai}.\arabic*)}
\newlength{\depthofsumsign}
\setlength{\depthofsumsign}{\depthof{$\sum$}}
\newcommand{\nsum}[1][1.4]{% only for \displaystyle
    \mathop{%
        \raisebox
            {-#1\depthofsumsign+1\depthofsumsign}
            {\scalebox
                {#1}
                {$\displaystyle\sum$}%
            }
    }
}
\def\scaleint#1{\vcenter{\hbox{\scaleto[3ex]{\displaystyle\int}{#1}}}}
\def\bs{\mkern-12mu}


\title{Coordenadas generalizadas\\ \large{Matemáticas Avanzadas de la Física}\vspace{-3ex}}
\author{M. en C. Abraham Lima Buendía}
\date{ }
\begin{document}
\vspace{-4cm}
\maketitle
\fontsize{14}{14}\selectfont
Con la finalidad de realizar el estudio de diferentes problemas de física se requiere hacer uso de sistemas de coordenadas diferentes al sistema cartesiano; en este material abordaremos su construcción, cabe señalar que se hará uso de diferentes resultados, propios de un curso avanzado de geometría diferencial, sin  embargo, se explicará la construcción de ellos y si así lo deseas puedes revisar a profundidad los detalles en la bibliografía complementaria: los primeros cuatro capítulos tanto de \cite{Bar} como de \cite{Nguyen} ofrecen un panorama completo sobre la manera en que se se sustenta lo que en este material se revisa.
\section{Variedades topológicas.}

Una $\mathbf{M}-1$ variedad topológica puede ser definida como un espacio topológico, tal que, para cada punto $p$, elemento de un conjunto abierto $U$ en la variedad, se tiene un homeomorfismo a $\mathbb{R}^{M}$.
\par
Recordemos la definición de homeomorfismo, sean $X$ e $Y$ espacios topológicos y $\varphi$ una función de $X$ a $Y$, decimos que $\varphi$ es un homeomorfismo si $\varphi$ es biyectiva, además que tanto $\varphi$, como $\varphi^{-1}$ son funciones continuas. Diremos que, si dos espacios son homeomorfos, entonces tienen exactamente las mismas propiedades topológicas. Cuando $\varphi$ es un homeomorfismo diferenciable, se tiene que $\varphi$ también lo es, entonces diremos que $\varphi$ es un difeomorfismo.
\par
Cada homeomorfismo $(U, \varphi)$ es llamado una carta o sistema de coordenadas. Hasta este punto hemos conseguido pasar de un espacio topológico a un $\mathbb{R}^{M}$, sin embargo, extenderemos la idea para poder llevar cada sistema de coordenadas a un conjunto de funciones coordenadas $x_{1}, x_{2}, \ldots, x_{m}$, es decir, construimos una función $r_{j} : \mathbb{R}^{M} \rightarrow \mathbb{R}$ tal que: $x_{j} = r_{j} \circ \varphi_{p}$, en otras palabras, trasladamos los puntos de cada abierto en la variedad a un vector, a un espacio $\mathbb{R}^{M}$.
\par
El concepto de variedad topológica puede no tener una representación geométrica y estar ligado a un ser un espacio meramente abstracto, sin embargo, en el caso de una representación geométrica la composición $x_{j} = r_{j} \circ \varphi_{p}$ nos permite tener las funciones coordenadas en términos del conjunto de parámetros a través de los cuales se caracteriza la variedad:
\begin{align}
x^{j} = x^{j} (q_{1}, \ldots q_{n})
\label{eq:ecuacion_01_01}
\end{align}
al conjunto de parámetros, que caracterizan las variedades, se les denomina \emph{coordenadas generalizadas}.
\par
Ahora sabemos cómo construir un mapeo que permite estudiar el comportamiento de un espacio topológico, desde un espacio $\mathbb{R}^{M}$, de esa forma, el siguiente paso es construir el espacio tangente a una variedad, para ello hacemos uso del $\mathbb{R}^{M}$ con el que la hemos asociado.
\section{Mapeo de vectores.}

Mediante un homeomorfismo (o más comúnmente un difeomorfismo), se construye la posición de un punto en $\mathbb{R}^{M}$, para ello hacemos uso de la base canónica de este espacio:
\begin{align}
\va{r} = \sum_{j=1}^{M} x^{j} \, (\va{q}) \, \vu{x}_{j} = x^{j} \, (\va{q}) \, \vu{x}_{j} = x^{j} (q_{1}, \ldots q_{n}) \, \vu{x}_{j}
\label{eq:ecuacion_01_02}
\end{align}
La segunda igualdad se conoce como \emph{convención de Einstein}, en la cual la aparición de dos índices repetidos consecutivamente tiene el significado de una suma sobre ellos. Por otro lado, los vectores tangentes a $\va{e}_{a}$, se construyen mediante la igualdad:
\begin{align}
\va{e}_{a} = \pdv{\va{r}}{q^{a}}
\label{eq:ecuacion_01_03}
\end{align}
es importante señalar que los vectores de la ec.(\ref{eq:ecuacion_01_03}) no son necesariamente ortogonales, cuando el sistema de coordenadas satisface esta condición, se conoce como sistema de coordenadas ortogonales.
\par
Notemos que las ecs. (\ref{eq:ecuacion_01_03}) dependen de cada punto $p$ en la variedad, consecuentemente, cada punto definirá un plano tangente a la variedad, la unión formada por todos los planos tangentes, es conocida como \emph{haz tangente} a la variedad, la teoría de haces vectoriales queda fuera de los alcances de este curso, sin embargo, seguiremos mediante la combinación de las ecuaciones (\ref{eq:ecuacion_01_02}) y (\ref{eq:ecuacion_01_03}):
\begin{align}
\va{e}_{a} = \pdv{\va{r}}{q^{a}} = \pdv{x^{j} (q_{1}, \ldots q_{n})}{q^{a}} \, \vu{x}_{j}
\label{eq:ecuacion_01_04}
\end{align}
Notemos que la ec. (\ref{eq:ecuacion_01_04}) forma un sistema de ecuaciones entre la nueva base vectorial y una base canónica de $\mathbb{R}^{M}$, la condición suficiente y necesaria para que el sistema de ecuaciones pueda invertirse es:
\begin{align}
{\rm Det} \, \abs{\pdv{x^{j} (q_{1}, \ldots q_{n})}{q^{a}}} \neq 0
\label{eq:ecuacion_01_05}
\end{align}
La ec.(\ref{eq:ecuacion_01_05}), puede interpretarse de diversas maneras: la primera de ellas es consecuencia directa del \emph{teorema de la función inversa}, el hecho de que la matriz Jacobiana posea un determinante distinto de cero se deriva del difeomorfismo de clase $C1$, con el cual hemos mapeado la variedad diferencial a un espacio $\mathbb{R}^{M}$, otra interpretación consiste en ver la ec.(\ref{eq:ecuacion_01_05}) como el triple producto escalar de los vectores de la nueva base vectorial, es decir, nuestra condición para que los vectores de la ec. (\ref{eq:ecuacion_01_03})  formen una base y que esta pueda invertirse es que el espacio tangente a la variedad se componga por vectores no coplanares, un resultado conocido del curso de geometría analítica.
\par
El que podamos cambiar entre bases vectoriales, nos permite traducir cualquier vector en la variedad,
pero descrito desde el espacio canónico de $\mathbb{R}^{M}$, es decir, para cualquier vector $\va{A}$, tenemos la siguiente igualdad:
\begin{align}
\va{A} = A^{x} \, \vu{x} + A^{y} \, \vu{y} + A^{z} \, \vu{z} = A^{1} \, \vu{e}_{1} + A^{2} \, \vu{e}_{2} + A^{3} \, \vu{e}_{3}
\label{eq:ecuacion_01_06}
\end{align}
Usando al ec. (\ref{eq:ecuacion_01_06}) escribimos el producto punto entre dos vectores:
\begin{align}
\begin{aligned}
\va{A} \cdot \va{B} &= \left( A^{1} \, \vu{e}_{1} + A^{2} \, \vu{e}_{2} + A^{3} \, \vu{e}_{3} \right) \cdot \left( B^{1} \, \vu{e}_{1} + B^{2} \, \vu{e}_{2} + B^{3} \, \vu{e}_{3} \right) \\[0.5em]
&= A^{l} \, B^{m} \, g_{lm}
\end{aligned}
\label{eq:ecuacion_01_07}
\end{align}
En el extremo derecho de la ec. (\ref{eq:ecuacion_01_07}) se ha empleado la convención de Einstein sobre índices repetidos, los elementos de matriz $g_{lm}$, están determinados mediante la siguiente igualdad:
\begin{align}
\begin{aligned}[b]
g_{lm} &= \va{e}_{l} \cdot \va{e}_{m} = \\[0.5em]
&= \left( \pdv{x^{r} (q_{1} \ldots q_{n})}{q^{l}} \, \vu{x}_{r} \right) \cdot \left( \pdv{x^{s} (q_{1} \ldots q_{n})}{q^{m}} \, \vu{x}_{s} \right) = \\[0.5em]
&= \pdv{x^{r} (q_{1} \ldots q_{n})}{q^{l}} \, \pdv{x^{s} (q_{1} \ldots q_{n})}{q^{m}} \, \delta_{r,s} = \\[0.5em]
&= \pdv{x^{r} (q_{1} \ldots q_{n})}{q^{l}} \, \pdv{x_{r} (q_{1} \ldots q_{n})}{q^{m}} = \\[0.5em]
&= \pdv{x^{r}}{q^{l}} \, \pdv{x_{r}}{q^{m}}
\end{aligned}
\label{eq:ecuacion_01_08}
\end{align}
la ec.(\ref{eq:ecuacion_01_08}) es conocida, en la literatura, como \emph{matriz métrica}. Un caso particular de ésta ocurre cuando el sistema de coordenadas es ortogonal, es decir: 
\begin{align*}
\va{e}_{l} \cdot \va{e}_{m} = \delta_{l,m}
\end{align*}
pues en ella la matriz métrica se vuelve diagonal.
\section{Derivadas temporales.}

Finalizaremos esta revisión analizando algunas propiedades de la ec.(\ref{eq:ecuacion_01_03}) y la ec. (\ref{eq:ecuacion_01_04}), notemos que los vectores $\va{e}_{a}$ son funciones de las coordenadas generalizadas, así que para construir las derivadas temporales de cada uno de estos vectores $\va{e}_{a}$, debemos seguir el siguiente procedimiento:
\begin{align}
\begin{aligned}[b]
\dv{\va{e}_{a}}{t} &= \dv{t} \left( \pdv{x^{j} (q_{1}, \ldots, q_{n})}{q^{a}} \, \vu{x}_{j} \right) = \\[0.5em]
&= \dot{q}^{b} \, \pdv[2]{x^{j} (q_{1}, \ldots, q_{n})}{q^{b}}{q^{a}} \, \vu{x}_{j} = \\[0.5em]
&= \dot{q}^{b} \, \pdv{\va{e}_{a}}{q^{b}}
\end{aligned}
\label{eq:ecuacion_01_09}
\end{align}
La construcción del último resultado con la derivada, abarca dos temas complejos que son el \emph{transporte paralelo} y la \emph{teoría de conexiones en variedades diferenciables}; consideremos esta diferenciación en una variedad libre de torsión y usando una conexión de Levi Civita, en estas condiciones la última derivada satisface lo siguiente:
\begin{align}
\pdv{\va{e}_{a}}{q^{b}} = {\mathlarger{\mathlarger\Gamma}}_{ab}^{c} \, \va{e}_{c}
\label{eq:ecuacion_01_10}
\end{align}
El coeficiente ${\mathlarger{\mathlarger\Gamma}}_{ab}^{c}$ es conocido como \emph{símbolo de Christoffel}. Una interpretación simple de la ec. (\ref{eq:ecuacion_01_10}) consiste en descomponer el nuevo vector $\pdv*{\va{e}_{a}}{q^{b}}$, en términos de la base vectorial $\va{e}_{c}$, cada uno de los símbolos de Christoffel corresponderá a las componentes de dicho vector en la base mencionada.
\vfill
\begin{thebibliography}{X}
\bibitem{Bar}{\textsc{Bär, Christian}, \textit{Elementary differential geometry}, Cambridge, UK, 2010.}
\bibitem{Nguyen}{\textsc{Nguyen, Hung Schäfer} y \textsc{Schmidt, Jan-Philip}, \textit{Tensor analysis and elementary differential geometry for physicists and engineers}, 2nd. ed., Springer, Berlín, 2017.}
\end{thebibliography}
\end{document}