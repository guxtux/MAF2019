\input{../Preambulos/preambulo_presentacion_Warsaw_seahorse}
\title{\large{Tema 1 - Sistema de coordenadas curvilíneas}}
\subtitle{La física y la geometría}
\author{M. en C. Gustavo Contreras Mayén}
\date{\today}
\institute{Facultad de Ciencias - UNAM}
\titlegraphic{\includegraphics[width=1.75cm]{../Imagenes/escudo-facultad-ciencias}\hspace*{4.75cm}~%
   \includegraphics[width=1.75cm]{../Imagenes/escudo-unam}
}
\setbeamertemplate{navigation symbols}{}
\begin{document}
\maketitle
\fontsize{14}{14}\selectfont
\spanishdecimal{.}
\section*{Contenido}
\frame{\tableofcontents[currentsection, hideallsubsections]}
\section{Coordenadas curvilíneas}
\frame{\tableofcontents[currentsection, hideothersubsections]}
\subsection{Introducción}
\begin{frame}
\frametitle{Coordenadas curvilíneas}
La posición de un punto en el espacio tridimensional (con respecto a algún origen) generalmente se especifica al dar sus tres coordenadas cartesianas $(x, y, z)$ o, lo que es equivalente, al especificar el vector de posición $\vb{R}$ del punto.
\end{frame}
\begin{frame}
\frametitle{Coordenadas curvilíneas}
A menudo es más conveniente describir la posición del punto por otro conjunto de coordenadas más apropiadas para el problema en cuestión, los ejemplos comunes son coordenadas esféricas y cilíndricas.
\\
\bigskip
\pause
Estos son solo casos especiales de \emph{sistemas de coordenadas curvilíneas}, cuyas propiedades generales proponemos examinar en detalle.
\end{frame}
\end{document}