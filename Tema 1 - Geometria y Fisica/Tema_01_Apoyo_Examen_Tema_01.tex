\documentclass[hidelinks,12pt]{article}
\usepackage[left=0.25cm,top=1cm,right=0.25cm,bottom=1cm]{geometry}
%\usepackage[landscape]{geometry}
\textwidth = 20cm
\hoffset = -1cm
\usepackage[utf8]{inputenc}
\usepackage[spanish,es-tabla]{babel}
\usepackage[autostyle,spanish=mexican]{csquotes}
\usepackage[tbtags]{amsmath}
\usepackage{nccmath}
\usepackage{amsthm}
\usepackage{amssymb}
\usepackage{mathrsfs}
\usepackage{graphicx}
\usepackage{subfig}
\usepackage{standalone}
\usepackage[outdir=./Imagenes/]{epstopdf}
\usepackage{siunitx}
\usepackage{physics}
\usepackage{color}
\usepackage{float}
\usepackage{hyperref}
\usepackage{multicol}
%\usepackage{milista}
\usepackage{anyfontsize}
\usepackage{anysize}
%\usepackage{enumerate}
\usepackage[shortlabels]{enumitem}
\usepackage{capt-of}
\usepackage{bm}
\usepackage{relsize}
\usepackage{placeins}
\usepackage{empheq}
\usepackage{cancel}
\usepackage{wrapfig}
\usepackage[flushleft]{threeparttable}
\usepackage{makecell}
\usepackage{fancyhdr}
\usepackage{tikz}
\usepackage{bigints}
\usepackage{scalerel}
\usepackage{pgfplots}
\usepackage{pdflscape}
\pgfplotsset{compat=1.16}
\spanishdecimal{.}
\renewcommand{\baselinestretch}{1.5} 
\renewcommand\labelenumii{\theenumi.{\arabic{enumii}})}
\newcommand{\ptilde}[1]{\ensuremath{{#1}^{\prime}}}
\newcommand{\stilde}[1]{\ensuremath{{#1}^{\prime \prime}}}
\newcommand{\ttilde}[1]{\ensuremath{{#1}^{\prime \prime \prime}}}
\newcommand{\ntilde}[2]{\ensuremath{{#1}^{(#2)}}}

\newtheorem{defi}{{\it Definición}}[section]
\newtheorem{teo}{{\it Teorema}}[section]
\newtheorem{ejemplo}{{\it Ejemplo}}[section]
\newtheorem{propiedad}{{\it Propiedad}}[section]
\newtheorem{lema}{{\it Lema}}[section]
\newtheorem{cor}{Corolario}
\newtheorem{ejer}{Ejercicio}[section]

\newlist{milista}{enumerate}{2}
\setlist[milista,1]{label=\arabic*)}
\setlist[milista,2]{label=\arabic{milistai}.\arabic*)}
\newlength{\depthofsumsign}
\setlength{\depthofsumsign}{\depthof{$\sum$}}
\newcommand{\nsum}[1][1.4]{% only for \displaystyle
    \mathop{%
        \raisebox
            {-#1\depthofsumsign+1\depthofsumsign}
            {\scalebox
                {#1}
                {$\displaystyle\sum$}%
            }
    }
}
\def\scaleint#1{\vcenter{\hbox{\scaleto[3ex]{\displaystyle\int}{#1}}}}
\def\bs{\mkern-12mu}


\usepackage{titling}
\setlength{\droptitle}{-3cm}
\title{Apoyo para el Problema 9 \\[0.3em]  \large{Examen Tarea 1} \vspace{-3ex}}
\author{M. en C. Gustavo Contreras Mayén}
\date{ }

\begin{document}
\vspace{-4cm}
\maketitle
\fontsize{14}{14}\selectfont

El Problema 9 del Examen Tarea pide resolver la siguiente integral:
\begin{align*}
\scaleint{5ex}_{\bs 0}^{1} \dfrac{x^{5}}{\sqrt[3]{1 - x^{4}}} \dd{x}
\end{align*}

Que podemos expresarla como:
\begin{align*}
\scaleint{5ex}_{\bs 0}^{1} x^{5} \, (1 - x^{4})^{\frac{1}{3}} \dd{x}
\end{align*}

Una integral de este tipo la podemos ver como de manera general como:
\begin{align}
\int_{0}^{a} x^{b} \, (a^{c} - x^{c})^{d} \dd{x}
\label{eq:integral_cambio}
\end{align}
donde $a, b, c, d$ son constantes, tales que $a > 0$, $c > 0$, $b > -1$, $d > 1$, $b + c \, d > -1$. 

Recordando la integral de la función Beta:
\begin{align}
B(x, y) = \int_{0}^{1} t^{x-1} \, (1 -t)^{y-1} \dd{t} \hspace{1cm} x > 0, y > 0
\label{eq:funcion_Beta}
\end{align}

Consideramos que las integrales (\ref{eq:integral_cambio}) y (\ref{eq:funcion_Beta}) \enquote{se parecen}, por lo que se tendría que buscar un cambio de variable que nos conduzca a la integral de la función de Beta y luego ocupar alguna de las identidades. El cambio de variable es el punto crítico, por ello presentamos una ayuda. 
\par
El cambio de variable propuesto es: $x^{c} = a^{c} \, u$, por lo que la integral (\ref{eq:integral_cambio}) resulta:
\begin{align*}
\dfrac{1}{c} \, a^{b+cd+1} \,\int_{0}^{1} u^{\frac{b+1}{c}} \, (1 - u)^{d} \dd{u}
\end{align*}
Que ya tiene la forma de la integral Beta (\ref{eq:funcion_Beta}). Al hacer uso de la identidad de la función Beta que la relaciona con la función Gamma, se tiene que:
\begin{align}
\scaleint{5ex}_{\bs 0}^{a} x^{b} \, (a^{c} - x^{c})^{d} \dd{x} = \dfrac{a^{b+cd+1} \, \mathlarger{\Gamma} \left( \dfrac{b + 1}{c} \right) \, \Gamma (d + 1)}{(b + c \, d + 1) \, \Gamma \left( \dfrac{b + c \, d + 1}{c} \right)}
\label{eq:identidad_Gamma}
\end{align}

\end{document}