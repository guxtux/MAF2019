\documentclass[12pt]{article}
\usepackage[utf8]{inputenc}
\usepackage[spanish,es-lcroman, es-tabla]{babel}
\usepackage[autostyle,spanish=mexican]{csquotes}
\usepackage{amsmath}
\usepackage{amssymb}
\usepackage{nccmath}
\numberwithin{equation}{section}
\usepackage{amsthm}
\usepackage{graphicx}
\usepackage{epstopdf}
\DeclareGraphicsExtensions{.pdf,.png,.jpg,.eps}
\usepackage{color}
\usepackage{float}
\usepackage{multicol}
\usepackage{enumerate}
\usepackage[shortlabels]{enumitem}
\usepackage{anyfontsize}
\usepackage{anysize}
\usepackage{array}
\usepackage{multirow}
\usepackage{enumitem}
\usepackage{cancel}
\usepackage{tikz}
\usepackage{circuitikz}
\usepackage{tikz-3dplot}
\usetikzlibrary{babel}
\usetikzlibrary{shapes}
\usepackage{bm}
\usepackage{mathtools}
\usepackage{esvect}
\usepackage{hyperref}
\usepackage{relsize}
\usepackage{siunitx}
\usepackage{physics}
%\usepackage{biblatex}
\usepackage{standalone}
\usepackage{mathrsfs}
\usepackage{bigints}
\usepackage{bookmark}
\spanishdecimal{.}

\setlist[enumerate]{itemsep=0mm}

\renewcommand{\baselinestretch}{1.5}

\let\oldbibliography\thebibliography

\renewcommand{\thebibliography}[1]{\oldbibliography{#1}

\setlength{\itemsep}{0pt}}
%\marginsize{1.5cm}{1.5cm}{2cm}{2cm}


\newtheorem{defi}{{\it Definición}}[section]
\newtheorem{teo}{{\it Teorema}}[section]
\newtheorem{ejemplo}{{\it Ejemplo}}[section]
\newtheorem{propiedad}{{\it Propiedad}}[section]
\newtheorem{lema}{{\it Lema}}[section]

\usepackage{tikz-3dplot}
%\author{M. en C. Gustavo Contreras Mayén. \texttt{curso.fisica.comp@gmail.com}}
\title{Ecuaciones de la Física Matemática \\ {\large Matemáticas Avanzadas de la Física}\vspace{-1.5\baselineskip}}
\date{}
\author{}
\begin{document}
\maketitle
\fontsize{14}{14}\selectfont
\section{Ecuaciones de la Física Matemática.}
A continuación se menciona un conjunto de ecuaciones diferenciales de gran importancia en la física matemática:
\begin{enumerate}[label=\alph*)]
\item $\square \, u = 0$ o bien $\laplacian{u} = \dfrac{1}{c^{2}} \displaystyle \pdv[2]{u}{t}$ (ecuación de ondas)
\item $\laplacian{u} = \dfrac{1}{k} \displaystyle \pdv{u}{t}$ (ecuación de conducción del calor o de difusión)
\item $\laplacian{u} = f(x, y, z)$ (ecuación de Poisson; si $f=0$ se llama ecuación de Laplace)
\item $\laplacian{u} + \lambda \, u = 0$ (ecuación de Helmholtz)
\item $\nabla^{4} = \laplacian{(\laplacian{u})} = 0$ (ecuación biarmónica)
\item $\nabla^{4} = - \left( \dfrac{1}{p^{2}} \right) \displaystyle \pdv[2]{u}{t}$ (ecuación biarmónica de onda)
\item $\laplacian{u} + \alpha [E - V(x, y, z)] \, u = 0$ (ecuación de Schrödinger)
\item $\square \, u + \lambda^{2} \, u = 0$ (ecuación de Klein-Gordon)
\item $\displaystyle \pdv[2]{u}{x} = A \, \displaystyle \pdv[2]{u}{t} + B \, \displaystyle \pdv{u}{t} + C \, u$ (ecuación del telégrafo)
\item $(u_{x})^{2} + (u_{y})^{2} + (u_{z})^{2} =  n (x, y, z)$ (ecuación de la óptica geométrica)
\item $\displaystyle \pdv{P}{t} = \beta \, \displaystyle \pdv{x} (P \, x) +  D \displaystyle \pdv[2]{P}{x}$ (ecuación de Fokker-Planck)
\end{enumerate}
Donde el operador de D'Alambert se define como
\begin{align*}
\square \, u = \laplacian{u} - \dfrac{1}{c^{2}} \displaystyle \pdv[2]{u}{t}
\end{align*}
Además deben considerarse las ecuaciones de Lagrange de la mecánica, las ecuaciones de Maxwell del electromagnetismo, las ecuaciones de Navier-Stokes en hidrodinámica, etc.
\end{document}