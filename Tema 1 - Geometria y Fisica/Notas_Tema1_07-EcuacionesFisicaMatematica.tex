\documentclass[12pt]{article}
\usepackage[utf8]{inputenc}
\usepackage[spanish,es-lcroman, es-tabla]{babel}
\usepackage[autostyle,spanish=mexican]{csquotes}
\usepackage{amsmath}
\usepackage{amssymb}
\usepackage{nccmath}
\numberwithin{equation}{section}
\usepackage{amsthm}
\usepackage{graphicx}
\usepackage{epstopdf}
\DeclareGraphicsExtensions{.pdf,.png,.jpg,.eps}
\usepackage{color}
\usepackage{float}
\usepackage{multicol}
\usepackage{enumerate}
\usepackage[shortlabels]{enumitem}
\usepackage{anyfontsize}
\usepackage{anysize}
\usepackage{array}
\usepackage{multirow}
\usepackage{enumitem}
\usepackage{cancel}
\usepackage{tikz}
\usepackage{circuitikz}
\usepackage{tikz-3dplot}
\usetikzlibrary{babel}
\usetikzlibrary{shapes}
\usepackage{bm}
\usepackage{mathtools}
\usepackage{esvect}
\usepackage{hyperref}
\usepackage{relsize}
\usepackage{siunitx}
\usepackage{physics}
%\usepackage{biblatex}
\usepackage{standalone}
\usepackage{mathrsfs}
\usepackage{bigints}
\usepackage{bookmark}
\spanishdecimal{.}

\setlist[enumerate]{itemsep=0mm}

\renewcommand{\baselinestretch}{1.5}

\let\oldbibliography\thebibliography

\renewcommand{\thebibliography}[1]{\oldbibliography{#1}

\setlength{\itemsep}{0pt}}
%\marginsize{1.5cm}{1.5cm}{2cm}{2cm}


\newtheorem{defi}{{\it Definición}}[section]
\newtheorem{teo}{{\it Teorema}}[section]
\newtheorem{ejemplo}{{\it Ejemplo}}[section]
\newtheorem{propiedad}{{\it Propiedad}}[section]
\newtheorem{lema}{{\it Lema}}[section]

\DeclareUnicodeCharacter{00A0}{ }
\author{}
\title{Ecuaciones de la Física Matemática \\ {\large Matemáticas Avanzadas de la Física}\vspace{-1.5\baselineskip}}
\date{}
\author{}
\begin{document}
\maketitle
\fontsize{14}{14}\selectfont
\section{Conceptos básicos y definiciones.}
Una gran cantidad de situaciones físicas puede ser descrita utilizando ecuaciones diferenciales que incluyen funciones de dos o más variables. Las conocemos como ecuaciones diferenciales parciales pues incluyen derivadas respecto a cada una de las variables.
\par
Se denomina \emph{ecuación diferencial parcial de segundo orden} en las variables independientes $x$ y $y$, a una relación entre la función incógnita $u(x, y)$ y sus derivadas parciales hasta el segundo orden:
\begin{align*}
F (x, y, \ldots, u, u_{x}, u_{y}, \ldots, u_{xx}, u_{xy}, u_{yy}, \ldots) = 0
\end{align*}
y análogamente para un número mayor de variables independientes. El subíndice en las variables dependientes indican una diferenciación
\begin{align*}
u_{x} = \pdv{u}{x} \hspace{1.5cm} u_{x y} = \pdv[2]{u}{y}{x}
\end{align*}
Aquí la ecuación inicial se considera en un dominio adecuado $D$ del espacio n-dimensional $\mathbb{R}^{n}$ en las variables independientes $x, y, \ldots$. Buscamos funciones $u = u (x, y, \ldots)$ Que satisfagan la ecuación de forma idéntica en el dominio $D$.
\par
Dichas funciones, si existen, se denominan \emph{soluciones de la EDP}. De estas muchas soluciones posibles, intentamos seleccionar una particular introduciendo condiciones adicionales adecuadas.
\par
El \emph{orden} de una EDP es el orden más alto de la derivada parcial que tiene la expresión, por ejemplo
\begin{align*}
u_{x x} + 2 \, x \, u_{x y} + u_{y y} = e^{y}
\end{align*}
es una EDP de segundo orden, mientras que
\begin{align*}
u_{x x y} + x \, u_{y y} + 8 \, u = 7 \, y
\end{align*}
es una EDP de tercer orden.
\par
Se dice que una EDP es \emph{lineal} si es lineal en la función desconocida y todas sus derivadas con coeficientes que dependen sólo de las variables independientes; se dice que es \emph{casi lineal} si es lineal en la derivada de orden más alto de la función desconocida. Por ejemplo, la ecuación
\begin{align*}
y \, u_{xx} + 2 \, x \, y \, u_{yy} + u = 1
\end{align*}
es una EDP2 lineal, mientras que
\begin{align*}
u_{x} \, u_{x x} + x \, u \, y_{y} = \sin y
\end{align*}
es una EDP casi lineal. Las ecuaciones que no son lineales, se les llama ecuaciones \emph{no lineales.}
\par
\textbf{Ejercicio a cuenta: } Para cada una de las siguientes EDP indica y justifica:
\begin{enumerate}[label=\alph*)]
\item Si la ecuación es lineal, casi lineal o no lineal.
\item Si la ecuación es homogénea o no homogénea.
\item Indica el orden de la EDP.
\end{enumerate}
\begin{enumerate}[label=(\roman*)]
\item $u \, u_{x} - 2 \, x \, y \, u_{y} = 0$
\item $u_{x}^{2} + u \, u_{y} = 1$
\item $u_{xx}^{2} + u_{x}^{2} + \sin u =  e^{y}$
\end{enumerate}
% Una ecuación diferencial parcial es lineal respecto a la derivada de segundo orden si tiene la forma:
% \[ A \: u_{xx} + B \: u_{xy} + C \: u_{yy} + F_{1} (x, y, u, u_{x}, u_{y} ) = 0 \]
% donde $A$, $B$, $C$ son en general, funciones de $x$ y $y$.
% \par
% La ecuación diferencial será lineal, si lo es respecto a la función $u$ y a sus primeras y segundas derivadas:
% \[ A \: u_{xx} + B \: u_{xy} + C \: u_{yy} + D \: u_{x} + E \: u_{y} + F \: u = G \]
% donde $A$, $B$, $C$, $D$, $E$, $F$, $G$ son en general, funciones de $x$ y $y$. La ecuación es homogénea si $G = 0$.
\section{Ecuaciones de la Física Matemática.}
A continuación se menciona un conjunto de ecuaciones diferenciales de gran importancia en la física matemática:
\begin{enumerate}[label=\alph*)]
\item $\square \, u = 0$ o bien $\laplacian{u} = \dfrac{1}{c^{2}} \displaystyle \pdv[2]{u}{t}$ (ecuación de ondas)
\item $\laplacian{u} = \dfrac{1}{k} \displaystyle \pdv{u}{t}$ (ecuación de conducción del calor o de difusión)
\item $\laplacian{u} = f(x, y, z)$ (ecuación de Poisson; si $f=0$ se llama ecuación de Laplace)
\item $\laplacian{u} + \lambda \, u = 0$ (ecuación de Helmholtz)
\item $\nabla^{4} = \laplacian{(\laplacian{u})} = 0$ (ecuación biarmónica)
\item $\nabla^{4} = - \left( \dfrac{1}{p^{2}} \right) \displaystyle \pdv[2]{u}{t}$ (ecuación biarmónica de onda)
\item $\laplacian{u} + \alpha [E - V(x, y, z)] \, u = 0$ (ecuación de Schrödinger)
\item $\square \, u + \lambda^{2} \, u = 0$ (ecuación de Klein-Gordon)
\item $\displaystyle \pdv[2]{u}{x} = A \, \displaystyle \pdv[2]{u}{t} + B \, \displaystyle \pdv{u}{t} + C \, u$ (ecuación del telégrafo)
\item $(u_{x})^{2} + (u_{y})^{2} + (u_{z})^{2} =  n (x, y, z)$ (ecuación de la óptica geométrica)
\item $\displaystyle \pdv{P}{t} = \beta \, \displaystyle \pdv{x} (P \, x) +  D \displaystyle \pdv[2]{P}{x}$ (ecuación de Fokker-Planck)
\end{enumerate}
Donde el operador de D'Alambert se define como
\begin{align*}
\square \, u = \laplacian{u} - \dfrac{1}{c^{2}} \pdv[2]{u}{t}
\end{align*}
Además deben considerarse las ecuaciones de Lagrange de la mecánica, las ecuaciones de Maxwell del electromagnetismo, las ecuaciones de Navier-Stokes en hidrodinámica, etc.
\section{Clasificación de las EDP de segundo orden.}
La clasificación de las Ecuaciones Diferenciales Parciales de segundo orden (EDP2) es un concepto importante debido a la teoría general y métodos de solución, que normalmente funciona para una clase determinada de ecuaciones. 
\par
Veamos inicialmente la clasificación de las EDP2 que involucran dos variables independientes.
\subsection{Clasificación con dos variables independientes.}
Consideremos la EDP2 lineal general con dos variables independientes
\begin{align}
A \, \pdv[2]{u}{x} + B \, \pdv[2]{u}{x}{y} + C \, \pdv[2]{u}{y} + D \, \pdv{u}{x} + E \, \pdv{u}{y} + F \, u + G = 0
\label{eq:ecuacion_01}
\end{align}
donde $A, B, C, D, E, F, G$ son funciones de las variables independientes $x$ e $y$. La ec. (\ref{eq:ecuacion_01}) puede escribirse de la forma
\begin{align}
A \, u_{xx} + B \, u_{xy} + C \, u_{yy} + f(x, y, u_{x}, u_{y}, u) = 0
\label{eq:ecuacion_02}
\end{align}
donde
\begin{align*}
u_{x} = \pdv{u}{x}, \hspace{1cm} u_{y} = \pdv{u}{y}, \hspace{1cm} u_{xx} = \pdv[2]{u}{x}, \hspace{1cm} u_{xy} = \pdv[2]{u}{x}{y}, \hspace{1cm} u_{yy} = \pdv[2]{u}{y}
\end{align*}
Supondremos que $A$, $B$, y $C$ son funciones continuas de $x$ e $y$ y que tienen derivadas parciales continuas de orden superior en caso de requerirse.
\par
La clasificación de las EDP2 se basa en la clasificación de las ecuaciones algebraicas de segundo orden en dos variables
\begin{align}
a \, x^{2} + b \, x \, y + c \, y^{2} + d \, x + e \, y + f = 0
\label{eq:ecuacion_03}
\end{align}
Sabemos que la naturaleza de las curvas se determina por la parte principal $a \, x^{2} + b \, x \, y + c \, y^{2}$, es decir, la parte que contiene el grado más alto. Dependiendo del signo del discriminante $b^{2} - 4 \, a \, c$, se clasifican las curvas como sigue:
\par
\begin{center}
    \fbox{%
        \parbox{22em}{
            Si $b^{2} - 4 \, a \, c > 0$ entonces se traza una hipérbola. \\
            Si $b^{2} - 4 \, a \, c = 0$ entonces se traza una parábola. \\
            Si $b^{2} - 4 \, a \, c < 0$ entonces se traza una elipse.
        }
    }
\end{center}
Con una adecuada transformación, podemos llevar la ec. (\ref{eq:ecuacion_03}) a una de las siguientes formas normales
\begin{align*}
\dfrac{x^{2}}{a^{2}} - \dfrac{y^{2}}{b^{2}} =  1 \hspace{1cm} &\text{hipérbola} \\[1em]
x^{2} = y \hspace{1cm} &\text{parábola} \\[1em]
\dfrac{x^{2}}{a^{2}} + \dfrac{y^{2}}{b^{2}} =  1 \hspace{1cm} &\text{elipse}
\end{align*}
\subsection{EDP lineales con coeficientes constantes.}
Consideremos el caso general de una EDP2 lineal en dos variables $x$ e $y$ con coeficientes constantes:
\begin{align}
A \, u_{xx} + B \, u_{xy} + C \, u_{yy} + D \, u_{x} + E \, u_{y} + F \, u + G = 0
\label{eq:ecuacion_04}    
\end{align}
donde los coeficientes $A, B, C, D, E, G, G$ son constantes. La naturaleza de la ec. (\ref{eq:ecuacion_04}) se determina por la parte principal que contiene a las derivadas de mayor orden, es decir:
\begin{align}
L \, u \equiv A \, u_{xx} + B \, u_{xy} + C \, u_{yy}
\label{eq:ecuacion_05}    
\end{align}
Para la clasificación, agregamos un símbolo a la ec. (\ref{eq:ecuacion_05}) como
\begin{align*}
P(x, y) = A \, x^{2} + B \, x \, y + C \, y^{2}
\end{align*}
en donde hemos reemplazado $x \to \displaystyle \pdv{x}$ e $y \to \displaystyle \pdv{y}$.
\par
Dependiendo del discriminante $(B^{2} - 4\, , A \, C)$, la clasificación de la ec. (\ref{eq:ecuacion_04}) es la siguiente:
\begin{center}
    \fbox{%
        \parbox{22em}{
            Si $B^{2} - 4 \, A \, C > 0$ la ec. (\ref{eq:ecuacion_04}) es hiperbólica. \\
            Si $B^{2} - 4 \, A \, C = 0$ la ec. (\ref{eq:ecuacion_04}) es parabólica. \\
            Si $B^{2} - 4 \, A \, C < 0$ la ec. (\ref{eq:ecuacion_04}) es elíptica.
        }
    }
\end{center}
\subsection{EDP lineales con coeficientes variables.}
La clasificación que se le dio a las ecuaciones del tipo (\ref{eq:ecuacion_04}) es válida si los coeficientes $A, B, C, D, E, F$ dependen de $x$ e $y$. En este caso, las condiciones anteriores deben de satisfacerse en cada punto $(x, y)$ en la región donde queremos describir su naturaleza, por ejemplo, en el caso de la ecuación de tipo elíptico necesitamos verificar que
\begin{align*}
B^{2} (x, y) - 4 \, A (x, y) \, C (x, y) < 0
\end{align*}
para cada $(x, y)$ en la región de interés- Entonces, clasificamos la EDP lineal con coeficientes variables como se indica a continuación:
\begin{center}
    \fbox{%
        \parbox{25em}{
            Si $B^{2}(x, y) - 4 \, A(x, y) \, C(x, y) > 0 \mbox{ en } (x, y) \Rightarrow$ la ec. (\ref{eq:ecuacion_04}) es hiperbólica en $(x, y)$ \\[1em]
            Si $B^{2}(x, y) - 4 \, A(x, y) \, C(x, y) = 0 \mbox{ en } (x, y) \Rightarrow$ la ec. (\ref{eq:ecuacion_04}) es parabólica en $(x, y)$ \\[1em]
            Si $B^{2}(x, y) - 4 \, A(x, y) \, C(x,y) < 0 \mbox{ en } (x, y) \Rightarrow$ la ec. (\ref{eq:ecuacion_04}) es elíptica en $(x, y)$
        }
    }
\end{center}
Como vemos, la ec. (\ref{eq:ecuacion_04}) es de tipo hiperbólico, parabólico o elíptico se define sólo con los coeficientes de las derivadas de segundo orden. No hay nada que hacer con las derivadas de primer orden, el término en $u$ o con el término no homogéneo.
\begin{ejemplo}
Considera la ecuación de Laplace
\begin{align*}
u_{xx} + u_{yy} = 0
\end{align*}
Entonces tenemos que
\begin{align*}
A = 1, \hspace{1cm} B = 0 \hspace{1cm} C = 1 \\
\Rightarrow B^{2} - 4 \, A \, C = - 4 < 0
\end{align*}
por lo tanto, la ecuación de Laplace es de tipo elíptico.
\end{ejemplo}
\begin{ejemplo}
Considera la ecuación de calor
\begin{align*}
u_{t} = u_{xx}
\end{align*}
Donde
\begin{align*}
A = -1, \hspace{1cm} B = 0 \hspace{1cm} C = 0 \\
\Rightarrow B^{2} - 4 \, A \, C = 0
\end{align*}
por lo tanto, la ecuación de calor es de tipo parabólico.
\end{ejemplo}
\begin{ejemplo}
Considera la ecuación de onda
\begin{align*}
u_{tt} - u_{xx} = 0
\end{align*}
Donde se tiene que:
\begin{align*}
A = -1, \hspace{1cm} B = 0 \hspace{1cm} C = 1 \\
\Rightarrow B^{2} - 4 \, A \, C = 4 > 0
\end{align*}
por lo tanto, la ecuación de onda es de tipo hiperbólico.
\end{ejemplo}
\begin{ejemplo}
Considera la ecuación de Tricomi:
\begin{align*}
u_{xx} - x \, u_{yy} = 0 \hspace{1.5cm} x \neq 0
\end{align*}
Se tiene entonces:
\begin{align*}
B^{2} - 4 \, A \, C = - 4 \, x
\end{align*}
Dada esta EDP tenemos que:
\begin{enumerate}
\item Para valores $x < 0$ la ecuación es de tipo hiperbólico.
\item Para valores $x > 0$ la ecuación es de tipo elíptico.
\end{enumerate}
Este es un ejemplo en donde se muestra que el tipo de ecuaciones con coeficientes variables puede cambiar de forma en diferentes regiones del dominio.
\end{ejemplo}
\textbf{Ejercicio a cuenta: } Estudia los dominios en los que las siguientes EDP son elípticas, parabólicas o hiperbólicas:
\begin{enumerate}[label=(\roman*)]
\item $(x - l) \, u_{x x} + 2 \, x \, y \, u_{x y} - y^{2} \, u_{y y } = 0, \hspace{1.5cm} l > 0$
\item $4 \, y^{2} \, u_{x x} - e^{2 x} \, u_{y y} - 4 \, y^{2} \, u_{y} = 0$
\item $x^{2} \, u_{x x} + 2 \, x \, y \, u_{x y} + y^{2} \, u_{y y} = 0$
\end{enumerate}
\section{Clasificación de las EDP con más de dos variables.}
Consideremos la forma general de una EDP2
\begin{align}
\sum_{i,j=1}^{n} a_{ij} \, \pdv[2]{u}{x_{i}}{x_{j}} + \sum_{i=1} b_{i} \, \pdv{u}{x_{i}} + c \, u + d = 0
\label{eq:ecuacion_12}    
\end{align}
donde los coeficientes $a_{i,j}, b_{i}, c, d$ son funciones reales de $x = (x_{1}, x_{2}, \ldots, x_{n})$ y $u = u(x_{1}, x_{2}, \ldots, x_{n})$.
\par
La parte principal de la EDP2 es
\begin{align}
L \equiv \sum_{i, j=1}^{n} a_{ij} \, \pdv[2]{u}{x_{i}}{x_{j}}
\label{eq:ecuacion_13}    
\end{align}
Basta con suponer que $A = [a_{ij}]$ es simétrica, en caso de que no  lo sea, hacemos $\overline{a}_{ij} = \dfrac{1}{2} (a_{ij} + a_{ji})$ y reescribimos
\begin{align}
L \equiv \sum_{i, j=1}^{n} \overline{a}_{ij} \, \pdv[2]{u}{x_{i}}{x_{j}}
\label{eq:ecuacion_14}    
\end{align}
Tomemos en cuenta que $\displaystyle \pdv[2]{x_{i}}{x_{j}} = \pdv[2]{x_{j}}{x_{i}}$. 
\par
En un espacio de dos dimensiones, podemos agregar una forma cuadrática $P$ a la ec. (\ref{eq:ecuacion_14}), al reemplazar $\displaystyle \pdv{u}{x_{i}} \to x_{i}$
\begin{align}
P(x_{1}, x_{2},\ldots, x_{n}) = \sum_{i,j=1}^{n} a_{ij} \, x_{i} \, x_{j}
\label{eq:ecuacion_015}
\end{align}
Ya que $A$ es una matriz real simétrica $(a_{ij} = a_{ji})$, entonces es diagonizable con valores propios (\emph{eigen valores} reales $(\lambda_{1}, \lambda_{2}, \ldots \lambda_{n})$ (considerando sus multiplicidades). En otras palabras, existe un correspondiente conjunto ortonormal de $n$ vectores propios (\emph{eigen vectores}), digamos $(\sigma_{1}, \sigma_{2}, \ldots, \sigma_{n})$ con $R = [\sigma_{1}, \sigma_{2}, \ldots, \sigma_{n}]$ como vectores columna, tales que
\begin{align}
R^{T} \, A \, R =
\mqty[
\lambda_{1} & & & & & & \\
& \lambda_{2} & & 0 & & & \\
& & & \ddots & &  & \\
& & & &\ddots & & \\
& & & 0 & & & \\
&  & &  & & \lambda_{n-1} & \\
&  & &  & & & \lambda_{n} \\
] = D
\label{eq:ecuacion_16}    
\end{align}
Entonces la clasificación de las ecuaciones tipo (\ref{eq:ecuacion_12}) dependerá del signo de los valores propios de $A$:
\begin{center}
    \fbox{%
        \parbox{30em}{
            \begin{enumerate}[label=(\alph*)]
            \item Si $\lambda_{i} > 0 \, \forall i$ o $\lambda_{i} < 0 \, \forall i$ entonces la ec. (\ref{eq:ecuacion_12}) es de tipo elíptico.
            \item Si uno o más de los $\lambda_{i} = 0$ entonces la ec. (\ref{eq:ecuacion_12}) es de tipo parabólico.
            \item Si uno de los $\lambda_{i} < 0$ o $\lambda_{i} > 0$ y todos los demás tienen un signo opuesto, entonces la ec. (\ref{eq:ecuacion_12}) es hiperbólico.
            \end{enumerate}
        }
    }
\end{center}
\begin{ejemplo}
Considera la ecuación
\begin{align*}
\laplacian{u} = u_{xx} + u_{yy} + u_{zz} = 0
\end{align*}
En este caso: $\lambda_{i} = 1 > 0 \quad \forall i = 1, 2, 3$. 
\par
Por lo que la EDP es de tipo elíptico, ya que todos los eigen valores son de un sólo signo.
\end{ejemplo}
\textbf{Ejercicio a cuenta:} Demuestra que
\begin{enumerate}[label=\roman*) ]
\item $u_{t} - \laplacian{u} = 0$ es de tipo parabólico.
\item $u_{tt} - \laplacian{u} = 0$ es de tipo hiperbólico.
\end{enumerate}
\begin{ejemplo}
Clasifica la EDP
\begin{align*}
u_{x_{1} x_{1}} + 2 (1 + c \, x_{2}) \, u_{x_{2} x_{3}} = 0
\end{align*}
Para contar con una simetría, escribimos la ecuación como
\begin{align*}
u_{x_{1} x_{1}} + (1 + c \, x_{2}) \, u_{x_{2} x_{3}} + (1 + c \, x_{2}) \, u_{x_{2} x_{3}} = 0
\end{align*}
es decir
\begin{align*}
\partial_{x}^{T} \, A \, \partial_{x} - c \, \partial_{x_{2}} = 0
\end{align*}
donde
\begin{align*}
A = \mqty[
1 & 0 & 0 \\
0 & 0 & 1 + c \, x_{2} \\
0 & 1 + c \, x_{2} & 0
] \hspace{2cm}
\partial_{x} = \mqty[
\partial_{x_{1}} \\
\partial_{x_{2}} \\
\partial_{x_{3}}
]
\end{align*}
Los valores propios son
\begin{align*}
\lambda_{1} &= 1 \\
\lambda_{2} &= 1 + c \, x_{2} \\
\lambda_{1} &= - (1 + c \, x_{2})
\end{align*}
y los eigen vectores normalizados son
\begin{align*}
\sigma_{1} = \mqty[ 1 \\ 0 \\ 0] \hspace{1.5cm} \sigma_{2} = \mqty[0 \\ 1/\sqrt{2} \\ 1/\sqrt{2}] \hspace{1.5cm} \sigma_{3} = \mqty[0 \\ 1/\sqrt{2} \\ -1/\sqrt{2}]
\end{align*}
Por lo que
\begin{align*}
R = \mqty[
1 & 0 & 0 \\
0 & 1/\sqrt{2} & 1/\sqrt{2} \\
0 & 1/\sqrt{2} & -1/\sqrt{2}
]
\end{align*}
Nótese que $R = R^{T} = R^{-1}$
\begin{align*}
R^{T} \, A \, R = \mqty[
1 & 0 & 0 \\
0 & 1+ c \, x_{2} & 0 \\
0 & 0 & -(1 + c \, x_{2})
] = D
\end{align*}
La EDP es de tipo:
\begin{enumerate}[label=\alph*)]
    \item Parabólica, si $x_{2} = - \dfrac{1}{c} \hspace{0.5cm} c \neq 0$
    \item Hiperbólica, si $x_{2} > - \dfrac{1}{c}$ y $x_{2} < - \dfrac{1}{c}$
    \item Para $c = 0$, $\lambda_{1} = \lambda_{2} = 1$ y $\lambda_{3} = -1$, es de tipo hiperbólico.
\end{enumerate}
\end{ejemplo}
\section{Formas canónicas.}
Con un respectivo cambio en las variables independientes, podemos demostrar que cualquier ecuación de la forma
\begin{align}
A \, u_{xx} + B \, u_{xy} + C \, u_{yy} + D \, u_{x} + E \, u_{y} + F \, u + G = 0
\label{eq:ecuacion_02_01}    
\end{align}
donde $A, B, C, D, E, F, G$ son funciones de las variables $x$ e $y$, se puede reducir a una \emph{forma canónica}. La ecuación transformada se asume como una forma simple tal que el subsecuente análisis para resolver la ecuación, se vuelve más sencillo.
\par
Consideremos la transformación de las variables independientes de $(x, y)$ a $(\xi, \eta)$ dada por
\begin{align}
\xi = \xi (x, y), \hspace{2cm} \eta = \eta (x, y)
\label{eq:ecuacion_02_02}
\end{align}
Las funciones $\xi$ y $\eta$ son funciones continuas y diferenciables, además el Jacobiano es tal que
\begin{align}
J = \pdv{(\xi, \eta)}{(x, y)} = \mdet{
\xi_{x} & \xi_y \\
\eta_{x} & \eta_{y}
} =
(\xi_{x} \, \eta_{y} - \xi_{y} \, \eta_{x}) \neq 0
\label{eq:ecuacion_02_03}    
\end{align}
en el dominio de la ec. (\ref{eq:ecuacion_02_01}).
\par
Usando la regla de la cadena para diferenciar las variables iniciales, tenemos que
\begin{align*}
u_{x} &= u_{\xi} \, \xi_{x} + u_{\eta} \, \eta_{x} \\
u_{y} &= u_{\xi} \, \xi_{y} + u_{\eta} \, \eta_{y} \\
u_{xx} &= u_{\xi \xi} \, \xi_{x}^{2} + 2 \, u_{\xi \eta} \, \xi_{x} \, \eta_{x} + u_{\eta \eta} \, \eta_{x}^{2} + u_{\xi} \xi_{x x} + u_{\eta} \, \eta_{xx} \\
u_{xy} &= u_{\xi \xi} \, \xi_{x} \, \xi_{y} + u_{\xi \eta} (\xi_{x} \, \eta_{y} + \xi_{y} \, \eta_{x}) + u_{\eta \eta} \, \eta_{x} \, \eta_{y} + u_{\eta} \, \xi_{x y} + u_{\eta} \, \eta_{x y} \\
u_{yy} &= u_{\xi \xi} \, \xi_{y}^{2} + 2 \, u_{\xi \eta} \, \xi_{y} \, \eta_{y} + u_{\eta \eta} \, \eta_{y}^{2} + u_{\xi} \, \xi_{y y} + u_{\eta} \, \eta_{yy}
\end{align*}
sustituyendo estas expresiones en la ec. (\ref{eq:ecuacion_02_01}), obtenemos
\begin{align}
\begin{aligned}
\overline{A} (\xi_{x}, \xi_{y}) \, u_{\xi \xi} &+ \overline{B} (\xi_{x}, \xi_{y}; \eta_{x},\eta_{y}) \, u_{\xi \eta} + \overline{C} (\eta_{x}, \eta_{y}) \, u_{\eta \eta} = \\
&= F (\xi, \eta, u(\xi, \eta), u_{\xi}( \xi, \eta), u_{\eta} (\xi, \eta))
\end{aligned}
\label{eq:ecuacion_02_04}    
\end{align}
donde
\begin{align*}
\overline{A} (\xi_{x}, \xi_{y}) &= A \, \xi_{x}^{2} + B \, \xi_{x} \, \xi_{y} + C \, \xi_{y}^{2} \\
\overline{B} (\xi_{x}, \xi_{y}; \eta_{x},\eta_{y}) &= 2 \, A \, \xi_{x} \, \eta_{x} + B (\xi_{x} \, \eta_{y} + \xi_{y} \, \eta_{x}) + 2 \, C \, \xi_{y} \, \eta_{y} \\
\overline{C} (\eta_{x}, \eta_{y}) &= A \, \eta_{x}^{2} + B \, \eta_{x} \eta_{y} + C \, \eta_{y}^{2}
\end{align*}
Una cuenta sencilla nos demuestra que
\begin{align}
\overline{B}^{\, 2} - 4 \, \overline{A} \, \overline{C} = (\xi_{x} \, \eta_{y} - \xi_{y} \, \eta_{x})^{2} \, (B^{2} - 4 \, A \, C)
\label{eq:ecuacion_02_05}    
\end{align}
La ec. (\ref{eq:ecuacion_02_05}) muestra que la transformación de variables independientes no modifica el tipo de EDP.
\par
Buscaremos darle a $\xi$ y a $\eta$ para que en la ec. (\ref{eq:ecuacion_02_04}) tomen la forma más simple posible. Ahora consideramos los siguientes casos:
\begin{enumerate}[label=\textbf{Caso \Roman*.}]
\item $B^{2} -  \, A \, C > 0$ Tipo hiperbólico.
\item $B^{2} -  \, A \, C = 0$ Tipo parabólico.
\item $B^{2} -  \, A \, C < 0$ Tipo elíptico.
\end{enumerate}
\textbf{Caso I. } Notemos que $B^{2} -  \, A \, C > 0$ implica que la ecuación
\begin{align*}
A \, \alpha^{2} + B \, \alpha + C = 0
\end{align*}
tiene dos raíces reales y distintas, digamos $\lambda_{1}$ y $\lambda_{2}$. Ahora bien, elegimos $\xi$ y $\eta$ tales que
\begin{align}
\begin{aligned}
\pdv{\xi}{x} &= \lambda_{1} \, \pdv{\xi}{y} \\[1em]
\pdv{\eta}{x} &= \lambda_{2} \, \pdv{\eta}{y}
\end{aligned}
\label{eq:ecuacion_02_06}    
\end{align}
Entonces los coeficientes de $u_{\xi \xi}$ y de $u_{\eta \eta}$ serán cero, debido a
\begin{align*}
\overline{A} (\xi_{x}, \xi_{y}) &= A \, \xi_{x}^{2} + B \, \xi_{x} \, \xi_{y} + C \, \xi_{y}^{2} = (A \, \lambda_{1}^{2} + B \, \lambda_{1} + C ) \, \xi_{y}^{2} = 0 \\[1em]
\overline{C} (\eta_{x}, \eta_{y}) &= A \, \eta_{x}^{2} + B \, \eta_{x} \eta_{y} + C \, \eta_{y}^{2} = (A \, \lambda_{2}^{2} + B \, \lambda_{2} + C ) \, \eta_{y}^{2} = 0
\end{align*}
Entonces, la ec. (\ref{eq:ecuacion_02_05}) se reduce a
\begin{align*}
\overline{B}^{2} = (B^{2} - 4 \, A \, C) (\eta_{x} \, \eta_{y} - \xi_{y} \, \eta_{x})^{2} > 0
\end{align*}
como se había obtenido $B^{2} - 4 \, A \, C > 0$. Nótese que la ec. (\ref{eq:ecuacion_02_06}) es una EDP lineal de primer orden en $\xi$ y $\eta$, cuyas curvas características satisfacen las EDO de primer orden
\begin{align}
\dv{y}{x} + \lambda_{i} (x, y) = 0, \hspace{1.5cm} i = 1, 2
\label{eq:ecuacion_02_07}    
\end{align}
Sea la familia de curvas que determina la solución de la ec. (\ref{eq:ecuacion_02_07}) para $i = 1$ e $i = 2$, tales que
\begin{align}
f_{1} (x, y) = c_{1} \hspace{2cm} f_{2}(x, y) = c_{2}
\label{eq:ecuacion_02_08}
\end{align}
respectivamente. Esas familias de curvas se denominan curvas características de la EDP (\ref{eq:ecuacion_02_05}). Con esta elección, dividimos la ec. (\ref{eq:ecuacion_02_04}) entre $\overline{B}$ y utilizamos las ecs. (\ref{eq:ecuacion_02_07}) y (\ref{eq:ecuacion_02_08}), para obtener
\begin{align}
\pdv[2]{u}{\xi}{\eta} = \phi (\xi, \eta, u, u_{\xi}, u_{\eta})
\label{eq:ecuacion_02_09}
\end{align}
la cual es la forma canónica de una ecuación de tipo hiperbólico.
\begin{ejemplo}\label{ejemplo_02_01}{Reducir la ecuación $u_{xx} = x^{2} \, u_{yy}$ a su forma canónica.}

Comparando la ec. (\ref{eq:ecuacion_02_01}), encontramos que $A = 1$, $B = 0$, $C = -x^{2}$
\par
Las raíces de la ecuación
\begin{align*}
A \, \alpha^{2} + B \, \alpha + C &= 0 \\
\alpha^{2} + x^{2} &= 0
\end{align*}
están dadas por $\lambda_{i} = \pm x$
\par
Las EDO para la familia de curvas características son
\begin{align*}
\dv{y}{x} \pm x = 0
\end{align*}
que tienen por solución
\begin{align*}
\xi &= y + \dfrac{1}{2} \, x^{2} \\[1em]
\eta &= y - \dfrac{1}{2} \, x^{2}
\end{align*}
Una serie de operaciones sencillas nos demuestran que
\begin{align*}
u_{x} &= u_{\xi} \, \xi_{x} + u_{\eta} \, \eta_{x} \\
u_{xx} &= u_{\xi \xi} \, \xi_{x}^{2} + 2 \, u_{\xi \eta} \, \xi_{x} \, \eta_{x} + u_{\eta \eta} \, \eta_{x}^{2} + u_{\xi} \xi_{x x} + u_{\eta} \, \eta_{xx} \\
&= u_{\xi \xi} \, x^{2} - 2 \, u_{\xi \eta} x^{2} + u_{\xi} - u_{\eta} \\
u_{yy} &= u_{\xi \xi} \, \xi_{y}^{2} + 2 \, u_{\xi \eta} \, \xi_{y} \, \eta_{y} + u_{\eta \eta} \, \eta_{y}^{2} + u_{\xi} \, \xi_{y y} + u_{\eta} \, \eta_{yy} \\
&= u_{\xi \xi} + 2 \, u_{\xi \eta} + u_{\eta \eta}
\end{align*}
sustituyendo esas expresiones en la ecuación inicial $u_{x x } = x^{2} \, u_{y y}$, nos lleva a
\begin{align*}
4 \, x^{2} \, u_{\xi \eta} &= (u_{xi} - u_{\eta}) \\
\mbox{o } \hspace{0.5cm} 4 \, (\xi - \eta) \, u_{\xi \eta} &= \dfrac{1}{4 (\xi - \eta)} (u_{\xi} - u_{\eta}) \\
\mbox{o } \hspace{2.3cm} u_{\xi \eta} &= \dfrac{1}{4 (\xi - \eta)} (u_{\xi} - u_{\eta})
\end{align*}
la cual, es la forma canónica requerida.
\end{ejemplo}
\textbf{Caso II: } De $B^{2} - 4 \, A \, C = 0$, entonces la ecuación $A \, \alpha^{2} + B \, \alpha + C = 0$ tiene dos raíces iguales, digamos $\lambda_{1} = \lambda_{2} = \lambda$.
\par
Sea
\begin{align*}
f_{1} (x, y) = c_{1}
\end{align*}
la solución de
\begin{align*}
\dv{y}{x} + \lambda (x, y) = 0
\end{align*}
Tomemos $\xi =  f_{1}(x, y)$ y $\eta$ que sea cualquier función de $x$ e $y$ independiente de $\xi$.
\par
Como vimos antes $\overline{A} (\xi_{x}, \xi_{y}) = 0$ y por tanto, de la ec. (\ref{eq:ecuacion_02_05}), se tiene que $\overline{B} = 0$, además vemos que $\overline{C} (\eta_{x}, \eta_{y}) \neq  0$, de otra manera, $\eta$ sería una función de $\xi$. Dividiendo la ec. (\ref{eq:ecuacion_02_04}) entre $\overline{C}$, la forma canónica de la ec. (\ref{eq:ecuacion_02_01}) es
\begin{align}
u_{\eta \eta} = \phi (\xi, \eta, u, u_{\xi}, u_{\eta})
\label{eq:ecuacion_02_10}
\end{align}
que corresponde a la forma canónica de una ecuación parabólica.
\begin{ejemplo}{Reducir la ecuación $u_{xx} + 2 \, u_{x y} + u_{yy} =  0$ a una forma canónica.}

Para este caso: $A = 1, B = 2, C = 1$. La ecuación $\alpha^{2} + 2 \, \alpha + 1 = 0$ tiene raíces iguales $\lambda = -1$.
\par
La solución de $\dv*{y}{x} - 1 = 0$ es $x - y = c_{1}$. Hacemos $\xi =  x - y$. Escogemos $\eta = x + y$. Repetimos el procedimiento que se hizo en el ejemplo (\ref{ejemplo_02_01}) para obtener $u_{\eta \eta} = 0$, que corresponde a la forma canónica de la EDP dada para el ejercicio.
\end{ejemplo}
\textbf{Caso III: } Cuando $B^{2} - 4 \, A \, C < 0$, las raíces de $A \, \alpha^{2} + B \, \alpha + C = 0$ son raíces complejas. Siguiendo el procedimiento como en el \textbf{Caso I}, encontramos que
\begin{align}
u_{\xi \eta} = \phi_{1} (\xi, \eta, u, u_{\xi}, u_{\eta})
\label{eq:ecuacion_02_11}
\end{align}
Las variables $\xi$ y $\eta$ son de hecho complejos conjugados. Para obtener una forma canónica real usamos la transformación
\begin{align*}
\alpha = \dfrac{1}{2} \, (\xi + \eta) \hspace{1.5cm} \beta = \dfrac{1}{2 \, i} \, (\xi - \eta)
\end{align*}
para obtener
\begin{align}
u_{\xi \eta} = \dfrac{1}{4} \, (u_{\alpha \alpha} + u_{\beta \beta})
\label{eq:ecuacion_02_12}
\end{align}
que se obtiene del siguiente cálculo
\begin{align*}
u_{\xi} &= u_{\alpha} \, \alpha_{\xi} + u_{\beta} \, \beta_{\xi} = \dfrac{1}{2} \, u_{\alpha} + \dfrac{1}{2 \, i} \, u_{\beta} \\
u_{\xi \eta} &= \dfrac{1}{2} \, (u_{\alpha \alpha} \, \alpha_{\eta} + u_{\alpha \beta} \, \beta_{\eta}) + \dfrac{1}{2 \, i} (u_{\beta \alpha} \, \alpha_{\eta} + u_{\beta \beta} \, \beta_{\eta}) \\
&= \dfrac{1}{4} \, (u_{\alpha \alpha} + u_{\beta \beta})
\end{align*}
Entonces la forma canónica deseada es
\begin{align}
u_{\alpha \alpha} + u_{\beta \beta} = \psi( \alpha, \beta, u(\alpha, \beta), u_{\alpha}(\alpha, \beta), u_{\beta}(\alpha, \beta))
\label{eq:ecuacion_02_13}
\end{align}
\begin{ejemplo}{Reducir la ecuación $u_{xx} + x^{2} \, u_{y y} = 0$ a su forma canónica.}

En este caso $A = 1, B = 0, C = x^{2}$. Las raíces son $\lambda_{1} = i \, x, \lambda_{2} = - i \, x$.
\par
Tomando
\begin{align*}
\xi &= i \, y + \dfrac{1}{2} \, x^{2} \\
\eta &= - i \, y + \dfrac{1}{2} \, x^{2}
\end{align*}
Entonces
\begin{align*}
\alpha &= \dfrac{1}{2} \, x^{2} \\
\beta &= y
\end{align*}
La forma canónica resultante es
\begin{align*}
u_{\alpha \alpha} + u_{\beta \beta} = - \dfrac{1}{2 \, \alpha} \, u_{\alpha}
\end{align*}
\end{ejemplo}
\textbf{Ejercicio a cuenta: } Reduce a una forma canónica y clasifica las siguientes EDP2:
\begin{enumerate}[label=(\Alph*)]
\item $2 \, u_{x x} - 4 \, u_{x y} + 2 \, u_{y y} + 3 \, u = 0$
\item $u_{x x} + y\, u_{y y} = 0$
\item $u_{x y} + u_{x} + u_{y} =  2 \, x$
\end{enumerate}
\section{Principio de superposición y problemas bien planteados.}
Un importante hecho respecto a las EDP lineales, es el principio de superposición, el cual establece que: una EDP lineal puede escribirse de la forma
\begin{align}
L \bqty{u} = f
\label{eq:ecuacion_03_01}
\end{align}
donde $L \bqty{u}$ indica una combinación lineal de $u$ y alguna de sus derivadas parciales, con coeficientes que reciben funciones de las variables independientes.
\begin{defi}{Principio de superposición.}

Sea $u_{1}$ una solución de la EDP lineal
\begin{align*}
L \bqty{u} = f_{1}
\end{align*}
y sea $u_{2}$ una solución a la EDP lineal
\begin{align*}
L \bqty{u} = f_{2}
\end{align*}
Entonces, para cualesquiera constantes $c_{1}$ y $c_{2}$, $c_{1} \, u_{1} + c_{2} \, u_{2}$ es una solución de
\begin{align}
L \bqty{c_{1} \, u_{1} + c_{2} \, u_{2}} = c_{1} \, f_{1} + c_{2} \, f_{2}
\label{eq:ecuacion_03_02}
\end{align}
En particular, cuando $f_{1} = 0$ y $f_{2} = 0$, la ec. (\ref{eq:ecuacion_03_02}) implica que si $u_{1}$ y $u_{2}$ son soluciones de la EDP lineal y homogénea $L \bqty{u} = 0$, entonces $c_{1} \, u_{1} + c_{2} \, u_{2}$ será también una solución de $L \bqty{u} = 0$.
\end{defi}
\begin{ejemplo}
Veamos que $u_{1}(x, y) = x^{3}$ es una solución de la EDP lineal $u_{xx} - u_{y} = 6 \, x$ y $u_{2}(x, y) = y^{2}$ es una solución de $u_{xx} - u_{y} = - 2 \, y$.
\par
Entonces, usando el principio de superposición, es fácil verificar que
\begin{align*}
3 \, u_{1} (x, y) - 4 \, u_{2} (x, y) \hspace{1.25cm} \mbox{será solución de} \hspace{1.25cm} u_{xx} - u_{y} = 18 \, x + 8 \, y
\end{align*}
\end{ejemplo}
\textsc{Nota: } Considera que el principio de superposición \emph{no es válido} para EDP no lineales. Estes punto hace que sea difícil formar familias de nuevas soluciones a partir de un par de soluciones originales.
\begin{ejemplo}
Considera la EDP no lineal de primer orden
\begin{align*}
u_{x} \, u_{y} - u \, (u_{x} + u_{y}) + u^{2} = 0
\end{align*}
Nótese que tanto $e^{x}$ como $e^{y}$ son dos soluciones para esta ecuación. Sin embargo, $c_{1} \, e^{x} + c_{2} \, e^{y}$ no será una solución, a menos que $c_{1} = 0$ o $c_{2} = 0$.
\par
Solución: Definimos
\begin{align*}
D \bqty{u} := (u_{x} - u)(u_{y} - u)
\end{align*}
Para cualquier $u, v \in C^{1}$, tenemos entonces que
\begin{align*}
D \bqty{u + v} &= (u_{x}  + v_{x} - u - v)(u_{y} - v_{y} - u -v) \\
&= D \bqty{u} + D \bqty{v} + (u_{y - u})(v_{x} - v) + (u_{x} - u)(v_{y} - v)
\end{align*}
Este cálculo nos muestra que en general
\begin{align*}
D \bqty{u + v} \neq D \bqty{u} + D \bqty{v}
\end{align*}
Haciendo $u = c_{1} \, e^{x}$ y $v = c_{2} \, e^{y}$, una cuenta sencilla nos lleva al resultado
\begin{align*}
D \bqty{c_{1} \, e^{x} + c_{2} \, e^{y}} = D \bqty{c_{1} \, e^{x}} + D \bqty{c_{2} \, e^{y}} + (-c_{1} \, e^{x})(-c_{2} \, e^{y}) = c_{1} \, c_{2} \, e^{x + y} 
\end{align*}
Entonces, $D \bqty{c_{1} \, e^{x} + c_{2} \, e^{y}} = 0$ sólo si $c_{1} = $ o $c_{2} = 0$.
\end{ejemplo}
\subsection{Problemas bien planteados.}
Se conocen tres requisitos que deben cumplirse al formular un problema de valores iniciales y/o condiciones de frontera (CDF). Se dice que un problema para el cual la EDP y los datos conducen a una solución, están \emph{bien planteados} o correctamente si las siguientes tres condiciones son satisfactorias:
\begin{enumerate}[label=\roman*)]
\item La solución debe existir. \label{condicion_01}
\item La solución debe ser única. \label{condicion_02}
\item La solución debe depender continuamente de las condiciones iniciales y/o las CDF. \label{condicion_03}
\end{enumerate}
Si no cumple con estos requisitos, el problema con la EDP se plantea incorrectamente.
\par
Las condiciones \ref{condicion_01} - \ref{condicion_02} requieren que la ecuación más las datos del problema sean tales que exista una única solución. La tercera condición establece que una pequeña variación de los datos para el problema debería causar una pequeña variación en la solución. Como los datos generalmente se obtienen de manera experimental y pueden estar sujetos a aproximaciones numéricas, requerimos que la solución sea estable bajo pequeñas variaciones en los valores iniciales y/o CDF. Es decir, no podemos permitir que ocurran grandes variaciones en la solución si los datos se alteran ligeramente.
\section{Condiciones de frontera.}
Por lo general, cuando sabemos que un sistema físico en algún momento está sometido a una ley que rige tal sistema, entonces seremos capaces de predecir la evolución de ese sistema. Tales valores iniciales son las condiciones de contorno (de frontera) más comunes que se asocian a las EDO y las EDP. La búsqueda de soluciones que responden a determinados puntos, curvas o superficies corresponde a problemas con condiciones de frontera. Las soluciones normalmente deben de satisfacer determinados condiciones de frontera impuestas (por ejemplo, asintóticas). Estas condiciones de frontera pueden clasificar en tres formas:
\begin{enumerate}
\item \textbf{Condiciones de frontera de Dirichlet}. El valor de una función se especifica en la frontera.
\item \textbf{Condiciones de frontera de Neumann}. La derivada normal (gradiente) de una función se especifica en la frontera. En el caso de la electrostática, serían $E_{n}$ y $\sigma$, la densidad de carga superficial.
\item \textbf{Condiciones de frontera de Cauchy}. El valor de una función y la derivada se especifican en la frontera. En la electrostática esto significaría, el potencial $\varphi$, y la componente normal del campo eléctrico $E_{n}$.
\end{enumerate}
%Sepulveda - EDP Cap. 3
Se puede hacer una segunda clasificación de las CDF, en función de su alcance: \emph{generales} y \emph{específicas}.
\subsection*{CDF Generales.}
Una condición de frontera general corresponde a situaciones donde un campo se extiende de un medio a otro. En estos casos no es posible especificar los valores de los campos y/o sus derivadas sobre las superficies de separación (interfases), sino alguna relación entre sus valores a ambos lados.
\par
Para un campo electrostático, por ejemplo, el potencial es continuo a través de la interfase $(\phi_{1} \eval_{S} = \phi_{2} \eval_{S})$ y, si no hay carga superficial en la interfase, la componente normal del vector de desplazamiento es continua a través de la interfase $(\vb{D}_{1} \vdot \vu{n} \eval_{S} = \vb{D}_{2} \vdot \vu{n} \eval_{S})$.
\par
En el caso del campo de temperatura, ésta es continua en la frontera de separación de dos medios ($T_{1} \eval_{S} = T_{2} \eval_{S})$.
\subsection*{CDF Específica.}
Una CDF específica establece los valores de los campos y/o sus derivadas espaciales y/o temporales en las fronteras espaciales (sean ellas superficies o líneas o puntos) y en puntos iniciales en el tiempo. A este tipo pertenecen las condiciones de Dirichlet, Neumann y Cauchy.
\par
Las condiciones de frontera específicas pueden subdividirse en homogéneas e inhomogéneas.
\begin{enumerate}
\item  Una CDF homogénea es del tipo:
\begin{align*}
\phi \eval_{S} = 0 \hspace{1cm} \mbox{o} \hspace{1cm} \pdv{\phi}{n} \eval_{S} \hspace{1.5cm} \hspace{1cm} \mbox{o} \hspace{1cm} \alpha \: \phi \eval_{S} + \beta \: \pdv{\phi}{n} \eval_{S} = 0
\end{align*}
La solución de problemas de tipo Sturm-Liouville, exige este tipo de condiciones.
\item Las CDF inhomogéneas son del tipo:
\begin{align*}
\phi \eval{S} = f(\vb{r}) \hspace{1cm} \mbox{o} \hspace{1cm}\pdv{\phi}{n} \eval{S} = g(\vb{r})\hspace{1cm} \mbox{o} \hspace{1cm} \psi(\vb{r}, t) \eval_{t=0} = h(\vb{r})
\end{align*}
entre otras.
\end{enumerate}
\subsection*{Recapitulación.}
La clasificación canónica anterior apunta hacia la conexión entre EDP y condiciones de frontera (CDF). Generalizando, podemos decir:
\begin{enumerate}[label=\alph*)]
\item Las ecuaciones elípticas satisfacen las CDF de tipo Dirichlet o Neumann o mixtas.
\item Las ecuaciones hiperbólicas satisfacen las CDF de tipo Cauchy.
\item Las ecuaciones parabólicas satisfacen las CDF de tipo la ecuación de calor.
\end{enumerate}
\newpage
\begin{landscape}
\begin{center}
\captionof{table}{Tabla que relaciona el tipo de EDP y las CDF.}
\begin{tabular}{ | c | c | c | c | c |} \hline
\multirow{3}{3.5cm}[-7pt]{\makecell{Condiciones \\ de frontera}} & \multirow{3}{3.5cm}[-7pt]{\makecell{Tipo de \\ superficie}} & \multicolumn{3}{c |}{Tipo de EDP} \\ \cline{3-5}
 & & \makecell{Elíptica} & \makecell{Hiperbólica} & \makecell{Parabólica} \\ \cline{3-5}
 & & \makecell{Laplace, Poisson \\ en $(x,y)$} & \makecell{Ecuación de onda \\ en $(x,t)$} & \makecell{Ecuación de difusión \\ en $(x,t)$} \\ \hline
\multirow{2}*{\textbf{Cauchy}} & \makecell{Superficie \\ abierta} & \makecell{Resultados sin \\ interpretación física} & \makecell{\underline{\textbf{Solución única}} \\ \underline{\textbf{estable}}} & \makecell{Demasiado \\ restrictiva} \\ \cline{2-5}
 & \makecell{Superficie \\ cerrada} & \makecell{Demasiado \\ restrictiva} & \makecell{Demasiado \\ restrictiva} & \makecell{Demasiado \\ restrictiva} \\ \hline
 \multirow{2}*{\textbf{Dirichlet}} & \makecell{Superficie \\ abierta} & \makecell{Insuficiente} & \makecell{Insuficiente}  &\makecell{\underline{\textbf{Solución única}} \\ \underline{\textbf{estable} en una dirección}} \\ \cline{2-5} 
& \makecell{Superficie \\ cerrada} & \makecell{\underline{Solución única} \\ \underline{estable}} & \makecell{Más de \\ una solución} & \makecell{Demasiado \\ restrictiva} \\ \hline
 \multirow{2}*{\textbf{Neumann}} & \makecell{Superficie \\ abierta} & \makecell{Insuficiente} & \makecell{Insuficiente}  &\makecell{\underline{\textbf{Solución única}} \\ \underline{\textbf{estable} en una dirección}} \\ \cline{2-5}
& \makecell{Superficie \\ cerrada} & \makecell{\underline{\textbf{Solución única}} \\ \underline{\textbf{estable}}} & \makecell{Más de \\ una solución} & \makecell{Demasiado \\ restrictiva} \\ \hline
\end{tabular}
\end{center}
\end{landscape}
\end{document}