\documentclass[12pt]{article}
\usepackage[utf8]{inputenc}
\usepackage[spanish,es-lcroman, es-tabla]{babel}
\usepackage[autostyle,spanish=mexican]{csquotes}
\usepackage{amsmath}
\usepackage{amssymb}
\usepackage{nccmath}
\numberwithin{equation}{section}
\usepackage{amsthm}
\usepackage{graphicx}
\usepackage{epstopdf}
\DeclareGraphicsExtensions{.pdf,.png,.jpg,.eps}
\usepackage{color}
\usepackage{float}
\usepackage{multicol}
\usepackage{enumerate}
\usepackage[shortlabels]{enumitem}
\usepackage{anyfontsize}
\usepackage{anysize}
\usepackage{array}
\usepackage{multirow}
\usepackage{enumitem}
\usepackage{cancel}
\usepackage{tikz}
\usepackage{circuitikz}
\usepackage{tikz-3dplot}
\usetikzlibrary{babel}
\usepackage{bm}
\usepackage{mathtools}
\usepackage{esvect}
\usepackage{hyperref}
\usepackage{relsize}
\usepackage{siunitx}
\usepackage{physics}
%\usepackage{biblatex}
\usepackage{standalone}
\usepackage{mathrsfs}
\usepackage{bigints}
\usepackage{bookmark}
\spanishdecimal{.}

\setlist[enumerate]{itemsep=0mm}

\renewcommand{\baselinestretch}{1.5}

\let\oldbibliography\thebibliography

\renewcommand{\thebibliography}[1]{\oldbibliography{#1}

\setlength{\itemsep}{0pt}}
%\marginsize{1.5cm}{1.5cm}{2cm}{2cm}


\newtheorem{defi}{{\it Definición}}[section]
\newtheorem{teo}{{\it Teorema}}[section]
\newtheorem{ejemplo}{{\it Ejemplo}}[section]
\newtheorem{propiedad}{{\it Propiedad}}[section]
\newtheorem{lema}{{\it Lema}}[section]

\author{}
\usepackage{tikz-3dplot}
%\author{M. en C. Gustavo Contreras Mayén. \texttt{curso.fisica.comp@gmail.com}}
\title{Ecuaciones de la Física Matemática \\ {\large Matemáticas Avanzadas de la Física}\vspace{-1.5\baselineskip}}
\date{}
\author{}
\begin{document}
\maketitle
\fontsize{14}{14}\selectfont
\section{Ecuaciones de la Física Matemática.}
A continuación se menciona un conjunto de ecuaciones diferenciales de gran importancia en la física matemática:
\begin{enumerate}[label=\alph*)]
\item $\square \, u = 0$ o bien $\laplacian{u} = \dfrac{1}{c^{2}} \displaystyle \pdv[2]{u}{t}$ (ecuación de ondas)
\item $\laplacian{u} = \dfrac{1}{k} \displaystyle \pdv{u}{t}$ (ecuación de conducción del calor o de difusión)
\item $\laplacian{u} = f(x, y, z)$ (ecuación de Poisson; si $f=0$ se llama ecuación de Laplace)
\item $\laplacian{u} + \lambda \, u = 0$ (ecuación de Helmholtz)
\item $\nabla^{4} = \laplacian{(\laplacian{u})} = 0$ (ecuación biarmónica)
\item $\nabla^{4} = - \left( \dfrac{1}{p^{2}} \right) \displaystyle \pdv[2]{u}{t}$ (ecuación biarmónica de onda)
\item $\laplacian{u} + \alpha [E - V(x, y, z)] \, u = 0$ (ecuación de Schrödinger)
\item $\square \, u + \lambda^{2} \, u = 0$ (ecuación de Klein-Gordon)
\item $\displaystyle \pdv[2]{u}{x} = A \, \displaystyle \pdv[2]{u}{t} + B \, \displaystyle \pdv{u}{t} + C \, u$ (ecuación del telégrafo)
\item $(u_{x})^{2} + (u_{y})^{2} + (u_{z})^{2} =  n (x, y, z)$ (ecuación de la óptica geométrica)
\item $\displaystyle \pdv{P}{t} = \beta \, \displaystyle \pdv{x} (P \, x) +  D \displaystyle \pdv[2]{P}{x}$ (ecuación de Fokker-Planck)
\end{enumerate}
Donde el operador de D'Alambert se define como
\begin{align*}
\square \, u = \laplacian{u} - \dfrac{1}{c^{2}} \displaystyle \pdv[2]{u}{t}
\end{align*}
Además deben considerarse las ecuaciones de Lagrange de la mecánica, las ecuaciones de Maxwell del electromagnetismo, las ecuaciones de Navier-Stokes en hidrodinámica, etc.
\section{Clasificación de las EDP de segundo orden.}
La clasificación de las Ecuaciones Diferenciales Parciales de segundo orden (EDP2) es un concepto importante debido a la teoría general y métodos de solución, que normalmente funciona para una clase determinada de ecuaciones. 
\par
Veamos inicialmente la clasificación de las EDP2 que involucran dos variables independientes.
\subsection{Clasificación con dos variables independientes.}
Consideremos la EDP2 lineal general con dos variables independientes
\begin{align}
A \, \pdv[2]{u}{x} + B \, \pdv[2]{u}{x}{y} + C \, \pdv[2]{u}{y} + D \, \pdv{u}{x} + E \, \pdv{u}{y} + F \, u + G = 0
\label{eq:ecuacion_01}
\end{align}
donde $A, B, C, D, E, F, G$ son funciones de las variables independientes $x$ e $y$. La ec. (\ref{eq:ecuacion_01}) puede escribirse de la forma
\begin{align}
A \, u_{xx} + B \, u_{xy} + C \, u_{yy} + f(x, y, u_{x}, u_{y}, u) = 0
\label{eq:ecuacion_02}
\end{align}
donde
\begin{align*}
u_{x} = \pdv{u}{x}, \hspace{1cm} u_{y} = \pdv{u}{y}, \hspace{1cm} u_{xx} = \pdv[2]{u}{x}, \hspace{1cm} u_{xy} = \pdv[2]{u}{x}{y}, \hspace{1cm} u_{yy} = \pdv[2]{u}{y}
\end{align*}
Supondremos que $A$, $B$, y $C$ son funciones continuas de $x$ e $y$ y que tienen derivadas parciales continuas de orden superior en caso de requerirse.
\par
La clasificación de las EDP2 se basa en la clasificación de las ecuaciones algebraicas de segundo orden en dos variables
\begin{align}
a \, x^{2} + b \, x \, y + c \, y^{2} + d \, x + e \, y + f = 0
\label{eq:ecuacion_03}
\end{align}
Sabemos que la naturaleza de las curvas se determina por la parte principal $a \, x^{2} + b \, x \, y + c \, y^{2}$, es decir, la parte que contiene el grado más alto. Dependiendo del signo del discriminante $b^{2} - 4 \, a \, c$, se clasifican las curvas como sigue:
\par
\begin{center}
    \fbox{%
        \parbox{22em}{
            Si $b^{2} - 4 \, a \, c > 0$ entonces se traza una hipérbola. \\
            Si $b^{2} - 4 \, a \, c = 0$ entonces se traza una parábola. \\
            Si $b^{2} - 4 \, a \, c < 0$ entonces se traza una elipse.
        }
    }
\end{center}
Con una adecuada transformación, podemos llevar la ec. (\ref{eq:ecuacion_03}) a una de las siguientes formas normales
\begin{align*}
\dfrac{x^{2}}{a^{2}} - \dfrac{y^{2}}{b^{2}} =  1 \hspace{1cm} &\text{hipérbola} \\[1em]
x^{2} = y \hspace{1cm} &\text{parábola} \\[1em]
\dfrac{x^{2}}{a^{2}} + \dfrac{y^{2}}{b^{2}} =  1 \hspace{1cm} &\text{elipse}
\end{align*}
\end{document}