\documentclass[12pt]{beamer}
\usepackage{../Estilos/BeamerMAF}
\input{../Preambulos/preambulo_Beamer_Warsaw_seahorse}
\makeatletter
\setbeamertemplate{footline}
{
  \leavevmode%
  \hbox{%
  \begin{beamercolorbox}[wd=.333333\paperwidth,ht=2.25ex,dp=1ex,center]{section in foot}%
    \usebeamerfont{section in foot} \insertsection
  \end{beamercolorbox}%
  \begin{beamercolorbox}[wd=.333333\paperwidth,ht=2.25ex,dp=1ex,center]{subsection in foot}%
    \usebeamerfont{subsection in foot}  \insertsubsection
  \end{beamercolorbox}%
  \begin{beamercolorbox}[wd=.333333\paperwidth,ht=2.25ex,dp=1ex,right]{date in head/foot}%
    \usebeamerfont{date in head/foot} \insertshortdate{} \hspace*{2em}
    \insertframenumber{} / \inserttotalframenumber \hspace*{2ex} 
  \end{beamercolorbox}}%
  \vskip0pt%
}
\makeatother
\date{24 de septiembre de 2021}
\title{Tensores especiales}
\subtitle{La física y la geometría}
\begin{document}
\maketitle
\fontsize{14}{14}\selectfont
\spanishdecimal{.}

%Ref. Heinbockel (1996) Introduction to tensor calculus and continuum mechanics.
\section*{Contenido}
\frame[allowframebreaks]{\tableofcontents[currentsection, hideallsubsections]}

\section{Introducción}
\frame[allowframebreaks]{\tableofcontents[currentsection, hideothersubsections]}
\subsection{Tensores especiales}

\begin{frame}
\frametitle{Introducción}
Saber cómo se definen los tensores y reconocer un tensor cuando se nos presenta son dos cosas diferentes.
\\
\bigskip
\pause
Algunas cantidades, que son tensores, aparecen con frecuencia en problemas aplicados y debemos aprender a reconocer estos tensores especiales cuando ocurren.
\end{frame}

\section{Midiendo distancias}
\frame{\tableofcontents[currentsection, hideothersubsections]}
\subsection{Distancia entre dos puntos}

\begin{frame}
\frametitle{Distancia entre dos puntos}
Definimos $y^{i}, i = 1, \ldots, N$ como coordenadas independientes en un sistema de coordenadas cartesiano ortogonal $N$ dimensional. \pause La distancia al cuadrado entre dos puntos $y^{i}$ e $y^{i} + \dd{y}^{i}, i = 1, \ldots, N$ está definida por la expresión:
\pause
\begin{eqnarray}
\begin{aligned}
\dd{s}^{2} &= \dd{y}^{m} \, \dd{y}^{m} = \\[0.5em] \pause
&= (\dd{y}^{1})^{2} + (\dd{y}^{2})^{2} + \ldots + (\dd{y}^{N})^{2}
\end{aligned}
\label{eq:ecuacion_01_03_01}
\end{eqnarray}
\end{frame}
\begin{frame}
\frametitle{Relación con coordenadas generalizadas}
Suponemos que las coordenadas $y^{i}$ están relacionadas con un conjunto de coordenadas independientes generalizadas $x^{i}, i = 1, \ldots, N$ mediante un conjunto de ecuaciones de transformación:
\pause
\begin{align}
y^{i} = y^{i} (x^{1}, x^{2}, \ldots, x^{N}), \hspace{1cm} i = 1, \ldots, N
\label{eq:ecuacion_01_03_02}
\end{align}
\pause
Para enfatizar que cada $y^{i}$ depende de las coordenadas $x$, a veces usamos la notación $y^{i} = y^{i} (x)$, para $i = 1, \ldots, N$.
\end{frame}
\begin{frame}
\frametitle{Diferencial de la coordenada}
El diferencial de cada coordenada se puede escribir como:
\pause
\begin{align}
\dd{y}^{m} = \pdv{y^{m}}{x^{j}} \dd{x}^{j}, \hspace{1cm} m = 1, \ldots, N
\label{eq:ecuacion_01_03_03}
\end{align}
\end{frame}
\begin{frame}
\frametitle{Distancia al cuadrado}
En consecuencia, en las coordenadas generalizadas $x$, la distancia al cuadrado, calculada a partir de la ec. (\ref{eq:ecuacion_01_03_03}), se convierte en una forma cuadrática.
\\
\bigskip
\pause
Sustituyendo la ec. (\ref{eq:ecuacion_01_03_03}) en la ec. (\ref{eq:ecuacion_01_03_01}) encontramos que:
\pause
\begin{align}
\dd{s}^{2} = \pdv{y^{m}}{x^{i}} \, \pdv{y^{m}}{x^{j}} \dd{x}^{i} \dd{x}^{j} = g_{ij} \dd{x}^{i} \dd{x}^{j}
\label{eq:ecuacion_01_03_04}
\end{align}
\end{frame}

\subsection{El tensor métrico}

\begin{frame}
\frametitle{El tensor métrico}
\begin{align*}
\dd{s}^{2} = g_{ij} \dd{x}^{i} \dd{x}^{j}
\end{align*}
En donde la cantidad:
\pause
\begin{align}
g_{ij} = \pdv{y^{m}}{x^{i}} \, \pdv{y^{m}}{x^{j}}
\end{align}
\pause
es llamada \textcolor{blue}{el tensor métrico} del espacio definido por las coordenadas $x^{i}$ donde \hfill \break $i = 1, \ldots, N$.
\end{frame}
\begin{frame}
\frametitle{Propiedad del tensor métrico}
Se tiene que $g_{ij}$ son \emph{funciones de las coordenadas} $x$, a veces, se escriben como $g_{ij} = g_{ij} (x)$.
\\
\bigskip
\pause
Además, las métricas $g_{ij}$ son \emph{simétricas en los índices} $i$ y $j$, de modo que $g_{ij} = g_{ji} \hspace{0.3cm} \forall \, i, j$ en el rango de los índices.
\end{frame}
\begin{frame}
\frametitle{Longitud de arco}
Si transformamos a otro sistema de coordenadas, digamos $\overline{x}^{i}, i = 1, \ldots, N$, entonces el elemento de longitud de arco al cuadrado se expresa en términos de las coordenadas barradas:
\pause
\begin{align*}
\dd{s}^{2} = \overline{g}_{ij} \dd{\overline{x}}^{i} \dd{\overline{x}}^{j}
\end{align*}
donde $\overline{g}_{ij} = \overline{g}_{ij} (\overline{x})$ es una función de las coordenadas barradas.
\end{frame}

\section{Coordenadas curvilíneas}
\frame{\tableofcontents[currentsection, hideothersubsections]}
\subsection{Reglas de transformación}

\begin{frame}
\frametitle{Reglas de transformación}
Considera un conjunto de ecuaciones de transformación generales desde coordenadas rectangulares $(x, y, z)$ a coordenadas curvilíneas $(u, v, w)$.
\end{frame}
\begin{frame}
\frametitle{Reglas de transformación}
Estas ecuaciones de transformación y las correspondientes transformaciones inversas se representan como:
\pause
\begin{align}
\begin{aligned}
x &= x(u, v, w) \hspace{1.3cm} u = u(x, y, z) \\[0.5em]
y &= y(u, v, w) \hspace{1.3cm} v = v(x, y, z) \\[0.5em]
z &= z(u, v, w) \hspace{1.3cm} w = w(x, y, z)
\end{aligned}
\label{eq:ecuacion_01_03_10}
\end{align}
\end{frame}
\begin{frame}
\frametitle{Identificando coordenadas}
Aquí: 
\pause
\begin{align*}
y^{1} = x, \hspace{0.5cm} y^{2} = y, \hspace{0.5cm} y^{3} = z
\end{align*}
son las coordenadas cartesianas, \pause mientras que:
\begin{align*}
x^{1} = u, \hspace{0.5cm} x^{2} = v, \hspace{0.5cm} x^{3} = w
\end{align*}
son las coordenadas generalizadas y $N = 3$.
\end{frame}
\begin{frame}
\frametitle{Superficies coordenadas}
La intersección de las superficies coordenadas $u = c_{1}, v = c_{2}$ y $w = c_{3}$ define las curvas coordenadas del sistema de coordenadas curvilíneas.
\end{frame}

\subsection{Vector de posición}

\begin{frame}
\frametitle{Vector de posición}
La sustitución de las ecuaciones de transformación dadas en la ec. (\ref{eq:ecuacion_01_03_10}) en el vector de posición $\va{r} = x\, \vu{e}_{1} + y\, \vu{e}_{2} + z\, \vu{e}_{3}$ genera el vector de posición que es una función de las coordenadas generalizadas y:
\pause
\begin{align*}
\va{r} &= \va{r} (u, v, w) = x (u, v, w) \, \vu{e}_{1} + y (u, v, w) \, \vu{e}_{2} + \\[0.5em]
&+ z (u, v, w) \, \vu{e}_{3}
\end{align*}
\end{frame}
\begin{frame}
\frametitle{Vector de posición}
En consecuencia:
\pause
\begin{align*}
\dd{\va{r}} = \pdv{\va{r}}{u} \dd{u} + \pdv{\va{r}}{v} \dd{v} + \pdv{\va{r}}{w} \dd{w}
\end{align*}
\end{frame}
\begin{frame}
\frametitle{Componentes del vector de posición}
Donde:
\pause
\begin{align}
\begin{aligned}
\va{E}_{1} &= \pdv{\va{r}}{u} = \pdv{x}{u} \, \vu{e}_{1} + \pdv{y}{u} \, \vu{e}_{2} + \pdv{z}{u} \, \vu{e}_{3} \\[0.5em]
\va{E}_{2} &= \pdv{\va{r}}{v} = \pdv{x}{v} \, \vu{e}_{1} + \pdv{y}{v} \, \vu{e}_{2} + \pdv{z}{v} \, \vu{e}_{3} \\[0.5em]
\va{E}_{3} &= \pdv{\va{r}}{w} = \pdv{x}{w} \, \vu{e}_{1} + \pdv{y}{w} \, \vu{e}_{2} + \pdv{z}{w} \, \vu{e}_{3}
\end{aligned}
\label{eq:ecuacion_01_03_11}
\end{align}
son los vectores tangentes a las curvas coordenadas.
\end{frame}

\subsection{Longitud de arco}

\begin{frame}
\frametitle{Longitud de arco}
El elemento de longitud de arco en las coordenadas curvilíneas es:
\pause
\begin{equation}
\begin{aligned}
&\dd{s}^{2} = \dd{\va{r}} \cdot \dd{\va{r}} = \\[0.5em] \pause
&= \pdv{\va{r}}{u} \cdot \pdv{\va{r}}{u} \dd{u} \dd{u} {+} \pdv{\va{r}}{u} \cdot \pdv{\va{r}}{v} \dd{u} \dd{v} {+} \pdv{\va{r}}{u} \cdot \pdv{\va{r}}{w} \dd{u} \dd{w} + \\[0.5em]
&+ \pdv{\va{r}}{v} \cdot \pdv{\va{r}}{u} \dd{v} \dd{u} {+} \pdv{\va{r}}{v} \cdot \pdv{\va{r}}{v} \dd{v} \dd{v} {+} \pdv{\va{r}}{v} \cdot \pdv{\va{r}}{w} \dd{v} \dd{w} + \\[0.5em]
&+ \pdv{\va{r}}{w} \cdot \pdv{\va{r}}{u} \dd{w} \dd{u} {+} \pdv{\va{r}}{w} \cdot \pdv{\va{r}}{v} \dd{w} \dd{v} {+} \pdv{\va{r}}{w} \cdot \pdv{\va{r}}{w} \dd{w} \dd{w}
\end{aligned}
\label{eq:ecuacion_01_03_12}
\end{equation}
\end{frame}
\begin{frame}
\frametitle{Usando notación de índices}
Utilizando la convención para la suma, la expresión anterior la podemos escribir con notación de índices.
\end{frame}
\begin{frame}
\frametitle{Usando notación de índices}
Por lo que se definen las cantidades:
\pause
\begin{align*}
g_{11} &= \pdv{\va{r}}{u} \cdot \pdv{\va{r}}{u} \hspace{0.8cm} g_{12} = \pdv{\va{r}}{u} \cdot \pdv{\va{r}}{v} \hspace{0.8cm} g_{13} = \pdv{\va{r}}{u} \cdot \pdv{\va{r}}{w} \\[0.5em]
g_{21} &= \pdv{\va{r}}{v} \cdot \pdv{\va{r}}{u} \hspace{0.8cm} g_{22} = \pdv{\va{r}}{v} \cdot \pdv{\va{r}}{v} \hspace{0.8cm} g_{23} = \pdv{\va{r}}{v} \cdot \pdv{\va{r}}{w} \\[0.5em]
g_{31} &= \pdv{\va{r}}{w} \cdot \pdv{\va{r}}{u} \hspace{0.8cm} g_{32} = \pdv{\va{r}}{w} \cdot \pdv{\va{r}}{v} \hspace{0.8cm} g_{33} = \pdv{\va{r}}{w} \cdot \pdv{\va{r}}{w}
\end{align*}
y haciendo que: $x^{1} = u$, $x^{2} = v$, $x^{3} = w$.
\end{frame}
\begin{frame}
\frametitle{Longitud de arco}
Entonces el elemento de longitud de arco de la ec. (\ref{eq:ecuacion_01_03_12}), se puede escribir como:
\pause
\begin{align*}
\dd{s}^{2} = \va{E}_{i} \cdot \va{E}_{j} \dd{x}^{i} \dd{x}^{j} = g_{ij} \dd{x}^{i} \dd{x}^{j}, \hspace{0.8cm} i, j = 1, 2, 3
\end{align*}
\pause
donde:
\pause
\begin{align}
\begin{aligned}
g_{ij} = \va{E}_{i} \cdot \va{E}_{j} &= \pdv{\va{r}}{x^{i}} \cdot \pdv{\va{r}}{x^{j}} = \pdv{y^{m}}{x^{i}} \, \pdv{y^{m}}{x^{j}} \\[0.5em]
&\mbox{con \quad} i, j \quad \mbox{ índices libres}
\end{aligned}
\label{eq:ecuacion_01_03_13}
\end{align}
son llamados \emph{componentes métricos del sistema coordenado curvilíneo}.
\end{frame}

\subsection{Matriz métrica}

\begin{frame}
\frametitle{Elemento de la matriz métrica}
Los componentes métricos pueden considerarse como los elementos de una matriz simétrica, ya que $g_{ij} = g_{ji}$.
\end{frame}
\begin{frame}
\frametitle{Coordenadas rectangulares}
En el sistema de coordenadas rectangular $x, y, z$, el elemento de la longitud del arco al cuadrado es $\dd{s}^{2} = \dd{x}^{2} + \dd{y}^{2} + \dd{z}^{2}$.
\pause
En este espacio los componentes métricos son:
\begin{align*}
g_{ij} = \mqty(
1 & 0 & 0 \\
0 & 1 & 0 \\
0 & 0 & 1 )
\end{align*}
\end{frame}
\begin{frame}
\frametitle{Coordenadas cilíndricas}
\textbf{Ejemplo: } Coordenadas cilíndricas $(r, \theta, z)$.
La regla de transformación de coordenadas cartesianas a coordenadas cilíndricas está dada por:
\par
\begin{align*}
x = r \, \cos \theta, \hspace{1cm} y = r \, \sin \theta, \hspace{1cm} z = z
\end{align*}
\end{frame}
\begin{frame}
\frametitle{Calculando el tensor métrico}
Hacemos que:
\pause
\begin{align*}
y^{1} = x, \hspace{1cm} y^{2} = y, \hspace{1cm} y^{3} = z
\end{align*}
\pause
y también que:
\begin{align*}
x^{1} = r, \hspace{1cm} x^{2} = \theta, \hspace{1cm} x^{3} = z
\end{align*}
\end{frame}
\begin{frame}
\frametitle{Vector de posición}
Por lo que el vector de posición se puede escribir de la forma:
\pause
\begin{align*}
\va{r} = \va{r} (r, \theta, z) = r \, \cos \theta \, \vu{e}_{1} + r \, \sin \theta \, \vu{e}_{2} + z \, \vu{e}_{3}
\end{align*}
\end{frame}
\begin{frame}
\frametitle{Vector de posición}
Calculamos las derivadas de ese vector de posición y encontramos que:
\pause
\begin{align*}
\va{E}_{1} &= \pdv{\va{r}}{r} = \cos \theta \, \vu{e}_{1} + \sin \theta \, \vu{e}_{2} \\[0.5em]
\va{E}_{2} &= \pdv{\va{r}}{\theta} = -r \, \sin \theta \, \vu{e}_{1} + r \, \cos \theta \, \vu{e}_{2} \\[0.5em]
\va{E}_{3} &= \pdv{\va{r}}{z} = \vu{e}_{3}
\end{align*}
\end{frame}
\begin{frame}
\frametitle{Componentes métricos}
Del resultado en la ec. (\ref{eq:ecuacion_01_03_13}), los componentes métricos de este espacio son:
\pause
\begin{align*}
g_{ij} = \pdv{y^{m}}{x^{i}} \, \pdv{y^{m}}{x^{j}} \hspace{1cm} i, j \quad \mbox{ índices libres}
\end{align*}
\end{frame}
\begin{frame}
\frametitle{Componentes métricos}
Entonces tendremos que:
\pause
\begin{eqnarray*}
g_{11} &=& \pdv{y^{1}}{x^{1}} \, \pdv{y^{1}}{x^{1}} + \pdv{y^{2}}{x^{1}} \, \pdv{y^{2}}{x^{1}} + \pdv{y^{3}}{x^{1}} \, \pdv{y^{3}}{x^{1}} = \\[0.5em] \pause
g_{11} &=& \cos \theta \, \cos \theta + \pause \sin \theta \, \sin \theta = \pause \cos^{2} \theta + \sin^{2} \theta = \pause 1 \\[0.5em] \pause
g_{12} &=& \pdv{y^{1}}{x^{1}} \, \pdv{y^{1}}{x^{2}} + \pdv{y^{2}}{x^{1}} \, \pdv{y^{2}}{x^{2}} + \pdv{y^{3}}{x^{1}} \, \pdv{y^{3}}{x^{2}} = \\[0.5em] \pause
g_{12} &=& - r \, \cos \theta \, \sin \theta + r \, \cos \theta \, \sin \theta = 0 \\[0.5em] \pause
g_{13} &=& 0
\end{eqnarray*}
\end{frame}
\begin{frame}
\frametitle{Componentes métricos}
\begin{eqnarray*}
g_{21} &=& - r \, \sin \theta \, \cos \theta + r \, \cos \theta \, \sin \theta = 0 \\[0.5em] \pause
g_{22} &=& (-r \, \sin \theta)(- r \, \sin \theta) + (r \, \cos \theta \, \cos \theta) = \\[0.5em]
&=& r^{2} \, (\sin^{2} \theta + \cos \theta) = r^{2} \\[0.5em] \pause
g_{23} &=& (- r \, \sin \theta)(0) + (r \, \cos \theta)(0) + (0)(1) = 0 \\[0.5em]
\end{eqnarray*}
\end{frame}
\begin{frame}
\frametitle{Componentes métricos}
\begin{eqnarray*}
g_{31} &=& (0)(\cos \theta) + 0 (\sin \theta) + (1)(0) = 0 \\[0.5em]
g_{32} &=& (0)(- r \, \cos \theta) + (0)(r \cos \theta) + (1)(0) = 0 \\[0.5em]
g_{33} &=& 0 + 0 + (1)(1) = 1
\end{eqnarray*}
\end{frame}
\begin{frame}
\frametitle{El tensor métrico}
Entonces el tensor métrico es:
\pause
\begin{align*}
g_{ij} = \mqty(
1 & 0 & 0 \\
0 & r^{2} & 0 \\
0 & 0 & 1 )
\end{align*}
\end{frame}
\begin{frame}
\frametitle{Longitud de arco}
Por lo que el diferencial de longitud de arco en el sistema coordenado cilíndrico es:
\pause
\begin{align*}
\dd{s}^{2} = g_{ij} \, \dd{x}^{i} \dd{x}^{j}
\end{align*}
\end{frame}  
\begin{frame}
\frametitle{Longitud de arco}
Así llegamos a:
\pause
\begin{align*}
\dd{s}^{2} = (\dd{r})^{2} + r^{2} \, (\dd{\theta})^{2} + (\dd{z})^{2}
\end{align*}
\end{frame}
\begin{frame}
\frametitle{Comentario relevante}
Si $\dd{s}^{2} > 0$, al elemento de longitud de arco que es un valor real, se le conoce como \textcolor{blue}{distancia elemental}.
\\
\bigskip
\pause
Pero si $\dd{s}^{2} < 0$ como en el espacio de Minkowski, al elemento de longitud de arco es un valor complejo, se le conoce como \textcolor{red}{intervalo elemental.}
\end{frame}
\begin{frame}
\frametitle{Punto importante}
Notemos que mientras $g_{ij} = 0$ cuando $i \neq j$, el sistema coordenado es ortogonal.
\end{frame}    
\begin{frame}
\frametitle{Componentes métricos}
Dado un conjunto de transformaciones de la forma encontrada en la ec. (\ref{eq:ecuacion_01_03_10}), se pueden calcular fácilmente los componentes métricos asociados con las coordenadas generalizadas.
\end{frame}
\begin{frame}
\frametitle{Componentes métricos sistema ortogonal}
Los componentes métricos de un sistema ortogonal tienen la forma:
\pause
\begin{align*}
g_{ij} = \mqty(
h_{1}^{2} & 0 & 0 \\
0 & h_{2}^{2} & 0 \\
0 & 0 & h_{3}^{2})
\end{align*}
donde los $h_{i}$ son los llamados \emph{factores de escala}.
\end{frame}
\begin{frame}
\frametitle{Longitud de arco}
Entonces el elemento de longitud de arco al cuadrado es:
\pause
\begin{align*}
\dd{s}^{2} = h_{1}^{2} \, (\dd{x}^{1})^{2} + h_{2}^{2} \, (\dd{x}^{2})^{2} + h_{3}^{2} \, (\dd{x}^{3})^{2}
\end{align*}
\end{frame}

\subsection{Otras medidas}

\begin{frame}
\frametitle{Elemento de área}
El elemento de área en la superficie generada por $x^{1} \, x^{2}$, está dado por:
\pause
\begin{eqnarray*}
\dd{A} &=& (g_{11})^{\frac{1}{2}} \, (g_{22})^{\frac{1}{2}} \, \dd{x}^{1} \dd{x}^{2} = \\[0.5em] \pause 
&=& (g_{11} \, g_{22})^{\frac{1}{2}} \, \dd{x}^{1} \dd{x}^{2}
\end{eqnarray*}
\end{frame}
\begin{frame}
\frametitle{Elemento de volumen}
El elemento de volumen se obtiene a partir de:
\begin{eqnarray*}
\dd{V} &=& (g_{11} \, g_{22} \, g_{33})^{\frac{1}{2}} \dd{x}^{1} \dd{x}^{2} \dd{x}^{3} = \\[0.5em] \pause
&=& g^{\frac{1}{2}} \dd{x}^{1} \dd{x}^{2} \dd{x}^{3}
\end{eqnarray*}
\end{frame}
\begin{frame}
\frametitle{Para ir practicando}
Trabajaremos con un sistema coordenado cónico, tal que:
\pause
\begin{align*}
y^{1} = x, \hspace{1cm} y^{2} = y, \hspace{1cm} y^{3} = z
\end{align*}
\pause
junto con:
\begin{align*}
x^{1} &= r \hspace{1.5cm} 0 \leq r < \infty \\[0.35em]
x^{2} &= \theta \hspace{1.5cm} b^{2} < \theta < c^{2} \\[0.35em]
x^{3} &= \lambda \hspace{1.5cm} 0 < \lambda < b^{2}
\end{align*}
donde $c^{2} > \theta^{2} > b^{2} > \lambda^{2} > 0$
\end{frame}
\begin{frame}
\frametitle{Reglas de transformación}
Además:
\begin{align*}
x &= \dfrac{r \, \theta \, \lambda}{b \, c} \\[0.5em]
y &= \dfrac{r}{b} \, \sqrt{\dfrac{(\theta^{2} - b^{2})(\lambda^{2} - b^{2})}{(b^{2} - c^{2})}} \\[0.5em]
z &= \dfrac{r}{c} \, \sqrt{\dfrac{(\theta^{2} - c^{2})(\lambda^{2} - c^{2})}{(c^{2} - b^{2})}}
\end{align*}
\pause
Y nos interesa conocer el tensor métrico de este sistema coordenado.
\end{frame}
\end{document}