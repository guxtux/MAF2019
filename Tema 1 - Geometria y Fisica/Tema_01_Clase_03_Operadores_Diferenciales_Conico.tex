\documentclass[12pt]{beamer}
\usepackage{../Estilos/BeamerMAF}
\input{../Preambulos/preambulo_Beamer_Warsaw_seahorse}

\setbeamercolor{section in foot}{bg=alizarin, fg=white}
\setbeamercolor{subsection in foot}{bg=brown, fg=white}
\setbeamercolor{date in foot}{bg=goldenrod, fg=white}

\makeatletter
\setbeamertemplate{footline}
{
  \leavevmode%
  \hbox{%
  \begin{beamercolorbox}[wd=.333333\paperwidth,ht=2.25ex,dp=1ex,center]{section in foot}%
    \usebeamerfont{section in foot} \insertsection
  \end{beamercolorbox}%
  \begin{beamercolorbox}[wd=.333333\paperwidth,ht=2.25ex,dp=1ex,center]{subsection in foot}%
    \usebeamerfont{subsection in foot}  \insertsubsection
  \end{beamercolorbox}%
  \begin{beamercolorbox}[wd=.333333\paperwidth,ht=2.25ex,dp=1ex,right]{date in head/foot}%
    \usebeamerfont{date in head/foot} \insertshortdate{} \hspace*{2em}
    \insertframenumber{} / \inserttotalframenumber \hspace*{2ex} 
  \end{beamercolorbox}}%
  \vskip0pt%
}
\makeatother

\date{22 de febrero de 2022}
\title{El sistema coordenado cónico - 2}
\subtitle{La física y la geometría}
\begin{document}
\maketitle
\fontsize{14}{14}\selectfont
\spanishdecimal{.}

\section*{Contenido}
\frame[allowframebreaks]{\frametitle{Contenido} \tableofcontents[currentsection, hideallsubsections]}

\section{Operadores diferenciales}
\frame{\tableofcontents[currentsection, hideothersubsections]}

\subsection{Gradiente}

\begin{frame}
\frametitle{El operador gradiente}
En un sistema coordenado generalizado $(q_{1}, q_{2}, q_{3})$, escribimos el gradiente de una función escalar $\psi (q_{1}, q_{2}, q_{3})$ como:
\pause
\begin{eqnarray*}
\begin{aligned}
\grad{\phi} &= \dfrac{1}{h_{1}} \, \vu{q}_{1} \, \pdv{\psi}{q_{1}} + \dfrac{1}{h_{2}} \, \vu{q}_{2} \, \pdv{\psi}{q_{2}} + \dfrac{1}{h_{3}} \, \vu{q}_{3} \, \pdv{\psi}{q_{3}} = \\[0.5em] \pause
&= \nsum_{i} \vu{q}_{i} \, \dfrac{1}{h_{i}} \, \pdv{\psi}{q_{i}}
\end{aligned}
\end{eqnarray*}
\end{frame}

\subsection{Divergencia}

\begin{frame}
\frametitle{Expresión para simplificar la escritura}
Definimos la cantidad:
\pause
\begin{align*}
h = \dfrac{1}{h_{1} \, h_{2} \, h_{3}}
\end{align*}
que nos servirá para abreviar la expresión.
\end{frame}
\begin{frame}
\frametitle{El operador divergencia}
La divergencia de un vector $\vb{B}$ se expresa como:
\pause
\begin{align*}
\div{\vb{B}} &= h \, \bigg[ \pdv{q_{1}} \bigg( B_{1} \, h_{2} \, h_{3} \bigg) + \\[0.5em]
&+ \pdv{q_{2}} \bigg( B_{2} \, h_{3} \, h_{1} \bigg) + \pdv{q_{3}} \bigg( B_{3} \, h_{1} \, h_{2} \bigg) \bigg]
\end{align*}
\end{frame}

\subsection{Rotacional}

\begin{frame}
\frametitle{El rotacional}
El operador rotacional de un vector $\vb{B}$ es:
\pause
\begin{align*}
\curl{\vb{B}} = \mqty|
\vu{q}_{1} \, h_{1} & \vu{q}_{2} \, h_{2} & \vu{q}_{3} \, h_{3} \\[1em]
\displaystyle \pdv{q_{1}} & \displaystyle \pdv{q_{2}} & \displaystyle \pdv{q_{3}} \\[1em]
h_{1} \, B_{1} & h_{2} \, B_{2} & h_{3} \, B_{3}
|
\end{align*}
\end{frame}

\subsection{Laplaciano}

\begin{frame}
\frametitle{El Laplaciano}
El operador Laplaciano de una función $\psi (q_{1}, q_{2}, q_{3})$ en un sistema coordenado generalizado, se define por la expresión:
\pause
\begin{align*}
\laplacian{\psi} &= \dfrac{1}{h} \bigg[ \pdv{q_{1}} \bigg( \dfrac{h_{2} h_{3}}{h_{1}} \, \pdv{\psi}{q_{1}} \bigg) + \\[0.5em]
&+ \pdv{q_{2}} \bigg( \dfrac{h_{3} h_{1}}{h_{2}} \, \pdv{\psi}{q_{2}} \bigg) + \pdv{q_{3}} \bigg( \dfrac{h_{1} h_{2}}{h_{3}} \, \pdv{\psi}{q_{3}} \bigg) \bigg]
\end{align*}
\end{frame}

\section{Aplicación: La ecuación de Helmholtz}
\frame{\tableofcontents[currentsection, hideothersubsections]}
\subsection{Construyendo la ecuación}

\begin{frame}
\frametitle{La ecuación de Helmholtz}
Sabemos que la ecuación de Helmholtz se expresa por:
\pause
\begin{align*}
\laplacian{E} + k \, E = 0
\end{align*}
donde $k$ es el número de onda, $E$ es la función escalar que es solución a la ecuación.
\\
\bigskip
\pause
Expresemos esta ecuación en el sistema coordenado cónico.
\end{frame}
\begin{frame}
\frametitle{El Laplaciano en coord. cónicas}
Ya conocemos la expresión para el Laplaciano en coordenadas generalizadas:
\pause
\begin{align*}
\laplacian{E} &= \dfrac{1}{h} \bigg[ \pdv{q_{1}} \bigg( \dfrac{h_{2} h_{3}}{h_{1}} \, \pdv{E}{q_{1}} \bigg) + \\[0.5em]
&+ \pdv{q_{2}} \bigg( \dfrac{h_{3} h_{1}}{h_{2}} \, \pdv{E}{q_{2}} \bigg) + \pdv{q_{3}} \bigg( \dfrac{h_{1} h_{2}}{h_{3}} \, \pdv{E}{q_{3}} \bigg) \bigg]
\end{align*}
\end{frame}
\begin{frame}
\frametitle{El Laplaciano en coord. cónicas}
Como ya conocemos los factores de escala:
\pause
\begin{align*}
h_{r} &= 1 \\[0.5em]
h_{\theta} &=  r \, \sqrt{\dfrac{(\theta^{2} - \lambda^{2})}{(\theta^{2} - b^{2})(c^{2} - \theta^{2})}} \\[0.5em]
h_{\lambda} &= r \, \sqrt{\dfrac{(\theta^{2} - \lambda^{2})}{(\lambda^{2} - b^{2})(\lambda^{2} - c^{2})}}
\end{align*}
Nos resta más que sustituir aquéllos y hacer un pequeño manejo.
\end{frame}
\begin{frame}
\frametitle{Simplificando las operaciones}
Con la finalidad de simplificar las operaciones, hagamos que:
\pause
\begin{eqnarray*}
\begin{aligned}
f(\theta) &= \sqrt{(\theta^{2} - b^{2})(c^{2} - \theta^{2})} \\[0.5em] \pause
f(\lambda) &= \sqrt{(\lambda^{2} - b^{2})(\lambda^{2} - c^{2})}
\end{aligned}
\end{eqnarray*}
\end{frame}
\begin{frame}
\frametitle{Valores con los factores de escala}
Calculemos los cocientes que involucran los factores de escala:
\pause
\begin{eqnarray*}
\begin{aligned}
\dfrac{1}{h_{1} \, h_{2} \, h_{3}} &= \dfrac{1}{\dfrac{r^{2} (\theta^{2} - \lambda^{2})}{f(\theta) \, f(\lambda)}} = \\[0.5em] \pause
&= \dfrac{f(\theta) \, f(\lambda)}{r^{2} (\theta^{2} - \lambda^{2})}
\end{aligned}
\end{eqnarray*}
\end{frame}
\begin{frame}
\frametitle{Las otras cantidades}
\begin{align*}
\dfrac{h_{2} \, h_{3}}{h_{1}} = \dfrac{r^{2} (\theta^{2} - \lambda^{2})}{f(\theta) \, f(\lambda)}
\end{align*}
\pause
\begin{eqnarray*}
\begin{aligned}
\dfrac{h_{3} \, h_{1}}{h_{2}} = \pause \dfrac{\dfrac{r \sqrt{\theta^{2} - \lambda^{2}}}{f(\lambda)}}{\dfrac{r \sqrt{\theta^{2} - \lambda^{2}}}{f(\theta)}} = \pause \dfrac{f (\theta)}{f (\lambda)}
\end{aligned}
\end{eqnarray*}
\end{frame}
\begin{frame}
\frametitle{Las otras cantidades}
\begin{eqnarray*}
\begin{aligned}
\dfrac{h_{1} \, h_{2}}{h_{3}} = \pause \dfrac{\dfrac{r \sqrt{\theta^{2} - \lambda^{2}}}{f(\theta)}}{\dfrac{r \sqrt{\theta^{2} - \lambda^{2}}}{f(\lambda)}} = \pause \dfrac{f (\lambda)}{f (\theta)}
\end{aligned}
\end{eqnarray*}
\end{frame}
\begin{frame}
\frametitle{Expresando el Laplaciano}
El operador Laplaciano en el sistema coordenado cónico se escribe como:
\pause
\begin{align*}
\laplacian{E} &= \dfrac{f(\theta) \, f(\lambda)}{r^{2} (\theta^{2} - \lambda^{2})} \bigg[ \pdv{r} \left( \dfrac{r^{2} (\theta^{2} - \lambda^{2})}{f(\theta) \, f(\lambda)} \, \pdv{E}{r} \right) + \\[0.5em]
&+ \pdv{\theta} \left( \dfrac{f(\theta)}{f(\lambda)} \, \pdv{E}{\theta} \right) + \pdv{\lambda} \left( \dfrac{f(\lambda)}{f(\theta)} \, \pdv{E}{\lambda} \right) \bigg]
\end{align*}
\pause
Que habrá que simplificar para entonces expresar la ecuación de Helmholtz.
\end{frame}
\begin{frame}
\frametitle{La ecuación de Helmholtz}
En el sistema coordenado cónico la ecuación de Helmholtz tiene la forma:
\begin{align*}
&\dfrac{1}{r^{2}} \pdv{r} \left( r^{2} \, \pdv{E}{r} \right) + \dfrac{f(\theta)}{r^{2} (\theta^{2} - \lambda^{2})} \, \pdv{\theta} \left( f(\theta) \, \pdv{E}{\theta} \right) + \\[0.5em]
&+ \dfrac{f(\lambda)}{r^{2} (\theta^{2} - \lambda^{2})} \, \pdv{\lambda} \left( f(\lambda) \, \pdv{E}{\lambda} \right) + k^{2} \, E = 0 \qed
\end{align*}
\end{frame}
\end{document}