\input{../Preambulos/preambulo_presentacion_Warsaw_seahorse}
\title{\large{Diferenciales y operadores diferenciales}}
\subtitle{Tema 1 - La física y la geometría}
\author{M. en C. Gustavo Contreras Mayén}
\date{\today}
\institute{Facultad de Ciencias - UNAM}
\titlegraphic{\includegraphics[width=1.75cm]{../Imagenes/escudo-facultad-ciencias}\hspace*{4.75cm}~%
   \includegraphics[width=1.75cm]{../Imagenes/escudo-unam}
}
\setbeamertemplate{navigation symbols}{}
\begin{document}
\maketitle
\fontsize{14}{14}\selectfont
\spanishdecimal{.}
\section*{Contenido}
\frame{\tableofcontents[currentsection, hideallsubsections]}
\section{Diferencial de línea}
\frame{\tableofcontents[currentsection, hideothersubsections]}
\subsection{Construcción}
\begin{frame}
\frametitle{Construcción}
Utilizando el resultado
\begin{align*}
\vu{e}_{i} = \dfrac{1}{h_{i}} \, \pdv{\vb{r}}{u_{i}}
\end{align*}
en la ecuación inicial
\begin{align*}
\dd{\vb{r}} = \sum_{i=1}^{3} \pdv{\vb{r}}{u_{i}} \dd{u_{i}}
\end{align*}
\end{frame}
\begin{frame}
\frametitle{Diferencial de línea}
Es posible escribir:
\begin{align}
\dd{\vb{r}} = \sum_{i=1}^{3} h_{i} \, \vu{e}_{i} \dd{u_{i}} = \sum_{i=1}^{3} \dd{\vb{l}_{i}}
\label{eq:ecuacion_01_21}
\end{align}
\pause
donde
\begin{align}
\dd{\vb{l}_{i}} = h_{i} \, \vu{e}_{i} \dd{u_{i}}
\label{eq:ecuacion_01_22}
\end{align}
que representa el \emph{elemento diferencial de línea} a lo largo del eje $u_{i}$.
\end{frame}
\begin{frame}
\frametitle{Diferencia de línea}
La ec. (\ref{eq:ecuacion_01_22}) asegura que cualquier elemento de línea con orientación arbitraria, puede descomponerse en una suma vectorial.
\end{frame}
\subsection*{Coordenadas esféricas}
\begin{frame}
\frametitle{Coordenadas esféricas}
En coordenadas esféricas tenemos que:
\begin{table}
\begin{tabular}{r  c  l}
$\dd{\vb{l}_{r}} = \vu{e}_{r} \dd{r}$ & $\longrightarrow$ & $\dd{l_{r}} = \dd{r}$ \\
$\dd{\vb{l}_{\theta}} = \vu{e}_{\theta} \, r \dd{\theta}$ & $\longrightarrow$ & $\dd{l_{\theta}} = r \, \dd{\theta}$ \\
$\dd{\vb{l}_{\varphi}} = \vu{e}_{\varphi} \, r \, \sin \theta \dd{\varphi}$ & $\longrightarrow$ & $\dd{l_{\varphi}} = r \, \sin \theta \dd{\varphi}$ \\
\end{tabular}
\end{table}
\end{frame}
\section{Diferencial de superficie}
\frame{\tableofcontents[currentsection, hideothersubsections]}
\subsection{Construcción}
\begin{frame}
\frametitle{Construción}
Las superficies diferenciales se describen como vectores perpendicular al área diferencial, como se ve en la siguiente figura (\ref{fig:figura_diferenciales_superficie}):
\end{frame}
\begin{frame}
\frametitle{Diferenciales de superficie}
\begin{figure}[h!]
    \centering
    \includegraphics[scale=0.5]{Imagenes/Diferenciales_Superficie_01.png}
    \caption{Elementos diferenciales de área en coordenadas curvilíneas.}
    \label{fig:figura_diferenciales_superficie}
\end{figure}
\end{frame}
\begin{frame}
\frametitle{Diferenciales de superficie}
Las superficies están orientadas según la regla de la mano derecha, por lo que:
\begin{align*}
\dd{\vb{S}_{1}} &= \dd{\vb{l}_{2}} \cp \dd{\vb{l}_{3}} \\[0.5em] 
\dd{\vb{S}_{2}} &= \dd{\vb{l}_{3}} \cp \dd{\vb{l}_{1}} \\[0.5em]
\dd{\vb{S}_{3}} &= \dd{\vb{l}_{1}} \cp \dd{\vb{l}_{2}}
\end{align*}
\end{frame}
\begin{frame}
\frametitle{Usando un resultado previo}
Usando las relaciones:
\begin{align*}
\vu{e}_{1} \cp \vu{e}_{2} = \vu{e}_{3} \\
\vu{e}_{2} \cp \vu{e}_{3} = \vu{e}_{1} \\
\vu{e}_{3} \cp \vu{e}_{1} = \vu{e}_{2}
\end{align*}
que son válidas para sistemas coordenados curvilíneos ortonormales, en espacios 3D euclidianos.
\end{frame}
\begin{frame}
\frametitle{Usando un resultado previo}
Que en forma sintética queda expresado por:
\begin{align}
\vu{e}_{i} \cp \vu{e}_{j} = \sum_{i=1}^{3} \epsilon_{ijk} \, \vu{e}_{k}
\end{align}
donde $\epsilon_{ijk}$ es el símbo de \emph{Levi-Civita} que definimos en la presentación anterior.
\end{frame}
\begin{frame}
\frametitle{Diferenciales de superficie}
Así encontramos que
\begin{align*}
\dd{\vb{S}_{1}} &= h_{2} \, h_{3} \, \vu{e}_{2} \cp \vu{e}_{3} \dd{u_{2}} \dd{u_{3}} = \\[0.5em]
&= h_{2} \, h_{3} \, \vu{e}_{1} \dd{u_{2}} \dd{u_{3}} = \\[0.5em]
&= \vu{e}_{1} \dd{S_{1}}
\end{align*}
\end{frame}
\begin{frame}
\frametitle{Superficies diferenciales}
Para los otros dos diferenciales de superficie:
\fontsize{12}{12}\selectfont
\begin{eqnarray*}
\dd{\vb{S}_{2}} &=& h_{3} \, h_{1} \, \vu{e}_{3} \cp \vu{e}_{1} \dd{u_{3}} \dd{u_{1}} = \\[0.5em]
&=& h_{3} \, h_{1} \, \vu{e}_{2} \dd{u_{3}} \dd{u_{1}} = \\[0.5em]
&=& \vu{e}_{2} \dd{S_{2}} \\[1em]
\pause
\dd{\vb{S}_{3}} &=& h_{1} \, h_{2} \, \vu{e}_{1} \cp \vu{e}_{2} \dd{u_{1}} \dd{u_{2}} = \\[0.5em]
&=& h_{1} \, h_{2} \, \vu{e}_{3} \dd{u_{1}} \dd{u_{2}} = \\[0.5em]
&=& \vu{e}_{3} \dd{S_{3}}
\end{eqnarray*}
\end{frame}
\section{Diferencial de volumen}
\frame{\tableofcontents[currentsection, hideothersubsections]}
\subsection{Construcción}
\begin{frame}
\frametitle{Construcción}
El elemento diferencial de volumen se define como
\begin{align*}
\dd{V} &= \dd{\vb{l}_{1}} \cdot \dd{\vb{l}_{2}} \cp \dd{\vb{l}_{3}} = \\
&= h_{1} \, h_{2} \, h_{3} \, \vu{e}_{1} \cdot \vu{e}_{2} \cp \vu{e}_{3} \dd{u_{1}} \dd{u_{2}} \dd{u_{3}} = \\
&= h_{1} \, h_{2} \, h_{3} \dd{u_{1}} \dd{u_{2}} \dd{u_{3}}
\end{align*}
\end{frame}
\subsection*{Coordenadas esféricas}
\begin{frame}
\frametitle{Coordenadas esféricas}
En coordenadas esféricas tenemos:
\begin{align*}
\dd{S_{1}} &= \dd{S_{r}} = h_{\theta} \, h_{\varphi} \dd{\theta} \dd{\varphi} = r^{2} \sin \theta \dd{\theta} \dd{\varphi} \\[0.5em]
\dd{S_{2}} &= \dd{S_{\theta}} = h_{\varphi} \, h_{r} \dd{\varphi} \dd{r} = r \sin \theta \dd{\theta} \dd{\varphi} \\[0.5em]
\dd{S_{3}} &= \dd{S_{\varphi}} = h_{r} \, h_{\theta} \dd{r} \dd{\theta} = r \dd{r} \dd{\theta}
\end{align*}
\end{frame}
\begin{frame}
\frametitle{Construcción}
Entonces el diferencial de volumen es:
\begin{align*}
\dd{V} &= h_{r} \, h_{\theta} \, h_{\varphi} \dd{r} \dd{\theta} \dd{\varphi} = \\[0.5em]
&= r^{2} \, \sin \theta \dd{r} \dd{\theta} \dd{\varphi}
\end{align*}
\end{frame}
\begin{frame}
\frametitle{Ejercicio a cuenta}
La velocidad y la aceleración se definen en la forma vectorial como:
\begin{align*}
\vb{v} = \dv{\vb{r}}{t} = \dot{\vb{r}} \hspace{1cm} \vb{a} = \dot{\vb{v}} = \ddot{\vb{r}}
\end{align*}
Calcula:
\setbeamercolor{item projected}{bg=blue!70!black,fg=yellow}
\setbeamertemplate{enumerate items}[circle]
\begin{enumerate}
\item $\dot{\vu{e}}_{r}$, $\dot{\vu{e}}_{\theta}$, $\dot{\vu{e}}_{\varphi}$ 
\item La velocidad $\vb{v}$
\item La aceleración $\vb{a}$
\end{enumerate}
\end{frame}
\begin{frame}
\frametitle{Ejercicio a cuenta}
Demuestra que para dos vectores $\vb{A}$ y $\vb{B}$:
\setbeamercolor{item projected}{bg=blue!70!black,fg=yellow}
\setbeamertemplate{enumerate items}[circle]
\begin{enumerate}
\item $\vb{A} \cp \vb{B} = \displaystyle \sum_{ijk} \, \vu{e}_{i} \, \epsilon_{ijk} \, A_{j} \, B_{k}$ \\[1em]
\item $\vb{A} \cdot \vb{B} \cp \vb{C} = \displaystyle \sum_{ijk} \, \epsilon_{ijk} \, A_{i} \, B_{j} \, C_{k}$
\end{enumerate}
\end{frame}
\section{Operadores diferenciales}
\frame{\tableofcontents[currentsection, hideothersubsections]}
\begin{frame}
\frametitle{Sobre los campos escalares y vectoriales}
En la presentación anterior mencionamos la naturaleza de los campos escalares y vectoriales.
\\
\bigskip
\pause
Asumiremos que los campos son funciones regulares, continuas y derivables, excepto posiblemente en algunos puntos aislados.
\end{frame}
\begin{frame}
\frametitle{Sobre los campos escalares y vectoriales}
En general los campos serán descritos por ecuaciones diferenciales parciales cuyas variables independientes serán la posición y el tiempo
\end{frame}
\subsection{Gradiente}
\begin{frame}
\frametitle{El gradiente}
Al pasar de un punto 
\begin{align*}
P(u_{1}, u_{2}, u_{3})
\end{align*}
a otro infinitisimalmente cercano 
\begin{align*}
P(u_{1} + \dd{u}_{1}, u_{2} + \dd{u_{2}} + u_{3} + \dd{u_{3}})
\end{align*}
\end{frame}
\begin{frame}
\frametitle{El gradiente}
El cambio diferencial de una función (o campo) escalar $\phi(u_{1}, u_{2}, u_{3})$ está dado por:
\begin{align}
\begin{aligned}
\dd{\phi} &= \pdv{\phi}{u_{1}} \dd{u_{1}} + \pdv{\phi}{u_{2}} \dd{u_{2}} + \pdv{\phi}{u_{3}} \dd{u_{3}} \\[0.5em]
&= \sum_{i=1}^{3} \pdv{\phi}{u_{i}} \dd{u_{i}}
\end{aligned}
\label{eq:ecuacion_01_26}
\end{align}
\end{frame}
\end{document}