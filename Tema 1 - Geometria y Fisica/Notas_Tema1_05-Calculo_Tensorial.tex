 \input{../preambulo_libro}
\author{}
\title{Cálculo Tensorial\\ {\large Matemáticas Avanzadas de la Física}\vspace{-1.5\baselineskip}}
\date{ }
\begin{document}
\maketitle
\fontsize{14}{14}\selectfont

\chapter{Antecedentes}

\section{Breve nota histórica.}
El desarrollo del cálculo tensorial (también llamado análisis multilineal) se haya ligado al desarrollo de la geometría diferencial; el trabajo de Karl F. Gauss (1777 - 1855) sobre la geometría intrínseca de las superficies curvas bidimensionales fue generalizado por Bernhard Riemann(1826 - 1866) quien desarrolló la geometría intrínseca para \enquote{superficies} no euclideanas de $n$-dimensiones (\emph{manifolds}); en un manifold $n$-dimensional cada punto está definido por $n$ coordenadas y en el caso $n = 2$ tenemos una superficie no euclideana que es el único tipo de manifold que podemos captar intuitivamente. En 1827 Gauss mostró que las propiedades métricas de una superficie se pueden expresar por medio de los coeficientes $g_{ij}$ con $i, j =1, 2$ de la siguiente forma diferencial:
\begin{align*}
\dd{\overline{s}^{\, 2}} = g_{11} \dd{u_{1}} \dd{u_{1}} + g_{12} \dd{u_{1}} \dd{u_{2}} + g_{21} \dd{u_{2}} \dd{u_{1}} + g_{22} \dd{u_{2}} \dd{u_{2}}
\end{align*}
siendo $\dd{\overline{s}^{\, 2}}$ el cuadrado de la distancia entre dos puntos de la superficie infinitamente cercanos y $\dd{u_{1}}$, $\dd{u_{2}}$, son los difenciales de las coordenadas intrínsecas de la superficie o coordenadas gaussianas.
\par
Fue Riemman quien generalizó en 1854 ésta fórmula para \enquote{superficies} de $n$ dimensiones, así:
\begin{align*}
\dd{\overline{s}^{\, 2}} = \sum_{i, j} g_{ij} \dd{u_{i}} \dd{u_{j}} \hspace{1.5cm} i, j = 1, 2, \ldots, n
\end{align*}
Por medio de los coeficientes $g_{ij}$ quedan determinadas todas las propiedades métricas en el manifold, por ejemplo: longitud de curvas, ángulo entre curvas, \enquote{áreas} sobre manifolds, etc.
\par
Después del año 1868 se despierta el interés de los matemáticos por algunos puntos tocados en los trabajos de Riemman; Christoffel y Lipzchitz introdujeron el concepto de diferenciación covariante, Beltrami y Kronecker estudiaron la curvatura de varios espacios y superficies $n$-dimensionales. Jordan generalizó las fórmulas de Serret-Frenet para curvas en el espacio $n$-dimensional.
\par
 Todos estos trabajos abrieron el camino a la gran generalización que hizo el geómetra italiano G. Ricci (1853 - 1925) a quien se considera fundador del cálculo tensorial; Ricci se apoyó en la métrica desarrollada por Riemann y en la diferenciación covariante de Christoffel; el profesor Tullio Levi-Civita, gran impulsador del cálculo de tensores afirmó: \blockquote{El desarrollo de los tensores como una rama sistemática de las matemáticas fue un proceso posterior, el crédito del cual se debe a Ricci quien durante los diez años desde 1887 a 1896 elaboró la teoría y realizó la elegante y comprensiva notación que permite adaptarla fácilmente a una gran variedad de temas de análisis, geometría y física.}
 \par
 El éxito de Ricci se debió a la gran intuición que tuvo cuando percibió que las propiedades de la geometría riemanianna son propiedades de ciertos vectores y tensores covariantes y contravariantes; esto le permitió simplificar de un modo notable todos los estudios anteriores a la vez que abrió horizontes para nuevas investigaciones.
 \par
El cálculo de Ricci despertó el interés general luego de que A. Einstein hizo uso del mismo en su formulación de la teoría general de la relatividad (1913 - 1916); en su teoría, Einstein necesitó trabajar con un espacio riemanniano de cuatro
dimensiones y encontró que toda la herramienta matemática necesaria ya había sido elaborada por Ricci.
\par
Finalmente, en 1917 Gerhard Hessenberg en su obra sobre la fundamentación vectorial de la geometría diferencial presenta un nuevo punto de vista sobre los tensores; según Hessenberg, un tensor puede ser mirado como una forma multilineal
homogénea dada en vectores base y que es invariante bajo transformación de coordenadas; el tensor se compone así de un conjunto de escalares (componentes del tensor) cada uno de ellos adscrito a un grupo de vectores base; según ésta presentación, un vector es un tensor de orden uno porque es una forma lineal homogénea de vectores base esto es:
\begin{align*}
\va{A} =  A_{1} \, \va{i}_{1} + A_{2} \, \va{i}_{2} + A_{3} \, \va{i}_{3}
\end{align*}
siendo $\va{i}_{1}, \va{i}_{2}, \va{i}_{3}$ la base vectorial en tres dimensiones.
\par
Las componentes del tensor cambian al cambiar de sistema coordenado, pero el tensor mismo permanece igual, es un invariante.
\par
En nuestra presentación, se tratarán algunos temas desde este punto de vista que desarrolló Hessenberg.
\newpage
\chapter{Bases recíprocas.}
\section{Vectores base.}
En el espacio tridimensional cualquier conjunto de tres vectores no coplanares pueden servir de base a todos los vectores de ese espacio, es decir, todo vector $\vb{A}$ puede expresarse como una combinación lineal de esos tres ya que siempre es posible construir un paralelepípedo tal que una de sus diagonales tenga la magnitud y dirección de $\vb{A}$, y que los tres lados partiendo de uno de los extremos de esa diagonal, tengan respectivamente la dirección de los vectores base:
\begin{align*}
\va{A} = \lambda_{1} \, \va{a}_{1} + \lambda_{2} \, \va{a}_{2} + \lambda_{3} \, \va{a}_{3}
\end{align*}
Como se puede ver en la figura (\ref{fig:figura_01}):
\begin{figure}[H]
    \centering
    \includestandalone{Figuras/Vectores_base}
    \caption{Vector como combinación lineal de los vectores base: $\va{a}_{1}, \va{a}_{2}, \va{a}_{3}$.}
    \label{fig:figura_01}
\end{figure}
\section{Bases recíprocas.}
Si $(\va{a}_{1}, \va{a}_{2}, \va{a}_{3})$ representan una base y se definen otros tres vectores $\va{a}^{\, 1}, \va{a}^{\, 2}, \va{a}^{\, 3}$ de la siguiente manera
\begin{align}
\begin{aligned}
\va{a}^{\, 1}&= \dfrac{\va{a}_{2} \cp \va{a}_{3}}{[\va{a}_{1} \, \va{a}_{2} \, \va{a}_{3}]} \\[1em]
\va{a}^{\, 2} &= \dfrac{\va{a}_{3} \cp \va{a}_{1}}{[\va{a}_{1} \, \va{a}_{2} \, \va{a}_{3}]}  \\[1em]
\va{a}^{\, 3} &= \dfrac{\va{a}_{1} \cp \va{a}_{2}}{[\va{a}_{1} \, \va{a}_{2} \, \va{a}_{3}]} 
\end{aligned}
\label{eq:ecuacion_02_01}
\end{align}
donde
\begin{align*}
\left[ \va{a}_{1} \, \va{a}_{2} \, \va{a}_{3} \right] = \va{a}_{1} \cdot (\va{a}_{2} \cp \va{a}_{3}) = \va{a}_{3} \cdot (\va{a}_{1} \cp \va{a}_{2}) = \va{a}_{2} \cdot (\va{a}_{3} \cp \va{a}_{1})
\end{align*}
entonces, las dos bases $(\va{a}_{1}, \va{a}_{2}, \va{a}_{3})$ y $(\va{a}^{\, 1}, \va{a}^{\, 2}, \va{a}^{\, 3})$ se llaman \emph{recíprocas}.
\par
Como se puede apreciar, de la definición los vectores $(\va{a}^{\, 1}, \va{a}^{\, 2}, \va{a}^{\, 3})$ son respectivamente perpendiculares a los planos determinados por los pares de vectores $(\va{a}_{2} \, \va{a}_{3})$,  $(\va{a}_{3} \, \va{a}_{1})$, $(\va{a}_{1} \, \va{a}_{2})$.
\par
Se deduce entonces que:
\begin{align}
\va{a}^{\, 1} \cdot \va{a}_{1} = \dfrac{(\va{a}_{2} \cp \va{a}_{3}) \cdot \va{a}_{1}}{\left[ \va{a}_{1} \, \va{a}_{2} \, \va{a}_{3} \right]} = \dfrac{\left[ \va{a}_{1} \, \va{a}_{2} \, \va{a}_{3} \right]}{\left[ \va{a}_{1} \, \va{a}_{2} \, \va{a}_{3} \right]} = 1
\label{ec:ecuacion_02_02}
\end{align}
de manera análoga
\begin{align*}
\va{a}^{\, 2} \cdot \va{a}_{2} &= 1 \\[0.5em]
\va{a}^{\, 3} \cdot \va{a}_{3} &= 1
\end{align*}
también se tiene que
\begin{align*}
\va{a}^{\, 2} \cdot \va{a}_{1} = \dfrac{(\va{a}_{3} \cp \va{a}_{1}) \cdot \va{a}_{1}}{\left[ \va{a}_{1} \, \va{a}_{2} \, \va{a}_{3} \right]} = 0
\end{align*}
por lo que
\begin{align*}
\va{a}^{\, 3} \cdot \va{a}_{2} = \va{a}^{\, 1} \cdot \va{a}_{3} = \va{a}^{\, 3} \cdot \va{a}_{1} = \va{a}^{\, 1} \cdot \va{a}_{2} = \va{a}^{\, 2} \cdot \va{a}_{3} = 0
\end{align*}
En resumen, se tiene que las dos bases definidas anteriormente, son tales que
\begin{align}
\va{a}^{\, i} \cdot \va{a}_{j} = \delta_{ij} \hspace{1.5cm} i, j = 1, 2 ,3
\label{eq:ecuacion_02_03}
\end{align}
Como $\va{a}^{\, i} \cdot \va{a}_{j} = 0$ para $ i \neq j$, se aprecia que cada $\va{a}_{j}$ es perpendicular a los dos vectores $\va{a}^{\, i}$ ($i \neq j$), por lo tanto $(\va{a}_{1} \, \va{a}_{2} \, \va{a}_{3})$ son respectivamente perpendiculares a los planos determinados por los pares de vectores $(\va{a}^{\, 2}, \va{a}^{\, 3})$, $(\va{a}^{\, 1}, \va{a}^{\, 3})$, $(\va{a}^{\, 1}, \va{a}^{\, 2})$, por lo que son proporcionales a $(\va{a}^{\, 2} \cp \va{a}^{\, 3})$, $(\va{a}^{\, 3} \cp \va{a}^{\, 1})$, $(\va{a}^{\, 1} \cp \va{a}^{\, 2})$ y en consecuencia:
\begin{align*}
\va{a}_{1} &= K_{1} \, \va{a}^{\, 2} \cp \va{a}^{\, 3} \\[0.5em]
\va{a}_{2} &= K_{2} \, \va{a}^{\, 3} \cp \va{a}^{\, 1} \\[0.5em]
\va{a}_{3} &= K_{3} \, \va{a}^{\, 1} \cp \va{a}^{\, 2}
\end{align*}
con $K_{1}, K_{2}, K_{3}$ escalares.
\par
Si se realiza el producto $a_{1} \cdot a^{1} = 1$, la ecuación (\ref{ec:ecuacion_02_02}) se expresa como:
\begin{align*}
K_{1} ( \va{a}^{\, 2} \cp \va{a}^{\, 3} ) \cdot \va{a}^{\, 1} = 1 \hspace{0.5cm} \Longrightarrow \hspace{0.5cm} K_{1} = \dfrac{1}{\left[ \va{a}_{1} \, \va{a}_{2} \, \va{a}_{3} \right]}
\end{align*}
de manera análoga
\begin{align*}
K_{2} = K_{3} = \dfrac{1}{\left[ \va{a}_{1} \, \va{a}_{2} \, \va{a}_{3} \right]}
\end{align*}
Esto quiere decir que los vectores $\va{a}_{1} , \va{a}_{2} , \va{a}_{3}$ se pueden expresar en función de los vectores $\va{a}^{\, 1} \, \va{a}^{\, 2} \, \va{a}^{\, 3}$ como
\begin{align}
\begin{aligned}
\va{a}_{1} &= \dfrac{\va{a}^{\, 2} \cp \va{a}^{\, 3}}{\left[ \va{a}_{1} \, \va{a}_{2} \, \va{a}_{3} \right]} \\[0.5em]
\va{a}_{2} &= \dfrac{\va{a}^{\, 3} \cp \va{a}^{\, 1}}{\left[ \va{a}_{1} \, \va{a}_{2} \, \va{a}_{3} \right]} \\[0.5em]
\va{a}_{3} &= \dfrac{\va{a}^{\, 1} \cp \va{a}^{\, 2}}{\left[ \va{a}_{1} \, \va{a}_{2} \, \va{a}_{3} \right]}
\end{aligned}
\label{eq:ecuacion_02_04}
\end{align}
Comparando las ecs. (\ref{eq:ecuacion_02_01}) y (\ref{eq:ecuacion_02_04}), se aprecia que las dos bases se comportan recíprocamente cuando se expresa una en función de la otra.
\par
Una relación importante entre las bases recíprocas es la correspondientes al triple producto mixto, que son inversos, es decir
\begin{align*}
\left[ \va{a}^{\, 1} \, \va{a}^{\, 2} \, \va{a}^{\, 3} \right] \cdot \left[ \va{a}_{1} \, \va{a}_{2} \, \va{a}_{3} \right]
= 1
\end{align*}
La demostración es como sigue:
\begin{align}
\va{a}^{\, 1} \cdot (\va{a}^{\, 2} \cp \va{a}^{\, 3}) = \dfrac{(\va{a}_{2} \cp \va{a}_{3}) \cdot [(\va{a}_{\, 3} \cp \va{a}_{1}) \cp (\va{a}_{1} \cp \va{a}_{2})]}{[ \va{a}_{1} \, \va{a}_{2} \, \va{a}_{3}]^{3}}
\label{eq:ecuacion_02_05}
\end{align}
el triple producto del numerador, se expande usando la regla
\begin{align*}
\va{a} \cp (\va{b} \cp \va{c}) = (\va{a} \cdot \va{c}) \, \va{b} - (\va{a} \cdot \va{b}) \, \va{c}
\end{align*}
por lo que
\begin{align*}
(\va{a}_{2} \cp \va{a}_{3}) \cdot [(\va{a}_{3} \cp \va{a}_{1}) \cp (\va{a}_{1} \cp \va{a}_{2})] &= (\va{a}_{2} \cp \va{a}_{3}) \cdot \left\{ [(\va{a}_{3} \cp \va{a}_{1}) \cdot \va{a}_{2}] \, \va{a}_{1} + \right. \\
&- \left. [(\va{a}_{3} \cp \va{a}_{1}) \cdot \va{a}_{1}] \, \va{a}_{2} \right\} = \\
&= (\va{a}_{2} \cp \va{a}_{3}) \cdot [(\va{a}_{3} \cp \va{a}_{1}) \cdot \va{a}_{2} ] \, \va{a}_{1})] = \\
&= [\va{a}_{1} \cdot (\va{a}_{2} \cp \va{a}_{3})] \, [\va{a}_{2} \cdot (\va{a}_{3} \cp \va{a}_{1})] = \\
&= [ \va{a}_{1} \, \va{a}_{2} \, \va{a}_{3}]^{2}
\end{align*}
que llevando este resultado a la ecuación (\ref{eq:ecuacion_02_05}) resulta
\begin{align*}
\va{a}^{\, 1} \cdot (\va{a}^{\, 2} \cp \va{a}^{\, 3}) &= \dfrac{[ \va{a}_{1} \, \va{a}_{2} \, \va{a}_{3}]^{2}}{[ \va{a}_{1} \, \va{a}_{2} \, \va{a}_{3}]^{3}} = \\[0.5em]
&= \dfrac{1}{[ \va{a}_{1} \, \va{a}_{2} \, \va{a}_{3}]}
\end{align*}
Si los vectores base $ \va{a}^{\, 1} \, \va{a}^{\, 2} \, \va{a}^{\, 3}$ son triplemente ortogonales, entonces como el vector $\va{a}_{1}$ (de la base recíproca $\va{a}_{1} \, \va{a}_{2} \, \va{a}_{3}$) debe de ser ortogonal con $\va{a}^{\, 2}$ y con $\va{a}^{\, 3}$, además coincide en dirección con $\va{a}^{\, 1}$; de la misma manera $\va{a}_{\, 2}$ coincide en dirección con $\va{a}^{\, 2}$ y $\va{a}_{\, 3}$ con $\va{a}^{\, 3}$; por lo tanto, en este caso, las dos bases coinciden por lo menos en cuanto a que sus correspondientes vectores tienen igual dirección y sentido; si además de ser ortogonal $ \va{a}^{\, 1} \, \va{a}^{\, 2} \, \va{a}^{\, 3}$, está formada por vectores unitarios, entonces los $\va{a}_{1} \, \va{a}_{2} \, \va{a}_{3}$ también serán unitarios, ya que 
\begin{align*}
\va{a}^{\, 1} \cdot \va{a}_{1} = 1
\end{align*} 
y como 
\begin{align*}
\abs{ \va{a}^{\, 1} } = 1 \hspace{1cm} \Rightarrow \hspace{1cm} \abs{\va{a}_{2}} = \abs{\va{a}_{3}} = 1
\end{align*}
Veamos un ejemplo sobre bases recíprocas: tenemos en el espacio bidimensional, dos vectores unitarios $\va{a}_{1}$ y $\va{a}_{2}$ dirigidos a lo largo de los ejes $X^{1}, X^{2}$ (después se verá justificado el que los vectores base lleven subíndices y los ejes correspondientes, superíndices); construyamos la base recíproca $\va{a}^{\, 1}, \va{a}^{\, 2}$:
\begin{figure}[H]
    \centering
    \includestandalone{Figuras/F_002-Ejemplo_Base_Reciproca}
    \caption{Dos bases recíprocas.}
    \label{fig:figura_02}
\end{figure}
De la ec. (\ref{eq:ecuacion_02_01}) vemos que $\va{a}^{\, 1}$ debe de ser normal a $\va{a}_{2}$ (es decir, debe de estar sobre el eje $X_{1}$, que forma un ángulo recto con el eje $X^{2}$) y el vector $\va{a}^{\, 2}$ debe de ser normal a $\va{a}_{1}$ (es decir, debe de estar sobre el eje $X_{2}$ que forma un ángulo recto con $X^{1}$).
\par
Del extremo de $\va{a}_{1}$ y $\va{a}_{2}$ bajemos a los ejes $X_{1}, X_{2}$ líneas que sean perpendiculares a $X^{1}, X^{2}$, vamos a demostrar que los vectores así obtenidos, son los vectores $(\va{a}^{1}, \va{a}^{2})$, recíprocos de $(\va{a}_{1}, \va{a}_{2})$; para esto debemos demostrar que éstos vectores satisfacen la ec. (\ref{eq:ecuacion_02_03}), es decir $\va{a}^{\, i} \cdot \va{a}_{j} = \delta_{j}^{i}$
\par
Tenemos que
\begin{align*}
\va{a}^{\, 1} \cdot \va{a}_{1} = \abs{\va{a}^{\, 1}} \abs{\va{a}_{1}} \, \sin \theta
\end{align*}
pero en el triángulo correspondiente a $\va{a}^{\, 1} \, \va{a}_{1}$, se tiene que
\begin{align*}
\sin \theta = \dfrac{\abs{\va{a}_{1}}}{\abs{\va{a}^{\, 1}}} \hspace{0.5 cm} \Rightarrow \hspace{0.5cm} \abs{\va{a}^{\, 1}}  = \dfrac{\abs{\va{a}_{1}}}{\sin \theta}
\end{align*}
entonces
\begin{align*}
\va{a}^{\, 1} \cdot \va{a}_{1} = \dfrac{\abs{\va{a}_{1}}}{\sin \theta} \, \abs{\va{a}_{1}} \, \sin \theta = 1
\end{align*}
ya que $\va{a}_{1}$ es unitario, además $\va{a}^{1} \cdot \va{a}_{2} = 0$, ya que $X_{1} \perp X^{2}$.
\par
De la misma manera se demuestra que $\va{a}^{\, 2} \cdot \va{a}_{2} = 1$ y que $\va{a}^{\, 2} \cdot \va{a}_{1} = 0$, por lo que se cumple que
\begin{align*}
\va{a}^{\, i} \cdot \va{a}_{j} = \delta_{j}^{i}
\end{align*}
es decir, los vectores $(\va{a}_{1}, \va{a}_{2})$
son unitarios y su base recíproca está formada por $(\va{a}^{1}, \va{a}^{2})$, obtenida en el procedimiento anterior.
\chapter{Coordenadas curvilíneas}
\section{Introducción.}
En física existe un gran número de problemas que se pueden resolver más fácilmente si se trabaja con las coordenadas apropiadas a un problema en particular, es decir, las coordenadas cartesianas no serán siempre las más convenientes para todo tipo de problema, por ejemplo, si estudiamos el flujo de calor a través de una esfera, evidentemente lo más práctico es trabajar con coordenadas esféricas; si estamos calculando la longitud de un arco de circunferencia lo más conveniente es trabajar con coordenadas polares, ya que en ese caso
\begin{align*}
S_{a, b} = \int_{\theta_{a}}^{\theta_{b}} R \cdot \dd{\theta} =  R (\theta_b) - \theta_{a})
\end{align*}
lo cual es más simple que si lo hacemos en coordenadas cartesianas:
\begin{align*}
S_{a, b} = \int_{x_{a}}^{x_{b}} \sqrt{1 + \left(\dv{y}{x} \right)^{2}} \dd{x}
\end{align*}
donde
\begin{align*}
\dv{y}{x} = -x \, (R^{2} - x^{2})^{-1/2}
\end{align*}
En vista de lo anterior, surge la necesidad de estudiar las coordenadas curvilíneas.
\section{Coordenadas curvilíneas.}
En un espacio dado, instalemos un sistema coordenado cartesiano $Y_{i}$, de tal manera que para cada punto del espacio, queda determinado en éste sistema, por una terna de valores $(y_{1}, y_{2}, y_{3})$; definamos ahora tres funciones $(x_{1}, x_{2}, x_{3})$, tales que
\begin{align}
\begin{aligned}
x_{1} &= x_{1} (y_{1}, y_{2}, y_{3}) \\
x_{2} &= x_{2} (y_{1}, y_{2}, y_{3}) \\
x_{3} &= x_{3} (y_{1}, y_{2}, y_{3})
\end{aligned}
\label{eq:ecuacion_03_01}
\end{align}
Estas funciones las vamos a suponer monovaluadas y derivables en todos los puntos del espacio, si en la ec. (\ref{eq:ecuacion_03_01}) hacemos $x_{1} = c_{1}, x_{2} = c_{2}, x_{3} =c_{3}$, las ecuaciones quedan de la forma
\begin{align}
\begin{aligned}
x_{1} (y_{1}, y_{2}, y_{3}) &= c_{1} \\
x_{2} (y_{1}, y_{2}, y_{3}) &= c_{2} \\
x_{3} (y_{1}, y_{2}, y_{3}) &= c_{3}
\end{aligned}
\label{eq:ecuacion_03_02}
\end{align}
Vemos que en cada una de esas tres ecuaciones se puede despejar una de las $y$, en función de las otras dos, por ejemplo
\begin{align*}
y_{1} = f_{1} (c_{1}, y_{2}, y_{3})
\end{align*}
esto implica que cada una de las ecuaciones en (\ref{eq:ecuacion_03_02}) representa una superficie por que la coordenada en una dirección ($y_{1}$ por ejemplo) es función de los puntos del plano normal a esa dirección (el plano $Y_{2}, Y_{3}$ por ejemplo); este es el caso de la ecuación $z = +\sqrt{R^{2} - x^{2} - y^{2}}$ que nos representa una semiesfera de radio $R$ centrada en el origen: a cada punto $(x, y)$ del plano $X \, Y$ corresponde un punto de coordenadas $(x, y , z)$ perteneciente a esa semiesfera y el punto $z$ dado precisamente por $\sqrt{R^{2} - x^{2} - y^{2}}$.
\par
Si en la ecuación $x_{1} (y_{1}, y_{2}, y_{3}) = c_{1}$, se va cambiando el valor de la constante $c_{1}$, se obtiene una familia de superficies (por ejemplo si en $z = \sqrt{R^{2} - x^{2} - y^{2}}$ se hace variar el valor de $R$, se obtiene una familia de superficies esféricas concéntricas) y lo mismo aplica para $x_{2} (y_{1}, y_{2}, y_{3}) = c_{2}$ y $x_{3} (y_{1}, y_{2}, y_{3}) = c_{3}$; de este modo el espacio considerado se puede pensar como lleno completamente con tres familias de superficies y en cada punto $(y_{1}, y_{2}, y_{3})$ de este espacio, se cortan tres superficies, precisamente aquéllas para las cuales se cumple:
\begin{align*}
y_{1} &= y_{1} (x_{1}, x_{2}, x_{3}) = y_{1} \mbox{ dado} \\
y_{2} &= y_{2} (x_{1}, x_{2}, x_{3}) = y_{2} \mbox{ dado} \\
y_{3} &= y_{3} (x_{1}, x_{2}, x_{3}) = y_{3} \mbox{ dado}
\end{align*}
Como vemos, para poder garantizar la intersección de tres superficies en cada punto del espacio, es necesario que en las ecuaciones (\ref{eq:ecuacion_03_02}) se pueda despejar $y_{i}$ en función de las $x_{i}$, de modo que para cada terna $(x_{1}, x_{2}, x_{3})$ exista una y sólo una terna $(y_{1}, y_{2}, y_{3})$, la condición algebraica que nos garantiza esto, la obtenemos si en la ecuación (\ref{eq:ecuacion_03_01}), expresamos los diferenciales totales de $x_{i}$ en función de los diferenciales totales de $y_{i}$, es decir:
\begin{align}
\begin{aligned}
\dd{x_{1}} &= \pdv{x_{1}}{y_{1}} \dd{y_{1}} + \pdv{x_{1}}{y_{2}} \dd{y_{2}} + \pdv{x_{1}}{y_{3}} \dd{y_{3}} \\[0.5em]
\dd{x_{2}} &= \pdv{x_{2}}{y_{1}} \dd{y_{1}} + \pdv{x_{2}}{y_{2}} \dd{y_{2}} + \pdv{x_{2}}{y_{3}} \dd{y_{3}} \\[0.5em]
\dd{x_{3}} &= \pdv{x_{3}}{y_{1}} \dd{y_{1}} + \pdv{x_{3}}{y_{2}} \dd{y_{2}} + \pdv{x_{3}}{y_{3}} \dd{y_{3}}
\end{aligned}
\label{eq:ecuacion_03_03}
\end{align}
Se demuestra en el análisis matemático avanzado que la condición necesaria y suficiente para que exista una correspondencia biunívoca entre los $x_{i}$ y los $y_{i}$, es que no se anule el determinante de los coeficientes del sistema de ecuaciones (\ref{eq:ecuacion_03_03}); este determinante se llama \emph{jacobiano} de la transformación que nos convierte a los $y_{i}$ en los $x_{i}$ a partir de las ecuaciones (\ref{eq:ecuacion_03_01}), es decir, se debe de cumplir:
\begin{equation*}
I  = 
\mdet{
\pdv{x_{1}}{y_{1}} & \pdv{x_{1}}{y_{2}} & \pdv{x_{1}}{y_{3}} \\[0.5em]
\pdv{x_{2}}{y_{1}} & \pdv{x_{2}}{y_{2}} & \pdv{x_{2}}{y_{3}} \\[0.5em]
\pdv{x_{3}}{y_{1}} & \pdv{x_{3}}{y_{2}} & \pdv{x_{3}}{y_{3}}
} \neq 0
\end{equation*}
Si en un punto en el espacio se tienen tres superficies intersectadas:
\begin{align*}
x_{1} &= x_{1} (y_{1}, y_{2}, y_{3}) = c_{1} \\
x_{2} &= x_{2} (y_{1}, y_{2}, y_{3}) = c_{2} \\
x_{3} &= x_{3} (y_{1}, y_{2}, y_{3}) = c_{3} 
\end{align*}
sabemos que ellas se cortarán dos a dos según una curva, por lo que la intersección de $x_{2} = c_{2}$ y $x_{3} = c_{3}$, se obtiene despejando $y_{i}$ en función de $x_{i}$
\begin{align}
\begin{aligned}
y_{1} &= y_{1} (x_{1}, c_{2}, c_{3}) \\
y_{2} &= y_{2} (x_{1}, c_{2}, c_{3}) \\
y_{3} &= y_{3} (x_{1}, c_{2}, c_{3}) \\
\end{aligned}
\label{eq:ecuacion_03_04}
\end{align}
Recordemos que los $y_{i}$ se refieren a coordenadas cartesianas del espacio, podemos observar en las anteriores ecuaciones que para el punto en cuestión (es decir, siendo $c_{2}, c_{3}$ constantes), los valores de $(y_{1}, y_{2}, y_{3})$ son funciones de un sólo parámetro $(x_{1})$, por lo tanto, las ecuaciones (\ref{eq:ecuacion_03_04}) son las ecuaciones paramétricas de la curva que resulta de la intersección de las superficies
\begin{align*}
x_{2} = x_{2} (y_{1}, y_{2}, y_{3}) = c_{2} \\
x_{3} = x_{3} (y_{1}, y_{2}, y_{3}) = c_{3}
\end{align*}
a esta curva la llamaremos la curva $X_{1}$; de manera análoga, podemos decir que la intersección de las superficies $X_{1} = C_{1}$ y $X_{3} = C_{3}$ representan una curva $X_{2}$; y la intersección de $X_{1} = C_{1}$ y $X_{2} = C_{2}$, es la curva $X_{3}$, en la siguiente gráfica se pueden apreciar estas curvas $X_{1}, X_{2}, X_{3}$

\end{document}