\input{../Preambulos/preambulo_presentacion_Warsaw_seahorse}
\title{\large{Ejercicios}}
\subtitle{Funciones Gamma y Beta}
\author{M. en C. Gustavo Contreras Mayén}
\date{\today}
\institute{Facultad de Ciencias - UNAM}
\titlegraphic{\includegraphics[width=1.75cm]{../Imagenes/escudo-facultad-ciencias}\hspace*{4.75cm}~%
   \includegraphics[width=1.75cm]{../Imagenes/escudo-unam}
}
\setbeamertemplate{navigation symbols}{}
\begin{document}
\maketitle
\fontsize{14}{14}\selectfont
\spanishdecimal{.}
\section*{Contenido}
\frame[allowframebreaks]{\tableofcontents[currentsection, hideallsubsections]}
\section{Función Gamma}
\frame{\tableofcontents[currentsection, hideothersubsections]}
\subsection{Identidad}
\begin{frame}
\frametitle{Propiedad función $\Gamma(x)$}
\textbf{Demuestra} la siguiente identidad:
\begin{align*}
\Gamma (x + 1) = x \, \Gamma (x)
\end{align*}
\end{frame}
\begin{frame}
\frametitle{Solución}
Antes de demostrar la identidad directamente de la integral Gamma para todo $x$ positivo, veamos que esta identidad se usa para definir la función Gamma primero para $-1 < x < 0$ escribiéndola en la forma 
\begin{align*}
\Gamma(x) = \dfrac{\Gamma (x + 1)}{x}
\end{align*}
\end{frame}
\begin{frame}
\frametitle{Solución}
Tendremos entonces que la expresión es válida para el intervalo $-2 < x < -1$, y así sucesivamente para todos los valores no enteros negativos de $x$.
\\
\bigskip
\pause
Por lo que nos queda demostrar que se cumple
\begin{align*}
\Gamma (x + 1) =  x \, \Gamma (x)
\end{align*}
para todo $x$ positivo.
\end{frame}
\begin{frame}
\frametitle{Demostración}
Haciendo que $x$ sea cualquier número positivo, escribimos la integral Gamma para el argumento $x + 1$:
\begin{eqnarray*}
\Gamma (x + 1) &= \displaystyle \int_{0}^{\infty} t^{(x+1)-1} \, e^{-t} \dd{t} = \\[0.5em] \pause
&= t^{x} \, e^{-t} \dd{t}
\end{eqnarray*}
\end{frame}
\begin{frame}
\frametitle{Demostración}
Resolvemos usando la integración por partes, haciendo:
\begin{align*}
u = t^{x} \hspace{1cm} \dd{v} = e^{-t} \dd{t}
\end{align*}
entonces
\pause
\begin{align*}
\Gamma (x + 1) = t^{x} \, \left( - e^{-t}\right)\eval_{0}^{\infty} - \int_{0}^{\infty} \left( -e^{-t}\right) \, x \, t^{x-1} \dd{t}
\end{align*}
\end{frame}
\begin{frame}
\frametitle{Demostración}
Por lo que:
\begin{align*}
\Gamma (x + 1) = - \lim_{t \to \infty} \dfrac{t^{x}}{e^t} + \dfrac{0^{x}}{e^{0}} + x \int_{0}^{\infty} t^{x-1} \, e^{-t} \dd{t}
\end{align*}
\pause
Veamos ahora lo que pasa con este resultado:
\end{frame}
\begin{frame}
\frametitle{Demostración}
El límite en el primer sumando de la expresión de la derecha sabemos que se anula, a partir de la indeterminación $\infty / \infty$, ya que usamos la regla de L'Hopital.
\begin{align*}
\Gamma (x + 1) = \cancelto{0}{- \lim_{t \to \infty} \dfrac{t^{x}}{e^t}} + \dfrac{0^{x}}{e^{0}} + x \int_{0}^{\infty} t^{x-1} \, e^{-t} \dd{t}
\end{align*}
\end{frame}
\begin{frame}
\frametitle{Demostración}
El segundo término de la derecha también se anula
\begin{align*}
\Gamma (x + 1) = \cancelto{0}{- \lim_{t \to \infty} \dfrac{t^{x}}{e^t}} + \cancelto{0}{\dfrac{0^{x}}{e^{0}}} + x \int_{0}^{\infty} t^{x-1} \, e^{-t} \dd{t}
\end{align*}
\pause
Por lo que nos queda el tercero, pero vemos que la integral que queda, es la definición de la función Gamma, así:
\end{frame}
\begin{frame}
\frametitle{Demostración}
Tenemos que
\begin{align*}
\Gamma (x + 1) = x \, \Gamma (x) \qed
\end{align*}
\end{frame}
\end{document}