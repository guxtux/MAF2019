Sistemas de coordenadas en movimiento.


Frecuentemente conviene, desde un punto de vista matemático, emplear la descripción cinemática del movimiento de una partícula con respecto a un sistema de coordenadas en movimiento. El movimiento de un sistema de coordenadas puede ser de traslación, de rotación o una combinación de ambos. Al tratar el movimiento limitado de una partícula cargada eléctricamente, bajo la acción combinada de un campo eléctrico central y otro magnético de poca intensidad, por ejemplo, veremos que la descripción cinemática del movimiento con respecto a un sistema de coordenadas giratorio apropiado será, aproximadamente, la misma que la descripción cinemática del movimiento de la partícula cargada con respecto a un sistema de coordenadas estacionario, bajo la acción del campo eléctrico únicamente, lo cual es un problema mucho más simple. Además, la descripción del movimiento de una partícula con respecto a un sistema de coordenadas fijo sobre la superficie de la Tierra implica, naturalmente, un sistema de coordenadas que se traslada y gira al mismo tiempo en el espacio. Gomo postularemos las leyes fundamentales de la mecánica con respecto a un sistema inercial, el cual es un sistema de coordenadas sin aceleración ni rotación, necesitaremos para poder aplicarlas a problemas específicos un conocimiento de las relaciones entre las descripciones cinemáticas del movimiento de una partícula con respecto a sistemas de coordenadas en que unos se mueven con respecto a los otros. 

3—1 Movimiento de traslación. 

Consideremos dos sistemas de coordenadas cuyas orientaciones en el espacio se observa que permanecen fijas. Para simplificar, supongamos además, que sus respectivos vectores base coordenados son paralelos entre sí.

En caso de que no fuera así, sólo se necesitaría la transformación de uno de ellos a un sistema de coordenadas paralelo al otro. Consideraremos tales transformaciones en la sección 3—4. El único movimiento que puede observarse que realizan entre sí estos dos sistemas de coordenadas es el movimiento de traslación de uno con respecto al otro. 

Un solo observador que vea estos dos sistemas de coordenadas podría describir el movimiento de una partícula con respecto a uno u otro de ellos y relacionar las dos descripciones diferentes así obtenidas. Procedemos a encontrar esta relación. 

Designando los orígenes de los dos sistemas coordenados por O y O', y suponiendo que el sistema coordenado sin prima permanece fijo con respecto al observador, conocerá éste la posición de un punto P, con respecto a O', si le son conocidas su posición con respecto a O y la posición de O' con respecto a O. Los respectivos vectores de posición r'(f)’, r(f) y R(í), como se puede ver en la figura 3-1, están relacionados de la siguiente manera : 

r(t) = R(t) + r'(t). (3-1) 

Mediante derivaciones sucesivas de la ecuación (3-1) hallamos las relaciones entre las velocidades y las aceleraciones del punto P con respecto a los dos puntos O y O'. Para las velocidades, obtenemos 

i(t) = R(f) + f'(t) 

o 

v(í) = V(Í) + v'(t), (3-2) 

donde v(f) es la velocidad de P con respecto a O, V(í) la velocidad de O’ con respecto a O, y v'(f) la velocidad de P ccu^ respecto a O'. Para las aceleraciones tenemos 

a(í) =A(t) +a'(í), (3-3) 

donde a(f),A(f) yo'(í) son, respectivamente, las aceleraciones del punto P con respecto al punto O, del punto O' con respecto a O, y del punto P con respecto a O'. 

Consideremos, como ejemplo, el problema de hallar la velocidad y la aceleración de un punto P en la circunferencia exterior de una rueda de radio R que desciende rodando sobre un plano inclinado que forma 
un ángulo 9 con la horizontal (fig. 3-2). 

La velocidad v del punto P con respecto al plano inclinado está relacionada a la velocidad V del punto P, con respecto al centro geométrico de la rueda y a la velocidad V de dicho centro con respecto al plano inclinado, como se ve en la ecuación ( 3—2 ) . De la figura 3—2 encontramos que en función de sus componentes 

y 

por lo que 

v' = R<j> sen </> i + R<j> eos </>¡ 

V = R(j> eos 9 i — R<f> sen 9 \ , 

v = R<¡>( eos 9 + sen <f>) i + i?<j>( eos </> — sen 9) ¡. 

Si el plano inclinado no permanece en reposo y deseamos hallar la velocidad del punto P con respecto al terreno o suelo, tendriamos que añadir, subsecuentemente, la velocidad del plano inclinado con respecto al suelo a la velocidad del punto P con respecto al plano inclinado. 

Para el punto de contacto entre la rueda y el plano inclinado en el cual vemos que su velocidad con respecto al plano inclinado es nula, 

v = 0. 

Esta ecuación expresa la condición de rodadura. Si un cuerpo rueda sobre otro, entonces la velocidad relativa de cada uno de los dos puntos de contacto respecto al otro es nula. 

La aceleración del punto P con respecto al plano inclinado está dada por la ecuación ( 3—3 ) . Con la figura 3-3 hallamos que la aceleración a' del punto P con respecto al centro de la rueda alrededor del cual se mueve en una circunferencia de radio R está dada por 

a' — ( R<j > 2 eos <¡> + R<j> sen <f>) i + ( R<¡> eos </> — R<j> z sen <f>) j. 

Con el mismo esquema también obtenemos, para la aceleración del centro de la rueda con respecto al plano inclinado, el resultado 

A — R<¡> eos 9\ — R<¡> sen 9\. 

Por tanto, por la ecuación (3-3) vemos que la aceleración del punto P con respecto al plano inclinado está dada por 

a =i [ R<¡> 2 eos <j> + R <¡>( sen <f> + eos 9 ) ]¡ + 

[R <¡> ( eos <f> — sen 9) — R<¡> 2 sen <j>]j.^ 

Queremos resaltar nuevamente que en nuestra explicación sólo consideramos un observador único, el cual describe la posición, velocidad y aceleración de los puntos P y O' con respecto al origen de su sistema de coordenadas. La posición, velocidad y aceleración relativas del punto P con respecto al CP representa meramente una especie de diferente sistema de registro contable cinemático para dicho observador. Las ecuaciones (3-1), (3-2) y (3-3) relacionan sus asientos de las diferentes descripciones cinemáticas del punto P para las posición, velocidad y aceleración relativas observadas de O' con respecto a O. Desde este punto de vista, vemos que las ecuaciones de transformación que hemos deducido son, también, completa 
y generalmente aplicables a la descripción relativista del movimiento de cuerpos puntiformes por un único observador. 

La mecánica relativista, de la que hablaremos en el Capítulo 13, difiere de la mecánica clásica en la descripción cinemática del movimiento de una partícula cuando un segundo observador entra en escena. Sólo difiere la mecánica clásica de la relativista cuando O y O* representan los orígenes de los sistemas de coordenadas de dos distintos observadores que se mueven respectivamente con estos puntos y cuando r, v y a y K, v' y a' son las descripciones cinemáticas deLífiovimiento del punto P con respecto a los dos observadores inerciales. La mecánica clásica supone que las ecuaciones que hemos hallado anteriormente son las de transformación correctas para relacionar las diferentes descripciones cinemáticas del movimiento del punto P por los dos observadores. Encontramos que para velocidades relativas de O' con respecto a O cuyas magnitudes sean pequeñas comparadas con la velocidad de la luz, las ecuaciones de transformación que obtuvimos son buenas aproximaciones a las relaciones relativistas correctas que existen entre las diferentes descripciones cinemáticas del movimiento de un punto P con respecto a dos observadores que se mueven respectivamente con los puntos O y O'. 

Una interesante aplicación de las ecuaciones (3—1) y (3-2) es la descripción de la propagación de una onda armónica plana con respecto a un sistema de coordenadas en movimiento. 

Consideremos la onda plana definida por 

'F = A sen ( kx — uf) = A sen k(x — vt). 

Es decir, una onda plana de frecuencia f — u/2ir y velocidad v = a/k que se mueve en el sentido en que x aumenta, donde k es el llamado número de onda, igual a 2— veces el recíproco de la longitud de onda. 

donde X representa la longitud de onda. 

Supongamos que ésta sea la descripción de la onda con respecto al medio de propagación, es decir, la descripción de la onda con respecto a un observador al cual el medio de transmisión le parece que está en reposo. A un segundo observador que vea el medio de transmisión moviéndose en la dirección y sentido de la propagación de la onda con una velocidad V le parecerá que la onda está moviéndose con una velocidad v + v M . El segundo observador, por tanto, definirá la onda plana por la ecuación 

4' = A sen k(x' — v't) = A sen (kx' — a’t), 

donde v' está definida por 

, u¡' 

V = V + V M ,= 

Esta última descripción de la onda por el segundo observador puede ser obtenida a partir de la descripción del primero sustituyendo a; por x' — v u t, donde x y x' son respectivamente las posiciones de un frente de onda en un instante o tiempo f con respecto a los dos observadores (fig. 3—4). 

La frecuencia /' atribuida a la onda por el segundo observador difiere de la f que le asigna a ella el primer observador. La relación entre las dos frecuencias es 
o + Vm 

V 

Frente de onda 

Fig. 3-4 

Esta fórmula es el clásico efecto Doppler para el cambio o variación de frecuencia que se produce por el movimiento de un observador, del medio transmisor o de ambos. 