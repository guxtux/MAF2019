\documentclass[12pt]{article}
%\usepackage[left=0.25cm,top=1cm,right=0.25cm,bottom=1cm]{geometry}
\usepackage{geometry}
\textwidth = 20cm
\hoffset = -1cm
\usepackage[utf8]{inputenc}
\usepackage[spanish,es-tabla]{babel}
\usepackage{amsmath}
\usepackage{nccmath}
\usepackage{amsthm}
\usepackage{amssymb}
\usepackage{graphicx}
\usepackage{color}
\usepackage{float}
\usepackage{multicol}
\usepackage{enumerate}
\usepackage{anyfontsize}
\usepackage{anysize}
\usepackage{enumitem}
\usepackage{capt-of}
\usepackage{bm}
\usepackage{relsize}
\usepackage{physics}
\usepackage{empheq}
\usepackage{mathtools}
\spanishdecimal{.}
\setlist[enumerate]{itemsep=0mm}
\renewcommand{\baselinestretch}{1.2}
\let\oldbibliography\thebibliography
\renewcommand{\thebibliography}[1]{\oldbibliography{#1}
\setlength{\itemsep}{0pt}}
%\marginsize{1.5cm}{1.5cm}{0cm}{2cm}
%\renewcommand\theenumii{\arabic{theenumii.enumii}}
\renewcommand\labelenumii{\theenumi.{\arabic{enumii}}}

\title{Ejercicios para el Tema 1 \\ \large{Matemáticas Avanzadas de la Física}}
%\subtitle{Fecha de entrega: 8 de marzo de 2016.}
\author{M. en C. Gustavo Contreras Mayén}
\date{ }

\begin{document}
\vspace{-4cm}

\maketitle
\fontsize{14}{14}\selectfont

\section*{Coordenadas cilíndricas elípticas}

El sistema de coordenadas cilíndricas elípticas $(\sigma, \tau, z)$ se define mediante las relaciones:
\begin{align*}
    x = 2 \, A \, \cosh \sigma \, \cos \tau \hspace{1cm} y = 2 \, A \, \senh \sigma \, \sen \tau \hspace{1cm} z = z
\end{align*}
Demuestra que:
\begin{enumerate}
\item Este sistema de coordenadas es ortogonal.
\item Los factores de escala son:
\begin{align*}
    h^{2}_{\sigma} = h^{2}_{\tau} = 4 \, A^{2} \left( \sinh 2 \sigma + \sin 2 \tau \right) \hspace{0.5cm} \text{y} \hspace{0.5cm} h_{z} = 1
\end{align*}
\end{enumerate}

\vspace{1cm}
\noindent
\textbf{Solución: } Consideremos las reglas de transformación:
\begin{align*}
x = 2\, A \, \cosh \sigma \, \cos \tau, \hspace{1cm}  y = 2 \, A \, \sinh \sigma \, \sin \tau \hspace{1cm} z = z
\end{align*}
con los dominios $\sigma \in \mathbb{R}, \, \tau \in [0, 2 \, \pi)$ y constante $(A > 0)$. Denotamos el vector posición:
\begin{align*}
\vb{r} (\sigma, \tau, z) = \left( 2 \,A \, \cosh \sigma \, \cos \tau, 2 \, A \, \sinh \sigma \, \sin \tau, z \right)
\end{align*}

\subsection*{Vectores base covariantes}
Calculamos las derivadas parciales:
\begin{align*}
\vb{e}_{\sigma} &= \pdv{\vb{r}}{\sigma} = \left( 2 \, A \, \sinh \sigma \, \cos \tau, 2 \, A \, \cosh \sigma \, \sin \tau, 0 \right) \\[1em]
\vb{e}_{\tau} &= \pdv{\vb{r}}{\tau} = \left( - 2 \, A \, \cosh \sigma \, \sin \tau, 2 \, A \, \sinh \sigma \, \cos \tau,  0 \right) \\[1em]
\vb{e}_{z} &= \pdv{\vb{r}}{z} = (0, 0, 1)
\end{align*}

\subsection*{Ortogonalidad}
Para que el sistema sea ortogonal necesitamos que los productos escalares cruzados se anulen:
\begin{align*}
&\vb{e}_{\sigma} \cdot \vb{e}_{\tau} = \\[1em]
&= \left( 2 A \sinh \sigma \, \cos \tau, 2 A \cosh \sigma \, \sin \tau, 0 \right) \cdot \left( - 2 A \cosh \sigma \, \sin \tau, 2 A \sinh \sigma \, \cos \tau, 0 \right) \\[1em]
&= \left( 2 A \sinh \sigma \, \cos \tau \right) \left( - 2 A \cosh \sigma \, \sin \tau \right) {+} \left( 2 A \cosh \sigma \, \sin \tau \right) \left( 2 A \sinh \sigma \, \cos \tau \right) + \\[1em]
&+ \left( 0 \cdot 0 \right) = \\[1em]
&= - (2 A)^{2} \sinh \sigma \, \cosh \sigma \, \cos \tau \, \sin \tau + (2 A)^{2} \sinh \sigma \, \cosh \sigma \, \cos \tau \, \sin \tau = \\[1em]
&= (2 \, A)^{2} \left( \sinh \sigma \, \cos \tau (- \cosh \sigma \, \sin \tau + \cosh \sigma \, \sin \tau \cdot \sinh \sigma \, \cos \tau \right) = \\[1em]
&\vb{e}_{\sigma} \cdot \vb{e}_{\tau} = 0
\end{align*}
Efectivamente los dos términos se cancelan entre sí, por lo que $( \vb{e}_{\sigma} \perp \vb{e}_{\tau} )$. Además:
\begin{align*}
\vb{e}_{\sigma} \cdot \vb{e}_{z} =\vb{e}_{\tau} \cdot \vb{e}_{z} = 0
\end{align*}
porque $\vb{e}_{z}$ sólo tiene componente en $z$.
\par
Por tanto $(\vb{e}_{\sigma}, \vb{e}_{\tau}, \vb{e}_{z})$ son mutuamente ortogonales y el sistema es ortogonal.

\subsection*{Factores de escala}
Los factores de escala $(h_{\sigma}, h_{\tau}, h_{z})$ se obtienen como las normas de los vectores base:
\begin{align*}
h_{\sigma}^{2} &= \vb{e}_{\sigma} \cdot \vb{e}_{\sigma}, \hspace{1cm} h_{\tau}^{2} = \vb{e}_{\tau} \cdot \vb{e}_{\tau}, \hspace{1cm} 
h_{z}^{2} = \vb{e}_{z} \cdot \vb{e}_{z}
\end{align*}
Calculemos $(h_{\sigma}^{2})$:
\begin{align*}
h_{\sigma}^{2} &= (2\, A)^{2} \left( \sinh^{2} \sigma \, \cos^{2} \tau + \cosh^{2} \sigma \, \sin^{2} \tau \right) = \\[4pt]
&= (2 \, A)^{2} \left( \sinh^{2} \sigma (\cos^{2} \tau + \sin^{2} \tau) + \sin^{2} \tau (\cosh^{2} \sigma - \sinh^{2} \sigma) \right) =\\[4pt]
&= (2 \, A)^{2} \left( \sinh^{2} \sigma + \sin^{2} \tau \right) = \\[4pt]
&= 4\, A^{2} \left( \sinh^{2} \sigma + \sin^{2} \tau \right)
\end{align*}
De forma análoga para $(h_{\tau}^{2})$:
\begin{align*}
h_{\tau}^{2} &=(2 \, A)^{ } \left( \cosh^{2} \sigma \, \sin^{2} \tau + \sinh^{2} \sigma \, \cos^{2} \tau \right) \\[1em]
&= (2\, A)^{2} \left( \sinh^{2} \sigma + \sin^{2} \tau \right) = 4\, A^{2} \left( \sinh^{2} \sigma + \sin^{2} \tau \right)
\end{align*}
Finalmente,
\begin{align*}
h_{z}^{2} = \vb{e}_{z} \cdot \vb{e}_{z} = 1 \quad \Longrightarrow \quad h_{z} = 1
\end{align*}
\bigskip
\noindent Por tanto:
\begin{align*}
\boxed{h_{\sigma} = h_{\tau} = 2\, A\ \, \sqrt{\sinh^{2} \sigma + \sin^{2} \tau}, \hspace{1cm} h_{z} = 1}
\end{align*}
y el sistema $(\sigma, \tau, z)$ es ortogonal.

\end{document}
