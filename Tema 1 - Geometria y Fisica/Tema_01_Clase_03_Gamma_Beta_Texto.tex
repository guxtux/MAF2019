\documentclass[12pt]{article}
\usepackage[left=0.25cm,top=1cm,right=0.25cm,bottom=1cm]{geometry}
\textwidth = 20cm
\hoffset = -1cm
\usepackage[utf8]{inputenc}
\usepackage[spanish,es-tabla]{babel}
\usepackage[autostyle,spanish=mexican]{csquotes}
\usepackage[tbtags]{amsmath}
\usepackage{nccmath}
\usepackage{amsthm}
\usepackage{amssymb}
\usepackage{graphicx}
\usepackage{standalone}
\usepackage[outdir=./]{epstopdf}
\usepackage{siunitx}
\usepackage{physics}
\usepackage{color}
\usepackage{float}
\usepackage{multicol}
%\usepackage{milista}
\usepackage{enumitem}
\usepackage{anyfontsize}
\usepackage{anysize}
\usepackage{enumitem}
\usepackage{capt-of}
\usepackage{bm}
\usepackage{relsize}
\usepackage{placeins}
\usepackage{empheq}
\usepackage{cancel}
\usepackage{wrapfig}
\spanishdecimal{.}
\renewcommand{\baselinestretch}{1.5} 
\renewcommand\labelenumii{\theenumi.{\arabic{enumii}}}
\newcommand{\ptilde}[1]{\ensuremath{{#1}^{\prime}}}
\newcommand{\stilde}[1]{\ensuremath{{#1}^{\prime \prime}}}
\newcommand{\ttilde}[1]{\ensuremath{{#1}^{\prime \prime \prime}}}
\newcommand{\ntilde}[2]{\ensuremath{{#1}^{(#2)}}}


\usepackage{apacite}
\title{Funciones Gamma y Beta \\[0.3em]  \large{Matemáticas Avanzadas de la Física}\vspace{-3ex}}
\author{M. en C. Gustavo Contreras Mayén}
\date{ }
\begin{document}
\vspace{-4cm}
\maketitle
\fontsize{14}{14}\selectfont
\tableofcontents
\newpage

\section{Identidades Función Gamma.}
\subsection{Primera identidad.}

%Ref. Farrell - 1-18
\textbf{Demuestra que}:
\begin{align*}
\sqrt{\pi} \, \Gamma(2 \, n + 1) = 2^{2n} \, \Gamma \left( n + \dfrac{1}{2} \right) \, \Gamma(n + 1)
\end{align*}
donde $n$ es cualquier valor entero positivo.
\par
Para resolver este ejercicio será necesario apoyarse de otras identidades que se indican en las notas de trabajo y también en los videos del canal de YouTube.
\par
La primera identidad que ocuparemos es:
\begin{align*}
\Gamma\left( n + \dfrac{1}{2} \right) = \dfrac{(2 \, n {-} 1)(2 \, n {-} 3)(2 \, n {-} 5) \ldots (3)(1)\sqrt{\pi}}{2^{n}}
\end{align*}
Como es una identidad fácil de demostrar, veremos su desarrollo para luego regresar a la identidad del problema inicial. El argumento $n + 1/2$ se puede escribir como: $(2 \, n + 1)/2$.
\par
Una identidad de apoyo que conocemos es:
\begin{align*}
\Gamma(x) = (x - 1) \, \Gamma (x - 1)
\end{align*}
Si hacemos que:
\begin{align*}
x = \dfrac{(2 \, n + 1)}{2}
\end{align*}
Tendremos que:
\begin{align*}
\Gamma \left( \dfrac{2 \, n + 1}{2} \right) = \left( \dfrac{2 \, n + 1}{2} - 1 \right) \, \Gamma \left( \dfrac{2 \, n + 1}{2} - 1 \right)
\end{align*}
Es decir:
\begin{align*}
\Gamma \left( n + \dfrac{1}{2} \right) = \left( \dfrac{2 \, n - 1}{2} \right) \, \Gamma \left( \dfrac{2 \, n - 1}{2} \right)
\end{align*}
Seguimos reduciendo el argumento en una unidad para $\Gamma [(2 \, n - 1)/2]$:
\begin{align*}
\Gamma \left( \dfrac{2 \, n - 1}{2} \right) = \left( \dfrac{2 \, n - 3}{2} \right) \, \Gamma \left( \dfrac{2 \, n - 3}{2} \right)
\end{align*}
Así tendremos que:
\begin{align*}
\Gamma \left( n + \dfrac{1}{2} \right) = \left( \dfrac{2 \, n - 1}{2} \right) \, \left( \dfrac{2 \, n - 3}{2} \right) \, \Gamma \left( \dfrac{2 \, n - 3}{2} \right)
\end{align*}

Recordemos que $\Gamma(\frac{1}{2}) = \sqrt{\pi}$,  queremos repetir el cambio en el argumento de la función Gamma, para llegar a $\Gamma(\frac{1}{2})$.
\par
Por ejemplo: podemos escribir:
\begin{align*}
\Gamma \left( \dfrac{7}{2} \right) = \dfrac{5}{2} \cdot \dfrac{3}{2} \cdot \dfrac{1}{2} \, \Gamma \left( \dfrac{1}{2} \right)
\end{align*}
Vemos que obtuvimos $\Gamma(\frac{1}{2})$ multiplicado por tres factores;  es más, este número tres es el mismo entero que se presenta en el argumento de $\Gamma(\frac{7}{2})$ cuando se escribe $\Gamma(3 + \frac{1}{2})$. Este $3$ corresponde a $n$ de $\Gamma \left(n + \frac{1}{2}\right)$.
\par
Po lo que podemos decir que si comenzamos con $\Gamma \left(n + \frac{1}{2}\right)$, tenemos que ir disminuyendo el argumento en una unidad $n$ veces para llegar a $\frac{1}{2}$.
\par
También observamos que cada vez que se repite el proceso hay otro $2$ en el denominador. Como el proceso se repetirá $n$ veces, podemos factorizar los $2$ y escribir $2^{n}$.
\par
Entonces llegamos al resultado:
\begin{align*}
\Gamma\left( n + \dfrac{1}{2} \right) = \dfrac{(2 \, n {-} 1)(2 \, n {-} 3)(2 \, n {-} 5) \ldots (3)(1)\sqrt{\pi}}{2^{n}}
\end{align*}    

\begin{mdframed}
 \textbf{Ejercicio. Un punto adicional.}
 
Demuestra que si $n$ es un entero positivo, se tiene que:
\begin{align*}
\Gamma\left( n - \dfrac{1}{2} \right) = \dfrac{(2 \, n {-} 3)(2 \, n {-} 5) \ldots (3)(1)\sqrt{\pi}}{2^{n-1}}
\end{align*}
\end{mdframed}

Con el resultado obtenido, regresamos al ejercicio inicial:
\begin{align*}
2^{2n} \, \Gamma \left( n + \dfrac{1}{2} \right) \, \Gamma(n + 1) =
\end{align*}    

Donde ahora usamos el resultado de la identidad de apoyo anterior:
\begin{align*}
= \dfrac{2^{2n} (2 \, n {-} 1)(2 \, n - 3) \ldots 3 \cdot 1 \, \sqrt{\pi} \, \Gamma(n + 1)}{2^{n}}
\end{align*}

Multiplicamos el lado derecho de la igualdad por un $1$:
\begin{align*}
\dfrac{2 \, n (2 \, n - 2)(2 \, n - 4) \ldots (4)(2)}{2 \, n (2 \, n - 2)(2 \, n - 4) \ldots (4)(2)}
\end{align*}
Para obtener:
\begin{align*}
= \dfrac{2^{2n} (2 n) (2 n {-} 2)(2 n {-} 4) \ldots 4 \cdot 3 \cdot 2 \cdot 1 \, \sqrt{\pi} \, \Gamma(n {+} 1)}{2^{n} (2 n)(2 n {-} 2)(2 n {-} 4) \ldots (4) (2)}
\end{align*}

% \begin{tikzpicture}[overlay]
%     \draw [fill, color=yellow!50, opacity=0.3] (1.4, 1.3) rectangle (8.1, 2); 
%     \draw [fill, color=blue!50, opacity=0.3] (2.3, 0.7) rectangle (9, 1.2);
% \end{tikzpicture}

El numerador del resultado anterior es:
\begin{align*}
2^{2n} (2 n)! \, \sqrt{\pi} \, \Gamma(n + 1)
\end{align*}
Mientras que el denominador es igual a:
\begin{align*}
2^{n} \, 2^{n} \, n!
\end{align*}

Si escribimos la cantidad:
\begin{align*}
(2 \, n)! = \Gamma(2 \, n + 1)
\end{align*}
y dejando que: $n! = \Gamma(n + 1)$, el resultado:
\begin{align*}
\dfrac{\Cancel[red]{2^{2n}} \, \sqrt{\pi} \, \Gamma(2 \, n + 1) \, \Cancel[blue]{\Gamma(n + 1)}}{\Cancel[red]{2^{2n}} \Cancel[blue]{\Gamma(n + 1)}}
\end{align*}

Entonces hemos demostrado que:
\begin{align*}
\sqrt{\pi} \, \Gamma(2 \, n + 1) = 2^{2n} \, \Gamma \left( n + \dfrac{1}{2} \right) \, \Gamma(n + 1) \qed
\end{align*}    

Si usamos:
\begin{align*}
2 \, n \, \Gamma( 2 \, n) &= \Gamma(2 \, n + 1) \\[0.5em] 
n \, \Gamma(n) &= \Gamma(n + 1)
\end{align*}

Es posible reescribir la ecuación obtenida como:
\begin{align*}
\sqrt{\pi} \, \Gamma(2 \, n) = 2^{2n-1} \, \Gamma(n) \, \Gamma \left( n + \dfrac{1}{2} \right)
\end{align*}

Que es conocida como la \emph{fórmula de duplicación de Legendre}. Es conveniente utilizar la fórmula de duplicación de Legendre expresada como una razón:
\begin{align*}
\dfrac{\Gamma(2 \, n)}{\Gamma (n)} = \dfrac{\Gamma \left( n + \dfrac{1}{2} \right)}{\sqrt{\pi} \, 2^{1- 2n}}
\end{align*}


\section{Problema con \texorpdfstring{$\Gamma(x)$}{G(x)} y \texorpdfstring{$B(x, y)$}{B(x,y)}.}
\subsection{Ejercicio 2.}

%REf. Farrell 1.31

\textbf{Demuestra que}: 
\begin{align*}
\Gamma(p) \, \Gamma(1 - p) = \dfrac{\pi}{\sin p \, \pi} \hspace{1cm} p \mbox{ no entero}
\end{align*}
La razón por que $p$ sea un valor no entero es evidente: No se puede ocupar cualquier valor entero para $p$, ya que el denominador sería cero. Nuevamente haremos uso de una serie de identidades tanto de la función Gamma como de la función Beta.
\par
La demostración de alguna de ellas se puede revisar en el canal de YouTube, mencionaremos la idea general para demostrar una identidad en particular.
\par
Ocuparemos las siguientes identidades:
\begin{align}
B(x, y) = \dfrac{\Gamma(x) \, \Gamma(y)}{\Gamma(x + y)}
\label{eq:ecuacion_01}
\end{align}

\begin{align}
B(p, 1 - p) = \dfrac{\pi}{\sin p \, \pi} \hspace{1cm} 0 < p < 1
\label{eq:ecuacion_02}
\end{align}
De esta identidad hablaremos más adelante. Las otras identidades que ocuparemos son:
\begin{align}
\Gamma(x + 1) &= x \, \Gamma(x) \label{eq:ecuacion_03} \\[0.5em]
\Gamma(-x) &= \dfrac{\Gamma(1 - x)}{-x}, \hspace{1cm} x \neq 0, 1, 2, \ldots \label{eq:ecuacion_04}
\end{align}

Comenzamos haciendo que $0 < p < 1$, así que ocupamos la ec. (\ref{eq:ecuacion_01}), que es la identidad que relaciona a la función Gamma con la función Beta:
\begin{align*}
\dfrac{\Gamma(p) \, \Gamma(1 - p)}{\Gamma(p + 1 - p)} = B(p, 1 - p) 
\end{align*}
Hacemos que $h = p + 1$, usando las ecs. (\ref{eq:ecuacion_03}) y (\ref{eq:ecuacion_04}) y el resultado de la diapositiva anterior, se tiene que:
\begin{align*}
\Gamma(h) \, \Gamma(1 - h) &=  \Gamma(p + 1) \, \Gamma(-p) = \\[0.5em] 
&= p \, \Gamma(p)  \, \dfrac{\Gamma(1 - p)}{- p} = \\[0.5em] 
&= \dfrac{-\pi}{\sin p \, \pi} \hspace{1cm} 0 < p < 1 \\[0.5em]
&= \dfrac{-\pi}{\sin (h - 1) \, \pi} \\[0.5em] 
&= \dfrac{-\pi}{- \sin h \, \pi} \\[0.5em] 
&= \dfrac{\pi}{\sin h \, \pi} \hspace{1cm} 1 < p < 2
\end{align*}

Se puede demostrar que la expresión sigue siendo válida para \hfill \break $2 < h < 3$, para luego por medio de inducción matemática ver que se cumple para cualquier \emph{valor no entero positivo} $h$.
\par
Ahora tomamos el intervalo $0 < p < 1$ y hacemos que $h = p - 1$,  procedemos de la misma manera que con los valores no enteros positivos.
\begin{align*}
\Gamma(h) \, \Gamma(1 - h) = \dfrac{\pi}{\sin h \, \pi} \hspace{1cm} -1 < h < 0
\end{align*}

Como se vio previamente, esta expresión es válida para $-2 < h < -1$,  se puede demostrar por inducción que se cumple para cualquier \emph{valor no entero negativo} $h$.
\par
Entonces llegamos a que:

\begin{align*}
\Gamma(p) \, \Gamma(1 - p) = \dfrac{\pi}{\sin p \, \pi}
\end{align*}
se cumple para cualquier valor no entero $p$ ya sea positivo o negativo.

\section{La identidad para \texorpdfstring{$B(x, y)$}{B(x, y)}.}
\subsection{Primer paso.}

Se requiere inicialmente demostrar que la función Beta:
\begin{align*}
B(x, y) = \int_{0}^{1} t^{x - 1} \, (1 - y)^{y-1} \dd{t}
\end{align*}
se puede expresar como una integral cuyo intervalo de integración es $[0, \infty)$
\par
El problema en primer lugar es buscar un cambio de la variable de integración,  $t = f(u)$,  de modo que a medida que $t$, en la integral de $B (x, y)$ recorra el intervalo de cero a la unidad. La nueva variable $u$ variará continua y monótonamente de cero a infinito. También requerimos que la relación $t = f(u)$ tenga una derivada continua para todo $u \geq 0$.
\par
Dado que tenemos que \enquote{extender} el intervalo $0 \leq t < 1$ hasta el intervalo $0 \leq u < \infty$,  luego de una breve reflexión que una fórmula fraccionaria funcionará:
\begin{align*}
t = \dfrac{u}{u + 1}
\end{align*}
Entonces:
\begin{align*}
\dv{t}{u} &= \dfrac{1}{(u + 1)^{2}} \\[1em] 
1 - t &= \dfrac{1}{u + 1}
\end{align*}

Entonces de la definición de la función $B(x, y)$:
\begin{align*}
B(x, y) = \int_{0}^{1} t^{x - 1} \, (1 - y)^{y-1} \dd{t}
\end{align*}
También se puede expresar por una integral cuyo intervalo va de $0$ a  $\infty$:
\begin{align*}
B(x, y) = \scaleint{5ex}_{\bs 0}^{\infty} \dfrac{u^{x-1}}{(u + 1)^{x+y}} \dd{u} \hspace{1cm} x > 0, y > 0
\end{align*}

\subsection{Segundo paso.}

Lo que queremos ahora es evaluar:
\begin{align*}
B(p, 1 - p)
\end{align*}
donde $p$ es cualquier número positivo menor que uno.

Haciendo que  $x = p$ con $0 < p < 1$  y $y = 1 - p$, entonces en la integral del resultado anterior:
\begin{align*}
B(x, y) = \scaleint{5ex}_{\bs 0}^{\infty} \dfrac{u^{x-1}}{(u + 1)^{x+y}} \dd{u} \hspace{1cm} x > 0, y > 0
\end{align*}
Tenemos que:
\begin{align*}
B(p, 1 - p) = \scaleint{5ex}_{\bs 0}^{\infty} \dfrac{u^{p-1}}{(u + 1)} \dd{u} \hspace{1cm} 0 < p < 1
\end{align*}

¿Cómo resolvemos esta integral? Si el intervalo de integración fuera de $(-\infty, \infty)$ podríamos evaluar la integral con un integral de contorno en el plano complejo de $z = x +  i \, y$ 
\par
Hagamos que el intervalo de integración se \enquote{extienda} a $(-\infty, \infty)$  mediante la transformación $u = e^{x}$, que deja a $x$ en $-\infty$ cuando $u = 0$, y deja a $x$ en $+\infty$ cuando $u$ está en $+\infty$. Como $\dv*{u}{x} = e^{x}$, se tiene que:
\begin{align*}
B(p, 1 - p) = \scaleint{5ex}_{\bs -\infty}^{\infty} \dfrac{e^{px}}{1 + e^{x}} \dd{x} \hspace{1cm} 0 < p < 1
\end{align*}

Para resolver la integral anterior, nos podemos apoyar con la siguiente expresión:
\begin{align*}
\scaleoint{6ex}_{C} = \dfrac{e^{pz}}{1 + e^{z}} \hspace{1cm} z = x + i \, y
\end{align*}
tomando el sentido positivo alrededor del rectángulo $C$ en el plano complejo, con vértices en $z = -R$, $z = R$, $z = R + 2 \,  \pi \, i$, $z = -R +  2 \, \pi \, i$. Como se revisó en el curso de Variable Compleja I, habrá que revisar si el integrando es analítico, qué puntos son polos, etc.
\par
Luego de resolver la integral con la teoría de variable compleja, podemos garantizar que la identidad para la función Beta es:
\begin{align*}
B(p, 1 - p) = \dfrac{\pi}{\sin p \, \pi} \hspace{1cm} 0 < p < 1
\end{align*}
\end{document}

