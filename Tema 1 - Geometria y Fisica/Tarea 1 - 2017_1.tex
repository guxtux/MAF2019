\documentclass[12pt]{article}
%\usepackage[left=0.25cm,top=1cm,right=0.25cm,bottom=1cm]{geometry}
\usepackage{geometry}
\textwidth = 20cm
\hoffset = -1cm
\usepackage[utf8]{inputenc}
\usepackage[spanish,es-tabla]{babel}
\usepackage{amsmath}
\usepackage{nccmath}
\usepackage{amsthm}
\usepackage{amssymb}
\usepackage{graphicx}
\usepackage{color}
\usepackage{float}
\usepackage{multicol}
\usepackage{enumerate}
\usepackage{anyfontsize}
\usepackage{anysize}
\usepackage{enumitem}
\usepackage{capt-of}
\usepackage{bm}
\usepackage{relsize}
\spanishdecimal{.}
\setlist[enumerate]{itemsep=0mm}
\renewcommand{\baselinestretch}{1.2}
\let\oldbibliography\thebibliography
\renewcommand{\thebibliography}[1]{\oldbibliography{#1}
\setlength{\itemsep}{0pt}}
%\marginsize{1.5cm}{1.5cm}{0cm}{2cm}
\title{Ejercicios para la Tarea 1 - Tema 1 \\ \large{Matemáticas Avanzadas de la Física}}
%\subtitle{Fecha de entrega: 8 de marzo de 2016.}
\date{ }
\begin{document}
\vspace{-4cm}
%\renewcommand\theenumii{\arabic{theenumii.enumii}}
\renewcommand\labelenumii{\theenumi.{\arabic{enumii}}}
\maketitle
\fontsize{14}{14}\selectfont
Para las respuestas de la tarea, te pedimos sea lo más claro posible, presentando de manera organizada tu solución. La calificación de esta tarea está en función del número de ejercicios que entregues, cuentas con el suficiente tiempo para resolverla y hacer un buen esfuerzo.
\begin{enumerate}
\item Para el sistema de coordenadas parabólicas tal que
\[ x = \dfrac{\xi^{2} - \eta^{2}}{2}  \hspace{1cm} y = \eta \xi \hspace{1cm} z = z\]
Calcula los operadores diferenciales:
\begin{enumerate}
\item $\nabla \phi$
\item $\nabla \cdot A$
\item $\nabla \times A$
\item $\nabla^{2} \phi$
\end{enumerate}
\item Para el sistema de coordenadas elípticas cilíndricas $(u, \theta, z)$ dadas por
\[ \begin{split}
x &= a \cosh u \cos \theta \\
y &= a \sinh u \sin \theta \\
z &= z
\end{split} \]
donde $a$ es una constante. Calcula los operadores diferenciales:
\begin{enumerate}
\item $\nabla \phi$
\item $\nabla \cdot A$
\item $\nabla \times A$
\item $\nabla^{2} \phi$
\end{enumerate}
\item Demuestra que el operador momento angular en coordenadas esféricas, está dado por:
\[ \mathbf{L} = - i (r \times \bm{\nabla}) = i \left( \bm{\theta}_{0} \dfrac{1}{\sin \theta} \; \dfrac{\partial}{\partial \varphi} - \bm{\varphi}_{0} \dfrac{\partial}{\partial \theta} \right)  \]
\item Partiendo del problema anterior, encuentra los operadores $L_{x}$, $L_{y}$, $L_{z}$, los operadores $L_{+}$, $L_{-}$ y el operador $L^{2}$.
\item Partiendo del Lagrangiano de una partícula libre en coordenadas esféricas:
\[ L = \dfrac{m}{2} \left( \dot{r}^{2} + r^{2} \dot{\theta}^{2} + r^{2} \sin \theta^{2} \dot{\varphi}^{2} \right)\]
Calcula los símbolos de Christoffel.
\item Las ecuaciones de Navier-Stokes para el flujo de un fluido incompresible
\[ - \bm{\nabla} \times ( \mathbf{v} \times (\bm{\nabla} \times \mathbf{v} )) =  \dfrac{\eta}{\rho_{0}} \bm{\nabla}^{2} (\bm{\nabla} \times \mathbf{v}) \]
Donde $\eta$ es la viscosidad y $\rho_{0}$ la densidad del fluido. Para un flujo axial dentro de un cilindro, consideremos que la velocidad $\mathbf{v}$ está dada por
\[ \mathbf{v} =  \mathbf{k} v (\rho) \]
Considera que
\[ \bm{\nabla} \times (\mathbf{v} \times (\bm{\nabla} \times \mathbf{v})) = 0 \]
para este valor de $\mathbf{v}$.
\\
Demuestra que
\[ \bm{\nabla}^{2} ( \bm{\nabla} \times \mathbf{v}) = 0  \]
nos lleva a la ecuación diferencial
\[ \dfrac{1}{\rho} \dfrac{d}{d \rho} \left( \rho \dfrac{d^{2} v}{d \rho^{2}} \right) -  \dfrac{1}{\rho^{2}} \dfrac{d v}{d \rho} = 0 \]
y que la siguiente expresión es solución de la misma ecuación diferencial
\[ v = v_{0} + a_{2} \rho^{2} \]
\item El cálculo de efecto ''pinch'' en magnetohidrodinámica, involucra la evaluación de $(\mathbf{B} \cdot \bm{\nabla}) \mathbf{B}$. Si la inducción magnética $\mathbf{B}$ se toma como $\mathbf{B} = \bm{\varphi}_{0} B_{\varphi} (\rho)$, demostrar que
\[ (\mathbf{B} \cdot \bm{\nabla}) \mathbf{B} = - \dfrac{\bm{\rho_{0}} B_{\varphi}^{2}}{\rho}	 \]
\item Un cierto campo de fuerza está dado por
\[ \mathbf{F} = \mathbf{r}_{0} \dfrac{2 P \cos \theta}{r^{3}} + \bm{\theta}_{0} \dfrac{P}{r^{3}} \sin \theta, \hspace{1.5cm} r \geq P/2 \]
en coordenadas esféricas.
\begin{enumerate}
\item Revisa $\bm{\nabla} \times \mathbf{F}$ para checar si existe un potencial.
\item Calcular $\mathlarger{\oint} \mathbf{F} \cdot d \bm{\lambda}$ para un círculo unitario en el plano $\theta = \pi/2$.\
¿Qué nos dice esto sobre la fuerza?¿Es conservativa o no conservativa?
\item Si considera que $\mathbf{F}$ se puede describir por $\mathbf{F} = - \bm{\nabla}\psi$, encuentra $\psi$. De otra manera argumenta que no es posible que un potencial exista.
\end{enumerate}
\item Usando las ecuaciones de Maxwell, demostrar que
\[ \dfrac{d}{dt} ( P_{\text{mec}} + P_{\text{campo}} )_{\alpha} = \sum_{\beta} \int_{V} \dfrac{\partial}{\partial x_{\beta}} T_{\alpha \beta} \; d^{3} x \]
con
\[ T_{\alpha \beta} =  \dfrac{1}{4 \pi} \left[ E_{\alpha} E_{\beta} + B_{\alpha} B_{\beta} - \dfrac{1}{2} (\mathbf{E} \cdot \mathbf{E} + \mathbf{B} \cdot \mathbf{B}) \delta_{\alpha \beta} \right]  \]
donde
\[ P_{\text{campo}} = \dfrac{1}{4 \pi c} \int_{V} (\mathbf{E} \times \mathbf{B}) \; d^{3} x \]
Nota: el ejercicio está hecho en el libro \emph{Classical Electrodynamics}, de John Jackson, en las secciones 6.8 y 6.9. En esencia la actividad consiste en el manejo de las ecuaciones de Maxwell (divergencias, rotacionales, etc.) y usar la notación de índices que se revisó en clase.
\end{enumerate}
\end{document}