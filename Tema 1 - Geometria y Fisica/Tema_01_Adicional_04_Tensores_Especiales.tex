\documentclass[hidelinks,12pt]{article}
\usepackage[left=0.25cm,top=1cm,right=0.25cm,bottom=1cm]{geometry}
%\usepackage[landscape]{geometry}
\textwidth = 20cm
\hoffset = -1cm
\usepackage[utf8]{inputenc}
\usepackage[spanish,es-tabla, es-lcroman]{babel}
\usepackage[autostyle,spanish=mexican]{csquotes}
\usepackage[tbtags]{amsmath}
\usepackage{nccmath}
\usepackage{amsthm}
\usepackage{amssymb}
\usepackage{mathrsfs}
\usepackage{graphicx}
\usepackage{subfig}
\usepackage{caption}
%\usepackage{subcaption}
\usepackage{standalone}
\graphicspath{{Imagenes/}{../Imagenes/}}
\usepackage[outdir=./Imagenes/]{epstopdf}
\usepackage{siunitx}
\usepackage{physics}
\AtBeginDocument{\RenewCommandCopy\qty\SI}
\ExplSyntaxOn
\msg_redirect_name:nnn { siunitx } { physics-pkg } { none }
\ExplSyntaxOff
\usepackage{color}
\usepackage{float}
\usepackage{hyperref}
\usepackage{multicol}
\usepackage{multirow}
%\usepackage{milista}
\usepackage{anyfontsize}
\usepackage{anysize}
%\usepackage{enumerate}
\usepackage[shortlabels]{enumitem}
\usepackage{capt-of}
\usepackage{bm}
\usepackage{mdframed}
\usepackage{relsize}
\usepackage{placeins}
\usepackage{empheq}
\usepackage{cancel}
\usepackage{pdfpages}
\usepackage{wrapfig}
\usepackage[flushleft]{threeparttable}
\usepackage{makecell}
\usepackage{fancyhdr}
\usepackage{tikz}
\usepackage{bigints}
\usepackage{tcolorbox}
\tcbuselibrary{breakable}
\usepackage{scalerel}
\usepackage{pgfplots}
\usepackage{pdflscape}
\usepackage{enumitem}
\pgfplotsset{compat=1.16}
\spanishdecimal{.}
\renewcommand{\baselinestretch}{1.5}
\renewcommand{\labelenumii}{\arabic{enumi}.\arabic{enumii}}
\renewcommand{\labelenumiii}{\arabic{enumi}.\arabic{enumii}.\arabic{enumiii}}

\newcommand{\ptilde}[1]{\ensuremath{{#1}^{\prime}}}
\newcommand{\stilde}[1]{\ensuremath{{#1}^{\prime \prime}}}
\newcommand{\ttilde}[1]{\ensuremath{{#1}^{\prime \prime \prime}}}
\newcommand{\ntilde}[2]{\ensuremath{{#1}^{(#2)}}}
\newcommand{\pderivada}[1]{\ensuremath{{#1}^{\prime}}}
\newcommand{\sderivada}[1]{\ensuremath{{#1}^{\prime \prime}}}
\newcommand{\tderivada}[1]{\ensuremath{{#1}^{\prime \prime \prime}}}
\newcommand{\nderivada}[2]{\ensuremath{{#1}^{(#2)}}}


\newtheorem{defi}{{\it Definición}}[section]
\newtheorem{teo}{{\it Teorema}}[section]
\newtheorem{ejemplo}{{\it Ejemplo}}[section]
\newtheorem{propiedad}{{\it Propiedad}}[section]
\newtheorem{lema}{{\it Lema}}[section]
\newtheorem{cor}{Corolario}
\newtheorem{ejer}{Ejercicio}[section]

\newlist{milista}{enumerate}{2}
\setlist[milista,1]{label=\arabic*)}
\setlist[milista,2]{label=\arabic{milistai}.\arabic*)}
\newlength{\depthofsumsign}
\setlength{\depthofsumsign}{\depthof{$\sum$}}
\newcommand{\nsum}[1][1.4]{% only for \displaystyle
    \mathop{%
        \raisebox
            {-#1\depthofsumsign+1\depthofsumsign}
            {\scalebox
                {#1}
                {$\displaystyle\sum$}%
            }
    }
}
\def\scaleint#1{\vcenter{\hbox{\scaleto[3ex]{\displaystyle\int}{#1}}}}
\def\scaleoint#1{\vcenter{\hbox{\scaleto[3ex]{\displaystyle\oint}{#1}}}}
\def\scaleiint#1{\vcenter{\hbox{\scaleto[3ex]{\displaystyle\iint}{#1}}}}
\def\scaleiiint#1{\vcenter{\hbox{\scaleto[3ex]{\displaystyle\iiint}{#1}}}}
\def\bs{\mkern-12mu}

\newcommand{\Cancel}[2][black]{{\color{#1}\cancel{\color{black}#2}}}

% \newcommand{\qed}{\tag*{$\blacksquare$}}
\renewcommand{\qed}{\hfill\blacksquare}


\usetikzlibrary{babel}
\usepackage{tikz-3dplot}
\setlength{\tabcolsep}{12pt}
\usepackage[outline]{contour} % glow around text
\colorlet{veccol}{green!50!black}
\colorlet{myred}{red!90!black}
\colorlet{myblue}{blue!80!black}
\colorlet{mydarkblue}{blue!50!black}
\tikzset{>=latex} % for LaTeX arrow head
\tikzstyle{proj}=[projcol!80,line width=0.08] %very thin
\tikzstyle{area}=[draw=veccol,fill=veccol!80,fill opacity=0.6]
\tikzstyle{vector}=[-stealth,myblue,thick,line cap=round]
\tikzstyle{unit vector}=[->,veccol,thick,line cap=round]
\tikzstyle{dark unit vector}=[unit vector,veccol!70!black]
\usetikzlibrary{angles,quotes} % for pic (angle labels)
\contourlength{1.3pt}

\title{Tensores Especiales\\ \large{Material de consulta previo}\vspace{-3ex}}
\author{M. en C. Gustavo Contreras Mayén}
\date{ }

\begin{document}
\vspace{-4cm}
\maketitle
\fontsize{14}{14}\selectfont
\tableofcontents
\newpage

%Ref. Heinbockel (1996) Introduction to tensor calculus and continuum mechanics.
\section{Tensores especiales.}

Saber cómo se definen los tensores y reconocer un tensor cuando se nos presenta son dos cosas diferentes. Algunas cantidades, que son tensores, aparecen con frecuencia en problemas aplicados y debemos aprender a reconocer estos tensores especiales cuando ocurren.

\subsection{Tensor métrico.}

Definimos $y^{i}, i = 1, \ldots, N$ como coordenadas independientes en un sistema de coordenadas cartesiano ortogonal $N$ dimensional. La distancia al cuadrado entre dos puntos $y^{i}$ e $y^{i} + \dd{y}^{i}, i = 1, \ldots, N$ está definida por la expresión:
\begin{align}
\dd{s}^{2} = \dd{y}^{m} \, \dd{y}^{m} = (\dd{y}^{1})^{2} + (\dd{y}^{2})^{2} + \ldots + (\dd{y}^{N})^{2}
\label{eq:ecuacion_01_03_01}
\end{align}
Suponemos que las coordenadas $y^{i}$ están relacionadas con un conjunto de coordenadas independientes generalizadas $x^{i}, i = 1, \ldots, N$ mediante un conjunto de ecuaciones de transformación:
\begin{align}
y^{i} = y^{i} (x^{1}, x^{2}, \ldots, x^{N}), \hspace{1cm} i = 1, \ldots, N
\label{eq:ecuacion_01_03_02}
\end{align}
Para enfatizar que cada $y^{i}$ depende de las coordenadas $x$, a veces usamos la notación $y^{i} = y^{i} (x)$, para $i = 1, \ldots, N$.
\par
El diferencial de cada coordenada se puede escribir como:
\begin{align}
\dd{y}^{m} = \pdv{y^{m}}{x^{j}} \dd{x}^{j}, \hspace{1cm} m = 1, \ldots, N
\label{eq:ecuacion_01_03_03}
\end{align}
y en consecuencia, en las coordenadas generalizadas $x$, la distancia al cuadrado, calculada a partir de la ec. (\ref{eq:ecuacion_01_03_03}), se convierte en una forma cuadrática. Sustituyendo la ec. (\ref{eq:ecuacion_01_03_03}) en la ec. (\ref{eq:ecuacion_01_03_01}) encontramos que:
\begin{align}
\dd{s}^{2} = \pdv{y^{m}}{x^{i}} \, \pdv{y^{m}}{x^{j}} \dd{x}^{i} \dd{x}^{j} = g_{ij} \, \dd{x}^{i} \dd{x}^{j}
\label{eq:ecuacion_01_03_04}
\end{align}
donde:
\begin{align}
g_{ij} = \pdv{y^{m}}{x^{i}} \, \pdv{y^{m}}{x^{j}} \hspace{1.5cm} i, j = 1, \ldots, N
\label{eq:ecuacion_01_03_05}
\end{align}
es llamado \textbf{el tensor métrico del espacio} definido por las coordenadas $x^{i}$ donde $i = 1, \ldots, N$.
\par
Aquí, la $g_{ij}$ son funciones de las coordenadas $x$, a veces, se escriben como $g_{ij} = g_{ij} (x)$. Además, las métricas $g_{ij}$ son simétricas en los índices $i$ y $j$, de modo que $g_{ij} = g_{ji}$ para todos los valores de $i$ y $j$ en el rango de los índices.
\par
Si transformamos a otro sistema de coordenadas, digamos $\overline{x}^{i}, i = 1, \ldots, N$, entonces el elemento de longitud de arco al cuadrado se expresa en términos de las coordenadas barradas:
\begin{align*}
\dd{s}^{2} = \overline{g}_{ij} \dd{\overline{x}}^{i} \dd{\overline{x}}^{j}
\end{align*}
donde $\overline{g}_{ij} = \overline{g}_{ij} (\overline{x})$ es una función de las coordenadas barradas.

\section{Coordenadas curvilíneas.}

Considera un conjunto de ecuaciones de transformación generales desde coordenadas rectangulares $(x, y, z)$ a coordenadas curvilíneas $(u, v, w)$. Estas ecuaciones de transformación y las correspondientes transformaciones inversas se representan como:
\begin{align}
\begin{aligned}
x &= x(u, v, w) \hspace{1.3cm} u = u(x, y, z) \\[0.5em]
y &= y(u, v, w) \hspace{1.3cm} v = v(x, y, z) \\[0.5em]
z &= z(u, v, w) \hspace{1.3cm} w = w(x, y, z)
\end{aligned}
\label{eq:ecuacion_01_03_10}
\end{align}
Aquí: 
\begin{align*}
y^{1} = x, \hspace{0.5cm} y^{2} = y, \hspace{0.5cm} y^{3} = z
\end{align*}
son las coordenadas cartesianas, mientras que:
\begin{align*}
x^{1} = u, \hspace{0.5cm} x^{2} = v, \hspace{0.5cm} x^{3} = w
\end{align*}
son las coordenadas generalizadas y $N = 3$.
\par
La intersección de las superficies coordenadas $u = c_{1}, v = c_{2}$ y $w = c_{3}$ define las curvas coordenadas del sistema de coordenadas curvilíneas.
\par
La sustitución de las ecuaciones de transformación dadas en la ec. (\ref{eq:ecuacion_01_03_10}) en el vector de posición $\va{r} = x\, \vu{e}_{1} + y\, \vu{e}_{2} + z\, \vu{e}_{3}$ genera el vector de posición que es una función de las coordenadas generalizadas y:
\begin{align*}
\va{r} = \va{r} (u, v, w) = x (u, v, w) \, \vu{e}_{1} + y (u, v, w) \, \vu{e}_{2} + z (u, v, w) \, \vu{e}_{3}
\end{align*}
y en consecuencia:
\begin{align*}
\dd{\va{r}} = \pdv{\va{r}}{u} \dd{u} + \pdv{\va{r}}{v} \dd{v} + \pdv{\va{r}}{w} \dd{w}
\end{align*}
donde:
\begin{align}
\begin{aligned}
\va{E}_{1} &= \pdv{\va{r}}{u} = \pdv{x}{u} \, \vu{e}_{1} + \pdv{y}{u} \, \vu{e}_{2} + \pdv{z}{u} \, \vu{e}_{3} \\[0.5em]
\va{E}_{2} &= \pdv{\va{r}}{v} = \pdv{x}{v} \, \vu{e}_{1} + \pdv{y}{v} \, \vu{e}_{2} + \pdv{z}{v} \, \vu{e}_{3} \\[0.5em]
\va{E}_{3} &= \pdv{\va{r}}{w} = \pdv{x}{w} \, \vu{e}_{1} + \pdv{y}{w} \, \vu{e}_{2} + \pdv{z}{w} \, \vu{e}_{3}
\end{aligned}
\label{eq:ecuacion_01_03_11}
\end{align}
son los vectores tangentes a las curvas coordenadas. El elemento de longitud de arco en las coordenadas curvilíneas es:
\begin{align}
\begin{aligned}
\dd{s}^{2} = \dd{\va{r}} \cdot \dd{\va{r}} &= \pdv{\va{r}}{u} \cdot \pdv{\va{r}}{u} \dd{u} \dd{u} + \pdv{\va{r}}{u} \cdot \pdv{\va{r}}{v} \dd{u} \dd{v} + \pdv{\va{r}}{u} \cdot \pdv{\va{r}}{w} \dd{u} \dd{w} + \\[0.5em]
&+ \pdv{\va{r}}{v} \cdot \pdv{\va{r}}{u} \dd{v} \dd{u} + \pdv{\va{r}}{v} \cdot \pdv{\va{r}}{v} \dd{v} \dd{v} + \pdv{\va{r}}{v} \cdot \pdv{\va{r}}{w} \dd{v} \dd{w} + \\[0.5em]
&+ \pdv{\va{r}}{w} \cdot \pdv{\va{r}}{u} \dd{w} \dd{u} + \pdv{\va{r}}{w} \cdot \pdv{\va{r}}{v} \dd{w} \dd{v} + \pdv{\va{r}}{w} \cdot \pdv{\va{r}}{w} \dd{w} \dd{w}
\end{aligned}
\label{eq:ecuacion_01_03_12}
\end{align}

Utilizando la convención para la suma, la expresión anterior la podemos escribir con notación de índices. Por lo que se definen las cantidades:
\begin{align*}
g_{11} &= \pdv{\va{r}}{u} \cdot \pdv{\va{r}}{u} \hspace{1cm} g_{12} = \pdv{\va{r}}{u} \cdot \pdv{\va{r}}{v} \hspace{1cm} g_{13} = \pdv{\va{r}}{u} \cdot \pdv{\va{r}}{w} \\[0.5em]
g_{21} &= \pdv{\va{r}}{v} \cdot \pdv{\va{r}}{u} \hspace{1cm} g_{22} = \pdv{\va{r}}{v} \cdot \pdv{\va{r}}{v} \hspace{1cm} g_{23} = \pdv{\va{r}}{v} \cdot \pdv{\va{r}}{w} \\[0.5em]
g_{31} &= \pdv{\va{r}}{w} \cdot \pdv{\va{r}}{u} \hspace{1cm} g_{32} = \pdv{\va{r}}{w} \cdot \pdv{\va{r}}{v} \hspace{1cm} g_{33} = \pdv{\va{r}}{w} \cdot \pdv{\va{r}}{w}
\end{align*}
y haciendo que: $x^{1} = u$, $x^{2} = v$, $x^{3} = w$. Entonces el elemento de longitud de arco de la ec. (\ref{eq:ecuacion_01_03_12}), se puede escribir como:
\begin{align*}
\dd{s}^{2} = \va{E}_{i} \cdot \va{E}_{j} \dd{x}^{i} \dd{x}^{j} = g_{ij} \dd{x}^{i} \dd{x}^{j}, \hspace{1cm} i, j = 1, 2, 3
\end{align*}
donde:
\begin{align}
g_{ij} = \va{E}_{i} \cdot \va{E}_{j} = \pdv{\va{r}}{x^{i}} \cdot \pdv{\va{r}}{x^{j}} = \pdv{y^{m}}{x^{i}} \, \pdv{y^{m}}{x^{j}} \hspace{1cm} i, j \quad \mbox{ índices libres}
\label{eq:ecuacion_01_03_13}
\end{align}
son llamados \textbf{componentes métricos del sistema coordenado curvilíneo}. Los componentes métricos pueden considerarse como los elementos de una matriz simétrica, ya que $g_{ij} = g_{ji}$.
\par
En el sistema de coordenadas rectangular $x, y, z$, el elemento de la longitud del arco al cuadrado es $\dd{s}^{2} = \dd{x}^{2} + \dd{y}^{2} + \dd{z}^{2}$. En este espacio los componentes métricos son:
\begin{align*}
g_{ij} = \mqty(
1 & 0 & 0 \\
0 & 1 & 0 \\
0 & 0 & 1 )
\end{align*}

\subsection{Coordenadas cilíndricas \texorpdfstring{$(r, \theta, z)$.}{r, t, z}}

La regla de transformación de coordenadas cartesianas a coordenadas cilíndricas está dada por:
\begin{align*}
x = r \, \cos \theta, \hspace{1cm} y = r \, \sin \theta, \hspace{1cm} z = z
\end{align*}
Hacemos que:
\begin{align*}
y^{1} = x, \hspace{1cm} y^{2} = y, \hspace{1cm} y^{3} = z
\end{align*}
y también que:
\begin{align*}
x^{1} = r, \hspace{1cm} x^{2} = \theta, \hspace{1cm} x^{3} = z
\end{align*}
por lo que el vector de posición se puede escribir de la forma:
\begin{align*}
\va{r} = \va{r} (r, \theta, z) = r \, \cos \theta \, \vu{e}_{1} + r \, \sin \theta \, \vu{e}_{2} + z \, \vu{e}_{3}
\end{align*}

Calculamos las derivadas de ese vector de posición y encontramos que:
\begin{align*}
\va{E}_{1} &= \pdv{\va{r}}{r} = \cos \theta \, \vu{e}_{1} + \sin \theta \, \vu{e}_{2} \\[0.5em]
\va{E}_{2} &= \pdv{\va{r}}{\theta} = -r \, \sin \theta \, \vu{e}_{1} + r \, \cos \theta \, \vu{e}_{2} \\[0.5em]
\va{E}_{3} &= \pdv{\va{r}}{z} = \vu{e}_{3}
\end{align*}
del resultado en la ec. (\ref{eq:ecuacion_01_03_13}), los componentes métricos de este espacio son:
\begin{align*}
g_{ij} = \pdv{y^{m}}{x^{i}} \, \pdv{y^{m}}{x^{j}} \hspace{1cm} i, j \quad \mbox{ índices libres}
\end{align*}
entonces tendremos que:
\begin{align*}
g_{11} &= \cos \theta \, \cos \theta + \sin \theta \, \sin \theta = \cos^{2} \theta + \sin^{2} \theta = 1 \\[0.5em]
g_{12} &= - r \, \cos \theta \, \sin \theta + r \, \cos \theta \, \sin \theta = 0 \\[0.5em]
g_{13} &= 0 \\[0.5em]
g_{21} &= - r \, \sin \theta \, \cos \theta + r \, \cos \theta \, \sin \theta = 0 \\[0.5em] 
g_{22} &= (-r \, \sin \theta)(- r \, \sin \theta) + (r \, \cos \theta \, \cos \theta) = r^{2} \, (\sin^{2} \theta + \cos \theta) = r^{2} \\[0.5em]
g_{23} &= (- r \, \sin \theta)(0) + (r \, \cos \theta)(0) + (0)(1) = 0 \\[0.5em]
g_{31} &= (0)(\cos \theta) + 0 (\sin \theta) + (1)(0) = 0 \\[0.5em]
g_{32} &= (0)(- r \, \cos \theta) + (0)(r \cos \theta) + (1)(0) = 0 \\[0.5em]
g_{33} &= 0 + 0 + (1)(1) = 1
\end{align*}
Entonces el tensor métrico es:
\begin{align*}
g_{ij} = \mqty(
1 & 0 & 0 \\
0 & r^{2} & 0 \\
0 & 0 & 1 )
\end{align*}

Por lo que el diferencial de longitud de arco es:
\begin{align*}
\dd{s}^{2} = g_{ij} \, \dd{x}^{i} \dd{x}^{j}
\end{align*}
así llegamos a:
\begin{align*}
\dd{s}^{2} = (\dd{r})^{2} + r^{2} \, (\dd{\theta})^{2} + (\dd{z})^{2}
\end{align*}

Notemos que mientras $g_{ij} = 0$ cuando $i \neq j$, el sistema coordenado es ortogonal.

\par
Dado un conjunto de transformaciones de la forma encontrada en la ec. (\ref{eq:ecuacion_01_03_10}), se pueden calcular fácilmente los componentes métricos asociados con las coordenadas generalizadas. Los componentes métricos de un sistema ortogonal tienen la forma:
\begin{align*}
g_{ij} = \mqty(
h_{1}^{2} & 0 & 0 \\
0 & h_{2}^{2} & 0 \\
0 & 0 & h_{3}^{2})
\end{align*}
donde los $h_{i}$ es el llamado \textbf{factor de escala}. Entonces el elemento de longitud de arco al cuadrado es:
\begin{align*}
\dd{s}^{2} = h_{1}^{2} \, (\dd{x}^{1})^{2} + h_{2}^{2} \, (\dd{x}^{2})^{2} + h_{3}^{2} \, (\dd{x}^{3})^{2}
\end{align*}

\end{document}