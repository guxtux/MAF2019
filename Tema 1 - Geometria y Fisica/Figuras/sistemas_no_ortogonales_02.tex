\documentclass[tikz,border=10pt]{standalone}
\usepackage{tikz}
\usepackage{tikz-3dplot}
\usepackage{amsmath}
\usepackage{physics}

\ExplSyntaxOn
\msg_redirect_name:nnn { siunitx } { physics-pkg } { none }
\ExplSyntaxOff

\begin{document}

% Ángulos de vista
\tdplotsetmaincoords{70}{110}

\begin{tikzpicture}[tdplot_main_coords, scale=2]

  % Vectores base directos
  \draw[->, thick] (0, 0, 0) -- (1.5, 1.8, 0) node[anchor=north] {$\vb{b}_1$};
  \draw[->, thick] (0, 0, 0) -- (-1.5, 1, 0) node[anchor=north west] {$\vb{b}_2$};
  \draw[->, thick] (0, 0, 0) -- (-0.5, -0.8, 0.5) node[anchor=south] {$\vb{b}_3$};

  % Vectores recíprocos (opuestos, en línea de trazos)
  \draw[->, thick] (0, 0, 0) -- (2.5, 1.5, 0) node[left, pos=0.8] {$b_{1}$};
  \draw[->, thick] (0, 0, 0) -- (0.5, 1, 1) node[anchor=south east] {$b_{2}$};
  \draw[->, thick] (0, 0, 0) -- (0, 0, 1.5) node[left, pos=0.8] {$b_{3}$};

  % Líneas de conexión para visualización
  % Para b3
  \draw[dashed] (0, 0, 0) -- (1, -0.5, 0);
  \draw[dashed] (1, -0.5, 0) -- (-0.5, -0.8, -0.75);
  \draw[dashed] (-0.5, -0.8, 0.5) -- (-0.5, -0.8, -0.75);

  % Para b2
  \draw[dashed] (0.5, 1, 1) -- (0.5, 1, 0.65);
  \draw[dashed] (0.5, 1, 0.65) -- (0, 0, 0);

  % Para b1
  \draw[dashed] (2.5, 1.5, 0) -- (2.5, 1.5, 0.35);
  \draw[dashed] (2.5, 1.5, 0.35) -- (0, 0, 0);

\end{tikzpicture}

\end{document}
