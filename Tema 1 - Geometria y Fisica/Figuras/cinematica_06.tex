\documentclass[tikz,border=3.14pt]{standalone}
\usepackage{tikz-3dplot}
\usetikzlibrary{arrows.meta}
\usepackage{amsmath}
\usepackage{physics}
\usepackage{calc}

\ExplSyntaxOn
\msg_redirect_name:nnn { siunitx } { physics-pkg } { none }
\ExplSyntaxOff

\begin{document}
\tdplotsetmaincoords{60}{120}
\begin{tikzpicture}[scale=3,
        tdplot_main_coords,
        axis/.style={->, thick},
        vector/.style={-stealth, thick}]

    %standard tikz coordinate definition using x, y, z coords
    \coordinate (O) at (0, 0, 0);
    
    %tikz-3dplot coordinate definition using r, theta, phi coords
    \tdplotsetcoord{P}{2.5}{50}{40}
    
    %draw axes
    \draw [axis] (0, 0, 0) -- (1.5, 0, 0) node[anchor=north east]{$x$};
    \draw [axis] (0, 0, 0) -- (0, 1.5, 0) node[anchor=north west]{$y$};
    \draw [axis] (0, 0, 0) -- (0, 0, 1) node[anchor=south]{$z$};
    
    %draw a vector from O to P
    \draw (O) -- (P);
    \draw [vector] (P) -- ++(-0.2, 0.2, 0) node [left, pos=1.6] {\small{$\vb{e}_{\phi}$}};
    \draw (O) -- (P);
    \draw [dashed] (P) -- ++(0, 0.5, 0) node [above, pos=1.2] {\small{$y$}};
    % \draw[->] (1.3, 1, 1.2) arc [start angle=0,end angle=60,x radius=0.5,y radius=0.5];
    \node at (1.2, 1.3, 1.47) {\small{$\phi$}};
    
    \draw [vector] (O) -- (P);
    \draw [vector] (P) -- ++(0.3, 0.25, 0);
    \node at (1, 1.12, 1.1) {\small{$\vb{e}_{\theta}$}};

    \draw (O) -- (P);
    \draw [dashed] (P) -- ++(0, 0, 0.5);
    
    \draw [dashed] (P) -- (Pxy);
    \draw [vector] (P) -- ++(0, 0.2, 0.25);
    \node at (0.9, 1.2, 1.5) {\small{$\vb{e}_{r}$}};
    \draw [dashed] (O) -- (Pxy);
    \draw [dashed] (P) -- (0, 0, 1);



    %define the rotated coordinate frame to lie in the "theta plane"
    \tdplotsetthetaplanecoords{50}

    \tdplotdrawarc[tdplot_rotated_coords]{(P)}{.35}{0}{48}{anchor=south}{\small{$\theta$}};

\end{tikzpicture}
\end{document}