\documentclass[tikz,border=3.14pt]{standalone}
\usepackage{tikz-3dplot}
\usetikzlibrary{arrows.meta}
\usepackage{amsmath}
\usepackage{physics}

\ExplSyntaxOn
\msg_redirect_name:nnn { siunitx } { physics-pkg } { none }
\ExplSyntaxOff

\begin{document}
\tdplotsetmaincoords{60}{120}
\begin{tikzpicture}[scale=3,
        tdplot_main_coords,
        axis/.style={->, thick},
        vector/.style={-stealth, very thick}]

    %standard tikz coordinate definition using x, y, z coords
    \coordinate (O) at (0, 0, 0);
    
    %tikz-3dplot coordinate definition using r, theta, phi coords
    \tdplotsetcoord{P}{2.5}{50}{45}
    
    %draw axes
    \draw[axis] (0, 0, 0) -- (1.5, 0, 0) node[anchor=north east]{$x$};
    \draw[axis] (0, 0, 0) -- (0, 1.5, 0) node[anchor=north west]{$y$};
    \draw[axis] (0, 0, 0) -- (0, 0, 1) node[anchor=south]{$z$};
    
    %draw a vector from O to P
    \draw[vector] (O) -- (P) node [midway, below] {$\vb{r}$};

    \draw (P) -- (Pxy);
    \draw (O) -- (Pxy);
    \tdplotdrawarc{(O)}{.35}{0}{45}{anchor=north}{$\phi$}

    %define the rotated coordinate frame to lie in the "theta plane"
        \tdplotsetthetaplanecoords{50}
        
        \tdplotdrawarc[tdplot_rotated_coords]{(O)}{.35}{0}{49}{anchor=south west}{$\theta$};

\end{tikzpicture}
\end{document}