\documentclass[12pt]{article}
\usepackage[left=0.25cm,top=1cm,right=0.25cm,bottom=1cm]{geometry}
\textwidth = 20cm
\hoffset = -1cm
\usepackage[utf8]{inputenc}
\usepackage[spanish,es-tabla]{babel}
\usepackage[autostyle,spanish=mexican]{csquotes}
\usepackage[tbtags]{amsmath}
\usepackage{nccmath}
\usepackage{amsthm}
\usepackage{amssymb}
\usepackage{graphicx}
\usepackage{standalone}
\usepackage[outdir=./]{epstopdf}
\usepackage{siunitx}
\usepackage{physics}
\usepackage{color}
\usepackage{float}
\usepackage{multicol}
%\usepackage{milista}
\usepackage{enumitem}
\usepackage{anyfontsize}
\usepackage{anysize}
\usepackage{enumitem}
\usepackage{capt-of}
\usepackage{bm}
\usepackage{relsize}
\usepackage{placeins}
\usepackage{empheq}
\usepackage{cancel}
\usepackage{wrapfig}
\spanishdecimal{.}
\renewcommand{\baselinestretch}{1.5} 
\renewcommand\labelenumii{\theenumi.{\arabic{enumii}}}
\newcommand{\ptilde}[1]{\ensuremath{{#1}^{\prime}}}
\newcommand{\stilde}[1]{\ensuremath{{#1}^{\prime \prime}}}
\newcommand{\ttilde}[1]{\ensuremath{{#1}^{\prime \prime \prime}}}
\newcommand{\ntilde}[2]{\ensuremath{{#1}^{(#2)}}}


\usepackage{titling}
\setlength{\droptitle}{-2cm}
\title{Ejercicio diario 01 \\[0.3em]  \large{Tema 1 - Funciones Gamma y Beta} \vspace{-3ex}}
\author{M. en C. Gustavo Contreras Mayén}
\date{ }


\begin{document}
\vspace{-7ex}
\maketitle
\fontsize{14}{14}\selectfont

Demuestra que si $n$ es un entero positivo:
\begin{align*}
\mathlarger{\Gamma} \left( n - \dfrac{1}{2} \right) = \dfrac{(2 \, n - 3)(2 \, n - 5) \ldots (3)(1) \sqrt{\pi}}{2^{n-1}}
\end{align*}

\noindent
\textbf{Solución}:

Usando la identidad: $\Gamma(x + 1) = x \, \Gamma(x)$ con $x = n - \frac{1}{2}$, entonces:
\begin{align*}
\Gamma \left( n - \dfrac{1}{2} + 1 \right) = \left(n - \dfrac{1}{2} \right) \, \Gamma \left( n - \dfrac{1}{2} \right)
\end{align*}
despejamos el término para $\Gamma \left( n - \dfrac{1}{2} \right)$, así:
\begin{align*}
\Gamma \left( n - \dfrac{1}{2} \right) = \dfrac{1}{\left(n - \dfrac{1}{2} \right)} \, \Gamma \left( n - \dfrac{1}{2} + 1 \right) = \left( \dfrac{2}{2 \, n - 1} \right) \Gamma \left( n + \dfrac{1}{2} \right)
\end{align*}
Aquí podemos usar el resultado que se demostró en la sesión de Zoom:
\begin{align*}
\Gamma \left( n + \dfrac{1}{2} \right) =  \dfrac{(2 \, n {-} 1)(2 \, n {-} 3)(2 \, n {-} 5) \ldots (3)(1)\sqrt{\pi}}{2^{n}}
\end{align*}
Al sustituir este valor en la expresión que hemos avanzado, se tiene que:
\begin{align*}
\Gamma \left( n - \dfrac{1}{2} \right) = \dfrac{2}{\Cancel[blue]{(2 \, n - 1)}} \,  \dfrac{\Cancel[blue]{(2 \, n {-} 1)}(2 \, n {-} 3)(2 \, n {-} 5) \ldots (3)(1)\sqrt{\pi}}{2^{n}}
\end{align*}
y como $\dfrac{2}{2^{n}} = \dfrac{1}{2^{-1} \cdot 2^{n}} = \dfrac{1}{2^{n-1}}$, entonces:
\begin{align*}
\Gamma \left( n - \dfrac{1}{2} \right) = \dfrac{(2 \, n {-} 3)(2 \, n {-} 5) \ldots (3)(1)\sqrt{\pi}}{2^{n-1}} \qed
\end{align*}

\end{document}