\documentclass[12pt]{article}
\usepackage[utf8]{inputenc}
\usepackage[spanish,es-lcroman, es-tabla]{babel}
\usepackage[autostyle,spanish=mexican]{csquotes}
\usepackage{amsmath}
\usepackage{amssymb}
\usepackage{nccmath}
\numberwithin{equation}{section}
\usepackage{amsthm}
\usepackage{graphicx}
\usepackage{epstopdf}
\DeclareGraphicsExtensions{.pdf,.png,.jpg,.eps}
\usepackage{color}
\usepackage{float}
\usepackage{multicol}
\usepackage{enumerate}
\usepackage[shortlabels]{enumitem}
\usepackage{anyfontsize}
\usepackage{anysize}
\usepackage{array}
\usepackage{multirow}
\usepackage{enumitem}
\usepackage{cancel}
\usepackage{tikz}
\usepackage{circuitikz}
\usepackage{tikz-3dplot}
\usetikzlibrary{babel}
\usetikzlibrary{shapes}
\usepackage{bm}
\usepackage{mathtools}
\usepackage{esvect}
\usepackage{hyperref}
\usepackage{relsize}
\usepackage{siunitx}
\usepackage{physics}
%\usepackage{biblatex}
\usepackage{standalone}
\usepackage{mathrsfs}
\usepackage{bigints}
\usepackage{bookmark}
\spanishdecimal{.}

\setlist[enumerate]{itemsep=0mm}

\renewcommand{\baselinestretch}{1.5}

\let\oldbibliography\thebibliography

\renewcommand{\thebibliography}[1]{\oldbibliography{#1}

\setlength{\itemsep}{0pt}}
%\marginsize{1.5cm}{1.5cm}{2cm}{2cm}


\newtheorem{defi}{{\it Definición}}[section]
\newtheorem{teo}{{\it Teorema}}[section]
\newtheorem{ejemplo}{{\it Ejemplo}}[section]
\newtheorem{propiedad}{{\it Propiedad}}[section]
\newtheorem{lema}{{\it Lema}}[section]

\usepackage{enumerate}
%\usepackage[shortlabels]{enumitem}
\usepackage{pifont}
\renewcommand{\labelitemi}{\ding{43}}
%\author{M. en C. Gustavo Contreras Mayén. \texttt{curso.fisica.comp@gmail.com}}
\title{{Examen Parcial 2} \\ {\large Matemáticas Avanzadas de la Física}}
\date{ }
\begin{document}
\vspace{-5cm}
%\renewcommand\theenumii{\arabic{theenumii.enumii}}
\renewcommand\labelenumii{\theenumi.{\arabic{enumii}}}
\maketitle
\fontsize{14}{14}\selectfont
\textbf{Indicaciones:}
\begin{itemize}
\item Responde lo más claro posible cada una de las preguntas.
\item Expresa con tus propias palabras las ideas e interpretaciones que consideres.
\end{itemize}
\begin{enumerate}
\item \begin{enumerate}[label=\alph*)]
\item En clase se explicó la conexión de la teoría Sturm Liouville y el procedimiento de ortonormalización de Gram-Schmidt, así como remover singularidades en las ecuaciones diferenciales, describe con tus palabras este procedimiento.
\item 	Usualmente las funciones de onda tienen expresiones de la forma
\[ \varphi_{n}(x) = N \; x^{\ell} \; \exp(-f(x)) \; G_{n}(x) \]
Donde $N$ es un coeficiente de normalización, $f(x)$ es una función positiva definida y $G$ es una función especial de orden $n$, explica por qué tienen esa estructura.
\end{enumerate}
\item  Menciona al menos cinco propiedades de las funciones especiales, decríbelas en términos generales, no olvides hablar de las representaciones alternativas.
\newpage
\item Complete la siguiente tabla con las discusiones realizadas en clase (de los casos vistos, escriba al menos 5)
\begin{center}
\begin{tabular}{| c | c |}
\hline
\multicolumn{1}{| p{6cm}|}{ \centering Función especial} & \multicolumn{1}{|p{6cm}|}{ \centering Problema físico} \\ \hline
 & \\ \hline
 & \\ \hline
 & \\ \hline
 & \\ \hline
 & \\ \hline
 & \\ \hline
 & \\ \hline
 & \\ \hline
 & \\ \hline
\end{tabular}
\end{center}


\end{enumerate}
\end{document}