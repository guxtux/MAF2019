\documentclass[12pt]{article}
\usepackage[utf8]{inputenc}
\usepackage[spanish,es-lcroman, es-tabla]{babel}
\usepackage[autostyle,spanish=mexican]{csquotes}
\usepackage{amsmath}
\usepackage{amssymb}
\usepackage{nccmath}
\numberwithin{equation}{section}
\usepackage{amsthm}
\usepackage{graphicx}
\usepackage{epstopdf}
\DeclareGraphicsExtensions{.pdf,.png,.jpg,.eps}
\usepackage{color}
\usepackage{float}
\usepackage{multicol}
\usepackage{enumerate}
\usepackage[shortlabels]{enumitem}
\usepackage{anyfontsize}
\usepackage{anysize}
\usepackage{array}
\usepackage{multirow}
\usepackage{enumitem}
\usepackage{cancel}
\usepackage{tikz}
\usepackage{circuitikz}
\usepackage{tikz-3dplot}
\usetikzlibrary{babel}
\usetikzlibrary{shapes}
\usepackage{bm}
\usepackage{mathtools}
\usepackage{esvect}
\usepackage{hyperref}
\usepackage{relsize}
\usepackage{siunitx}
\usepackage{physics}
%\usepackage{biblatex}
\usepackage{standalone}
\usepackage{mathrsfs}
\usepackage{bigints}
\usepackage{bookmark}
\spanishdecimal{.}

\setlist[enumerate]{itemsep=0mm}

\renewcommand{\baselinestretch}{1.5}

\let\oldbibliography\thebibliography

\renewcommand{\thebibliography}[1]{\oldbibliography{#1}

\setlength{\itemsep}{0pt}}
%\marginsize{1.5cm}{1.5cm}{2cm}{2cm}


\newtheorem{defi}{{\it Definición}}[section]
\newtheorem{teo}{{\it Teorema}}[section]
\newtheorem{ejemplo}{{\it Ejemplo}}[section]
\newtheorem{propiedad}{{\it Propiedad}}[section]
\newtheorem{lema}{{\it Lema}}[section]

\usepackage{enumerate}
\usepackage{pifont}
\renewcommand{\labelitemi}{\ding{43}}
%\author{M. en C. Gustavo Contreras Mayén. \texttt{curso.fisica.comp@gmail.com}}
\title{Comentarios a tu Examen final \\ Marco Antonio Sandoval Espinosa}
\date{ }
\begin{document}
%\renewcommand\theenumii{\arabic{theenumii.enumii}}
\renewcommand\labelenumii{\theenumi.{\arabic{enumii}})}
\maketitle
\fontsize{14}{14}\selectfont
\begin{enumerate}
\item A grandes distancias de su fuente, el dipolo eléctrico tiene por campo eléctrico y magnético
\[  \mathbf{E} =a_{E} \sin \theta \dfrac{e^{i(k r - \omega t)}}{r} \bm{\theta}_{0}, \hspace{1cm} \mathbf{B} = a_{B} \sin \theta \dfrac{e^{i(k r-\omega t)}}{r} \bm{\varphi}_{0} \]
Demostrar que las ecuaciones de Maxwell se cumplen
\[ \bm{\nabla \times E} =  -\dfrac{\partial \bm{B}}{\partial t}, \hspace{1cm} \bm{\nabla \times B} =  \varepsilon_{0} \mu_{0} \dfrac{\partial \bm{E}}{\partial t}\]
si hacemos que
\[ \dfrac{a_{E}}{a_{B}} = \dfrac{\omega}{k} =  c = (\varepsilon_{0} \mu_{0})^{-1/2} \]
Considera que si $r$ es grande, los términos de orden $r^{-2}$ pueden descartarse.
\\
Comentarios: La suposición inicial con la que partes, debes de argumentar el por qué consideras que el problema ''está'' en coordenadas esféricas, la naturaleza del problema te ayuda mucho, es decir, se indica que tienes un dipolo eléctrico, si te apoyas con ese elemento, podrías argumentar sin contratiempos.
\\
Es recomendable que asignes un número a todas las ecuaciones que uses, ya que comienzas a presentar expresiones pero más adelante es donde usas un identificador. La expresión que dejas entre paréntesis no me queda clara, ya que parecería que dice ''por comprobar''.
\item La ecuación de onda unidimensional de Schrödinger para una partícula en un potencial $V=\frac{1}{2} k x^{2}$ es
\[ - \dfrac{\hbar^{2}}{2m} \dfrac{d^{2} \psi}{d x^{2}} + \dfrac{1}{2} k x^{2} \psi =  E \psi(x)\]
\begin{enumerate}
\item Usando $\xi = ax$ y una constante $\lambda$, donde
\begin{eqnarray*}
a &=& \left( \dfrac{m k}{\hbar^{2}} \right)^{1/4}  \\ \nonumber
\lambda &=& \dfrac{2E}{\hbar} \left(\dfrac{m}{k} \right)^{1/2} \nonumber
\end{eqnarray*}
demostrar que
\[ \dfrac{d^{2} \psi(\xi)}{d \xi^{2}} + (\lambda - \xi^{2}) \psi(\xi) = 0 \]
\item Sustituyendo
\[ \psi(\xi) = y(\xi) e^{-\xi^{2}/2}\]
demuestra que $y(\xi)$ satisface la ecuación diferencial de Hermite.
\end{enumerate}
Inicias bien con la solución del problema, pero en el segundo inciso es donde hay un ''salto'' que das por hecho: cuando llegas a la expresión general de la ecuación diferencial de segundo grado homogénea, anotas ''si sabemos que $k=m\omega^{2}$ y $E=(n +1/2) \hbar \omega$'' mencionas que ''salió del problema'', mi pregunta es: ¿en qué momento salió como resultado? deberías de haber resuelto para el caso específico de la ecuación de onda, por más que se hayan revisado este tipo de problemas en clase, no puedes dar por hecho y menos en un examen final, aunque la solución a la que llegas es la esperada, en tu respuesta deberías de haber hecho el desarrollo completo del problema.
\item 
\begin{enumerate}
\item Demostrar que
\[  y'' + \dfrac{1 - \alpha^{2}}{4 x^{2}} y = 0\]
tiene dos soluciones:
\begin{eqnarray*}
y_{1}(x) &=& a_{0} x^{(1+\alpha)/2} \\
y_{2}(x) &=& a_{0} x^{(1-\alpha)/2}
\end{eqnarray*}
\item Para $\alpha =0$ las dos soluciones linealmente independientes del inciso anterior se reducen a $y_{10} = a_{0} x^{1/2}$. Usando
\[ y_{2}(x) =  y_{1}(x) \int^{x} \dfrac{dx_{2}}{[y_{1}(x_{2})]^{2}}\]
llega a una segunda solución
\[ y_{20}(x) = a_{0} x^{1/2} ln x \]
\end{enumerate}
También comienzas bien, pero es hasta el punto donde:
\[ \sum_{\lambda=0}^{\infty} a_{\lambda} (k + \lambda)(k+\lambda-1) x^{k+\lambda-2} + \dfrac{1-\alpha^{2}}{4x^{2}} \left( \sum_{\lambda=0}^{\infty} a_{\lambda} x^{k+\lambda} \right) = 0 \]
separas la segunda suma, mi pregunta es ¿cómo para qué? de acuerdo a lo que revisamos, podrías hacer que esa función quede expresada en términos de una $\omega^{2}$
\[ \sum_{\lambda=0}^{\infty} a_{\lambda} (k + \lambda)(k+\lambda-1) x^{k+\lambda-2} + \omega^{2} \left( \sum_{\lambda=0}^{\infty} a_{\lambda} x^{k+\lambda} \right) = 0 \]
para entonces definir la \emph{ecuación indicial}, y así estimar las raíces. Mencionas ''notamos que se harían cero en los coeficientes'', va la pregunta: ¿por qué se harían cero? existe un argumento muy poderoso para que funcione, pero no basta con que se note, hay que expresarlo, la ecuación indicial que manejas de repente hay términos que no se identifica cómo es que llegas a ellos, de
\[ \sum_{\lambda=0}^{\infty} a_{\lambda} \left[ (k-\lambda)(k+\lambda-1) + \dfrac{1}{4} - \alpha^{2} \right] = 0 \]
tengo mi duda de cómo es que llegas al coeficiente $1/4$ y $\alpha^{2}$ ya que en el denominador permanece $4x^{2}$, luego con $\lambda=0$
\[ a_{0} x^{k-2} [(k(k-1) + 1/2 - \alpha^{2})] = 0 \]
en la expresión de la suma, el coeficiente aumenta de $1/4$ a $1/2$ ??
A partir de la revisión de las raíces de la ecuación indicial, debes de llegar a una expresión que te permita definir los coeficientes de la suma, mediante una regla de recurrencia, pero que en tu respuesta no se presenta.
\end{enumerate}
Evaluación:
\begin{enumerate}
\item Problema 1: 1 punto.
\item Problema 2: 0.8 puntos.
\item Problema 3: 0.5 puntos.
\end{enumerate}
Calificación final: \textbf{7.65}
\\
Como se requiere que haya una calificación entera y aplicando la regla de que si tienes una calificación $\geq 0.5$ pasa al siguiente entero, tu calificación del curso de Matemáticas Avanzadas de la Física es: \textbf{8, (ocho)}.
\end{document}