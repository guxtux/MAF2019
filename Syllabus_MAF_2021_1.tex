\documentclass[12pt]{article}
\usepackage[left=0.3cm,top=2cm,right=0.3cm,bottom=2cm]{geometry}
\usepackage[utf8]{inputenc}
\usepackage[spanish,es-tabla]{babel}
\usepackage[autostyle,spanish=mexican]{csquotes}
\usepackage{amsmath}
\usepackage{amsthm}
\usepackage{graphicx}
\usepackage{color}
\usepackage{float}
\usepackage{multicol}
\usepackage{enumerate}
\usepackage{anyfontsize}
\usepackage{anysize}
\usepackage{natbib}
\usepackage{enumitem}
\usepackage{capt-of}
\usepackage{titlesec}
\usepackage{enumitem}
\setlist{nolistsep}
\titlespacing*{\section}{0pt}{0.25\baselineskip}{0.25\baselineskip}
\titlespacing*{\subsection}{0pt}{0.25\baselineskip}{0.25\baselineskip}
\spanishdecimal{.}
\setlist[enumerate]{itemsep=0mm}
\setlength{\parskip}{\baselineskip}
\renewcommand{\baselinestretch}{1.1}
\let\oldbibliography\thebibliography
\renewcommand{\thebibliography}[1]{\oldbibliography{#1}
\setlength{\itemsep}{0pt}}
\marginsize{1.5cm}{1.5cm}{1cm}{2cm}
\author{M. en C. Gustavo Contreras Mayén. \texttt{gux7avo@ciencias.unam.mx}\\
M. en C. Abraham Lima Buendía. \texttt{abraham3081@ciencias.unam.mx}}
\title{Matemáticas Avanzadas de la Física \\ {\large Semestre 2021-1}}
\date{ }
\makeatletter
\renewcommand{\@biblabel}[1]{}
\renewenvironment{thebibliography}[1]
{\section*{\refname}%
	\@mkboth{\MakeUppercase\refname}{\MakeUppercase\refname}%
	\list{}%
	{\labelwidth=0pt
		\labelsep=0pt
		\leftmargin1.5em
		\itemindent=-1.5em
		\advance\leftmargin\labelsep
		\@openbib@code
	}%
	\sloppy
	\clubpenalty4000
	\@clubpenalty \clubpenalty
	\widowpenalty4000%
	\sfcode`\.\@m}
\makeatother
\usepackage{breakcites}	
\begin{document}
\vspace{-4cm}
%\renewcommand\theenumii{\arabic{theenumii.enumii}}
\renewcommand\labelenumii{\theenumi.{\arabic{enumii}}}
\maketitle
\fontsize{14}{14}\selectfont
\textbf{Horario: } Lunes a viernes de 15 a 16 pm.
\section{Objetivos.}
De acuerdo con el mapa curricular de la carrera de Física, los objetivos de la asigantura Matemáticas Avanzadas de la Físca son los siguientes:
\par
El alumno:
\begin{itemize}
\setlength{\itemsep}{0mm}
\item Reconocerá las ideas básicas del análisis de ecuaciones que involucran a funciones de varias variables.
\item Formulará aproximaciones consistentes a soluciones, con el fin de cuantificar los distintos mecanismos de la física que se involucran.
\item Consultará la literatura matemática que sea relevante para los problemas de física.
\item Identificará el papel moderno que juegan las funciones especiales, como auxiliares poderosos en el análisis cualitativo de problemas en varias variables.
\end{itemize}

También es nuestro objetivo demostrar al alumno que \emph{las funciones especiales y las transformadas integrales} no son solamente un tema matemático, que involucra las ramas de la geometría diferencial, las ecuaciones diferenciales y el análisis matemático.
\par
Veremos \emph{son las técnicas de estudio fundamentales} en la electrostática, la electrodinámica, la mecánica cuántica en los límites relativista y no relativista, la dinámica de medios deformables, la hidrodinámica clásica entre otras ramas de la física.

\section{Metodología de enseñanza.}

Para este curso se utilizará la plataforma Moodle, para favorecer una estandarización con las demás asignaturas que se imparten en la Facultad de Ciencias.
\par
Se proporcionarán las credenciales para ingresar a la plataforma en donde encontrarán las actividades de trabajo, materiales de consulta y referencias adicionales para el curso.

\subsection{Formato de trabajo en línea.}
Considerando el potencial de la enseñanza a distancia, tendremos un formato de trabajo mixto:
\begin{enumerate}
\item Con sesiones síncronas en el horario de 3 a 4 pm.
\item Sesiones asíncronas en donde podrán ingresar a la plataforma.
\end{enumerate}

Se tendrán sesiones de videoconferencia al inicio de cada tema del curso, utilizando el servicio de videoconferencia Zoom, con la finalidad de presentar tanto el objetivo como el alcance del tema, así como para señalar las actividades de trabajo y de evaluación del mismo. Las sesiones síncronas se llevarán a cabo en el horario de la asignatura: 3 a 4 pm, el día de la semana podrá variar en función del avance del curso. 
\par
También se tendrán sesiones de videoconferencia los viernes de cada semana, con la finalidad de revisar los avances en la solución de los ejercicios de tarea y de evaluación del tema, siendo muy importante participar en estas sesiones síncronas.

\subsection{Sesiones asíncronas.}

La modalidad de enseñanza a distancia requiere que el alumno realice actividades de manera asíncrona dentro de la plataforma, éstas actividades consistirán en la revisión de materiales de trabajo, lecturas adicionales, consulta de textos complementarios y una serie de ejercicios.
\par
Las actividades (ejercicios y tareas) se deberán de completar en los tiempos señalados, la modalidad a distancia requiere también de un cumplimiento para el alcance de los objetivos.

\section{Temario.}
\begin{enumerate}[leftmargin=2cm, label=Tema \arabic*.]
\item La física y la geometría.
\item Primeras técnicas de solución.
\item Bases completas y ortogonales.
\item Separación de variables en coordenadas esféricas.
\item Funciones especiales.
\item Transformadas integrales.
\end{enumerate}
\section{Evaluación.}
\subsection{Ejercicios semanales en clase.}

Durante cada semana se presentarán ejercicios que se deberán de resolver a modo de tarea, para que puedan trabajarlos oportunamente, hacer consultas, resolver dudas, etc.
\par
Al concluir cada semana se tendrá un conjunto de ejercicios que deberán de ser resueltos, tendrán dos semanas para completarlos y hacer el envío el día viernes correspondiente.
\par
\textbf{Muy importante: } Se considerarán como ejercicios a cuenta de calificación, aquellos ejercicios resueltos que representen el $100\%$ del total, es decir, deberán de entregar todos los ejercicios a cuenta.
\par
En el caso de que no se entreguen todos los ejercicios de la semana, sólo se revisarán éstos pero no contarán para el porcentaje de calificación.
\subsection{Exámenes parciales.}

Habrá 2 exámenes parciales durante el semestre, el primero de ellos cubrirá los tres primeros temas, mientras que el segundo examen, los tres temas restantes.
\par
Se entregará con suficiente el tiempo el listado de problemas para cada examen, el examen es individual por lo que cada alumno devolverá su examen resuelto.
\par
El examen parcial deberá de devolverse con el $100\%$ de los ejercicios resueltos, sólo de esta manera se tomará en cuenta para la calificación, de lo contrario, sólo se revisarán los ejercicios entregados sin que aporten puntuación.
\par
En esta asignatura \emph{se revisa y se evalúa el proceso de resolución de un problema, es decir, será necesario detallar cada paso en la solución}, por lo que deberán de considerar que tanto los ejercicios semanales como los exámenes tendrán que ser resueltos a mano, para posteriomente escanear las hojas que hayan ocupado y enviarlas mediante la plataforma. En caso de no contar con un escáner para la digitalización de las soluciones, se podrá enviar un archivo con las imágenes de la solución, por lo que se pedirá encarecidamente, ser lo más claros en la escritura y orden en el envío de las soluciones.

\subsection{Peso para la calificación.}

El porcentaje para cada elemento de evaluación es el siguiente:
\begin{itemize}
\setlength{\itemsep}{0mm}
\item Ejercicios semanales: $\mathbf{50\%}$.
\item Exámenes parciales: $\mathbf{50\%}$.
\end{itemize}
\section{Primera reunión.}
La primera reunión se llevará a cabo de la siguiente forma:
\begin{itemize}
\item Día: 21 de septiembre de 2020.
\item Hora: De 15 a 16 pm.
\item Plataforma: Zoom.
\end{itemize}
Con la finalidad de garantizar la seguridad de todos en la videoconferencia, es decir, que no haya bots, trolls, etc. la sesión va a requerir de una contraseña para ingresar, por lo que ésta se enviará oportunamente a la cuenta de correo del alumno.
\par
Se recomienda que utilices la cuenta de correo institucional, es decir, aquella con el dominio \texttt{@ciencias}, \texttt{@fciencias} o \texttt{@unam}, ya que se ocuparán los recursos digitales y educativos que la UNAM ofrece, al tener la cuenta institucional, se facilitará el uso de esos recursos; en caso de que no tengas aún la cuenta institucional, podrías revisar el trámite necesario y obtenerla.
\par
Si te interesa llevar el curso, te pedimos gentilmente envíes un correo a la cuenta \texttt{gux7avo@ciencias.unam.mx} para hacerte llegar un mensaje con la clave de acceso, en el asunto del mensaje anota: VIDEOCONFERENCIA MAF 21/09
\par
En la videoconferencia se revisará a detalle este syllabus y tendremos una sesión de preguntas para aclarar dudas.

\end{document}