\documentclass[12pt]{article}
\usepackage[utf8]{inputenc}
\usepackage[spanish,es-lcroman, es-tabla]{babel}
\usepackage[autostyle,spanish=mexican]{csquotes}
\usepackage{amsmath}
\usepackage{amssymb}
\usepackage{nccmath}
\numberwithin{equation}{section}
\usepackage{amsthm}
\usepackage{graphicx}
\usepackage{epstopdf}
\DeclareGraphicsExtensions{.pdf,.png,.jpg,.eps}
\usepackage{color}
\usepackage{float}
\usepackage{multicol}
\usepackage{enumerate}
\usepackage[shortlabels]{enumitem}
\usepackage{anyfontsize}
\usepackage{anysize}
\usepackage{array}
\usepackage{multirow}
\usepackage{enumitem}
\usepackage{cancel}
\usepackage{tikz}
\usepackage{circuitikz}
\usepackage{tikz-3dplot}
\usetikzlibrary{babel}
\usepackage{bm}
\usepackage{mathtools}
\usepackage{esvect}
\usepackage{hyperref}
\usepackage{relsize}
\usepackage{siunitx}
\usepackage{physics}
%\usepackage{biblatex}
\usepackage{standalone}
\usepackage{mathrsfs}
\usepackage{bigints}
\usepackage{bookmark}
\spanishdecimal{.}

\setlist[enumerate]{itemsep=0mm}

\renewcommand{\baselinestretch}{1.5}

\let\oldbibliography\thebibliography

\renewcommand{\thebibliography}[1]{\oldbibliography{#1}

\setlength{\itemsep}{0pt}}
%\marginsize{1.5cm}{1.5cm}{2cm}{2cm}


\newtheorem{defi}{{\it Definición}}[section]
\newtheorem{teo}{{\it Teorema}}[section]
\newtheorem{ejemplo}{{\it Ejemplo}}[section]
\newtheorem{propiedad}{{\it Propiedad}}[section]
\newtheorem{lema}{{\it Lema}}[section]

\usepackage{standalone}
\usepackage{enumerate}
\usepackage{hyperref}
\usepackage[left=1.5cm,top=1.5cm,right=1.5cm,bottom=1.5cm]{geometry}
\title{Tarea 3 y 4 - Matemáticas Avanzadas de la Física}
\date{ }
\begin{document}
\vspace{-4cm}
%\renewcommand\theenumii{\arabic{theenumii.enumii}}
\renewcommand\labelenumii{\theenumi.{\arabic{enumii}}}
\maketitle
\fontsize{14}{14}\selectfont
Para los problemas 1,2 y 3 te pedimos que consultes la referencia \cite[pág. 45]{Greiner}, en donde se explica parte de la solución, tu tarea consiste es detallar todo el proceso, sin omitir pasos y explicando lo más posible el desarrollo. Estos ejercicios tienen el objetivo de guiar el uso del Teorema de Green para la solución de problemas en electrodinámica.
\begin{enumerate}
\item Construye la siguiente ecuación:
\[  \int_{V} \left[ \varphi(\mathbf{r}') \Delta' \psi (\mathbf{r}') - \psi (\mathbf{r}') \Delta' \varphi (\mathbf{r}') \right] dV' = \oint_{S} \left[ \varphi(\mathbf{r}') \dfrac{\partial \psi (\mathbf{r}')}{\partial n'} - \psi (\mathbf{r}') \dfrac{\partial \varphi (\mathbf{r}')}{\partial n'} \right] da' \]
donde $\nabla^{2} =  \Delta$. La expresión es la representación integral del potencial, siendo una representación más general que se obtiene de los teoremas de Green y de la ecuación diferencial de Poisson, la ecuación de partida es
\[ \phi (\mathbf{r}) =  \int	\dfrac{\rho(\mathbf{r}')}{\vert \mathbf{r} - \mathbf{r}' \vert} dV' \]
\item Resuelve el problema del potencial para un punto con carga cerca de una esfera aterrizada ($\Phi = 0$ en la superficie)
\item Resuelve el problema de dos semiesferas conductoras a diferentes potenciales: la semiesfera superior a un potencial $+V$ y la semiesfera inferior a un potencial $-V$
\item Dentro del contexto de la mecánica cuántica, demuestra que el momento \textbf{p}, es un operador hermitiano:
\[ \mathbf{p} =  - i \hbar \nabla \equiv i \dfrac{h}{2 \pi} \nabla \]
\item Como en el inciso anterior, demuestra ahora que el momento angular \textbf{L}, es un operador hermitiano:
\[ \mathbf{L} = - i \hbar \mathbf{r} \times \nabla \equiv i \dfrac{h}{2 \pi} \mathbf{r} \times \nabla \]
\item En una distribución tipo Maxwell la fracción de partículas moviéndose con velocidad $v$ y $v+dv$ es
\[ \dfrac{dN}{N} = 4 \pi \left( \dfrac{m}{2 \pi k T} \right)^{3/2} \exp(-mv^{2}/ k T) v^{2} dv \]
$N$ es el número total de partículas. El promedio o valor esperado de $v^{n}$ se define como $<v^{n}> = N^{-1} \int v^{n} dN$. Demostrar que
\[ < v^{n} > = \left( \dfrac{2 k T}{m} \right)^{n/2} \left( \dfrac{n+1}{2} \right) ! \Bigg/ \dfrac{1}{2} ! \]
\item Demostrar que
\[ \int_{0}^{\infty} e^{-x^{4}} dx = \left( \dfrac{1}{4} \right) !\]
\item Los polinomios de Legendre pueden escribirse como
\[ \begin{split}
P_{n}(\cos \theta) &= 2 \dfrac{(2n-1)!!}{(2n)!!} \left[ \cos n\theta + \dfrac{1}{1} \cdot \dfrac{n}{2n-1} \cos(n-2) \theta + \right. \\
&+ \dfrac{1 \cdot 3}{1 \cdot 2} \dfrac{n(n-1)}{(2n-1)(2n-3)} \cos(n-4) \theta + \\
&+ \left. \dfrac{1 \cdot 3 \cdot 5}{1 \cdot 2 \cdot 3} \dfrac{n(n-1)(n-2)}{(2n-1)(2n-3)(2n-5)} \cos(n-6) \theta + \ldots \right]
\end{split}
\]
Para $n=2s+1$, tenemos que
\[ P_{n}(\cos \theta) =  P_{2s+1} (\cos \theta) =  \sum_{m=0}^{s} a_{m} \cos(2m+1) \theta \]
Encontrar los $a_{m}$ en términos de factoriales y dobles factoriales.
\item Comprueba las siguientes identidades de la función Beta:
\begin{enumerate}
\item $B(a,b) = B(a+1,b) + B(a,b+1)$
\item $B(a,b) = \frac{a+b}{b} B(a,b+1)$ 
\item $B(a,b) = \frac{b-1}{a} B(a+1,b-1)$
\item $B(a,b) B(a+b,c) = B(b,c) B(a,b+c)$
\end{enumerate}
\item Demostrar que
\[ \int_{-1}^{1} (1-x^{2})^{1/2} x^{2n} dx =  
\begin{cases}
\pi/2 & n = 0 \\
\pi \dfrac{(2n-1)!!}{(2n+2)!!} & n=1,2,3,\ldots  \end{cases}
 \]
\item Demuestra que 
\[ \Gamma(\frac{1}{2} - n) \Gamma(\frac{1}{2} + n) = (-1)^{n} \pi \]
\item Considera la siguiente propiedad de la función Gamma
\[ \Gamma(x) = (x-1) \Gamma (x-1) \]
\begin{enumerate}[label=\alph{*})]
\item Usando reiteradamente la propiedad anterior, demuestra que
\[ \Gamma(a+n) = (a+n-1)(a+n-2) \ldots (a+n-k) \Gamma(a+n-k) \]
\item Haciendo $k=n$ en la ecuación anterior, ahora demuestra
\[ a(a+a) \ldots (a+n-1) = \dfrac{\Gamma(a+n)}{\Gamma(a)} \]
\item Usando el inciso anterior, demuestra que
\[ \alpha (\alpha-1) \ldots (\alpha -n + 1) = (-1)^{n} \dfrac{\Gamma(n-\alpha)}{\Gamma(-\alpha)} \]
\end{enumerate}
\item Usando la integral 
\[ \int_{0}^{\pi/2} \sin^{2n} t dt = \dfrac{(2n-1)!!}{(2n)!!} \dfrac{\pi}{2} \]
demuestra que la integral elíptica de primera clase
\[ K(k) = \int_{0}^{\pi/2} \dfrac{dt}{\sqrt{1-k^{2} \sin^{2}t}} \]
puede expresarse como
\[ K(k) = \dfrac{\pi}{2} \left[ 1 + \sum_{n=1}^{\infty} \left[ \dfrac{(2n-1)!!}{(2n)!!} \right]^{2} k^{2n} \right]  \]
\item Demuestra que la integral elíptica de primera clase está relacionada con la función hipergeométrica de la siguiente manera
\[ K(k) = \dfrac{\pi}{4} F \left( \dfrac{1}{2}, \dfrac{1}{2};1,k^{2} \right) \]
\end{enumerate}
\begin{thebibliography}{99}
\bibitem{Greiner} Walter Greiner: Classical Electrodynamics, Springer Verlag, New York, 1998.
\end{thebibliography}
\end{document}