\documentclass[12pt]{article}
\usepackage[left=0.25cm,top=1cm,right=0.25cm,bottom=1cm]{geometry}
%\usepackage{geometry}
\textwidth = 20cm
\hoffset = -1cm
\usepackage[utf8]{inputenc}
\usepackage[spanish,es-tabla]{babel}
\usepackage[autostyle,spanish=mexican]{csquotes}
\usepackage{amsmath}
\usepackage{nccmath}
\usepackage{amsthm}
\usepackage{amssymb}
\usepackage{graphicx}
\usepackage{physics}
\usepackage{color}
\usepackage{float}
\usepackage{multicol}
%\usepackage{milista}
\usepackage{enumitem}
\usepackage{anyfontsize}
\usepackage{anysize}
\usepackage{enumitem}
\usepackage{capt-of}
\usepackage{bm}
\usepackage{relsize}
\newlist{milista}{enumerate}{2}
\setlist[milista,1]{label=\arabic*)}
\setlist[milista,2]{label=\arabic{milistai}.\arabic*)}
\spanishdecimal{.}
\renewcommand{\baselinestretch}{1.5} 
\title{Problemas a cuenta - Semana 4 \\ {\large Matemáticas Avanzadas de la Física}\vspace{-3ex}}
\date{ }
\author{}
\begin{document}
\renewcommand\labelenumii{\theenumi.{\arabic{enumii}}}
\maketitle
\fontsize{14}{14}\selectfont
Ejercicios.
\par
\textbf{Usando la delta de Dirac}:
\begin{enumerate}
\item Demuestra que 
\begin{align*}
\nabla \vdot \left( \dfrac{e_{r}}{r^{2}} \right) = 0 \hspace{1cm} \mbox{para } r > 0
\end{align*}
Tip: Se puede resolver en coordenadas cartesianas, pero es más fácil usando coordenadas esféricas.
\item Demuestra que 
\begin{align*}
\nabla \left( \dfrac{1}{r} \right) = - \dfrac{e_{r}}{r^{2}}
\end{align*}
\end{enumerate}
\par
\textbf{Función de Green}.
\begin{enumerate}[resume]
\item Demostrar la propiedad de simetría de la función de Green:
\begin{align*}
G(a, b) =  G(b, a) \hspace{1cm} a \neq b
\end{align*}
Nota: Se demuestra suponiendo que hay dos funciones de Green que cumplen con condiciones de frontera de tipo Dirichlet, para luego continuar con el desarrollo a partir de la definición. En el la sección 9.7 del Arfken hay un desarrollo parecido, pero no tan formal en donde concluye que la función de Green es simétrica.
\end{enumerate}
\end{document}