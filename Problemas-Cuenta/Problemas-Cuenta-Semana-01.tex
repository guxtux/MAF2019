\documentclass[12pt]{article}
\usepackage[utf8]{inputenc}
\usepackage[spanish,es-lcroman, es-tabla]{babel}
\usepackage[autostyle,spanish=mexican]{csquotes}
\usepackage{amsmath}
\usepackage{amssymb}
\usepackage{nccmath}
\numberwithin{equation}{section}
\usepackage{amsthm}
\usepackage{graphicx}
\usepackage{epstopdf}
\DeclareGraphicsExtensions{.pdf,.png,.jpg,.eps}
\usepackage{color}
\usepackage{float}
\usepackage{multicol}
\usepackage{enumerate}
\usepackage[shortlabels]{enumitem}
\usepackage{anyfontsize}
\usepackage{anysize}
\usepackage{array}
\usepackage{multirow}
\usepackage{enumitem}
\usepackage{cancel}
\usepackage{tikz}
\usepackage{circuitikz}
\usepackage{tikz-3dplot}
\usetikzlibrary{babel}
\usepackage{bm}
\usepackage{mathtools}
\usepackage{esvect}
\usepackage{hyperref}
\usepackage{relsize}
\usepackage{siunitx}
\usepackage{physics}
%\usepackage{biblatex}
\usepackage{standalone}
\usepackage{mathrsfs}
\usepackage{bigints}
\usepackage{bookmark}
\spanishdecimal{.}

\setlist[enumerate]{itemsep=0mm}

\renewcommand{\baselinestretch}{1.5}

\let\oldbibliography\thebibliography

\renewcommand{\thebibliography}[1]{\oldbibliography{#1}

\setlength{\itemsep}{0pt}}
%\marginsize{1.5cm}{1.5cm}{2cm}{2cm}


\newtheorem{defi}{{\it Definición}}[section]
\newtheorem{teo}{{\it Teorema}}[section]
\newtheorem{ejemplo}{{\it Ejemplo}}[section]
\newtheorem{propiedad}{{\it Propiedad}}[section]
\newtheorem{lema}{{\it Lema}}[section]

\usepackage[top=2cm, bottom=2cm, left=1.5cm, right=1.5cm,headsep=0pt]{geometry}
\title{Primeros problemas a cuenta \\ {\large Matemáticas Avanzadas de la Física}}
\date{ }
\begin{document}
\renewcommand\labelenumii{\theenumi.{\arabic{enumii}}}
\maketitle
\fontsize{14}{14}\selectfont
Los problemas a cuenta que se dejan durante la semana, se entregan al siguiente viernes, recuerden que para tomarse en cuenta en el $40\%$ de la calificación final del curso, deberán de entregar al menos el $50\%$ de los ejercicios, en caso contrario, sólo se van a revisar y a comentar, pero ya no se toman en cuenta para la calificación.
\par
Se debe de entregar la solución a mano, de manera ordenada, con las hojas engrapadas y muy importante: \textbf{con su nombre completo}.
\par
Ejercicios:
\begin{enumerate}
\item A partir de las ecuaciones de Maxwell sin fuentes, obtener la ecuación de onda.
\item En el espacio de Minkowski se define $x_{1} = x, x_{2} = y, x_{3} = z, x_{0} = c \: t$. Esto se hace para que el elemento de la métrica sea 
\[ \dd{s^{2}} = \dd{x_{0}}^{2} - \dd{x_{1}}^{2} - \dd{x_{2}}^{2} - \dd{x_{3}}^{2} \]
donde $c$ es la velocidad de la luz. Demuestra que la métrica del espacio de Minkowski es
\begin{align*}
(g_{ij}) = \begin{pmatrix}
1 & 0 & 0 & 0 \\
0 & -1 & 0 & 0 \\
0 & 0 & -1 & 0 \\
0 & 0 & 0 & -1
\end{pmatrix}
\end{align*}
\item Considerando el sistema de coordenadas esférico, en donde las relaciones de transformación están dadas por:
\begin{align*}
\begin{aligned}
x &= r \sin \theta \cos \varphi \\
y &= r \sin \theta \sin \varphi \\
z &= r \cos \theta
\end{aligned}
\end{align*}
Realiza las respectivas operaciones, cuentas, etc. para obtener de manera explícita (es decir, debe de mostrarse el procedimiento completo):
\begin{enumerate}
\item Los factores de escala $h_{r}$, $h_{\theta}$, $h_{\varphi}$
\item Los elementos de línea, área y volumen.
\item Los operadores diferenciales: $\nabla \psi$, $\nabla \vdot \vb{V}$, $\nabla \vdot \nabla \psi$, $\nabla \cp \vb{V}$  
\end{enumerate}
\end{enumerate}
\end{document}