\documentclass[12pt]{article}
\usepackage[left=0.25cm,top=1cm,right=0.25cm,bottom=1cm]{geometry}
%\usepackage{geometry}
\textwidth = 20cm
\hoffset = -1cm
\usepackage[utf8]{inputenc}
\usepackage[spanish,es-tabla]{babel}
\usepackage[autostyle,spanish=mexican]{csquotes}
\usepackage{amsmath}
\usepackage{nccmath}
\usepackage{amsthm}
\usepackage{amssymb}
\usepackage{graphicx}
\usepackage{physics}
\usepackage{color}
\usepackage{float}
\usepackage{multicol}
%\usepackage{milista}
\usepackage{enumitem}
\usepackage{anyfontsize}
\usepackage{anysize}
\usepackage{enumitem}
\usepackage{capt-of}
\usepackage{bm}
\usepackage{relsize}
\newlist{milista}{enumerate}{2}
\setlist[milista,1]{label=\arabic*)}
\setlist[milista,2]{label=\arabic{milistai}.\arabic*)}
\spanishdecimal{.}
\renewcommand{\baselinestretch}{1.5} 
\title{Problemas a cuenta \\ {\large Matemáticas Avanzadas de la Física}}
\date{ }
\begin{document}
\renewcommand\labelenumii{\theenumi.{\arabic{enumii}}}
\maketitle
\fontsize{14}{14}\selectfont
Ejercicios:
\begin{enumerate}
\item Demuestra que 
\begin{align*}
\dv{t} (\vb{P}_{\mbox{mec}} + \vb{P}_{\mbox{campo}})_{\alpha} = \sum_{\beta} \int_{V} \pdv{x_{\beta}} T_{\alpha \beta} \dd[3]{x}
\end{align*}
donde $\vb{P}_{\mbox{mec}}$ es el momento mecánico,  $\vb{P}_{\mbox{campo}}$ es el momento de campo y $T_{\alpha \beta}$ es el tensor de energía momento:
\begin{align*}
T_{\alpha \beta} = \dfrac{1}{4 \, \pi} \left[ E_{\alpha} \, E_{\beta} + B_{\alpha} \, B_{\beta} - \dfrac{1}{2} \left( \vb{E} \vdot \vb{E} + \vb{B} \vdot \vb{B} \right)  \, \delta_{\alpha \, \beta} \right]
\end{align*}
\textbf{Nota: } Revisa en el libro de Clasiscal Electrodynamics, en el capítulo 6.8, se indica la solución pero hay que desarrollar los pasos intermedios.
\par
\textit{Sugerencia para la solución (no se comparte con los alumnos)} Utiliza la relación:
\begin{align*}
\dv{P_{\mbox{mec}}}{t} = \mbox{Fuerza de Lorentz } = \int \left( p \, E + \dfrac{J \cp B}{c} \right) \dd[3]{x}
\end{align*}
\textbf{Nota: } Revisa en el libro de Clasiscal Electrodynamics, en el capítulo 6.8, se indica la solución pero hay que desarrollar los pasos intermedios.
\item Para el vector $\va{A} = - 5 \: \vu{i} + 6 \: \vu{j}$ y con vectores base $\va{e}_{1} = \vu{i} +  2 \: \vu{j}$ y $\va{e}_{2} = - 2 \: \vu{i} - \vu{j}$, calcular:
\begin{enumerate}
\item Los vectores base duales $\va{e}^{\: 1}$ y $\va{e}^{\: 2}$ de los vectores base $\va{e}_{1}$ y $\va{e}_{2}$
\item Las componentes contravariantes $\va{A}_{1}$ y $\va{A}_{2}$
\item Las componentes covariantes $\va{A}^{1}$ y $\va{A}^{2}$
\end{enumerate}
\textit{Sugerencia para la solución (no se comparte con los alumnos)}
s\begin{enumerate}[label=\alph*)]
\item Expresa los vectores $\vu{i}$ y $\vu{j}$ como combinaciones lineales de $\va{e}_{1}$ y $\va{e}_{2}$.
\item Construye la matriz métrica $g_{ij} = e_{i} \vdot e_{2}$
\item Construye la matriz métrica inversa $g^{ij} = [g_{ij}]$
\item Usando la relación $e^{i} = g^{ij} \, e_{j}$ encuentra las componentes covariantes para cada uno de los elementos base.
\item Expresa el vector $\vb{A}$ en términos de la base $e^{i}$
\end{enumerate}
\end{enumerate}
\end{document}