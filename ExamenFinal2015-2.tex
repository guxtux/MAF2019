\documentclass[12pt]{article}
\usepackage[utf8]{inputenc}
\usepackage[spanish,es-lcroman, es-tabla]{babel}
\usepackage[autostyle,spanish=mexican]{csquotes}
\usepackage{amsmath}
\usepackage{amssymb}
\usepackage{nccmath}
\numberwithin{equation}{section}
\usepackage{amsthm}
\usepackage{graphicx}
\usepackage{epstopdf}
\DeclareGraphicsExtensions{.pdf,.png,.jpg,.eps}
\usepackage{color}
\usepackage{float}
\usepackage{multicol}
\usepackage{enumerate}
\usepackage[shortlabels]{enumitem}
\usepackage{anyfontsize}
\usepackage{anysize}
\usepackage{array}
\usepackage{multirow}
\usepackage{enumitem}
\usepackage{cancel}
\usepackage{tikz}
\usepackage{circuitikz}
\usepackage{tikz-3dplot}
\usetikzlibrary{babel}
\usetikzlibrary{shapes}
\usepackage{bm}
\usepackage{mathtools}
\usepackage{esvect}
\usepackage{hyperref}
\usepackage{relsize}
\usepackage{siunitx}
\usepackage{physics}
%\usepackage{biblatex}
\usepackage{standalone}
\usepackage{mathrsfs}
\usepackage{bigints}
\usepackage{bookmark}
\spanishdecimal{.}

\setlist[enumerate]{itemsep=0mm}

\renewcommand{\baselinestretch}{1.5}

\let\oldbibliography\thebibliography

\renewcommand{\thebibliography}[1]{\oldbibliography{#1}

\setlength{\itemsep}{0pt}}
%\marginsize{1.5cm}{1.5cm}{2cm}{2cm}


\newtheorem{defi}{{\it Definición}}[section]
\newtheorem{teo}{{\it Teorema}}[section]
\newtheorem{ejemplo}{{\it Ejemplo}}[section]
\newtheorem{propiedad}{{\it Propiedad}}[section]
\newtheorem{lema}{{\it Lema}}[section]

\usepackage{enumerate}
\usepackage{pifont}
\renewcommand{\labelitemi}{\ding{43}}
%\author{M. en C. Gustavo Contreras Mayén. \texttt{curso.fisica.comp@gmail.com}}
\title{{Examen final} \\ {\large Matemáticas Avanzadas de la Física}}
\date{ }
\begin{document}
%\renewcommand\theenumii{\arabic{theenumii.enumii}}
\renewcommand\labelenumii{\theenumi.{\arabic{enumii}})}
\maketitle
\fontsize{14}{14}\selectfont
Fecha de entrega: \textbf{Jueves 28 de mayo de 2015.} En el cubículo del Profesor en el laboratorio de Biofísica, Piso 4, Departamenteo de Física, Facultad de Ciencias a las 3 pm (hora del DF)
\\
\begin{enumerate}
\item A grandes distancias de su fuente, el dipolo eléctrico tiene por campo eléctrico y magnético
\[  \mathbf{E} =a_{E} \sin \theta \dfrac{e^{i(k r - \omega t)}}{r} \bm{\theta}_{0}, \hspace{1cm} \mathbf{B} = a_{B} \sin \theta \dfrac{e^{i(k r-\omega t)}}{r} \bm{\varphi}_{0} \]
Demostrar que las ecuaciones de Maxwell se cumplen
\[ \bm{\nabla \times E} =  -\dfrac{\partial \bm{B}}{\partial t}, \hspace{1cm} \bm{\nabla \times B} =  \varepsilon_{0} \mu_{0} \dfrac{\partial \bm{E}}{\partial t}\]
si hacemos que
\[ \dfrac{a_{E}}{a_{B}} = \dfrac{\omega}{k} =  c = (\varepsilon_{0} \mu_{0})^{-1/2} \]
Considera que si $r$ es grande, los términos de orden $r^{-2}$ pueden descartarse.
\item La ecuación de onda unidimensional de Schrödinger para una partícula en un potencial $V=\frac{1}{2} k x^{2}$ es
\[ - \dfrac{\hbar^{2}}{2m} \dfrac{d^{2} \psi}{d x^{2}} + \dfrac{1}{2} k x^{2} \psi =  E \psi(x)\]
\begin{enumerate}
\item Usando $\xi = ax$ y una constante $\lambda$, donde
\begin{eqnarray*}
a &=& \left( \dfrac{m k}{\hbar^{2}} \right)^{1/4}  \\ \nonumber
\lambda &=& \dfrac{2E}{\hbar} \left(\dfrac{m}{k} \right)^{1/2} \nonumber
\end{eqnarray*}
demostrar que
\[ \dfrac{d^{2} \psi(\xi)}{d \xi^{2}} + (\lambda - \xi^{2}) \psi(\xi) = 0 \]
\item Sustituyendo
\[ \psi(\xi) = y(\xi) e^{-\xi^{2}/2}\]
demuestra que $y(\xi)$ satisface la ecuación diferencial de Hermite.
\end{enumerate}
\item 
\begin{enumerate}
\item Demostrar que
\[  y'' + \dfrac{1 - \alpha^{2}}{4 x^{2}} y = 0\]
tiene dos soluciones:
\begin{eqnarray*}
y_{1}(x) &=& a_{0} x^{(1+\alpha)/2} \\
y_{2}(x) &=& a_{0} x^{(1-\alpha)/2}
\end{eqnarray*}
\item Para $\alpha =0$ las dos soluciones linealmente independientes del inciso anterior se reducen a $y_{10} = a_{0} x^{1/2}$. Usando
\[ y_{2}(x) =  y_{1}(x) \int^{x} \dfrac{dx_{2}}{[y_{1}(x_{2})]^{2}}\]
llega a una segunda solución
\[ y_{20}(x) = a_{0} x^{1/2} ln x \]
\end{enumerate}
\end{enumerate}
\end{document}